% !TeX TXS-program:compile = txs:///pdflatex/[--shell-escape]
\documentclass[a4paper,11pt,twoside]{article}
%%%%%%%%%%%%%%%%%%%%%%%%%%%%%%%%%%%%%%%%%%%%%%%%%%%%%%%%%%%%%%%%%%%%%%%%%%%
% A generic report compiler for publishing rables and plots for a partcular choise of lags in the NARX system
% A path to media corresponding to the setting is defined below, as well as chosen parameter settings
\makeatletter
\def\input@path{{../Results_cv_set_VS_ny_4_nu_4/}}
\def\dataset{HS}
\def\datasett{VS}
\def\ny{4}
\def\nu{4}
\def\order{3}
\makeatother
%%%%%%%%%%%%%%%%%%%%%%%%%%%%%%%%%%%%%%%%%%%%%%%%%%%%%%%%%%%%%%%%%%%%%%%%%%%
% Packages
\usepackage[final]{pdfpages}
\usepackage{verbatim}
\usepackage{inputenc}
\usepackage{graphicx} 
\usepackage{amsmath,amssymb,mathrsfs,amsfonts}
\usepackage{mathtools}
\usepackage{amsthm}
\usepackage{mathtools}
\usepackage{calrsfs}
\usepackage{graphicx}
\usepackage{subfig}
\usepackage{eucal}    
\usepackage{amssymb}  
\usepackage{pifont}
\usepackage{color} 
\usepackage{cancel}
\usepackage[toc,page]{appendix}
\usepackage{pgfplots}
\pgfplotsset{compat=newest,every axis/.append style={line width=0.5pt},x label style={font={\small},at={(axis description cs:0.5,-0.15)},anchor=north},y label style={font={\small},at={(axis description cs:-0.15,0.5)},anchor=south},z label style={font={\small},at={(axis description cs:-0.25,0.5)},anchor=north},label style={font=\small},tick label style={font=\small},title style={font=\small},} %% y tick label style={/pgf/number format/.cd,fixed,precision=3, set thousands separator={},},z tick label style={/pgf/number format/.cd,fixed,precision=3, set thousands separator={},
\usetikzlibrary{shapes,shadows,arrows,backgrounds,patterns,positioning,automata,calc,decorations.markings,decorations.pathreplacing,bayesnet,arrows.meta,decorations.fractals,spy}
\usepackage{varwidth}
\usepackage{lscape}
\usepackage{array} 
\usepackage[colorlinks=false,pdfborder={0 0 0}]{hyperref}
\usepackage{tabularx}
\usepackage{textcomp}
\usepackage{multicol} 
\usepackage{booktabs}
\usepackage{multirow}
\usepackage[font=small,labelfont=bf]{caption}                                                           
\usepackage{textcase}
\usepackage{bbm} 
\usepackage{fancyhdr}
\usepackage{enumitem}
\usepackage{soul}
\usepackage{wrapfig}
%%%%%%%%%%%%%%%%%%%%%%%%%%%%%%%%%%%%%%%%%%%%%%%%%%%%%%%%%%%%%%%%%%%%%%%%%%%
% Geometry
\setlength{\parindent}{2em}
\setlength{\parskip}{0.5em}
\renewcommand{\baselinestretch}{1.2}
\usepackage[left=2cm, right=2cm, top=2.5cm, bottom=3cm, headheight=13.6pt]{geometry}
\allowdisplaybreaks 
%%%%%%%%%%%%%%%%%%%%%%%%%%%%%%%%%%%%%%%%%%%%%%%%%%%%%%%%%%%%%%%%%%%%%%%%%%%
% Bibliography
\usepackage[backend=bibtex,style=ieee,sorting=none]{biblatex} 
\bibliography{bibliography}
\renewcommand*{\bibfont}{\scriptsize}
%%%%%%%%%%%%%%%%%%%%%%%%%%%%%%%%%%%%%%%%%%%%%%%%%%%%%%%%%%%%%%%%%%%%%%%%%%%
% Custom commands and operators
\makeatletter
\newcommand*{\rom}[1]{\expandafter\@slowromancap\romannumeral #1@}
\newcommand{\ie}{\textit{i.e.} }
\newcommand{\eg}{\textit{e.g.} }
\newcommand\id{\ensuremath{\mathbbm{1}}} 
\newcommand{\norm}[1]{\left\lVert#1\right\rVert}
\DeclareMathOperator{\E}{\mathbb{E}}
\DeclareMathOperator{\eye}{\mathbb{I}}
\DeclareMathOperator{\zeros}{\mathbb{O}}
\DeclareMathOperator{\tr}{\textrm{tr}}
\DeclareMathOperator{\vvec}{\textrm{vec}}
\DeclareMathOperator{\ik}{\mathrm{k}}
\DeclareMathOperator{\ip}{\mathrm{p}}
\DeclareMathOperator{\inn}{\mathrm{n}}
\DeclareMathOperator{\im}{\mathrm{m}}
\DeclareMathOperator{\td}{\mathrm{t}}
\DeclareMathOperator{\kd}{\mathrm{k}}
\DeclareMathOperator{\T}{\mathrm{T}}
\DeclareMathOperator{\K}{\mathrm{K}}
\DeclareMathOperator{\rk}{\mathrm{rk}}
\DeclareMathOperator{\vc}{\mathrm{vec}}
\DeclareSymbolFontAlphabet{\mathcal} {symbols}
\DeclareSymbolFont{symbols}{OMS}{cm}{m}{n}
\DeclareMathAlphabet{\mathbfit}{OML}{cmm}{b}{it}
\makeatother
% Number equations
%\numberwithin{equation}{section}

%%%%%%%%%%%%%%%%%%%%%%%%%%%%%%%%%%%%%%%%%%%%%%%%%%%%%%%%%%%%%%%%%%%%%%%%%%%
%Theorems
\newtheoremstyle{mytheoremstyle} % name
{.5em}                    % Space above
{.8em}                    % Space below
{\itshape}                % Body font
{1em}                           % Indent amount
{\bfseries}                   % Theorem head font
{:}                          % Punctuation after theorem head
{.5em}                       % Space after theorem head
{}  % Theorem head spec (can be left empty, meaning ‘normal’)

\theoremstyle{mytheoremstyle}
\newtheorem{theorem}{Theorem}[section]
\newtheorem{remark}{Remark}[section]
\newtheorem{assumption}{Assumption}[section]
\newtheorem{lemma}{Lemma}[section]
\newtheorem{condition}{Condition}[section]
\newtheorem{definition}{Definition}[section]
\newtheorem{property}{Property}[section]
\newtheorem{corollary}{Corollary}[section]
\renewcommand\qedsymbol{$\blacksquare$}
%%%%%%%%%%%%%%%%%%%%%%%%%%%%%%%%%%%%%%%%%%%%%%%%%%%%%%%%%%%%%%%%%%%%%%%%%%%
% Nomenclature
\usepackage[intoc]{nomencl}
\makenomenclature

\usepackage[ruled]{algorithm}
\usepackage{float}

\usepackage{algorithmic}
\algsetup{linenosize=\scriptsize}
\usepackage{etoolbox}
\AtBeginEnvironment{algorithmic}{\scriptsize}
\renewcommand{\thealgorithm}{\thechapter.\arabic{algorithm}} 
%\usepackage{chngcntr}
%\counterwithin{algorithm}{section}
% correct bad hyphenation here
\hyphenation{op-tical net-works semi-conduc-tor}
%%%%%%%%%%%%%%%%%%%%%%%%%%%%%%%%%%%%%%%%%%%%%%%%%%%%%%%%%%%%%%%%%%%%%%%%%%%%%
\usepackage{titling}
\setlength{\droptitle}{-6em}     % Eliminate the default vertical space
\addtolength{\droptitle}{4pt}   % Only a guess. Use this for adjustment	tubnode/.style={midway, right=2pt},

%%%%%%%%%%%%%%%%%%%%%%%%%%%%%%%%%%%%%%%%%%%%%%%%%%%%%%%%%%%%%%%%%%%%%%%%%%%%%
\usepackage[explicit]{titlesec}
\usepackage{titletoc}
\interfootnotelinepenalty=10000
\title{Identification framework overview}
\date{}
\begin{document}
	\maketitle
	\vspace{-2cm}
\section{Framework structure}
\par The schematic overview is of the implemented identification framework is presented in Figure 1, with further details on the constituent parts described in Table 1. The notation is consistent with the previous report.\\
\begin{figure}[!h]
\begin{tikzpicture}
\centering
[spy using outlines={circle, magnification=4, size=2cm, connect spies}]
\tikzstyle{block} = [rectangle, draw, fill=white, 
text width=10em, text centered, rounded corners, minimum height=3em]
\tikzstyle{line} = [draw, -latex']
%\draw [help lines] (0,0) grid (17,12);
\node at (8.5,12.5) (A0){};
\node[block,align=center,text width=18em] at (8.5,11) (A1){EFOR-based common model structure selection};
\node[block,align=center,text width=18em] at (8.5,9) (A2){Formulating the regression in terms of polynomial coefficients};
\node[block,text width=7em,align=center] at (1.8,9) (A21){Design parameter mapping};
\node[block,align=center,text width=18em] at (8.5,7) (A3){Splitting the regression into training and testing subsets};
\node[block,align=center,text width=38em,text height=5em] at (8.5,4.5) (A4){Cross-validation};
\node[block,align=center] at (4,4.6) (A41){Ridge regression};
\node[block,align=center] at (8.5,4.6) (A42){LASSO};
\node[block,align=center] at (13,4.6) (A43){Sparse group LASSO};
\node[block,align=center,text width=18em] at (8.5,2) (A5){Estimation with optimal hyper-parameters};
\node[block,align=center,text width=18em] at (8.5,0) (A6){Validation on testing datasets};
\node[block,text width=7em,align=center] at (1.8,0) (A61){Testing datasets};
\path[->,draw,thick]
(A0) edge node[right] {$y^{k}_{train}(t), u^{k}_{train}(t), k = 1,\dots,K$} (A1)
(A1) edge node[right] {$f(y,u,t,\theta)$} (A2)
(A21) edge node[above] {$g(\xi,\mathbf{B})$} (A2)
(A2) edge node[right] {$\Phi, \mathbf{Y}$} (A3)
(A3) edge node[right] {$\bar{\mathbf{Y}}, \bar{\Phi}, \tilde{\mathbf{Y}}, \tilde{\Phi}$} (A4)
(A4) edge node[right] {$\gamma^{\ast},\alpha^{\ast}$} (A5)
(A5) edge node[right] {$\mathbf{B}^{\ast}_{ridge}, \mathbf{B}^{\ast}_{lasso}, \mathbf{B}^{\ast}_{spgl}$} (A6)
(A61) edge node[above] {$y(t), u(t)$} (A6)
;
\end{tikzpicture}
\caption{A schematic overview of the identification framework.}
\end{figure}
\begin{table}
	\centering
		\caption{Constituent parts of the framework}\label{tab:criteria}
		\small
		\begin{tabular}{>{\raggedright\arraybackslash}p{1.5cm}
				>{\raggedright\arraybackslash}p{3cm}
				>{\raggedright\arraybackslash}p{3cm}
			>{\raggedright\arraybackslash}p{8cm}}
			Algorithm & Input & Output & Criterion \\ 
			\hline 
			EFOR &Dictionary of regressors & Model structure & \begin{tabular}{@{}l@{}} AERR - for determining significant terms \\ BIC - for truncating number of terms \end{tabular} \\ 
			Tikhnonov & Training data, $\lbrace \bar{\mathbf{Y}}, \bar{\Phi} \rbrace$; Lagrangian, $\gamma$ & Polynomial coefficients, $\hat{\mathbf{B}}^{\ast}(\gamma)$  & $  \Biggl \{\norm{ \biggl(\bar{\mathbf{Y}} - \bar{\Phi} \mathbf{K}\bar{\mathbf{B}} \biggr) }_{2}^{2} + \gamma \norm{\bar{\mathbf{B}}}^{2}_{2} \Biggr\}$ \\
			LASSO & Training data, $\lbrace \bar{\mathbf{Y}}, \bar{\Phi} \rbrace$; Lagrangian, $\gamma$ & Polynomial coefficients, $\hat{\mathbf{B}}^{\ast}(\gamma)$ &  $  \Biggl \{\norm{ \biggl(\bar{\mathbf{Y}} - \bar{\Phi} \mathbf{K}\bar{\mathbf{B}} \biggr) }_{2}^{2} + \gamma \norm{\bar{\mathbf{B}}}_{1} \Biggr\}$ \\ 
			Sparse group LASSO &Training data, $\lbrace \bar{\mathbf{Y}}, \bar{\Phi} \rbrace$; Lagrangians, $\gamma, \alpha$ & Polynomial coefficients, $\hat{\mathbf{B}}^{\ast}(\gamma, \alpha)$  &  $  \Biggl \{\norm{ \biggl(\bar{\mathbf{Y}} - \bar{\Phi} \mathbf{K}\bar{\mathbf{B}} \biggr) }_{2}^{2} + \gamma \left((1-\alpha) \sum_{i}^{N}\sqrt{L_i} \norm{B_i}^2_{2} + \alpha \sum_{i}^{N} \norm{B_i}_{1}\right)\Biggr\}$  \\
			CV & Testing data, $\lbrace \tilde{\mathbf{Y}}, \tilde{\Phi} \rbrace$; Parameter estimates,  $\hat{\mathbf{B}}^{\ast}(\gamma, \alpha)$  & Optimal regularisation parameters, $\gamma^{\ast},\alpha^{\ast}$ &  $\mathrm{PRESS} = \sum_{l=1}^{L} \sum_{t} (\tilde{y}^{l}(t) - \hat{y}^{l}(t))$, where $L$ is number of folds  \\ 
			\hline 
		\end{tabular}	
	\end{table}
\par In table 1, the regressor matrix $\Phi$ reflects both the common structure of the dynamical model identified using EFOR algorithm,  $f$, and the arbitrary structure of the polynomial parameter mapping, $g$. The joint regression matrix and output vector are partitioned for the cross validation into the training data, $\bar{\Phi}, \bar{\mathbf{Y}}$, and the testing data, $\tilde{\Phi}, \tilde{\mathbf{Y}}$. The estimation algorithms are performed simultaneously with the same regularisation coefficient. Optimal regularisation coefficients are found using random search algorithm with 200 iterations. The testing datasets for the final stage are not utilised in the model identification and differ from the testing subsets in the cross-validation algorithm.
\par Previously demonstrated results were obtained using K-folds validation that implicitly relies on OSA-validation as the data partition does not preserve the time-series structure in $\tilde{\Phi}, \tilde{\mathbf{Y}}$. The identified model structure thus corresponds to the minimal OSA error, while model-predicted output  differs from the measured one(see Figure 2).
\par The framework has been amended so that the PRESS is obtained using model-predicted output
\begin{equation}
\hat{y}(t+1) = f(\hat{y}(t),\dots,\hat{y}(t-n_y),u(t),\dots,u(t-n_u),\theta(\mathbf{B})). 
\end{equation}
In order to accommodate MPO validation into the CV algorithm, the regression was partitioned so that the testing is performed against a dataset from one experiment at a time. See the blocks of the regression matrix in Figure 3.
\section{Results file}
The matlab file with results contains the following variables:
\begin{itemize}
	\item Betas\_nonreg\_opt -- estimate of polynomial coefficients obtained using OLS
	\item Betas\_tikh\_opt -- estimate of polynomial coefficients obtained using ridge regression
	\item Betas\_lasso\_opt -- estimate of polynomial coefficients obtained using LASSO algorithm
	\item Betas\_spl\_opt -- estimate of polynomial coefficients obtained using sparse group LASSO regression
	\item Files -- indices of files used for model training
	\item testFiles -- indices of files used for model testing
	\item A -- matrix of polynomial model terms for training files so that the following relationship holds
	\begin{equation*}
	\Theta = B A^{\top}.
	\end{equation*}
	\item A\_valid -- matrix of polynomial model terms for testing files
	\item A\_symb -- symbolic array of polynomial model terms that (structure of $g(\cdot)$)
	\item f\_model -- inline functions for lagged terms in the dynamical model $f(\cdot)$
	\item g\_model -- inline functions for polynomial terms in the parameter model  $g(\cdot)$
	\item Terms -- symbolic array with identified significant model terms (structure of $f(\cdot)$)
	\item extParams -- design parameter vector indexed according to the tables 1 and 2 in the previous report
\end{itemize}
\begin{figure}[!h]
	\includegraphics[width=\linewidth]{mpo_osa.png}
	\caption{Model output from the model identified with OSA-PRESS optimal regularisation parameters for dataset HS4.}
\end{figure}
\begin{figure}[!h]
	\includegraphics[width=\linewidth]{blocks.png}
	\caption{Block partitions of the regression data. Each testing dataset has time-series structure.}
\end{figure}
\section{Validation results for modified framework}
\par Figures below present validation results for experimental data obtained for soft auxetic foams. The test file indexing corresponds to the tables 2 in the previous report.
%\begin{figure}[!h]
%	\definecolor{mycolor1}{rgb}{0.00000,0.44700,0.74100}%
%	\definecolor{mycolor2}{rgb}{0.85000,0.32500,0.09800}%
%	\centering
%	% This file was created by matlab2tikz.
%
\definecolor{mycolor1}{rgb}{0.00000,0.44700,0.74100}%
\definecolor{mycolor2}{rgb}{0.85000,0.32500,0.09800}%
\definecolor{mycolor3}{rgb}{0.92900,0.69400,0.12500}%
%
\begin{tikzpicture}

\begin{axis}[%
width=6.159cm,
height=1.831cm,
at={(0cm,10.169cm)},
scale only axis,
xmin=1000,
xmax=1405,
xlabel style={font=\color{white!15!black}},
xlabel={Sample index},
ymin=-58.594,
ymax=0,
ylabel style={font=\color{white!15!black}},
ylabel={$y(t)$},
axis background/.style={fill=white},
title style={font=\bfseries},
title={C1: RMSE(OSA) = 1.5371, RMSE(MPO) = 1.5952},
legend style={legend cell align=left, align=left, draw=white!15!black}
]
\addplot [color=mycolor1, line width=2.0pt]
  table[row sep=crcr]{%
1006	-28.076\\
1007	-32.9590000000001\\
1008	-25.635\\
1009	-14.6479999999999\\
1010	-20.752\\
1011	-17.0899999999999\\
1012	-18.3109999999999\\
1013	-20.752\\
1014	-7.32400000000007\\
1015	-4.88300000000004\\
1016	-4.88300000000004\\
1017	-13.4280000000001\\
1018	-23.193\\
1020	-25.635\\
1021	-19.5309999999999\\
1022	-12.2070000000001\\
1023	-24.414\\
1025	-14.6479999999999\\
1026	-20.752\\
1027	-18.3109999999999\\
1028	-17.0899999999999\\
1029	-29.297\\
1030	-25.635\\
1031	-18.3109999999999\\
1032	-28.076\\
1033	-28.076\\
1034	-21.973\\
1035	-18.3109999999999\\
1036	-15.8689999999999\\
1037	-15.8689999999999\\
1038	-21.973\\
1041	-21.973\\
1042	-23.193\\
1043	-30.518\\
1044	-29.297\\
1045	-19.5309999999999\\
1046	-18.3109999999999\\
1047	-10.9860000000001\\
1048	-14.6479999999999\\
1049	-15.8689999999999\\
1050	-12.2070000000001\\
1051	-13.4280000000001\\
1052	-18.3109999999999\\
1053	-19.5309999999999\\
1054	-18.3109999999999\\
1055	-29.297\\
1056	-21.973\\
1057	-12.2070000000001\\
1058	-12.2070000000001\\
1059	-13.4280000000001\\
1060	-17.0899999999999\\
1061	-9.76600000000008\\
1062	-10.9860000000001\\
1063	-14.6479999999999\\
1064	-9.76600000000008\\
1065	-7.32400000000007\\
1066	-13.4280000000001\\
1067	-13.4280000000001\\
1068	-21.973\\
1069	-20.752\\
1070	-25.635\\
1071	-21.973\\
1072	-25.635\\
1073	-20.752\\
1074	-20.752\\
1075	-18.3109999999999\\
1076	-18.3109999999999\\
1077	-17.0899999999999\\
1078	-30.518\\
1079	-41.5039999999999\\
1080	-41.5039999999999\\
1081	-42.7249999999999\\
1082	-28.076\\
1083	-37.8420000000001\\
1084	-46.3869999999999\\
1085	-46.3869999999999\\
1086	-35.4000000000001\\
1087	-47.607\\
1088	-58.5940000000001\\
1089	-43.9449999999999\\
1091	-24.414\\
1093	-17.0899999999999\\
1094	-21.973\\
1096	-12.2070000000001\\
1097	-9.76600000000008\\
1098	-12.2070000000001\\
1099	-18.3109999999999\\
1101	-18.3109999999999\\
1102	-21.973\\
1103	-23.193\\
1104	-30.518\\
1105	-28.076\\
1106	-30.518\\
1108	-20.752\\
1109	-24.414\\
1110	-18.3109999999999\\
1111	-13.4280000000001\\
1112	-19.5309999999999\\
1113	-20.752\\
1114	-12.2070000000001\\
1115	-15.8689999999999\\
1116	-12.2070000000001\\
1117	-13.4280000000001\\
1118	-13.4280000000001\\
1119	-9.76600000000008\\
1120	-10.9860000000001\\
1122	-23.193\\
1123	-25.635\\
1124	-26.855\\
1125	-15.8689999999999\\
1126	-20.752\\
1127	-32.9590000000001\\
1128	-24.414\\
1129	-28.076\\
1130	-29.297\\
1131	-18.3109999999999\\
1132	-12.2070000000001\\
1133	-17.0899999999999\\
1134	-18.3109999999999\\
1135	-30.518\\
1136	-36.6210000000001\\
1137	-36.6210000000001\\
1138	-28.076\\
1139	-28.076\\
1140	-25.635\\
1142	-25.635\\
1144	-15.8689999999999\\
1146	-13.4280000000001\\
1147	-13.4280000000001\\
1148	-20.752\\
1149	-31.7380000000001\\
1150	-26.855\\
1151	-17.0899999999999\\
1152	-15.8689999999999\\
1153	-15.8689999999999\\
1154	-9.76600000000008\\
1155	-7.32400000000007\\
1156	-7.32400000000007\\
1157	-15.8689999999999\\
1158	-10.9860000000001\\
1159	-14.6479999999999\\
1160	-14.6479999999999\\
1161	-13.4280000000001\\
1163	-6.10400000000004\\
1164	-4.88300000000004\\
1165	-7.32400000000007\\
1166	-18.3109999999999\\
1167	-26.855\\
1168	-28.076\\
1170	-13.4280000000001\\
1171	-10.9860000000001\\
1172	-6.10400000000004\\
1173	-12.2070000000001\\
1174	-14.6479999999999\\
1175	-18.3109999999999\\
1176	-26.855\\
1177	-39.0630000000001\\
1178	-47.607\\
1180	-37.8420000000001\\
1181	-28.076\\
1182	-30.518\\
1183	-31.7380000000001\\
1184	-34.1800000000001\\
1185	-24.414\\
1186	-23.193\\
1187	-23.193\\
1188	-21.973\\
1189	-23.193\\
1190	-18.3109999999999\\
1191	-30.518\\
1192	-37.8420000000001\\
1194	-18.3109999999999\\
1195	-18.3109999999999\\
1196	-30.518\\
1199	-45.1659999999999\\
1200	-45.1659999999999\\
1201	-34.1800000000001\\
1202	-32.9590000000001\\
1203	-36.6210000000001\\
1204	-41.5039999999999\\
1206	-17.0899999999999\\
1207	-23.193\\
1208	-20.752\\
1209	-13.4280000000001\\
1210	-18.3109999999999\\
1211	-19.5309999999999\\
1212	-14.6479999999999\\
1213	-13.4280000000001\\
1214	-15.8689999999999\\
1215	-13.4280000000001\\
1216	-17.0899999999999\\
1217	-25.635\\
1218	-25.635\\
1219	-15.8689999999999\\
1220	-14.6479999999999\\
1221	-24.414\\
1222	-40.2829999999999\\
1223	-29.297\\
1224	-20.752\\
1225	-14.6479999999999\\
1226	-18.3109999999999\\
1227	-15.8689999999999\\
1228	-12.2070000000001\\
1229	-13.4280000000001\\
1232	-20.752\\
1233	-15.8689999999999\\
1234	-23.193\\
1235	-29.297\\
1236	-25.635\\
1237	-20.752\\
1238	-23.193\\
1239	-30.518\\
1240	-21.973\\
1241	-8.54500000000007\\
1242	-13.4280000000001\\
1243	-13.4280000000001\\
1244	-17.0899999999999\\
1245	-17.0899999999999\\
1246	-13.4280000000001\\
1247	-17.0899999999999\\
1248	-19.5309999999999\\
1249	-13.4280000000001\\
1250	-15.8689999999999\\
1251	-15.8689999999999\\
1252	-12.2070000000001\\
1253	-15.8689999999999\\
1254	-9.76600000000008\\
1255	-10.9860000000001\\
1256	-8.54500000000007\\
1257	-8.54500000000007\\
1259	-20.752\\
1260	-24.414\\
1261	-35.4000000000001\\
1263	-13.4280000000001\\
1265	-10.9860000000001\\
1266	-13.4280000000001\\
1267	-10.9860000000001\\
1268	-12.2070000000001\\
1269	-14.6479999999999\\
1270	-24.414\\
1271	-21.973\\
1272	-26.855\\
1273	-23.193\\
1274	-17.0899999999999\\
1275	-13.4280000000001\\
1278	-6.10400000000004\\
1279	-9.76600000000008\\
1280	-14.6479999999999\\
1281	-18.3109999999999\\
1282	-18.3109999999999\\
1283	-19.5309999999999\\
1284	-29.297\\
1285	-25.635\\
1286	-18.3109999999999\\
1287	-24.414\\
1288	-24.414\\
1289	-31.7380000000001\\
1290	-26.855\\
1291	-17.0899999999999\\
1292	-9.76600000000008\\
1293	-6.10400000000004\\
1294	-7.32400000000007\\
1295	-9.76600000000008\\
1296	-8.54500000000007\\
1297	-8.54500000000007\\
1298	-12.2070000000001\\
1299	-17.0899999999999\\
1300	-15.8689999999999\\
1301	-15.8689999999999\\
1302	-19.5309999999999\\
1303	-13.4280000000001\\
1304	-4.88300000000004\\
1305	-9.76600000000008\\
1306	-21.973\\
1307	-19.5309999999999\\
1308	-21.973\\
1310	-14.6479999999999\\
1311	-19.5309999999999\\
1312	-25.635\\
1313	-25.635\\
1314	-19.5309999999999\\
1315	-26.855\\
1316	-26.855\\
1317	-23.193\\
1318	-28.076\\
1319	-26.855\\
1320	-20.752\\
1321	-19.5309999999999\\
1322	-29.297\\
1323	-40.2829999999999\\
1324	-30.518\\
1325	-18.3109999999999\\
1326	-17.0899999999999\\
1327	-18.3109999999999\\
1328	-21.973\\
1329	-26.855\\
1330	-19.5309999999999\\
1331	-19.5309999999999\\
1332	-26.855\\
1333	-28.076\\
1334	-20.752\\
1335	-14.6479999999999\\
1336	-20.752\\
1337	-34.1800000000001\\
1338	-30.518\\
1339	-31.7380000000001\\
1340	-21.973\\
1341	-17.0899999999999\\
1342	-17.0899999999999\\
1343	-10.9860000000001\\
1344	-6.10400000000004\\
1345	-6.10400000000004\\
1346	-4.88300000000004\\
1347	-10.9860000000001\\
1348	-19.5309999999999\\
1349	-18.3109999999999\\
1350	-15.8689999999999\\
1351	-14.6479999999999\\
1352	-20.752\\
1353	-14.6479999999999\\
1354	-9.76600000000008\\
1355	-13.4280000000001\\
1356	-8.54500000000007\\
1357	-10.9860000000001\\
1358	-18.3109999999999\\
1359	-23.193\\
1360	-17.0899999999999\\
1361	-9.76600000000008\\
1363	-12.2070000000001\\
1364	-9.76600000000008\\
1365	-13.4280000000001\\
1366	-31.7380000000001\\
1367	-25.635\\
1369	-37.8420000000001\\
1370	-36.6210000000001\\
1371	-26.855\\
1372	-20.752\\
1373	-18.3109999999999\\
1374	-21.973\\
1375	-23.193\\
1376	-29.297\\
1377	-29.297\\
1378	-36.6210000000001\\
1379	-42.7249999999999\\
1380	-46.3869999999999\\
1381	-37.8420000000001\\
1382	-35.4000000000001\\
1383	-40.2829999999999\\
1384	-31.7380000000001\\
1385	-34.1800000000001\\
1386	-31.7380000000001\\
1387	-18.3109999999999\\
1388	-12.2070000000001\\
1389	-13.4280000000001\\
1390	-18.3109999999999\\
1391	-17.0899999999999\\
1392	-8.54500000000007\\
1393	-14.6479999999999\\
1394	-13.4280000000001\\
1395	-6.10400000000004\\
1396	-10.9860000000001\\
1397	-12.2070000000001\\
1398	-7.32400000000007\\
1399	-18.3109999999999\\
1400	-20.752\\
1401	-14.6479999999999\\
1402	-23.193\\
1403	-37.8420000000001\\
1404	-32.9590000000001\\
1405	-37.8420000000001\\
};
\addlegendentry{True output}

\addplot [color=mycolor2, dashed, line width=2.0pt]
  table[row sep=crcr]{%
1006	-28.3734904745222\\
1007	-30.8287513378341\\
1008	-26.7121079728224\\
1009	-13.5860530618645\\
1010	-16.4336285164409\\
1011	-18.0813431859847\\
1012	-21.1232930613435\\
1013	-19.1577704514355\\
1014	-7.12390642486844\\
1015	-5.89416150748207\\
1016	-4.81634156735004\\
1017	-14.2923236455918\\
1018	-24.4340855194887\\
1019	-24.2896482113324\\
1020	-25.4344269145213\\
1021	-18.7848558530757\\
1022	-11.553106580503\\
1023	-24.6619604734353\\
1024	-19.5118651615578\\
1025	-12.1583932123615\\
1026	-20.6764482067044\\
1027	-21.1487620215053\\
1028	-16.4293698150568\\
1029	-25.4821520580792\\
1030	-24.6880878948268\\
1031	-20.050811105916\\
1032	-26.6866423022043\\
1033	-26.9231510055297\\
1034	-22.0993977449975\\
1035	-19.1342961391426\\
1036	-14.9651186829628\\
1038	-20.9312408058815\\
1039	-19.5796477019251\\
1040	-19.9790542854985\\
1041	-22.3410282906068\\
1042	-22.6502575932\\
1043	-31.4637410557232\\
1044	-27.5799406802948\\
1045	-19.7887899257832\\
1046	-19.1079831019179\\
1047	-12.2600344778321\\
1048	-10.211058775888\\
1049	-15.8359702108485\\
1050	-13.5916344114144\\
1051	-13.0736881828668\\
1052	-17.551926007214\\
1053	-20.049261301757\\
1054	-18.614860547247\\
1055	-27.0069650246455\\
1056	-21.9622351700104\\
1057	-13.4646403733109\\
1058	-11.299285619468\\
1059	-15.0874893888686\\
1060	-16.7714543805246\\
1061	-8.87734730713419\\
1062	-9.25073574906196\\
1063	-13.6073877810256\\
1064	-12.0643678263325\\
1065	-9.21313840836592\\
1066	-11.9793485331149\\
1067	-13.2637781097189\\
1068	-21.5744478783779\\
1069	-20.8368324837097\\
1070	-22.9316748597273\\
1071	-22.5789799784959\\
1072	-23.887554364278\\
1073	-21.1332431936823\\
1074	-19.8628635443345\\
1075	-19.3603299077945\\
1076	-17.1822750039773\\
1077	-18.4555310410824\\
1078	-27.6438822783884\\
1079	-41.4872978384326\\
1080	-41.663147689078\\
1081	-39.1598254006401\\
1082	-25.4664889403091\\
1083	-37.1776638025449\\
1084	-46.5674243686324\\
1085	-44.674657970316\\
1086	-36.3591097907001\\
1087	-44.2098118473284\\
1088	-57.8726719565182\\
1089	-44.3530782641892\\
1090	-36.8430882184132\\
1092	-19.8885733781146\\
1093	-16.7602652398009\\
1094	-20.7097619250392\\
1095	-16.6886813190852\\
1096	-10.0350916249524\\
1097	-9.57115150001209\\
1098	-13.2011305273475\\
1099	-19.1265693304372\\
1100	-17.9330466353567\\
1101	-17.7008252391431\\
1102	-23.0494295782341\\
1103	-23.9397172510664\\
1104	-32.2659562045319\\
1105	-28.1555887451022\\
1106	-29.8460155965849\\
1107	-24.6932447366862\\
1108	-20.6395574975229\\
1109	-25.1871362541788\\
1110	-19.019441549427\\
1111	-17.2232048591584\\
1112	-19.7558776348164\\
1113	-19.8163044803612\\
1114	-15.9581525250644\\
1115	-15.9323526483511\\
1116	-12.144993909862\\
1117	-12.5804710579657\\
1118	-15.1249914453597\\
1119	-11.2907622353896\\
1120	-12.1671383944458\\
1121	-15.1767684880781\\
1122	-23.5477613978094\\
1123	-24.9653360929092\\
1124	-26.1741977523131\\
1125	-14.9403299118285\\
1126	-20.6440285859205\\
1127	-34.7936671394118\\
1128	-25.6346317169111\\
1129	-27.295552669377\\
1130	-28.5048015010721\\
1131	-19.356061016809\\
1132	-12.5604735106426\\
1133	-17.14727820704\\
1134	-20.371942345497\\
1135	-28.8447478384307\\
1136	-37.6787992539059\\
1137	-36.5143822319253\\
1138	-26.998880117362\\
1139	-28.6151753453066\\
1140	-26.5384323983149\\
1141	-25.8640521851253\\
1142	-25.9912332235572\\
1143	-17.8807469078511\\
1144	-15.4500546384784\\
1145	-16.5252719924235\\
1146	-14.4612411848991\\
1147	-15.8304800398539\\
1148	-21.3071987779908\\
1149	-33.16482562691\\
1150	-25.9664576571654\\
1151	-17.5378208259649\\
1152	-16.1286679049006\\
1153	-16.1272101312848\\
1154	-10.4719565172957\\
1155	-6.65352125594086\\
1156	-7.86547244433814\\
1157	-15.197554415673\\
1158	-12.9068082398442\\
1159	-12.0243019784473\\
1160	-13.9598342633092\\
1161	-13.4777854094646\\
1162	-10.6434629914393\\
1163	-7.69269312234223\\
1164	-5.40534434077404\\
1165	-7.76840184679759\\
1166	-19.2589231680308\\
1167	-26.3807712287783\\
1168	-29.4670154225987\\
1169	-18.3253840258817\\
1170	-13.6681767217053\\
1171	-11.0534471452206\\
1172	-7.1983055817866\\
1173	-14.0106405580775\\
1174	-15.6093074637661\\
1175	-19.3388389635174\\
1176	-25.5547622978822\\
1177	-40.2919235994832\\
1178	-50.6462494099057\\
1179	-40.4591568557278\\
1180	-38.2521870585906\\
1181	-27.9561720414308\\
1182	-28.9788439744154\\
1183	-31.3852680178886\\
1184	-32.0426824993347\\
1185	-27.3473067966593\\
1186	-24.0386732865095\\
1187	-25.0171273931139\\
1188	-21.2162376776892\\
1189	-24.4373462552892\\
1190	-19.8388769655089\\
1191	-32.2126922913014\\
1192	-36.9670129953731\\
1194	-16.7846303547315\\
1195	-18.7889897290086\\
1196	-30.0922300541915\\
1197	-36.0356873748003\\
1198	-43.3754448506274\\
1199	-44.2601502931061\\
1200	-43.8433025751171\\
1201	-35.9771496010308\\
1202	-33.5413144851714\\
1203	-36.5395254692487\\
1204	-40.2503735975092\\
1205	-27.7346793725981\\
1206	-17.0542918594106\\
1207	-25.368751223338\\
1208	-22.1001879739708\\
1209	-12.8425009865582\\
1210	-18.6257712327501\\
1211	-21.7168705251172\\
1212	-17.4986848472299\\
1213	-13.1055601490943\\
1214	-16.4154825193959\\
1215	-15.4533115045867\\
1216	-18.2130948104254\\
1217	-27.094855377047\\
1218	-22.7320152221307\\
1219	-17.1733291769071\\
1220	-18.1795136160761\\
1221	-27.0753658943606\\
1222	-39.3273444268716\\
1223	-30.5914201436449\\
1224	-20.8874705321916\\
1225	-15.9146779876273\\
1226	-16.7634362087595\\
1227	-17.0579855596332\\
1228	-12.9274013270326\\
1229	-13.8517506250075\\
1230	-17.548319302515\\
1231	-20.6230713039779\\
1232	-21.5626197591603\\
1233	-14.8738728806456\\
1234	-24.442222278595\\
1235	-29.7025087062848\\
1236	-26.4792775064634\\
1237	-21.6090179836615\\
1238	-23.8942292084344\\
1239	-30.146056957492\\
1240	-20.0118417906897\\
1241	-9.27515620902682\\
1242	-12.1763320351895\\
1243	-15.4682784500972\\
1244	-19.0656580796535\\
1245	-18.8831876408105\\
1246	-12.8808376890095\\
1247	-18.9913190432901\\
1248	-19.6012046675585\\
1249	-14.0541239748757\\
1250	-17.1085515875066\\
1251	-15.5361153537049\\
1252	-14.3480370757043\\
1253	-17.1412068272161\\
1254	-9.35760985095067\\
1255	-9.78098234425352\\
1256	-9.37645000570137\\
1257	-9.69561230735121\\
1258	-17.9513028719266\\
1259	-19.8022965428906\\
1260	-25.8195502957617\\
1261	-33.4090004870184\\
1262	-23.0633979145543\\
1263	-14.3319388962777\\
1264	-12.1139521881039\\
1265	-11.3210518378071\\
1266	-15.259749857732\\
1267	-10.8789682968479\\
1268	-10.4752345999107\\
1269	-14.6980910838872\\
1270	-23.4875869202933\\
1271	-23.2418204727751\\
1272	-29.2117190514712\\
1273	-20.2162192762821\\
1274	-18.4100949022863\\
1275	-10.7850011460291\\
1276	-9.59907987039787\\
1277	-7.93921036100596\\
1278	-7.82661413250548\\
1279	-9.01481597093243\\
1280	-15.8063288661783\\
1282	-18.1859445268478\\
1283	-18.7873950526659\\
1284	-31.0204362667009\\
1285	-29.2699285105048\\
1286	-19.3799561617495\\
1287	-23.7524939890691\\
1288	-24.4164777265175\\
1289	-29.4630692980734\\
1290	-24.4988167996505\\
1291	-18.0714004227893\\
1292	-9.59876371476093\\
1293	-7.05150429832247\\
1294	-6.86166591393658\\
1295	-8.85487954294445\\
1296	-9.43684186697033\\
1297	-8.77660647550147\\
1298	-12.0581378854527\\
1299	-16.4928282942301\\
1300	-15.6974918155383\\
1301	-16.4648528523458\\
1302	-18.0254294156448\\
1303	-11.2135219762936\\
1304	-7.32467002557655\\
1305	-12.8492013298633\\
1306	-20.4750076242005\\
1307	-17.6383285684201\\
1308	-19.9257246624034\\
1309	-19.6334946246332\\
1310	-14.299368560942\\
1311	-18.2184778709916\\
1312	-25.4591182499644\\
1313	-25.884626520753\\
1314	-18.18309449507\\
1315	-30.2152132645995\\
1316	-25.2977119323\\
1317	-20.9476824429464\\
1318	-26.2029926715102\\
1319	-25.3353197838683\\
1320	-21.6203716882465\\
1321	-18.5647320517821\\
1322	-26.6895281615232\\
1323	-40.7096719378258\\
1324	-32.1691391459335\\
1325	-17.8863524552244\\
1327	-18.6180652527789\\
1328	-23.0049684708645\\
1329	-26.3586242479685\\
1330	-19.2846232323834\\
1331	-21.4022997257671\\
1332	-27.0038235884988\\
1333	-28.5187531682591\\
1334	-19.5208877363305\\
1335	-16.5513383690313\\
1336	-20.9125555894641\\
1337	-32.2384733965296\\
1338	-31.8520733260107\\
1339	-28.7846047373655\\
1340	-19.4935182464042\\
1341	-17.7474201862035\\
1342	-18.8642780857547\\
1343	-12.6582552036523\\
1344	-6.84733769662648\\
1345	-5.07042597523309\\
1346	-4.51194303372677\\
1347	-11.1525463318685\\
1348	-18.0826056199996\\
1349	-19.6729734851026\\
1350	-15.7392099090976\\
1351	-15.5028365739188\\
1352	-18.4453723178985\\
1353	-11.545902890196\\
1354	-12.6190828977556\\
1355	-13.5696886400683\\
1356	-10.0262609571171\\
1357	-11.1951953795606\\
1358	-19.0305195629476\\
1359	-23.5409999207359\\
1360	-14.3074673522888\\
1361	-11.508646008492\\
1362	-10.3399140963884\\
1363	-12.172118100038\\
1364	-9.22913847192717\\
1365	-17.916195830911\\
1366	-31.2288974184771\\
1367	-22.31663814546\\
1368	-29.3896318940351\\
1369	-37.1853745077678\\
1370	-35.2015703683298\\
1371	-29.4353931269513\\
1372	-20.798586313435\\
1373	-21.136418871562\\
1374	-21.4378288486121\\
1375	-23.8346032765978\\
1376	-26.9499708127048\\
1377	-26.8944753435931\\
1378	-35.5778076175479\\
1379	-42.095362875312\\
1380	-47.0903161750061\\
1381	-34.9726237493214\\
1382	-35.3624369487741\\
1383	-42.0097922391353\\
1384	-31.4048940285238\\
1385	-36.1853532138796\\
1386	-32.3519656528199\\
1387	-18.5443235122784\\
1388	-11.6593703968015\\
1389	-11.8209731979011\\
1390	-19.198806513959\\
1391	-16.3935305453624\\
1392	-10.905345011651\\
1393	-14.5428818695389\\
1394	-12.707902249203\\
1395	-8.40051222832403\\
1396	-10.8032991266375\\
1397	-12.3857056300174\\
1398	-9.82798759240973\\
1399	-14.7556507508116\\
1400	-20.8437381223227\\
1401	-17.1254003684314\\
1402	-25.760176769101\\
1403	-38.2576352694002\\
1404	-33.9709693295435\\
1405	-41.5069219498037\\
};
\addlegendentry{OSA predition}

\addplot [color=mycolor3, dotted, line width=2.0pt]
  table[row sep=crcr]{%
1006	-28.076\\
1007	-32.9590000000001\\
1008	-25.635\\
1009	-14.6479999999999\\
1010	-16.4336285164411\\
1011	-17.7144588465353\\
1012	-20.2025659964947\\
1013	-18.85949493576\\
1014	-7.29025560289961\\
1015	-5.6372418921444\\
1016	-4.76597514576815\\
1017	-14.3725898254361\\
1018	-24.6801550278421\\
1019	-24.5887177708835\\
1020	-25.9104105278475\\
1021	-19.0524843516321\\
1022	-11.6413684831164\\
1023	-24.5544940179636\\
1024	-19.3417782075783\\
1025	-12.1035748919921\\
1026	-20.4036385625691\\
1027	-20.5261733037967\\
1028	-16.2231839538847\\
1029	-25.7518034659938\\
1030	-24.5857167647321\\
1031	-19.1549948366717\\
1032	-25.890280973053\\
1033	-26.6774396370388\\
1034	-21.5850500272804\\
1035	-18.533994709901\\
1036	-14.6588471849263\\
1038	-20.8450631717931\\
1039	-19.7448277610233\\
1040	-19.7761157890868\\
1041	-21.5524058391782\\
1042	-21.7526774952476\\
1043	-30.7538946358261\\
1044	-27.12890032156\\
1045	-19.5178744036634\\
1046	-18.5805681529498\\
1047	-12.0610411741116\\
1048	-10.2999750131019\\
1049	-15.7150468010727\\
1050	-12.7602215650281\\
1051	-12.5850397400477\\
1052	-17.4265528166513\\
1053	-19.8736230341683\\
1054	-18.355039382159\\
1055	-26.8913231470342\\
1056	-21.8287998214644\\
1057	-12.9337802330333\\
1058	-11.0772729670432\\
1059	-15.0472760567952\\
1060	-16.7695068937683\\
1061	-9.12195192292688\\
1062	-9.25880569868332\\
1063	-13.3126163255295\\
1064	-11.4919311390458\\
1065	-8.8983389720579\\
1066	-12.2608725769464\\
1067	-13.6757045500985\\
1068	-21.7134912400659\\
1069	-20.7413352377007\\
1070	-22.876999501417\\
1071	-22.3395582685648\\
1072	-23.2912366845335\\
1073	-20.6661377930202\\
1074	-19.2822705269921\\
1075	-18.9114699474505\\
1076	-16.8114406779296\\
1077	-18.250531259134\\
1078	-27.4722300328167\\
1079	-41.1956915317974\\
1080	-41.1644074203641\\
1081	-38.5940536777309\\
1082	-25.1168472048248\\
1083	-35.8790023328636\\
1084	-44.8651359524126\\
1085	-43.4739436188563\\
1086	-35.5468151024706\\
1087	-43.1316989800996\\
1088	-57.0964131712067\\
1089	-43.2245950641898\\
1090	-35.8154610926683\\
1091	-27.9854713715567\\
1092	-20.4204965761562\\
1093	-17.8229101224906\\
1094	-21.2905963756875\\
1095	-16.8850867635163\\
1096	-9.91859826133555\\
1097	-9.25562305568428\\
1098	-12.5777851223022\\
1099	-18.6659377689004\\
1100	-17.8937056232514\\
1101	-17.7766342350812\\
1102	-22.9821705734839\\
1103	-23.8452496271252\\
1104	-32.4196876068588\\
1105	-28.5554898695475\\
1106	-30.4562830754653\\
1107	-25.0879823207763\\
1108	-20.7096310517366\\
1109	-25.0598877964183\\
1110	-18.9867640200723\\
1111	-17.4058828284192\\
1112	-20.3633106319396\\
1113	-20.9156181689318\\
1114	-16.669694798429\\
1115	-16.6156161290314\\
1116	-13.2743204904839\\
1117	-13.4103434305835\\
1118	-15.6133386239069\\
1119	-11.625756183955\\
1120	-12.8514088189891\\
1121	-16.0715160847969\\
1122	-24.2864108598064\\
1123	-25.2113864991841\\
1124	-26.2754435839247\\
1125	-14.8738666867737\\
1126	-20.3952875420168\\
1127	-34.3977200104205\\
1128	-25.4718745090884\\
1129	-27.607410055492\\
1130	-28.9446899051809\\
1131	-19.3934761728349\\
1132	-12.5581187045946\\
1133	-17.3695586928613\\
1134	-20.6308518452906\\
1135	-29.2068581700771\\
1136	-38.1881363546709\\
1137	-36.7194682449183\\
1138	-27.2766046997942\\
1139	-28.7825260393831\\
1140	-26.4570933792381\\
1141	-25.9737999207471\\
1142	-26.2942357507002\\
1143	-18.1277737276741\\
1144	-15.4542105069545\\
1145	-15.9671205137267\\
1146	-14.1819698153479\\
1147	-16.0511693704323\\
1148	-21.9196519982108\\
1149	-34.0518760059235\\
1150	-26.7629961941132\\
1151	-18.2659690502692\\
1152	-16.5315367184674\\
1153	-16.5276800715556\\
1154	-10.7986494649515\\
1155	-6.99903008398269\\
1156	-8.16044003316006\\
1157	-15.3922041679307\\
1158	-13.0322482769768\\
1159	-12.2085609049998\\
1160	-14.1378713770171\\
1161	-13.1477511192938\\
1162	-10.3179572152389\\
1163	-7.55585483693812\\
1164	-5.62247989770299\\
1165	-8.20227614828946\\
1166	-19.8096426361926\\
1167	-26.9056097368868\\
1168	-29.9744405245092\\
1169	-18.641910661883\\
1170	-13.9787979455709\\
1171	-10.8165596542844\\
1172	-7.17407802508455\\
1173	-14.0899090834209\\
1174	-16.0051136801883\\
1175	-20.0263502516202\\
1176	-26.3825211462042\\
1177	-41.0400077461338\\
1178	-51.1106315522495\\
1179	-40.8862738415989\\
1180	-39.3504990973922\\
1181	-28.0805937403109\\
1182	-29.1899409265911\\
1183	-31.4947401014713\\
1184	-31.7585590033466\\
1185	-26.9647102207728\\
1186	-23.5596637888739\\
1187	-25.282133849011\\
1188	-21.732171869691\\
1189	-25.0550949338544\\
1190	-20.2097800373685\\
1191	-32.8280162211893\\
1192	-37.9153566500236\\
1193	-27.7371709648642\\
1194	-17.0743979029858\\
1195	-18.6862858706827\\
1196	-29.7663499291225\\
1197	-35.8110560410721\\
1198	-43.2650735722355\\
1199	-44.303017107723\\
1200	-44.6038411615593\\
1201	-36.2233877767824\\
1202	-33.4649667400026\\
1203	-36.9552587430608\\
1204	-40.7611574691125\\
1205	-27.9294460790832\\
1206	-16.8316318939576\\
1207	-24.8942696686404\\
1208	-21.9459914034355\\
1209	-13.2637934239649\\
1210	-19.1412647431564\\
1211	-21.9913243143451\\
1212	-17.9052461779768\\
1213	-14.0433607890295\\
1214	-17.6021474275267\\
1215	-16.253073637017\\
1216	-19.0543995985001\\
1217	-28.2509944929147\\
1218	-23.8207614344515\\
1219	-17.947907302517\\
1220	-18.3563946069762\\
1221	-27.6387924391879\\
1222	-40.8357334041411\\
1223	-31.9784715636019\\
1224	-21.6490549039472\\
1225	-16.8106878257174\\
1226	-17.5530869139286\\
1227	-17.7111764573854\\
1228	-13.1905485499767\\
1229	-14.332054553603\\
1230	-18.1020458651117\\
1231	-21.2371543096594\\
1232	-22.4716427414317\\
1233	-15.9437213509386\\
1234	-25.4326000477483\\
1235	-30.3037310634738\\
1236	-27.1108101678346\\
1237	-22.2377437169091\\
1238	-24.5880953502547\\
1239	-30.9015454908363\\
1240	-20.583159202067\\
1241	-9.40447431029133\\
1242	-12.0234238073131\\
1243	-15.3596852309372\\
1244	-19.0264301788613\\
1245	-19.2872211077647\\
1246	-13.6531978993037\\
1247	-19.8707463246617\\
1248	-20.2761537440322\\
1249	-14.8231206800451\\
1250	-17.8060184079332\\
1251	-16.2069955312581\\
1252	-15.0057754521345\\
1253	-17.8003823575139\\
1254	-10.2387812882248\\
1255	-10.5816892129392\\
1256	-9.74825556827113\\
1257	-9.84404533984298\\
1258	-18.2749196135446\\
1259	-20.5813553115338\\
1260	-26.8554589722364\\
1261	-34.3595876881659\\
1262	-23.6527703597951\\
1263	-14.2501882666343\\
1264	-11.976557976993\\
1265	-11.3669059187639\\
1266	-15.3367169610401\\
1267	-11.1372750879475\\
1268	-10.9338136701037\\
1269	-14.8962509642581\\
1270	-23.3318519925706\\
1271	-23.0089217249949\\
1272	-29.0654600177659\\
1273	-20.4737757683972\\
1274	-18.7896975693918\\
1275	-10.5413711441693\\
1276	-9.50055712042877\\
1277	-7.22480844991946\\
1278	-7.13052589548624\\
1279	-8.56854828849623\\
1280	-15.6268367372443\\
1281	-16.9677336047998\\
1282	-18.1186203986024\\
1283	-18.5875969093311\\
1284	-30.6453799748397\\
1285	-29.0696413425519\\
1286	-19.8015750032832\\
1287	-24.8505051318571\\
1288	-25.3747833910111\\
1289	-30.0156526471785\\
1290	-24.7270443799662\\
1291	-17.6421796934951\\
1292	-9.02879378444732\\
1293	-6.84885007970774\\
1294	-6.75435358185518\\
1295	-8.89087771574918\\
1296	-9.300622738627\\
1297	-8.63302380875302\\
1298	-12.07845607374\\
1299	-16.5926609787439\\
1300	-15.6652030929351\\
1301	-16.3283369068745\\
1302	-17.9461502737497\\
1303	-11.1420542738363\\
1304	-6.79932022493563\\
1305	-12.3340925970488\\
1306	-20.6567096594572\\
1307	-18.2605409077369\\
1308	-19.9989616341752\\
1309	-19.0991268348691\\
1310	-13.7504734229149\\
1311	-17.9799796360344\\
1312	-25.2038758347132\\
1313	-25.392672569649\\
1314	-17.8650427562338\\
1315	-29.8846297420589\\
1316	-25.0653501235981\\
1317	-21.2751589854395\\
1318	-26.0735263564529\\
1319	-24.5847923877686\\
1320	-20.6854187289359\\
1321	-17.7177706351326\\
1322	-26.0573340374935\\
1323	-39.9832838136638\\
1324	-31.1463516872068\\
1325	-17.5085327535839\\
1326	-18.1863502159374\\
1327	-18.5963155524446\\
1328	-23.1720095274891\\
1329	-26.679396222738\\
1330	-19.6155352520861\\
1331	-21.5536855725595\\
1332	-27.1989604002035\\
1333	-29.0149828725321\\
1334	-19.9007005146191\\
1335	-16.8089031981456\\
1336	-21.057324069777\\
1337	-32.6183484268086\\
1338	-32.1994541388226\\
1339	-28.6457984536903\\
1340	-19.5327923147236\\
1341	-16.9616430568924\\
1342	-17.9029396444753\\
1343	-12.3318027743353\\
1344	-7.0648362763327\\
1345	-5.54010715136087\\
1346	-4.8207670929462\\
1347	-11.2147389111674\\
1348	-18.0187861150005\\
1349	-19.5602080394935\\
1350	-15.5593033279843\\
1351	-15.557698746961\\
1352	-18.5931637690842\\
1353	-11.5758229439548\\
1354	-11.964730514572\\
1355	-12.8616682385941\\
1356	-10.0459680011602\\
1357	-11.3908767450459\\
1358	-19.3624800470332\\
1359	-23.9773506533229\\
1360	-14.6785059361066\\
1361	-11.5856670980163\\
1362	-10.0884642142191\\
1363	-12.2042445644026\\
1364	-9.1413995562782\\
1365	-17.7598200201642\\
1366	-31.4526084613242\\
1367	-23.1100683208431\\
1369	-36.5808775788919\\
1370	-34.2834848543127\\
1371	-28.6859715006531\\
1372	-20.1553296158811\\
1373	-21.1594647485774\\
1374	-21.7217555695242\\
1375	-24.4388042698802\\
1376	-27.449657684524\\
1377	-27.1134592910012\\
1378	-35.2399022135858\\
1379	-41.177867793122\\
1380	-46.1965446407098\\
1381	-34.325772384632\\
1382	-34.8601352750179\\
1383	-41.0118329894212\\
1384	-30.7807821607951\\
1385	-36.095311549632\\
1386	-32.3047058923742\\
1387	-18.967973678157\\
1388	-12.0199924847236\\
1389	-12.098981311821\\
1390	-19.1538753254242\\
1391	-16.1645886480528\\
1392	-10.8694727856223\\
1393	-14.6271327890802\\
1394	-13.1418521555865\\
1395	-8.56167088713505\\
1396	-11.053769667499\\
1397	-12.8867107906938\\
1398	-10.1645186329088\\
1399	-15.2641393385366\\
1400	-21.3024864199438\\
1401	-16.9171812987233\\
1402	-25.6047122779742\\
1403	-38.9704777206937\\
1404	-34.9748398418565\\
1405	-42.4420780085477\\
};
\addlegendentry{MPO prediction}

\end{axis}

\begin{axis}[%
width=6.159cm,
height=1.831cm,
at={(8.104cm,10.169cm)},
scale only axis,
xmin=1000,
xmax=1405,
xlabel style={font=\color{white!15!black}},
xlabel={Sample index},
ymin=-50,
ymax=0,
ylabel style={font=\color{white!15!black}},
ylabel={$y(t)$},
axis background/.style={fill=white},
title style={font=\bfseries},
title={C2: RMSE(OSA) = 1.5452, RMSE(MPO) = 1.5631},
legend style={legend cell align=left, align=left, draw=white!15!black}
]
\addplot [color=mycolor1, line width=2.0pt]
  table[row sep=crcr]{%
1006	-24.414\\
1007	-28.076\\
1008	-23.193\\
1009	-13.4280000000001\\
1010	-17.0899999999999\\
1011	-17.0899999999999\\
1012	-18.3109999999999\\
1013	-15.8689999999999\\
1014	-8.54500000000007\\
1015	-6.10400000000004\\
1016	-7.32400000000007\\
1017	-9.76600000000008\\
1018	-21.973\\
1019	-23.193\\
1020	-23.193\\
1021	-19.5309999999999\\
1022	-10.9860000000001\\
1023	-17.0899999999999\\
1024	-19.5309999999999\\
1025	-13.4280000000001\\
1026	-18.3109999999999\\
1027	-19.5309999999999\\
1028	-14.6479999999999\\
1029	-24.414\\
1030	-21.973\\
1031	-15.8689999999999\\
1032	-23.193\\
1033	-25.635\\
1034	-19.5309999999999\\
1036	-14.6479999999999\\
1037	-17.0899999999999\\
1038	-20.752\\
1039	-18.3109999999999\\
1040	-18.3109999999999\\
1042	-20.752\\
1043	-28.076\\
1044	-25.635\\
1045	-17.0899999999999\\
1046	-15.8689999999999\\
1047	-12.2070000000001\\
1048	-10.9860000000001\\
1049	-15.8689999999999\\
1050	-12.2070000000001\\
1051	-12.2070000000001\\
1052	-15.8689999999999\\
1053	-18.3109999999999\\
1054	-15.8689999999999\\
1055	-25.635\\
1056	-21.973\\
1057	-12.2070000000001\\
1058	-9.76600000000008\\
1059	-13.4280000000001\\
1060	-15.8689999999999\\
1061	-10.9860000000001\\
1062	-10.9860000000001\\
1063	-12.2070000000001\\
1064	-9.76600000000008\\
1065	-8.54500000000007\\
1066	-12.2070000000001\\
1068	-17.0899999999999\\
1069	-18.3109999999999\\
1070	-23.193\\
1071	-21.973\\
1072	-21.973\\
1073	-17.0899999999999\\
1075	-17.0899999999999\\
1076	-15.8689999999999\\
1077	-17.0899999999999\\
1078	-24.414\\
1079	-35.4000000000001\\
1080	-36.6210000000001\\
1081	-34.1800000000001\\
1082	-24.414\\
1083	-29.297\\
1084	-36.6210000000001\\
1085	-40.2829999999999\\
1086	-31.7380000000001\\
1087	-37.8420000000001\\
1088	-48.828\\
1089	-39.0630000000001\\
1090	-30.518\\
1091	-25.635\\
1092	-19.5309999999999\\
1093	-14.6479999999999\\
1094	-19.5309999999999\\
1095	-15.8689999999999\\
1096	-9.76600000000008\\
1097	-9.76600000000008\\
1098	-12.2070000000001\\
1099	-17.0899999999999\\
1100	-15.8689999999999\\
1101	-15.8689999999999\\
1102	-19.5309999999999\\
1103	-19.5309999999999\\
1104	-28.076\\
1105	-23.193\\
1106	-25.635\\
1107	-23.193\\
1108	-18.3109999999999\\
1109	-20.752\\
1110	-17.0899999999999\\
1111	-14.6479999999999\\
1112	-18.3109999999999\\
1114	-15.8689999999999\\
1116	-10.9860000000001\\
1118	-13.4280000000001\\
1119	-10.9860000000001\\
1120	-9.76600000000008\\
1122	-19.5309999999999\\
1123	-23.193\\
1124	-23.193\\
1125	-17.0899999999999\\
1126	-18.3109999999999\\
1127	-30.518\\
1128	-23.193\\
1129	-24.414\\
1130	-26.855\\
1131	-17.0899999999999\\
1132	-10.9860000000001\\
1133	-17.0899999999999\\
1134	-18.3109999999999\\
1135	-26.855\\
1136	-34.1800000000001\\
1137	-31.7380000000001\\
1138	-25.635\\
1140	-23.193\\
1141	-23.193\\
1142	-21.973\\
1144	-14.6479999999999\\
1146	-12.2070000000001\\
1147	-13.4280000000001\\
1149	-25.635\\
1150	-21.973\\
1151	-14.6479999999999\\
1152	-13.4280000000001\\
1153	-14.6479999999999\\
1154	-12.2070000000001\\
1155	-8.54500000000007\\
1156	-8.54500000000007\\
1157	-13.4280000000001\\
1158	-12.2070000000001\\
1161	-12.2070000000001\\
1162	-8.54500000000007\\
1163	-7.32400000000007\\
1164	-4.88300000000004\\
1165	-7.32400000000007\\
1166	-17.0899999999999\\
1167	-24.414\\
1168	-24.414\\
1169	-19.5309999999999\\
1170	-12.2070000000001\\
1171	-12.2070000000001\\
1172	-8.54500000000007\\
1173	-10.9860000000001\\
1174	-14.6479999999999\\
1175	-14.6479999999999\\
1177	-34.1800000000001\\
1178	-40.2829999999999\\
1179	-32.9590000000001\\
1180	-31.7380000000001\\
1181	-23.193\\
1182	-25.635\\
1184	-28.076\\
1185	-21.973\\
1187	-19.5309999999999\\
1188	-19.5309999999999\\
1189	-21.973\\
1190	-19.5309999999999\\
1191	-23.193\\
1192	-29.297\\
1193	-24.414\\
1194	-14.6479999999999\\
1195	-15.8689999999999\\
1196	-25.635\\
1197	-32.9590000000001\\
1198	-35.4000000000001\\
1199	-39.0630000000001\\
1200	-37.8420000000001\\
1201	-30.518\\
1202	-26.855\\
1203	-30.518\\
1204	-35.4000000000001\\
1205	-25.635\\
1206	-14.6479999999999\\
1207	-19.5309999999999\\
1208	-20.752\\
1209	-12.2070000000001\\
1210	-13.4280000000001\\
1211	-18.3109999999999\\
1212	-14.6479999999999\\
1213	-12.2070000000001\\
1214	-18.3109999999999\\
1215	-10.9860000000001\\
1216	-14.6479999999999\\
1217	-24.414\\
1218	-21.973\\
1219	-14.6479999999999\\
1220	-14.6479999999999\\
1221	-20.752\\
1222	-34.1800000000001\\
1223	-28.076\\
1224	-15.8689999999999\\
1225	-14.6479999999999\\
1226	-14.6479999999999\\
1227	-12.2070000000001\\
1228	-10.9860000000001\\
1229	-10.9860000000001\\
1230	-13.4280000000001\\
1231	-17.0899999999999\\
1232	-19.5309999999999\\
1233	-14.6479999999999\\
1234	-19.5309999999999\\
1235	-26.855\\
1236	-23.193\\
1237	-17.0899999999999\\
1238	-21.973\\
1239	-25.635\\
1240	-21.973\\
1241	-8.54500000000007\\
1242	-12.2070000000001\\
1243	-13.4280000000001\\
1244	-15.8689999999999\\
1245	-15.8689999999999\\
1246	-10.9860000000001\\
1247	-17.0899999999999\\
1248	-15.8689999999999\\
1249	-10.9860000000001\\
1250	-14.6479999999999\\
1251	-14.6479999999999\\
1252	-12.2070000000001\\
1253	-14.6479999999999\\
1254	-10.9860000000001\\
1255	-9.76600000000008\\
1256	-12.2070000000001\\
1257	-8.54500000000007\\
1258	-17.0899999999999\\
1259	-18.3109999999999\\
1260	-20.752\\
1261	-29.297\\
1262	-20.752\\
1263	-10.9860000000001\\
1264	-10.9860000000001\\
1265	-8.54500000000007\\
1266	-13.4280000000001\\
1267	-12.2070000000001\\
1268	-9.76600000000008\\
1269	-15.8689999999999\\
1270	-19.5309999999999\\
1271	-20.752\\
1272	-23.193\\
1273	-23.193\\
1274	-17.0899999999999\\
1275	-15.8689999999999\\
1276	-10.9860000000001\\
1277	-10.9860000000001\\
1278	-7.32400000000007\\
1279	-8.54500000000007\\
1280	-10.9860000000001\\
1281	-17.0899999999999\\
1282	-18.3109999999999\\
1283	-17.0899999999999\\
1284	-24.414\\
1285	-25.635\\
1286	-15.8689999999999\\
1289	-26.855\\
1290	-25.635\\
1291	-17.0899999999999\\
1292	-10.9860000000001\\
1293	-7.32400000000007\\
1294	-6.10400000000004\\
1295	-8.54500000000007\\
1296	-9.76600000000008\\
1297	-8.54500000000007\\
1298	-10.9860000000001\\
1299	-17.0899999999999\\
1300	-15.8689999999999\\
1301	-13.4280000000001\\
1302	-15.8689999999999\\
1303	-12.2070000000001\\
1304	-6.10400000000004\\
1305	-9.76600000000008\\
1306	-20.752\\
1307	-17.0899999999999\\
1308	-18.3109999999999\\
1309	-15.8689999999999\\
1310	-10.9860000000001\\
1312	-23.193\\
1313	-23.193\\
1314	-17.0899999999999\\
1315	-26.855\\
1316	-25.635\\
1317	-18.3109999999999\\
1318	-23.193\\
1319	-25.635\\
1320	-19.5309999999999\\
1321	-15.8689999999999\\
1322	-24.414\\
1323	-35.4000000000001\\
1324	-29.297\\
1325	-17.0899999999999\\
1326	-17.0899999999999\\
1327	-18.3109999999999\\
1329	-23.193\\
1330	-19.5309999999999\\
1331	-17.0899999999999\\
1332	-24.414\\
1333	-25.635\\
1334	-18.3109999999999\\
1335	-15.8689999999999\\
1336	-18.3109999999999\\
1337	-28.076\\
1338	-26.855\\
1339	-24.414\\
1340	-19.5309999999999\\
1341	-17.0899999999999\\
1342	-18.3109999999999\\
1343	-13.4280000000001\\
1344	-6.10400000000004\\
1346	-6.10400000000004\\
1347	-9.76600000000008\\
1348	-19.5309999999999\\
1349	-15.8689999999999\\
1351	-13.4280000000001\\
1352	-17.0899999999999\\
1354	-9.76600000000008\\
1355	-12.2070000000001\\
1356	-9.76600000000008\\
1357	-8.54500000000007\\
1358	-17.0899999999999\\
1359	-20.752\\
1360	-17.0899999999999\\
1361	-10.9860000000001\\
1362	-10.9860000000001\\
1363	-13.4280000000001\\
1364	-9.76600000000008\\
1365	-10.9860000000001\\
1366	-28.076\\
1367	-20.752\\
1368	-25.635\\
1369	-35.4000000000001\\
1370	-30.518\\
1372	-18.3109999999999\\
1373	-18.3109999999999\\
1374	-19.5309999999999\\
1375	-21.973\\
1376	-25.635\\
1377	-25.635\\
1378	-31.7380000000001\\
1379	-34.1800000000001\\
1380	-41.5039999999999\\
1381	-32.9590000000001\\
1382	-28.076\\
1383	-35.4000000000001\\
1384	-26.855\\
1385	-29.297\\
1386	-28.076\\
1387	-17.0899999999999\\
1388	-10.9860000000001\\
1389	-12.2070000000001\\
1390	-17.0899999999999\\
1391	-14.6479999999999\\
1392	-9.76600000000008\\
1393	-13.4280000000001\\
1394	-13.4280000000001\\
1395	-7.32400000000007\\
1396	-8.54500000000007\\
1397	-12.2070000000001\\
1398	-7.32400000000007\\
1399	-14.6479999999999\\
1400	-20.752\\
1401	-13.4280000000001\\
1403	-30.518\\
1404	-29.297\\
1405	-35.4000000000001\\
};
\addlegendentry{True output}

\addplot [color=mycolor2, dashed, line width=2.0pt]
  table[row sep=crcr]{%
1006	-24.5192122402532\\
1007	-26.6277885528164\\
1008	-23.4342653885014\\
1009	-13.1738057908742\\
1010	-15.5068595562459\\
1011	-17.4156327787528\\
1012	-17.7106683869799\\
1013	-17.1123404759057\\
1014	-7.81679636714625\\
1015	-6.42585167004813\\
1016	-6.11855903763626\\
1017	-13.6452442546772\\
1018	-21.4168089231403\\
1019	-20.9816378275495\\
1020	-21.2851054458656\\
1021	-17.2790022293648\\
1022	-12.2225520590887\\
1023	-22.4708634750045\\
1024	-17.8900663675774\\
1025	-11.1026688142447\\
1026	-19.2993575338644\\
1027	-19.4689242771492\\
1028	-15.4586186024897\\
1029	-22.0728064675438\\
1030	-23.65865587335\\
1031	-16.9129596673963\\
1032	-23.3933541205961\\
1033	-23.1376029713401\\
1034	-18.6278406558285\\
1035	-16.8958427208463\\
1036	-14.6409275126377\\
1037	-16.214324800749\\
1038	-19.4843831125293\\
1039	-18.5864377382049\\
1040	-19.1186598061443\\
1041	-18.8507486485432\\
1042	-19.9244193690035\\
1043	-26.1371652268281\\
1044	-24.8278511302324\\
1045	-17.6849977729164\\
1046	-17.1815123501633\\
1047	-11.2809806299078\\
1048	-10.2925969418077\\
1049	-14.6992425506628\\
1050	-13.214592086985\\
1051	-11.9338085652014\\
1052	-16.46592831099\\
1053	-18.2119822597269\\
1054	-16.7325685803414\\
1055	-23.275577271982\\
1056	-20.6265059200296\\
1057	-12.6226719177819\\
1058	-11.3830372871796\\
1059	-13.9839018705643\\
1060	-15.1094078763717\\
1061	-9.54306447718977\\
1062	-9.2445084849976\\
1063	-13.5767050531292\\
1064	-12.5016103802027\\
1065	-8.76556507584633\\
1066	-11.254593080474\\
1067	-13.3253939564095\\
1068	-18.4760816893354\\
1069	-18.9181508684926\\
1070	-19.4048260487282\\
1071	-19.6446837795834\\
1072	-23.2559311336465\\
1073	-20.5467673610744\\
1074	-17.0640322928934\\
1075	-16.1624130908667\\
1076	-16.09959010175\\
1077	-15.8754944511995\\
1078	-24.3089408705644\\
1079	-34.1005770016866\\
1080	-34.64963481208\\
1081	-34.7979104975652\\
1082	-23.2843385569761\\
1083	-31.3904580142166\\
1084	-37.7933433513217\\
1085	-36.336168249413\\
1086	-29.9897663551619\\
1087	-37.9989861880949\\
1088	-49.5959185116551\\
1089	-36.5118021500371\\
1090	-32.8937043318988\\
1091	-22.8700357547871\\
1092	-18.0118206597006\\
1093	-16.2713431497948\\
1094	-18.9854872641911\\
1095	-16.6100137214657\\
1096	-9.6545044312661\\
1097	-9.23156260468636\\
1098	-12.1102216647057\\
1099	-17.3728609435329\\
1100	-16.5793316284082\\
1101	-16.0797653955894\\
1102	-20.2356216995991\\
1103	-21.6270298485254\\
1104	-26.1285002720711\\
1105	-24.0386938213517\\
1106	-26.2068655866997\\
1107	-22.8676880187584\\
1108	-18.7561945020634\\
1109	-21.6118983843\\
1110	-17.771391715665\\
1111	-15.226841311292\\
1112	-18.4002330615497\\
1113	-18.2223209272781\\
1114	-14.1643535819023\\
1115	-14.6295884309782\\
1116	-12.2402686534451\\
1117	-12.5273949222797\\
1118	-13.0710997002668\\
1119	-10.40963798503\\
1120	-11.9852465073543\\
1121	-14.869012930352\\
1122	-20.3004283872435\\
1123	-21.0366305072348\\
1124	-22.2367791700885\\
1125	-15.6808296967267\\
1126	-20.4780548832289\\
1127	-29.3257948799144\\
1128	-24.3520365858365\\
1129	-24.8535093225985\\
1130	-27.2245918983899\\
1131	-15.9046031415762\\
1132	-12.7003083270681\\
1133	-15.7573973326896\\
1134	-18.6346144055326\\
1135	-27.2900682958891\\
1136	-32.4501806428668\\
1137	-31.8970689369835\\
1138	-26.18594235572\\
1139	-25.8566976360164\\
1140	-23.2684226232061\\
1141	-22.8340445330155\\
1142	-23.200385786178\\
1143	-17.2535711774269\\
1144	-14.7522752677737\\
1145	-14.1895278002012\\
1146	-13.0868275459648\\
1147	-14.3916202595249\\
1148	-18.8441835578606\\
1149	-26.8999415187989\\
1150	-22.7661505101605\\
1151	-16.1846204929998\\
1152	-14.9882579486211\\
1153	-14.198686286408\\
1154	-9.47577881611437\\
1155	-7.3368964206752\\
1156	-8.75375048887327\\
1157	-14.6428495561877\\
1158	-12.6137585703884\\
1159	-11.7532739928126\\
1160	-12.8832843054511\\
1161	-12.6865058343733\\
1162	-9.80224374206273\\
1163	-7.4484509065403\\
1164	-6.08885416283374\\
1165	-8.14759679067288\\
1166	-17.0875932378353\\
1167	-22.6327893563825\\
1168	-24.4194240440911\\
1169	-18.479508805483\\
1170	-13.2502954166448\\
1171	-10.4897393817043\\
1172	-8.22444499523021\\
1173	-14.2858684377115\\
1174	-15.237350316969\\
1175	-17.1007297149065\\
1176	-21.8325638982808\\
1177	-34.6341934184852\\
1178	-40.3639344177002\\
1179	-35.0212772906325\\
1180	-33.8690677854422\\
1181	-21.5947716827084\\
1182	-25.1926470369967\\
1183	-26.295949115301\\
1184	-27.7804428749894\\
1185	-23.7518397853637\\
1186	-20.7768424435528\\
1187	-21.2888509640638\\
1188	-19.1983426545571\\
1189	-21.544792768085\\
1190	-17.9271340705143\\
1191	-26.4964495625372\\
1192	-32.0099084863459\\
1193	-22.9694648796428\\
1194	-15.045435017963\\
1195	-15.8390262707931\\
1196	-26.0229055791685\\
1197	-30.354273701535\\
1198	-35.315092500686\\
1199	-38.1980803532022\\
1200	-38.2573233296916\\
1201	-30.4040840409443\\
1202	-28.576526137075\\
1203	-31.2775663885377\\
1204	-34.9642339351471\\
1205	-24.1880613685071\\
1206	-15.3641937051987\\
1207	-21.6039340663976\\
1208	-19.9554082365694\\
1209	-12.8826125724995\\
1210	-15.4726798329052\\
1211	-18.9877530637584\\
1212	-15.359279648633\\
1213	-12.5810212152155\\
1214	-14.8518576202816\\
1215	-15.27526384211\\
1216	-16.8248426207301\\
1217	-23.1895722222421\\
1218	-21.1421729954554\\
1219	-16.2310039596532\\
1220	-16.1120957366202\\
1221	-22.0575835956133\\
1222	-34.7598850518114\\
1223	-26.4047159708155\\
1224	-17.26623656425\\
1225	-14.24917846528\\
1226	-16.9677786622287\\
1227	-15.651626061356\\
1228	-11.8956748512371\\
1229	-11.9002772052843\\
1230	-14.3969144820255\\
1231	-17.2724268318673\\
1232	-19.6849534774447\\
1233	-14.314198111646\\
1234	-21.2525743065376\\
1235	-25.780752595615\\
1236	-22.3806594706427\\
1237	-20.4576387458999\\
1238	-20.3341868824687\\
1239	-26.6193089771696\\
1240	-19.0146504120135\\
1241	-9.218244642738\\
1242	-11.5799904332698\\
1243	-15.2360564396092\\
1244	-17.863926271122\\
1245	-16.2389868657442\\
1246	-12.484093248929\\
1247	-17.6055781323562\\
1248	-18.1300931707656\\
1249	-13.1794932583043\\
1250	-14.6398473151398\\
1251	-14.9672303058098\\
1252	-12.6634760894401\\
1253	-14.6743321957354\\
1254	-10.8269138291569\\
1255	-10.9392616080111\\
1256	-9.72945115506263\\
1257	-10.244024964343\\
1258	-15.3363293150703\\
1259	-18.3496156446204\\
1260	-21.9940188238509\\
1261	-29.8954129662202\\
1262	-21.6164922015928\\
1263	-12.6575017907335\\
1264	-10.9346463859895\\
1265	-10.6833766885065\\
1266	-13.9341078548673\\
1267	-10.1797309282804\\
1268	-10.1795298518409\\
1269	-13.7836663450682\\
1270	-22.8754834059721\\
1271	-20.8711932252761\\
1272	-24.9465967392573\\
1273	-18.9802665348213\\
1274	-16.9814825125397\\
1275	-11.0743016503279\\
1276	-10.5777455007683\\
1277	-9.4823533423787\\
1278	-9.04191919921504\\
1279	-10.3936480958964\\
1280	-14.1304931356112\\
1281	-15.6908806874301\\
1282	-15.1119289781921\\
1283	-16.9276099395975\\
1284	-26.9196896741705\\
1285	-24.9678089671525\\
1286	-17.937900816436\\
1287	-20.7397425574841\\
1288	-21.6091299453535\\
1289	-26.5455893152687\\
1290	-22.3327432382014\\
1291	-16.3185934698492\\
1292	-9.93652871725044\\
1293	-8.26170974592378\\
1294	-8.36641503768988\\
1295	-9.23006809605295\\
1296	-9.14111401651121\\
1297	-8.68421331302625\\
1298	-10.5715550128348\\
1299	-16.0644182038227\\
1300	-15.1957504141819\\
1301	-14.9399160188868\\
1302	-16.5109431550845\\
1303	-11.4109287569622\\
1304	-6.98905151594363\\
1305	-11.2974404714862\\
1306	-18.2770625654609\\
1307	-16.8069076381259\\
1308	-17.8881504207366\\
1309	-17.1662138466452\\
1310	-13.3501128825656\\
1311	-16.5866465606268\\
1312	-21.5980660741131\\
1313	-22.0780471498558\\
1314	-15.9258480760893\\
1315	-25.112074695671\\
1316	-23.9100348229542\\
1317	-20.0029575172991\\
1318	-23.8787277578583\\
1319	-23.2502918415717\\
1320	-18.8251419143626\\
1321	-16.9464145663019\\
1322	-26.1038050536627\\
1323	-33.4114306387582\\
1324	-27.0169766201197\\
1325	-17.133200162978\\
1326	-17.7092122924371\\
1327	-18.0634835233163\\
1328	-20.6384694095661\\
1329	-23.5844951473521\\
1330	-18.6170762667582\\
1331	-19.1336956398509\\
1332	-23.7837423340379\\
1333	-25.2689761739603\\
1334	-18.6044837331831\\
1335	-15.1514209545769\\
1336	-19.0622885060995\\
1337	-29.7557556104794\\
1338	-26.2290348294025\\
1339	-24.7774738609908\\
1340	-18.0892880607819\\
1341	-15.3295231762297\\
1342	-16.1792380991508\\
1343	-12.499217295715\\
1344	-8.40016076485813\\
1345	-6.91219895232348\\
1346	-6.38565761484938\\
1347	-10.8300766925233\\
1348	-15.4817131683583\\
1349	-18.5154451218473\\
1350	-14.2468529493351\\
1351	-14.9864270686314\\
1352	-17.0090090979975\\
1353	-11.3291475391043\\
1354	-11.36574631297\\
1355	-12.4143097192161\\
1356	-9.71144431078892\\
1357	-11.4058869465234\\
1358	-17.4215566770188\\
1359	-20.337050761935\\
1360	-13.7295378152678\\
1361	-11.1147312994499\\
1362	-11.3392065997284\\
1363	-12.0932177987702\\
1364	-9.70898219629248\\
1365	-16.3646840966094\\
1366	-26.8211772029902\\
1367	-20.5796131034472\\
1368	-25.1395529834558\\
1369	-31.0275663546636\\
1370	-30.6635834703079\\
1371	-25.3351167106978\\
1372	-19.3946058423312\\
1373	-18.8600986809513\\
1374	-19.3268319663375\\
1375	-21.4245463366844\\
1376	-23.9395180004628\\
1377	-24.0044300864693\\
1378	-30.758062125889\\
1379	-35.9693665377804\\
1380	-39.6348501325299\\
1381	-29.976522076289\\
1382	-30.7233955612037\\
1383	-35.1452430089869\\
1384	-28.6205591288126\\
1385	-30.4010562115902\\
1386	-27.9378556425568\\
1387	-16.0200984962655\\
1388	-11.0575504896799\\
1389	-11.4574894913935\\
1390	-16.6425939099111\\
1391	-15.1860039969936\\
1392	-10.5198627597024\\
1393	-13.8617436837078\\
1394	-12.2219089556538\\
1395	-8.99397112953898\\
1396	-10.7224682683573\\
1397	-12.0126678893948\\
1398	-9.61971302946768\\
1399	-13.6206904818223\\
1400	-19.5639509496264\\
1401	-14.7962548697726\\
1402	-24.0839263972762\\
1403	-34.4296612620979\\
1404	-28.5303599116482\\
1405	-33.8299220730519\\
};
\addlegendentry{OSA predition}

\addplot [color=mycolor3, dotted, line width=2.0pt]
  table[row sep=crcr]{%
1006	-24.414\\
1007	-28.076\\
1008	-23.193\\
1009	-13.4280000000001\\
1010	-15.5068595562461\\
1011	-17.1847031180312\\
1012	-17.433281705943\\
1013	-16.7525491016265\\
1014	-7.72764797090122\\
1015	-6.40305417813624\\
1016	-6.06948565647485\\
1017	-13.3644657667048\\
1018	-21.8103711525121\\
1019	-21.2890404020602\\
1020	-21.605401481715\\
1021	-16.8443742353577\\
1022	-11.341902406567\\
1023	-21.5996407628204\\
1024	-18.1198509086569\\
1025	-11.6306940295665\\
1026	-19.6868540559749\\
1027	-19.2795978726663\\
1028	-15.3547687583598\\
1029	-22.342939582144\\
1030	-23.5262447937885\\
1031	-16.827397898596\\
1032	-23.4316266523983\\
1033	-23.5498701563452\\
1034	-18.6112254906068\\
1035	-16.4466911995567\\
1036	-14.1037862277783\\
1037	-15.7795625074818\\
1038	-19.0377763708316\\
1039	-17.9969935533754\\
1040	-18.5189274939933\\
1041	-18.4865241717762\\
1042	-19.6916246290923\\
1043	-25.8308576147522\\
1044	-24.1803241999569\\
1045	-16.9070623990478\\
1046	-16.504134004068\\
1047	-11.0930507786927\\
1048	-10.1589095286004\\
1049	-14.4694234261954\\
1050	-12.7339444852516\\
1051	-11.6106716120682\\
1052	-16.1798958990028\\
1053	-18.1919534574088\\
1054	-16.6957298300906\\
1055	-23.4175982470131\\
1056	-20.4875258723957\\
1057	-12.0978913674585\\
1058	-10.862356995746\\
1059	-13.8065873094965\\
1060	-15.2844533316159\\
1061	-9.63189285422027\\
1062	-9.02156284877105\\
1063	-12.9497557796142\\
1064	-12.0637053938915\\
1065	-8.96599076234861\\
1066	-11.7540551288773\\
1067	-13.7290547691834\\
1068	-18.4413115227019\\
1069	-18.8902133490858\\
1070	-19.5389142605854\\
1071	-19.4460067064977\\
1072	-22.4124597838561\\
1073	-19.5860669208514\\
1074	-16.9823593377803\\
1075	-16.5779515028548\\
1076	-16.4117523589598\\
1077	-15.9449179989297\\
1078	-24.1225808435647\\
1079	-33.9797451187944\\
1080	-34.2246160334601\\
1081	-34.2075934920852\\
1082	-22.6131274128099\\
1083	-30.5978201953963\\
1084	-37.4845224696894\\
1085	-36.2677023518702\\
1086	-29.9990362116921\\
1087	-37.2455717779915\\
1088	-48.1162272097852\\
1089	-35.8840518612753\\
1090	-32.312471120739\\
1091	-22.2639130406308\\
1092	-17.4962830793709\\
1093	-15.421905485347\\
1094	-18.175465298749\\
1095	-16.122455866363\\
1096	-9.43073629347305\\
1097	-9.09297046189818\\
1098	-11.9277402568907\\
1099	-17.1544597070565\\
1100	-16.4151259778241\\
1101	-16.0989171461811\\
1102	-20.3692986725177\\
1103	-21.8903854213277\\
1104	-26.6751637845668\\
1105	-24.4684036517174\\
1106	-26.6059039038444\\
1107	-23.0948909694182\\
1108	-19.0595806863016\\
1109	-21.9266253139965\\
1110	-18.0984053169439\\
1111	-15.6957853986273\\
1112	-18.9862961321442\\
1113	-18.7290351210818\\
1114	-14.7053316666947\\
1115	-14.9059019990539\\
1116	-12.48443077964\\
1117	-12.919894770333\\
1118	-13.6281182583807\\
1119	-10.8096485170111\\
1120	-12.1614614601415\\
1121	-15.2698737340984\\
1122	-20.7793698325715\\
1123	-21.7714247299241\\
1124	-22.4821386117083\\
1125	-15.5059823997794\\
1126	-19.9058607589923\\
1127	-29.0267019517003\\
1128	-24.1721229153457\\
1129	-24.9354868603064\\
1130	-27.250560275502\\
1131	-16.1024125994452\\
1132	-12.7320433769182\\
1133	-15.9193630312545\\
1134	-18.6287359948815\\
1135	-27.5104040230772\\
1136	-32.3745879545179\\
1137	-31.858879161658\\
1138	-25.9723904429372\\
1139	-25.6315177538563\\
1140	-23.4361750707976\\
1141	-23.1688307912787\\
1142	-23.4777767045668\\
1143	-17.5298460586478\\
1144	-14.9374887498013\\
1145	-14.2983403925414\\
1146	-13.2225132585897\\
1147	-14.7214548271986\\
1148	-19.403912918373\\
1149	-27.3937678646623\\
1150	-23.2222522298816\\
1151	-16.7099291449408\\
1152	-15.7765435561043\\
1153	-15.1975435291697\\
1154	-10.2651209738988\\
1155	-7.4452168799105\\
1156	-8.39327312276077\\
1157	-14.0851734804749\\
1158	-12.4527617083825\\
1159	-11.8342612501392\\
1160	-12.9618991567484\\
1161	-12.7991421512465\\
1162	-9.98915752707671\\
1163	-7.83979214443298\\
1164	-6.50743523713186\\
1165	-8.72810180803754\\
1166	-17.7569157982803\\
1167	-23.3850355345141\\
1168	-24.7603656920901\\
1169	-18.4927603518272\\
1170	-13.0811751766257\\
1171	-10.4228024778431\\
1172	-8.02470881486988\\
1173	-14.0079917642925\\
1174	-15.3511771951712\\
1175	-17.5403516963743\\
1176	-23.002334728259\\
1177	-35.2821176697294\\
1178	-41.0340693427936\\
1179	-35.1152448843407\\
1180	-34.3806590860843\\
1181	-22.4408011691191\\
1182	-26.0593801657942\\
1183	-26.7858500713103\\
1184	-27.812369315159\\
1185	-23.7478277451751\\
1186	-20.9646200130046\\
1187	-21.619609969955\\
1188	-19.8032292414316\\
1189	-22.0953464190504\\
1190	-18.3190887690364\\
1191	-26.4937237148004\\
1192	-32.2340882435922\\
1193	-23.7225690578086\\
1194	-15.8551394239755\\
1195	-16.4552947063457\\
1196	-26.3049556601622\\
1197	-30.8001335831864\\
1198	-35.4052742835477\\
1199	-37.9615199943084\\
1200	-37.7322758948271\\
1201	-30.0867241237006\\
1202	-28.3130272719295\\
1203	-31.2940883475749\\
1204	-35.2409222205747\\
1205	-24.5311492119647\\
1206	-15.3358194911673\\
1207	-21.4818171701122\\
1208	-20.1689828719545\\
1209	-13.2104795659534\\
1210	-15.8612537662159\\
1211	-19.5118584025922\\
1212	-16.1020988599303\\
1213	-13.3837426901598\\
1214	-15.5959041572237\\
1215	-15.3730760080111\\
1216	-17.2598642961443\\
1217	-23.8271951066833\\
1218	-22.2145571053356\\
1219	-16.686392353422\\
1220	-16.4742556904646\\
1221	-22.7370208539935\\
1222	-35.9508728108633\\
1223	-27.3318239541456\\
1224	-17.8118565042528\\
1225	-14.6837029956744\\
1226	-17.3022285636512\\
1227	-16.3169307441426\\
1228	-13.0802572801215\\
1229	-13.3685674938617\\
1230	-15.9479281395929\\
1231	-18.6281315002486\\
1232	-20.8949297707409\\
1233	-15.2055827480167\\
1234	-22.0001090028436\\
1235	-26.5746282886046\\
1236	-22.9478726924885\\
1237	-20.7911167276388\\
1238	-20.8601680583156\\
1239	-27.1104114708512\\
1240	-19.5007444449127\\
1241	-9.2317740434778\\
1242	-11.3683621807347\\
1243	-14.8867771273544\\
1244	-17.9643225375401\\
1245	-16.6909114274399\\
1246	-13.130682772083\\
1247	-18.5378077817579\\
1248	-18.9522044735554\\
1249	-14.1960176983737\\
1250	-16.0096111802591\\
1251	-16.301271750006\\
1252	-13.7185112783441\\
1253	-15.5542674978699\\
1254	-11.4811656005327\\
1255	-11.4528343954871\\
1256	-10.2507687532566\\
1257	-10.34247717933\\
1258	-15.6165205609793\\
1259	-18.1411868519365\\
1260	-22.0171736418938\\
1261	-29.7490958904077\\
1262	-21.8096011342452\\
1263	-12.9980078922406\\
1264	-11.54911051562\\
1265	-11.3081595474716\\
1266	-14.8585059965169\\
1267	-11.0270247968231\\
1268	-10.6451968607626\\
1269	-14.0611638071352\\
1270	-22.5403565474001\\
1271	-21.0824968970549\\
1272	-25.1742762114184\\
1273	-19.6650329264137\\
1274	-17.0413959631019\\
1275	-10.6042011856871\\
1276	-9.45410887350181\\
1277	-8.12970711334788\\
1278	-7.70962623356036\\
1279	-9.5312790616515\\
1280	-13.7625665776354\\
1281	-16.2329008287334\\
1282	-15.6262694842399\\
1283	-16.9037364799021\\
1284	-26.3255276301336\\
1285	-24.6284239635302\\
1286	-18.0210761471621\\
1287	-21.1937295088846\\
1288	-22.1422498044112\\
1289	-27.141061739741\\
1290	-22.5289249032103\\
1291	-15.9118850873238\\
1292	-9.14878181003746\\
1293	-7.48531466164218\\
1294	-7.78375305398231\\
1295	-9.22677937405251\\
1296	-9.47635742359444\\
1297	-8.97587831210922\\
1298	-10.7629306931344\\
1299	-16.0717038179159\\
1300	-15.0681517903572\\
1301	-14.6332204332009\\
1302	-16.3723499886275\\
1303	-11.56405110879\\
1304	-7.04096490922257\\
1305	-11.4632930732027\\
1306	-18.5877441080129\\
1307	-16.9212939171036\\
1308	-17.8295616995629\\
1309	-16.8028048102974\\
1310	-13.2911114030828\\
1311	-17.0153528394605\\
1312	-22.1751361328311\\
1313	-22.4056309695825\\
1314	-15.7225219455258\\
1315	-24.5452130177812\\
1316	-23.1872691465369\\
1317	-19.0299756686766\\
1318	-23.0245147932553\\
1319	-22.8338308211009\\
1320	-18.443118473074\\
1321	-16.2691780090611\\
1322	-25.3295602786254\\
1323	-33.2148809541493\\
1324	-27.0057907329956\\
1325	-16.5677395697853\\
1326	-16.8981808522014\\
1327	-17.4761489272582\\
1328	-20.2933110160236\\
1329	-23.307624353803\\
1330	-18.4158305756225\\
1331	-18.8890874741396\\
1332	-23.8328776220783\\
1333	-25.3149822915839\\
1334	-18.6650294569686\\
1335	-15.1373650996775\\
1336	-18.9422207745179\\
1337	-29.7824741106597\\
1338	-26.4146936503087\\
1339	-25.1243452653459\\
1340	-18.3481830246142\\
1341	-15.3117769581022\\
1342	-15.7818448677383\\
1343	-11.6859889609898\\
1344	-7.46206646434962\\
1345	-6.48813680310786\\
1346	-6.4020047072936\\
1348	-15.8709561806124\\
1349	-18.2848441018091\\
1350	-14.1765099149304\\
1351	-14.8309781704809\\
1352	-17.4048305047497\\
1353	-11.6684669401604\\
1354	-11.3226841474429\\
1355	-12.4404399216996\\
1356	-9.82654863539869\\
1357	-11.6242574711857\\
1358	-18.0196548904223\\
1359	-21.0149921523118\\
1360	-14.3130577889201\\
1361	-10.9811016676749\\
1362	-10.9006849694169\\
1363	-11.6931818191019\\
1364	-9.31807025700982\\
1365	-15.97911005275\\
1366	-27.0261658224765\\
1367	-21.1525827471953\\
1368	-26.026085308109\\
1369	-31.1187853604997\\
1370	-30.353725749218\\
1372	-18.8234270132516\\
1373	-18.7668248647662\\
1374	-19.5122166424555\\
1375	-21.6100578638645\\
1376	-23.9971377273175\\
1377	-23.7446526531937\\
1378	-30.1270049147211\\
1379	-35.034794990795\\
1380	-38.8850727274287\\
1381	-29.6138189300216\\
1382	-29.9369212580241\\
1383	-34.1395108360866\\
1384	-28.183354249034\\
1385	-30.544156145497\\
1386	-28.1876528806108\\
1387	-16.3812310506796\\
1388	-11.1555309889698\\
1389	-11.4076678507417\\
1390	-16.4272642918679\\
1391	-14.9414255808717\\
1392	-10.3600854369267\\
1393	-13.8658538338755\\
1394	-12.4083103648722\\
1395	-8.99478119569176\\
1396	-10.8913084907379\\
1397	-12.5334138154844\\
1398	-10.2409590401664\\
1399	-14.6269414712569\\
1400	-20.2283416297996\\
1401	-15.1397947288756\\
1402	-24.3604422715473\\
1403	-34.9102978326171\\
1404	-29.7139524141623\\
1405	-35.3060352508978\\
};
\addlegendentry{MPO prediction}

\end{axis}

\begin{axis}[%
width=6.159cm,
height=1.831cm,
at={(0cm,7.627cm)},
scale only axis,
xmin=1000,
xmax=1405,
xlabel style={font=\color{white!15!black}},
xlabel={Sample index},
ymin=-40.283,
ymax=0,
ylabel style={font=\color{white!15!black}},
ylabel={$y(t)$},
axis background/.style={fill=white},
title style={font=\bfseries},
title={C3: RMSE(OSA) = 1.6731, RMSE(MPO) = 1.7983},
legend style={legend cell align=left, align=left, draw=white!15!black}
]
\addplot [color=mycolor1, line width=2.0pt]
  table[row sep=crcr]{%
1006	-18.3109999999999\\
1007	-23.193\\
1008	-18.3109999999999\\
1009	-9.76600000000008\\
1010	-15.8689999999999\\
1011	-12.2070000000001\\
1012	-14.6479999999999\\
1013	-10.9860000000001\\
1014	-3.66200000000003\\
1015	-2.44100000000003\\
1016	-4.88300000000004\\
1017	-6.10400000000004\\
1018	-14.6479999999999\\
1019	-17.0899999999999\\
1020	-17.0899999999999\\
1022	-9.76600000000008\\
1023	-18.3109999999999\\
1025	-10.9860000000001\\
1026	-13.4280000000001\\
1028	-10.9860000000001\\
1029	-20.752\\
1030	-19.5309999999999\\
1031	-12.2070000000001\\
1032	-20.752\\
1033	-20.752\\
1034	-15.8689999999999\\
1035	-13.4280000000001\\
1036	-9.76600000000008\\
1037	-14.6479999999999\\
1041	-14.6479999999999\\
1042	-15.8689999999999\\
1043	-21.973\\
1044	-20.752\\
1045	-13.4280000000001\\
1046	-13.4280000000001\\
1047	-8.54500000000007\\
1048	-12.2070000000001\\
1049	-12.2070000000001\\
1050	-13.4280000000001\\
1051	-10.9860000000001\\
1052	-13.4280000000001\\
1053	-14.6479999999999\\
1054	-13.4280000000001\\
1055	-20.752\\
1056	-13.4280000000001\\
1057	-9.76600000000008\\
1058	-7.32400000000007\\
1059	-8.54500000000007\\
1060	-10.9860000000001\\
1061	-6.10400000000004\\
1062	-9.76600000000008\\
1063	-10.9860000000001\\
1064	-6.10400000000004\\
1065	-6.10400000000004\\
1066	-9.76600000000008\\
1067	-9.76600000000008\\
1068	-15.8689999999999\\
1069	-14.6479999999999\\
1070	-17.0899999999999\\
1071	-15.8689999999999\\
1072	-18.3109999999999\\
1073	-13.4280000000001\\
1074	-14.6479999999999\\
1075	-12.2070000000001\\
1076	-10.9860000000001\\
1077	-12.2070000000001\\
1078	-23.193\\
1079	-28.076\\
1080	-30.518\\
1081	-29.297\\
1082	-18.3109999999999\\
1083	-28.076\\
1084	-32.9590000000001\\
1085	-31.7380000000001\\
1086	-20.752\\
1087	-34.1800000000001\\
1088	-40.2829999999999\\
1089	-29.297\\
1091	-19.5309999999999\\
1093	-12.2070000000001\\
1094	-14.6479999999999\\
1095	-9.76600000000008\\
1096	-7.32400000000007\\
1097	-7.32400000000007\\
1098	-10.9860000000001\\
1099	-13.4280000000001\\
1100	-12.2070000000001\\
1101	-12.2070000000001\\
1102	-15.8689999999999\\
1103	-18.3109999999999\\
1104	-19.5309999999999\\
1105	-15.8689999999999\\
1106	-21.973\\
1107	-15.8689999999999\\
1108	-15.8689999999999\\
1109	-17.0899999999999\\
1110	-12.2070000000001\\
1111	-12.2070000000001\\
1112	-15.8689999999999\\
1113	-14.6479999999999\\
1114	-8.54500000000007\\
1115	-12.2070000000001\\
1116	-8.54500000000007\\
1117	-9.76600000000008\\
1118	-9.76600000000008\\
1119	-7.32400000000007\\
1121	-12.2070000000001\\
1122	-17.0899999999999\\
1124	-17.0899999999999\\
1125	-10.9860000000001\\
1126	-12.2070000000001\\
1127	-23.193\\
1128	-15.8689999999999\\
1129	-20.752\\
1130	-21.973\\
1131	-12.2070000000001\\
1132	-9.76600000000008\\
1133	-12.2070000000001\\
1134	-13.4280000000001\\
1135	-21.973\\
1136	-25.635\\
1137	-26.855\\
1138	-19.5309999999999\\
1139	-20.752\\
1140	-18.3109999999999\\
1141	-17.0899999999999\\
1142	-18.3109999999999\\
1143	-13.4280000000001\\
1144	-12.2070000000001\\
1145	-9.76600000000008\\
1146	-9.76600000000008\\
1147	-10.9860000000001\\
1148	-15.8689999999999\\
1149	-23.193\\
1150	-17.0899999999999\\
1151	-13.4280000000001\\
1152	-10.9860000000001\\
1153	-10.9860000000001\\
1154	-7.32400000000007\\
1155	-6.10400000000004\\
1156	-6.10400000000004\\
1157	-12.2070000000001\\
1158	-7.32400000000007\\
1159	-8.54500000000007\\
1161	-8.54500000000007\\
1162	-7.32400000000007\\
1163	-4.88300000000004\\
1164	-3.66200000000003\\
1165	-8.54500000000007\\
1166	-15.8689999999999\\
1168	-20.752\\
1170	-10.9860000000001\\
1171	-8.54500000000007\\
1172	-4.88300000000004\\
1173	-9.76600000000008\\
1174	-8.54500000000007\\
1175	-15.8689999999999\\
1176	-17.0899999999999\\
1177	-29.297\\
1178	-32.9590000000001\\
1179	-25.635\\
1180	-25.635\\
1181	-18.3109999999999\\
1182	-23.193\\
1183	-21.973\\
1184	-24.414\\
1185	-15.8689999999999\\
1186	-17.0899999999999\\
1187	-17.0899999999999\\
1188	-15.8689999999999\\
1189	-15.8689999999999\\
1190	-13.4280000000001\\
1191	-23.193\\
1192	-25.635\\
1194	-13.4280000000001\\
1195	-14.6479999999999\\
1196	-23.193\\
1197	-24.414\\
1198	-28.076\\
1199	-30.518\\
1200	-29.297\\
1201	-23.193\\
1202	-21.973\\
1203	-25.635\\
1204	-28.076\\
1205	-18.3109999999999\\
1206	-12.2070000000001\\
1207	-19.5309999999999\\
1209	-9.76600000000008\\
1210	-13.4280000000001\\
1211	-14.6479999999999\\
1212	-10.9860000000001\\
1214	-10.9860000000001\\
1215	-8.54500000000007\\
1216	-10.9860000000001\\
1217	-18.3109999999999\\
1218	-17.0899999999999\\
1219	-12.2070000000001\\
1220	-12.2070000000001\\
1221	-18.3109999999999\\
1222	-26.855\\
1223	-17.0899999999999\\
1224	-14.6479999999999\\
1225	-10.9860000000001\\
1226	-13.4280000000001\\
1228	-8.54500000000007\\
1229	-8.54500000000007\\
1231	-13.4280000000001\\
1232	-14.6479999999999\\
1233	-12.2070000000001\\
1234	-15.8689999999999\\
1235	-20.752\\
1236	-17.0899999999999\\
1237	-12.2070000000001\\
1238	-15.8689999999999\\
1239	-20.752\\
1240	-13.4280000000001\\
1241	-7.32400000000007\\
1242	-13.4280000000001\\
1243	-10.9860000000001\\
1244	-12.2070000000001\\
1245	-9.76600000000008\\
1246	-8.54500000000007\\
1247	-13.4280000000001\\
1248	-13.4280000000001\\
1249	-9.76600000000008\\
1250	-12.2070000000001\\
1252	-7.32400000000007\\
1253	-12.2070000000001\\
1254	-8.54500000000007\\
1255	-9.76600000000008\\
1257	-4.88300000000004\\
1258	-12.2070000000001\\
1259	-15.8689999999999\\
1260	-17.0899999999999\\
1261	-23.193\\
1262	-15.8689999999999\\
1263	-9.76600000000008\\
1264	-8.54500000000007\\
1265	-8.54500000000007\\
1266	-10.9860000000001\\
1267	-8.54500000000007\\
1268	-4.88300000000004\\
1269	-10.9860000000001\\
1270	-15.8689999999999\\
1271	-13.4280000000001\\
1272	-20.752\\
1273	-15.8689999999999\\
1274	-14.6479999999999\\
1275	-8.54500000000007\\
1276	-10.9860000000001\\
1277	-4.88300000000004\\
1278	-3.66200000000003\\
1279	-4.88300000000004\\
1280	-9.76600000000008\\
1283	-13.4280000000001\\
1284	-20.752\\
1285	-17.0899999999999\\
1286	-14.6479999999999\\
1288	-19.5309999999999\\
1289	-20.752\\
1291	-13.4280000000001\\
1292	-7.32400000000007\\
1294	-4.88300000000004\\
1295	-7.32400000000007\\
1296	-7.32400000000007\\
1297	-4.88300000000004\\
1299	-12.2070000000001\\
1300	-10.9860000000001\\
1301	-12.2070000000001\\
1302	-12.2070000000001\\
1304	-4.88300000000004\\
1305	-9.76600000000008\\
1306	-15.8689999999999\\
1307	-13.4280000000001\\
1308	-15.8689999999999\\
1309	-13.4280000000001\\
1310	-9.76600000000008\\
1311	-14.6479999999999\\
1312	-18.3109999999999\\
1313	-18.3109999999999\\
1314	-13.4280000000001\\
1315	-20.752\\
1316	-15.8689999999999\\
1319	-19.5309999999999\\
1320	-13.4280000000001\\
1321	-13.4280000000001\\
1322	-18.3109999999999\\
1323	-26.855\\
1324	-21.973\\
1325	-13.4280000000001\\
1327	-13.4280000000001\\
1328	-15.8689999999999\\
1329	-17.0899999999999\\
1330	-12.2070000000001\\
1331	-14.6479999999999\\
1332	-18.3109999999999\\
1333	-20.752\\
1334	-13.4280000000001\\
1335	-12.2070000000001\\
1336	-15.8689999999999\\
1337	-23.193\\
1338	-20.752\\
1339	-21.973\\
1340	-14.6479999999999\\
1341	-14.6479999999999\\
1343	-9.76600000000008\\
1344	-4.88300000000004\\
1345	-3.66200000000003\\
1346	-3.66200000000003\\
1347	-7.32400000000007\\
1348	-13.4280000000001\\
1349	-14.6479999999999\\
1350	-9.76600000000008\\
1351	-12.2070000000001\\
1352	-13.4280000000001\\
1353	-9.76600000000008\\
1354	-8.54500000000007\\
1355	-8.54500000000007\\
1356	-6.10400000000004\\
1357	-9.76600000000008\\
1358	-14.6479999999999\\
1359	-15.8689999999999\\
1360	-10.9860000000001\\
1361	-8.54500000000007\\
1363	-8.54500000000007\\
1364	-7.32400000000007\\
1365	-10.9860000000001\\
1366	-20.752\\
1367	-18.3109999999999\\
1368	-18.3109999999999\\
1369	-24.414\\
1370	-23.193\\
1371	-18.3109999999999\\
1372	-14.6479999999999\\
1374	-14.6479999999999\\
1376	-19.5309999999999\\
1377	-19.5309999999999\\
1378	-25.635\\
1379	-26.855\\
1380	-31.7380000000001\\
1381	-24.414\\
1383	-26.855\\
1384	-21.973\\
1385	-24.414\\
1386	-20.752\\
1387	-13.4280000000001\\
1388	-9.76600000000008\\
1389	-9.76600000000008\\
1390	-12.2070000000001\\
1392	-7.32400000000007\\
1393	-10.9860000000001\\
1395	-6.10400000000004\\
1396	-7.32400000000007\\
1397	-9.76600000000008\\
1398	-6.10400000000004\\
1399	-12.2070000000001\\
1400	-14.6479999999999\\
1401	-10.9860000000001\\
1402	-17.0899999999999\\
1403	-24.414\\
1404	-24.414\\
1405	-28.076\\
};
\addlegendentry{True output}

\addplot [color=mycolor2, dashed, line width=2.0pt]
  table[row sep=crcr]{%
1006	-19.1871419588074\\
1007	-20.2170256872116\\
1008	-17.7656157699494\\
1009	-11.1116533934828\\
1010	-12.1603887820268\\
1011	-13.6587985830529\\
1012	-14.5359354288282\\
1013	-13.6960986312022\\
1014	-6.39674896463907\\
1015	-5.30978301921891\\
1016	-4.41165996277505\\
1017	-8.85448981018226\\
1018	-15.6474567597757\\
1019	-15.1890751492519\\
1020	-15.8478001617268\\
1021	-12.9597209762608\\
1022	-9.35724642547029\\
1023	-15.9884431788246\\
1024	-14.870828738721\\
1025	-10.1917655877414\\
1026	-14.6469178733025\\
1027	-15.6165205088018\\
1028	-11.3021244306881\\
1029	-16.3846198209856\\
1030	-17.6101041219652\\
1031	-13.6813226430804\\
1032	-18.3517851747495\\
1033	-19.4024094229137\\
1034	-15.2048159201363\\
1035	-13.9470283760816\\
1036	-12.5328293329912\\
1037	-12.4325061356897\\
1038	-15.1061312075722\\
1039	-14.7459706488548\\
1040	-14.0976343104767\\
1041	-14.7359286310905\\
1042	-15.5691664450028\\
1043	-19.9876024195914\\
1044	-19.7134984782649\\
1045	-14.1461320159465\\
1046	-13.3741841940898\\
1047	-9.62302373440184\\
1048	-8.37144719965067\\
1049	-11.8384458875828\\
1050	-11.2810886873463\\
1051	-11.0015801341203\\
1052	-13.6125403828021\\
1053	-15.2346400792774\\
1054	-13.7190317744546\\
1055	-18.7448521765066\\
1056	-16.9596752019934\\
1057	-9.44556636993957\\
1058	-8.9509614392914\\
1059	-10.4784603324104\\
1060	-12.0585063604306\\
1061	-7.78420006620127\\
1062	-6.29344667862165\\
1063	-10.0400108618151\\
1064	-10.2429935263383\\
1065	-7.44130979115243\\
1066	-8.84925775870829\\
1067	-10.4477172692546\\
1068	-14.0535394606468\\
1069	-15.0216371283334\\
1070	-15.6587699626061\\
1071	-15.3978865405622\\
1072	-17.0029573252998\\
1073	-15.6222121036674\\
1074	-13.6545796996631\\
1075	-13.3428095976433\\
1076	-12.7809267628199\\
1077	-12.3619807205084\\
1078	-18.0880836835793\\
1079	-26.5818709337407\\
1080	-27.0413182320337\\
1081	-26.8295800779952\\
1082	-18.4692563520625\\
1083	-24.7595630474277\\
1084	-29.76354088012\\
1085	-28.7113279020266\\
1086	-24.7308893021577\\
1087	-29.2461420345176\\
1088	-35.8510028950855\\
1089	-28.9559526520502\\
1090	-25.9993383491171\\
1091	-18.8939503937315\\
1092	-14.8654169979318\\
1093	-13.1651126185943\\
1094	-14.6033023615125\\
1095	-13.1969225000482\\
1096	-7.7927588503062\\
1097	-7.03966033352413\\
1098	-9.03149144674512\\
1099	-13.7695109520519\\
1100	-13.7961738648171\\
1101	-13.0381579860218\\
1102	-15.3949668476289\\
1103	-16.2522336780112\\
1104	-21.8551992447619\\
1105	-19.6664173609151\\
1106	-19.8557553010762\\
1108	-13.9970647514028\\
1109	-17.386515086625\\
1110	-14.4420470618086\\
1111	-11.8403852439535\\
1112	-14.3575036948728\\
1113	-15.028451189168\\
1114	-12.4013816350168\\
1115	-11.3910957192049\\
1116	-10.1630575764775\\
1117	-9.2215521732121\\
1118	-11.5295459358549\\
1119	-8.41136653499484\\
1120	-8.82649699046215\\
1121	-12.0844777053885\\
1122	-16.3319668092438\\
1123	-16.8342096878853\\
1124	-16.8724318940233\\
1125	-12.7200552613563\\
1126	-15.215621236664\\
1127	-21.9705716767301\\
1128	-17.3973029026974\\
1129	-18.3786606102021\\
1130	-19.5005026749957\\
1131	-13.7664163438101\\
1132	-10.7033903251117\\
1133	-12.5337637099551\\
1134	-14.4147370660435\\
1135	-20.2487332788594\\
1136	-24.8991645912172\\
1137	-23.5429165709904\\
1138	-20.2095309212309\\
1139	-19.8805989308853\\
1140	-19.7073348892309\\
1141	-17.8876973726747\\
1142	-17.5407497776844\\
1143	-13.4687241311372\\
1144	-11.5530069326619\\
1145	-11.8104150276563\\
1146	-10.9616300459645\\
1147	-11.3083367679678\\
1148	-15.1177287845992\\
1149	-21.2685539742379\\
1150	-19.0785743362892\\
1151	-13.1988854608785\\
1152	-12.0004519129875\\
1153	-11.8126283002434\\
1154	-8.27959414594807\\
1155	-6.1315761103258\\
1156	-6.87758871347705\\
1157	-11.1362333870645\\
1158	-10.5594421863736\\
1159	-8.79053941239727\\
1160	-10.3152636662169\\
1161	-9.48955762203673\\
1162	-7.7032063565016\\
1163	-5.77465401635754\\
1164	-5.08817003390504\\
1165	-6.66504221256537\\
1166	-13.4546930147901\\
1167	-18.3599300190735\\
1168	-19.7842062358252\\
1169	-14.1172749052453\\
1170	-11.2390782346008\\
1171	-9.27843382852188\\
1172	-7.15262494937724\\
1173	-10.4794667357355\\
1174	-12.5480349036673\\
1175	-13.1850211312033\\
1176	-17.484712432898\\
1177	-26.124626939275\\
1178	-31.0884369442597\\
1179	-26.0717558892593\\
1180	-26.1453996404691\\
1181	-20.1211532832765\\
1182	-18.9279321806027\\
1183	-21.2045704154641\\
1184	-22.0735756444856\\
1185	-19.0817990755227\\
1186	-17.087106974577\\
1187	-17.3187782124573\\
1188	-15.4272757227343\\
1189	-17.9249784418987\\
1190	-14.7570864412553\\
1191	-19.2103218440543\\
1192	-24.2240511973289\\
1193	-18.4979187099789\\
1194	-13.3786200443299\\
1195	-14.0793719395567\\
1196	-20.5886970632719\\
1197	-24.4705229838714\\
1198	-27.6300212257611\\
1199	-29.3232568211226\\
1200	-28.8480493498178\\
1201	-23.4579925317764\\
1202	-22.0340749036579\\
1203	-23.8797520480987\\
1204	-26.0523998030485\\
1205	-18.9915092707301\\
1206	-13.0648079657076\\
1207	-16.4500329413406\\
1208	-16.0464596755999\\
1209	-10.6078374610827\\
1211	-15.5337465762914\\
1212	-12.7076680015541\\
1213	-10.0335477650378\\
1214	-12.2634749347155\\
1215	-12.035667290522\\
1216	-13.3079465590658\\
1217	-17.6383091743601\\
1218	-15.9973200128068\\
1219	-12.3811012690517\\
1220	-12.6357811026005\\
1221	-17.7705454104878\\
1222	-26.1707957861584\\
1223	-21.8427753067137\\
1224	-13.6246150203469\\
1225	-12.0144730097929\\
1226	-11.8426292923646\\
1227	-13.1381025237715\\
1228	-10.0088557214663\\
1229	-10.4202584838783\\
1230	-11.5862550145189\\
1231	-14.1144202498979\\
1232	-15.2987815254107\\
1233	-10.8569718416352\\
1235	-19.1853539934787\\
1236	-18.2103876308449\\
1237	-15.4118042200139\\
1238	-15.6990763807078\\
1239	-19.874900626231\\
1240	-14.5558949934493\\
1241	-7.59548378620843\\
1242	-9.08738676541611\\
1243	-12.0947702534559\\
1244	-14.7935114451923\\
1245	-13.6826700228062\\
1246	-9.78439756576427\\
1247	-12.8016445339695\\
1248	-13.7589804641566\\
1249	-10.4052113671696\\
1250	-12.1766695952999\\
1251	-12.2424330556153\\
1252	-10.7242345588486\\
1253	-11.0944620518821\\
1254	-8.42501181828447\\
1255	-8.40153828032953\\
1256	-8.46679364826582\\
1257	-8.27840541825162\\
1258	-12.5379456065812\\
1259	-14.4627150826159\\
1260	-17.1844157382684\\
1261	-22.5671761971155\\
1262	-16.7993553844356\\
1263	-10.7586342256482\\
1264	-9.18223025572979\\
1265	-8.8289229035488\\
1266	-11.4766253543758\\
1267	-8.93516855437383\\
1268	-8.32820076509142\\
1269	-10.4751718104978\\
1270	-15.6297852068371\\
1271	-15.5215297076782\\
1272	-18.0322133362645\\
1273	-14.7465759674878\\
1274	-12.8625909343798\\
1275	-9.55056282464807\\
1277	-7.36475498932077\\
1278	-6.32731382428051\\
1279	-7.59396684685953\\
1280	-11.4361763576183\\
1281	-11.2921909872039\\
1282	-12.0619246569565\\
1283	-13.1628509185407\\
1284	-19.500077413825\\
1285	-19.6181069249717\\
1286	-13.5035553322548\\
1287	-16.1033191169345\\
1288	-17.0349532747402\\
1289	-21.1547348492518\\
1290	-18.5153189991593\\
1291	-13.0973065041492\\
1292	-8.6078188111569\\
1293	-6.68845975705108\\
1294	-6.62681660263161\\
1295	-7.53186000127539\\
1296	-7.64505445099985\\
1297	-7.20296019003581\\
1298	-8.34502401179634\\
1299	-12.1819961613357\\
1300	-12.039550679686\\
1301	-11.0341671020867\\
1302	-13.0925794322427\\
1304	-5.96226263850622\\
1305	-9.11415114869715\\
1306	-14.4966249730574\\
1307	-13.905075615899\\
1308	-13.8750598705806\\
1309	-13.7018129109208\\
1310	-11.0236543737478\\
1311	-13.0693288982905\\
1312	-17.6485264264913\\
1313	-17.7704232821425\\
1314	-13.3312306358166\\
1315	-20.0057140699221\\
1316	-18.2782567011914\\
1317	-15.2030288631515\\
1318	-18.2158233918117\\
1319	-17.4591479568057\\
1320	-14.8753764019204\\
1321	-13.0525843823887\\
1322	-17.9831664463597\\
1323	-25.2132269197211\\
1324	-20.1905459947166\\
1325	-13.2114817443785\\
1326	-13.2207951904402\\
1328	-16.0993801265101\\
1329	-17.9857559411334\\
1330	-14.2163613117\\
1331	-14.5800593157185\\
1332	-18.017283456858\\
1333	-19.1268841769077\\
1334	-14.6356612887394\\
1335	-12.4468035524621\\
1336	-15.1043632565897\\
1337	-22.4230630610266\\
1338	-20.5653465266682\\
1339	-20.0671788949337\\
1340	-14.2735359348492\\
1341	-12.8289238724628\\
1342	-13.9153665006304\\
1343	-9.26837705568528\\
1344	-7.2498113266995\\
1345	-5.76481775215575\\
1346	-4.90080319641697\\
1347	-8.05691282372641\\
1348	-12.5561379453843\\
1349	-13.7461697350102\\
1350	-10.9227717801427\\
1351	-11.4171030795533\\
1352	-13.1438184007889\\
1353	-9.86018409690314\\
1354	-9.31945660436895\\
1355	-10.1058983030359\\
1356	-8.26765629044075\\
1357	-8.34436021204556\\
1358	-12.6212005823656\\
1359	-16.1094838406445\\
1360	-11.2873165511239\\
1361	-9.12509792703668\\
1362	-8.62938482960135\\
1363	-9.73509092646964\\
1364	-7.66012346533489\\
1365	-12.8564694808319\\
1366	-19.8979665766578\\
1367	-15.9672230675935\\
1368	-20.0749125696009\\
1369	-24.2606181184306\\
1370	-22.9559854202935\\
1371	-18.5967460484794\\
1372	-14.672648537673\\
1373	-14.55015625434\\
1374	-15.1040316645303\\
1375	-16.6294203030441\\
1376	-18.4831573222189\\
1377	-18.967462976553\\
1378	-22.2579849355054\\
1379	-25.9176692936812\\
1380	-29.7103674737239\\
1381	-23.2986423491152\\
1382	-23.2997483812678\\
1383	-27.2513112446782\\
1384	-22.6171498977515\\
1385	-23.2122927840994\\
1386	-21.6628008564071\\
1387	-12.5119737069303\\
1388	-9.99909399694775\\
1389	-10.004427696377\\
1390	-14.1165329737285\\
1391	-12.6300392434557\\
1392	-8.26778990140997\\
1393	-10.2320143441889\\
1394	-9.38470482461025\\
1395	-6.8958969503758\\
1396	-8.38317148766873\\
1397	-9.69936351612523\\
1398	-7.73411441576695\\
1399	-10.9974344805216\\
1400	-15.8072748688385\\
1401	-12.0994416220224\\
1402	-16.4636782683135\\
1403	-24.0906885235286\\
1404	-22.3005738832203\\
1405	-26.3466451626552\\
};
\addlegendentry{OSA predition}

\addplot [color=mycolor3, dotted, line width=2.0pt]
  table[row sep=crcr]{%
1006	-18.3109999999999\\
1007	-23.193\\
1008	-18.3109999999999\\
1009	-9.76600000000008\\
1010	-12.160388782027\\
1011	-13.093176870166\\
1012	-14.1665198579817\\
1013	-13.1280842127801\\
1014	-6.64630453244968\\
1015	-6.08570920220427\\
1016	-5.68281081067425\\
1017	-10.0893831355354\\
1018	-17.0465737442628\\
1019	-16.2497043649435\\
1020	-16.6584322576093\\
1021	-13.3188214169186\\
1022	-9.24834446102227\\
1023	-15.7732496511994\\
1024	-14.4026430971803\\
1025	-9.74185140648865\\
1026	-13.9408541967307\\
1027	-15.3902376637623\\
1028	-11.6550140008144\\
1029	-16.9634412531404\\
1030	-17.7525589901629\\
1031	-13.1573404699327\\
1032	-17.5880721491633\\
1033	-18.6404445442631\\
1034	-14.5821305493932\\
1035	-12.9657402007681\\
1036	-11.8094913591333\\
1037	-12.3955836955915\\
1038	-14.9083441391861\\
1039	-14.8075203698631\\
1040	-13.861250495863\\
1041	-14.5979567993452\\
1042	-15.4914327909407\\
1043	-19.7880539623723\\
1044	-19.3411677723147\\
1045	-13.5939434572563\\
1046	-12.8139897432409\\
1047	-9.25717899149981\\
1048	-8.37575797555451\\
1049	-11.2162452161915\\
1050	-10.70710759244\\
1051	-9.89268957917966\\
1052	-12.7724408675772\\
1053	-14.5129368157268\\
1054	-13.3395052837293\\
1055	-18.5591449475467\\
1056	-16.6066726648621\\
1057	-9.60842131187565\\
1058	-9.03056439293005\\
1059	-11.2127300301336\\
1060	-12.8114526943939\\
1061	-8.65523007310617\\
1062	-7.39752008692858\\
1063	-10.3928597399417\\
1064	-10.2918445670982\\
1065	-7.82563291919109\\
1066	-9.49364798609463\\
1067	-11.2452451667443\\
1068	-14.682722309517\\
1069	-15.055506379191\\
1070	-15.7907033423037\\
1071	-15.1242415860952\\
1072	-16.7427108000948\\
1073	-15.0947511852869\\
1074	-13.5211615548949\\
1075	-13.1085969651226\\
1076	-12.9889351367099\\
1077	-12.6760786500747\\
1078	-18.5542824937827\\
1079	-26.4415102300288\\
1080	-26.4128390490632\\
1081	-25.3818095281079\\
1082	-17.0144004625265\\
1083	-23.0653318631039\\
1084	-28.0246553335887\\
1085	-27.1854344602677\\
1086	-22.5856230933844\\
1087	-27.5416132643102\\
1088	-34.3918908431676\\
1089	-27.9624694852025\\
1090	-23.7034072004465\\
1091	-17.2325176711299\\
1092	-13.9676559652564\\
1093	-12.348676977269\\
1094	-13.8772384938804\\
1095	-12.6777564308327\\
1096	-8.0988078703283\\
1097	-7.46374217551033\\
1098	-9.63682947928896\\
1099	-13.8146966067109\\
1100	-13.7318525915041\\
1101	-13.11642506154\\
1102	-15.7576234550281\\
1103	-16.6357312725579\\
1104	-21.8229018373079\\
1105	-19.7488059263785\\
1106	-20.4122579120572\\
1108	-14.7920482255897\\
1109	-17.3243472062545\\
1110	-14.5679394522465\\
1111	-12.1232983246134\\
1112	-14.6800285201093\\
1113	-15.2660684160564\\
1114	-12.4069752336841\\
1115	-11.8970146883159\\
1116	-10.7027068711766\\
1117	-10.1737123075227\\
1118	-11.9797744189352\\
1119	-9.12645198350401\\
1120	-9.56025530389593\\
1121	-12.6783541316017\\
1122	-16.7596452319458\\
1123	-16.8826346569126\\
1124	-16.8689669134817\\
1125	-12.6248009877509\\
1126	-15.394759470322\\
1127	-22.6356333905392\\
1128	-18.0345731501623\\
1129	-19.2409010615795\\
1130	-19.6252134086155\\
1131	-13.5905708943137\\
1132	-10.425931250549\\
1133	-12.3541946357161\\
1134	-14.6909495700538\\
1135	-20.6315543150779\\
1136	-24.9493681334247\\
1137	-23.5153726097235\\
1138	-19.5488696558266\\
1139	-19.2119798444535\\
1140	-18.832918411227\\
1141	-17.6135833822905\\
1142	-17.4025916352668\\
1143	-13.4485158458704\\
1144	-11.5341165607745\\
1145	-11.5689083101954\\
1146	-11.1267456685855\\
1147	-11.6614807204894\\
1148	-15.704307464867\\
1149	-21.6440143969626\\
1150	-19.03264632174\\
1151	-13.2563016979889\\
1152	-11.9862770372335\\
1153	-12.2290594199562\\
1154	-8.66319261354693\\
1155	-6.65327138384237\\
1156	-7.37884496224819\\
1157	-11.6960261415677\\
1158	-10.7692536091945\\
1159	-9.49499335508131\\
1160	-10.8854893387756\\
1161	-10.4921105867827\\
1162	-8.58731501189754\\
1163	-6.5791576141271\\
1164	-5.87586808935134\\
1165	-7.52066583667079\\
1166	-13.8718447269607\\
1167	-18.2983517536195\\
1168	-19.479359066416\\
1169	-13.5958220933783\\
1170	-10.6453938258455\\
1172	-6.7861317342456\\
1173	-10.6760956547178\\
1174	-12.9537728139553\\
1175	-14.2919777854831\\
1176	-18.0338097200029\\
1177	-26.8128491782552\\
1178	-30.8765543507177\\
1179	-25.6921683069502\\
1180	-25.3963858169429\\
1181	-19.6121369926072\\
1182	-19.002660516951\\
1183	-20.8158404845396\\
1184	-21.562170158714\\
1185	-17.8408445240802\\
1186	-16.5810327660276\\
1187	-16.9447566402021\\
1188	-15.5879423075421\\
1189	-17.8398406347296\\
1190	-14.9558711611021\\
1192	-24.3747976642362\\
1193	-18.2103520188907\\
1194	-12.4877654293391\\
1195	-13.3743023496811\\
1196	-19.9348159468302\\
1197	-23.6541784867138\\
1198	-26.8221942067421\\
1199	-28.407971790622\\
1200	-28.2046258108107\\
1201	-22.7651454635381\\
1202	-21.3445265189512\\
1203	-23.4224219557138\\
1204	-25.5762048442903\\
1205	-18.2742672806894\\
1206	-12.2893979482392\\
1207	-15.8620985011542\\
1208	-15.4398610022474\\
1209	-10.2205549405173\\
1210	-12.5963328911994\\
1211	-15.4778200186315\\
1212	-12.8235656132676\\
1213	-10.3222706710594\\
1214	-12.5258264988813\\
1215	-12.527765962898\\
1216	-14.111621507272\\
1217	-18.9135620069012\\
1218	-17.2318710272091\\
1219	-13.1439416869869\\
1220	-13.0228452744561\\
1221	-18.0794594110973\\
1222	-26.4331664749645\\
1223	-21.9472743960659\\
1224	-14.210007816635\\
1225	-12.5816709971864\\
1226	-12.9019687940156\\
1227	-13.3868383353044\\
1228	-10.5511807753758\\
1229	-11.0476455712908\\
1230	-12.6570853447747\\
1231	-15.1463491303009\\
1232	-16.2790910923788\\
1233	-11.6857566301364\\
1235	-19.4294814865932\\
1236	-17.9616114690639\\
1237	-15.2760646275708\\
1238	-16.0419323242577\\
1239	-20.4517903889168\\
1240	-15.0964183408723\\
1241	-7.94660920546721\\
1242	-9.38519136703303\\
1243	-11.8248911241926\\
1244	-14.5628602128304\\
1245	-13.5402687692635\\
1246	-10.6678757182681\\
1247	-14.0969798650808\\
1248	-14.8733464180361\\
1249	-11.1839129129214\\
1250	-12.7921223796045\\
1251	-12.7898841975082\\
1252	-11.5637779281517\\
1253	-12.3401871430879\\
1254	-9.48349053331867\\
1255	-9.3432147826345\\
1256	-8.70230092589736\\
1257	-8.59722052419534\\
1258	-13.3017028662903\\
1259	-15.3029469759551\\
1260	-17.8653071854208\\
1261	-22.9271119096265\\
1262	-16.8185038358122\\
1263	-10.9580890765008\\
1264	-9.49907332490807\\
1265	-9.29611482996393\\
1266	-11.9707629794643\\
1267	-9.37348685557117\\
1268	-8.73842770233387\\
1269	-11.4180917761983\\
1270	-16.3679653788063\\
1271	-16.2829993679691\\
1272	-18.7411636923273\\
1273	-14.9622640453977\\
1274	-12.9391135123774\\
1275	-8.97771441809709\\
1276	-8.08063871838499\\
1277	-6.69041451398243\\
1278	-6.24961264037142\\
1279	-7.84514176886046\\
1280	-12.4183275606185\\
1281	-12.5056488636374\\
1282	-13.1498336575844\\
1283	-14.0314188430302\\
1284	-20.0709602413929\\
1285	-19.823690421335\\
1286	-13.923789482918\\
1287	-16.275033668526\\
1288	-17.2898781141455\\
1289	-20.7015454680309\\
1290	-18.0274162638609\\
1291	-12.8248207397924\\
1292	-8.55661724243168\\
1293	-6.9287645556999\\
1294	-6.87917114515426\\
1295	-8.14215705405877\\
1296	-8.19436216546478\\
1297	-7.75519053578046\\
1298	-9.12901565691368\\
1299	-12.7970691656035\\
1300	-12.6542766269574\\
1301	-11.5658739618902\\
1302	-13.3238955755255\\
1303	-9.90861840186631\\
1304	-6.31388994000713\\
1305	-9.70676983213502\\
1306	-14.9436435436598\\
1307	-14.0197128696595\\
1308	-13.8932824013218\\
1309	-13.3336619763804\\
1310	-10.8012019427101\\
1311	-12.9113834927909\\
1312	-17.430013231176\\
1313	-17.5853967636863\\
1314	-12.8754155410002\\
1315	-19.5905163725822\\
1316	-17.8871235806641\\
1317	-15.2149862641518\\
1318	-17.9888289439207\\
1319	-17.4701645696261\\
1320	-14.3026395889408\\
1321	-12.7713786787911\\
1322	-17.5967442201556\\
1323	-25.112742113158\\
1324	-19.8267024466061\\
1325	-12.5494097013811\\
1326	-12.4415026742286\\
1327	-13.9198437604416\\
1328	-15.7976195506744\\
1329	-17.8342243533757\\
1330	-14.3575486456584\\
1331	-15.0215889603153\\
1332	-18.5138233454504\\
1333	-19.6260054243703\\
1334	-14.6595180714439\\
1335	-12.5238013009161\\
1336	-15.1264595013249\\
1337	-22.5576967181862\\
1338	-20.4894929237441\\
1339	-19.8243656803911\\
1340	-13.8227006621612\\
1341	-12.3243659190066\\
1342	-13.038538894152\\
1343	-8.86905169919964\\
1344	-6.85923297509999\\
1345	-5.99709791699547\\
1346	-5.47173290508317\\
1347	-8.9559253102168\\
1348	-13.4695611921159\\
1349	-14.2803767910452\\
1350	-11.1765737798212\\
1351	-11.6645329716764\\
1352	-13.1868331008986\\
1353	-9.94661362126772\\
1354	-9.28815852845037\\
1355	-10.1933923203562\\
1356	-8.65016394823874\\
1357	-9.09894606854664\\
1358	-13.1370423874935\\
1359	-16.2536394471497\\
1360	-11.1494169070247\\
1361	-8.95673100895056\\
1362	-8.72008120680061\\
1363	-9.85894122507034\\
1364	-7.9608291005718\\
1365	-13.173347533741\\
1366	-20.5454474464157\\
1367	-16.3591651478821\\
1368	-20.1560611360508\\
1369	-24.2830608101913\\
1370	-22.8341613052078\\
1372	-14.7339943945019\\
1373	-14.5631895418674\\
1374	-15.1794995414834\\
1375	-16.7312550188881\\
1376	-18.4931898270647\\
1377	-18.8689259868067\\
1378	-21.9749386116566\\
1379	-25.1833268583498\\
1380	-28.841804656083\\
1381	-22.0944648517773\\
1382	-22.1262208384726\\
1383	-25.8036187988507\\
1384	-21.4183196396896\\
1385	-22.1951171107748\\
1386	-20.9728195828741\\
1387	-12.078481652271\\
1388	-9.45447544254171\\
1389	-9.7095166476538\\
1390	-13.813949802216\\
1391	-12.7696970859547\\
1392	-8.9408113801594\\
1393	-11.1807102620453\\
1394	-10.1639843071769\\
1395	-7.53074585575996\\
1396	-8.92849375706919\\
1397	-10.4123862523791\\
1398	-8.30602530279498\\
1399	-11.7578389211528\\
1400	-16.1852841908103\\
1401	-12.6250735821445\\
1402	-16.9318716696046\\
1403	-24.5517461029542\\
1404	-22.6334384626007\\
1405	-26.1890427368044\\
};
\addlegendentry{MPO prediction}

\end{axis}

\begin{axis}[%
width=6.159cm,
height=1.831cm,
at={(8.104cm,7.627cm)},
scale only axis,
xmin=1000,
xmax=1405,
xlabel style={font=\color{white!15!black}},
xlabel={Sample index},
ymin=-26.855,
ymax=0,
ylabel style={font=\color{white!15!black}},
ylabel={$y(t)$},
axis background/.style={fill=white},
title style={font=\bfseries},
title={C4: RMSE(OSA) = 1.501, RMSE(MPO) = 1.4703},
legend style={legend cell align=left, align=left, draw=white!15!black}
]
\addplot [color=mycolor1, line width=2.0pt]
  table[row sep=crcr]{%
1006	-14.6479999999999\\
1007	-17.0899999999999\\
1008	-10.9860000000001\\
1009	-7.32400000000007\\
1010	-14.6479999999999\\
1011	-8.54500000000007\\
1012	-9.76600000000008\\
1013	-6.10400000000004\\
1014	-3.66200000000003\\
1016	-3.66200000000003\\
1017	-2.44100000000003\\
1018	-13.4280000000001\\
1020	-13.4280000000001\\
1021	-9.76600000000008\\
1022	-7.32400000000007\\
1023	-10.9860000000001\\
1024	-12.2070000000001\\
1025	-6.10400000000004\\
1027	-13.4280000000001\\
1028	-7.32400000000007\\
1029	-15.8689999999999\\
1031	-10.9860000000001\\
1032	-17.0899999999999\\
1033	-13.4280000000001\\
1035	-8.54500000000007\\
1036	-8.54500000000007\\
1037	-10.9860000000001\\
1038	-12.2070000000001\\
1040	-9.76600000000008\\
1041	-10.9860000000001\\
1042	-10.9860000000001\\
1043	-15.8689999999999\\
1044	-14.6479999999999\\
1045	-10.9860000000001\\
1046	-10.9860000000001\\
1047	-6.10400000000004\\
1048	-4.88300000000004\\
1049	-8.54500000000007\\
1050	-7.32400000000007\\
1051	-7.32400000000007\\
1052	-10.9860000000001\\
1053	-9.76600000000008\\
1054	-9.76600000000008\\
1055	-15.8689999999999\\
1056	-8.54500000000007\\
1057	-6.10400000000004\\
1060	-9.76600000000008\\
1061	-4.88300000000004\\
1062	-7.32400000000007\\
1064	-7.32400000000007\\
1065	-6.10400000000004\\
1069	-10.9860000000001\\
1070	-10.9860000000001\\
1072	-13.4280000000001\\
1073	-10.9860000000001\\
1074	-12.2070000000001\\
1075	-9.76600000000008\\
1077	-9.76600000000008\\
1078	-17.0899999999999\\
1079	-20.752\\
1080	-19.5309999999999\\
1081	-19.5309999999999\\
1082	-15.8689999999999\\
1083	-20.752\\
1084	-23.193\\
1085	-23.193\\
1086	-14.6479999999999\\
1087	-23.193\\
1088	-26.855\\
1089	-21.973\\
1090	-15.8689999999999\\
1091	-14.6479999999999\\
1092	-10.9860000000001\\
1093	-8.54500000000007\\
1094	-10.9860000000001\\
1096	-6.10400000000004\\
1097	-6.10400000000004\\
1098	-8.54500000000007\\
1100	-10.9860000000001\\
1101	-9.76600000000008\\
1102	-12.2070000000001\\
1103	-10.9860000000001\\
1104	-15.8689999999999\\
1105	-13.4280000000001\\
1106	-18.3109999999999\\
1107	-12.2070000000001\\
1108	-10.9860000000001\\
1109	-12.2070000000001\\
1110	-9.76600000000008\\
1111	-8.54500000000007\\
1112	-10.9860000000001\\
1113	-10.9860000000001\\
1114	-8.54500000000007\\
1115	-8.54500000000007\\
1116	-7.32400000000007\\
1118	-7.32400000000007\\
1119	-4.88300000000004\\
1120	-6.10400000000004\\
1121	-8.54500000000007\\
1122	-13.4280000000001\\
1123	-12.2070000000001\\
1124	-13.4280000000001\\
1125	-8.54500000000007\\
1126	-15.8689999999999\\
1127	-17.0899999999999\\
1128	-13.4280000000001\\
1129	-14.6479999999999\\
1130	-14.6479999999999\\
1132	-7.32400000000007\\
1133	-8.54500000000007\\
1134	-8.54500000000007\\
1135	-17.0899999999999\\
1136	-20.752\\
1137	-19.5309999999999\\
1138	-14.6479999999999\\
1139	-15.8689999999999\\
1140	-13.4280000000001\\
1141	-12.2070000000001\\
1142	-12.2070000000001\\
1143	-9.76600000000008\\
1144	-8.54500000000007\\
1145	-8.54500000000007\\
1146	-9.76600000000008\\
1147	-8.54500000000007\\
1148	-13.4280000000001\\
1149	-15.8689999999999\\
1150	-10.9860000000001\\
1151	-8.54500000000007\\
1152	-9.76600000000008\\
1153	-8.54500000000007\\
1155	-3.66200000000003\\
1156	-4.88300000000004\\
1157	-8.54500000000007\\
1158	-9.76600000000008\\
1159	-4.88300000000004\\
1160	-7.32400000000007\\
1161	-7.32400000000007\\
1162	-3.66200000000003\\
1163	-3.66200000000003\\
1164	-2.44100000000003\\
1165	-4.88300000000004\\
1166	-10.9860000000001\\
1167	-12.2070000000001\\
1168	-14.6479999999999\\
1169	-13.4280000000001\\
1170	-8.54500000000007\\
1171	-7.32400000000007\\
1172	-4.88300000000004\\
1173	-7.32400000000007\\
1174	-6.10400000000004\\
1175	-10.9860000000001\\
1176	-12.2070000000001\\
1177	-21.973\\
1178	-23.193\\
1179	-21.973\\
1180	-18.3109999999999\\
1181	-13.4280000000001\\
1182	-17.0899999999999\\
1183	-15.8689999999999\\
1184	-19.5309999999999\\
1185	-12.2070000000001\\
1186	-13.4280000000001\\
1187	-12.2070000000001\\
1188	-13.4280000000001\\
1189	-13.4280000000001\\
1190	-10.9860000000001\\
1191	-18.3109999999999\\
1192	-17.0899999999999\\
1193	-13.4280000000001\\
1194	-7.32400000000007\\
1195	-12.2070000000001\\
1196	-15.8689999999999\\
1197	-17.0899999999999\\
1199	-21.973\\
1200	-20.752\\
1201	-15.8689999999999\\
1202	-14.6479999999999\\
1203	-18.3109999999999\\
1204	-19.5309999999999\\
1205	-13.4280000000001\\
1206	-9.76600000000008\\
1207	-13.4280000000001\\
1208	-9.76600000000008\\
1209	-7.32400000000007\\
1210	-9.76600000000008\\
1211	-10.9860000000001\\
1212	-7.32400000000007\\
1213	-7.32400000000007\\
1214	-9.76600000000008\\
1215	-6.10400000000004\\
1216	-10.9860000000001\\
1217	-13.4280000000001\\
1218	-13.4280000000001\\
1219	-8.54500000000007\\
1220	-9.76600000000008\\
1221	-12.2070000000001\\
1222	-18.3109999999999\\
1223	-15.8689999999999\\
1224	-9.76600000000008\\
1225	-8.54500000000007\\
1226	-9.76600000000008\\
1227	-7.32400000000007\\
1228	-8.54500000000007\\
1229	-7.32400000000007\\
1230	-8.54500000000007\\
1231	-10.9860000000001\\
1232	-10.9860000000001\\
1233	-6.10400000000004\\
1234	-10.9860000000001\\
1235	-13.4280000000001\\
1236	-13.4280000000001\\
1237	-9.76600000000008\\
1238	-12.2070000000001\\
1239	-13.4280000000001\\
1240	-10.9860000000001\\
1241	-4.88300000000004\\
1242	-10.9860000000001\\
1243	-8.54500000000007\\
1244	-8.54500000000007\\
1245	-6.10400000000004\\
1246	-6.10400000000004\\
1247	-10.9860000000001\\
1248	-10.9860000000001\\
1249	-6.10400000000004\\
1250	-8.54500000000007\\
1251	-8.54500000000007\\
1252	-4.88300000000004\\
1253	-8.54500000000007\\
1254	-4.88300000000004\\
1255	-7.32400000000007\\
1256	-4.88300000000004\\
1257	-4.88300000000004\\
1258	-10.9860000000001\\
1259	-12.2070000000001\\
1260	-12.2070000000001\\
1261	-15.8689999999999\\
1263	-6.10400000000004\\
1264	-7.32400000000007\\
1265	-4.88300000000004\\
1266	-7.32400000000007\\
1267	-6.10400000000004\\
1268	-3.66200000000003\\
1269	-8.54500000000007\\
1270	-9.76600000000008\\
1271	-9.76600000000008\\
1272	-17.0899999999999\\
1273	-10.9860000000001\\
1274	-13.4280000000001\\
1275	-7.32400000000007\\
1276	-8.54500000000007\\
1277	-3.66200000000003\\
1278	-1.221\\
1279	-6.10400000000004\\
1280	-9.76600000000008\\
1281	-8.54500000000007\\
1282	-9.76600000000008\\
1283	-9.76600000000008\\
1284	-17.0899999999999\\
1285	-10.9860000000001\\
1287	-10.9860000000001\\
1289	-15.8689999999999\\
1290	-15.8689999999999\\
1292	-6.10400000000004\\
1293	-3.66200000000003\\
1295	-6.10400000000004\\
1296	-4.88300000000004\\
1297	-6.10400000000004\\
1298	-6.10400000000004\\
1299	-9.76600000000008\\
1300	-9.76600000000008\\
1301	-8.54500000000007\\
1302	-9.76600000000008\\
1303	-6.10400000000004\\
1304	-3.66200000000003\\
1306	-10.9860000000001\\
1307	-8.54500000000007\\
1308	-12.2070000000001\\
1309	-8.54500000000007\\
1310	-7.32400000000007\\
1311	-8.54500000000007\\
1312	-12.2070000000001\\
1313	-14.6479999999999\\
1314	-10.9860000000001\\
1315	-17.0899999999999\\
1316	-9.76600000000008\\
1317	-12.2070000000001\\
1318	-13.4280000000001\\
1319	-13.4280000000001\\
1320	-9.76600000000008\\
1321	-9.76600000000008\\
1322	-12.2070000000001\\
1323	-18.3109999999999\\
1324	-15.8689999999999\\
1325	-8.54500000000007\\
1326	-12.2070000000001\\
1327	-9.76600000000008\\
1328	-12.2070000000001\\
1329	-13.4280000000001\\
1330	-9.76600000000008\\
1331	-9.76600000000008\\
1332	-14.6479999999999\\
1334	-12.2070000000001\\
1335	-8.54500000000007\\
1336	-9.76600000000008\\
1337	-15.8689999999999\\
1338	-14.6479999999999\\
1339	-15.8689999999999\\
1340	-12.2070000000001\\
1343	-8.54500000000007\\
1344	-4.88300000000004\\
1345	-3.66200000000003\\
1346	-3.66200000000003\\
1348	-10.9860000000001\\
1350	-8.54500000000007\\
1351	-10.9860000000001\\
1353	-8.54500000000007\\
1354	-3.66200000000003\\
1355	-6.10400000000004\\
1357	-6.10400000000004\\
1359	-10.9860000000001\\
1360	-7.32400000000007\\
1361	-6.10400000000004\\
1362	-7.32400000000007\\
1363	-7.32400000000007\\
1364	-6.10400000000004\\
1365	-8.54500000000007\\
1366	-15.8689999999999\\
1367	-14.6479999999999\\
1368	-17.0899999999999\\
1370	-17.0899999999999\\
1371	-13.4280000000001\\
1372	-10.9860000000001\\
1373	-9.76600000000008\\
1375	-12.2070000000001\\
1376	-14.6479999999999\\
1377	-14.6479999999999\\
1380	-21.973\\
1381	-18.3109999999999\\
1382	-21.973\\
1383	-19.5309999999999\\
1384	-14.6479999999999\\
1385	-17.0899999999999\\
1386	-14.6479999999999\\
1387	-10.9860000000001\\
1388	-8.54500000000007\\
1389	-8.54500000000007\\
1390	-10.9860000000001\\
1391	-7.32400000000007\\
1394	-7.32400000000007\\
1395	-3.66200000000003\\
1396	-7.32400000000007\\
1397	-7.32400000000007\\
1398	-6.10400000000004\\
1399	-8.54500000000007\\
1400	-12.2070000000001\\
1401	-8.54500000000007\\
1402	-12.2070000000001\\
1403	-18.3109999999999\\
1404	-17.0899999999999\\
1405	-19.5309999999999\\
};
\addlegendentry{True output}

\addplot [color=mycolor2, dashed, line width=2.0pt]
  table[row sep=crcr]{%
1006	-14.2292537067281\\
1007	-14.7888697203041\\
1008	-13.4363392139035\\
1009	-7.9286579076861\\
1010	-9.15567496596918\\
1011	-9.52642590232381\\
1012	-11.206107571217\\
1013	-10.3550266929815\\
1014	-4.47936290165558\\
1015	-3.87672029792566\\
1016	-2.90150343416394\\
1017	-6.78115289550624\\
1018	-12.8346419557627\\
1019	-12.4665361521752\\
1020	-12.3651208709316\\
1021	-10.3541037452317\\
1022	-7.1139828221983\\
1023	-11.5531085748041\\
1024	-10.4232026288066\\
1025	-6.53528646201812\\
1026	-11.1392116820984\\
1027	-11.9720757392533\\
1028	-8.82571217021791\\
1029	-12.6928475190805\\
1030	-13.1740918589678\\
1031	-9.65521564787809\\
1032	-13.4126879129419\\
1033	-13.817098679564\\
1034	-11.4407088486048\\
1035	-10.1449486247575\\
1036	-8.52513322273808\\
1037	-8.92609227446951\\
1038	-10.9633879419782\\
1039	-10.7786080110643\\
1040	-11.0195090968637\\
1041	-11.2288007390443\\
1042	-11.2036630543057\\
1043	-14.512470582903\\
1044	-14.0124116337317\\
1045	-9.89513533495619\\
1046	-9.66019581288856\\
1047	-6.87799952607065\\
1048	-6.24927090576352\\
1049	-8.43643598075937\\
1050	-7.34704204722448\\
1051	-6.42047509771214\\
1052	-9.16763234284986\\
1053	-10.5855469198123\\
1054	-9.62692243314837\\
1055	-13.2978129765042\\
1056	-12.0662501056272\\
1057	-7.0238109694933\\
1058	-6.68909287692804\\
1059	-6.80237218483217\\
1060	-8.70836447597162\\
1061	-6.09578041928171\\
1062	-5.11719399485401\\
1063	-7.53138908963592\\
1064	-7.11667504723755\\
1065	-5.30007081368649\\
1066	-6.44069960320485\\
1067	-7.85587656120629\\
1068	-11.4869286118555\\
1069	-11.2489357502618\\
1070	-11.9213224713646\\
1071	-11.563091964101\\
1072	-12.3131095466058\\
1073	-10.8292027062421\\
1074	-10.376788417042\\
1075	-10.1278121368732\\
1076	-9.18540076166641\\
1077	-9.64873088330592\\
1078	-13.5786640011893\\
1079	-19.2151042552443\\
1080	-20.4309925051282\\
1081	-19.4011934568148\\
1082	-13.9846018462665\\
1083	-17.5820935659285\\
1084	-21.3914735569547\\
1085	-20.9894371365499\\
1086	-19.2251129801361\\
1087	-20.6567985350146\\
1088	-25.871540397085\\
1089	-20.1040084686842\\
1090	-19.7788734176554\\
1091	-14.0547919030103\\
1092	-10.5020775769694\\
1093	-8.93948635895595\\
1094	-11.9095211130746\\
1095	-9.74022782075849\\
1096	-5.90471713389843\\
1097	-6.00028633532452\\
1098	-7.18321881403131\\
1099	-10.5253796365896\\
1100	-10.6239949520661\\
1101	-9.58476993754812\\
1102	-12.1227085870494\\
1103	-12.396746947778\\
1104	-15.7277964974055\\
1105	-13.998104857161\\
1106	-15.3174934437861\\
1107	-12.9288283530607\\
1108	-11.1973067148197\\
1109	-12.9829019920442\\
1110	-10.1244754782817\\
1111	-9.17638357805026\\
1112	-10.9813025964627\\
1113	-11.2641596204735\\
1114	-8.07178111958888\\
1115	-8.78209716521906\\
1116	-7.3400757490474\\
1117	-6.98962072961626\\
1118	-8.25921154359844\\
1119	-6.53933528702919\\
1120	-6.87248281729808\\
1121	-8.42679818985221\\
1122	-12.1345377799273\\
1123	-13.2318263067621\\
1124	-13.1849452153433\\
1125	-8.86895014546189\\
1126	-10.4700855614419\\
1127	-16.5199350593346\\
1128	-14.1341405450662\\
1129	-14.0662491672206\\
1130	-15.2944942466017\\
1131	-9.76272482503055\\
1132	-7.87360739568294\\
1133	-9.67621835086311\\
1134	-11.4232770819779\\
1135	-15.3539914967366\\
1136	-19.1266233856859\\
1137	-18.6738119604634\\
1138	-14.5209016685303\\
1139	-15.011625102487\\
1140	-14.5340154403361\\
1141	-13.3793567461551\\
1142	-13.3025421126588\\
1143	-9.62030860126197\\
1144	-8.24464968831148\\
1145	-8.39287377115738\\
1146	-8.11105520804358\\
1147	-8.92847612562673\\
1148	-11.6697682023273\\
1149	-16.093087477873\\
1150	-13.9200428072577\\
1151	-10.1933919276889\\
1152	-8.91658967399758\\
1153	-8.48155572090559\\
1154	-5.92440504004321\\
1155	-4.57700238576831\\
1156	-4.78266726750849\\
1157	-8.15687491983908\\
1158	-7.0805856913928\\
1159	-6.1612259347869\\
1160	-7.67345839587915\\
1161	-7.55029988198112\\
1162	-5.50681834519605\\
1163	-4.19905370328888\\
1164	-3.38041113617692\\
1165	-4.08491600793036\\
1166	-9.62703669004009\\
1167	-14.0755446195146\\
1168	-14.696865264699\\
1169	-10.7684982725\\
1170	-7.99602773501988\\
1171	-6.67348339275327\\
1172	-5.32367119440414\\
1173	-7.29279942747394\\
1174	-9.81198104281407\\
1175	-10.4706540381392\\
1176	-12.7525031722078\\
1177	-19.8384004505224\\
1178	-22.918227150124\\
1179	-20.3474579150579\\
1180	-19.9298469468868\\
1181	-14.2655558042409\\
1182	-15.1290099706657\\
1183	-16.3873488583124\\
1184	-15.9699316071353\\
1185	-14.1455177599073\\
1186	-12.8807447747108\\
1187	-13.8902505682761\\
1188	-11.5917954227659\\
1189	-13.2774310707789\\
1190	-11.5954324617232\\
1191	-15.5730271829786\\
1192	-18.8394038952938\\
1193	-13.8816498581666\\
1194	-10.3782253513564\\
1195	-9.92022381071251\\
1196	-15.2767894078243\\
1197	-18.0769476030248\\
1198	-21.0224988363236\\
1199	-21.1126299285631\\
1200	-21.0906852738883\\
1201	-17.3808153708005\\
1202	-16.4248726900355\\
1203	-17.5628424189294\\
1204	-19.119659533992\\
1205	-14.5398965673073\\
1206	-9.80192311144719\\
1207	-12.9378266943684\\
1208	-12.1715982277294\\
1209	-7.98938096083589\\
1210	-9.63340065777425\\
1211	-11.4165954293437\\
1212	-9.21527699490935\\
1213	-8.1150508430635\\
1214	-9.42037161621442\\
1215	-8.77903618911932\\
1216	-10.2836654308715\\
1217	-13.9377230713051\\
1218	-13.055110628118\\
1219	-9.64342812764312\\
1220	-9.49931449758219\\
1221	-13.2486645393858\\
1222	-20.6153290634618\\
1223	-17.2142646588563\\
1224	-10.6267243179461\\
1225	-9.3828858767622\\
1226	-9.77188456379417\\
1227	-9.74099518557841\\
1228	-7.54729372400971\\
1229	-7.79037529350285\\
1230	-9.25226196432845\\
1231	-11.1424200178358\\
1232	-11.5840050447398\\
1233	-9.19877733528665\\
1234	-11.7371298515982\\
1235	-14.420486911348\\
1236	-12.9877044991565\\
1237	-11.2822929772171\\
1238	-12.149052166099\\
1239	-14.5951788410357\\
1240	-10.7169915813411\\
1241	-5.71698502656091\\
1242	-6.61013963880578\\
1243	-8.68365034626413\\
1244	-10.6946467232635\\
1245	-10.1332513797397\\
1246	-6.99716677617562\\
1247	-9.54108861490454\\
1248	-10.3221358313149\\
1249	-7.62969049227968\\
1250	-9.12404324153567\\
1251	-9.03208371465757\\
1252	-7.76341122771987\\
1253	-8.79051266287206\\
1254	-6.44608015066683\\
1255	-5.8890643359066\\
1256	-5.90286882016949\\
1257	-5.1006239783635\\
1258	-9.21018485215723\\
1259	-10.8714062431882\\
1260	-13.3899415019494\\
1261	-16.4322221205612\\
1262	-12.3446746719892\\
1263	-7.69357000215405\\
1264	-6.71202854156968\\
1265	-6.16600513081835\\
1266	-8.558653012994\\
1267	-7.05891728443567\\
1268	-6.25772581592992\\
1269	-7.76575210029955\\
1270	-12.9077781734527\\
1271	-12.5327935549237\\
1272	-14.2362364074163\\
1273	-10.6595119531592\\
1274	-9.37342403819116\\
1275	-7.092914610055\\
1276	-5.90309986920442\\
1277	-5.64558445390639\\
1278	-4.53471942083138\\
1279	-5.44809308147455\\
1280	-8.09365565425128\\
1281	-9.09720717688697\\
1282	-9.20426488603948\\
1283	-9.78168668812918\\
1284	-14.5509421993681\\
1285	-14.2610222821343\\
1286	-10.2742550611081\\
1287	-12.2734030995721\\
1288	-12.4442003833285\\
1289	-14.5890623638497\\
1290	-12.9816121247413\\
1291	-9.91051054613331\\
1292	-6.05807881231885\\
1293	-5.39933640659956\\
1294	-4.72279353237423\\
1295	-5.18232006141648\\
1296	-5.59931864753185\\
1297	-5.23208326753115\\
1298	-6.27409721652907\\
1299	-9.19192379160995\\
1300	-8.80574870329815\\
1301	-7.96543625564595\\
1302	-9.36663975124634\\
1303	-6.7225437143361\\
1304	-4.12729823677682\\
1305	-6.41779486963264\\
1306	-10.8213369269854\\
1307	-10.0090302163976\\
1308	-10.1304524003872\\
1309	-9.93344781374367\\
1310	-7.04180533879048\\
1311	-8.99565311145125\\
1312	-13.3939066286905\\
1313	-12.7301403513688\\
1314	-9.51925453195167\\
1315	-13.7374500880801\\
1316	-13.6390720304655\\
1317	-11.4204654252405\\
1318	-12.9933390653828\\
1319	-12.6064105995024\\
1320	-10.759869990475\\
1321	-9.87222304271972\\
1322	-13.2599974609559\\
1323	-18.0630605728004\\
1324	-15.1474151343816\\
1325	-9.7676796687133\\
1326	-10.0451806857047\\
1327	-10.2955321988341\\
1328	-12.0493146693568\\
1329	-13.5525256465517\\
1330	-10.6650526497306\\
1331	-11.1847731816526\\
1332	-13.2315198034332\\
1333	-14.2281136471527\\
1334	-9.82620868964159\\
1335	-9.33502901839506\\
1336	-11.9984112766574\\
1337	-16.4249940595621\\
1338	-15.5066874717907\\
1339	-15.1995833611672\\
1340	-10.309398523224\\
1341	-9.23332837100838\\
1342	-10.4832640563839\\
1343	-7.339590531574\\
1344	-5.14345936194491\\
1345	-4.18722799976445\\
1346	-3.83265555283629\\
1347	-6.23048202941641\\
1348	-10.1661969674683\\
1349	-10.9263030407637\\
1350	-8.26059971905534\\
1351	-8.18875359654294\\
1352	-9.63553070361559\\
1353	-6.63409783118982\\
1354	-6.69754881880885\\
1355	-7.24577794144716\\
1356	-5.53383929910615\\
1357	-5.70248435984672\\
1358	-9.80969975129278\\
1359	-12.1607027646826\\
1360	-7.8666167797835\\
1361	-6.12232931898666\\
1362	-5.72277092240961\\
1363	-6.28931942162421\\
1364	-5.49940331402831\\
1365	-9.16136567902026\\
1366	-14.9693895647581\\
1367	-13.0491596827928\\
1368	-15.1487898387186\\
1369	-17.4672884669312\\
1370	-17.513670025385\\
1371	-14.1529269776986\\
1372	-10.8941903058567\\
1373	-10.7348043964066\\
1374	-11.1370228521639\\
1375	-12.4174660881088\\
1376	-13.7760612981936\\
1377	-14.1654465188799\\
1378	-17.8890853860576\\
1379	-19.808225135728\\
1380	-22.4927120075943\\
1381	-17.7199476547464\\
1382	-17.6397154162739\\
1383	-19.8607312879099\\
1384	-15.829443356228\\
1385	-18.1952448190057\\
1386	-16.3447546913017\\
1387	-9.98359117275868\\
1388	-7.50059558070802\\
1389	-7.03334756354138\\
1390	-10.6328836840751\\
1391	-10.14218572864\\
1392	-6.79383622047681\\
1393	-8.29325901665379\\
1394	-7.43252082583922\\
1395	-5.38225046798743\\
1396	-5.9976082279145\\
1397	-7.25563368779672\\
1398	-6.05965616598428\\
1399	-8.60973646062257\\
1400	-11.6061632097164\\
1401	-9.11305841491185\\
1403	-17.9023425827522\\
1404	-17.2543256664778\\
1405	-19.8322895980052\\
};
\addlegendentry{OSA predition}

\addplot [color=mycolor3, dotted, line width=2.0pt]
  table[row sep=crcr]{%
1006	-14.6479999999999\\
1007	-17.0899999999999\\
1008	-10.9860000000001\\
1009	-7.32400000000007\\
1010	-9.15567496596873\\
1011	-9.31783769544859\\
1012	-10.8344800628556\\
1013	-9.65800110641703\\
1014	-4.8140064257027\\
1015	-4.36120419579242\\
1016	-3.61668848126578\\
1017	-6.99250901817891\\
1018	-12.887658298481\\
1019	-12.7427397806966\\
1020	-12.3980767585117\\
1021	-10.2145935196754\\
1022	-6.916922056472\\
1023	-11.4329647793566\\
1024	-10.4742475358926\\
1025	-6.4339986879038\\
1026	-11.0714746166375\\
1027	-11.8544075123627\\
1028	-8.88847615145573\\
1029	-12.7332012443435\\
1030	-13.0104065509565\\
1031	-9.64720621907099\\
1032	-13.0246384443301\\
1033	-13.4872153861907\\
1034	-10.9433196669913\\
1035	-9.41146598968749\\
1036	-8.53642742338934\\
1037	-8.98377989584742\\
1038	-10.9870317007692\\
1039	-10.5931853828861\\
1040	-10.6951893405835\\
1041	-11.0683551414597\\
1042	-11.2179622581291\\
1043	-14.5756437353252\\
1044	-13.994964166639\\
1045	-9.81851353451475\\
1046	-9.33092674888758\\
1047	-6.58903085037286\\
1048	-5.92721614681022\\
1049	-8.24463657968226\\
1050	-7.46612451939131\\
1051	-6.54276092367809\\
1052	-9.12793182795963\\
1053	-10.4686394358782\\
1054	-9.45335247663616\\
1055	-13.3065715064338\\
1056	-12.0121086433167\\
1057	-6.91228058996057\\
1058	-6.49992201971872\\
1059	-7.44119523464951\\
1060	-8.73095119883305\\
1061	-5.84766657989553\\
1062	-4.90161857384078\\
1063	-7.36376937720752\\
1064	-7.06848465517487\\
1065	-4.97041136718644\\
1066	-6.36811790379966\\
1067	-7.71833778644168\\
1068	-11.4342020603265\\
1069	-11.2090543644704\\
1070	-12.0191384706045\\
1071	-11.7830530415622\\
1072	-12.4043903842787\\
1073	-10.914380894744\\
1074	-10.2424345600832\\
1075	-9.87997615046856\\
1076	-9.02571801181102\\
1077	-9.34082814423323\\
1078	-13.501455366988\\
1079	-19.106853131052\\
1080	-20.1184864464817\\
1081	-19.049595549327\\
1082	-13.6245918864315\\
1083	-17.5351754089943\\
1084	-21.0378814283602\\
1085	-20.4483681645529\\
1086	-18.4233517679581\\
1087	-20.2452371704621\\
1088	-25.8856770620366\\
1089	-20.5128815176461\\
1090	-19.1004753845546\\
1091	-13.7997635327245\\
1092	-10.3929608850708\\
1093	-9.56441998324249\\
1094	-11.7605909827444\\
1095	-9.73093712572381\\
1096	-6.20992075485128\\
1097	-6.24868843747913\\
1098	-7.41150807864847\\
1099	-10.5100340949812\\
1100	-10.5855140684544\\
1101	-9.50135592782431\\
1102	-12.1307872386026\\
1103	-12.3395465586525\\
1104	-15.7688537012325\\
1105	-14.0833789485578\\
1106	-15.4406609179614\\
1107	-12.8602382910044\\
1108	-11.1193359914889\\
1109	-12.5730925179932\\
1110	-10.2485375964286\\
1111	-9.24955410651842\\
1112	-11.1006775791325\\
1113	-11.3876935390222\\
1114	-8.22442028025034\\
1115	-8.82275075210077\\
1116	-7.40063369908626\\
1117	-6.96510073878653\\
1118	-8.28810413538599\\
1119	-6.56440805704779\\
1120	-6.95824386207005\\
1121	-8.68390454573819\\
1122	-12.2032222095127\\
1123	-13.2203669391827\\
1124	-13.1500620944253\\
1125	-8.6934156623679\\
1126	-10.5896534394776\\
1127	-16.3206404720836\\
1128	-13.7361105883931\\
1129	-13.4773830467825\\
1130	-15.140560779452\\
1131	-9.7525152391222\\
1132	-7.63167339430015\\
1133	-9.68088149697064\\
1134	-11.3251029886651\\
1135	-15.5299197983807\\
1136	-19.2070684558046\\
1137	-18.6049395323701\\
1138	-14.0956776065996\\
1139	-14.6383090670295\\
1140	-14.2607811162309\\
1141	-13.2157526035855\\
1142	-13.2000493688561\\
1143	-9.89564640330673\\
1144	-8.5283136159369\\
1145	-8.59817162233503\\
1146	-8.14132378017166\\
1147	-8.86968049637676\\
1148	-11.5690262002092\\
1149	-16.0858005072666\\
1150	-13.8135759113982\\
1151	-9.9758868572917\\
1152	-9.21436701292987\\
1153	-9.09260540181731\\
1154	-6.19688559206134\\
1155	-4.53124873684237\\
1156	-4.91599056059977\\
1157	-8.23255719376948\\
1158	-7.17412474136836\\
1159	-6.05238664539979\\
1160	-7.50338080155348\\
1161	-7.34097096201754\\
1162	-5.66717323380021\\
1163	-4.31294632134473\\
1164	-3.54694631585016\\
1165	-4.45193547563804\\
1166	-9.6752618135788\\
1167	-13.7601015841906\\
1168	-14.7301250650521\\
1169	-10.7014789725306\\
1170	-8.18338464632825\\
1172	-4.77683426889462\\
1173	-7.15387883029553\\
1174	-9.72355723976648\\
1175	-10.589143179571\\
1176	-12.9876192043757\\
1177	-19.2485611525699\\
1178	-22.9569543615446\\
1179	-20.1862789287948\\
1180	-19.491691457559\\
1181	-14.092528274796\\
1182	-14.8947322289209\\
1183	-16.555435960405\\
1184	-15.9318157630435\\
1185	-13.6646298500757\\
1186	-12.7667851758647\\
1187	-13.3468880355183\\
1188	-11.8245381584229\\
1189	-13.2186184522097\\
1190	-11.6608466681798\\
1191	-15.4473175647436\\
1192	-18.7576529305586\\
1193	-13.8461194442887\\
1194	-9.89594018998901\\
1195	-10.3772296412783\\
1196	-15.4754662961361\\
1197	-18.0211628952591\\
1198	-21.1034650903509\\
1199	-21.2105676379751\\
1200	-21.2747938274472\\
1201	-17.65536206301\\
1202	-16.3951861494261\\
1203	-17.8665083572853\\
1204	-19.4944748809644\\
1205	-14.8760634774303\\
1206	-9.7614891603987\\
1207	-13.0439462433228\\
1208	-12.4392675468757\\
1209	-8.07100810217366\\
1210	-9.79803586017169\\
1211	-11.8763977598182\\
1212	-9.39109113513246\\
1213	-8.26888555629534\\
1214	-9.74531414728017\\
1215	-9.16829503094436\\
1216	-10.5085270191416\\
1217	-14.1477034078084\\
1218	-13.3147774779077\\
1219	-9.62978044590454\\
1220	-9.65185988261533\\
1221	-13.3176001736367\\
1222	-20.557872440125\\
1223	-17.3818707357548\\
1224	-11.1108832471214\\
1225	-10.0362787962463\\
1226	-10.2679685752339\\
1227	-10.1143344344885\\
1228	-7.96523646794003\\
1229	-8.06620300900454\\
1231	-11.204545310685\\
1232	-11.7272227921421\\
1233	-9.41321267371472\\
1234	-11.9449057413763\\
1235	-14.7547090800535\\
1236	-13.4424530798512\\
1237	-11.5503244520676\\
1238	-12.4206871369647\\
1239	-14.7497531807539\\
1240	-11.1021259606009\\
1241	-5.88115993960696\\
1242	-6.92719658670035\\
1243	-8.63664902687651\\
1244	-10.4859997027406\\
1245	-9.81525712504981\\
1246	-7.25936335765323\\
1247	-10.0587289205014\\
1248	-10.8179681376691\\
1249	-7.76042797770469\\
1250	-9.06403289978016\\
1251	-9.14058893653942\\
1252	-8.04989265571248\\
1253	-9.02773428190471\\
1254	-6.79622473332438\\
1255	-6.4689139496013\\
1256	-6.1097512017484\\
1257	-5.37470033175259\\
1258	-9.27560949690974\\
1259	-10.878523370571\\
1260	-13.2066270360583\\
1261	-16.4832648576692\\
1263	-7.98984988484995\\
1264	-7.0059073828711\\
1265	-6.54060194268959\\
1266	-8.85159230597469\\
1267	-7.19672276994834\\
1268	-6.6078854112186\\
1269	-8.15688786046303\\
1270	-13.0787612708714\\
1271	-12.7614717620195\\
1272	-14.605433822461\\
1273	-11.3203212561145\\
1274	-9.72643654674107\\
1275	-6.50658855588267\\
1276	-5.64155034167516\\
1277	-4.82071824684681\\
1278	-4.21558042269817\\
1279	-5.23286861000406\\
1280	-8.27359980365145\\
1281	-9.06910474126221\\
1282	-9.12266178942696\\
1283	-9.68083259441505\\
1284	-14.4681997273822\\
1285	-14.1267991931049\\
1286	-10.1885601813588\\
1287	-12.1342670486183\\
1288	-12.8476485276035\\
1289	-14.6168013840663\\
1290	-13.0426479542461\\
1291	-9.60526446065774\\
1292	-5.53665251586222\\
1293	-4.7731348781233\\
1294	-4.43481121385025\\
1295	-5.14588800768797\\
1296	-5.64711960346858\\
1297	-5.15719080482563\\
1298	-6.21690506812524\\
1299	-9.13271471469761\\
1300	-8.74458144860523\\
1301	-7.88718493543342\\
1302	-9.21613736752283\\
1303	-6.47709049225432\\
1304	-3.97804426629386\\
1305	-6.36416426638721\\
1306	-10.8191479326335\\
1307	-9.94941711761453\\
1308	-10.1293137318573\\
1309	-9.92852624436159\\
1310	-7.18479239289127\\
1311	-8.83416855473729\\
1312	-13.4429390645068\\
1313	-12.7987864139209\\
1314	-9.57894784532118\\
1315	-13.6540689242331\\
1316	-13.1664129203441\\
1317	-11.0890496774823\\
1318	-12.7517040065914\\
1319	-12.9312384552006\\
1320	-10.5295836485102\\
1321	-9.73936123037333\\
1322	-13.2527161878359\\
1323	-18.1012973836666\\
1324	-15.2186237910723\\
1325	-9.95629999236735\\
1326	-10.0130748623922\\
1327	-10.1941442732304\\
1328	-12.072567751879\\
1329	-13.3834027744856\\
1330	-10.7045312521866\\
1331	-11.2009380516863\\
1332	-13.3501285899738\\
1333	-14.371185601645\\
1334	-10.0432211558714\\
1335	-9.09719794050943\\
1336	-11.9932631669917\\
1337	-16.4745000218197\\
1338	-15.7140442292366\\
1339	-15.4990277180441\\
1340	-10.5238634245445\\
1341	-9.3568032248827\\
1342	-10.2266775001121\\
1343	-6.90263607465477\\
1344	-4.80044451512663\\
1345	-4.13167176946968\\
1346	-3.57031450126374\\
1347	-6.23817185043617\\
1348	-10.1069430248608\\
1349	-10.8328223217843\\
1350	-8.12287678992243\\
1351	-8.13355347917582\\
1352	-9.61587881993546\\
1353	-6.35924040180794\\
1354	-6.18306155482924\\
1355	-7.14073360429961\\
1356	-5.40916883806221\\
1357	-6.0371221292271\\
1358	-9.82434243928492\\
1359	-12.2319222514639\\
1360	-7.98612719581047\\
1361	-6.44836632595911\\
1362	-6.0123535341304\\
1363	-6.37517970746012\\
1364	-5.43136276167252\\
1365	-9.03160502147648\\
1366	-15.0465732657142\\
1367	-13.0319689513497\\
1368	-15.0233158589615\\
1369	-17.2338496644415\\
1370	-17.240811014432\\
1371	-13.8017209906718\\
1372	-10.9350229708293\\
1373	-10.793156439182\\
1374	-11.239278940433\\
1375	-12.5003773068845\\
1376	-13.908430424585\\
1377	-14.1928741621634\\
1378	-17.8298084255853\\
1379	-19.7726517388774\\
1380	-22.5494324150336\\
1381	-17.8904550110913\\
1382	-17.7162080458199\\
1383	-19.756515475462\\
1384	-15.41251213484\\
1385	-17.5802354866073\\
1386	-16.4379127429631\\
1387	-10.2716167291121\\
1388	-7.72692793554029\\
1389	-7.33581987533717\\
1390	-10.4116788287322\\
1391	-9.90902712666275\\
1392	-6.62995729113368\\
1393	-8.3613222202639\\
1394	-7.84393612517692\\
1395	-5.37661352118334\\
1396	-6.25190027106987\\
1397	-7.41183553160113\\
1398	-6.22041832597642\\
1399	-8.56428956179843\\
1400	-11.6089331937219\\
1401	-9.10481872672221\\
1402	-13.4791651588353\\
1403	-18.0234732322385\\
1404	-17.3413075491781\\
1405	-19.8329242062991\\
};
\addlegendentry{MPO prediction}

\end{axis}

\begin{axis}[%
width=6.159cm,
height=1.831cm,
at={(0cm,5.085cm)},
scale only axis,
xmin=1000,
xmax=1405,
xlabel style={font=\color{white!15!black}},
xlabel={Sample index},
ymin=-28.1498026372379,
ymax=0,
ylabel style={font=\color{white!15!black}},
ylabel={$y(t)$},
axis background/.style={fill=white},
title style={font=\bfseries},
title={C5: RMSE(OSA) = 1.5754, RMSE(MPO) = 1.5597},
legend style={legend cell align=left, align=left, draw=white!15!black}
]
\addplot [color=mycolor1, line width=2.0pt]
  table[row sep=crcr]{%
1006	-15.8689999999999\\
1007	-14.6479999999999\\
1008	-14.6479999999999\\
1009	-7.32400000000007\\
1010	-8.54500000000007\\
1011	-10.9860000000001\\
1012	-9.76600000000008\\
1013	-9.76600000000008\\
1015	-2.44100000000003\\
1016	-3.66200000000003\\
1017	-2.44100000000003\\
1018	-12.2070000000001\\
1019	-13.4280000000001\\
1021	-10.9860000000001\\
1022	-6.10400000000004\\
1023	-6.10400000000004\\
1024	-1.221\\
1026	-10.9860000000001\\
1027	-12.2070000000001\\
1028	-8.54500000000007\\
1029	-14.6479999999999\\
1030	-14.6479999999999\\
1031	-9.76600000000008\\
1032	-13.4280000000001\\
1033	-12.2070000000001\\
1034	-9.76600000000008\\
1035	-9.76600000000008\\
1036	-8.54500000000007\\
1037	-8.54500000000007\\
1038	-12.2070000000001\\
1039	-10.9860000000001\\
1042	-10.9860000000001\\
1043	-14.6479999999999\\
1044	-15.8689999999999\\
1045	-9.76600000000008\\
1046	-8.54500000000007\\
1047	-8.54500000000007\\
1048	-6.10400000000004\\
1049	-7.32400000000007\\
1050	-7.32400000000007\\
1051	-4.88300000000004\\
1052	-9.76600000000008\\
1053	-10.9860000000001\\
1054	-8.54500000000007\\
1055	-13.4280000000001\\
1056	-15.8689999999999\\
1057	-8.54500000000007\\
1058	-6.10400000000004\\
1059	-6.10400000000004\\
1060	-9.76600000000008\\
1061	-6.10400000000004\\
1062	-4.88300000000004\\
1063	-7.32400000000007\\
1064	-7.32400000000007\\
1065	-3.66200000000003\\
1068	-10.9860000000001\\
1069	-10.9860000000001\\
1070	-9.76600000000008\\
1071	-13.4280000000001\\
1072	-10.9860000000001\\
1073	-12.2070000000001\\
1074	-10.9860000000001\\
1075	-10.9860000000001\\
1076	-7.32400000000007\\
1077	-8.54500000000007\\
1079	-20.752\\
1080	-20.752\\
1081	-19.5309999999999\\
1082	-13.4280000000001\\
1083	-15.8689999999999\\
1084	-24.414\\
1085	-23.193\\
1086	-15.8689999999999\\
1088	-28.076\\
1089	-23.193\\
1090	-17.0899999999999\\
1091	-14.6479999999999\\
1092	-13.4280000000001\\
1093	-9.76600000000008\\
1095	-12.2070000000001\\
1096	-4.88300000000004\\
1097	-4.88300000000004\\
1098	-7.32400000000007\\
1099	-10.9860000000001\\
1100	-9.76600000000008\\
1101	-9.76600000000008\\
1102	-10.9860000000001\\
1103	-13.4280000000001\\
1105	-15.8689999999999\\
1106	-14.6479999999999\\
1107	-17.0899999999999\\
1108	-12.2070000000001\\
1109	-12.2070000000001\\
1110	-10.9860000000001\\
1111	-8.54500000000007\\
1112	-9.76600000000008\\
1113	-12.2070000000001\\
1114	-8.54500000000007\\
1115	-9.76600000000008\\
1116	-7.32400000000007\\
1117	-6.10400000000004\\
1118	-8.54500000000007\\
1119	-6.10400000000004\\
1120	-4.88300000000004\\
1121	-8.54500000000007\\
1122	-13.4280000000001\\
1124	-10.9860000000001\\
1125	-7.32400000000007\\
1126	-8.54500000000007\\
1127	-19.5309999999999\\
1128	-14.6479999999999\\
1129	-10.9860000000001\\
1130	-17.0899999999999\\
1131	-12.2070000000001\\
1132	-6.10400000000004\\
1133	-9.76600000000008\\
1134	-8.54500000000007\\
1135	-14.6479999999999\\
1136	-15.8689999999999\\
1137	-19.5309999999999\\
1138	-14.6479999999999\\
1139	-14.6479999999999\\
1140	-13.4280000000001\\
1141	-13.4280000000001\\
1142	-14.6479999999999\\
1144	-7.32400000000007\\
1145	-7.32400000000007\\
1146	-6.10400000000004\\
1148	-10.9860000000001\\
1149	-15.8689999999999\\
1150	-14.6479999999999\\
1151	-8.54500000000007\\
1152	-8.54500000000007\\
1153	-9.76600000000008\\
1155	-2.44100000000003\\
1157	-7.32400000000007\\
1158	-8.54500000000007\\
1159	-6.10400000000004\\
1160	-7.32400000000007\\
1162	-4.88300000000004\\
1163	-4.88300000000004\\
1164	-1.221\\
1165	-3.66200000000003\\
1166	-12.2070000000001\\
1167	-14.6479999999999\\
1168	-15.8689999999999\\
1169	-12.2070000000001\\
1170	-7.32400000000007\\
1171	-6.10400000000004\\
1172	-3.66200000000003\\
1173	-7.32400000000007\\
1175	-9.76600000000008\\
1176	-12.2070000000001\\
1177	-18.3109999999999\\
1178	-18.3109999999999\\
1179	-19.5309999999999\\
1180	-18.3109999999999\\
1181	-15.8689999999999\\
1182	-15.8689999999999\\
1184	-18.3109999999999\\
1185	-12.2070000000001\\
1186	-10.9860000000001\\
1187	-12.2070000000001\\
1188	-10.9860000000001\\
1189	-12.2070000000001\\
1190	-9.76600000000008\\
1191	-15.8689999999999\\
1192	-18.3109999999999\\
1193	-12.2070000000001\\
1194	-7.32400000000007\\
1195	-9.76600000000008\\
1196	-17.0899999999999\\
1197	-18.3109999999999\\
1198	-18.3109999999999\\
1199	-21.973\\
1200	-20.752\\
1201	-17.0899999999999\\
1202	-15.8689999999999\\
1204	-20.752\\
1205	-17.0899999999999\\
1206	-8.54500000000007\\
1207	-14.6479999999999\\
1208	-12.2070000000001\\
1209	-7.32400000000007\\
1210	-10.9860000000001\\
1213	-7.32400000000007\\
1214	-8.54500000000007\\
1215	-8.54500000000007\\
1216	-9.76600000000008\\
1217	-13.4280000000001\\
1218	-14.6479999999999\\
1219	-7.32400000000007\\
1220	-8.54500000000007\\
1221	-12.2070000000001\\
1222	-17.0899999999999\\
1223	-15.8689999999999\\
1224	-8.54500000000007\\
1225	-8.54500000000007\\
1226	-9.76600000000008\\
1229	-6.10400000000004\\
1230	-8.54500000000007\\
1231	-9.76600000000008\\
1233	-9.76600000000008\\
1235	-14.6479999999999\\
1236	-15.8689999999999\\
1237	-10.9860000000001\\
1238	-13.4280000000001\\
1239	-17.0899999999999\\
1240	-12.2070000000001\\
1241	-3.66200000000003\\
1242	-9.76600000000008\\
1243	-8.54500000000007\\
1244	-9.76600000000008\\
1245	-8.54500000000007\\
1247	-8.54500000000007\\
1248	-10.9860000000001\\
1249	-7.32400000000007\\
1250	-7.32400000000007\\
1251	-9.76600000000008\\
1252	-6.10400000000004\\
1253	-8.54500000000007\\
1255	-3.66200000000003\\
1256	-6.10400000000004\\
1257	-4.88300000000004\\
1258	-10.9860000000001\\
1259	-12.2070000000001\\
1260	-12.2070000000001\\
1261	-18.3109999999999\\
1262	-10.9860000000001\\
1263	-4.88300000000004\\
1264	-6.10400000000004\\
1265	-6.10400000000004\\
1266	-8.54500000000007\\
1267	-6.10400000000004\\
1268	-8.54500000000007\\
1269	-7.32400000000007\\
1270	-13.4280000000001\\
1271	-9.76600000000008\\
1272	-14.6479999999999\\
1273	-10.9860000000001\\
1274	-9.76600000000008\\
1275	-6.10400000000004\\
1276	-3.66200000000003\\
1277	-3.66200000000003\\
1278	-1.221\\
1279	-2.44100000000003\\
1280	-8.54500000000007\\
1281	-8.54500000000007\\
1282	-7.32400000000007\\
1283	-9.76600000000008\\
1284	-15.8689999999999\\
1285	-10.9860000000001\\
1286	-9.76600000000008\\
1287	-9.76600000000008\\
1288	-13.4280000000001\\
1289	-14.6479999999999\\
1290	-12.2070000000001\\
1291	-12.2070000000001\\
1292	-6.10400000000004\\
1293	-2.44100000000003\\
1294	-4.88300000000004\\
1295	-4.88300000000004\\
1296	-6.10400000000004\\
1297	-3.66200000000003\\
1298	-4.88300000000004\\
1299	-9.76600000000008\\
1300	-7.32400000000007\\
1301	-7.32400000000007\\
1302	-10.9860000000001\\
1303	-7.32400000000007\\
1304	-4.88300000000004\\
1305	-9.76600000000008\\
1306	-12.2070000000001\\
1307	-9.76600000000008\\
1308	-10.9860000000001\\
1309	-7.32400000000007\\
1310	-7.32400000000007\\
1312	-14.6479999999999\\
1313	-13.4280000000001\\
1314	-8.54500000000007\\
1315	-14.6479999999999\\
1316	-13.4280000000001\\
1319	-13.4280000000001\\
1320	-10.9860000000001\\
1321	-10.9860000000001\\
1322	-12.2070000000001\\
1323	-18.3109999999999\\
1324	-17.0899999999999\\
1325	-10.9860000000001\\
1327	-10.9860000000001\\
1328	-13.4280000000001\\
1329	-13.4280000000001\\
1330	-9.76600000000008\\
1332	-14.6479999999999\\
1333	-14.6479999999999\\
1334	-9.76600000000008\\
1335	-8.54500000000007\\
1336	-10.9860000000001\\
1337	-18.3109999999999\\
1338	-14.6479999999999\\
1339	-13.4280000000001\\
1340	-9.76600000000008\\
1341	-10.9860000000001\\
1342	-9.76600000000008\\
1343	-6.10400000000004\\
1345	-3.66200000000003\\
1346	-3.66200000000003\\
1348	-10.9860000000001\\
1349	-9.76600000000008\\
1350	-6.10400000000004\\
1351	-7.32400000000007\\
1352	-9.76600000000008\\
1353	-6.10400000000004\\
1354	-6.10400000000004\\
1355	-9.76600000000008\\
1356	-6.10400000000004\\
1357	-7.32400000000007\\
1358	-12.2070000000001\\
1359	-12.2070000000001\\
1361	-4.88300000000004\\
1363	-7.32400000000007\\
1364	-6.10400000000004\\
1365	-7.32400000000007\\
1366	-14.6479999999999\\
1367	-9.76600000000008\\
1368	-17.0899999999999\\
1369	-20.752\\
1370	-18.3109999999999\\
1371	-13.4280000000001\\
1372	-10.9860000000001\\
1373	-10.9860000000001\\
1377	-15.8689999999999\\
1378	-20.752\\
1379	-20.752\\
1380	-24.414\\
1381	-15.8689999999999\\
1382	-20.752\\
1383	-20.752\\
1384	-14.6479999999999\\
1385	-17.0899999999999\\
1386	-18.3109999999999\\
1387	-9.76600000000008\\
1388	-7.32400000000007\\
1390	-9.76600000000008\\
1391	-7.32400000000007\\
1392	-6.10400000000004\\
1393	-7.32400000000007\\
1394	-6.10400000000004\\
1395	-3.66200000000003\\
1396	-4.88300000000004\\
1397	-7.32400000000007\\
1398	-2.44100000000003\\
1399	-10.9860000000001\\
1400	-8.54500000000007\\
1401	-7.32400000000007\\
1402	-15.8689999999999\\
1403	-19.5309999999999\\
1405	-19.5309999999999\\
};
\addlegendentry{True output}

\addplot [color=mycolor2, dashed, line width=2.0pt]
  table[row sep=crcr]{%
1006	-14.141662001158\\
1007	-16.1340647066668\\
1008	-13.3988444853255\\
1009	-7.68188441804045\\
1010	-9.28272331540165\\
1011	-10.5180980613625\\
1012	-10.4673830598222\\
1013	-9.40922167579311\\
1014	-3.9387865614608\\
1015	-3.42261315070732\\
1016	-3.16052161173116\\
1017	-6.98885576132352\\
1018	-12.5269754750441\\
1019	-11.9027905663277\\
1020	-12.3137341190943\\
1021	-9.931095639369\\
1022	-5.93881194264577\\
1023	-12.1883163357256\\
1024	-11.3683960416033\\
1025	-6.44161294997116\\
1026	-11.0368072201609\\
1027	-10.1389652448552\\
1028	-8.63662178178447\\
1029	-13.259397877441\\
1030	-12.8084671409363\\
1031	-9.49016818113751\\
1032	-13.9783574909138\\
1033	-13.6038921857487\\
1034	-10.874052190841\\
1035	-9.71527850215875\\
1036	-7.95241780820902\\
1037	-8.86872910948591\\
1038	-10.8253273106063\\
1039	-10.1170155039781\\
1040	-10.5131231073383\\
1041	-11.0648782752162\\
1042	-11.8744035021662\\
1043	-14.8444983632228\\
1044	-13.8482677981485\\
1045	-9.52356463363708\\
1046	-9.85208467605207\\
1047	-5.82376654951963\\
1048	-5.92008940336359\\
1049	-8.35191233215937\\
1050	-7.51433870435881\\
1051	-6.50590809503592\\
1052	-9.02786428172976\\
1053	-9.9740362419584\\
1054	-9.06799348950221\\
1055	-13.6234942570302\\
1056	-11.6089250741159\\
1057	-5.98688829891648\\
1058	-6.00517031349909\\
1059	-9.00926292478903\\
1060	-8.83318636240392\\
1061	-5.02636486825213\\
1062	-4.80129448071443\\
1063	-7.61922027492369\\
1064	-6.10642739706554\\
1065	-4.91818316384479\\
1066	-6.64363266305213\\
1067	-7.13561652760313\\
1068	-9.61082532980186\\
1069	-10.481906760217\\
1070	-11.0664600846044\\
1071	-11.314493035334\\
1072	-11.8394710904524\\
1073	-10.2306996731797\\
1074	-9.95144290782537\\
1075	-9.3294467209646\\
1076	-9.50775122838763\\
1077	-9.38260705623588\\
1078	-13.2115413075408\\
1079	-19.1373895105799\\
1080	-19.7318948973814\\
1081	-19.1940809603263\\
1082	-13.6222265756633\\
1083	-18.7729192270453\\
1084	-21.9462860185379\\
1085	-21.6623558249466\\
1086	-18.2786014318483\\
1087	-22.7241228174587\\
1088	-28.0683195398872\\
1089	-22.9671975004412\\
1090	-19.3739514254507\\
1091	-14.5271372517855\\
1092	-10.4121203351533\\
1093	-10.159739386226\\
1094	-12.1565478605801\\
1095	-10.2127303698326\\
1096	-5.88456675821635\\
1097	-6.27770380208949\\
1098	-8.19879071992\\
1099	-10.1835258753347\\
1100	-10.1318530122162\\
1101	-9.79754785397154\\
1102	-11.8297534173832\\
1103	-12.5975717456076\\
1104	-15.7971678553927\\
1105	-14.3229464165845\\
1106	-15.1449517026626\\
1107	-13.2420162540227\\
1108	-11.0845244887087\\
1109	-13.150358429669\\
1110	-11.0438629458995\\
1111	-9.17922100714759\\
1112	-11.0417043886598\\
1113	-11.1966600849582\\
1114	-7.97108711868373\\
1115	-8.94535111662731\\
1116	-7.41240629292861\\
1117	-7.53969560236601\\
1118	-8.5101759842687\\
1119	-6.09005666154098\\
1120	-6.70286595708671\\
1121	-9.2655841624487\\
1122	-11.3750602363684\\
1123	-12.4782606495696\\
1124	-13.3440466768354\\
1125	-8.71089499519803\\
1126	-11.0636489378194\\
1127	-16.6096525320056\\
1128	-13.2975065308451\\
1129	-14.7362743125489\\
1130	-15.8036032300099\\
1131	-8.93097802303464\\
1132	-6.87525270080232\\
1133	-11.0121804725577\\
1134	-11.1435847560642\\
1135	-15.9138953789229\\
1136	-18.7918417698875\\
1137	-17.8661981921744\\
1138	-14.3079724630099\\
1139	-14.2523036732543\\
1140	-15.0463291539768\\
1141	-13.0490932951902\\
1142	-13.7272548932986\\
1143	-9.95566977329736\\
1144	-9.23392373659658\\
1145	-9.04672194533146\\
1146	-7.91297783973346\\
1147	-8.71448241020744\\
1148	-11.9086811691566\\
1149	-16.5441658859042\\
1150	-13.8963824147052\\
1151	-9.47145341790178\\
1152	-9.09763566287097\\
1153	-9.12894577818088\\
1154	-5.3751333650016\\
1155	-4.09344786617021\\
1156	-5.08994340046002\\
1157	-8.01764261947915\\
1158	-6.96408670736946\\
1159	-6.11856173374144\\
1160	-7.62501707769593\\
1161	-6.78127971976119\\
1162	-5.17581871865445\\
1163	-3.83352361792686\\
1164	-3.07909044948792\\
1165	-4.47956740024938\\
1166	-9.32305291461125\\
1167	-13.3318141425771\\
1168	-15.0784710862561\\
1169	-10.1905758363016\\
1170	-7.97136025435179\\
1171	-6.00276433892031\\
1172	-3.96002154813209\\
1173	-7.34756327213563\\
1174	-9.01967486707804\\
1175	-10.029209728265\\
1176	-12.938965963802\\
1177	-19.4277409070837\\
1178	-22.703095405152\\
1179	-19.4985471263321\\
1180	-19.1503516445439\\
1181	-13.4066900801395\\
1182	-15.0001712542203\\
1183	-16.5160086098472\\
1184	-16.6441366826809\\
1185	-14.4404716746178\\
1186	-12.9223914165927\\
1187	-13.4511926932032\\
1188	-12.4754268429447\\
1189	-13.4242856799804\\
1190	-10.1545000447004\\
1191	-15.1779881621892\\
1192	-18.8549731614628\\
1193	-13.8977213443827\\
1194	-9.93461697203634\\
1195	-9.82575149801255\\
1196	-15.4987593109911\\
1197	-18.5484685098338\\
1198	-21.8764413128522\\
1199	-21.9821614536183\\
1200	-21.5288965158213\\
1201	-17.8947851699131\\
1202	-16.3855608438234\\
1203	-19.142824889168\\
1204	-21.0735993351029\\
1205	-14.3627377687744\\
1206	-9.97190921529022\\
1207	-13.9858880922022\\
1208	-13.3530529458058\\
1209	-7.22793572838941\\
1210	-10.9180450991566\\
1211	-12.1840232075861\\
1212	-9.67556453267593\\
1213	-7.8297482777125\\
1214	-9.765814511288\\
1215	-8.9766600684909\\
1216	-10.5264720949401\\
1217	-13.8815164822463\\
1218	-12.8096483414945\\
1219	-9.48840589841552\\
1220	-10.2632961065276\\
1221	-13.9174857844303\\
1222	-19.5296074203557\\
1223	-15.8422694068934\\
1224	-10.018172452837\\
1225	-8.27044082272391\\
1226	-10.3427629815812\\
1227	-9.52680433294449\\
1228	-7.01436417342552\\
1229	-7.86346054508113\\
1230	-9.70004222893499\\
1231	-11.328655563633\\
1232	-11.5907085596134\\
1233	-8.30067074579392\\
1234	-13.6919021655413\\
1235	-15.8693603367292\\
1236	-13.7366066981176\\
1237	-11.9038441668299\\
1238	-12.817743253079\\
1239	-15.6253561574254\\
1240	-10.9756608534872\\
1241	-5.23789599137604\\
1242	-8.26436370397187\\
1243	-8.67274870811957\\
1244	-10.389559043399\\
1245	-9.99904528922639\\
1246	-7.36307738937398\\
1247	-10.274522419634\\
1248	-11.0004152531058\\
1249	-7.52751216055162\\
1250	-9.20471027980921\\
1251	-9.13987172600878\\
1252	-7.47519913731958\\
1253	-8.89920773799395\\
1254	-6.28658597240656\\
1255	-6.42869303306588\\
1256	-6.00142187130746\\
1257	-5.20968702063965\\
1258	-9.23485095653314\\
1259	-10.7514233969816\\
1260	-13.1525262121877\\
1261	-16.7153073705451\\
1262	-12.1881178129609\\
1263	-7.50127597463461\\
1264	-6.54242130958824\\
1265	-6.09938032946548\\
1266	-8.03876270237924\\
1267	-5.95893535935261\\
1268	-5.56297222230137\\
1269	-8.01089239808471\\
1270	-11.951081598412\\
1271	-11.9477100273484\\
1272	-14.1709805356661\\
1273	-10.8138516477684\\
1274	-9.09261683638715\\
1275	-4.57063434262477\\
1276	-5.12652776092477\\
1277	-4.25375321200204\\
1278	-3.68094753195282\\
1279	-4.86011752429795\\
1280	-7.46205172374994\\
1281	-8.77888006613193\\
1282	-8.76874709109688\\
1283	-9.77122659054953\\
1284	-13.93425181253\\
1285	-13.8200558129461\\
1286	-10.535503429039\\
1287	-12.0013267634731\\
1288	-12.5704982773182\\
1289	-14.7178347451688\\
1290	-14.2353808904879\\
1291	-8.89438111176059\\
1292	-4.95218122160804\\
1293	-3.83484221476579\\
1294	-4.58825519192442\\
1295	-4.90446870609162\\
1296	-5.11401409198766\\
1297	-4.72764098221501\\
1298	-5.88145093831827\\
1299	-9.00850120610653\\
1300	-8.25899911032161\\
1301	-8.14279568457141\\
1302	-9.19952261117442\\
1303	-5.87903915094694\\
1304	-3.41340560889535\\
1305	-6.96706330359166\\
1306	-10.3521141717824\\
1307	-9.87719065374131\\
1308	-10.3411293832069\\
1309	-9.19282061255672\\
1310	-7.17032931363224\\
1311	-9.2761869024539\\
1312	-12.7919587735546\\
1313	-12.6764017141529\\
1314	-9.70344304486912\\
1315	-14.9824418180783\\
1316	-13.20689304951\\
1317	-10.7914999266723\\
1318	-13.673264422433\\
1319	-12.9873039382007\\
1320	-11.0835119015705\\
1321	-9.64384143019151\\
1322	-15.7560272043822\\
1323	-19.8044556129266\\
1324	-15.8661753244944\\
1325	-9.5878955359351\\
1326	-9.99308011079302\\
1327	-11.1589443104197\\
1328	-12.5223869577624\\
1329	-13.7971075600008\\
1330	-11.2957114610067\\
1331	-11.2467887597065\\
1332	-14.4592375689226\\
1333	-14.1392685795374\\
1334	-12.0388329666894\\
1335	-8.62802502604018\\
1336	-11.8804571504822\\
1337	-16.8345802216243\\
1338	-16.2519339069331\\
1339	-15.2158824377259\\
1340	-10.6971535242335\\
1341	-8.6410849311419\\
1342	-10.0397418654156\\
1343	-6.01907525696447\\
1344	-4.09211626526167\\
1345	-3.54550731947347\\
1346	-3.16425183270121\\
1347	-5.23026367044076\\
1348	-8.98043688254029\\
1349	-9.87364583515318\\
1350	-8.10633493566093\\
1351	-8.16021883590406\\
1352	-9.33400344812048\\
1353	-5.71210827300979\\
1354	-5.97501781925325\\
1355	-6.85514964310096\\
1356	-5.01522394244034\\
1357	-6.21165999665209\\
1358	-9.7733464002149\\
1359	-11.6751926789386\\
1360	-7.79733403012324\\
1361	-6.40579743334888\\
1362	-5.94715834182466\\
1363	-6.10588103322129\\
1364	-4.81726686044772\\
1365	-8.72394444950464\\
1366	-15.0782145097369\\
1367	-12.3744155573386\\
1368	-15.3238950224088\\
1369	-16.994645882533\\
1370	-17.7614464898181\\
1371	-14.8273159128435\\
1372	-11.4154203383155\\
1373	-10.8285790463417\\
1374	-11.285533758961\\
1375	-12.3186144120625\\
1376	-13.9539162674071\\
1377	-14.302781507143\\
1378	-17.762052129108\\
1379	-19.9966134321228\\
1380	-23.0240808163303\\
1381	-18.4622472728288\\
1382	-18.0707412849483\\
1383	-20.8763986071424\\
1384	-16.7472254569655\\
1385	-18.9539065273134\\
1386	-17.5830788804381\\
1387	-9.86846437279769\\
1388	-6.10174050713317\\
1389	-7.68168029572894\\
1390	-10.6908496332378\\
1391	-9.5626004430319\\
1392	-6.534307578783\\
1393	-8.6471275246945\\
1394	-7.26167330903422\\
1395	-4.79364855624658\\
1396	-6.21541334615517\\
1397	-7.05872459375155\\
1398	-5.29231408478927\\
1399	-8.40698006649541\\
1400	-10.8679527554159\\
1401	-8.42089606089689\\
1402	-13.0291227117873\\
1403	-17.9630491745859\\
1404	-17.3405056628303\\
1405	-20.2937810332573\\
};
\addlegendentry{OSA predition}

\addplot [color=mycolor3, dotted, line width=2.0pt]
  table[row sep=crcr]{%
1006	-15.8689999999999\\
1007	-14.6479999999999\\
1008	-14.6479999999999\\
1009	-7.32400000000007\\
1010	-9.28272331540325\\
1011	-10.4891043680996\\
1012	-10.5255303255769\\
1013	-9.43876563203344\\
1014	-3.94118987418415\\
1015	-3.55519974573622\\
1016	-2.91789213836205\\
1017	-6.61059074557238\\
1018	-12.0825594844218\\
1019	-12.215466052982\\
1020	-12.2707222872989\\
1021	-9.7903416616457\\
1022	-5.84488281007179\\
1023	-12.1577683722001\\
1024	-10.9860494370273\\
1025	-6.28674067020302\\
1026	-12.431875242044\\
1027	-11.1280640800046\\
1028	-8.63350026890043\\
1029	-13.3410295446527\\
1030	-12.7007526084794\\
1031	-9.44912891672107\\
1032	-13.7192041775852\\
1033	-13.3832798828848\\
1034	-10.8671573001445\\
1035	-9.79500297977233\\
1036	-8.1767144448047\\
1037	-9.03700004510802\\
1038	-10.773043930709\\
1039	-10.1418989505867\\
1041	-10.8661179421567\\
1042	-11.7848972314703\\
1043	-14.7990164257085\\
1044	-13.8935748165593\\
1045	-9.62514248953062\\
1046	-9.74614028721885\\
1047	-5.56146550679659\\
1048	-6.14122722802335\\
1049	-8.38325337786182\\
1050	-7.12648653706469\\
1051	-6.61790167031518\\
1052	-9.03609392898397\\
1053	-10.071317646999\\
1054	-9.18058530679446\\
1055	-13.5017162168517\\
1056	-11.5783756788669\\
1057	-6.10655909281036\\
1058	-5.86104895799826\\
1059	-8.00956046121405\\
1060	-8.49238571249089\\
1061	-5.1871622503827\\
1062	-5.00439141035417\\
1063	-7.33772048612332\\
1064	-6.03410760086558\\
1065	-5.00711388695413\\
1066	-6.49367514440587\\
1067	-7.07160542017141\\
1068	-9.74506265602463\\
1069	-10.4304819658971\\
1070	-10.969946849737\\
1071	-11.1776234197819\\
1072	-11.9205686736645\\
1073	-10.214409128537\\
1074	-9.77237624851773\\
1075	-9.39993601505716\\
1076	-9.21796186638608\\
1077	-9.08317322897074\\
1078	-13.1809996343713\\
1079	-19.0912210068648\\
1080	-19.4978281126409\\
1081	-18.9603140182792\\
1082	-13.3019759103524\\
1083	-18.5559025757034\\
1084	-21.9562903193464\\
1085	-21.5830785516659\\
1086	-18.3064884412897\\
1087	-22.3179518122224\\
1088	-28.1498026372378\\
1089	-23.2955064549851\\
1090	-19.4080498999481\\
1091	-14.6307773874939\\
1092	-10.5829600900779\\
1093	-10.6101864718951\\
1094	-11.9127045831135\\
1095	-9.84087955497193\\
1096	-6.13770493334914\\
1097	-6.18044476981368\\
1098	-7.75933681859283\\
1099	-10.4066406784073\\
1100	-10.2397931441287\\
1101	-9.7652808361961\\
1102	-11.8014438606465\\
1103	-12.6233389005015\\
1104	-15.8484396868291\\
1105	-14.3461331695141\\
1106	-15.1331131813338\\
1107	-13.3123631908734\\
1108	-10.8657077794123\\
1109	-13.0394937977037\\
1110	-10.422029323533\\
1111	-9.11903712687467\\
1112	-11.129354836653\\
1113	-11.1082317411383\\
1114	-8.15290398269531\\
1115	-9.07113334639689\\
1116	-7.25274815178682\\
1117	-7.4649066201182\\
1118	-8.35239597607938\\
1119	-6.1773785255366\\
1120	-6.85327134192244\\
1121	-9.12491298005011\\
1122	-11.468285607715\\
1123	-12.5600923040759\\
1124	-13.2170989761607\\
1125	-8.44203880887017\\
1126	-11.2209123452878\\
1127	-16.828221621558\\
1128	-13.41019741379\\
1129	-14.7392874445645\\
1130	-15.3186090393806\\
1131	-9.0061837917624\\
1132	-7.33037929323837\\
1133	-10.3313608908393\\
1134	-10.7677123550236\\
1135	-15.9212668826669\\
1136	-18.8928296785414\\
1137	-18.060768313647\\
1138	-14.5502193081745\\
1139	-14.5844628781781\\
1140	-14.766901701496\\
1141	-13.0470779538387\\
1142	-13.807738009416\\
1143	-10.1166771822816\\
1144	-9.13605413318896\\
1145	-8.79065584261912\\
1146	-7.84707168351429\\
1147	-8.98092325065181\\
1148	-12.1187981014814\\
1149	-16.4987347620836\\
1150	-13.9742151827043\\
1151	-9.64708613790481\\
1152	-9.08900114964536\\
1153	-9.05672026117986\\
1154	-5.56847526041133\\
1155	-4.11834745785109\\
1156	-4.83573682769088\\
1157	-8.04371425431896\\
1158	-7.06454837418528\\
1159	-6.19880769072779\\
1160	-7.6112010805673\\
1161	-6.61383750132723\\
1162	-5.19514776789515\\
1163	-3.87385280334979\\
1164	-3.20769502429948\\
1165	-4.3410445437944\\
1166	-9.37286993734801\\
1167	-13.0521607260644\\
1168	-14.7967734876529\\
1169	-9.79249312422348\\
1170	-7.81366614084732\\
1171	-5.72712270767488\\
1172	-3.80273685815223\\
1173	-7.44972632803933\\
1174	-8.97854857231982\\
1175	-10.0226811763025\\
1176	-12.9512062310782\\
1177	-19.3597703602381\\
1178	-22.7443956539034\\
1179	-19.879437198523\\
1180	-19.548057912338\\
1181	-14.135312814564\\
1182	-15.0726188147066\\
1183	-16.6069037938562\\
1184	-16.3418921938551\\
1185	-14.2714286597334\\
1186	-12.7176439917373\\
1187	-13.2401005741942\\
1188	-12.9015483999628\\
1189	-13.649198452852\\
1190	-10.3702551521324\\
1191	-15.4822559124898\\
1192	-18.9758840293548\\
1193	-13.9488817174799\\
1194	-9.89735767790421\\
1195	-9.94637011437612\\
1196	-15.9886315951437\\
1197	-18.601618233859\\
1198	-21.7785794895885\\
1199	-22.1784200892571\\
1200	-21.8067364834412\\
1201	-18.4679234319965\\
1202	-16.4883405073015\\
1203	-19.428490546441\\
1204	-21.3344135502728\\
1205	-14.5481725547684\\
1206	-10.1127692321188\\
1207	-13.8734608693676\\
1208	-13.0056165364447\\
1209	-7.39824299213456\\
1210	-10.8160359536839\\
1211	-12.3345247056814\\
1212	-9.59645945964212\\
1213	-7.9209704406685\\
1214	-10.198746366186\\
1215	-9.05323990860688\\
1216	-10.6458268818642\\
1217	-13.9862642815331\\
1218	-12.8695306531959\\
1219	-9.626167845385\\
1220	-10.1248447562514\\
1221	-13.7373399388216\\
1222	-19.5939356567671\\
1223	-16.0960675636386\\
1224	-10.3678351622482\\
1225	-8.52958345034313\\
1226	-10.4727875742965\\
1227	-9.76793357670817\\
1228	-6.98427847199059\\
1229	-8.05153293069429\\
1230	-9.74907091685509\\
1231	-11.3543862753484\\
1232	-11.735015793776\\
1233	-8.44826036209565\\
1234	-13.9972372033089\\
1235	-15.8776097544051\\
1236	-13.7443620497211\\
1237	-12.1931231402643\\
1238	-12.8084253065467\\
1239	-15.4239196315038\\
1240	-11.0880092138859\\
1241	-5.08600279211237\\
1242	-7.80768385388797\\
1243	-8.64343391897205\\
1244	-10.4070256921225\\
1245	-9.80905381718048\\
1246	-7.38839430039002\\
1247	-10.4772381622104\\
1248	-10.9863610264604\\
1249	-7.49534545818005\\
1250	-9.44647762313525\\
1251	-9.03570769639123\\
1252	-7.64451092707827\\
1253	-9.03311494252375\\
1254	-6.24410786621797\\
1255	-6.65134520937022\\
1256	-5.88527391237017\\
1257	-5.40092361675261\\
1258	-9.49470678537728\\
1259	-10.7828136334454\\
1260	-13.0810024122648\\
1261	-16.7011535239324\\
1262	-12.0718514983753\\
1263	-7.50823180379712\\
1264	-6.26876557440278\\
1265	-6.45692489887574\\
1266	-8.42481142299516\\
1267	-5.96804929340033\\
1268	-5.60056085170481\\
1269	-8.1238108518387\\
1270	-11.7480195162279\\
1271	-12.0302144326424\\
1272	-14.0394711365632\\
1273	-10.7625423450404\\
1274	-9.35771643212365\\
1275	-4.49317437612808\\
1276	-5.16724242382952\\
1277	-4.02925009788987\\
1278	-3.53785080319562\\
1279	-4.96125171764766\\
1280	-7.43533044145056\\
1281	-8.92500176869589\\
1282	-8.69301421534919\\
1283	-9.65236689906396\\
1284	-14.0294689222092\\
1285	-13.828680603455\\
1286	-10.3527662759539\\
1287	-11.9287929346094\\
1288	-12.9515282992706\\
1289	-14.8614935453406\\
1290	-14.3842871935851\\
1291	-8.7966143205465\\
1292	-5.21191465944185\\
1293	-3.80708281477905\\
1294	-3.90424581543903\\
1295	-4.93737385196982\\
1296	-5.21005594106373\\
1297	-4.67499403716124\\
1298	-5.79149701788606\\
1299	-8.99038010095387\\
1300	-8.38198813446684\\
1301	-8.0850349421969\\
1302	-9.14695198673735\\
1303	-6.11053535607516\\
1304	-3.42096523828036\\
1305	-6.59517801242464\\
1306	-10.311497108934\\
1307	-9.77923153251936\\
1308	-10.1789790099135\\
1309	-9.07747420994792\\
1310	-7.07421646877242\\
1311	-9.28265540255006\\
1312	-12.9814642818703\\
1313	-12.5546386535252\\
1314	-9.44436281857861\\
1315	-14.7559797156709\\
1316	-13.1898436707438\\
1317	-10.905581076855\\
1318	-13.7064091164307\\
1319	-12.7999397161332\\
1320	-10.8102744799817\\
1321	-9.68040615831774\\
1322	-15.6979734253234\\
1323	-19.7193408031135\\
1324	-16.0562137322659\\
1325	-10.1581844770883\\
1326	-10.1275888304031\\
1327	-10.9158145838189\\
1328	-12.3728035186759\\
1329	-13.7417177555519\\
1330	-11.2335441935211\\
1331	-11.0923816121963\\
1332	-14.6290296937614\\
1333	-14.2080164354122\\
1334	-11.8960888793001\\
1335	-8.5542079818556\\
1336	-11.936186310686\\
1337	-17.0486369330956\\
1338	-16.2403609901635\\
1339	-15.2262215769472\\
1340	-10.6142189050177\\
1341	-8.96702897461068\\
1342	-10.4656826289038\\
1343	-5.94603420837166\\
1344	-3.96727578154355\\
1345	-3.69177363766084\\
1346	-3.08200335487072\\
1347	-5.12591663236526\\
1348	-9.07512563228693\\
1349	-9.86756469634975\\
1350	-7.90636186735878\\
1351	-7.9347901247927\\
1352	-9.44650282447151\\
1353	-5.96156415946552\\
1354	-6.01414363338699\\
1355	-6.79806578972421\\
1356	-5.12419700637543\\
1357	-6.09028914162718\\
1358	-9.55262165186787\\
1359	-11.7401548127198\\
1360	-7.55546689364564\\
1361	-6.13999437489178\\
1362	-5.79156731665989\\
1363	-6.0584998182344\\
1364	-5.01975598334502\\
1365	-8.66964210075662\\
1366	-15.0928612669472\\
1367	-12.473464491605\\
1368	-15.2582227286614\\
1369	-17.13919619238\\
1370	-17.6079771152345\\
1371	-14.3311953131015\\
1372	-10.8971706806906\\
1373	-10.821825065399\\
1374	-11.4450104059231\\
1375	-12.3008802727172\\
1376	-13.9071878673656\\
1377	-14.1677095650093\\
1378	-17.6536321953356\\
1379	-19.720292672396\\
1380	-22.7681382955175\\
1381	-17.8679222095238\\
1382	-17.9120204495698\\
1383	-20.6024870720653\\
1384	-16.8557255918388\\
1385	-18.5659549006589\\
1386	-17.7924596438093\\
1387	-10.2478862889952\\
1388	-6.13572040192412\\
1389	-7.65890887386172\\
1390	-10.7694300097482\\
1391	-9.3393221372371\\
1392	-6.42024598486455\\
1393	-8.91114709139197\\
1394	-7.45579774655789\\
1395	-4.8266966066542\\
1396	-6.45447962569233\\
1397	-7.1939019253748\\
1398	-5.48614671905148\\
1399	-8.33808382247707\\
1400	-11.105566563012\\
1401	-8.40305435493383\\
1402	-13.0501704571464\\
1403	-17.9669562861075\\
1404	-17.0616465905389\\
1405	-20.1264173348761\\
};
\addlegendentry{MPO prediction}

\end{axis}

\begin{axis}[%
width=6.159cm,
height=1.831cm,
at={(8.104cm,5.085cm)},
scale only axis,
xmin=1000,
xmax=1405,
xlabel style={font=\color{white!15!black}},
xlabel={Sample index},
ymin=-404.053,
ymax=0,
ylabel style={font=\color{white!15!black}},
ylabel={$y(t)$},
axis background/.style={fill=white},
title style={font=\bfseries},
title={C6: RMSE(OSA) = 5.8768, RMSE(MPO) = 7.5894},
legend style={legend cell align=left, align=left, draw=white!15!black}
]
\addplot [color=mycolor1, line width=2.0pt]
  table[row sep=crcr]{%
1006	-185.547\\
1007	-219.727\\
1008	-169.678\\
1009	-86.6700000000001\\
1010	-107.422\\
1011	-102.539\\
1012	-126.953\\
1013	-103.76\\
1014	-41.5039999999999\\
1015	-31.7380000000001\\
1016	-28.076\\
1017	-89.1110000000001\\
1018	-157.471\\
1019	-167.236\\
1020	-174.561\\
1022	-74.463\\
1023	-158.691\\
1024	-111.084\\
1025	-83.008\\
1026	-142.822\\
1028	-103.76\\
1029	-173.34\\
1030	-164.795\\
1031	-124.512\\
1032	-180.664\\
1033	-179.443\\
1034	-142.822\\
1036	-89.1110000000001\\
1037	-108.643\\
1038	-137.939\\
1039	-130.615\\
1040	-136.719\\
1041	-140.381\\
1042	-151.367\\
1043	-213.623\\
1044	-191.65\\
1045	-126.953\\
1046	-115.967\\
1047	-64.6970000000001\\
1048	-69.5799999999999\\
1049	-96.4359999999999\\
1050	-74.463\\
1051	-78.125\\
1052	-111.084\\
1053	-128.174\\
1054	-119.629\\
1055	-190.43\\
1056	-139.16\\
1057	-81.787\\
1058	-63.4770000000001\\
1059	-85.4490000000001\\
1060	-101.318\\
1061	-51.27\\
1062	-58.5940000000001\\
1063	-81.787\\
1064	-65.9180000000001\\
1065	-56.152\\
1066	-74.463\\
1067	-86.6700000000001\\
1068	-139.16\\
1069	-130.615\\
1070	-163.574\\
1071	-146.484\\
1072	-170.898\\
1073	-139.16\\
1074	-133.057\\
1075	-115.967\\
1076	-111.084\\
1077	-111.084\\
1079	-280.762\\
1080	-291.748\\
1081	-280.762\\
1082	-175.781\\
1084	-303.955\\
1085	-317.383\\
1086	-239.258\\
1087	-313.721\\
1088	-404.053\\
1089	-289.307\\
1090	-238.037\\
1091	-158.691\\
1092	-122.07\\
1093	-102.539\\
1094	-129.395\\
1095	-92.7729999999999\\
1096	-61.0350000000001\\
1097	-59.8140000000001\\
1098	-75.684\\
1099	-118.408\\
1100	-107.422\\
1101	-111.084\\
1102	-146.484\\
1103	-152.588\\
1104	-207.52\\
1105	-167.236\\
1106	-208.74\\
1107	-148.926\\
1108	-130.615\\
1109	-151.367\\
1110	-108.643\\
1111	-98.877\\
1112	-122.07\\
1113	-123.291\\
1114	-85.4490000000001\\
1115	-95.2149999999999\\
1116	-67.1389999999999\\
1117	-76.904\\
1118	-81.787\\
1119	-57.373\\
1121	-93.9939999999999\\
1122	-151.367\\
1123	-152.588\\
1124	-170.898\\
1125	-97.6559999999999\\
1126	-140.381\\
1127	-211.182\\
1128	-161.133\\
1129	-186.768\\
1130	-191.65\\
1131	-112.305\\
1132	-75.684\\
1133	-107.422\\
1134	-123.291\\
1135	-205.078\\
1136	-252.686\\
1137	-252.686\\
1138	-184.326\\
1139	-189.209\\
1140	-167.236\\
1141	-163.574\\
1142	-162.354\\
1143	-108.643\\
1144	-96.4359999999999\\
1145	-86.6700000000001\\
1146	-78.125\\
1147	-87.8910000000001\\
1148	-133.057\\
1149	-203.857\\
1150	-161.133\\
1151	-109.863\\
1152	-92.7729999999999\\
1153	-89.1110000000001\\
1154	-52.49\\
1155	-41.5039999999999\\
1156	-47.607\\
1157	-90.3320000000001\\
1158	-67.1389999999999\\
1159	-78.125\\
1160	-85.4490000000001\\
1161	-80.566\\
1162	-54.932\\
1163	-37.8420000000001\\
1164	-30.518\\
1165	-54.932\\
1166	-119.629\\
1167	-172.119\\
1168	-194.092\\
1169	-136.719\\
1170	-91.5530000000001\\
1171	-64.6970000000001\\
1172	-43.9449999999999\\
1173	-98.877\\
1174	-91.5530000000001\\
1175	-137.939\\
1176	-168.457\\
1177	-275.879\\
1178	-316.162\\
1179	-260.01\\
1180	-249.023\\
1181	-159.912\\
1182	-185.547\\
1183	-195.313\\
1184	-212.402\\
1185	-158.691\\
1186	-144.043\\
1187	-142.822\\
1188	-129.395\\
1189	-156.25\\
1190	-109.863\\
1191	-195.313\\
1192	-234.375\\
1193	-175.781\\
1194	-104.98\\
1195	-111.084\\
1196	-192.871\\
1197	-234.375\\
1198	-289.307\\
1199	-303.955\\
1200	-302.734\\
1201	-228.271\\
1202	-205.078\\
1203	-235.596\\
1204	-275.879\\
1205	-170.898\\
1206	-104.98\\
1207	-152.588\\
1208	-122.07\\
1209	-79.346\\
1210	-109.863\\
1211	-122.07\\
1212	-95.2149999999999\\
1213	-75.684\\
1214	-101.318\\
1215	-80.566\\
1216	-115.967\\
1217	-162.354\\
1218	-147.705\\
1219	-100.098\\
1220	-101.318\\
1221	-157.471\\
1222	-261.23\\
1224	-118.408\\
1225	-85.4490000000001\\
1226	-101.318\\
1227	-90.3320000000001\\
1228	-67.1389999999999\\
1229	-76.904\\
1230	-92.7729999999999\\
1231	-120.85\\
1232	-134.277\\
1233	-89.1110000000001\\
1234	-156.25\\
1235	-179.443\\
1236	-170.898\\
1237	-140.381\\
1238	-150.146\\
1239	-202.637\\
1241	-53.711\\
1242	-78.125\\
1243	-85.4490000000001\\
1244	-119.629\\
1245	-101.318\\
1246	-74.463\\
1247	-118.408\\
1248	-112.305\\
1249	-79.346\\
1250	-102.539\\
1251	-90.3320000000001\\
1252	-80.566\\
1253	-95.2149999999999\\
1254	-54.932\\
1255	-68.3589999999999\\
1256	-47.607\\
1257	-51.27\\
1258	-106.201\\
1259	-122.07\\
1260	-167.236\\
1261	-219.727\\
1262	-142.822\\
1263	-83.008\\
1264	-63.4770000000001\\
1265	-62.2560000000001\\
1266	-91.5530000000001\\
1267	-54.932\\
1269	-85.4490000000001\\
1270	-150.146\\
1271	-133.057\\
1272	-190.43\\
1273	-124.512\\
1274	-117.188\\
1275	-56.152\\
1276	-62.2560000000001\\
1277	-34.1800000000001\\
1278	-46.3869999999999\\
1279	-51.27\\
1280	-95.2149999999999\\
1281	-100.098\\
1282	-117.188\\
1283	-123.291\\
1284	-197.754\\
1285	-170.898\\
1286	-128.174\\
1287	-153.809\\
1288	-163.574\\
1289	-213.623\\
1292	-53.711\\
1293	-40.2829999999999\\
1294	-41.5039999999999\\
1295	-52.49\\
1296	-54.932\\
1297	-51.27\\
1298	-70.8009999999999\\
1299	-108.643\\
1300	-98.877\\
1301	-103.76\\
1302	-118.408\\
1303	-69.5799999999999\\
1304	-36.6210000000001\\
1306	-125.732\\
1307	-125.732\\
1308	-142.822\\
1309	-118.408\\
1310	-87.8910000000001\\
1311	-123.291\\
1312	-172.119\\
1313	-172.119\\
1314	-120.85\\
1315	-207.52\\
1316	-155.029\\
1317	-151.367\\
1318	-173.34\\
1319	-172.119\\
1320	-130.615\\
1321	-112.305\\
1323	-266.113\\
1324	-202.637\\
1325	-124.512\\
1326	-118.408\\
1327	-115.967\\
1328	-150.146\\
1329	-173.34\\
1330	-125.732\\
1331	-134.277\\
1332	-177.002\\
1333	-192.871\\
1334	-120.85\\
1335	-97.6559999999999\\
1336	-130.615\\
1337	-218.506\\
1338	-195.313\\
1339	-203.857\\
1340	-115.967\\
1342	-101.318\\
1343	-63.4770000000001\\
1344	-41.5039999999999\\
1345	-34.1800000000001\\
1346	-29.297\\
1347	-78.125\\
1348	-108.643\\
1349	-131.836\\
1350	-93.9939999999999\\
1352	-119.629\\
1353	-72.021\\
1354	-79.346\\
1355	-73.242\\
1356	-54.932\\
1357	-75.684\\
1358	-120.85\\
1359	-156.25\\
1360	-95.2149999999999\\
1361	-73.242\\
1362	-58.5940000000001\\
1363	-74.463\\
1364	-45.1659999999999\\
1366	-195.313\\
1367	-163.574\\
1368	-216.064\\
1369	-241.699\\
1370	-249.023\\
1371	-181.885\\
1372	-131.836\\
1374	-129.395\\
1375	-155.029\\
1376	-184.326\\
1377	-181.885\\
1378	-247.803\\
1379	-270.996\\
1380	-328.369\\
1381	-217.285\\
1382	-238.037\\
1383	-264.893\\
1384	-205.078\\
1385	-239.258\\
1386	-200.195\\
1387	-113.525\\
1388	-74.463\\
1389	-79.346\\
1390	-117.188\\
1391	-97.6559999999999\\
1392	-64.6970000000001\\
1393	-90.3320000000001\\
1394	-63.4770000000001\\
1395	-48.828\\
1396	-64.6970000000001\\
1397	-72.021\\
1398	-48.828\\
1399	-100.098\\
1400	-135.498\\
1401	-97.6559999999999\\
1403	-241.699\\
1404	-220.947\\
1405	-279.541\\
};
\addlegendentry{True output}

\addplot [color=mycolor2, dashed, line width=2.0pt]
  table[row sep=crcr]{%
1006	-172.010378631054\\
1007	-211.918675751263\\
1008	-172.693139661322\\
1009	-81.6122750116901\\
1010	-107.028756733575\\
1011	-109.839332893192\\
1012	-130.587043111721\\
1013	-102.443671178399\\
1014	-33.9792912834178\\
1015	-36.9054311491623\\
1016	-29.7570413425472\\
1018	-149.56199066639\\
1019	-164.504927196799\\
1020	-162.170388873548\\
1021	-125.718149822819\\
1022	-75.7444945801651\\
1023	-163.270660450885\\
1024	-116.164816992277\\
1025	-79.1423274264052\\
1026	-143.851868783818\\
1027	-121.283419821125\\
1028	-106.213157021848\\
1029	-169.287129032768\\
1030	-161.662189447781\\
1031	-118.569696772933\\
1032	-177.649833953728\\
1033	-174.159890507108\\
1034	-140.393893458525\\
1036	-94.8897078948096\\
1037	-104.342522203841\\
1038	-130.304929452446\\
1039	-124.589086906544\\
1040	-134.265801266709\\
1041	-142.613247917592\\
1042	-149.888362021168\\
1043	-198.460758504744\\
1044	-183.533826223289\\
1045	-127.99456176254\\
1046	-122.499169432911\\
1047	-65.8237074759011\\
1048	-69.3480179338617\\
1049	-93.096347025833\\
1050	-76.4915595180537\\
1051	-75.5685957712178\\
1052	-105.796000313186\\
1053	-128.834817587964\\
1054	-118.845657820564\\
1055	-181.204439213791\\
1056	-145.769413847177\\
1057	-83.7132903959027\\
1058	-68.2974685153993\\
1060	-100.265033540657\\
1061	-53.0621990895941\\
1062	-51.4795148063581\\
1063	-83.3903843328039\\
1065	-54.0474262881153\\
1066	-76.0866729297575\\
1067	-86.0893746861386\\
1068	-129.876500852092\\
1069	-138.016489551653\\
1070	-153.900727759592\\
1071	-139.720623356452\\
1072	-170.196244906942\\
1073	-134.098062706719\\
1074	-139.14422829565\\
1075	-115.264683972467\\
1076	-111.537654197326\\
1077	-112.512177705091\\
1079	-273.321316596471\\
1080	-281.222626089532\\
1081	-262.913328317447\\
1082	-172.461933192623\\
1083	-243.851281934752\\
1084	-299.386959583766\\
1085	-294.008568452356\\
1086	-233.844659280978\\
1087	-297.220767051334\\
1088	-392.557078217418\\
1089	-299.982401995104\\
1090	-250.466353819492\\
1091	-148.223380972505\\
1092	-119.779477563513\\
1093	-103.147013769748\\
1094	-119.783543826208\\
1095	-94.4888557995434\\
1096	-54.6655126050284\\
1097	-45.1640040706407\\
1098	-72.4084193883175\\
1099	-119.129800469032\\
1100	-115.430770214369\\
1101	-114.36962047701\\
1102	-146.958885949437\\
1103	-152.000245135991\\
1104	-211.163735523403\\
1105	-179.830808105325\\
1106	-198.140728437546\\
1107	-151.612439570118\\
1108	-133.074509664312\\
1109	-156.622302162165\\
1110	-111.239656105171\\
1111	-105.009641448059\\
1112	-119.066678776077\\
1113	-129.992506770892\\
1114	-87.7850590115884\\
1115	-98.0644136686747\\
1116	-67.9091740018837\\
1117	-80.2668883825384\\
1118	-83.8454464532915\\
1119	-59.9417783030813\\
1120	-67.7123413289169\\
1121	-93.7825293425835\\
1122	-153.234876046248\\
1123	-157.861023472323\\
1124	-169.759627789853\\
1125	-94.4394219069172\\
1126	-135.863816962161\\
1127	-215.234699913015\\
1128	-166.851812754179\\
1129	-190.162908167856\\
1130	-185.882611975015\\
1131	-112.387576397249\\
1132	-83.2939708112485\\
1134	-123.788025028863\\
1135	-203.641055845625\\
1136	-251.362783792388\\
1137	-248.877123536768\\
1138	-184.968151191233\\
1139	-189.06695055443\\
1140	-176.740103281937\\
1141	-173.240959225824\\
1142	-160.174294262926\\
1143	-104.865667107103\\
1144	-91.8570875021273\\
1145	-88.3526186256474\\
1146	-83.6785625639386\\
1147	-89.2216546392303\\
1148	-129.922160364613\\
1149	-208.979695151951\\
1150	-167.413541719031\\
1151	-107.583645241869\\
1153	-91.5904163447765\\
1154	-49.2025350250012\\
1155	-35.7194726250057\\
1156	-41.0456011529675\\
1157	-91.2408897702458\\
1158	-75.8468425263545\\
1159	-79.5467480430782\\
1160	-79.4853692063525\\
1161	-78.8914257953447\\
1162	-61.491016326906\\
1163	-39.3467185165136\\
1164	-28.6256253265665\\
1165	-50.7456546837589\\
1166	-119.48985516071\\
1167	-168.783838036087\\
1168	-195.773265863671\\
1169	-139.564597129007\\
1170	-90.4367127066805\\
1171	-69.6436942820812\\
1172	-43.62506863607\\
1173	-92.7913881581139\\
1174	-95.0110971610843\\
1175	-130.632065719438\\
1176	-178.84309200932\\
1177	-281.873297124665\\
1178	-342.085934907652\\
1179	-267.242570102423\\
1180	-248.918298099846\\
1181	-158.332607250558\\
1182	-173.573999825525\\
1183	-202.126015919515\\
1184	-205.177345042683\\
1185	-159.668810867505\\
1186	-144.039988202243\\
1187	-151.201783694902\\
1188	-131.599768139614\\
1189	-153.082832181148\\
1190	-115.50656480661\\
1191	-197.025107491673\\
1192	-225.817602875582\\
1193	-182.403580560032\\
1194	-102.855888676345\\
1195	-109.739175578894\\
1196	-194.885573958672\\
1198	-274.200096422486\\
1199	-296.718053411409\\
1200	-299.392423226058\\
1201	-225.750295311026\\
1202	-215.325673885541\\
1203	-233.087599846553\\
1204	-269.331834052804\\
1205	-172.043694980231\\
1206	-99.9370807162647\\
1207	-149.208081145306\\
1208	-124.290761692881\\
1209	-82.782629539766\\
1210	-99.9726997004284\\
1211	-127.169285333673\\
1213	-78.0667619346\\
1214	-100.076061493403\\
1215	-88.1831985983063\\
1216	-118.303154699553\\
1217	-175.762841554248\\
1218	-143.685294691163\\
1219	-105.970743931327\\
1220	-108.90819680847\\
1221	-158.335536945043\\
1222	-263.433923184239\\
1223	-197.192763494919\\
1224	-115.89050122403\\
1225	-85.1897926017293\\
1226	-100.246545223459\\
1227	-102.862144729026\\
1228	-69.2616608023952\\
1229	-72.6122569884262\\
1230	-98.7387698537748\\
1231	-121.81418085399\\
1232	-136.783193417222\\
1233	-89.9813154243311\\
1234	-162.22059292837\\
1235	-185.523858088795\\
1236	-164.743244681433\\
1237	-151.595722172219\\
1238	-147.492857424924\\
1239	-204.717802917181\\
1241	-51.6895887184155\\
1242	-74.3455737424729\\
1243	-88.4639399545993\\
1244	-125.10618008064\\
1245	-108.354267372771\\
1246	-80.3040267652418\\
1247	-120.397573688828\\
1248	-116.780280388955\\
1249	-86.1078036463741\\
1250	-99.7179213708625\\
1251	-91.687388584214\\
1252	-89.0554762613067\\
1253	-99.0848077435978\\
1254	-54.0374197206784\\
1255	-70.4400692639917\\
1256	-48.8189109443024\\
1257	-55.101554594416\\
1258	-111.978273052232\\
1259	-117.401486024748\\
1260	-173.005070833644\\
1261	-221.448676271775\\
1263	-85.8267572045936\\
1264	-61.506897545325\\
1265	-68.8409560843779\\
1266	-93.0630405610052\\
1267	-54.7102816461893\\
1268	-64.4638049734949\\
1269	-86.6605679873637\\
1270	-154.175072070732\\
1271	-140.193728981227\\
1272	-191.364168437148\\
1273	-120.071867336293\\
1274	-123.074747602554\\
1275	-53.4314959199783\\
1276	-60.3651167450437\\
1277	-36.3211937878655\\
1278	-43.7240999869207\\
1279	-47.3806659826453\\
1280	-95.712028402\\
1281	-101.643985528772\\
1282	-115.77472328983\\
1283	-126.749510365143\\
1284	-197.675097655567\\
1285	-193.290216761539\\
1286	-131.979098168984\\
1287	-147.521271152958\\
1288	-161.007418798995\\
1289	-206.508242268968\\
1291	-112.342036836474\\
1292	-52.206406512053\\
1293	-43.9530233364783\\
1294	-39.2525694040178\\
1295	-51.3661459842533\\
1296	-57.4100670811156\\
1297	-49.2559931891587\\
1298	-67.9515643762334\\
1299	-112.210285273333\\
1300	-105.225376328502\\
1301	-95.9437407561702\\
1302	-116.726134894121\\
1303	-72.683705501189\\
1304	-40.4263530337475\\
1306	-126.353187045652\\
1307	-116.551416708723\\
1308	-133.949480056039\\
1309	-126.326263715904\\
1310	-89.8760868879085\\
1311	-119.102832266091\\
1312	-170.970092014129\\
1313	-166.654384106255\\
1314	-123.24113863983\\
1315	-203.361515633951\\
1316	-158.622494651585\\
1317	-152.390704382425\\
1318	-173.446588960485\\
1319	-163.621339991637\\
1320	-132.456588966823\\
1321	-119.977074393513\\
1322	-180.806041839734\\
1323	-261.328126382444\\
1324	-207.793981996785\\
1325	-120.719047161332\\
1327	-119.904682295797\\
1328	-148.232718923076\\
1329	-164.658501015255\\
1330	-133.348122169703\\
1331	-138.119246240745\\
1332	-173.940734382032\\
1333	-181.857833733938\\
1334	-123.740458077907\\
1335	-105.611909827082\\
1336	-134.604700240745\\
1337	-206.042488020533\\
1338	-199.585895882446\\
1339	-198.874069978816\\
1340	-111.497866110996\\
1341	-110.236413865282\\
1342	-112.86201291432\\
1343	-61.4297838144601\\
1344	-33.6201218680321\\
1345	-30.3204543016211\\
1346	-28.1817975764441\\
1347	-74.5706591798992\\
1348	-107.535228279811\\
1349	-128.217291438948\\
1350	-99.3993407375488\\
1351	-106.543756652836\\
1352	-115.841520293287\\
1353	-75.065618703321\\
1354	-77.110852367397\\
1355	-76.874051624028\\
1356	-56.8905067015123\\
1357	-75.8326140611437\\
1359	-152.359772473188\\
1360	-95.4883627701033\\
1361	-80.7805618667157\\
1362	-60.2692027170924\\
1363	-73.9992690561317\\
1364	-49.2462627345969\\
1366	-189.496982430347\\
1367	-163.232589208815\\
1368	-210.49221620448\\
1369	-245.720953676278\\
1370	-246.368543167541\\
1371	-187.567819342392\\
1372	-138.141058868737\\
1373	-126.580996660381\\
1374	-133.655672907787\\
1376	-173.074961717554\\
1377	-174.677882343506\\
1378	-238.20426309487\\
1379	-269.115615846569\\
1380	-324.93790972504\\
1381	-212.689893706159\\
1382	-243.464885499779\\
1383	-269.608691292822\\
1384	-199.165043050324\\
1385	-239.656902256143\\
1386	-194.393844136425\\
1387	-112.662857611324\\
1388	-66.4012761489539\\
1389	-70.6710392205491\\
1390	-123.825483990049\\
1391	-105.869605576148\\
1392	-64.6105979731101\\
1393	-85.8613081178476\\
1394	-67.6252548251589\\
1395	-45.5502039227611\\
1396	-65.6125500557976\\
1397	-69.2296217322751\\
1398	-56.4579524802289\\
1400	-142.136036742025\\
1401	-100.792686457733\\
1402	-171.099398547213\\
1403	-257.620918185745\\
1404	-233.226263319512\\
1405	-286.035208922001\\
};
\addlegendentry{OSA predition}

\addplot [color=mycolor3, dotted, line width=2.0pt]
  table[row sep=crcr]{%
1006	-185.547\\
1007	-219.727\\
1008	-169.678\\
1009	-86.6700000000001\\
1010	-107.02875673358\\
1011	-109.742353078777\\
1012	-132.228258882878\\
1013	-106.199948836171\\
1014	-36.0545924083908\\
1015	-36.677625508552\\
1016	-29.4583817312864\\
1018	-151.589373245242\\
1019	-164.550348159093\\
1020	-159.925376608622\\
1021	-121.633385648453\\
1022	-69.6243599834315\\
1023	-159.449108828175\\
1024	-113.828579960995\\
1025	-80.0282099586857\\
1026	-143.885054355085\\
1027	-120.646685826593\\
1028	-106.019623204646\\
1029	-168.885921600512\\
1030	-161.351628813077\\
1031	-116.132522488355\\
1032	-174.137851204946\\
1033	-169.313597131689\\
1034	-135.627248317323\\
1036	-90.776516585111\\
1037	-102.440299245025\\
1038	-128.705440168341\\
1039	-120.126983961716\\
1042	-144.921531819087\\
1043	-194.133097922997\\
1044	-176.710037565898\\
1045	-117.781099767332\\
1046	-114.099064429139\\
1047	-61.1526253436978\\
1048	-66.8659957628715\\
1049	-90.5976928346638\\
1050	-73.9378913019675\\
1051	-73.2461199888071\\
1052	-103.82501663298\\
1053	-125.084529906819\\
1054	-115.457828725234\\
1055	-178.079143243077\\
1056	-141.233449147452\\
1057	-79.9329522035518\\
1058	-68.2653379854025\\
1060	-101.784241471462\\
1061	-52.8680847999894\\
1062	-52.283632424331\\
1063	-82.3028815714511\\
1064	-66.5886391989322\\
1065	-54.4072699942546\\
1066	-75.5712389230805\\
1067	-85.868143157667\\
1068	-129.871352591398\\
1069	-135.562008141486\\
1070	-152.0042483338\\
1071	-138.340610772503\\
1072	-164.181664798409\\
1073	-129.704554121198\\
1074	-133.683487134068\\
1075	-111.484121723139\\
1076	-110.039760367618\\
1077	-110.24435565383\\
1079	-271.513890067712\\
1080	-277.8672213762\\
1081	-257.318676600674\\
1082	-163.513137611743\\
1083	-230.768954695468\\
1084	-289.635147484739\\
1085	-286.419122788572\\
1086	-223.842307931251\\
1087	-280.846838149509\\
1088	-377.277588720642\\
1089	-284.477020712003\\
1090	-238.433628925385\\
1091	-144.92644788725\\
1092	-115.442153602362\\
1093	-95.9255542313545\\
1094	-115.064111044564\\
1095	-87.8086818296613\\
1096	-48.0250641169127\\
1097	-39.5372426752642\\
1098	-61.5930564121884\\
1099	-107.173605819466\\
1100	-106.185433377793\\
1101	-108.68211246467\\
1102	-144.332990251969\\
1103	-150.033093578503\\
1104	-209.496665867009\\
1105	-179.094780070125\\
1106	-201.166800011964\\
1107	-154.362715392183\\
1108	-131.784337399363\\
1109	-159.044391093993\\
1110	-113.102088752473\\
1111	-109.144235785035\\
1112	-123.671919003575\\
1113	-134.641865341041\\
1114	-91.2984592353489\\
1115	-104.105109376371\\
1116	-72.4144907542202\\
1117	-85.4672368968857\\
1118	-88.5260684141645\\
1119	-65.0338082857338\\
1120	-72.5159374567861\\
1121	-96.4257642096447\\
1122	-154.063364764192\\
1123	-159.390544423795\\
1124	-172.229149684278\\
1125	-97.1872082722641\\
1126	-136.704816226961\\
1127	-214.706455064641\\
1128	-166.308131173746\\
1129	-192.296771414817\\
1130	-188.956469081097\\
1131	-113.912328366762\\
1132	-82.9222732072255\\
1133	-106.684907267462\\
1134	-126.146145089833\\
1135	-205.166229233614\\
1136	-252.618968644483\\
1137	-249.090758625005\\
1138	-184.482266198198\\
1139	-187.749815009811\\
1140	-176.341722140381\\
1141	-174.519842991211\\
1142	-165.910702282484\\
1143	-109.605855540261\\
1144	-93.970546444235\\
1146	-83.5169798053435\\
1147	-91.2617327231617\\
1148	-132.691851976394\\
1149	-210.963960940298\\
1150	-169.104561723793\\
1151	-112.028404061018\\
1152	-103.079692104972\\
1153	-95.5235694956341\\
1154	-54.8440718936133\\
1155	-38.6508019663345\\
1156	-41.7756878503437\\
1157	-89.2808201054315\\
1158	-73.7062111206135\\
1159	-80.5238533351192\\
1160	-81.935938309638\\
1161	-78.9593353448874\\
1163	-40.3271740799087\\
1164	-30.8661070512267\\
1165	-51.8257875662014\\
1166	-119.443406408962\\
1167	-167.980236914523\\
1168	-194.41473491222\\
1169	-138.098050601255\\
1170	-90.9407737666345\\
1171	-69.7837616569605\\
1172	-44.7346454186911\\
1173	-94.7654538172419\\
1174	-94.5247557055661\\
1175	-130.386801044206\\
1176	-177.557161715826\\
1177	-281.747538185038\\
1178	-344.715220919636\\
1179	-272.065183347905\\
1180	-262.147953103643\\
1181	-165.335754919433\\
1182	-181.658098463664\\
1183	-204.854793717264\\
1184	-206.610009181511\\
1185	-162.2679319317\\
1186	-142.664606094905\\
1187	-152.43709030476\\
1188	-133.126247682629\\
1189	-157.581274560409\\
1190	-117.268745534593\\
1191	-200.172075230252\\
1192	-229.832372965031\\
1193	-183.541012827121\\
1194	-103.073758878921\\
1195	-112.317543335281\\
1196	-195.040097174163\\
1198	-275.000505513828\\
1199	-294.919111600359\\
1200	-292.951098696117\\
1201	-220.837765464645\\
1202	-209.165181301827\\
1203	-229.947222444166\\
1204	-268.777814830419\\
1205	-168.423880972122\\
1206	-97.0500667602764\\
1207	-145.618952061781\\
1208	-119.253260162056\\
1209	-79.534581608151\\
1210	-98.3143032245662\\
1211	-123.858702769629\\
1212	-98.5043746453862\\
1213	-79.4204355374959\\
1214	-101.71792558725\\
1215	-89.7850521738803\\
1216	-121.183229448058\\
1217	-180.45992610545\\
1218	-149.843526299785\\
1219	-113.238751679018\\
1220	-113.756609573883\\
1221	-167.209008574574\\
1222	-272.445798985773\\
1223	-203.435393908812\\
1224	-123.812531604106\\
1225	-93.7592960708873\\
1226	-105.559746751226\\
1227	-108.017099503378\\
1228	-75.4562530385001\\
1229	-81.4569889480756\\
1231	-127.669263590464\\
1232	-142.995846595019\\
1233	-94.5882932267787\\
1234	-167.858242065041\\
1235	-191.254617051218\\
1236	-171.914290160172\\
1237	-156.677023469334\\
1238	-152.653858581045\\
1239	-211.891080194716\\
1240	-127.654079854894\\
1241	-56.1430971656657\\
1242	-76.0875770949374\\
1243	-89.3654012821805\\
1244	-125.707321623022\\
1245	-111.165126254659\\
1246	-84.7125170294464\\
1247	-127.047882483386\\
1248	-123.240992607964\\
1249	-92.2351016286184\\
1250	-107.660826631168\\
1251	-98.2509849941171\\
1252	-93.7056898867525\\
1253	-106.015783131413\\
1254	-61.4117080594322\\
1255	-76.6052368825115\\
1256	-53.9571793403964\\
1257	-60.1963394246857\\
1258	-117.609995936145\\
1259	-124.047744707065\\
1260	-178.633611512216\\
1261	-226.630149966572\\
1263	-91.7616078310518\\
1264	-70.3875537795457\\
1265	-74.6058711137985\\
1266	-99.9715022973453\\
1267	-61.3783372350881\\
1268	-69.6060081051714\\
1269	-89.6504724197509\\
1270	-156.130159715046\\
1271	-143.171914041991\\
1272	-196.144108165421\\
1273	-125.015300160264\\
1274	-125.70075061788\\
1275	-56.0778190841729\\
1276	-63.6434927869102\\
1277	-36.7525125022066\\
1278	-45.2811136070923\\
1279	-47.9796431563104\\
1280	-94.8434948369584\\
1281	-100.55332862739\\
1282	-115.476947263984\\
1283	-126.293878572685\\
1284	-197.870162369563\\
1285	-194.202182770783\\
1286	-136.884421365508\\
1287	-158.055967967111\\
1288	-166.531374700729\\
1289	-210.643757201658\\
1291	-111.774458139284\\
1292	-54.6285556715266\\
1293	-46.5214826834258\\
1294	-41.4263398033499\\
1295	-53.8015213502715\\
1296	-58.0514348218128\\
1297	-50.7825636236673\\
1298	-68.881837706748\\
1299	-111.97884574229\\
1301	-98.5808858467869\\
1302	-117.784555300411\\
1303	-70.9268649063488\\
1304	-41.0341009214924\\
1306	-128.374981270849\\
1307	-118.057078582158\\
1308	-133.285301124217\\
1309	-121.675201581545\\
1310	-87.5269151317459\\
1311	-119.189920556843\\
1312	-169.736320639704\\
1313	-164.883678289676\\
1314	-120.798157029035\\
1315	-200.338354231919\\
1316	-156.759881480952\\
1317	-149.909477930313\\
1318	-173.250921424805\\
1319	-162.801694322398\\
1320	-130.427292705376\\
1321	-116.49462528095\\
1322	-180.900002942325\\
1323	-259.808044565773\\
1324	-202.729428518556\\
1325	-119.125780124246\\
1327	-117.563577564425\\
1328	-148.246273436566\\
1329	-164.515937962945\\
1330	-130.758433379143\\
1331	-136.168724131236\\
1332	-175.414946457691\\
1333	-182.094865579545\\
1334	-121.145228936967\\
1335	-101.850421570094\\
1336	-135.042742364613\\
1337	-208.097435229465\\
1338	-199.438059447692\\
1339	-196.456938948919\\
1340	-111.910357057636\\
1341	-106.034787167633\\
1342	-110.620739811959\\
1343	-62.461964234742\\
1344	-36.4648008679387\\
1345	-28.6071486824508\\
1346	-25.4514755648011\\
1347	-71.2681468443639\\
1348	-103.623530710263\\
1349	-124.200327647454\\
1350	-95.6359918876165\\
1351	-103.861260719729\\
1352	-114.843820117595\\
1353	-72.3128640290854\\
1354	-75.5651454852359\\
1355	-75.6599902086728\\
1356	-56.0155975698058\\
1357	-76.7335666474098\\
1359	-151.592018005383\\
1360	-92.6486353026935\\
1361	-78.6112936344787\\
1362	-60.4090427998856\\
1363	-75.9659429225107\\
1364	-50.1534405955929\\
1366	-192.952130449929\\
1367	-165.03626155489\\
1368	-210.863850428635\\
1369	-245.274364369432\\
1370	-245.12054208262\\
1372	-137.757040420858\\
1373	-130.433229650303\\
1374	-135.882014181161\\
1376	-175.895658832525\\
1377	-173.572260956235\\
1378	-233.801089397334\\
1379	-262.802181888207\\
1380	-317.121786129802\\
1381	-208.025109898807\\
1383	-264.392319703629\\
1384	-197.753844065014\\
1385	-237.420031334386\\
1386	-191.133662698239\\
1387	-110.277735004903\\
1388	-61.7995035588119\\
1389	-65.8662389236831\\
1390	-115.083675257164\\
1391	-100.048715700574\\
1392	-64.0215016650327\\
1393	-85.7686272410194\\
1395	-44.819652751647\\
1396	-65.2538057090349\\
1397	-67.9121895650139\\
1398	-55.6558620825349\\
1400	-143.35996422068\\
1401	-101.802343362286\\
1402	-174.839479154904\\
1403	-260.654949664821\\
1404	-237.636901183788\\
1405	-296.177879506107\\
};
\addlegendentry{MPO prediction}

\end{axis}

\begin{axis}[%
width=6.159cm,
height=1.831cm,
at={(0cm,2.542cm)},
scale only axis,
xmin=1000,
xmax=1405,
xlabel style={font=\color{white!15!black}},
xlabel={Sample index},
ymin=-400,
ymax=0,
ylabel style={font=\color{white!15!black}},
ylabel={$y(t)$},
axis background/.style={fill=white},
title style={font=\bfseries},
title={C7: RMSE(OSA) = 5.0395, RMSE(MPO) = 6.4094},
legend style={legend cell align=left, align=left, draw=white!15!black}
]
\addplot [color=mycolor1, line width=2.0pt]
  table[row sep=crcr]{%
1006	-147.705\\
1007	-175.781\\
1008	-134.277\\
1009	-68.3589999999999\\
1010	-80.566\\
1011	-80.566\\
1012	-100.098\\
1013	-81.787\\
1014	-34.1800000000001\\
1015	-24.414\\
1016	-23.193\\
1018	-128.174\\
1020	-137.939\\
1021	-92.7729999999999\\
1022	-56.152\\
1023	-125.732\\
1024	-86.6700000000001\\
1025	-68.3589999999999\\
1026	-114.746\\
1028	-85.4490000000001\\
1029	-142.822\\
1030	-133.057\\
1031	-102.539\\
1032	-150.146\\
1033	-145.264\\
1034	-113.525\\
1036	-72.021\\
1037	-86.6700000000001\\
1038	-112.305\\
1039	-103.76\\
1040	-109.863\\
1041	-112.305\\
1042	-122.07\\
1043	-175.781\\
1044	-155.029\\
1045	-102.539\\
1046	-92.7729999999999\\
1047	-53.711\\
1048	-58.5940000000001\\
1049	-79.346\\
1050	-59.8140000000001\\
1051	-62.2560000000001\\
1052	-89.1110000000001\\
1053	-102.539\\
1054	-95.2149999999999\\
1055	-151.367\\
1056	-111.084\\
1057	-64.6970000000001\\
1058	-51.27\\
1059	-69.5799999999999\\
1060	-81.787\\
1061	-43.9449999999999\\
1062	-45.1659999999999\\
1063	-67.1389999999999\\
1064	-51.27\\
1065	-42.7249999999999\\
1066	-61.0350000000001\\
1067	-70.8009999999999\\
1068	-109.863\\
1069	-104.98\\
1070	-129.395\\
1071	-120.85\\
1072	-137.939\\
1073	-109.863\\
1074	-107.422\\
1075	-92.7729999999999\\
1076	-91.5530000000001\\
1077	-87.8910000000001\\
1079	-224.609\\
1080	-231.934\\
1081	-225.83\\
1082	-140.381\\
1083	-205.078\\
1084	-252.686\\
1085	-261.23\\
1086	-200.195\\
1088	-335.693\\
1089	-235.596\\
1090	-191.65\\
1091	-128.174\\
1092	-98.877\\
1093	-84.229\\
1094	-104.98\\
1095	-74.463\\
1096	-50.049\\
1097	-48.828\\
1098	-63.4770000000001\\
1099	-95.2149999999999\\
1100	-86.6700000000001\\
1101	-89.1110000000001\\
1102	-119.629\\
1103	-123.291\\
1104	-167.236\\
1105	-134.277\\
1106	-169.678\\
1107	-122.07\\
1108	-108.643\\
1109	-125.732\\
1110	-89.1110000000001\\
1111	-80.566\\
1112	-97.6559999999999\\
1113	-98.877\\
1114	-68.3589999999999\\
1115	-73.242\\
1116	-54.932\\
1117	-61.0350000000001\\
1118	-64.6970000000001\\
1119	-45.1659999999999\\
1121	-72.021\\
1122	-117.188\\
1123	-122.07\\
1124	-136.719\\
1125	-76.904\\
1126	-109.863\\
1127	-173.34\\
1128	-131.836\\
1129	-157.471\\
1130	-157.471\\
1131	-85.4490000000001\\
1132	-59.8140000000001\\
1133	-85.4490000000001\\
1134	-98.877\\
1135	-161.133\\
1136	-202.637\\
1137	-198.975\\
1138	-145.264\\
1139	-150.146\\
1140	-135.498\\
1141	-131.836\\
1142	-126.953\\
1143	-85.4490000000001\\
1144	-76.904\\
1146	-62.2560000000001\\
1147	-70.8009999999999\\
1148	-107.422\\
1149	-164.795\\
1151	-86.6700000000001\\
1152	-73.242\\
1153	-70.8009999999999\\
1154	-42.7249999999999\\
1155	-31.7380000000001\\
1156	-39.0630000000001\\
1157	-72.021\\
1158	-57.373\\
1159	-59.8140000000001\\
1160	-67.1389999999999\\
1161	-63.4770000000001\\
1162	-43.9449999999999\\
1163	-29.297\\
1164	-25.635\\
1165	-43.9449999999999\\
1166	-96.4359999999999\\
1167	-140.381\\
1168	-155.029\\
1169	-101.318\\
1170	-69.5799999999999\\
1171	-50.049\\
1172	-34.1800000000001\\
1173	-83.008\\
1174	-81.787\\
1175	-119.629\\
1176	-145.264\\
1177	-230.713\\
1178	-262.451\\
1179	-211.182\\
1180	-202.637\\
1181	-136.719\\
1182	-151.367\\
1183	-159.912\\
1184	-172.119\\
1185	-129.395\\
1186	-118.408\\
1187	-115.967\\
1188	-103.76\\
1189	-123.291\\
1190	-87.8910000000001\\
1191	-153.809\\
1192	-189.209\\
1194	-78.125\\
1195	-85.4490000000001\\
1196	-146.484\\
1197	-181.885\\
1198	-229.492\\
1199	-241.699\\
1200	-240.479\\
1201	-180.664\\
1202	-163.574\\
1203	-187.988\\
1204	-222.168\\
1205	-139.16\\
1206	-81.787\\
1207	-122.07\\
1208	-102.539\\
1209	-65.9180000000001\\
1210	-87.8910000000001\\
1211	-98.877\\
1212	-76.904\\
1213	-58.5940000000001\\
1214	-80.566\\
1215	-65.9180000000001\\
1216	-91.5530000000001\\
1217	-133.057\\
1218	-122.07\\
1219	-79.346\\
1220	-83.008\\
1221	-126.953\\
1222	-209.961\\
1224	-92.7729999999999\\
1225	-69.5799999999999\\
1226	-83.008\\
1227	-74.463\\
1228	-56.152\\
1229	-62.2560000000001\\
1230	-74.463\\
1231	-96.4359999999999\\
1232	-107.422\\
1233	-72.021\\
1234	-122.07\\
1235	-144.043\\
1236	-137.939\\
1237	-106.201\\
1238	-118.408\\
1239	-158.691\\
1241	-41.5039999999999\\
1242	-58.5940000000001\\
1243	-65.9180000000001\\
1244	-91.5530000000001\\
1245	-81.787\\
1246	-59.8140000000001\\
1247	-96.4359999999999\\
1248	-91.5530000000001\\
1249	-65.9180000000001\\
1250	-83.008\\
1251	-76.904\\
1252	-67.1389999999999\\
1253	-79.346\\
1254	-46.3869999999999\\
1255	-53.711\\
1256	-40.2829999999999\\
1257	-43.9449999999999\\
1258	-86.6700000000001\\
1259	-100.098\\
1261	-177.002\\
1262	-115.967\\
1263	-68.3589999999999\\
1264	-48.828\\
1265	-48.828\\
1266	-73.242\\
1267	-46.3869999999999\\
1268	-52.49\\
1269	-70.8009999999999\\
1270	-119.629\\
1271	-107.422\\
1272	-150.146\\
1273	-102.539\\
1274	-91.5530000000001\\
1275	-47.607\\
1276	-47.607\\
1277	-31.7380000000001\\
1278	-34.1800000000001\\
1279	-41.5039999999999\\
1280	-75.684\\
1281	-83.008\\
1282	-92.7729999999999\\
1283	-97.6559999999999\\
1284	-152.588\\
1285	-130.615\\
1286	-97.6559999999999\\
1287	-119.629\\
1288	-129.395\\
1289	-167.236\\
1290	-133.057\\
1291	-84.229\\
1292	-43.9449999999999\\
1293	-31.7380000000001\\
1294	-34.1800000000001\\
1295	-41.5039999999999\\
1296	-43.9449999999999\\
1297	-40.2829999999999\\
1298	-56.152\\
1299	-87.8910000000001\\
1300	-79.346\\
1301	-81.787\\
1302	-96.4359999999999\\
1303	-58.5940000000001\\
1304	-29.297\\
1306	-97.6559999999999\\
1307	-96.4359999999999\\
1308	-109.863\\
1309	-93.9939999999999\\
1310	-69.5799999999999\\
1311	-97.6559999999999\\
1312	-137.939\\
1313	-139.16\\
1314	-97.6559999999999\\
1315	-162.354\\
1316	-128.174\\
1317	-118.408\\
1318	-137.939\\
1319	-137.939\\
1320	-103.76\\
1321	-90.3320000000001\\
1323	-214.844\\
1324	-164.795\\
1325	-92.7729999999999\\
1326	-92.7729999999999\\
1327	-91.5530000000001\\
1329	-136.719\\
1330	-100.098\\
1331	-104.98\\
1332	-140.381\\
1333	-153.809\\
1334	-103.76\\
1335	-79.346\\
1336	-104.98\\
1337	-175.781\\
1338	-159.912\\
1339	-162.354\\
1340	-95.2149999999999\\
1341	-84.229\\
1342	-83.008\\
1343	-52.49\\
1344	-34.1800000000001\\
1345	-28.076\\
1346	-23.193\\
1347	-59.8140000000001\\
1348	-86.6700000000001\\
1349	-104.98\\
1350	-76.904\\
1352	-97.6559999999999\\
1353	-59.8140000000001\\
1354	-63.4770000000001\\
1355	-61.0350000000001\\
1356	-45.1659999999999\\
1357	-57.373\\
1358	-97.6559999999999\\
1359	-124.512\\
1360	-78.125\\
1361	-56.152\\
1362	-46.3869999999999\\
1363	-59.8140000000001\\
1364	-41.5039999999999\\
1365	-91.5530000000001\\
1366	-158.691\\
1367	-122.07\\
1368	-167.236\\
1369	-194.092\\
1370	-197.754\\
1371	-140.381\\
1372	-106.201\\
1373	-106.201\\
1374	-104.98\\
1375	-120.85\\
1376	-150.146\\
1377	-147.705\\
1378	-200.195\\
1379	-220.947\\
1380	-263.672\\
1381	-178.223\\
1382	-190.43\\
1383	-212.402\\
1384	-163.574\\
1385	-189.209\\
1386	-163.574\\
1387	-91.5530000000001\\
1388	-58.5940000000001\\
1389	-62.2560000000001\\
1390	-95.2149999999999\\
1391	-75.684\\
1392	-51.27\\
1393	-72.021\\
1394	-52.49\\
1395	-40.2829999999999\\
1396	-48.828\\
1397	-56.152\\
1398	-40.2829999999999\\
1400	-109.863\\
1401	-78.125\\
1403	-195.313\\
1404	-178.223\\
1405	-222.168\\
};
\addlegendentry{True output}

\addplot [color=mycolor2, dashed, line width=2.0pt]
  table[row sep=crcr]{%
1006	-146.955168029269\\
1007	-166.895354589941\\
1008	-129.710872769074\\
1009	-68.6168586921294\\
1010	-81.8609510509511\\
1011	-84.9407264605738\\
1012	-101.996457766837\\
1013	-84.7751445265026\\
1014	-25.9893787035569\\
1015	-29.2646169576824\\
1016	-25.3057688251185\\
1017	-77.2750626110778\\
1018	-119.754432843242\\
1019	-124.840842669245\\
1020	-144.124888860455\\
1021	-87.4927763298506\\
1022	-60.2668112531605\\
1023	-131.381560388364\\
1024	-86.180920575265\\
1025	-65.540133976001\\
1026	-110.700652709088\\
1028	-90.8049661333985\\
1029	-134.849030185633\\
1030	-135.614641563021\\
1031	-97.5274104269338\\
1032	-145.776891973042\\
1033	-142.063955049464\\
1034	-114.66262583584\\
1035	-95.8045765129066\\
1036	-71.8191200501615\\
1037	-87.2338545585637\\
1038	-105.923777231205\\
1039	-96.177942762225\\
1040	-107.271818737302\\
1041	-108.735129646577\\
1042	-118.47081167193\\
1043	-169.673956239786\\
1044	-150.589707090588\\
1045	-108.957568781545\\
1046	-95.8608162247524\\
1047	-49.3363469719707\\
1048	-56.1316180332456\\
1049	-75.922041374321\\
1050	-66.215852076064\\
1051	-59.7138043332793\\
1052	-89.3499820860086\\
1053	-100.531669085553\\
1054	-94.8287071878003\\
1055	-151.108065464178\\
1056	-111.663975420561\\
1057	-65.6855583463107\\
1058	-54.6133481364145\\
1059	-68.3042478589045\\
1060	-84.9550771111483\\
1061	-43.2600821431486\\
1062	-42.9351301050372\\
1063	-65.2301405839482\\
1065	-44.1789382193458\\
1066	-57.4716509399843\\
1067	-67.7768911198268\\
1068	-112.806080167753\\
1069	-103.307606584719\\
1070	-126.765271497552\\
1071	-116.653251419995\\
1072	-133.391855369351\\
1073	-112.040695476922\\
1074	-103.375968899755\\
1075	-98.8216901068636\\
1076	-91.3714879119586\\
1077	-87.2415073405357\\
1079	-219.560665950438\\
1080	-226.316613827266\\
1081	-210.551227802517\\
1082	-136.223261777004\\
1083	-202.582346938719\\
1084	-244.540489992504\\
1085	-244.121024330946\\
1086	-200.504945902605\\
1087	-256.830632014538\\
1088	-331.653081908424\\
1089	-237.932131099399\\
1090	-204.051603331821\\
1091	-122.133581746307\\
1092	-97.6478567980171\\
1093	-79.5516758344759\\
1094	-106.20202414745\\
1096	-43.8756049215365\\
1097	-39.7669588486492\\
1098	-57.3626056081621\\
1099	-93.8554588769762\\
1100	-93.0080943456881\\
1101	-88.1099854527622\\
1102	-121.974980370702\\
1103	-117.780351078145\\
1104	-181.56102374565\\
1105	-134.855716957794\\
1106	-169.954794732337\\
1107	-123.637555220838\\
1108	-113.917522295949\\
1109	-124.776187985356\\
1110	-95.4788549561342\\
1111	-84.0027055368048\\
1112	-96.8874951712432\\
1113	-100.730673789136\\
1114	-69.6140443401559\\
1115	-75.0254403326921\\
1116	-56.3140025620166\\
1118	-66.7024808631086\\
1119	-50.7873307169298\\
1120	-55.4247781733829\\
1121	-75.1486759961927\\
1122	-122.530295637238\\
1123	-119.218157825645\\
1124	-132.817318152912\\
1125	-74.0536436614987\\
1126	-118.40571947343\\
1127	-175.320574755302\\
1128	-134.176610500488\\
1129	-162.105426460861\\
1130	-148.069886465967\\
1131	-87.3578434595352\\
1132	-67.274564103664\\
1133	-78.1709971392252\\
1134	-101.565229459047\\
1135	-164.437463617798\\
1136	-202.092037406038\\
1137	-191.505625170563\\
1138	-144.906601550175\\
1139	-148.746252300442\\
1140	-138.931641923268\\
1141	-131.620899543403\\
1142	-137.89089719457\\
1143	-85.0280269266241\\
1144	-72.6453686510215\\
1145	-70.1776942743581\\
1146	-63.4890737988978\\
1147	-74.772742770504\\
1148	-105.603704324476\\
1149	-169.58881524923\\
1151	-89.7222082628728\\
1152	-74.9541329092481\\
1153	-71.9690538481432\\
1154	-41.9196728352222\\
1155	-27.3735672292746\\
1156	-34.9399954949292\\
1157	-77.6639809196333\\
1158	-55.3364711945114\\
1159	-60.5795126841033\\
1160	-69.8194778281622\\
1161	-63.3306334119425\\
1162	-45.9848151395011\\
1163	-33.3990861654595\\
1164	-24.6232784642898\\
1165	-40.1324153938008\\
1166	-101.760195055519\\
1167	-133.377135584582\\
1168	-158.868500497121\\
1169	-106.490253462923\\
1170	-69.6543531403813\\
1172	-33.9251712172843\\
1173	-83.8542307419336\\
1174	-82.6484184177993\\
1175	-118.43478808139\\
1176	-145.057000961991\\
1177	-239.281952847047\\
1178	-279.219276747907\\
1179	-220.250446075673\\
1180	-196.926360229417\\
1181	-132.994548338439\\
1182	-156.57703191114\\
1183	-159.008825109268\\
1184	-172.026568957833\\
1185	-131.727051556375\\
1186	-123.566623285041\\
1187	-114.425675151595\\
1188	-110.692501773604\\
1189	-120.897295257538\\
1190	-91.2321718443077\\
1191	-147.048931291545\\
1192	-189.797288674102\\
1193	-129.482850050119\\
1194	-83.186543035627\\
1195	-76.3273010017217\\
1196	-158.709907435124\\
1197	-176.096135812153\\
1198	-227.758263778255\\
1199	-228.974895788914\\
1200	-236.848375505436\\
1201	-185.787208966803\\
1202	-165.564395689653\\
1203	-197.588426127992\\
1204	-204.184918178192\\
1205	-127.080871144749\\
1206	-90.1981679732974\\
1207	-116.933954791607\\
1208	-112.530966272017\\
1209	-62.2639012261745\\
1210	-86.2407458560067\\
1211	-99.4473559249727\\
1213	-63.5747194005164\\
1214	-80.8280542079679\\
1215	-69.9833367305646\\
1216	-96.1185917246867\\
1217	-136.362932789559\\
1218	-129.726718358242\\
1219	-75.6455805949122\\
1220	-90.6816245267851\\
1221	-122.216751444261\\
1222	-224.764454794659\\
1224	-93.0577778929028\\
1225	-68.4443537785405\\
1226	-80.4063698323355\\
1227	-79.6283745229816\\
1228	-60.0123013494033\\
1229	-57.6270736027216\\
1231	-103.30009856597\\
1232	-102.30282997286\\
1233	-74.0869449464928\\
1234	-136.997239239132\\
1235	-145.181694849334\\
1236	-130.485562985132\\
1237	-110.318086227489\\
1238	-124.227653179938\\
1239	-156.897826596184\\
1241	-38.4994914685201\\
1242	-59.1735138813842\\
1243	-71.5130105662554\\
1244	-99.1005054749594\\
1245	-84.3247225097903\\
1246	-62.7938533746828\\
1247	-101.089757452922\\
1248	-89.7351825546054\\
1249	-69.1137890197651\\
1250	-85.3619746997131\\
1251	-79.4261498160988\\
1252	-70.8214760703809\\
1253	-81.8555726927893\\
1254	-47.9590463315374\\
1255	-53.7669110521831\\
1256	-44.6012039916868\\
1257	-46.7360824901066\\
1258	-94.502259314751\\
1259	-97.2660346729733\\
1260	-142.985936466051\\
1261	-181.703388601958\\
1263	-61.2573179307678\\
1264	-56.2146909860512\\
1265	-54.2517865313889\\
1266	-77.5952070950218\\
1267	-43.9262226219946\\
1268	-53.1545316297545\\
1269	-69.639267953052\\
1270	-129.032519578669\\
1271	-109.549817740588\\
1272	-148.633376546009\\
1273	-103.312702161353\\
1274	-91.6980241610011\\
1275	-45.0648638995619\\
1276	-49.0947194442717\\
1277	-29.6562056116361\\
1278	-36.0567198093738\\
1279	-44.3638217692501\\
1280	-80.1826023413382\\
1281	-76.8893594321012\\
1282	-97.1710369080088\\
1283	-99.0289520025667\\
1284	-153.77063441483\\
1285	-139.404691511036\\
1286	-104.153999423034\\
1287	-118.733429542844\\
1288	-120.570938100865\\
1289	-160.313803236661\\
1290	-136.531247781147\\
1292	-40.746263463732\\
1293	-38.132785745664\\
1294	-34.0015383040359\\
1295	-40.4306831530519\\
1296	-43.9458645656919\\
1297	-41.7251193484894\\
1298	-55.3826526502617\\
1299	-90.8803736887655\\
1300	-80.6143549807273\\
1301	-80.4275744905588\\
1302	-98.283984639477\\
1303	-54.8268756399095\\
1304	-32.27816558105\\
1305	-59.9644529055086\\
1306	-97.3084731101899\\
1307	-93.3661528486762\\
1308	-109.142993228648\\
1309	-93.6948525591613\\
1310	-68.7363289221598\\
1311	-97.4881057513587\\
1312	-136.181950674456\\
1313	-137.245850358488\\
1314	-94.2019060261505\\
1315	-162.636709965971\\
1316	-129.552155711737\\
1317	-122.865925231816\\
1318	-134.023312875237\\
1319	-127.4939134022\\
1321	-91.8284325846935\\
1322	-155.427194237411\\
1323	-206.073916537526\\
1324	-165.594015136128\\
1325	-92.0667417058537\\
1326	-90.7517909438659\\
1327	-95.164611451982\\
1328	-117.881472461255\\
1329	-134.56843374381\\
1330	-99.2299385923432\\
1331	-108.44092854911\\
1332	-136.783646607342\\
1333	-143.175034648781\\
1334	-102.131902080793\\
1335	-82.3344585858426\\
1336	-114.315714731704\\
1337	-174.0569427446\\
1338	-162.692861200529\\
1339	-155.603725671675\\
1340	-88.2367102293222\\
1341	-84.7485924522462\\
1342	-93.6625416906465\\
1343	-46.2342800091988\\
1344	-30.5773973356363\\
1345	-24.3052360709182\\
1346	-23.6300308564187\\
1347	-60.1930791566119\\
1348	-91.9540679077522\\
1349	-99.9847349793511\\
1350	-74.5863583764394\\
1351	-88.5120493345219\\
1352	-98.5175986240881\\
1353	-61.5000713335467\\
1354	-60.5878216385995\\
1355	-58.5880773291785\\
1356	-48.5651973157335\\
1357	-59.9542379648112\\
1358	-95.3151900727539\\
1359	-122.021779487206\\
1360	-80.3411957558274\\
1361	-55.5018414347696\\
1362	-55.4948105965677\\
1363	-57.5793954253766\\
1364	-38.9686214021278\\
1365	-95.4097300413659\\
1366	-162.307935721323\\
1367	-122.224036196152\\
1368	-166.323717077296\\
1369	-189.549892775779\\
1370	-200.660810265345\\
1371	-137.194857230308\\
1372	-111.069431878924\\
1373	-109.322927036247\\
1374	-110.395926188573\\
1375	-120.545427312857\\
1376	-137.616626436727\\
1377	-144.560493406965\\
1378	-188.494934252637\\
1379	-216.02084100008\\
1380	-262.638678049623\\
1381	-186.381997255498\\
1382	-186.716058473452\\
1383	-214.671870982853\\
1384	-158.806856227022\\
1385	-192.441356197629\\
1386	-160.862754182041\\
1387	-85.4331991660247\\
1388	-55.0910517447267\\
1389	-56.9011786986785\\
1390	-94.8841550380587\\
1391	-78.7664785818324\\
1392	-51.3191876133953\\
1393	-67.4704294831952\\
1394	-60.0556808370814\\
1395	-35.4388329570102\\
1396	-50.2937679067848\\
1397	-57.2269087341849\\
1398	-41.416059149517\\
1399	-85.0411328963671\\
1400	-105.554125517566\\
1401	-81.6893213083206\\
1403	-201.656382293216\\
1404	-184.175914727061\\
1405	-213.354918575294\\
};
\addlegendentry{OSA predition}

\addplot [color=mycolor3, dotted, line width=2.0pt]
  table[row sep=crcr]{%
1006	-147.705\\
1007	-175.781\\
1008	-134.277\\
1009	-68.3589999999999\\
1010	-81.86095105096\\
1011	-85.2137402545661\\
1012	-103.437469082222\\
1013	-87.1400496922327\\
1014	-28.1294081441858\\
1015	-30.1088796697099\\
1016	-24.7345671943613\\
1017	-79.2134489602472\\
1018	-122.513208037514\\
1019	-126.028971594574\\
1020	-141.467111731313\\
1021	-86.0489316839448\\
1022	-59.9582120913501\\
1023	-129.718781344367\\
1024	-88.349842316265\\
1025	-66.8232892661727\\
1026	-111.491823203439\\
1027	-99.3589768529746\\
1028	-89.5915636452621\\
1029	-135.0613967266\\
1030	-135.400733601153\\
1031	-95.1631044362075\\
1032	-144.683228983265\\
1033	-138.087540812665\\
1034	-111.429886404727\\
1035	-92.295459707846\\
1036	-70.5952976280396\\
1037	-86.2919920776997\\
1038	-105.191799340058\\
1039	-94.5376747416522\\
1040	-102.780777617254\\
1041	-103.653774852349\\
1042	-113.33756341606\\
1043	-163.561954339057\\
1044	-144.529293484735\\
1045	-102.288078718241\\
1046	-91.2500417688352\\
1047	-48.3484299315\\
1048	-53.8538104986526\\
1049	-72.8753874826609\\
1050	-62.9943638837524\\
1051	-57.8370182688448\\
1052	-88.7612437834148\\
1053	-98.8339001716545\\
1054	-93.9449759293286\\
1055	-149.286574725565\\
1056	-110.718305432252\\
1057	-64.7375860350892\\
1058	-54.4026932212273\\
1059	-68.8763431875811\\
1060	-85.8778552468518\\
1061	-43.8108523038654\\
1062	-44.4617845879893\\
1063	-65.2218585491412\\
1064	-53.982859027094\\
1065	-44.1265896365358\\
1066	-58.5620263433134\\
1067	-67.7627462018982\\
1068	-111.662855437388\\
1069	-102.578205518644\\
1070	-126.505483337226\\
1071	-115.181181604692\\
1072	-131.19802709754\\
1073	-108.519289950922\\
1074	-100.409527409403\\
1075	-96.3842494531241\\
1076	-89.2027031898074\\
1077	-87.7400294403321\\
1079	-218.658230967668\\
1080	-224.188021989118\\
1081	-207.17607439697\\
1082	-130.561647048115\\
1083	-192.558246111025\\
1084	-236.289145702737\\
1085	-235.694472823726\\
1086	-190.514696785921\\
1087	-243.18998475006\\
1088	-320.266535061585\\
1089	-224.959867364836\\
1090	-195.969905488275\\
1091	-117.00884964438\\
1093	-74.0890649877117\\
1094	-102.754930552992\\
1095	-69.9233607010099\\
1096	-42.1260439900236\\
1097	-35.791250155065\\
1098	-51.3374975028596\\
1099	-85.7793971226952\\
1100	-85.7761935989795\\
1101	-83.6045863618172\\
1102	-119.067359625594\\
1103	-115.173624274374\\
1104	-178.988895070494\\
1105	-133.620291313636\\
1106	-173.461938690879\\
1107	-123.832759677363\\
1108	-116.282723520334\\
1109	-127.13269672654\\
1110	-98.8260118052428\\
1111	-86.725189959403\\
1112	-102.244099229474\\
1113	-104.432825451707\\
1114	-72.8965521045984\\
1115	-78.4534811329936\\
1116	-59.3889665238303\\
1118	-69.5082303938927\\
1119	-53.3974922194386\\
1120	-59.2218618720653\\
1121	-78.6193766842471\\
1122	-125.419492609556\\
1123	-123.554066402327\\
1124	-136.422078005585\\
1125	-74.6925028437222\\
1126	-118.349200179197\\
1127	-176.246328265539\\
1128	-137.600923308487\\
1129	-164.710397269828\\
1130	-152.101824881288\\
1131	-89.069650242127\\
1132	-66.293375215088\\
1133	-81.0976505831056\\
1134	-102.950068259901\\
1135	-164.85207485291\\
1136	-203.873113484681\\
1137	-193.06705533045\\
1138	-144.66144183832\\
1139	-146.599842382714\\
1140	-137.885920660092\\
1141	-130.398889386009\\
1142	-138.311049884531\\
1143	-86.4622403555088\\
1144	-77.2225364000965\\
1146	-64.4939016318583\\
1147	-75.7776082031901\\
1148	-107.576707418931\\
1149	-171.776601293002\\
1150	-129.927837628146\\
1151	-93.2994195403587\\
1152	-78.0118008953507\\
1153	-75.9244537373133\\
1154	-44.9373343589132\\
1155	-30.0104833345715\\
1156	-35.7238767015479\\
1157	-76.8012728406429\\
1158	-55.3346427458332\\
1159	-61.5089021057754\\
1160	-69.5639234630323\\
1161	-64.6108972203281\\
1162	-46.939785123614\\
1163	-34.5280817677706\\
1164	-26.9240247645616\\
1165	-42.3409238568552\\
1166	-102.8549927294\\
1167	-134.724665432042\\
1168	-159.691577190463\\
1169	-105.246409475319\\
1170	-72.5829471675738\\
1172	-36.2120724911283\\
1173	-86.2422034690755\\
1174	-84.4621858204123\\
1175	-120.496932553448\\
1176	-146.618705628769\\
1177	-240.587040544677\\
1178	-281.248641532224\\
1179	-225.700577447631\\
1180	-206.495221856331\\
1181	-139.854476131303\\
1182	-160.256254104003\\
1183	-162.814796060702\\
1184	-176.332937155466\\
1185	-133.830007719581\\
1186	-126.565998554574\\
1187	-118.029866893095\\
1188	-114.570965276635\\
1189	-124.48098122709\\
1190	-95.9669664343864\\
1191	-150.196521548411\\
1192	-192.859741387361\\
1193	-128.713504398891\\
1194	-84.2409123068246\\
1195	-75.8226540136575\\
1196	-158.728433400262\\
1197	-175.235807737082\\
1198	-230.037129112411\\
1199	-227.103735248996\\
1200	-235.400114236572\\
1201	-179.081222376103\\
1202	-163.047841788494\\
1203	-195.753151891316\\
1204	-204.833645982712\\
1205	-127.545286087086\\
1206	-80.9882246582602\\
1207	-111.76050000074\\
1208	-108.783797406584\\
1209	-59.1286328590343\\
1210	-86.6129624321773\\
1211	-96.4499561866032\\
1212	-81.208242449329\\
1213	-63.2469139128741\\
1214	-83.215154351135\\
1215	-72.2045319368269\\
1216	-98.9178128757251\\
1217	-140.648431479368\\
1218	-134.275231444066\\
1219	-80.841909151437\\
1220	-96.1289781160087\\
1221	-126.81333064439\\
1222	-230.388567014241\\
1224	-103.759931770636\\
1225	-76.2734752428614\\
1226	-87.6164179413856\\
1227	-84.187823093579\\
1228	-64.3438247497099\\
1229	-63.3318133848452\\
1231	-106.905351411747\\
1232	-108.351706637626\\
1233	-78.0103905949024\\
1234	-140.172356737064\\
1235	-151.454799813873\\
1236	-138.513325901464\\
1237	-113.728219981712\\
1238	-127.250479283837\\
1239	-161.862143156791\\
1241	-39.3614808663444\\
1242	-58.9034655004421\\
1243	-70.916759991052\\
1244	-100.04832413857\\
1245	-87.6783399763067\\
1246	-66.9745739569546\\
1247	-105.612257341519\\
1248	-94.8439356982126\\
1249	-73.274326161514\\
1250	-88.9129775597505\\
1251	-83.8751904718699\\
1252	-74.6665521848267\\
1253	-86.5747207160041\\
1254	-52.3944951025298\\
1255	-58.10669213029\\
1256	-48.2047880919044\\
1257	-50.4919621630156\\
1258	-99.4042692591233\\
1259	-102.9887640789\\
1260	-149.077948196231\\
1261	-186.930468541756\\
1263	-66.284023730577\\
1264	-59.9605144603618\\
1265	-56.6737374199147\\
1266	-83.565458852129\\
1267	-48.641415780744\\
1268	-57.9105806814221\\
1269	-72.6043518376882\\
1270	-132.113149762788\\
1271	-113.021249383139\\
1272	-154.477597042052\\
1273	-106.360820141871\\
1274	-94.7086490718293\\
1275	-47.3666966802962\\
1276	-50.468666020801\\
1277	-30.4314554015641\\
1278	-36.9056270956876\\
1279	-44.6387172231639\\
1280	-81.7595532681221\\
1281	-79.2973572397134\\
1282	-98.7444075515812\\
1283	-99.5381084189958\\
1284	-156.019347797325\\
1285	-140.884549733413\\
1286	-107.615053939452\\
1287	-124.770814886906\\
1288	-125.943850671874\\
1289	-162.949283074177\\
1290	-135.238922201668\\
1292	-42.5466062983305\\
1293	-39.5560382157348\\
1294	-35.4658539334389\\
1295	-43.5967435912119\\
1296	-45.0417142130752\\
1297	-43.0820315242322\\
1298	-56.5338596594668\\
1299	-92.1055734988877\\
1300	-81.8213945223254\\
1301	-82.4895069867405\\
1302	-99.5826800266666\\
1303	-55.8978346057211\\
1304	-32.8808002278697\\
1305	-60.0074012033488\\
1306	-97.3810855576355\\
1307	-91.8962603616224\\
1308	-107.963676516113\\
1309	-91.4976241767542\\
1310	-67.5577448203919\\
1311	-95.8242241737305\\
1312	-134.725648555742\\
1313	-135.724983191509\\
1314	-92.3476515811703\\
1315	-159.875573264505\\
1316	-126.86601764121\\
1317	-121.439266926487\\
1318	-133.71648850568\\
1319	-127.620818477297\\
1320	-106.095290371249\\
1321	-88.6091274724454\\
1322	-154.653048817551\\
1323	-205.404184715016\\
1324	-164.59504011704\\
1325	-88.4067923957234\\
1326	-89.6413635362564\\
1327	-92.3436041612092\\
1328	-116.616391099253\\
1329	-134.99941027929\\
1330	-99.8575030026918\\
1331	-107.914414708265\\
1332	-137.321657309394\\
1333	-143.561220201303\\
1334	-99.1068134440322\\
1335	-77.7452515801765\\
1336	-111.462224357496\\
1337	-173.64707676112\\
1339	-155.762622885996\\
1340	-89.0905113164347\\
1341	-80.8966632906868\\
1342	-90.5958084828555\\
1343	-46.3303826113304\\
1344	-31.6589199464856\\
1345	-21.4781817624657\\
1346	-21.5891637055599\\
1348	-89.7703986298695\\
1349	-99.2914635674385\\
1350	-74.3361591141986\\
1352	-97.050757035172\\
1353	-60.4543046806336\\
1354	-60.3531553215344\\
1355	-58.0219663598591\\
1356	-46.8227868501733\\
1357	-59.1994704507026\\
1358	-95.798640549938\\
1359	-122.284794581697\\
1360	-79.3567283952821\\
1361	-55.1260543133662\\
1362	-55.5601727909266\\
1363	-59.1434667853728\\
1364	-41.9843739573573\\
1365	-96.2254510917362\\
1366	-163.77502489012\\
1367	-124.639171815788\\
1368	-169.052379733309\\
1369	-191.573041806387\\
1370	-201.398408221955\\
1371	-136.941201731048\\
1372	-111.898937154666\\
1373	-109.135226588547\\
1374	-113.186601451983\\
1375	-123.402786465649\\
1376	-141.618637183191\\
1377	-144.542031732932\\
1378	-185.452129046839\\
1379	-211.773539725567\\
1380	-254.997963036379\\
1381	-181.675374863018\\
1382	-182.886533306616\\
1383	-213.686858973466\\
1384	-155.603731116195\\
1385	-191.252035276851\\
1386	-157.688176197777\\
1387	-85.535510450454\\
1388	-51.1070966205675\\
1389	-52.8780180646872\\
1390	-89.3869016592039\\
1391	-73.7156525811224\\
1392	-48.6118052606732\\
1393	-65.4630654720615\\
1394	-57.3513064699671\\
1395	-34.1655643118734\\
1396	-50.2457014673821\\
1397	-55.2999691778462\\
1398	-41.919467412572\\
1399	-84.9278249094557\\
1400	-107.777946296915\\
1401	-83.8988735740447\\
1403	-205.694894553032\\
1404	-189.70327526838\\
1405	-220.251963668666\\
};
\addlegendentry{MPO prediction}

\end{axis}

\begin{axis}[%
width=6.159cm,
height=1.831cm,
at={(8.104cm,2.542cm)},
scale only axis,
xmin=1000,
xmax=1405,
xlabel style={font=\color{white!15!black}},
xlabel={Sample index},
ymin=-279.541,
ymax=0,
ylabel style={font=\color{white!15!black}},
ylabel={$y(t)$},
axis background/.style={fill=white},
title style={font=\bfseries},
title={C8: RMSE(OSA) = 4.1858, RMSE(MPO) = 5.2038},
legend style={legend cell align=left, align=left, draw=white!15!black}
]
\addplot [color=mycolor1, line width=2.0pt]
  table[row sep=crcr]{%
1006	-125.732\\
1007	-153.809\\
1008	-115.967\\
1009	-61.0350000000001\\
1010	-74.463\\
1011	-73.242\\
1012	-86.6700000000001\\
1013	-73.242\\
1014	-34.1800000000001\\
1015	-20.752\\
1016	-23.193\\
1017	-58.5940000000001\\
1018	-100.098\\
1019	-109.863\\
1020	-114.746\\
1022	-52.49\\
1023	-104.98\\
1025	-59.8140000000001\\
1026	-97.6559999999999\\
1027	-83.008\\
1028	-72.021\\
1029	-118.408\\
1030	-107.422\\
1031	-83.008\\
1032	-119.629\\
1033	-123.291\\
1034	-95.2149999999999\\
1035	-80.566\\
1036	-62.2560000000001\\
1037	-75.684\\
1038	-95.2149999999999\\
1039	-87.8910000000001\\
1040	-91.5530000000001\\
1041	-96.4359999999999\\
1042	-102.539\\
1043	-141.602\\
1044	-128.174\\
1045	-85.4490000000001\\
1046	-79.346\\
1047	-47.607\\
1048	-48.828\\
1049	-68.3589999999999\\
1050	-52.49\\
1051	-57.373\\
1052	-75.684\\
1053	-89.1110000000001\\
1054	-81.787\\
1055	-128.174\\
1056	-98.877\\
1057	-57.373\\
1058	-45.1659999999999\\
1059	-62.2560000000001\\
1060	-72.021\\
1061	-40.2829999999999\\
1062	-42.7249999999999\\
1063	-56.152\\
1064	-46.3869999999999\\
1065	-40.2829999999999\\
1066	-52.49\\
1067	-62.2560000000001\\
1068	-92.7729999999999\\
1069	-90.3320000000001\\
1070	-107.422\\
1071	-98.877\\
1072	-115.967\\
1073	-91.5530000000001\\
1074	-91.5530000000001\\
1075	-79.346\\
1076	-75.684\\
1077	-76.904\\
1079	-189.209\\
1080	-194.092\\
1081	-189.209\\
1082	-119.629\\
1083	-170.898\\
1084	-213.623\\
1085	-219.727\\
1086	-169.678\\
1087	-219.727\\
1088	-279.541\\
1089	-197.754\\
1090	-161.133\\
1091	-109.863\\
1092	-85.4490000000001\\
1093	-73.242\\
1094	-91.5530000000001\\
1095	-62.2560000000001\\
1096	-46.3869999999999\\
1097	-42.7249999999999\\
1098	-54.932\\
1099	-79.346\\
1100	-74.463\\
1101	-75.684\\
1102	-100.098\\
1103	-103.76\\
1104	-141.602\\
1105	-114.746\\
1106	-142.822\\
1107	-104.98\\
1108	-90.3320000000001\\
1109	-103.76\\
1110	-76.904\\
1111	-69.5799999999999\\
1112	-84.229\\
1113	-86.6700000000001\\
1114	-59.8140000000001\\
1115	-64.6970000000001\\
1116	-48.828\\
1117	-53.711\\
1118	-57.373\\
1119	-41.5039999999999\\
1120	-52.49\\
1121	-65.9180000000001\\
1122	-101.318\\
1123	-104.98\\
1124	-114.746\\
1125	-68.3589999999999\\
1126	-93.9939999999999\\
1127	-150.146\\
1128	-114.746\\
1129	-125.732\\
1130	-128.174\\
1131	-80.566\\
1132	-53.711\\
1133	-75.684\\
1134	-85.4490000000001\\
1135	-137.939\\
1136	-172.119\\
1137	-170.898\\
1138	-123.291\\
1139	-130.615\\
1140	-115.967\\
1142	-111.084\\
1143	-76.904\\
1144	-68.3589999999999\\
1145	-61.0350000000001\\
1146	-57.373\\
1147	-62.2560000000001\\
1148	-91.5530000000001\\
1149	-140.381\\
1151	-79.346\\
1152	-64.6970000000001\\
1153	-62.2560000000001\\
1154	-40.2829999999999\\
1155	-29.297\\
1156	-36.6210000000001\\
1157	-63.4770000000001\\
1158	-51.27\\
1159	-52.49\\
1160	-58.5940000000001\\
1161	-54.932\\
1162	-37.8420000000001\\
1163	-28.076\\
1164	-23.193\\
1165	-39.0630000000001\\
1166	-83.008\\
1167	-117.188\\
1168	-130.615\\
1169	-86.6700000000001\\
1170	-58.5940000000001\\
1172	-32.9590000000001\\
1173	-67.1389999999999\\
1174	-65.9180000000001\\
1176	-117.188\\
1177	-186.768\\
1178	-216.064\\
1179	-177.002\\
1180	-168.457\\
1181	-109.863\\
1182	-122.07\\
1183	-131.836\\
1184	-146.484\\
1185	-108.643\\
1186	-100.098\\
1187	-98.877\\
1188	-89.1110000000001\\
1189	-106.201\\
1190	-78.125\\
1191	-130.615\\
1192	-159.912\\
1194	-70.8009999999999\\
1195	-73.242\\
1196	-123.291\\
1197	-153.809\\
1198	-189.209\\
1199	-202.637\\
1200	-202.637\\
1201	-153.809\\
1202	-137.939\\
1203	-158.691\\
1204	-187.988\\
1205	-109.863\\
1206	-72.021\\
1207	-101.318\\
1208	-86.6700000000001\\
1209	-53.711\\
1210	-74.463\\
1211	-84.229\\
1213	-51.27\\
1214	-70.8009999999999\\
1215	-58.5940000000001\\
1216	-79.346\\
1217	-107.422\\
1218	-100.098\\
1219	-69.5799999999999\\
1220	-70.8009999999999\\
1221	-107.422\\
1222	-177.002\\
1224	-79.346\\
1225	-61.0350000000001\\
1226	-70.8009999999999\\
1227	-64.6970000000001\\
1228	-48.828\\
1229	-53.711\\
1230	-65.9180000000001\\
1231	-83.008\\
1232	-91.5530000000001\\
1233	-63.4770000000001\\
1234	-104.98\\
1235	-124.512\\
1236	-118.408\\
1237	-91.5530000000001\\
1238	-100.098\\
1239	-135.498\\
1241	-42.7249999999999\\
1242	-57.373\\
1243	-62.2560000000001\\
1244	-81.787\\
1245	-72.021\\
1246	-52.49\\
1247	-79.346\\
1248	-80.566\\
1249	-57.373\\
1250	-70.8009999999999\\
1252	-56.152\\
1253	-67.1389999999999\\
1254	-41.5039999999999\\
1255	-48.828\\
1256	-36.6210000000001\\
1257	-40.2829999999999\\
1258	-75.684\\
1259	-84.229\\
1260	-113.525\\
1261	-146.484\\
1262	-98.877\\
1263	-58.5940000000001\\
1264	-46.3869999999999\\
1265	-43.9449999999999\\
1266	-63.4770000000001\\
1267	-42.7249999999999\\
1268	-47.607\\
1269	-61.0350000000001\\
1270	-102.539\\
1271	-93.9939999999999\\
1272	-134.277\\
1273	-89.1110000000001\\
1274	-81.787\\
1275	-46.3869999999999\\
1276	-45.1659999999999\\
1277	-28.076\\
1278	-35.4000000000001\\
1279	-37.8420000000001\\
1280	-67.1389999999999\\
1281	-70.8009999999999\\
1282	-79.346\\
1283	-85.4490000000001\\
1284	-125.732\\
1285	-112.305\\
1286	-84.229\\
1287	-102.539\\
1288	-109.863\\
1289	-144.043\\
1290	-115.967\\
1291	-75.684\\
1292	-40.2829999999999\\
1293	-29.297\\
1294	-29.297\\
1295	-36.6210000000001\\
1296	-39.0630000000001\\
1297	-37.8420000000001\\
1298	-50.049\\
1299	-74.463\\
1300	-68.3589999999999\\
1301	-70.8009999999999\\
1302	-83.008\\
1303	-51.27\\
1304	-30.518\\
1305	-58.5940000000001\\
1306	-90.3320000000001\\
1307	-87.8910000000001\\
1308	-97.6559999999999\\
1309	-83.008\\
1310	-61.0350000000001\\
1311	-85.4490000000001\\
1312	-118.408\\
1313	-118.408\\
1314	-84.229\\
1315	-136.719\\
1316	-100.098\\
1317	-101.318\\
1318	-117.188\\
1319	-115.967\\
1320	-86.6700000000001\\
1321	-76.904\\
1323	-178.223\\
1324	-139.16\\
1325	-79.346\\
1326	-76.904\\
1327	-79.346\\
1328	-98.877\\
1329	-114.746\\
1330	-85.4490000000001\\
1331	-90.3320000000001\\
1332	-120.85\\
1333	-131.836\\
1334	-87.8910000000001\\
1335	-68.3589999999999\\
1336	-91.5530000000001\\
1337	-156.25\\
1338	-137.939\\
1339	-140.381\\
1340	-84.229\\
1341	-76.904\\
1342	-72.021\\
1343	-47.607\\
1344	-30.518\\
1346	-23.193\\
1347	-54.932\\
1349	-87.8910000000001\\
1350	-65.9180000000001\\
1351	-73.242\\
1352	-83.008\\
1353	-53.711\\
1354	-56.152\\
1355	-53.711\\
1356	-40.2829999999999\\
1357	-51.27\\
1358	-84.229\\
1359	-106.201\\
1360	-63.4770000000001\\
1361	-48.828\\
1362	-40.2829999999999\\
1363	-50.049\\
1364	-35.4000000000001\\
1365	-76.904\\
1366	-130.615\\
1367	-104.98\\
1368	-139.16\\
1369	-162.354\\
1370	-164.795\\
1371	-124.512\\
1372	-90.3320000000001\\
1373	-89.1110000000001\\
1374	-89.1110000000001\\
1375	-106.201\\
1376	-125.732\\
1377	-125.732\\
1378	-168.457\\
1379	-183.105\\
1380	-219.727\\
1381	-147.705\\
1383	-184.326\\
1384	-140.381\\
1385	-162.354\\
1386	-139.16\\
1387	-80.566\\
1388	-52.49\\
1389	-56.152\\
1390	-81.787\\
1391	-65.9180000000001\\
1392	-43.9449999999999\\
1393	-62.2560000000001\\
1394	-47.607\\
1395	-36.6210000000001\\
1396	-46.3869999999999\\
1397	-52.49\\
1398	-36.6210000000001\\
1399	-70.8009999999999\\
1400	-92.7729999999999\\
1401	-68.3589999999999\\
1403	-164.795\\
1404	-150.146\\
1405	-196.533\\
};
\addlegendentry{True output}

\addplot [color=mycolor2, dashed, line width=2.0pt]
  table[row sep=crcr]{%
1006	-115.93992834947\\
1007	-148.823399870046\\
1008	-118.538982872481\\
1009	-58.037148910912\\
1010	-72.3788963339402\\
1011	-72.6251268487251\\
1012	-92.8187592071399\\
1013	-72.1916890384689\\
1014	-25.7625632319164\\
1015	-27.1600003348044\\
1016	-23.2944172131229\\
1017	-62.032906346079\\
1018	-95.2345530002549\\
1019	-105.353854578872\\
1020	-109.226666818455\\
1021	-83.6360375832242\\
1022	-50.4507996596958\\
1023	-111.511290702861\\
1024	-77.1479846338204\\
1025	-59.7404954418926\\
1026	-102.741337172466\\
1028	-71.7749609790185\\
1029	-115.517957629535\\
1030	-104.772912279705\\
1031	-83.7607699649436\\
1032	-116.019044406542\\
1033	-114.883222450035\\
1036	-65.259547806892\\
1037	-77.0431445441541\\
1038	-87.3039622257377\\
1039	-88.8960474134115\\
1040	-89.290433429223\\
1041	-92.2248139123938\\
1042	-96.1328160217083\\
1043	-135.916565351667\\
1044	-127.143260340508\\
1045	-90.9911961460568\\
1046	-80.6628842605992\\
1047	-45.0346614361306\\
1048	-48.0750649443883\\
1049	-67.6793365348919\\
1050	-54.4738843557482\\
1051	-54.7180010195525\\
1052	-77.649509631186\\
1053	-86.3914268856222\\
1054	-81.3236476865195\\
1055	-127.806380012888\\
1056	-95.1331742882039\\
1057	-58.9489718789928\\
1058	-49.508099057032\\
1059	-60.9302321264538\\
1060	-73.7199242039349\\
1061	-38.1585372230052\\
1062	-41.3249751153337\\
1063	-58.6307770780888\\
1065	-39.5601877417462\\
1066	-54.0546543825249\\
1067	-57.4104558234783\\
1068	-98.2183421987779\\
1069	-85.0603887827331\\
1070	-106.792127258186\\
1071	-99.0126332117056\\
1072	-107.61592353183\\
1073	-93.0981276806685\\
1074	-91.2163886867911\\
1075	-81.0809574299631\\
1076	-81.6271792271307\\
1077	-75.3880274491592\\
1078	-121.190639258704\\
1079	-177.914370151446\\
1080	-189.571232660362\\
1081	-183.720902154167\\
1082	-114.851993412575\\
1083	-173.955269282282\\
1084	-201.372611593973\\
1085	-207.695436133982\\
1086	-170.014427629482\\
1087	-212.061716063864\\
1088	-278.269321072586\\
1089	-201.235871372318\\
1090	-172.169325166941\\
1091	-100.665321194169\\
1092	-83.811723391429\\
1093	-69.964626357093\\
1094	-87.859828314392\\
1095	-61.3246428444943\\
1096	-37.7955311560893\\
1097	-35.6015969958978\\
1098	-55.644123501082\\
1099	-82.9745696617842\\
1100	-75.8854550991061\\
1101	-77.5056436485183\\
1102	-99.4400461914177\\
1103	-106.309295958048\\
1104	-143.329566256406\\
1105	-126.526922029497\\
1106	-137.560367390696\\
1107	-104.27545731434\\
1108	-93.6220684922512\\
1109	-105.46123533924\\
1110	-80.6466764283282\\
1111	-71.7174319631047\\
1112	-87.1691903004164\\
1113	-87.1404539865923\\
1114	-65.5135820274177\\
1115	-67.263680129962\\
1116	-47.8125573274726\\
1117	-51.4556536720556\\
1118	-60.4254716849584\\
1119	-46.2373289106592\\
1120	-52.671910607573\\
1121	-66.8199073781739\\
1122	-105.070574921709\\
1123	-98.6271071350186\\
1124	-117.143971607898\\
1125	-65.3825077550537\\
1126	-98.6911618384686\\
1127	-148.380795052137\\
1128	-118.600311509229\\
1129	-130.528277938279\\
1130	-123.288154082686\\
1131	-80.0346708281379\\
1132	-53.2340751357026\\
1133	-76.3172084962305\\
1134	-86.1055207605418\\
1135	-138.12638559118\\
1136	-168.271858733656\\
1137	-166.98904736523\\
1138	-119.668148686621\\
1139	-138.31656599238\\
1140	-117.855694452064\\
1141	-114.427837958283\\
1142	-114.08428964669\\
1143	-74.3425727761817\\
1144	-69.2636862546844\\
1145	-61.050652026343\\
1146	-56.6637035598017\\
1147	-65.9233139303672\\
1148	-94.6764672371148\\
1149	-133.996563269191\\
1150	-120.538224913424\\
1151	-77.4638966346486\\
1152	-67.4347732672443\\
1153	-65.5768273726976\\
1154	-36.7913453139829\\
1155	-26.8774075838357\\
1156	-34.7839739648859\\
1157	-62.3287273218853\\
1158	-50.9353623875545\\
1159	-56.6181746896825\\
1160	-59.4175574633932\\
1161	-58.2103030449193\\
1162	-39.6644169535773\\
1163	-30.2696130702036\\
1164	-22.0421405295015\\
1165	-35.6204439229748\\
1166	-82.7322879356566\\
1167	-111.729129475326\\
1168	-127.318802250074\\
1169	-89.378242154404\\
1170	-60.2407312639239\\
1171	-45.8856378049218\\
1172	-33.8729121075305\\
1173	-66.0967130311903\\
1174	-65.1270181773216\\
1175	-89.2702446770334\\
1176	-116.679378665307\\
1177	-191.888170117714\\
1178	-223.696106030967\\
1179	-185.230471511957\\
1180	-164.941908355678\\
1181	-105.740892540964\\
1182	-124.512128293444\\
1183	-133.685208097376\\
1184	-141.393651023643\\
1185	-114.562888171162\\
1186	-96.8366202718855\\
1187	-104.077595387804\\
1188	-96.0071375236487\\
1189	-103.144641059149\\
1190	-79.0168618153907\\
1191	-129.40017197917\\
1192	-161.450177238967\\
1194	-72.9169536642323\\
1195	-70.9116968767048\\
1196	-120.371825266038\\
1197	-151.744578587033\\
1198	-176.860866313396\\
1199	-193.786989951709\\
1200	-203.764821891694\\
1201	-151.217569597925\\
1202	-145.318330137958\\
1203	-158.103392535546\\
1204	-181.107306325907\\
1205	-104.946529589974\\
1206	-71.239853821892\\
1207	-105.403526166714\\
1208	-89.7289327532224\\
1209	-52.755774724621\\
1210	-72.3704041417725\\
1211	-85.9953980478001\\
1212	-70.7196930987502\\
1213	-59.1689070041377\\
1214	-71.206282968679\\
1215	-61.5733083151324\\
1216	-75.3362937932257\\
1217	-116.133987647634\\
1218	-103.227459439589\\
1219	-73.9090411049335\\
1220	-72.5102172760373\\
1221	-107.016266008999\\
1222	-174.421085802227\\
1223	-141.60985433954\\
1224	-77.5262746863145\\
1225	-62.1914290248321\\
1226	-71.4515741877326\\
1227	-69.7272298753153\\
1228	-47.018216921995\\
1229	-55.3746610801979\\
1230	-66.5790742179963\\
1231	-82.9027653131868\\
1232	-96.5143410890792\\
1233	-66.7524585729218\\
1234	-105.298999056955\\
1235	-128.270174057368\\
1236	-116.651156826358\\
1237	-92.9061169507875\\
1238	-102.258472021907\\
1239	-131.431477033748\\
1240	-99.9249873002038\\
1241	-35.4356868984146\\
1243	-65.6522710976728\\
1244	-88.4268972782763\\
1245	-76.5169071285545\\
1246	-57.847404550336\\
1247	-85.4660931645274\\
1248	-76.8829634143187\\
1249	-57.2580210382037\\
1250	-77.0643695827152\\
1251	-66.1478892891892\\
1252	-57.7814614517431\\
1253	-71.1709698138488\\
1254	-42.4906590661551\\
1255	-50.1487825073104\\
1256	-36.2562085216271\\
1257	-41.4302539869655\\
1258	-78.9917133747435\\
1259	-81.483867439701\\
1260	-109.758308655455\\
1261	-150.763515199632\\
1263	-58.8969155870307\\
1264	-46.3155741648034\\
1265	-46.1275663414001\\
1266	-69.9001816278676\\
1267	-42.988513657554\\
1268	-46.0272422782912\\
1269	-60.9708156980187\\
1270	-103.902369209356\\
1271	-95.1256358853643\\
1272	-134.976159908913\\
1273	-93.2144057205824\\
1274	-79.3062541075278\\
1275	-40.2060291259893\\
1276	-46.5271522508242\\
1277	-29.5266671926749\\
1279	-38.8440773295331\\
1280	-66.2564616670497\\
1281	-71.246712759749\\
1282	-77.6848685273078\\
1283	-83.1802141995829\\
1284	-127.315410744539\\
1285	-125.7218004148\\
1286	-86.2378900841011\\
1288	-110.341266928658\\
1289	-138.074154095419\\
1290	-112.461696607889\\
1291	-78.8892618512452\\
1292	-36.5511489105149\\
1293	-34.0189597570518\\
1294	-31.8407235130153\\
1295	-37.8903474554513\\
1296	-38.7504475044748\\
1297	-37.291747504222\\
1298	-46.1476929776904\\
1299	-79.4386010045062\\
1300	-72.0848101046856\\
1301	-67.1532079117101\\
1302	-80.4866237524045\\
1303	-53.7385587165688\\
1304	-32.1128545466099\\
1305	-57.1795187294399\\
1306	-88.1050561998802\\
1307	-83.9938017941342\\
1308	-98.8631923683033\\
1310	-62.5826142384424\\
1311	-83.3127152906429\\
1312	-112.650253940999\\
1313	-115.706325917255\\
1314	-85.5890237623339\\
1315	-136.398197827352\\
1316	-105.447327742581\\
1317	-97.5728073651824\\
1318	-118.772320153952\\
1319	-114.401097596011\\
1320	-90.8220279851166\\
1321	-82.0505902106506\\
1322	-123.25571891508\\
1323	-174.139200048974\\
1324	-137.96960704228\\
1325	-77.7187784609496\\
1326	-74.3833158200839\\
1327	-84.6096817451928\\
1329	-108.083601290122\\
1330	-87.8940449039878\\
1331	-97.0530379004845\\
1332	-118.291811359817\\
1333	-121.852293652248\\
1334	-88.6939288690401\\
1335	-79.1954110767701\\
1336	-93.5107121793048\\
1337	-155.885965106936\\
1338	-135.956449517832\\
1339	-134.962580321526\\
1340	-81.2889687177349\\
1341	-76.5461776246\\
1342	-73.3693717158549\\
1343	-47.8158050306515\\
1344	-27.5595456311758\\
1345	-24.2097220996354\\
1346	-22.5672084417888\\
1347	-50.9268808774914\\
1348	-72.4713056803869\\
1349	-88.2236108986276\\
1350	-67.3346566396324\\
1351	-70.4766044844746\\
1352	-82.1631233451221\\
1353	-53.0205242295101\\
1354	-56.1035426874919\\
1355	-55.5587323599543\\
1356	-42.3048341320712\\
1357	-52.7224312629269\\
1358	-81.0318881666444\\
1359	-102.423045179955\\
1360	-63.0701632979512\\
1361	-50.3850965158904\\
1362	-45.5618874925972\\
1363	-54.7485440927996\\
1364	-33.1727791959106\\
1366	-127.312579809651\\
1367	-107.082474624176\\
1368	-135.045814831101\\
1369	-158.986316553142\\
1370	-165.707847303161\\
1371	-120.51983980323\\
1372	-91.6003992463106\\
1373	-95.4030962282002\\
1374	-90.1229835441814\\
1375	-104.609080767116\\
1376	-121.194169411157\\
1377	-119.432575420174\\
1378	-162.371278684482\\
1379	-182.348830102669\\
1380	-212.896510393907\\
1381	-147.388784125922\\
1382	-173.41726206602\\
1383	-174.138540426379\\
1384	-146.471114989037\\
1385	-162.924244201796\\
1386	-137.825709673518\\
1387	-79.6259383271611\\
1388	-47.2710559562927\\
1389	-51.6809773636371\\
1390	-79.7154621249872\\
1391	-68.1383183066716\\
1392	-46.4408036348241\\
1393	-65.5181047627093\\
1394	-48.4939614961245\\
1395	-37.3296604982213\\
1396	-46.5951047044343\\
1397	-49.5649767895009\\
1398	-41.8288305553344\\
1400	-93.853474375097\\
1401	-72.584873560156\\
1403	-176.675996975104\\
1404	-156.954141528294\\
1405	-195.175782443863\\
};
\addlegendentry{OSA predition}

\addplot [color=mycolor3, dotted, line width=2.0pt]
  table[row sep=crcr]{%
1006	-125.732\\
1007	-153.809\\
1008	-115.967\\
1009	-61.0350000000001\\
1010	-72.3788963339414\\
1011	-72.1910258293001\\
1012	-91.8732467146499\\
1013	-72.8176863041749\\
1014	-27.4399196704139\\
1015	-25.3329781314335\\
1016	-22.287117408325\\
1017	-62.3119600621367\\
1018	-96.7479300521532\\
1019	-105.924917282651\\
1020	-107.697238097631\\
1021	-81.1522817356833\\
1022	-47.5838698568871\\
1023	-108.566334454978\\
1024	-75.637189661508\\
1025	-59.7109426912969\\
1026	-100.754613859637\\
1028	-73.3801505394597\\
1029	-117.859233828287\\
1030	-105.70471808449\\
1031	-83.5253888433147\\
1032	-115.543927948748\\
1033	-114.124234212662\\
1034	-95.2288574509666\\
1035	-78.9299152688136\\
1036	-64.6114973568358\\
1038	-88.1294119863869\\
1039	-87.7801834122331\\
1040	-87.0300449702154\\
1041	-90.8529977470243\\
1042	-93.0065782654351\\
1043	-131.401731537255\\
1044	-121.301134779581\\
1045	-86.0401302480757\\
1046	-77.587819899614\\
1047	-44.4305604535791\\
1048	-46.3489772432613\\
1049	-65.9292264534799\\
1050	-52.9446792320753\\
1051	-53.7415340906136\\
1052	-76.604903732964\\
1053	-85.27749996532\\
1054	-80.6669390440136\\
1055	-126.225577796557\\
1056	-94.3055311550613\\
1057	-57.2166015767421\\
1058	-47.7920387529771\\
1060	-74.0617103859549\\
1061	-38.0655669844828\\
1062	-41.6671741098312\\
1063	-57.7479380125906\\
1065	-40.4256900248683\\
1066	-54.8899141788108\\
1067	-58.2409422322692\\
1068	-98.2967341203312\\
1069	-85.0676205281957\\
1070	-107.197607820813\\
1071	-97.3153969341543\\
1072	-107.155569722283\\
1073	-90.5546680950454\\
1074	-87.8735889588513\\
1075	-79.5751295350983\\
1076	-79.8063684163101\\
1077	-76.0941471357712\\
1078	-122.253016874409\\
1079	-176.798854714503\\
1080	-184.332404537266\\
1081	-177.070065547869\\
1082	-109.411304716342\\
1083	-165.666472525645\\
1084	-194.259586129728\\
1085	-201.585215406673\\
1086	-160.119750734127\\
1087	-201.206866220495\\
1088	-267.660479805598\\
1089	-192.46832860712\\
1090	-166.407633755924\\
1091	-98.5352284143726\\
1092	-83.0419046444333\\
1093	-64.8067587702433\\
1094	-84.8475537524853\\
1095	-56.2013786739712\\
1096	-34.3138304896438\\
1097	-30.2334494169097\\
1098	-47.8940242595158\\
1099	-75.2811049150505\\
1100	-71.1187980216509\\
1101	-74.281284902743\\
1102	-97.1068623147703\\
1103	-104.650832676948\\
1104	-142.033061530103\\
1105	-126.592463900745\\
1106	-139.784596976579\\
1107	-108.130900143281\\
1108	-93.2160695117007\\
1109	-107.707239279719\\
1110	-82.3119207290263\\
1111	-74.4653276963138\\
1112	-90.6029758612949\\
1113	-90.7460312891558\\
1114	-68.8886820716245\\
1115	-71.1422263954134\\
1116	-52.7753526981098\\
1117	-55.0218854698578\\
1118	-63.0288952682754\\
1119	-48.3144695993838\\
1120	-56.345164321194\\
1121	-70.5330715527512\\
1122	-108.924271494029\\
1123	-102.420174907934\\
1124	-119.954686336354\\
1125	-65.4616530025878\\
1126	-100.178141948084\\
1127	-149.391630330301\\
1128	-120.772220343759\\
1129	-131.684164015474\\
1130	-126.866673014214\\
1131	-81.726448514486\\
1132	-53.1471947795587\\
1133	-76.9032284044831\\
1134	-86.1741728922539\\
1135	-138.641051372045\\
1136	-168.868303501582\\
1137	-166.920931180302\\
1138	-117.948125567958\\
1139	-135.696539497114\\
1140	-116.367821186727\\
1141	-115.907899417406\\
1142	-114.652685239407\\
1143	-76.1941237785331\\
1144	-70.4625587030901\\
1145	-61.4607020986455\\
1146	-57.7049550639399\\
1147	-66.2210325476294\\
1148	-95.8139634040383\\
1149	-136.28700139978\\
1150	-121.631557520897\\
1151	-78.0593226710489\\
1152	-71.2207163094108\\
1153	-66.5361095207752\\
1154	-40.678454510904\\
1155	-28.2147415604688\\
1156	-35.2254935519293\\
1157	-61.973152064528\\
1158	-50.1501239733423\\
1159	-55.7591256923588\\
1160	-59.5662790416177\\
1161	-59.338184261431\\
1162	-40.8257611925621\\
1164	-24.0374888368626\\
1165	-37.7025480322161\\
1166	-83.9634060740341\\
1167	-112.043131205706\\
1168	-126.452301110381\\
1169	-86.8433855084816\\
1170	-59.0810748965407\\
1171	-45.756076532032\\
1172	-33.9144403007804\\
1173	-66.5679929576831\\
1174	-65.4012026643736\\
1175	-89.1483005615451\\
1176	-115.985297904745\\
1177	-190.660782737652\\
1178	-223.080108576178\\
1179	-187.024910271582\\
1180	-168.957477680418\\
1181	-109.749450612355\\
1182	-125.531363034432\\
1183	-134.927935117463\\
1184	-143.236284511048\\
1185	-115.233450146393\\
1186	-97.1144321685008\\
1187	-105.956060341002\\
1188	-96.3514772955943\\
1189	-107.357081126885\\
1190	-81.7788195749208\\
1191	-131.975732708376\\
1192	-163.657100030572\\
1193	-116.362526431902\\
1194	-74.9706867588932\\
1195	-72.9998432949153\\
1196	-122.450960611142\\
1197	-152.415486389718\\
1198	-176.558599082716\\
1199	-191.466075986021\\
1200	-197.52665477312\\
1201	-146.233227518739\\
1202	-141.033962246575\\
1203	-154.602462674163\\
1204	-180.444594753555\\
1205	-101.807223565511\\
1206	-67.2426752249357\\
1207	-100.50647710364\\
1208	-87.0022538625735\\
1209	-52.2639949546767\\
1210	-71.8393025969458\\
1211	-84.9067709131218\\
1212	-69.9414956291655\\
1213	-59.8754824819339\\
1214	-73.9207674502743\\
1215	-65.3571719211777\\
1216	-78.6350884011219\\
1217	-119.369257757821\\
1218	-105.932354520085\\
1219	-79.5794014229921\\
1220	-76.9679137920057\\
1221	-113.300778398441\\
1222	-180.683665015234\\
1223	-145.224100257132\\
1224	-81.6297349850392\\
1225	-69.3564816268217\\
1226	-74.9846944459639\\
1227	-74.7268096657158\\
1228	-50.6157927181155\\
1230	-69.516392597186\\
1231	-86.5629083787933\\
1232	-99.2605894934281\\
1233	-69.8627558255466\\
1234	-109.963275040844\\
1235	-132.853089438921\\
1236	-120.752403926128\\
1237	-96.7158532955982\\
1238	-104.908278241836\\
1239	-135.032762726694\\
1240	-101.811434105769\\
1241	-37.8076771937601\\
1242	-54.5978021350288\\
1243	-64.1463584541762\\
1244	-88.1567617463288\\
1245	-77.9760139260356\\
1246	-61.0547879663675\\
1247	-90.1204656243001\\
1248	-82.7971125024033\\
1249	-61.6228051975684\\
1250	-80.0801733881619\\
1251	-70.299547815988\\
1252	-62.5647915738107\\
1253	-75.8291164601956\\
1254	-46.9610657364117\\
1255	-54.8810410080614\\
1256	-39.9020627604816\\
1257	-44.8469061786859\\
1258	-82.3635310839431\\
1259	-85.1085668360624\\
1260	-113.030714246842\\
1261	-152.677160912486\\
1262	-104.497018029193\\
1263	-62.2469580914189\\
1264	-49.0525157667294\\
1265	-48.5119690811641\\
1266	-72.4979956561756\\
1267	-46.6206298468514\\
1268	-50.0966164107676\\
1269	-63.8546891541314\\
1270	-106.646170862544\\
1271	-97.3246064529098\\
1272	-137.647496355532\\
1273	-95.2005616120928\\
1274	-81.923387005879\\
1275	-42.5060381734472\\
1276	-45.9781718058093\\
1277	-28.7325928580467\\
1278	-34.6521548018329\\
1279	-38.6991050795427\\
1280	-66.3550595182405\\
1281	-71.298200412976\\
1282	-77.5760705301486\\
1283	-82.9613865561687\\
1284	-126.26585409441\\
1285	-124.793219736862\\
1286	-88.6612357650736\\
1287	-103.423441167956\\
1288	-113.248929410278\\
1289	-140.401082009914\\
1290	-113.344139996222\\
1291	-77.0430667724613\\
1292	-36.2939058464476\\
1293	-33.4301718304798\\
1294	-31.2240775005118\\
1295	-39.5652983414179\\
1296	-39.9765667775644\\
1297	-38.7334166814042\\
1298	-46.9575886954019\\
1299	-79.3928309353432\\
1300	-72.2735635816362\\
1301	-69.4463189002809\\
1302	-81.7503744643243\\
1303	-53.2962769394596\\
1304	-32.4413927349981\\
1305	-58.1020268494906\\
1306	-88.8999121643371\\
1307	-83.8638171754085\\
1308	-97.6497858532359\\
1310	-61.9324385252435\\
1311	-82.2617625545142\\
1312	-111.880152364373\\
1313	-113.338239394076\\
1314	-82.4978752172537\\
1315	-133.136680748439\\
1316	-103.557335583843\\
1317	-96.6584428736435\\
1318	-118.623186023527\\
1319	-113.026437978775\\
1320	-90.9368905332028\\
1321	-81.6056496961921\\
1322	-125.452203801892\\
1323	-176.090983315827\\
1324	-137.205404046497\\
1325	-77.1267952271553\\
1326	-73.0158158353715\\
1327	-82.7429964410976\\
1329	-108.024932902812\\
1330	-85.1849445838016\\
1331	-94.6302568684816\\
1332	-118.077517473884\\
1333	-122.419025327338\\
1334	-85.965393618259\\
1335	-75.6705848919355\\
1336	-93.4776870608521\\
1337	-157.877780682203\\
1338	-137.46308677095\\
1339	-136.011841653553\\
1340	-80.4348627203058\\
1341	-74.2558617399031\\
1342	-71.234678195108\\
1343	-46.5025558198138\\
1344	-26.9075935350897\\
1345	-22.868825800517\\
1346	-20.4135636890785\\
1347	-48.5391907220176\\
1348	-69.271002554322\\
1349	-85.249470445983\\
1350	-65.9888387653289\\
1351	-69.1587525202144\\
1352	-81.0143405359513\\
1353	-51.078127794008\\
1354	-54.7735706188585\\
1355	-54.0299259102856\\
1356	-41.7929236905818\\
1357	-52.9101636106768\\
1358	-81.8415034567504\\
1359	-102.746955507418\\
1360	-61.6708046297053\\
1361	-49.0052236571803\\
1362	-44.7622107010047\\
1363	-55.4685753783851\\
1364	-35.6925227402617\\
1366	-130.127369671749\\
1367	-108.72396615863\\
1368	-136.193314124427\\
1369	-159.987419883211\\
1370	-164.702180143391\\
1371	-119.883209634501\\
1372	-90.5814712836136\\
1373	-93.5828586630475\\
1374	-90.893865354034\\
1376	-122.353721751802\\
1377	-119.260680848866\\
1378	-160.111489116406\\
1379	-178.249713038798\\
1380	-207.944425628685\\
1381	-143.985235702909\\
1382	-167.5069813562\\
1383	-171.542314516932\\
1384	-144.44881225497\\
1385	-158.643960260941\\
1386	-138.321540460171\\
1387	-77.0587304368225\\
1388	-46.6289755189737\\
1389	-48.8448894712944\\
1390	-75.6713771748196\\
1391	-63.9242831719964\\
1392	-43.7672026088715\\
1393	-63.9126071647686\\
1394	-48.3707187979192\\
1395	-37.9294586122228\\
1396	-47.2134298006463\\
1397	-50.3236593902047\\
1398	-41.6966840048667\\
1399	-69.8543086024576\\
1400	-95.0017476061732\\
1401	-72.7197057453377\\
1403	-180.457757551776\\
1404	-163.056569457911\\
1405	-204.575085460168\\
};
\addlegendentry{MPO prediction}

\end{axis}

\begin{axis}[%
width=6.159cm,
height=1.831cm,
at={(0cm,0cm)},
scale only axis,
xmin=1000,
xmax=1405,
xlabel style={font=\color{white!15!black}},
xlabel={Sample index},
ymin=-200,
ymax=0,
ylabel style={font=\color{white!15!black}},
ylabel={$y(t)$},
axis background/.style={fill=white},
title style={font=\bfseries},
title={C9: RMSE(OSA) = 3.0702, RMSE(MPO) = 3.593},
legend style={legend cell align=left, align=left, draw=white!15!black}
]
\addplot [color=mycolor1, line width=2.0pt]
  table[row sep=crcr]{%
1006	-86.6700000000001\\
1007	-103.76\\
1008	-79.346\\
1009	-45.1659999999999\\
1010	-57.373\\
1011	-51.27\\
1012	-63.4770000000001\\
1013	-48.828\\
1014	-24.414\\
1015	-17.0899999999999\\
1016	-17.0899999999999\\
1017	-43.9449999999999\\
1018	-73.242\\
1019	-76.904\\
1020	-79.346\\
1022	-36.6210000000001\\
1023	-76.904\\
1024	-53.711\\
1025	-43.9449999999999\\
1026	-69.5799999999999\\
1027	-61.0350000000001\\
1028	-48.828\\
1029	-84.229\\
1030	-79.346\\
1031	-59.8140000000001\\
1032	-87.8910000000001\\
1033	-85.4490000000001\\
1034	-67.1389999999999\\
1035	-54.932\\
1036	-46.3869999999999\\
1037	-56.152\\
1038	-64.6970000000001\\
1039	-64.6970000000001\\
1040	-67.1389999999999\\
1041	-68.3589999999999\\
1042	-73.242\\
1043	-100.098\\
1044	-87.8910000000001\\
1045	-61.0350000000001\\
1046	-56.152\\
1047	-34.1800000000001\\
1048	-34.1800000000001\\
1049	-48.828\\
1050	-37.8420000000001\\
1051	-39.0630000000001\\
1052	-56.152\\
1053	-62.2560000000001\\
1054	-58.5940000000001\\
1055	-90.3320000000001\\
1056	-64.6970000000001\\
1057	-41.5039999999999\\
1058	-34.1800000000001\\
1059	-45.1659999999999\\
1060	-51.27\\
1061	-28.076\\
1062	-25.635\\
1063	-41.5039999999999\\
1064	-34.1800000000001\\
1065	-29.297\\
1066	-40.2829999999999\\
1067	-43.9449999999999\\
1068	-68.3589999999999\\
1069	-63.4770000000001\\
1070	-79.346\\
1071	-69.5799999999999\\
1072	-79.346\\
1073	-63.4770000000001\\
1074	-62.2560000000001\\
1075	-54.932\\
1076	-53.711\\
1077	-53.711\\
1079	-126.953\\
1080	-131.836\\
1081	-129.395\\
1082	-83.008\\
1083	-114.746\\
1084	-141.602\\
1085	-147.705\\
1086	-111.084\\
1087	-146.484\\
1088	-185.547\\
1089	-131.836\\
1090	-112.305\\
1091	-75.684\\
1092	-61.0350000000001\\
1093	-52.49\\
1094	-64.6970000000001\\
1095	-46.3869999999999\\
1096	-31.7380000000001\\
1097	-31.7380000000001\\
1098	-40.2829999999999\\
1099	-57.373\\
1100	-52.49\\
1101	-53.711\\
1102	-72.021\\
1103	-72.021\\
1104	-97.6559999999999\\
1105	-79.346\\
1106	-101.318\\
1107	-72.021\\
1108	-64.6970000000001\\
1109	-73.242\\
1110	-54.932\\
1111	-51.27\\
1112	-59.8140000000001\\
1113	-62.2560000000001\\
1114	-43.9449999999999\\
1115	-47.607\\
1116	-34.1800000000001\\
1117	-40.2829999999999\\
1118	-42.7249999999999\\
1119	-30.518\\
1121	-47.607\\
1122	-72.021\\
1123	-74.463\\
1124	-81.787\\
1125	-48.828\\
1126	-70.8009999999999\\
1127	-101.318\\
1128	-84.229\\
1129	-93.9939999999999\\
1130	-91.5530000000001\\
1131	-54.932\\
1132	-40.2829999999999\\
1133	-56.152\\
1134	-61.0350000000001\\
1135	-97.6559999999999\\
1136	-119.629\\
1137	-117.188\\
1138	-89.1110000000001\\
1139	-92.7729999999999\\
1140	-81.787\\
1142	-79.346\\
1143	-54.932\\
1144	-48.828\\
1145	-45.1659999999999\\
1146	-40.2829999999999\\
1147	-45.1659999999999\\
1148	-64.6970000000001\\
1149	-96.4359999999999\\
1151	-52.49\\
1152	-45.1659999999999\\
1153	-46.3869999999999\\
1154	-29.297\\
1155	-23.193\\
1156	-26.855\\
1157	-46.3869999999999\\
1158	-35.4000000000001\\
1159	-41.5039999999999\\
1161	-39.0630000000001\\
1162	-28.076\\
1163	-21.973\\
1164	-17.0899999999999\\
1165	-29.297\\
1166	-58.5940000000001\\
1167	-80.566\\
1168	-91.5530000000001\\
1169	-59.8140000000001\\
1170	-43.9449999999999\\
1171	-32.9590000000001\\
1172	-23.193\\
1173	-47.607\\
1174	-46.3869999999999\\
1175	-67.1389999999999\\
1176	-81.787\\
1177	-131.836\\
1178	-147.705\\
1179	-120.85\\
1180	-117.188\\
1181	-78.125\\
1182	-81.787\\
1183	-91.5530000000001\\
1184	-98.877\\
1185	-73.242\\
1186	-68.3589999999999\\
1187	-69.5799999999999\\
1188	-62.2560000000001\\
1189	-75.684\\
1190	-53.711\\
1191	-93.9939999999999\\
1192	-108.643\\
1194	-51.27\\
1195	-54.932\\
1196	-92.7729999999999\\
1197	-109.863\\
1198	-134.277\\
1199	-140.381\\
1200	-140.381\\
1201	-106.201\\
1202	-96.4359999999999\\
1203	-111.084\\
1204	-129.395\\
1205	-78.125\\
1206	-50.049\\
1207	-73.242\\
1208	-57.373\\
1209	-39.0630000000001\\
1210	-56.152\\
1211	-61.0350000000001\\
1212	-46.3869999999999\\
1213	-37.8420000000001\\
1214	-50.049\\
1215	-39.0630000000001\\
1216	-53.711\\
1217	-76.904\\
1218	-72.021\\
1219	-51.27\\
1220	-51.27\\
1221	-78.125\\
1222	-123.291\\
1223	-86.6700000000001\\
1224	-56.152\\
1225	-43.9449999999999\\
1226	-48.828\\
1227	-43.9449999999999\\
1228	-34.1800000000001\\
1229	-37.8420000000001\\
1230	-47.607\\
1231	-59.8140000000001\\
1232	-64.6970000000001\\
1233	-46.3869999999999\\
1234	-74.463\\
1235	-85.4490000000001\\
1236	-83.008\\
1237	-65.9180000000001\\
1238	-73.242\\
1239	-96.4359999999999\\
1240	-61.0350000000001\\
1241	-29.297\\
1242	-42.7249999999999\\
1243	-43.9449999999999\\
1244	-58.5940000000001\\
1245	-50.049\\
1246	-40.2829999999999\\
1247	-58.5940000000001\\
1248	-54.932\\
1249	-39.0630000000001\\
1250	-48.828\\
1252	-39.0630000000001\\
1253	-50.049\\
1254	-29.297\\
1255	-36.6210000000001\\
1256	-26.855\\
1257	-31.7380000000001\\
1258	-53.711\\
1259	-61.0350000000001\\
1260	-76.904\\
1261	-101.318\\
1262	-67.1389999999999\\
1263	-42.7249999999999\\
1264	-32.9590000000001\\
1265	-31.7380000000001\\
1266	-47.607\\
1267	-30.518\\
1268	-36.6210000000001\\
1269	-43.9449999999999\\
1270	-65.9180000000001\\
1271	-62.2560000000001\\
1272	-86.6700000000001\\
1273	-59.8140000000001\\
1274	-56.152\\
1275	-31.7380000000001\\
1276	-32.9590000000001\\
1277	-20.752\\
1278	-24.414\\
1279	-26.855\\
1280	-47.607\\
1281	-47.607\\
1282	-56.152\\
1283	-59.8140000000001\\
1284	-93.9939999999999\\
1285	-79.346\\
1286	-62.2560000000001\\
1287	-73.242\\
1288	-79.346\\
1289	-101.318\\
1290	-80.566\\
1291	-52.49\\
1292	-30.518\\
1293	-21.973\\
1294	-23.193\\
1295	-29.297\\
1296	-29.297\\
1297	-26.855\\
1298	-36.6210000000001\\
1299	-53.711\\
1300	-47.607\\
1301	-51.27\\
1302	-57.373\\
1303	-37.8420000000001\\
1304	-20.752\\
1306	-63.4770000000001\\
1307	-61.0350000000001\\
1308	-69.5799999999999\\
1309	-58.5940000000001\\
1310	-45.1659999999999\\
1311	-59.8140000000001\\
1312	-84.229\\
1313	-85.4490000000001\\
1314	-59.8140000000001\\
1315	-102.539\\
1316	-76.904\\
1317	-74.463\\
1318	-84.229\\
1319	-83.008\\
1320	-63.4770000000001\\
1321	-54.932\\
1323	-124.512\\
1324	-95.2149999999999\\
1325	-57.373\\
1326	-53.711\\
1327	-54.932\\
1328	-72.021\\
1329	-83.008\\
1330	-59.8140000000001\\
1331	-64.6970000000001\\
1332	-83.008\\
1333	-90.3320000000001\\
1334	-61.0350000000001\\
1335	-50.049\\
1336	-63.4770000000001\\
1337	-104.98\\
1338	-92.7729999999999\\
1339	-95.2149999999999\\
1340	-58.5940000000001\\
1341	-53.711\\
1342	-51.27\\
1343	-34.1800000000001\\
1344	-23.193\\
1345	-20.752\\
1346	-17.0899999999999\\
1347	-40.2829999999999\\
1348	-54.932\\
1349	-62.2560000000001\\
1350	-47.607\\
1352	-57.373\\
1353	-37.8420000000001\\
1354	-39.0630000000001\\
1355	-39.0630000000001\\
1356	-28.076\\
1357	-39.0630000000001\\
1358	-57.373\\
1359	-73.242\\
1360	-48.828\\
1361	-35.4000000000001\\
1362	-31.7380000000001\\
1363	-36.6210000000001\\
1364	-26.855\\
1365	-54.932\\
1366	-93.9939999999999\\
1367	-72.021\\
1368	-97.6559999999999\\
1369	-113.525\\
1370	-114.746\\
1371	-85.4490000000001\\
1372	-63.4770000000001\\
1374	-63.4770000000001\\
1375	-75.684\\
1376	-89.1110000000001\\
1377	-86.6700000000001\\
1378	-115.967\\
1379	-125.732\\
1380	-150.146\\
1381	-96.4359999999999\\
1382	-109.863\\
1383	-122.07\\
1384	-92.7729999999999\\
1385	-107.422\\
1386	-92.7729999999999\\
1387	-54.932\\
1388	-39.0630000000001\\
1389	-40.2829999999999\\
1390	-57.373\\
1391	-42.7249999999999\\
1392	-32.9590000000001\\
1393	-45.1659999999999\\
1394	-34.1800000000001\\
1395	-26.855\\
1396	-34.1800000000001\\
1397	-37.8420000000001\\
1398	-25.635\\
1399	-48.828\\
1400	-64.6970000000001\\
1401	-46.3869999999999\\
1403	-115.967\\
1404	-106.201\\
1405	-133.057\\
};
\addlegendentry{True output}

\addplot [color=mycolor2, dashed, line width=2.0pt]
  table[row sep=crcr]{%
1006	-85.587535083137\\
1007	-103.446388886553\\
1008	-79.3093233003451\\
1009	-43.000243339997\\
1010	-50.664046363177\\
1011	-53.0319049845652\\
1012	-63.911156782998\\
1013	-56.3272253347911\\
1014	-18.7213987728378\\
1015	-19.8775879843879\\
1016	-18.0184345312612\\
1017	-47.0708929546763\\
1018	-67.6671673556446\\
1019	-77.2986178889901\\
1020	-77.8748904992831\\
1021	-59.2117241578303\\
1022	-36.8052378539992\\
1023	-76.0019760954226\\
1024	-55.454063696264\\
1025	-41.7280021248084\\
1026	-67.4012934034795\\
1027	-61.5858346135415\\
1028	-52.9180190895445\\
1029	-84.7067673582408\\
1030	-78.0615267545163\\
1031	-58.3502341818553\\
1032	-81.7946758410026\\
1033	-85.7246604387735\\
1034	-65.7432175296374\\
1035	-59.8366707497496\\
1036	-45.0802864229756\\
1038	-62.293325143936\\
1039	-63.0044117359516\\
1040	-62.7554087253993\\
1041	-66.1817131783398\\
1042	-74.0233007649124\\
1043	-94.9842336240708\\
1044	-90.5095424797055\\
1045	-58.5398426486115\\
1046	-60.8393528497247\\
1047	-33.4957912985437\\
1048	-33.8553613410904\\
1049	-45.0282370270731\\
1050	-38.0167325646369\\
1051	-42.3224250346764\\
1052	-51.2543873456193\\
1053	-62.843211560536\\
1054	-58.0364492034007\\
1055	-86.7953487454902\\
1056	-69.7984319591526\\
1057	-43.3714300172981\\
1058	-35.0140960636202\\
1059	-41.8782314091955\\
1060	-50.1164146500789\\
1061	-28.9109500868499\\
1062	-30.830309279595\\
1063	-39.2601622542561\\
1064	-33.3755181006497\\
1065	-28.4484653605712\\
1066	-38.4157084590349\\
1067	-43.6793693814075\\
1068	-66.6422262101648\\
1069	-65.2397751960989\\
1070	-74.171252375105\\
1071	-70.4639108167221\\
1072	-76.2935795348556\\
1073	-62.7236992542134\\
1074	-62.3849030275885\\
1075	-56.160688606142\\
1076	-55.9068538767137\\
1077	-54.8784510291225\\
1078	-87.9750818496689\\
1079	-126.093337127038\\
1080	-128.210152392728\\
1081	-124.071999712553\\
1082	-80.582916474754\\
1083	-113.111662117789\\
1084	-139.474104336113\\
1085	-136.675186449817\\
1086	-104.48436685978\\
1087	-145.303316998764\\
1088	-178.72852795287\\
1089	-137.512583955425\\
1090	-111.739034658081\\
1091	-76.0666237904695\\
1092	-60.830724267822\\
1093	-52.214105260658\\
1094	-62.4646559757207\\
1096	-27.4593570613144\\
1097	-27.6620207222259\\
1098	-36.1315609972773\\
1099	-60.627501362625\\
1100	-54.5827548039933\\
1101	-57.9648769739638\\
1102	-68.3566560716745\\
1103	-72.1575604788122\\
1104	-98.8168989094177\\
1105	-91.0729350503098\\
1106	-95.8254993122914\\
1107	-77.3595305146305\\
1108	-63.5526743760756\\
1109	-76.1617983609606\\
1110	-55.855441636834\\
1111	-50.2045379588837\\
1112	-61.5173490345651\\
1113	-62.3623753939878\\
1114	-42.1122865931741\\
1115	-48.8933841459339\\
1116	-37.7284958437674\\
1117	-39.480518823801\\
1118	-42.1014585972482\\
1119	-31.9417324946594\\
1120	-37.253366223061\\
1121	-47.685856277315\\
1122	-74.6257891928935\\
1123	-75.8554380829505\\
1124	-77.9955115816872\\
1125	-51.7884884830828\\
1126	-67.3243728959544\\
1127	-108.03363098301\\
1128	-82.0850529614274\\
1129	-93.6585863176779\\
1130	-89.8606336822045\\
1131	-56.7050810861751\\
1132	-40.979866099528\\
1133	-54.0502752746706\\
1134	-62.8100346140238\\
1135	-96.1919794995895\\
1136	-118.786212712364\\
1137	-112.493755342286\\
1138	-93.9661560613151\\
1139	-91.5847020058654\\
1140	-86.8215482546418\\
1141	-81.4595537220425\\
1142	-77.838226093691\\
1143	-53.3604362822823\\
1144	-51.7589317743018\\
1145	-44.9650981125137\\
1146	-41.1013924550384\\
1147	-47.4598998114855\\
1148	-62.1916099375062\\
1149	-96.578747574463\\
1150	-78.1581779451958\\
1151	-54.8479419643113\\
1152	-46.8404926568744\\
1153	-47.2151482033273\\
1154	-27.2958323976018\\
1155	-20.0136463181377\\
1156	-24.0238259052294\\
1157	-46.2957049464262\\
1158	-35.1902217509573\\
1159	-39.7589765713471\\
1160	-40.2404442342965\\
1161	-41.2506925812029\\
1162	-32.7546404529571\\
1163	-21.5846300623587\\
1164	-16.8670540677426\\
1165	-27.0392687371241\\
1166	-58.8152029927271\\
1167	-81.3744583174928\\
1168	-90.122692012475\\
1169	-59.9823479759207\\
1170	-44.9216921434424\\
1171	-34.4103213224782\\
1172	-25.9484904961412\\
1173	-46.4606009315555\\
1174	-46.4687363515081\\
1175	-63.1522132900852\\
1176	-82.5268011784326\\
1177	-135.313682369927\\
1178	-156.482404771228\\
1179	-124.21988404521\\
1180	-121.212038872265\\
1181	-74.5047142981668\\
1182	-85.837525559527\\
1183	-88.9260765469792\\
1184	-95.5881933117828\\
1185	-78.990807084062\\
1186	-68.882497146662\\
1187	-68.9965333470984\\
1188	-65.7448346043982\\
1189	-74.2249376268919\\
1190	-56.8096875995261\\
1191	-90.1725674110614\\
1192	-109.221290634248\\
1193	-83.1125465524508\\
1194	-49.149610068014\\
1195	-55.5046681493527\\
1196	-89.8466542503129\\
1197	-108.821347099474\\
1198	-135.310409894752\\
1199	-135.906535665664\\
1200	-138.136114740253\\
1201	-109.705284371177\\
1202	-94.9714669198358\\
1203	-111.339839374529\\
1204	-121.78570167618\\
1205	-73.5359666491106\\
1206	-53.5886408395245\\
1207	-68.3425931789211\\
1208	-65.1586872551275\\
1209	-39.7412050162645\\
1210	-50.8516682193217\\
1211	-59.5126374745178\\
1212	-53.1811317278762\\
1213	-38.2442988342589\\
1214	-51.6123245303429\\
1215	-44.6162526443097\\
1216	-52.6979738522691\\
1217	-77.4167631660086\\
1218	-71.2298063544395\\
1219	-52.4746075963719\\
1220	-54.2435493082974\\
1221	-76.6670814500383\\
1222	-128.244674344346\\
1223	-97.9689439484848\\
1224	-54.1145722210872\\
1225	-41.1444718380696\\
1226	-50.9656510163625\\
1227	-46.6843110333775\\
1228	-36.135220916655\\
1229	-39.091555118395\\
1230	-47.0946313506963\\
1231	-58.6476091844065\\
1232	-65.93445391259\\
1233	-47.1100415725493\\
1234	-79.0796062550505\\
1235	-89.3415173360454\\
1236	-81.7963500953686\\
1237	-67.8822573533719\\
1238	-69.7235588986134\\
1239	-95.3972142612638\\
1240	-66.4419405992865\\
1241	-27.2472129582982\\
1242	-36.7103678082892\\
1243	-44.919524435529\\
1244	-64.5094414829944\\
1245	-52.3239063373808\\
1246	-38.5962046784525\\
1247	-58.3609008731962\\
1248	-57.3476431357267\\
1249	-41.0536869387474\\
1250	-53.144849886052\\
1251	-46.1433503629155\\
1252	-44.0362136029519\\
1253	-46.0727339505954\\
1254	-30.321283448596\\
1255	-35.2074866853509\\
1256	-26.7098854361864\\
1257	-31.4269127614241\\
1258	-54.9454970615398\\
1259	-60.9719026305431\\
1260	-78.690787676946\\
1261	-106.312424371496\\
1262	-71.8314509402564\\
1263	-42.0441909528759\\
1264	-34.0239193858254\\
1265	-33.2964097553968\\
1266	-46.9462045494749\\
1267	-31.0838140588744\\
1268	-34.6525161799173\\
1269	-44.3947275064932\\
1270	-69.8871412641754\\
1271	-64.9332181560922\\
1272	-86.9586444877664\\
1273	-59.041078453175\\
1274	-55.4719909376747\\
1275	-30.2343775499339\\
1276	-30.5539330214465\\
1277	-22.6139203745479\\
1278	-23.3088295776254\\
1279	-28.1863116352883\\
1280	-45.7667878770985\\
1281	-51.8206425526412\\
1282	-54.0857020860626\\
1283	-60.1920155261969\\
1284	-95.2752906018295\\
1285	-88.2460633174039\\
1286	-62.8464934646895\\
1287	-73.1475151524185\\
1288	-74.5077161646125\\
1289	-96.4684689239132\\
1290	-82.8705585201221\\
1291	-54.048265663585\\
1292	-29.1628521904147\\
1293	-26.6294047369711\\
1294	-23.7089306958139\\
1295	-26.67941209992\\
1296	-28.9056333210458\\
1297	-27.7977516586086\\
1298	-35.2625457253812\\
1299	-55.5807672632513\\
1300	-49.6143261314073\\
1301	-49.090403161948\\
1302	-57.0125677281035\\
1303	-38.609037956056\\
1304	-23.3750720670407\\
1305	-39.8752710157603\\
1306	-62.3206192401426\\
1307	-59.796585038712\\
1308	-66.5593079072999\\
1309	-59.3793318141306\\
1310	-45.0279029124331\\
1311	-61.1572568280603\\
1312	-83.5097510710655\\
1313	-80.6260088386348\\
1314	-58.4282517971133\\
1315	-103.971532791087\\
1316	-79.1708525411852\\
1317	-76.1141512426595\\
1318	-82.4086313884786\\
1319	-82.5080257838754\\
1320	-62.9381883807155\\
1321	-58.7996672057964\\
1322	-87.0831096263432\\
1323	-123.56210822175\\
1324	-93.6544351715984\\
1325	-55.0942267997611\\
1326	-55.2928558253971\\
1327	-55.9846727326876\\
1328	-71.1343529379922\\
1329	-80.6187823483979\\
1330	-58.908128407433\\
1331	-70.2660291557752\\
1332	-83.7759012107736\\
1333	-83.4119811363842\\
1334	-60.8877329234926\\
1335	-49.953119426541\\
1336	-65.7473877342018\\
1337	-104.447971672795\\
1338	-96.3098345997653\\
1339	-91.9065542199842\\
1340	-57.2278637606519\\
1341	-52.579549369596\\
1342	-56.110009843798\\
1343	-33.087481354921\\
1344	-20.369138031524\\
1345	-18.2053083891064\\
1346	-17.0712662202723\\
1347	-38.47865944968\\
1348	-53.2242762308094\\
1349	-62.9255437965048\\
1350	-49.1835380250452\\
1351	-50.0558463874747\\
1352	-58.5436401029697\\
1353	-37.8767771400228\\
1354	-39.0692333931545\\
1355	-39.2108648086612\\
1356	-31.6751838707123\\
1357	-34.2511440106005\\
1358	-58.8749691771302\\
1359	-72.6041391378869\\
1360	-46.1090278732345\\
1361	-36.6673928496666\\
1362	-32.7544403614686\\
1363	-39.3022692777281\\
1364	-25.375568035686\\
1365	-58.8309113046168\\
1366	-94.9644384189803\\
1367	-72.691929345252\\
1368	-97.422071777396\\
1369	-110.901293877675\\
1370	-112.30109536893\\
1371	-87.4745567713394\\
1372	-65.8023844763188\\
1373	-64.3576568120336\\
1374	-65.6309351071236\\
1376	-83.0889946863908\\
1377	-86.823329932357\\
1378	-113.998780149798\\
1379	-123.09859564089\\
1380	-145.003648101382\\
1381	-97.5932110832557\\
1383	-117.145578097926\\
1384	-98.0213939857842\\
1385	-111.301452671045\\
1386	-90.1936841592064\\
1387	-53.6046372808078\\
1388	-34.5091627854492\\
1389	-37.9359722378065\\
1390	-56.2513145754911\\
1391	-50.5326582602058\\
1392	-34.2762187205624\\
1393	-41.7505191167083\\
1394	-36.448709276568\\
1395	-25.0123060883436\\
1396	-32.9486602931861\\
1397	-37.1696709910777\\
1398	-30.4423807756723\\
1399	-49.2411509572823\\
1400	-66.3671681835144\\
1401	-53.0809559052832\\
1402	-86.1773535650907\\
1403	-124.312466988419\\
1404	-107.539339631597\\
1405	-126.041385468434\\
};
\addlegendentry{OSA predition}

\addplot [color=mycolor3, dotted, line width=2.0pt]
  table[row sep=crcr]{%
1006	-86.6700000000001\\
1007	-103.76\\
1008	-79.346\\
1009	-45.1659999999999\\
1010	-50.6640463631813\\
1011	-51.6874789838469\\
1012	-61.9198899965184\\
1013	-55.6449847156277\\
1014	-19.1802681181916\\
1015	-20.7901394370617\\
1016	-17.5537380540059\\
1017	-48.0787687895449\\
1018	-69.2237495144382\\
1019	-77.9745106034952\\
1020	-77.3939199117885\\
1021	-59.2542612892169\\
1022	-36.0938023716094\\
1023	-75.9263219131674\\
1024	-55.036426357487\\
1025	-41.7353507531234\\
1026	-67.2008974440148\\
1027	-60.5154895370686\\
1028	-52.0754149864467\\
1029	-84.7190123054604\\
1030	-78.9294280893171\\
1031	-58.5117605387395\\
1032	-81.6765662047228\\
1033	-84.2550420501527\\
1034	-63.5308228539363\\
1035	-58.3016635628035\\
1036	-44.1300371927903\\
1037	-53.6806188954279\\
1038	-61.1969309240255\\
1039	-61.6026123388424\\
1040	-60.7470013310813\\
1041	-63.538250640464\\
1042	-70.6002196697364\\
1043	-91.9124858422647\\
1044	-87.518412999774\\
1045	-55.5463308869564\\
1046	-58.9082668369438\\
1047	-31.7260183024052\\
1048	-33.8616433512875\\
1049	-44.0759792994681\\
1050	-36.9765235414764\\
1051	-40.5709345621915\\
1053	-61.8032461516968\\
1054	-56.450535242652\\
1055	-85.5174398449001\\
1056	-67.975207564147\\
1057	-42.4751320719622\\
1058	-35.8654691149925\\
1059	-42.4643125524935\\
1060	-50.3641468261008\\
1061	-28.0351519891619\\
1062	-30.5215956210866\\
1063	-40.0530107644602\\
1064	-34.4196665379066\\
1065	-28.3006508271346\\
1066	-38.4638640785661\\
1067	-42.9370246088633\\
1068	-65.7780469692404\\
1069	-64.19464946178\\
1070	-73.3249521773532\\
1071	-69.3074508228208\\
1072	-74.3093961643895\\
1073	-61.3972773061437\\
1074	-60.0141083676531\\
1075	-54.7105916474916\\
1077	-54.6961676605658\\
1078	-88.290283524997\\
1079	-126.076581344702\\
1080	-127.355042047166\\
1081	-123.069558196723\\
1082	-78.2172801863373\\
1083	-109.605547144749\\
1084	-135.943113750605\\
1085	-133.532205926831\\
1086	-100.974524454012\\
1087	-137.988747112204\\
1088	-171.735438114002\\
1089	-133.129025028836\\
1090	-106.021345982353\\
1091	-74.3873796055591\\
1092	-57.5726059333738\\
1093	-50.7370528834508\\
1094	-60.4369012942095\\
1096	-25.048344332095\\
1097	-24.6525311434\\
1098	-31.9648163264883\\
1099	-55.6339292833559\\
1100	-50.7373911296916\\
1101	-55.8990226743774\\
1102	-67.5247761014837\\
1103	-71.6935309581004\\
1104	-97.6080833539313\\
1105	-90.6275681325999\\
1106	-97.3753275071733\\
1108	-64.3467948948414\\
1109	-79.1666581958841\\
1110	-57.0159642270489\\
1111	-53.0149921580389\\
1112	-63.0577151097793\\
1113	-64.1065769228908\\
1114	-43.670774685269\\
1115	-49.7636302365061\\
1116	-38.4004338513009\\
1118	-43.7140460085022\\
1119	-32.7373835232652\\
1120	-38.3603915525705\\
1121	-48.3504100671921\\
1122	-74.9986986896527\\
1123	-76.6680775097834\\
1124	-79.3071254305482\\
1125	-52.1955353095473\\
1126	-67.62355370878\\
1127	-108.219112945793\\
1128	-82.1673795026511\\
1129	-95.3222492355333\\
1130	-89.8531300096176\\
1131	-57.2498549495174\\
1132	-40.8098405375315\\
1133	-54.8614036148833\\
1134	-62.8327409948038\\
1135	-96.3926278893994\\
1136	-118.930351231181\\
1137	-112.150368026047\\
1138	-93.1552055489449\\
1139	-90.4235549229475\\
1140	-87.3207118234893\\
1141	-81.611356434908\\
1142	-79.8971655227976\\
1143	-53.8921029493065\\
1144	-52.1552162008193\\
1145	-45.4088253032307\\
1146	-42.1715317595979\\
1147	-48.1342335900661\\
1148	-63.6259988770291\\
1149	-97.6376766539079\\
1150	-78.3856578008447\\
1151	-56.0828076859284\\
1152	-48.8214009038925\\
1153	-49.4732777202032\\
1154	-29.4594893878391\\
1155	-21.3720995973099\\
1156	-24.0987501485638\\
1157	-45.3789599529503\\
1158	-34.1016799380454\\
1159	-38.8780074662207\\
1160	-39.0536035324478\\
1161	-40.072139918718\\
1162	-32.3920269089879\\
1163	-22.5968911530147\\
1164	-18.2175851688551\\
1165	-27.8473215670051\\
1166	-59.2126423506486\\
1167	-81.4341502968823\\
1168	-90.2681113635149\\
1169	-59.9550692575281\\
1170	-44.6195586174194\\
1171	-34.5854293095836\\
1172	-26.4284001122273\\
1173	-47.7283287822299\\
1174	-47.6357643016643\\
1175	-63.9817819176801\\
1176	-82.520950083237\\
1177	-134.855129947629\\
1178	-156.272435221408\\
1179	-125.425084986175\\
1180	-124.855443148185\\
1181	-77.4814917280871\\
1182	-89.2550809311435\\
1183	-90.9976333530237\\
1184	-98.3833905660842\\
1185	-79.2050367622207\\
1186	-70.235671619191\\
1187	-71.2383322214848\\
1188	-67.0896895920666\\
1189	-76.2517002658838\\
1190	-58.683209139464\\
1191	-92.093745304179\\
1192	-110.920226519225\\
1193	-83.1298136109538\\
1194	-50.614312119671\\
1195	-56.262599739018\\
1196	-90.3751520258029\\
1197	-108.919921916841\\
1198	-134.641796988022\\
1199	-135.384417248828\\
1200	-137.594433077074\\
1201	-107.759333761461\\
1202	-93.9219860040964\\
1203	-110.8644853865\\
1204	-120.908911607564\\
1205	-72.6066123461219\\
1206	-49.5603663680006\\
1207	-65.6144235149054\\
1208	-62.5810406798612\\
1209	-37.9599188072436\\
1210	-51.6192703139977\\
1211	-58.240817449861\\
1212	-51.5029227183709\\
1213	-38.1493849959838\\
1214	-52.7290690442321\\
1215	-45.4258875030966\\
1216	-55.1998797889648\\
1217	-80.0671635406293\\
1218	-73.0191570978852\\
1219	-54.0962277382523\\
1220	-55.448543851845\\
1221	-78.7250420741959\\
1222	-130.36814765603\\
1223	-99.4281861689142\\
1224	-58.6130719117214\\
1225	-46.2683345907394\\
1226	-53.6857257391441\\
1227	-49.4105661689796\\
1228	-39.0303939288281\\
1229	-42.2976020446101\\
1230	-50.3188352817092\\
1231	-61.427372960263\\
1232	-67.8679196018547\\
1233	-48.6077075555866\\
1234	-80.9006651635209\\
1235	-91.7165803076457\\
1236	-85.1895632560106\\
1237	-70.760214103504\\
1238	-72.2482663058745\\
1239	-97.462041093348\\
1240	-66.6738104743808\\
1241	-28.9755203062027\\
1242	-38.4684148639258\\
1243	-44.4219840348812\\
1244	-63.6666208010729\\
1245	-53.1098778985879\\
1246	-40.4684557909916\\
1247	-59.8924772625874\\
1248	-58.2634936695886\\
1249	-42.3073414427024\\
1250	-54.9676839945612\\
1251	-48.7001685977457\\
1252	-47.2838597380648\\
1253	-49.9283892892997\\
1254	-33.3515439760083\\
1255	-36.8700930596933\\
1256	-28.5074983781944\\
1257	-32.1092653689846\\
1258	-55.8022957083042\\
1259	-61.7280085428599\\
1260	-79.585693814359\\
1261	-107.416796142568\\
1262	-73.6684104684564\\
1263	-45.2064472647689\\
1264	-37.1400210874388\\
1265	-35.7766393378754\\
1266	-49.7348130435425\\
1267	-33.0499230949531\\
1268	-36.3357585519946\\
1269	-45.5439345779951\\
1270	-70.6709504455287\\
1271	-66.418388588527\\
1272	-89.3726984900359\\
1273	-61.1035076982416\\
1274	-57.0868574719848\\
1275	-31.1931193033051\\
1276	-30.9696385142763\\
1277	-22.1527482191925\\
1278	-23.0642924873175\\
1279	-28.111318182755\\
1280	-45.7169399281363\\
1281	-51.6925918912154\\
1282	-54.4355270529747\\
1283	-60.905202975667\\
1284	-95.6239479219555\\
1285	-88.8768295789077\\
1286	-64.8728614993847\\
1287	-76.8665750902501\\
1288	-76.985056564354\\
1289	-98.1543531511097\\
1290	-82.1360875195887\\
1291	-53.5024354580303\\
1292	-29.6159128396762\\
1293	-26.6540156488966\\
1294	-24.6119560152399\\
1295	-28.3689439703701\\
1296	-29.3751292452378\\
1297	-27.9490107727352\\
1298	-35.5318127096871\\
1299	-55.6056831679425\\
1300	-49.7965874377173\\
1301	-50.0596114350676\\
1302	-57.5260709995166\\
1303	-38.4885861061309\\
1304	-23.6780377251373\\
1305	-40.6424440672265\\
1306	-62.8436135718043\\
1307	-59.4489935862837\\
1308	-66.0381854558445\\
1309	-58.0578511328154\\
1310	-43.7465111429503\\
1311	-60.2501477198487\\
1312	-82.8550839030336\\
1313	-80.3289620012454\\
1314	-57.0641923037804\\
1315	-101.648607284942\\
1316	-77.6677678159908\\
1317	-75.4565857553291\\
1318	-82.562643460898\\
1319	-82.5546630267377\\
1320	-62.4353941733445\\
1321	-58.4127396603176\\
1322	-87.274275129407\\
1323	-123.922901823315\\
1324	-92.8246433121378\\
1325	-54.7524145281632\\
1326	-53.7417798384658\\
1327	-54.9099210255961\\
1328	-70.7062483386994\\
1329	-80.2031712563571\\
1330	-58.0195615623618\\
1331	-68.9141172794384\\
1332	-83.5857924338416\\
1333	-84.4630216595715\\
1334	-60.1701913509924\\
1335	-48.2098651700467\\
1336	-64.5840995517106\\
1337	-103.55038657646\\
1338	-96.1725622985098\\
1339	-91.9649494444602\\
1340	-57.8444868539673\\
1341	-51.4580948751727\\
1342	-55.3791245653056\\
1343	-32.9587357793846\\
1344	-21.2543702661919\\
1345	-17.4838528601751\\
1346	-15.9769342001816\\
1347	-37.0135793074921\\
1348	-51.5697982659779\\
1349	-60.9910441514739\\
1350	-47.7326874763733\\
1351	-49.2588733278658\\
1352	-57.5859220786963\\
1353	-36.8778389194672\\
1354	-38.7468223413252\\
1355	-38.6586058937728\\
1356	-31.4640998309114\\
1357	-34.780303678923\\
1358	-58.8747809683439\\
1359	-72.1868174999336\\
1360	-46.4533718836899\\
1361	-35.6324694360947\\
1362	-32.1151374430717\\
1363	-39.2072652609099\\
1364	-25.8874635226277\\
1365	-59.4324586855055\\
1366	-96.2405494781053\\
1367	-74.3963167820361\\
1368	-99.1019000501756\\
1369	-112.488387636476\\
1370	-113.055387675975\\
1371	-87.1729069002513\\
1372	-65.5835770472618\\
1373	-65.1404760967196\\
1374	-66.6654171385253\\
1376	-84.440260003953\\
1377	-86.5327676739321\\
1378	-112.72382633045\\
1379	-121.990709443312\\
1380	-143.23303726787\\
1381	-95.7763534587807\\
1382	-104.576825721329\\
1383	-115.169907039313\\
1384	-94.9475115893147\\
1385	-108.72266086733\\
1386	-90.2161070507159\\
1387	-53.2379528256706\\
1388	-33.5142562343246\\
1389	-36.1015870203532\\
1390	-53.2692396781424\\
1391	-47.9169869675634\\
1392	-33.609657761408\\
1393	-42.8879840402894\\
1394	-36.2281529192999\\
1395	-25.1297737355271\\
1396	-33.0639217318069\\
1397	-36.4319396846549\\
1398	-29.8684456609517\\
1399	-49.5101030702924\\
1400	-67.4502781725114\\
1401	-53.988071044367\\
1402	-88.9227104220186\\
1403	-128.500249735253\\
1404	-112.159327370518\\
1405	-132.278304356364\\
};
\addlegendentry{MPO prediction}

\end{axis}

\begin{axis}[%
width=6.159cm,
height=1.831cm,
at={(8.104cm,0cm)},
scale only axis,
xmin=1000,
xmax=1405,
xlabel style={font=\color{white!15!black}},
xlabel={Sample index},
ymin=-200,
ymax=0,
ylabel style={font=\color{white!15!black}},
ylabel={$y(t)$},
axis background/.style={fill=white},
title style={font=\bfseries},
title={C10: RMSE(OSA) = 7.4133, RMSE(MPO) = 11.5662},
legend style={legend cell align=left, align=left, draw=white!15!black}
]
\addplot [color=mycolor1, line width=2.0pt]
  table[row sep=crcr]{%
1006	-76.904\\
1007	-90.3320000000001\\
1008	-68.3589999999999\\
1009	-40.2829999999999\\
1010	-48.828\\
1011	-46.3869999999999\\
1012	-54.932\\
1013	-45.1659999999999\\
1014	-21.973\\
1015	-17.0899999999999\\
1016	-18.3109999999999\\
1018	-62.2560000000001\\
1019	-67.1389999999999\\
1020	-69.5799999999999\\
1022	-34.1800000000001\\
1023	-64.6970000000001\\
1024	-52.49\\
1025	-39.0630000000001\\
1026	-61.0350000000001\\
1027	-53.711\\
1028	-45.1659999999999\\
1029	-73.242\\
1030	-67.1389999999999\\
1031	-52.49\\
1032	-75.684\\
1033	-74.463\\
1034	-58.5940000000001\\
1036	-41.5039999999999\\
1037	-47.607\\
1038	-58.5940000000001\\
1039	-54.932\\
1040	-59.8140000000001\\
1041	-59.8140000000001\\
1042	-65.9180000000001\\
1043	-85.4490000000001\\
1044	-76.904\\
1045	-54.932\\
1046	-50.049\\
1047	-35.4000000000001\\
1048	-34.1800000000001\\
1049	-45.1659999999999\\
1050	-34.1800000000001\\
1051	-36.6210000000001\\
1052	-51.27\\
1053	-54.932\\
1054	-51.27\\
1055	-78.125\\
1056	-62.2560000000001\\
1057	-36.6210000000001\\
1058	-30.518\\
1059	-40.2829999999999\\
1060	-46.3869999999999\\
1061	-29.297\\
1062	-29.297\\
1063	-36.6210000000001\\
1065	-26.855\\
1066	-35.4000000000001\\
1067	-40.2829999999999\\
1068	-58.5940000000001\\
1069	-56.152\\
1070	-65.9180000000001\\
1071	-61.0350000000001\\
1072	-70.8009999999999\\
1073	-57.373\\
1074	-58.5940000000001\\
1075	-51.27\\
1077	-48.828\\
1079	-111.084\\
1080	-114.746\\
1081	-112.305\\
1082	-74.463\\
1084	-125.732\\
1085	-128.174\\
1086	-96.4359999999999\\
1087	-124.512\\
1088	-158.691\\
1089	-115.967\\
1090	-97.6559999999999\\
1091	-68.3589999999999\\
1092	-53.711\\
1093	-47.607\\
1094	-56.152\\
1095	-45.1659999999999\\
1096	-30.518\\
1097	-30.518\\
1098	-36.6210000000001\\
1099	-50.049\\
1101	-47.607\\
1102	-62.2560000000001\\
1103	-64.6970000000001\\
1104	-85.4490000000001\\
1105	-72.021\\
1106	-85.4490000000001\\
1107	-63.4770000000001\\
1108	-54.932\\
1109	-64.6970000000001\\
1110	-48.828\\
1111	-43.9449999999999\\
1112	-53.711\\
1113	-54.932\\
1114	-40.2829999999999\\
1115	-43.9449999999999\\
1116	-32.9590000000001\\
1118	-40.2829999999999\\
1119	-30.518\\
1120	-34.1800000000001\\
1121	-43.9449999999999\\
1122	-63.4770000000001\\
1123	-65.9180000000001\\
1124	-72.021\\
1125	-45.1659999999999\\
1126	-56.152\\
1127	-89.1110000000001\\
1128	-70.8009999999999\\
1129	-81.787\\
1130	-80.566\\
1131	-50.049\\
1132	-36.6210000000001\\
1133	-47.607\\
1134	-52.49\\
1135	-81.787\\
1136	-103.76\\
1137	-102.539\\
1138	-76.904\\
1139	-80.566\\
1140	-72.021\\
1141	-69.5799999999999\\
1142	-68.3589999999999\\
1143	-51.27\\
1145	-39.0630000000001\\
1146	-37.8420000000001\\
1147	-41.5039999999999\\
1148	-56.152\\
1149	-84.229\\
1150	-70.8009999999999\\
1151	-50.049\\
1152	-42.7249999999999\\
1153	-40.2829999999999\\
1154	-28.076\\
1155	-23.193\\
1156	-24.414\\
1157	-41.5039999999999\\
1158	-32.9590000000001\\
1159	-34.1800000000001\\
1160	-37.8420000000001\\
1161	-35.4000000000001\\
1162	-26.855\\
1163	-21.973\\
1164	-18.3109999999999\\
1165	-25.635\\
1166	-51.27\\
1167	-73.242\\
1168	-78.125\\
1169	-54.932\\
1170	-39.0630000000001\\
1172	-24.414\\
1173	-41.5039999999999\\
1174	-42.7249999999999\\
1175	-54.932\\
1176	-73.242\\
1177	-108.643\\
1178	-125.732\\
1179	-103.76\\
1180	-101.318\\
1181	-67.1389999999999\\
1182	-75.684\\
1183	-79.346\\
1184	-86.6700000000001\\
1185	-69.5799999999999\\
1186	-62.2560000000001\\
1187	-61.0350000000001\\
1188	-56.152\\
1189	-67.1389999999999\\
1190	-52.49\\
1191	-76.904\\
1192	-97.6559999999999\\
1194	-45.1659999999999\\
1195	-47.607\\
1196	-78.125\\
1197	-93.9939999999999\\
1198	-112.305\\
1199	-119.629\\
1200	-119.629\\
1201	-91.5530000000001\\
1202	-84.229\\
1203	-96.4359999999999\\
1204	-111.084\\
1205	-74.463\\
1206	-46.3869999999999\\
1207	-62.2560000000001\\
1208	-54.932\\
1209	-34.1800000000001\\
1210	-47.607\\
1211	-53.711\\
1212	-42.7249999999999\\
1213	-34.1800000000001\\
1214	-43.9449999999999\\
1215	-40.2829999999999\\
1216	-52.49\\
1217	-65.9180000000001\\
1218	-62.2560000000001\\
1219	-45.1659999999999\\
1220	-45.1659999999999\\
1221	-67.1389999999999\\
1222	-104.98\\
1223	-79.346\\
1224	-51.27\\
1225	-40.2829999999999\\
1226	-47.607\\
1227	-40.2829999999999\\
1228	-31.7380000000001\\
1229	-35.4000000000001\\
1230	-42.7249999999999\\
1231	-51.27\\
1232	-57.373\\
1233	-42.7249999999999\\
1234	-63.4770000000001\\
1235	-75.684\\
1236	-70.8009999999999\\
1237	-57.373\\
1238	-62.2560000000001\\
1239	-81.787\\
1240	-52.49\\
1241	-28.076\\
1242	-36.6210000000001\\
1243	-42.7249999999999\\
1244	-51.27\\
1245	-46.3869999999999\\
1246	-36.6210000000001\\
1247	-52.49\\
1248	-52.49\\
1249	-37.8420000000001\\
1250	-47.607\\
1251	-41.5039999999999\\
1252	-37.8420000000001\\
1253	-45.1659999999999\\
1254	-29.297\\
1255	-31.7380000000001\\
1257	-26.855\\
1258	-47.607\\
1259	-53.711\\
1260	-69.5799999999999\\
1261	-89.1110000000001\\
1263	-36.6210000000001\\
1264	-31.7380000000001\\
1265	-29.297\\
1266	-40.2829999999999\\
1267	-31.7380000000001\\
1268	-30.518\\
1269	-39.0630000000001\\
1270	-61.0350000000001\\
1271	-53.711\\
1272	-79.346\\
1273	-54.932\\
1274	-48.828\\
1275	-32.9590000000001\\
1276	-29.297\\
1277	-23.193\\
1278	-23.193\\
1279	-25.635\\
1280	-40.2829999999999\\
1281	-46.3869999999999\\
1283	-51.27\\
1284	-79.346\\
1285	-73.242\\
1286	-53.711\\
1287	-63.4770000000001\\
1288	-69.5799999999999\\
1289	-85.4490000000001\\
1290	-72.021\\
1291	-47.607\\
1292	-28.076\\
1293	-21.973\\
1294	-21.973\\
1295	-26.855\\
1296	-26.855\\
1297	-25.635\\
1298	-32.9590000000001\\
1299	-47.607\\
1300	-43.9449999999999\\
1301	-45.1659999999999\\
1302	-52.49\\
1303	-35.4000000000001\\
1304	-20.752\\
1305	-35.4000000000001\\
1306	-56.152\\
1307	-54.932\\
1308	-59.8140000000001\\
1309	-53.711\\
1310	-40.2829999999999\\
1311	-51.27\\
1312	-72.021\\
1313	-70.8009999999999\\
1314	-51.27\\
1315	-81.787\\
1316	-62.2560000000001\\
1317	-63.4770000000001\\
1318	-73.242\\
1319	-70.8009999999999\\
1320	-54.932\\
1321	-47.607\\
1323	-108.643\\
1324	-85.4490000000001\\
1325	-50.049\\
1326	-51.27\\
1327	-51.27\\
1328	-61.0350000000001\\
1329	-72.021\\
1330	-54.932\\
1331	-56.152\\
1332	-73.242\\
1333	-79.346\\
1334	-56.152\\
1335	-45.1659999999999\\
1336	-57.373\\
1337	-87.8910000000001\\
1338	-78.125\\
1339	-80.566\\
1340	-54.932\\
1341	-45.1659999999999\\
1342	-47.607\\
1343	-31.7380000000001\\
1344	-20.752\\
1345	-20.752\\
1346	-17.0899999999999\\
1347	-31.7380000000001\\
1348	-48.828\\
1349	-54.932\\
1350	-40.2829999999999\\
1351	-45.1659999999999\\
1352	-52.49\\
1353	-35.4000000000001\\
1355	-35.4000000000001\\
1356	-29.297\\
1357	-31.7380000000001\\
1358	-51.27\\
1359	-64.6970000000001\\
1360	-41.5039999999999\\
1361	-32.9590000000001\\
1362	-28.076\\
1363	-32.9590000000001\\
1364	-26.855\\
1365	-45.1659999999999\\
1366	-80.566\\
1367	-63.4770000000001\\
1369	-98.877\\
1370	-97.6559999999999\\
1371	-76.904\\
1372	-58.5940000000001\\
1373	-54.932\\
1374	-57.373\\
1375	-65.9180000000001\\
1376	-76.904\\
1377	-76.904\\
1378	-100.098\\
1379	-108.643\\
1380	-130.615\\
1381	-91.5530000000001\\
1382	-98.877\\
1383	-108.643\\
1384	-85.4490000000001\\
1385	-96.4359999999999\\
1386	-85.4490000000001\\
1387	-51.27\\
1388	-34.1800000000001\\
1389	-36.6210000000001\\
1390	-51.27\\
1391	-43.9449999999999\\
1392	-31.7380000000001\\
1393	-41.5039999999999\\
1394	-32.9590000000001\\
1395	-23.193\\
1396	-31.7380000000001\\
1397	-34.1800000000001\\
1398	-25.635\\
1400	-58.5940000000001\\
1401	-45.1659999999999\\
1402	-68.3589999999999\\
1403	-100.098\\
1404	-87.8910000000001\\
1405	-109.863\\
};
\addlegendentry{True output}

\addplot [color=mycolor2, dashed, line width=2.0pt]
  table[row sep=crcr]{%
1006	-79.926407842808\\
1007	-91.0901261788222\\
1008	-64.5059333488762\\
1009	-26.1175019066841\\
1010	-47.3096810284665\\
1011	-56.407105750196\\
1012	-56.1351685513298\\
1013	-52.2048238964771\\
1014	-10.4558418262607\\
1015	-9.6452484467186\\
1016	-15.513470775704\\
1017	-46.0537637125149\\
1018	-66.894160232235\\
1019	-67.1109087040682\\
1020	-75.5557970195978\\
1021	-53.4413343723786\\
1022	-26.2445523201925\\
1023	-86.473077841747\\
1024	-47.0084608310062\\
1025	-38.2999480960282\\
1026	-73.8416111307797\\
1027	-53.9640037064985\\
1028	-53.5930930911263\\
1029	-80.1801004309402\\
1030	-70.790984783557\\
1031	-52.9966596993227\\
1032	-81.0830058162794\\
1033	-78.4602612430317\\
1034	-62.7431275593678\\
1035	-50.7051138481684\\
1036	-43.2696618713987\\
1037	-50.5548199903417\\
1038	-63.1991819297848\\
1039	-56.9749114544461\\
1040	-58.5933394504204\\
1041	-62.5140670146413\\
1042	-64.8306049247778\\
1043	-97.7045225949646\\
1044	-81.2695683701925\\
1045	-50.5202292390604\\
1046	-52.1922467408799\\
1047	-23.3863730882413\\
1048	-33.6968720706468\\
1049	-46.9503996029223\\
1050	-37.877183109191\\
1051	-39.6444014886702\\
1052	-58.5436038421676\\
1053	-57.9787920755261\\
1054	-52.4076688580062\\
1055	-90.2570294195314\\
1057	-34.2142827872572\\
1058	-28.0350575760399\\
1059	-43.6825765015376\\
1060	-51.868188179393\\
1061	-27.2375981523026\\
1062	-27.2682388774974\\
1063	-44.4096959517224\\
1064	-34.3245411449147\\
1065	-29.8966727395657\\
1066	-38.0407836695861\\
1067	-41.0408579212174\\
1068	-66.693223140102\\
1069	-59.8401899096766\\
1070	-69.9239945923052\\
1071	-69.3917065216724\\
1072	-68.9638062504762\\
1073	-60.809140480935\\
1074	-58.6115934918425\\
1075	-54.2748422443731\\
1076	-52.2679321853673\\
1077	-56.747556140652\\
1078	-89.7066401514621\\
1079	-120.012311284395\\
1080	-119.622944639664\\
1081	-108.993907704005\\
1082	-58.8759480689946\\
1083	-120.929869292731\\
1084	-141.233701203731\\
1085	-124.285171493629\\
1086	-86.3962092109218\\
1087	-144.46703019715\\
1088	-181.430248484543\\
1089	-106.784343443303\\
1090	-90.7702854093372\\
1091	-44.5816271709839\\
1092	-35.0350687444541\\
1093	-41.6557754599132\\
1094	-59.1705896035949\\
1095	-41.4106672041496\\
1096	-20.1562987914808\\
1097	-23.5222504845203\\
1098	-38.2406945849154\\
1099	-59.3863108222379\\
1100	-51.253070532779\\
1101	-52.6223358972406\\
1102	-68.8480231990961\\
1103	-68.7885109657584\\
1104	-94.3841560405874\\
1105	-81.5784247125487\\
1106	-88.9303799117743\\
1107	-61.4425148671189\\
1108	-57.9646510928731\\
1109	-70.963039262665\\
1110	-50.424125003864\\
1111	-48.0625326239231\\
1112	-57.6803402910496\\
1113	-58.0547540627906\\
1114	-42.131799947168\\
1115	-44.9486063120046\\
1116	-37.1162873754381\\
1117	-38.8167846394867\\
1118	-40.3057185474199\\
1119	-32.4851916858711\\
1120	-37.9930517385781\\
1121	-48.75212430084\\
1122	-69.3623423494716\\
1123	-70.6981588492529\\
1124	-75.8059399123501\\
1125	-39.2225402014792\\
1126	-67.5632448838148\\
1127	-111.081285849685\\
1128	-72.3080961219289\\
1129	-82.7750316304309\\
1130	-86.4951143233779\\
1131	-41.6722396379764\\
1132	-30.2553605375476\\
1133	-52.4654270249825\\
1134	-60.1949794557543\\
1135	-94.4163101881381\\
1136	-107.50349354592\\
1137	-103.932999162574\\
1138	-73.7290054469743\\
1139	-85.1857340134059\\
1140	-70.8612312036864\\
1141	-75.5730277343025\\
1142	-75.127564255567\\
1143	-43.1767633338025\\
1144	-40.0073574842565\\
1145	-42.579537432722\\
1146	-37.9742434055356\\
1147	-46.0415870314121\\
1148	-65.2211651247758\\
1149	-92.6641491095711\\
1150	-68.3281488861878\\
1151	-50.6529097350681\\
1152	-41.5662032363305\\
1153	-43.5486325953657\\
1154	-22.7024429948349\\
1155	-16.1396391094559\\
1156	-26.4868883271456\\
1157	-47.0623622851094\\
1158	-36.2449694414456\\
1159	-37.3896382079661\\
1160	-41.6338817426981\\
1161	-38.3361896348292\\
1162	-28.3683948794985\\
1163	-19.3622449121744\\
1164	-17.3860749302344\\
1165	-29.7742061234633\\
1166	-57.5375893404132\\
1167	-73.7345906504618\\
1168	-84.2003762505076\\
1169	-54.2961306746022\\
1170	-32.6237775442519\\
1172	-19.0524072918342\\
1173	-51.1109887760895\\
1174	-45.9562854545577\\
1175	-59.7359098005297\\
1176	-79.1019299067166\\
1177	-127.210665749716\\
1178	-143.932787353846\\
1179	-105.997831974624\\
1180	-108.604036403086\\
1181	-48.2855194822937\\
1182	-80.7923648522831\\
1183	-82.0972425855998\\
1184	-97.5937311823193\\
1185	-63.0137865059639\\
1186	-64.0895034335217\\
1187	-66.3664555037801\\
1188	-58.2226630566124\\
1189	-70.4219150003671\\
1190	-50.7495887504842\\
1191	-95.8465043353842\\
1192	-101.244696459916\\
1194	-35.582236967326\\
1195	-45.2825513667276\\
1196	-94.3169342951774\\
1197	-99.4762279203514\\
1198	-117.654855743167\\
1199	-125.828976057689\\
1200	-129.115213767591\\
1201	-80.0931039188031\\
1202	-89.7020981305329\\
1203	-104.049968213744\\
1204	-115.31737395686\\
1205	-62.6842556571542\\
1206	-26.6749411792402\\
1207	-68.4892751961477\\
1208	-60.2889168814552\\
1209	-31.7234386622831\\
1210	-49.2926730975951\\
1211	-56.6021895711187\\
1212	-48.1705493342131\\
1213	-36.1136852823711\\
1214	-48.9586856251226\\
1215	-43.8419159474111\\
1216	-51.1284684050941\\
1217	-81.3001506395938\\
1218	-67.7703419783693\\
1219	-46.704837598762\\
1220	-50.2689750823088\\
1221	-73.6114270414637\\
1222	-121.032969767742\\
1224	-43.533534518353\\
1225	-25.6692167578351\\
1226	-49.9401459428743\\
1227	-47.9203805314376\\
1228	-32.8417111789265\\
1229	-35.9075448780802\\
1230	-50.4870605455951\\
1231	-54.5091434733897\\
1232	-61.9731097224299\\
1233	-39.7927774097573\\
1234	-76.8840942906988\\
1235	-89.3791201766137\\
1237	-61.1771403221858\\
1238	-67.3691663588611\\
1239	-95.4555230071092\\
1241	-13.0572576750928\\
1242	-32.2703009613699\\
1243	-43.0236570561731\\
1244	-58.5414035172782\\
1245	-51.1758400855872\\
1246	-37.8143976938652\\
1247	-60.4765548880937\\
1248	-56.2007996859829\\
1249	-41.251379906801\\
1250	-55.2868404087781\\
1251	-46.7135785762232\\
1252	-42.1175891014411\\
1253	-46.5401449649089\\
1254	-28.8252481609541\\
1255	-37.1008890766905\\
1256	-24.1137256130212\\
1257	-30.6254225589696\\
1258	-58.8655998937008\\
1259	-56.6297730689776\\
1260	-73.1627487994058\\
1261	-102.374962419485\\
1263	-28.06706473074\\
1264	-22.325529681633\\
1265	-30.2432981477407\\
1266	-49.638512637209\\
1267	-27.4357056057122\\
1268	-34.1220168069458\\
1269	-44.5730835242937\\
1270	-66.7423413346999\\
1271	-63.4330776334536\\
1272	-86.1851767008293\\
1273	-54.3940134866934\\
1274	-52.4331092286989\\
1275	-17.8669065922277\\
1276	-26.7520750875542\\
1277	-20.3727728480435\\
1278	-22.6695335115605\\
1279	-30.726232849591\\
1280	-47.5728826513125\\
1281	-46.5339757241952\\
1282	-53.4007512244634\\
1283	-56.7427895514666\\
1284	-96.5427034368115\\
1285	-79.4240109609752\\
1286	-55.1643375775004\\
1287	-71.5035016333843\\
1288	-75.0742826753228\\
1289	-93.1439980341133\\
1290	-75.9249077841955\\
1291	-38.9256494620506\\
1292	-11.7665773186868\\
1293	-14.246755086461\\
1294	-22.9205370169018\\
1295	-28.1215806069399\\
1296	-29.4717543040174\\
1297	-26.6735083188439\\
1298	-35.3988484045556\\
1299	-54.1785241622422\\
1300	-49.1097874226539\\
1301	-45.608918881084\\
1302	-56.6631576115797\\
1303	-33.4367400907968\\
1304	-20.3050495318116\\
1305	-44.0076872493364\\
1306	-65.1423439994799\\
1307	-53.2295960310892\\
1308	-65.3537525094555\\
1309	-53.6133233477776\\
1310	-39.9014069531961\\
1311	-59.0185543239388\\
1312	-82.4474691143839\\
1313	-76.4274262619851\\
1314	-53.8709211268897\\
1315	-94.2816324565483\\
1316	-68.798401468325\\
1317	-61.1027173268092\\
1318	-79.4512831781492\\
1319	-74.8177742518469\\
1320	-57.2101998231979\\
1321	-51.9453738275306\\
1322	-90.9812321191534\\
1323	-115.275669018575\\
1324	-87.5548562866456\\
1325	-36.6904967395601\\
1326	-44.2852309378998\\
1328	-68.4471956743637\\
1329	-73.9563099978222\\
1330	-57.0203781249108\\
1331	-58.254089539008\\
1332	-84.9025742368119\\
1333	-83.7902475345172\\
1334	-54.9429782049303\\
1335	-41.724569262753\\
1336	-64.0671514571011\\
1337	-100.172292190244\\
1338	-83.3306998324181\\
1339	-83.7487584452174\\
1340	-38.5686859206094\\
1341	-47.7319143033665\\
1342	-50.9705301528627\\
1343	-25.9306834383344\\
1344	-15.8940424412467\\
1345	-13.9541945466083\\
1346	-16.9810396011883\\
1347	-38.5977881047431\\
1348	-49.2762788263801\\
1349	-57.962146072487\\
1350	-46.0953237191495\\
1351	-49.6014041889882\\
1352	-58.2352701744028\\
1353	-32.5125081930764\\
1354	-37.4475376455298\\
1355	-38.9022698807112\\
1356	-28.8427614162197\\
1357	-38.2635701122124\\
1358	-59.8052783127969\\
1359	-66.9249123870086\\
1360	-38.3828122050488\\
1361	-32.4880148777627\\
1362	-30.7636002041477\\
1363	-39.8411892408017\\
1364	-23.8849852829269\\
1366	-89.7300979451138\\
1367	-63.6938461347993\\
1368	-93.3406750595664\\
1369	-109.593567405679\\
1370	-102.912090871803\\
1371	-75.1863666117124\\
1372	-50.0100879047457\\
1373	-62.0742271387869\\
1374	-56.8618915938146\\
1375	-70.4805304865674\\
1376	-81.3058735024163\\
1377	-77.8316592558958\\
1378	-112.658922810053\\
1379	-121.7867499215\\
1380	-132.772765420428\\
1381	-82.3349028004841\\
1382	-107.942057299831\\
1383	-122.573483175184\\
1384	-78.5785277870177\\
1385	-106.166227847068\\
1386	-85.3974443109103\\
1387	-34.1752115592597\\
1388	-17.1958845559548\\
1389	-28.1950685775107\\
1390	-61.1159344227244\\
1391	-43.8151205167094\\
1392	-30.222012787267\\
1393	-42.1297403563681\\
1394	-35.5368407509889\\
1395	-25.9923958532659\\
1396	-31.7911713785606\\
1397	-38.3941593599191\\
1398	-28.4451598576966\\
1399	-52.5894890718432\\
1400	-63.0991783409436\\
1401	-45.3460820430855\\
1402	-81.7600526592501\\
1403	-121.32311693172\\
1404	-90.9358563839435\\
1405	-127.648059995245\\
};
\addlegendentry{OSA predition}

\addplot [color=mycolor3, dotted, line width=2.0pt]
  table[row sep=crcr]{%
1006	-76.904\\
1007	-90.3320000000001\\
1008	-68.3589999999999\\
1009	-40.2829999999999\\
1010	-47.309681028468\\
1011	-56.0096802179141\\
1012	-58.1716468524949\\
1013	-55.4479135578713\\
1014	-11.0088049074693\\
1015	-9.23423216849369\\
1016	-9.7185291057483\\
1017	-37.6178520948404\\
1018	-57.6214539897251\\
1019	-61.4209224575143\\
1020	-71.6702317342551\\
1021	-51.9551088363276\\
1022	-27.2057123691923\\
1023	-84.6522282668002\\
1024	-49.0678403245313\\
1025	-43.1210755527682\\
1026	-79.8271486060687\\
1027	-60.1929905061336\\
1028	-60.2787661657846\\
1029	-92.4326121707745\\
1030	-80.9327483180166\\
1031	-61.5554981935788\\
1032	-93.5188574499462\\
1033	-88.3344578715669\\
1034	-71.6799601845623\\
1035	-58.5531415813277\\
1036	-50.6555928784842\\
1037	-57.9507512928235\\
1038	-71.3976305812162\\
1039	-64.6180686589189\\
1040	-66.5730177460687\\
1041	-69.5846748329291\\
1042	-71.4604678073338\\
1043	-105.005465620792\\
1044	-87.5368023313736\\
1045	-58.1896909442448\\
1046	-57.1002130815586\\
1047	-26.2293165460137\\
1048	-33.8560718804622\\
1049	-45.7977824798406\\
1050	-38.0434815823162\\
1051	-40.1186570728275\\
1052	-60.7036710440059\\
1053	-62.1708939080133\\
1054	-57.354611699127\\
1055	-98.0943266631057\\
1057	-40.64480596206\\
1058	-30.1952896841319\\
1059	-47.113857481138\\
1060	-55.0320022531043\\
1061	-31.5976793350337\\
1062	-30.0706965916604\\
1063	-47.5914929203204\\
1064	-38.3949088667869\\
1065	-34.7911876283299\\
1066	-43.9559862922868\\
1067	-47.5522205505958\\
1068	-74.9193506263298\\
1069	-67.8903617815117\\
1070	-79.4635440686561\\
1071	-79.0611475864725\\
1072	-79.7181164847298\\
1073	-69.7188629729294\\
1074	-66.504602952326\\
1075	-61.6944881958907\\
1076	-58.6910572173315\\
1077	-63.9112399715943\\
1078	-100.798277845529\\
1079	-132.743920337869\\
1080	-132.080646769783\\
1081	-120.29086232358\\
1082	-63.9898807121779\\
1083	-127.708011226842\\
1084	-143.520818374054\\
1085	-128.791584933237\\
1086	-91.5275808199574\\
1087	-151.861391856126\\
1088	-179.102811322071\\
1089	-108.686085264161\\
1090	-96.2562842662217\\
1091	-41.9614029155773\\
1092	-30.8353871067502\\
1093	-26.7410035293358\\
1094	-43.0124765711694\\
1095	-29.9869828559226\\
1096	-14.0009061463654\\
1097	-13.1607068979558\\
1098	-25.4969325988368\\
1099	-44.6253722075157\\
1100	-42.502073730645\\
1101	-46.4124297295723\\
1102	-64.3983372106163\\
1103	-67.1009926324227\\
1104	-94.5071522242172\\
1105	-83.8290084479136\\
1106	-94.3210637958985\\
1107	-66.5383907165512\\
1108	-62.1658276016942\\
1109	-76.0143378156477\\
1110	-55.6035086674431\\
1111	-53.4948498784893\\
1112	-64.5112807704422\\
1113	-65.2507142383276\\
1114	-48.0729105542716\\
1115	-51.5245784939409\\
1116	-42.4309347084604\\
1117	-44.9485216861517\\
1118	-46.6696347465365\\
1119	-37.1614841794863\\
1120	-43.3015933176482\\
1121	-55.177259778435\\
1122	-77.8804611520065\\
1123	-79.8477006740384\\
1124	-85.9426067217862\\
1125	-45.202298642289\\
1126	-75.2608857360433\\
1127	-123.177959144447\\
1128	-81.1048024331449\\
1129	-97.9119140462537\\
1130	-99.9069728935601\\
1131	-50.3757400039215\\
1132	-34.8581218139709\\
1133	-55.8226690709478\\
1134	-63.4395075762905\\
1135	-98.388943730133\\
1136	-116.463857339535\\
1137	-113.056641373945\\
1138	-79.3193891937224\\
1139	-91.6161344112149\\
1140	-75.0145749290873\\
1141	-80.5452327109458\\
1142	-79.8328075044778\\
1143	-49.2893473649344\\
1144	-42.6985803149182\\
1145	-42.6932857461545\\
1146	-39.1675984309331\\
1147	-46.9143558828307\\
1148	-67.9743558126397\\
1149	-97.7891868829299\\
1150	-74.1867721976514\\
1151	-55.1025717329462\\
1152	-44.5974976980801\\
1153	-46.7586754830329\\
1154	-24.6667124197786\\
1155	-17.0724424039968\\
1156	-24.2135423460034\\
1157	-44.8340081814297\\
1158	-36.1189946703857\\
1159	-38.6070243173938\\
1160	-44.2256589953397\\
1161	-41.8320508595662\\
1163	-22.4223725112788\\
1164	-18.9844959248246\\
1165	-31.4861104546674\\
1166	-60.0950501609339\\
1167	-77.7775867582927\\
1168	-89.8800887684731\\
1169	-57.8537929490055\\
1170	-36.7913810170758\\
1171	-25.3951611497114\\
1172	-17.1780984592301\\
1173	-46.7969737879769\\
1174	-43.9775961936496\\
1175	-58.509937002\\
1176	-81.85466153794\\
1177	-131.057717079348\\
1178	-148.698078188954\\
1179	-111.969487383654\\
1180	-120.131825219782\\
1181	-50.9848659132499\\
1182	-85.4135457684995\\
1183	-82.3894549347979\\
1184	-98.428402060732\\
1185	-64.2087320562914\\
1186	-67.3263969192521\\
1187	-67.356748583705\\
1188	-61.7369735075181\\
1189	-74.0534390402522\\
1190	-54.336273318278\\
1191	-100.93997675666\\
1192	-108.037010090719\\
1193	-75.9919659534607\\
1194	-35.7193880080486\\
1195	-46.0124658559341\\
1196	-93.4781518899354\\
1197	-97.9383075590406\\
1198	-120.915146474704\\
1199	-132.238051797591\\
1200	-133.702918704699\\
1201	-83.7883638104474\\
1202	-94.5373394929186\\
1204	-118.514321099548\\
1205	-64.7794439369179\\
1206	-25.4210807558306\\
1207	-61.4584505451976\\
1208	-53.5619158970808\\
1209	-33.2199989202018\\
1210	-46.8035912505948\\
1211	-55.8051089631474\\
1212	-48.2871708845323\\
1213	-38.1009125310161\\
1214	-51.8785095594355\\
1215	-47.7305499323411\\
1216	-56.4373757825056\\
1217	-87.8550496891307\\
1218	-74.9714504928299\\
1219	-55.9422760430855\\
1220	-58.6470298298229\\
1221	-85.4161613148231\\
1222	-136.063434573951\\
1223	-89.2303004630801\\
1224	-52.1226609722773\\
1225	-26.6159394052988\\
1226	-49.6843496115243\\
1227	-45.4620441732029\\
1228	-35.2994885807946\\
1229	-37.3754154375972\\
1230	-53.3209231925243\\
1231	-59.0605395951252\\
1232	-67.5010056842232\\
1233	-44.1682041412455\\
1234	-83.550291898868\\
1235	-98.0738994154613\\
1237	-71.2865123796805\\
1238	-79.0705203954833\\
1239	-109.314255856183\\
1241	-19.3785503921561\\
1242	-34.5980498723229\\
1243	-43.4870368937964\\
1244	-56.5629324606025\\
1245	-51.686365497196\\
1246	-40.4363256292804\\
1247	-64.3762459057086\\
1248	-61.44771215233\\
1249	-47.0395581912494\\
1250	-62.7815828797102\\
1251	-54.8354052819489\\
1252	-50.9910605374594\\
1253	-56.601138637486\\
1254	-35.8430976240343\\
1255	-44.2880829983842\\
1256	-30.2380988057253\\
1257	-35.5276872917213\\
1258	-67.3307228888441\\
1259	-64.9001845773853\\
1260	-84.2252921108054\\
1261	-118.445246508335\\
1262	-73.1937936560178\\
1263	-36.5390350488879\\
1264	-23.6749033904098\\
1265	-30.2641607304463\\
1266	-48.6067415380467\\
1267	-30.1585568445041\\
1268	-35.2374614122396\\
1269	-47.3136625009454\\
1270	-70.2277152121292\\
1271	-68.709398987447\\
1272	-95.8060315149141\\
1273	-61.9358023358452\\
1274	-59.7671217713421\\
1275	-21.2182057327218\\
1276	-27.2411702617796\\
1277	-16.3860377482079\\
1278	-21.1021231835323\\
1279	-27.0906237864199\\
1280	-44.7657920732945\\
1281	-46.9916801120155\\
1282	-54.8741256168159\\
1283	-59.5144930126257\\
1284	-100.844813617382\\
1285	-86.1908020423202\\
1286	-63.66851572222\\
1287	-81.9710841148903\\
1288	-85.911316077412\\
1289	-105.619778379659\\
1290	-86.2667049511431\\
1291	-46.4725860460624\\
1292	-12.2406399379518\\
1293	-8.62923124016061\\
1294	-15.628739494533\\
1295	-21.0268936584796\\
1296	-23.1705965220092\\
1297	-22.5923540344997\\
1298	-31.9200263374066\\
1299	-51.1905203558642\\
1300	-48.7931113561854\\
1301	-47.4182668314372\\
1302	-59.232333452881\\
1303	-35.737802213431\\
1304	-22.2330539804368\\
1306	-69.1832946496099\\
1307	-59.2236974031971\\
1308	-72.6646374411175\\
1310	-45.7698577369677\\
1311	-64.9930170102366\\
1312	-90.5556335271824\\
1313	-85.2048839888439\\
1314	-62.1291018168868\\
1315	-108.528157382368\\
1316	-78.3099964961557\\
1317	-73.5994807785673\\
1318	-92.9968802512335\\
1319	-85.7646316341486\\
1320	-67.0583961052275\\
1321	-60.5724171033596\\
1322	-104.547439142113\\
1323	-129.747824519237\\
1324	-98.2833304132225\\
1325	-40.0652695651172\\
1326	-47.1739739239581\\
1327	-55.0585925323937\\
1328	-67.1279814071634\\
1329	-74.5452972011285\\
1330	-59.0887708624646\\
1331	-60.8263952829197\\
1332	-88.4383932314474\\
1333	-88.9856420520894\\
1334	-61.0243579707571\\
1335	-45.24862610234\\
1336	-68.3838229134792\\
1337	-105.65686958407\\
1338	-88.7835827162453\\
1339	-92.4070698382031\\
1340	-42.3413899244588\\
1341	-49.2175820393586\\
1342	-50.1895129497082\\
1343	-30.1777453316399\\
1344	-14.4210199448921\\
1345	-11.668301997895\\
1346	-12.8182587696947\\
1347	-32.7734266068087\\
1349	-54.563584392972\\
1350	-44.5529791608315\\
1351	-50.1200997572946\\
1352	-60.2192843521777\\
1353	-35.3831988373786\\
1354	-40.0855286002509\\
1355	-41.364756123447\\
1356	-32.3724622486664\\
1357	-41.3889942458475\\
1358	-65.9873976170884\\
1359	-74.5492603625473\\
1360	-43.8818965569358\\
1361	-35.7788229311616\\
1362	-33.5283055328457\\
1363	-43.3061963973933\\
1364	-28.2923291807454\\
1365	-61.4001853417883\\
1366	-101.567712828127\\
1367	-71.7190461623327\\
1368	-106.750463344503\\
1369	-123.945263705645\\
1370	-114.987520925376\\
1372	-56.8337146951619\\
1373	-67.3943068230333\\
1374	-61.2743003400062\\
1375	-75.2748898603124\\
1376	-87.4956645347636\\
1377	-83.9795691457816\\
1378	-120.897157360328\\
1379	-129.557939241828\\
1380	-142.418805439797\\
1381	-88.6840010177425\\
1382	-116.126725045385\\
1383	-127.477631767074\\
1384	-85.0019631439498\\
1385	-115.068309792755\\
1386	-89.9332637501675\\
1387	-42.0588263117252\\
1388	-12.3985312127802\\
1389	-20.3028660153579\\
1390	-48.1034233223422\\
1391	-36.3335844260125\\
1392	-26.5913109379419\\
1393	-36.7979828379378\\
1394	-32.2160737223353\\
1395	-24.1841107839439\\
1397	-37.9373647795107\\
1398	-29.2743706927745\\
1399	-54.4646389543339\\
1400	-68.4557792595403\\
1401	-51.1093827925692\\
1403	-133.849304624378\\
1404	-99.9350923832749\\
1405	-148.485873471472\\
};
\addlegendentry{MPO prediction}

\end{axis}
\end{tikzpicture}%
%	\caption{Validation results for model estimated using OLS. Compression direction d1.}\label{fig:ols_hs}
%\end{figure}
%\begin{figure}[!h]
%	\definecolor{mycolor1}{rgb}{0.00000,0.44700,0.74100}%
%	\definecolor{mycolor2}{rgb}{0.85000,0.32500,0.09800}%
%	\centering
%	% This file was created by matlab2tikz.
%
\definecolor{mycolor1}{rgb}{0.00000,0.44700,0.74100}%
\definecolor{mycolor2}{rgb}{0.85000,0.32500,0.09800}%
\definecolor{mycolor3}{rgb}{0.92900,0.69400,0.12500}%
%
\begin{tikzpicture}

\begin{axis}[%
width=6.159cm,
height=1.831cm,
at={(0cm,10.169cm)},
scale only axis,
xmin=1000,
xmax=2000,
xlabel style={font=\color{white!15!black}},
xlabel={Sample index},
ymin=-68.7387310771141,
ymax=0,
ylabel style={font=\color{white!15!black}},
ylabel={$y(t)$},
axis background/.style={fill=white},
title style={font=\bfseries},
title={C1: RMSE(OSA) = 2.6487, RMSE(MPO) = 5.431},
legend style={legend cell align=left, align=left, draw=white!15!black}
]
\addplot [color=mycolor1, line width=2.0pt]
  table[row sep=crcr]{%
1006	-28.076\\
1007	-32.9590000000001\\
1008	-25.635\\
1009	-14.6479999999999\\
1010	-20.752\\
1011	-17.0899999999999\\
1012	-18.3109999999999\\
1013	-20.752\\
1014	-7.32400000000007\\
1015	-4.88300000000004\\
1016	-4.88300000000004\\
1017	-13.4280000000001\\
1018	-23.193\\
1020	-25.635\\
1021	-19.5309999999999\\
1022	-12.2070000000001\\
1023	-24.414\\
1025	-14.6479999999999\\
1026	-20.752\\
1027	-18.3109999999999\\
1028	-17.0899999999999\\
1029	-29.297\\
1030	-25.635\\
1031	-18.3109999999999\\
1032	-28.076\\
1033	-28.076\\
1034	-21.973\\
1035	-18.3109999999999\\
1036	-15.8689999999999\\
1037	-15.8689999999999\\
1038	-21.973\\
1041	-21.973\\
1042	-23.193\\
1043	-30.518\\
1044	-29.297\\
1045	-19.5309999999999\\
1046	-18.3109999999999\\
1047	-10.9860000000001\\
1048	-14.6479999999999\\
1049	-15.8689999999999\\
1050	-12.2070000000001\\
1051	-13.4280000000001\\
1052	-18.3109999999999\\
1053	-19.5309999999999\\
1054	-18.3109999999999\\
1055	-29.297\\
1056	-21.973\\
1057	-12.2070000000001\\
1058	-12.2070000000001\\
1059	-13.4280000000001\\
1060	-17.0899999999999\\
1061	-9.76600000000008\\
1062	-10.9860000000001\\
1063	-14.6479999999999\\
1064	-9.76600000000008\\
1065	-7.32400000000007\\
1066	-13.4280000000001\\
1067	-13.4280000000001\\
1068	-21.973\\
1069	-20.752\\
1070	-25.635\\
1071	-21.973\\
1072	-25.635\\
1073	-20.752\\
1074	-20.752\\
1075	-18.3109999999999\\
1076	-18.3109999999999\\
1077	-17.0899999999999\\
1078	-30.518\\
1079	-41.5039999999999\\
1080	-41.5039999999999\\
1081	-42.7249999999999\\
1082	-28.076\\
1083	-37.8420000000001\\
1084	-46.3869999999999\\
1085	-46.3869999999999\\
1086	-35.4000000000001\\
1087	-47.607\\
1088	-58.5940000000001\\
1089	-43.9449999999999\\
1091	-24.414\\
1093	-17.0899999999999\\
1094	-21.973\\
1096	-12.2070000000001\\
1097	-9.76600000000008\\
1098	-12.2070000000001\\
1099	-18.3109999999999\\
1101	-18.3109999999999\\
1102	-21.973\\
1103	-23.193\\
1104	-30.518\\
1105	-28.076\\
1106	-30.518\\
1108	-20.752\\
1109	-24.414\\
1110	-18.3109999999999\\
1111	-13.4280000000001\\
1112	-19.5309999999999\\
1113	-20.752\\
1114	-12.2070000000001\\
1115	-15.8689999999999\\
1116	-12.2070000000001\\
1117	-13.4280000000001\\
1118	-13.4280000000001\\
1119	-9.76600000000008\\
1120	-10.9860000000001\\
1122	-23.193\\
1123	-25.635\\
1124	-26.855\\
1125	-15.8689999999999\\
1126	-20.752\\
1127	-32.9590000000001\\
1128	-24.414\\
1129	-28.076\\
1130	-29.297\\
1131	-18.3109999999999\\
1132	-12.2070000000001\\
1133	-17.0899999999999\\
1134	-18.3109999999999\\
1135	-30.518\\
1136	-36.6210000000001\\
1137	-36.6210000000001\\
1138	-28.076\\
1139	-28.076\\
1140	-25.635\\
1142	-25.635\\
1144	-15.8689999999999\\
1146	-13.4280000000001\\
1147	-13.4280000000001\\
1148	-20.752\\
1149	-31.7380000000001\\
1150	-26.855\\
1151	-17.0899999999999\\
1152	-15.8689999999999\\
1153	-15.8689999999999\\
1154	-9.76600000000008\\
1155	-7.32400000000007\\
1156	-7.32400000000007\\
1157	-15.8689999999999\\
1158	-10.9860000000001\\
1159	-14.6479999999999\\
1160	-14.6479999999999\\
1161	-13.4280000000001\\
1163	-6.10400000000004\\
1164	-4.88300000000004\\
1165	-7.32400000000007\\
1166	-18.3109999999999\\
1167	-26.855\\
1168	-28.076\\
1170	-13.4280000000001\\
1171	-10.9860000000001\\
1172	-6.10400000000004\\
1173	-12.2070000000001\\
1174	-14.6479999999999\\
1175	-18.3109999999999\\
1176	-26.855\\
1177	-39.0630000000001\\
1178	-47.607\\
1180	-37.8420000000001\\
1181	-28.076\\
1182	-30.518\\
1183	-31.7380000000001\\
1184	-34.1800000000001\\
1185	-24.414\\
1186	-23.193\\
1187	-23.193\\
1188	-21.973\\
1189	-23.193\\
1190	-18.3109999999999\\
1191	-30.518\\
1192	-37.8420000000001\\
1194	-18.3109999999999\\
1195	-18.3109999999999\\
1196	-30.518\\
1199	-45.1659999999999\\
1200	-45.1659999999999\\
1201	-34.1800000000001\\
1202	-32.9590000000001\\
1203	-36.6210000000001\\
1204	-41.5039999999999\\
1206	-17.0899999999999\\
1207	-23.193\\
1208	-20.752\\
1209	-13.4280000000001\\
1210	-18.3109999999999\\
1211	-19.5309999999999\\
1212	-14.6479999999999\\
1213	-13.4280000000001\\
1214	-15.8689999999999\\
1215	-13.4280000000001\\
1216	-17.0899999999999\\
1217	-25.635\\
1218	-25.635\\
1219	-15.8689999999999\\
1220	-14.6479999999999\\
1221	-24.414\\
1222	-40.2829999999999\\
1223	-29.297\\
1224	-20.752\\
1225	-14.6479999999999\\
1226	-18.3109999999999\\
1227	-15.8689999999999\\
1228	-12.2070000000001\\
1229	-13.4280000000001\\
1232	-20.752\\
1233	-15.8689999999999\\
1234	-23.193\\
1235	-29.297\\
1236	-25.635\\
1237	-20.752\\
1238	-23.193\\
1239	-30.518\\
1240	-21.973\\
1241	-8.54500000000007\\
1242	-13.4280000000001\\
1243	-13.4280000000001\\
1244	-17.0899999999999\\
1245	-17.0899999999999\\
1246	-13.4280000000001\\
1247	-17.0899999999999\\
1248	-19.5309999999999\\
1249	-13.4280000000001\\
1250	-15.8689999999999\\
1251	-15.8689999999999\\
1252	-12.2070000000001\\
1253	-15.8689999999999\\
1254	-9.76600000000008\\
1255	-10.9860000000001\\
1256	-8.54500000000007\\
1257	-8.54500000000007\\
1259	-20.752\\
1260	-24.414\\
1261	-35.4000000000001\\
1263	-13.4280000000001\\
1265	-10.9860000000001\\
1266	-13.4280000000001\\
1267	-10.9860000000001\\
1268	-12.2070000000001\\
1269	-14.6479999999999\\
1270	-24.414\\
1271	-21.973\\
1272	-26.855\\
1273	-23.193\\
1274	-17.0899999999999\\
1275	-13.4280000000001\\
1278	-6.10400000000004\\
1279	-9.76600000000008\\
1280	-14.6479999999999\\
1281	-18.3109999999999\\
1282	-18.3109999999999\\
1283	-19.5309999999999\\
1284	-29.297\\
1285	-25.635\\
1286	-18.3109999999999\\
1287	-24.414\\
1288	-24.414\\
1289	-31.7380000000001\\
1290	-26.855\\
1291	-17.0899999999999\\
1292	-9.76600000000008\\
1293	-6.10400000000004\\
1294	-7.32400000000007\\
1295	-9.76600000000008\\
1296	-8.54500000000007\\
1297	-8.54500000000007\\
1298	-12.2070000000001\\
1299	-17.0899999999999\\
1300	-15.8689999999999\\
1301	-15.8689999999999\\
1302	-19.5309999999999\\
1303	-13.4280000000001\\
1304	-4.88300000000004\\
1305	-9.76600000000008\\
1306	-21.973\\
1307	-19.5309999999999\\
1308	-21.973\\
1310	-14.6479999999999\\
1311	-19.5309999999999\\
1312	-25.635\\
1313	-25.635\\
1314	-19.5309999999999\\
1315	-26.855\\
1316	-26.855\\
1317	-23.193\\
1318	-28.076\\
1319	-26.855\\
1320	-20.752\\
1321	-19.5309999999999\\
1322	-29.297\\
1323	-40.2829999999999\\
1324	-30.518\\
1325	-18.3109999999999\\
1326	-17.0899999999999\\
1327	-18.3109999999999\\
1328	-21.973\\
1329	-26.855\\
1330	-19.5309999999999\\
1331	-19.5309999999999\\
1332	-26.855\\
1333	-28.076\\
1334	-20.752\\
1335	-14.6479999999999\\
1336	-20.752\\
1337	-34.1800000000001\\
1338	-30.518\\
1339	-31.7380000000001\\
1340	-21.973\\
1341	-17.0899999999999\\
1342	-17.0899999999999\\
1343	-10.9860000000001\\
1344	-6.10400000000004\\
1345	-6.10400000000004\\
1346	-4.88300000000004\\
1347	-10.9860000000001\\
1348	-19.5309999999999\\
1349	-18.3109999999999\\
1350	-15.8689999999999\\
1351	-14.6479999999999\\
1352	-20.752\\
1353	-14.6479999999999\\
1354	-9.76600000000008\\
1355	-13.4280000000001\\
1356	-8.54500000000007\\
1357	-10.9860000000001\\
1358	-18.3109999999999\\
1359	-23.193\\
1360	-17.0899999999999\\
1361	-9.76600000000008\\
1363	-12.2070000000001\\
1364	-9.76600000000008\\
1365	-13.4280000000001\\
1366	-31.7380000000001\\
1367	-25.635\\
1369	-37.8420000000001\\
1370	-36.6210000000001\\
1371	-26.855\\
1372	-20.752\\
1373	-18.3109999999999\\
1374	-21.973\\
1375	-23.193\\
1376	-29.297\\
1377	-29.297\\
1378	-36.6210000000001\\
1379	-42.7249999999999\\
1380	-46.3869999999999\\
1381	-37.8420000000001\\
1382	-35.4000000000001\\
1383	-40.2829999999999\\
1384	-31.7380000000001\\
1385	-34.1800000000001\\
1386	-31.7380000000001\\
1387	-18.3109999999999\\
1388	-12.2070000000001\\
1389	-13.4280000000001\\
1390	-18.3109999999999\\
1391	-17.0899999999999\\
1392	-8.54500000000007\\
1393	-14.6479999999999\\
1394	-13.4280000000001\\
1395	-6.10400000000004\\
1396	-10.9860000000001\\
1397	-12.2070000000001\\
1398	-7.32400000000007\\
1399	-18.3109999999999\\
1400	-20.752\\
1401	-14.6479999999999\\
1402	-23.193\\
1403	-37.8420000000001\\
1404	-32.9590000000001\\
1405	-37.8420000000001\\
1406	-32.9590000000001\\
1407	-29.297\\
1408	-20.752\\
1409	-14.6479999999999\\
1410	-17.0899999999999\\
1411	-13.4280000000001\\
1412	-10.9860000000001\\
1413	-14.6479999999999\\
1414	-13.4280000000001\\
1415	-3.66200000000003\\
1416	-7.32400000000007\\
1417	-15.8689999999999\\
1418	-20.752\\
1420	-23.193\\
1421	-21.973\\
1422	-26.855\\
1424	-17.0899999999999\\
1425	-17.0899999999999\\
1426	-14.6479999999999\\
1427	-20.752\\
1428	-18.3109999999999\\
1429	-13.4280000000001\\
1430	-18.3109999999999\\
1431	-26.855\\
1432	-20.752\\
1433	-15.8689999999999\\
1434	-13.4280000000001\\
1435	-14.6479999999999\\
1436	-12.2070000000001\\
1437	-8.54500000000007\\
1438	-14.6479999999999\\
1439	-19.5309999999999\\
1440	-18.3109999999999\\
1441	-15.8689999999999\\
1442	-19.5309999999999\\
1443	-18.3109999999999\\
1444	-10.9860000000001\\
1445	-10.9860000000001\\
1446	-21.973\\
1447	-30.518\\
1448	-29.297\\
1449	-23.193\\
1450	-21.973\\
1451	-19.5309999999999\\
1452	-18.3109999999999\\
1453	-15.8689999999999\\
1454	-19.5309999999999\\
1455	-17.0899999999999\\
1456	-13.4280000000001\\
1457	-17.0899999999999\\
1459	-21.973\\
1460	-25.635\\
1461	-25.635\\
1462	-32.9590000000001\\
1463	-41.5039999999999\\
1464	-45.1659999999999\\
1465	-31.7380000000001\\
1466	-17.0899999999999\\
1467	-13.4280000000001\\
1468	-8.54500000000007\\
1470	-13.4280000000001\\
1471	-17.0899999999999\\
1472	-18.3109999999999\\
1473	-13.4280000000001\\
1474	-19.5309999999999\\
1475	-24.414\\
1477	-26.855\\
1478	-39.0630000000001\\
1479	-35.4000000000001\\
1480	-25.635\\
1481	-20.752\\
1482	-20.752\\
1483	-19.5309999999999\\
1484	-23.193\\
1486	-15.8689999999999\\
1487	-17.0899999999999\\
1488	-10.9860000000001\\
1489	-14.6479999999999\\
1490	-28.076\\
1491	-19.5309999999999\\
1492	-15.8689999999999\\
1493	-34.1800000000001\\
1495	-21.973\\
1496	-34.1800000000001\\
1497	-31.7380000000001\\
1498	-20.752\\
1499	-15.8689999999999\\
1500	-13.4280000000001\\
1501	-23.193\\
1502	-35.4000000000001\\
1503	-37.8420000000001\\
1504	-36.6210000000001\\
1505	-29.297\\
1506	-20.752\\
1507	-15.8689999999999\\
1508	-14.6479999999999\\
1509	-10.9860000000001\\
1510	-10.9860000000001\\
1512	-15.8689999999999\\
1513	-10.9860000000001\\
1514	-13.4280000000001\\
1515	-21.973\\
1516	-21.973\\
1517	-26.855\\
1519	-43.9449999999999\\
1520	-28.076\\
1522	-28.076\\
1523	-24.414\\
1524	-18.3109999999999\\
1525	-20.752\\
1526	-34.1800000000001\\
1527	-37.8420000000001\\
1528	-24.414\\
1529	-12.2070000000001\\
1530	-10.9860000000001\\
1531	-3.66200000000003\\
1532	-8.54500000000007\\
1533	-9.76600000000008\\
1534	-12.2070000000001\\
1535	-13.4280000000001\\
1536	-13.4280000000001\\
1537	-7.32400000000007\\
1538	-8.54500000000007\\
1539	-10.9860000000001\\
1540	-8.54500000000007\\
1541	-12.2070000000001\\
1542	-14.6479999999999\\
1543	-24.414\\
1544	-28.076\\
1545	-24.414\\
1546	-28.076\\
1547	-35.4000000000001\\
1548	-35.4000000000001\\
1549	-40.2829999999999\\
1550	-40.2829999999999\\
1551	-29.297\\
1552	-19.5309999999999\\
1553	-18.3109999999999\\
1554	-30.518\\
1555	-37.8420000000001\\
1557	-15.8689999999999\\
1558	-9.76600000000008\\
1559	-4.88300000000004\\
1560	-6.10400000000004\\
1562	-6.10400000000004\\
1563	-12.2070000000001\\
1564	-14.6479999999999\\
1566	-9.76600000000008\\
1567	-4.88300000000004\\
1568	-8.54500000000007\\
1570	-8.54500000000007\\
1571	-7.32400000000007\\
1572	-4.88300000000004\\
1573	-12.2070000000001\\
1574	-25.635\\
1575	-24.414\\
1576	-31.7380000000001\\
1577	-25.635\\
1578	-32.9590000000001\\
1579	-46.3869999999999\\
1581	-39.0630000000001\\
1582	-34.1800000000001\\
1583	-31.7380000000001\\
1584	-32.9590000000001\\
1585	-20.752\\
1586	-20.752\\
1587	-13.4280000000001\\
1588	-14.6479999999999\\
1589	-23.193\\
1590	-20.752\\
1591	-15.8689999999999\\
1592	-19.5309999999999\\
1593	-17.0899999999999\\
1594	-17.0899999999999\\
1595	-20.752\\
1596	-19.5309999999999\\
1597	-24.414\\
1598	-20.752\\
1599	-15.8689999999999\\
1600	-18.3109999999999\\
1601	-14.6479999999999\\
1602	-2.44100000000003\\
1603	-9.76600000000008\\
1604	-13.4280000000001\\
1605	-8.54500000000007\\
1606	-4.88300000000004\\
1608	-12.2070000000001\\
1609	-12.2070000000001\\
1610	-14.6479999999999\\
1611	-13.4280000000001\\
1612	-19.5309999999999\\
1613	-19.5309999999999\\
1614	-12.2070000000001\\
1615	-19.5309999999999\\
1616	-17.0899999999999\\
1617	-7.32400000000007\\
1618	-7.32400000000007\\
1619	-3.66200000000003\\
1620	-10.9860000000001\\
1621	-8.54500000000007\\
1622	-7.32400000000007\\
1623	-19.5309999999999\\
1624	-20.752\\
1625	-10.9860000000001\\
1626	-14.6479999999999\\
1627	-12.2070000000001\\
1629	-14.6479999999999\\
1630	-8.54500000000007\\
1631	-10.9860000000001\\
1632	-7.32400000000007\\
1633	-7.32400000000007\\
1634	-9.76600000000008\\
1635	-14.6479999999999\\
1636	-14.6479999999999\\
1637	-9.76600000000008\\
1640	-9.76600000000008\\
1642	-26.855\\
1643	-18.3109999999999\\
1645	-15.8689999999999\\
1646	-21.973\\
1647	-21.973\\
1648	-13.4280000000001\\
1649	-14.6479999999999\\
1650	-13.4280000000001\\
1651	-17.0899999999999\\
1652	-12.2070000000001\\
1653	-8.54500000000007\\
1654	-10.9860000000001\\
1655	-15.8689999999999\\
1656	-12.2070000000001\\
1657	-14.6479999999999\\
1658	-14.6479999999999\\
1659	-21.973\\
1660	-18.3109999999999\\
1661	-25.635\\
1662	-35.4000000000001\\
1663	-28.076\\
1664	-18.3109999999999\\
1665	-15.8689999999999\\
1666	-23.193\\
1667	-28.076\\
1668	-20.752\\
1669	-23.193\\
1670	-18.3109999999999\\
1671	-7.32400000000007\\
1672	-8.54500000000007\\
1673	-20.752\\
1674	-18.3109999999999\\
1675	-18.3109999999999\\
1676	-23.193\\
1677	-23.193\\
1679	-15.8689999999999\\
1680	-18.3109999999999\\
1681	-23.193\\
1682	-19.5309999999999\\
1683	-18.3109999999999\\
1684	-19.5309999999999\\
1685	-13.4280000000001\\
1686	-14.6479999999999\\
1687	-13.4280000000001\\
1688	-13.4280000000001\\
1689	-20.752\\
1690	-25.635\\
1691	-23.193\\
1692	-23.193\\
1693	-17.0899999999999\\
1694	-12.2070000000001\\
1695	-15.8689999999999\\
1696	-15.8689999999999\\
1697	-21.973\\
1698	-24.414\\
1699	-28.076\\
1700	-23.193\\
1701	-13.4280000000001\\
1702	-24.414\\
1703	-36.6210000000001\\
1704	-24.414\\
1705	-23.193\\
1706	-29.297\\
1707	-21.973\\
1708	-20.752\\
1709	-20.752\\
1710	-18.3109999999999\\
1711	-20.752\\
1712	-25.635\\
1713	-19.5309999999999\\
1714	-21.973\\
1715	-21.973\\
1717	-19.5309999999999\\
1718	-21.973\\
1719	-17.0899999999999\\
1720	-18.3109999999999\\
1721	-17.0899999999999\\
1722	-17.0899999999999\\
1723	-8.54500000000007\\
1724	-2.44100000000003\\
1725	-2.44100000000003\\
1726	-8.54500000000007\\
1727	-19.5309999999999\\
1728	-25.635\\
1729	-23.193\\
1730	-17.0899999999999\\
1731	-20.752\\
1732	-19.5309999999999\\
1734	-12.2070000000001\\
1735	-14.6479999999999\\
1736	-14.6479999999999\\
1737	-12.2070000000001\\
1738	-15.8689999999999\\
1741	-8.54500000000007\\
1742	-14.6479999999999\\
1743	-21.973\\
1744	-17.0899999999999\\
1746	-17.0899999999999\\
1747	-23.193\\
1748	-20.752\\
1749	-28.076\\
1750	-21.973\\
1751	-23.193\\
1752	-23.193\\
1753	-15.8689999999999\\
1754	-20.752\\
1755	-19.5309999999999\\
1757	-21.973\\
1758	-25.635\\
1759	-19.5309999999999\\
1760	-24.414\\
1761	-30.518\\
1762	-25.635\\
1763	-23.193\\
1764	-18.3109999999999\\
1765	-21.973\\
1766	-19.5309999999999\\
1767	-14.6479999999999\\
1768	-13.4280000000001\\
1769	-15.8689999999999\\
1770	-13.4280000000001\\
1771	-26.855\\
1772	-35.4000000000001\\
1773	-29.297\\
1775	-29.297\\
1776	-18.3109999999999\\
1777	-14.6479999999999\\
1778	-12.2070000000001\\
1779	-10.9860000000001\\
1780	-10.9860000000001\\
1781	-18.3109999999999\\
1782	-23.193\\
1783	-25.635\\
1784	-21.973\\
1785	-15.8689999999999\\
1786	-25.635\\
1787	-30.518\\
1788	-31.7380000000001\\
1789	-25.635\\
1790	-42.7249999999999\\
1791	-32.9590000000001\\
1792	-18.3109999999999\\
1793	-13.4280000000001\\
1794	-13.4280000000001\\
1795	-19.5309999999999\\
1796	-26.855\\
1797	-31.7380000000001\\
1798	-26.855\\
1799	-19.5309999999999\\
1800	-13.4280000000001\\
1802	-13.4280000000001\\
1803	-15.8689999999999\\
1804	-9.76600000000008\\
1805	-9.76600000000008\\
};
\addlegendentry{True output}

\addplot [color=mycolor2, dashed, line width=2.0pt]
  table[row sep=crcr]{%
1006	-29.2441057622657\\
1007	-32.3902621771167\\
1008	-27.3837868832763\\
1009	-14.6035404206339\\
1010	-18.9702236356895\\
1011	-19.8026453614564\\
1012	-21.9878645537181\\
1013	-18.9974074558529\\
1014	-11.6245016558196\\
1015	-13.0114128118389\\
1016	-8.87855040430918\\
1017	-11.8211575016446\\
1018	-30.421687712791\\
1019	-26.7945075155137\\
1020	-27.0326040654804\\
1021	-19.8633501147619\\
1022	-15.180756800467\\
1023	-25.4041835296027\\
1024	-18.3198379289081\\
1025	-16.1698381379583\\
1026	-21.4757525440721\\
1027	-21.3135122293925\\
1028	-17.8126458077127\\
1029	-26.803174520121\\
1030	-26.3862533635854\\
1031	-21.6445632067059\\
1032	-27.1897321016911\\
1033	-28.9326427729061\\
1034	-23.7544127232402\\
1035	-20.3288765507627\\
1036	-16.7491385163812\\
1037	-19.9934364336852\\
1038	-20.9761557890552\\
1039	-21.0808789952125\\
1040	-21.7075069302161\\
1041	-23.5392674778184\\
1042	-23.6246188996804\\
1043	-33.7845326475715\\
1044	-27.9789449562516\\
1045	-21.9377762251288\\
1046	-20.43317292446\\
1047	-14.057231888793\\
1048	-13.5915117943616\\
1049	-17.5778479462767\\
1050	-13.6061209678821\\
1051	-14.8098973216788\\
1052	-18.206103539751\\
1053	-21.1695696784259\\
1054	-19.540230434314\\
1055	-28.2918940362799\\
1056	-23.1002545841443\\
1057	-16.3767058401261\\
1058	-13.8377335140979\\
1059	-16.7031002665767\\
1060	-16.8524021080566\\
1061	-11.6794337753126\\
1062	-12.4897546658931\\
1063	-13.8399493788068\\
1064	-13.3751856903202\\
1065	-11.4120156890317\\
1066	-12.2478690085213\\
1067	-14.3164847342191\\
1068	-22.6018558059161\\
1069	-21.5229271310757\\
1070	-24.699475850374\\
1071	-24.2997105797372\\
1072	-25.1538987961308\\
1073	-22.9511429764293\\
1074	-21.51697208835\\
1075	-20.9669957258202\\
1076	-18.4892297899514\\
1077	-20.2708941425278\\
1078	-28.745884991303\\
1079	-47.9541166175095\\
1080	-44.4857156662245\\
1081	-40.0110822427678\\
1082	-26.9277529630888\\
1083	-39.8179864641368\\
1084	-50.7997905609093\\
1085	-47.7907162255442\\
1086	-36.8076120396045\\
1087	-45.5986924099361\\
1088	-65.5490495633719\\
1089	-45.1098819580925\\
1090	-35.4938259370304\\
1091	-27.1453913842499\\
1092	-19.860396224175\\
1093	-18.943990753678\\
1094	-21.9397955671989\\
1095	-16.8194885126024\\
1096	-12.4648793738438\\
1097	-12.5954827637338\\
1098	-13.8753552997593\\
1099	-19.3240647786965\\
1100	-18.2914755171623\\
1101	-18.9643526408747\\
1102	-23.8946661236835\\
1103	-24.538408203955\\
1104	-34.3731260420118\\
1105	-28.1926572928164\\
1106	-31.4318205797679\\
1107	-25.2397057417897\\
1108	-22.4310263772677\\
1109	-26.5108375904322\\
1110	-19.277628556684\\
1111	-19.191327959817\\
1112	-19.4824214084231\\
1113	-20.9472914392018\\
1114	-17.237080160721\\
1115	-15.8934438031984\\
1116	-13.6929793982513\\
1117	-14.5473663537368\\
1118	-15.6520903317762\\
1119	-12.1384571710635\\
1120	-13.5235426506692\\
1121	-15.1872086984681\\
1122	-25.4881311847948\\
1123	-25.6490064173765\\
1124	-27.664321832933\\
1125	-16.3908606914918\\
1126	-22.5441951071061\\
1127	-37.5575251586649\\
1128	-25.4025827613966\\
1129	-28.4413622611935\\
1130	-29.5122062376054\\
1131	-20.6303860135924\\
1132	-15.021848494343\\
1133	-18.643554296963\\
1134	-20.9752994416904\\
1135	-30.4497758615598\\
1136	-41.5601469140336\\
1137	-37.3852745697473\\
1138	-27.1800246011858\\
1139	-30.1858492516922\\
1140	-27.1996678966614\\
1141	-27.2521057869837\\
1142	-26.9505617291684\\
1143	-18.390544429823\\
1144	-18.3359759552654\\
1145	-17.6205341424829\\
1146	-15.2070503900477\\
1147	-17.1192787388072\\
1148	-21.3412059784332\\
1149	-36.2798523546123\\
1150	-25.1428081326419\\
1151	-19.2237327343498\\
1152	-17.4082347722974\\
1153	-17.6501301763437\\
1154	-12.6659445505036\\
1155	-10.8638476014783\\
1156	-10.0745206194581\\
1157	-14.4423249410149\\
1158	-13.2862249227737\\
1159	-13.566279058778\\
1160	-15.3085561550656\\
1161	-14.8394741856132\\
1162	-12.8718216276047\\
1163	-10.9805828728367\\
1164	-9.28588527083366\\
1165	-8.31770260493909\\
1166	-19.181924198369\\
1167	-31.8793437999955\\
1168	-32.4558736310173\\
1169	-18.7671503733395\\
1170	-17.5877080479781\\
1171	-13.6564586296267\\
1172	-11.6375344983876\\
1173	-13.7133203441224\\
1174	-14.4198319094289\\
1175	-20.9495802746537\\
1176	-26.5364912138937\\
1177	-45.9825858493223\\
1178	-57.6815891014542\\
1179	-40.1063246561389\\
1180	-40.1173785056458\\
1181	-27.4331023718223\\
1182	-30.885010628107\\
1183	-33.5004148381336\\
1184	-33.4428280860102\\
1185	-28.2917697712332\\
1186	-23.7936967989733\\
1187	-26.2014558884123\\
1188	-21.5129426746396\\
1189	-25.9327754132566\\
1190	-19.3384926418237\\
1191	-33.8316054371371\\
1192	-38.7696044115808\\
1193	-26.8219134280498\\
1194	-17.6243917798424\\
1195	-20.6567831308309\\
1196	-32.1445289477003\\
1198	-46.1694855301928\\
1199	-44.9113914575933\\
1200	-46.1515264324528\\
1201	-36.2330987791222\\
1202	-33.277759560827\\
1203	-38.3585893508052\\
1204	-42.7499253019885\\
1205	-26.9066576148068\\
1206	-17.7137756251173\\
1207	-26.9506658806217\\
1208	-21.2588954078165\\
1209	-14.6706926404822\\
1210	-19.7873836811832\\
1211	-21.5959582378875\\
1212	-17.7863135819375\\
1213	-13.8710535891619\\
1214	-17.6813978208604\\
1215	-15.5941622723299\\
1216	-18.710634866359\\
1217	-28.1503836854126\\
1218	-22.4334858352961\\
1219	-19.4612598042597\\
1220	-18.6521639946013\\
1221	-27.6793511652938\\
1222	-44.153023932864\\
1223	-30.5505184445756\\
1224	-20.6490816948433\\
1225	-17.1743532713881\\
1226	-18.60521759262\\
1227	-18.5738386325186\\
1228	-13.9271892899233\\
1229	-15.1504463115941\\
1230	-17.9898901280235\\
1231	-20.984732470461\\
1232	-21.7753616230814\\
1233	-15.7190266484702\\
1234	-25.7200603673361\\
1235	-30.503518829863\\
1236	-27.5963120725082\\
1237	-22.1195811707312\\
1238	-24.8516143574295\\
1239	-31.838865984317\\
1240	-20.0884358289152\\
1241	-13.6349570053799\\
1242	-14.0638141173588\\
1243	-16.0874734007243\\
1244	-19.7530556296958\\
1245	-18.9285002140468\\
1246	-14.17108634636\\
1247	-20.0692769784437\\
1248	-19.116828737702\\
1249	-16.0065890355843\\
1250	-18.0322729431525\\
1251	-15.6873683689701\\
1253	-17.1815442265502\\
1254	-10.8337844369271\\
1255	-12.9508998914189\\
1256	-10.5241616746168\\
1257	-11.2991984658559\\
1258	-17.4606124231188\\
1259	-19.912511205868\\
1260	-28.3445506601395\\
1261	-35.3668848402069\\
1262	-23.947267788006\\
1263	-17.5824177940424\\
1264	-14.4625225440129\\
1265	-13.9906373863223\\
1266	-16.1638788380205\\
1267	-11.2063210417596\\
1268	-13.0997390609407\\
1269	-15.2020259222923\\
1270	-24.6005094150753\\
1271	-24.6901355296784\\
1272	-30.6675968165671\\
1273	-19.9533922152368\\
1274	-21.902214118503\\
1275	-12.3846377403972\\
1276	-14.4811565319715\\
1277	-11.0555289947006\\
1278	-10.6483893833345\\
1279	-9.23071236577016\\
1280	-16.0054674581932\\
1281	-17.4750560953707\\
1282	-20.1451341808925\\
1283	-19.8365485779975\\
1284	-33.3239173894799\\
1285	-30.4191322329084\\
1286	-20.1021081799713\\
1287	-24.9889826180379\\
1288	-25.66605700319\\
1289	-31.3469604106158\\
1290	-25.353300104191\\
1291	-20.3773797563456\\
1292	-12.561998856431\\
1293	-12.8971463012981\\
1294	-9.61548774065091\\
1295	-9.39102905458594\\
1296	-10.4129383589359\\
1297	-10.0260628257738\\
1298	-12.4073821091802\\
1299	-17.0276967251286\\
1300	-17.0134831371056\\
1302	-18.848970421413\\
1303	-14.1325677324844\\
1304	-13.3643081242567\\
1305	-11.8640863838996\\
1306	-20.4981972158355\\
1307	-19.8068306545385\\
1308	-21.4787991347368\\
1309	-20.7791497121495\\
1310	-16.3233953824406\\
1311	-19.8146236734078\\
1312	-27.1307893071546\\
1313	-27.3674323745477\\
1314	-19.4812693774397\\
1315	-31.9390361527917\\
1316	-24.3577069844987\\
1317	-23.8664399348811\\
1318	-27.6639836667694\\
1319	-26.5726148125038\\
1320	-23.6075149797912\\
1321	-19.701520370367\\
1322	-28.6844996161258\\
1323	-46.2666147136285\\
1324	-31.9955493440712\\
1325	-18.7760920752009\\
1326	-20.2312941256582\\
1327	-19.7006147218435\\
1328	-24.6480670268929\\
1329	-27.0194932162765\\
1330	-19.8135004230439\\
1331	-22.7168889951236\\
1332	-27.4154657004879\\
1333	-30.2175720776113\\
1334	-19.6851413121492\\
1335	-18.6587028552392\\
1336	-21.4256970459442\\
1337	-35.0622326744624\\
1338	-33.8532389505092\\
1339	-28.528391259782\\
1340	-21.1091485220568\\
1341	-20.4462199153479\\
1342	-19.8262525189568\\
1343	-14.0584789380353\\
1344	-11.1778894028\\
1345	-9.48640925629434\\
1346	-8.56664679466053\\
1347	-9.52401486514373\\
1348	-18.5850028120685\\
1349	-22.1914244724599\\
1350	-16.7183101117375\\
1351	-17.4221406151105\\
1352	-19.2612055169736\\
1353	-14.0822423732996\\
1354	-15.9851077278834\\
1355	-13.0704940894504\\
1357	-12.4783902149059\\
1359	-25.2975708654033\\
1360	-15.3070371492347\\
1361	-15.7173702067867\\
1362	-11.887918300994\\
1363	-13.9544882607622\\
1364	-10.9303682367649\\
1365	-17.9573550273556\\
1366	-34.6754994879961\\
1367	-23.3248212495071\\
1368	-31.667173888088\\
1369	-39.364443365117\\
1370	-37.6007074133288\\
1371	-30.9463446522072\\
1372	-21.1829985310292\\
1373	-23.4409341416751\\
1374	-21.9603836282081\\
1375	-25.5754261633617\\
1376	-27.7305049731231\\
1377	-28.2997364387631\\
1378	-38.1023772608503\\
1379	-44.9869593480403\\
1380	-50.735244032682\\
1381	-34.5920998839783\\
1382	-37.3515222966316\\
1383	-44.2204798307275\\
1384	-31.1674301861601\\
1385	-38.1094616086666\\
1386	-31.8518099808496\\
1387	-18.7648393136928\\
1388	-13.7339894304223\\
1389	-14.4168261552263\\
1390	-20.5679175116632\\
1391	-15.810715991737\\
1392	-13.6662354048633\\
1393	-14.7306711959793\\
1395	-11.90433052918\\
1396	-11.1363796096516\\
1397	-12.7600132839025\\
1398	-11.6634411152411\\
1399	-14.293743345034\\
1400	-22.6428248214013\\
1401	-17.8050145182449\\
1402	-26.0213027648713\\
1403	-42.470224520818\\
1404	-36.0486833309251\\
1405	-43.0289495321913\\
1406	-28.3762800395116\\
1407	-33.6951579613024\\
1408	-21.1357267055794\\
1409	-16.2960062875197\\
1410	-18.5461032833991\\
1411	-14.9665431158094\\
1412	-12.8513305384679\\
1413	-15.8854560606592\\
1414	-10.0435472823153\\
1415	-12.6970073396494\\
1416	-8.37117878782169\\
1417	-14.7984669986583\\
1418	-23.6745743401937\\
1419	-25.3122014476228\\
1420	-24.8401279683251\\
1421	-22.7057593405113\\
1422	-25.6951752005639\\
1423	-24.0258583132679\\
1424	-19.3438032013939\\
1425	-20.4220158122373\\
1426	-16.6064407322879\\
1427	-23.0596182339905\\
1428	-15.8463198289624\\
1429	-16.0257879457556\\
1430	-20.3468387410837\\
1431	-27.9300021404083\\
1432	-19.7625210172073\\
1433	-18.4958774600063\\
1434	-16.1492418657865\\
1435	-17.6094095310364\\
1436	-13.1588701511871\\
1437	-12.6396014325305\\
1438	-14.7266338545469\\
1439	-17.7643410219844\\
1440	-21.0751545713997\\
1441	-17.4130248110894\\
1442	-19.830175138329\\
1443	-20.9382146768519\\
1444	-13.1521400015786\\
1445	-14.2476905285696\\
1446	-21.8145186015113\\
1447	-33.8712591069952\\
1448	-28.8271614873142\\
1449	-26.5884008756141\\
1450	-23.2490055124824\\
1451	-21.0859795448407\\
1452	-19.0815343591357\\
1453	-17.9724272354185\\
1454	-21.3828293583399\\
1455	-18.3229366725207\\
1456	-14.1121777607896\\
1457	-18.3326577093735\\
1458	-21.0345995296316\\
1459	-23.422612580857\\
1460	-25.545718411237\\
1461	-24.4694404665372\\
1462	-34.7436660507915\\
1463	-43.2639446732856\\
1464	-51.1275674738079\\
1465	-29.2567010845073\\
1466	-20.4478877202655\\
1467	-15.1414974166344\\
1468	-12.8063968380507\\
1469	-13.2549557323064\\
1470	-12.0488434969\\
1471	-20.9769096373327\\
1472	-18.2126691658596\\
1473	-17.2741301438668\\
1474	-19.4947676102088\\
1475	-23.1954873851289\\
1476	-27.177776035996\\
1477	-28.6925175709241\\
1478	-41.642201400153\\
1479	-39.9294789974397\\
1480	-27.3180215524885\\
1481	-19.7120060477673\\
1482	-21.9967950966652\\
1483	-22.340210662861\\
1484	-26.5337792142036\\
1485	-18.0698298518753\\
1486	-20.5982105707844\\
1487	-18.2597784383192\\
1488	-12.0055889424409\\
1489	-17.2884296280399\\
1490	-24.6781519129938\\
1491	-19.7374331317546\\
1492	-20.9182814005151\\
1493	-35.6462751846427\\
1494	-27.7210567773395\\
1495	-26.0679247617463\\
1496	-36.4987072781469\\
1497	-33.005341732827\\
1498	-20.9181539962435\\
1499	-14.7518086902719\\
1500	-17.9940603760715\\
1501	-23.7943808785958\\
1502	-41.8210362638372\\
1503	-38.9908641441652\\
1504	-38.4898551991498\\
1505	-29.8441364401165\\
1506	-20.5222579262445\\
1507	-19.8361545960138\\
1508	-14.3890295112717\\
1509	-14.5683888538488\\
1510	-13.3374886033603\\
1511	-15.3403678187817\\
1512	-17.6549969253579\\
1513	-11.4247040518603\\
1514	-15.291953093219\\
1515	-21.4153399471556\\
1516	-24.3502297766295\\
1517	-28.9574172542261\\
1518	-37.6036257700734\\
1519	-47.9660896881264\\
1520	-34.8406660704504\\
1521	-27.2606566263801\\
1522	-30.6798428555551\\
1523	-25.427348548161\\
1524	-18.6417074057256\\
1525	-21.4550124682271\\
1526	-34.9836161519027\\
1527	-40.660546582995\\
1528	-21.5770441274494\\
1529	-16.0759204314249\\
1530	-12.6632878313478\\
1531	-11.8716825342772\\
1532	-9.23380731310203\\
1533	-9.91530916713418\\
1534	-12.5820556199783\\
1535	-17.4004702272966\\
1536	-14.6169625540831\\
1537	-12.0096785833455\\
1538	-11.6993734829334\\
1539	-9.75490629309206\\
1540	-11.6622484730169\\
1541	-11.1161380159106\\
1542	-16.1888335207454\\
1543	-26.2651330557371\\
1544	-28.110037798343\\
1545	-25.8107224404221\\
1546	-29.9533220080839\\
1547	-36.6867332612694\\
1548	-38.844271700488\\
1549	-41.3787171523429\\
1550	-40.8030150367754\\
1551	-29.0400290960351\\
1552	-23.0378392274424\\
1553	-20.9037618232958\\
1554	-32.3150390637791\\
1555	-36.412154562689\\
1556	-26.257062829505\\
1557	-19.567945727548\\
1558	-11.3576003607232\\
1559	-13.02919390229\\
1560	-8.99734714884903\\
1561	-7.44740127560772\\
1562	-8.79580089139085\\
1563	-12.3258882206112\\
1564	-12.8394654003316\\
1565	-13.5785204881145\\
1566	-10.882163122935\\
1567	-11.1427401939472\\
1568	-8.85698403261677\\
1569	-9.75298017725504\\
1570	-10.930273194482\\
1571	-8.86485612416845\\
1572	-9.37845289567463\\
1573	-9.59531159160792\\
1574	-29.4170267761938\\
1575	-37.5183111874319\\
1576	-33.1125270938976\\
1577	-24.743484714799\\
1578	-32.465282563586\\
1579	-52.3655304956499\\
1580	-43.5085213291059\\
1581	-43.8657568327328\\
1582	-30.5484700736076\\
1583	-34.3048511577908\\
1584	-36.0417159432996\\
1585	-24.2014444197471\\
1586	-21.6518896368302\\
1587	-12.7628997803206\\
1588	-14.0981992954673\\
1589	-24.325571993882\\
1590	-16.8932861196968\\
1591	-20.4477880427221\\
1592	-21.6704463207363\\
1593	-19.0821945117134\\
1594	-19.8836592115822\\
1595	-20.4299784333919\\
1596	-22.5139749326968\\
1597	-22.874321855073\\
1598	-22.4401099517684\\
1599	-17.3261024992862\\
1600	-21.3681311924167\\
1601	-11.2677942931603\\
1602	-14.5303159820248\\
1603	-8.36809112924266\\
1604	-11.5880823288492\\
1605	-9.61613809616279\\
1606	-13.056813292349\\
1607	-6.9832279450477\\
1608	-9.68588627456938\\
1609	-15.2120306184943\\
1610	-17.0331131524674\\
1611	-15.1652527616313\\
1612	-22.2185590346521\\
1613	-20.3554463294913\\
1614	-15.5837773291873\\
1615	-18.6255982215787\\
1616	-12.8928189108783\\
1617	-15.2970192522923\\
1618	-10.2525286416756\\
1619	-10.317637100711\\
1620	-7.11849422508067\\
1621	-8.35854550274439\\
1622	-10.6175820477029\\
1623	-16.8614735140156\\
1624	-18.5499630444854\\
1625	-16.3639372468256\\
1626	-14.5340457001705\\
1627	-13.914996701372\\
1628	-15.708449592131\\
1629	-12.1111711774813\\
1630	-13.7984726157204\\
1631	-10.9934428611973\\
1632	-10.640757842428\\
1633	-10.5492585639986\\
1634	-8.78876970783062\\
1635	-13.173985617379\\
1636	-13.9399420277489\\
1638	-12.2998178749101\\
1639	-10.5180642905534\\
1640	-11.2516327827518\\
1641	-18.1096963553914\\
1642	-29.9200663026852\\
1643	-19.6878502472398\\
1644	-19.3887789486828\\
1645	-14.4473399960902\\
1646	-21.6133971486161\\
1647	-22.2823402255508\\
1648	-15.682582399498\\
1649	-18.1504062779475\\
1650	-12.9261431925033\\
1651	-17.0594351836451\\
1652	-12.5677638579432\\
1653	-13.4889548207932\\
1654	-13.370894606338\\
1655	-15.4929969914513\\
1656	-13.4192068985496\\
1657	-13.685657311351\\
1658	-15.1108458784511\\
1659	-21.3433385095911\\
1660	-19.9494528296179\\
1661	-27.8926416688387\\
1662	-38.1627845270511\\
1664	-21.5884921116917\\
1665	-16.5232333497138\\
1666	-22.517625931099\\
1667	-29.7847007366142\\
1668	-21.8123621481443\\
1669	-24.6417886430033\\
1670	-17.0765174815983\\
1671	-13.9455587426912\\
1672	-11.6947228576075\\
1673	-17.9519402169065\\
1674	-20.7019788949258\\
1675	-16.4199560387945\\
1676	-24.919032362282\\
1677	-23.2596691869187\\
1678	-21.9494591441646\\
1679	-16.7778916447919\\
1680	-19.1848421679836\\
1681	-24.7641622993017\\
1682	-20.2592414606238\\
1683	-20.6568825015643\\
1684	-18.4813150671625\\
1685	-16.564970039753\\
1686	-16.7571888936279\\
1687	-13.4296261643954\\
1688	-15.115092091841\\
1689	-21.4376527198776\\
1690	-25.2233655924997\\
1691	-25.3713851719001\\
1692	-22.8932999033022\\
1693	-18.5591184007683\\
1694	-15.2436506933689\\
1695	-15.1486585648515\\
1696	-18.6775053159909\\
1697	-21.2144193756774\\
1698	-25.5750870638717\\
1699	-30.0680380028618\\
1700	-24.4911773050458\\
1701	-16.2276495678639\\
1702	-23.8707763879277\\
1703	-36.0562082829163\\
1704	-25.9950236498814\\
1705	-24.6582283409189\\
1706	-28.9865038684168\\
1707	-24.0949826672472\\
1708	-19.3935041412847\\
1709	-23.4030738219208\\
1710	-21.0894592752284\\
1711	-21.131128128256\\
1712	-26.670448163346\\
1713	-20.5069751354215\\
1714	-23.2502940248928\\
1715	-22.3365271824171\\
1716	-23.1184756249045\\
1717	-18.2142980391159\\
1718	-22.5646487190825\\
1719	-17.7069694784545\\
1720	-18.5636036616215\\
1721	-20.2689038359458\\
1722	-19.7937098744749\\
1723	-11.6982725776836\\
1724	-11.6293180747334\\
1725	-8.41974100382549\\
1726	-5.91147050027325\\
1727	-17.7700235187756\\
1728	-30.6098725562902\\
1729	-24.86542017494\\
1730	-18.8242881498752\\
1731	-20.1615314609503\\
1732	-20.2680230024505\\
1733	-17.826717506601\\
1734	-13.5466439943609\\
1735	-15.6824045034984\\
1736	-15.2666127097107\\
1738	-16.2687494070533\\
1739	-15.010268499939\\
1740	-14.5823292499176\\
1741	-10.5199190482867\\
1742	-12.0222180591647\\
1743	-22.1203980702105\\
1744	-16.7414923722215\\
1745	-20.2790849772691\\
1746	-18.0434536486891\\
1747	-22.2229426612789\\
1748	-23.3659547396753\\
1749	-27.9635670442269\\
1750	-22.7119924776093\\
1751	-23.414152479656\\
1752	-25.6239701710044\\
1753	-14.7507211889063\\
1754	-22.8232358085966\\
1755	-16.5892885422049\\
1756	-21.3502350631363\\
1757	-22.1040751173887\\
1758	-26.1399005279136\\
1759	-22.5097489643933\\
1760	-24.9396170249906\\
1761	-31.8911741418433\\
1762	-24.9630735510668\\
1763	-24.1420543814809\\
1764	-20.34688283271\\
1765	-21.7412110416928\\
1766	-20.24847180558\\
1767	-18.8968323912125\\
1768	-15.5801718156299\\
1769	-16.6221631220003\\
1770	-16.1138238634542\\
1771	-26.1134061849077\\
1772	-36.4377907091607\\
1773	-30.4222732194096\\
1774	-29.1174129888907\\
1775	-28.5485825676046\\
1776	-19.1577623790449\\
1777	-18.4012095293908\\
1778	-15.1383515487657\\
1779	-13.2913328501661\\
1780	-13.6637792643437\\
1781	-18.5467553599108\\
1782	-23.8315739676357\\
1783	-26.2249073530024\\
1784	-22.4593786357052\\
1785	-17.1715876896553\\
1786	-25.9469346818983\\
1787	-32.7357704723252\\
1788	-34.3550146272814\\
1789	-27.276809084047\\
1790	-43.4556458934808\\
1791	-34.9525132890869\\
1792	-19.1021380799411\\
1793	-16.3340029970041\\
1794	-14.2158690928243\\
1796	-26.2915196616489\\
1797	-35.2176456908223\\
1798	-26.7849972015915\\
1799	-22.1563982498169\\
1800	-14.8510504170952\\
1801	-14.5114865644503\\
1802	-15.979570113371\\
1803	-15.5312822566418\\
1804	-12.5562390588645\\
1805	-13.1062638324208\\
};
\addlegendentry{OSA predition}

\addplot [color=mycolor3, dotted, line width=2.0pt]
  table[row sep=crcr]{%
1006	-28.076\\
1007	-32.9590000000001\\
1008	-25.635\\
1009	-14.6479999999999\\
1010	-18.9702236356895\\
1011	-19.137769017907\\
1012	-22.5296921457948\\
1013	-20.6396618397639\\
1014	-12.0023264489146\\
1015	-14.9240775587991\\
1016	-13.6183003265924\\
1017	-16.7687312280152\\
1018	-34.977932301025\\
1020	-33.4407821553698\\
1021	-25.4666282944581\\
1022	-19.9911612221304\\
1023	-30.9402457828623\\
1024	-23.0372220338973\\
1025	-19.4678534805953\\
1026	-25.1545582492249\\
1027	-24.5541160329026\\
1028	-21.5040130853959\\
1029	-30.2801738997134\\
1030	-28.496666969228\\
1031	-23.9102861963349\\
1032	-30.1883605266164\\
1033	-30.8738196680672\\
1034	-25.9469257497451\\
1035	-22.8910802134285\\
1036	-19.3455066520978\\
1037	-22.4913934913204\\
1038	-24.7606496697954\\
1039	-23.7676116972443\\
1040	-23.7817670687491\\
1041	-25.3885015826813\\
1042	-25.5946498873043\\
1043	-35.5516548403539\\
1044	-30.777757262382\\
1045	-23.6860152883944\\
1046	-22.7958036070447\\
1047	-16.8185060283024\\
1048	-16.6726183948481\\
1049	-19.8553711151017\\
1050	-16.2364695166334\\
1051	-17.538139471199\\
1052	-20.8511282275206\\
1053	-23.491518386945\\
1054	-22.1930006561847\\
1055	-30.9909914190289\\
1056	-24.90955938107\\
1057	-18.3412365342976\\
1058	-16.9508379579245\\
1059	-19.6334343852436\\
1060	-20.6404205105162\\
1061	-14.6493943185287\\
1062	-15.5882071408441\\
1063	-17.0444663817545\\
1064	-15.5643344080891\\
1065	-14.6746715262002\\
1066	-16.33238487836\\
1067	-17.0576473317028\\
1068	-25.9541364175823\\
1069	-24.5358367816966\\
1070	-27.3737016321807\\
1071	-26.2333665344372\\
1072	-27.7958088363275\\
1073	-24.8328841116449\\
1074	-23.8857780136377\\
1075	-23.2614519285253\\
1076	-21.2851256927927\\
1077	-22.6960159694684\\
1078	-32.2043480247921\\
1079	-50.3383555617272\\
1080	-49.3624521929084\\
1081	-44.889034467006\\
1082	-29.7196618993203\\
1083	-42.5336976530002\\
1084	-53.8670901159094\\
1085	-51.6876581202775\\
1086	-40.5605701134862\\
1087	-49.7475858586806\\
1088	-68.7387310771142\\
1089	-50.4689336007277\\
1090	-40.0794556237684\\
1091	-31.1683524644966\\
1092	-24.4950230189388\\
1093	-22.2096133522232\\
1094	-25.4888843559763\\
1095	-19.7409666074038\\
1096	-14.5097962051182\\
1097	-14.4368889023362\\
1098	-16.4353212366791\\
1099	-21.9976372044434\\
1100	-20.9525135534809\\
1101	-21.3301905758835\\
1102	-26.2176586892451\\
1103	-27.184180008888\\
1104	-37.1286793950142\\
1105	-32.0253434507254\\
1106	-34.7049661039352\\
1107	-28.3704575179008\\
1108	-24.9684464925988\\
1109	-29.2550767693035\\
1110	-22.2559395929413\\
1111	-21.9022839618196\\
1112	-24.0171281885257\\
1113	-24.5737064308225\\
1114	-20.3695124314868\\
1115	-20.6314943460297\\
1116	-17.1846427288781\\
1117	-17.9891854337804\\
1118	-19.1881571327406\\
1119	-15.6694086259995\\
1120	-17.2962605277269\\
1121	-19.2664919456543\\
1122	-28.4793778303992\\
1123	-29.3915105412048\\
1124	-30.6313202759727\\
1125	-18.8713634102514\\
1126	-25.0209124699606\\
1127	-40.4701413483749\\
1128	-29.4964072589949\\
1129	-32.1464631605886\\
1130	-32.9791423245404\\
1131	-23.6716479915829\\
1132	-18.1719978178116\\
1133	-22.238292253697\\
1134	-24.5450995525509\\
1135	-34.8677390214625\\
1136	-45.4801652994604\\
1137	-42.7979559886196\\
1138	-31.741252626261\\
1139	-33.7227216387892\\
1140	-31.2293536120198\\
1141	-31.0647323704991\\
1142	-30.6745154393839\\
1143	-21.945070545727\\
1144	-20.3283656939516\\
1145	-20.3097129766361\\
1146	-18.4592419694216\\
1147	-20.2293440499677\\
1148	-25.6215532199133\\
1149	-40.3654572903185\\
1150	-30.4187136668622\\
1151	-22.748597676595\\
1152	-20.992847565054\\
1153	-21.3754083538602\\
1154	-15.9921938182774\\
1155	-14.5778770715133\\
1156	-14.3025075208623\\
1157	-19.0995865925152\\
1158	-16.6416526297785\\
1159	-17.4700677680962\\
1160	-18.0152420919362\\
1161	-17.2521773307292\\
1162	-15.4539935224786\\
1163	-14.000563188097\\
1164	-13.3572365579817\\
1165	-13.1555509193668\\
1166	-24.0497210660019\\
1167	-37.101741273159\\
1168	-38.8961737714903\\
1169	-25.2540310654456\\
1170	-21.8274346235285\\
1171	-18.7221738540097\\
1172	-16.7082833829743\\
1173	-19.8282816896854\\
1174	-20.13663798125\\
1176	-32.3362164667558\\
1177	-50.8419077508274\\
1178	-65.0796700336546\\
1179	-49.8489727209314\\
1180	-47.2793950907731\\
1181	-34.4552671246856\\
1182	-36.9014608353568\\
1183	-38.4426372556154\\
1184	-38.4911225739193\\
1185	-32.0916338109744\\
1186	-28.4568240290707\\
1187	-30.2620876587412\\
1188	-25.9390934935755\\
1189	-29.6640581319461\\
1190	-23.4003511679327\\
1191	-37.8359022562179\\
1192	-43.5753169861487\\
1193	-31.1809272718949\\
1194	-20.4921275376591\\
1195	-22.9529208472766\\
1196	-35.0990188250046\\
1197	-42.1498938003372\\
1198	-50.3693467812607\\
1199	-50.8060894399539\\
1200	-51.1302123189851\\
1201	-41.0806351027022\\
1202	-38.2385062704961\\
1203	-42.5249406281798\\
1204	-47.207720405512\\
1205	-31.0316551377766\\
1206	-19.9222865868876\\
1207	-29.2975687594094\\
1208	-24.610266447329\\
1209	-17.2480115161375\\
1210	-22.5399855645614\\
1211	-24.652069115517\\
1212	-20.9840903333713\\
1213	-17.5684398622848\\
1214	-20.8998615158814\\
1215	-19.0269138814001\\
1216	-22.4928888124016\\
1217	-31.9978512276471\\
1218	-26.6769486605444\\
1219	-21.705732786545\\
1220	-22.0116630306204\\
1221	-31.9221795118369\\
1222	-49.0743231565218\\
1223	-36.4841187930133\\
1224	-25.9008128967669\\
1225	-21.214369870193\\
1226	-23.037008038136\\
1227	-22.2911691088616\\
1228	-17.8773087049485\\
1229	-19.0871744337294\\
1230	-21.9062406179155\\
1231	-25.2509628659252\\
1232	-26.3792799507942\\
1233	-19.7668409309545\\
1234	-29.3581039458636\\
1235	-34.7632698378318\\
1236	-31.4874792993176\\
1237	-26.019141623822\\
1238	-28.7396049870565\\
1239	-35.8839595599209\\
1240	-23.8647389967091\\
1241	-15.6972680199299\\
1242	-17.7515581402686\\
1243	-19.2692013412291\\
1244	-23.4975463838293\\
1245	-23.2369564419007\\
1246	-18.1649025868903\\
1247	-23.8732810115052\\
1248	-23.509419911705\\
1249	-19.3406212415377\\
1250	-21.8593138161696\\
1251	-19.6513686805815\\
1252	-19.5082024858668\\
1253	-21.5513864350828\\
1254	-14.6452712884125\\
1255	-16.3244309188108\\
1256	-14.1251776538709\\
1257	-14.9195645848433\\
1258	-21.6431705406324\\
1259	-24.5571748545212\\
1260	-32.1937703372882\\
1261	-40.5886995322508\\
1262	-27.9867210492503\\
1263	-20.4217195805772\\
1264	-18.4641987097659\\
1265	-17.8793145927511\\
1266	-20.4815999550981\\
1267	-15.8019761165206\\
1268	-16.9022561918125\\
1269	-18.876814006042\\
1270	-28.1218348171674\\
1271	-27.7192165586703\\
1272	-34.39693985447\\
1273	-24.2766913272142\\
1274	-24.1402854140399\\
1275	-15.9557679012626\\
1276	-17.0856196506825\\
1277	-13.9271795589616\\
1278	-14.1919534958608\\
1279	-13.5368168384528\\
1280	-19.5994157650009\\
1282	-23.1436248114105\\
1283	-22.9566595799399\\
1284	-36.1639020632738\\
1285	-34.4252141486759\\
1286	-25.0632264548929\\
1287	-29.7163047942931\\
1288	-30.1154416171246\\
1289	-35.9438949870359\\
1290	-28.9026784613029\\
1291	-22.6433149770178\\
1292	-15.5170293584417\\
1293	-16.1667368125873\\
1294	-14.4183694460171\\
1295	-14.1324871551169\\
1296	-14.2514643797645\\
1297	-14.2214887943271\\
1298	-16.3137126878289\\
1299	-20.3519127877034\\
1300	-19.9454448001143\\
1301	-20.8604312140785\\
1302	-21.9582084045489\\
1303	-16.2371397256495\\
1304	-15.2728538901783\\
1305	-16.7580448872818\\
1306	-24.9356035301114\\
1307	-23.4188554396362\\
1308	-25.2226557053316\\
1309	-23.4588873863986\\
1310	-19.2297870982582\\
1311	-22.8567914314142\\
1312	-29.9281150785848\\
1313	-30.5230852668344\\
1314	-22.6124316182179\\
1315	-34.6294295082421\\
1316	-28.6137243792703\\
1317	-26.300786941692\\
1318	-30.2320301088357\\
1319	-28.7198436775661\\
1320	-25.0557134338144\\
1321	-22.0191628245871\\
1322	-30.5971320358594\\
1323	-48.0168238162398\\
1324	-35.8399215368208\\
1325	-22.0960935603375\\
1326	-22.9385970124936\\
1327	-23.4941218230697\\
1328	-28.2721233187422\\
1329	-31.1681706428728\\
1330	-23.3106643873498\\
1331	-25.8349415907996\\
1332	-31.3836192039876\\
1333	-33.6392268108414\\
1334	-23.3136218604509\\
1335	-21.2727735017029\\
1336	-25.1713397070116\\
1337	-38.4800962369513\\
1338	-37.2987484515515\\
1339	-32.8485880127932\\
1340	-23.1717055547551\\
1341	-21.8644910069379\\
1342	-22.502016556921\\
1343	-16.8616934507411\\
1344	-14.3029310991419\\
1345	-13.820054350087\\
1346	-13.0708297399165\\
1347	-14.8320383927412\\
1348	-22.8408907091712\\
1349	-25.9214633648235\\
1350	-21.1896842643578\\
1351	-21.0125696143134\\
1352	-23.3976954061022\\
1353	-16.9398923825916\\
1354	-18.1644290994541\\
1355	-17.2961098712665\\
1356	-15.6266543768636\\
1357	-16.505064971366\\
1358	-23.170297736134\\
1359	-29.221498359035\\
1360	-19.3983217112755\\
1361	-18.1822610347544\\
1362	-16.0600657548819\\
1363	-17.6258398890502\\
1364	-14.2987446167115\\
1365	-21.7430450529616\\
1366	-39.8091033397291\\
1367	-28.522151699699\\
1368	-35.4928347112223\\
1369	-43.1779388707016\\
1370	-41.2835116075912\\
1371	-34.1041605561381\\
1372	-25.2448641845808\\
1373	-26.955873654646\\
1374	-26.8568939221555\\
1375	-29.7233650104017\\
1376	-32.3451257202782\\
1377	-31.7264738828649\\
1378	-40.8747400488635\\
1379	-48.0309273348851\\
1380	-54.005057773004\\
1381	-38.9255477934807\\
1382	-39.8238538159344\\
1383	-47.5196295747462\\
1384	-35.3242677296062\\
1385	-41.1025881825037\\
1386	-36.1311034015512\\
1387	-22.269089036608\\
1388	-16.2999156498697\\
1389	-17.2971307989808\\
1390	-23.3330914519188\\
1391	-18.9341538773485\\
1392	-15.6471328328223\\
1393	-18.4072958175664\\
1395	-14.0073299141977\\
1396	-15.3945462190138\\
1397	-15.9547705729619\\
1398	-14.4888379322892\\
1399	-18.6620163823063\\
1400	-24.5504519885394\\
1401	-20.4781562130129\\
1402	-29.3941908787313\\
1403	-46.2387189789147\\
1404	-41.3223155295784\\
1405	-48.8955862292148\\
1406	-35.1666157993338\\
1407	-37.7701725945217\\
1408	-26.1564785364692\\
1409	-20.6086124542182\\
1410	-22.2363608215392\\
1411	-18.8046525430454\\
1412	-16.3757313925969\\
1413	-19.5999116409919\\
1414	-13.334646165004\\
1415	-13.96080735757\\
1416	-12.9332947004616\\
1417	-18.5669347387577\\
1418	-26.9429914710595\\
1419	-29.8773613869998\\
1420	-29.4560345688928\\
1421	-26.9960945084872\\
1422	-29.8939299599651\\
1423	-27.2174118880075\\
1424	-22.7172352957521\\
1425	-23.9469462726161\\
1426	-20.5700243068065\\
1427	-27.3027154791064\\
1428	-20.1794985359984\\
1429	-18.6447477254096\\
1430	-23.7803045161638\\
1431	-31.5092888683457\\
1432	-22.9415955108113\\
1433	-20.8752806576338\\
1434	-19.1348506543254\\
1435	-21.0028766921471\\
1436	-16.7954842302499\\
1437	-15.9734979547186\\
1438	-19.2096942346798\\
1440	-23.7514176572829\\
1441	-20.882396358207\\
1442	-23.0171269450211\\
1443	-23.6283871313333\\
1444	-16.4177416126981\\
1445	-17.6535384607387\\
1446	-25.9491245969205\\
1447	-37.5993510002061\\
1448	-33.5532670150785\\
1449	-30.2106692066545\\
1450	-27.4881535891714\\
1451	-25.091174460295\\
1452	-22.8215921077181\\
1453	-21.4713611909233\\
1454	-25.1566489624445\\
1455	-22.0941798475694\\
1456	-17.5036531283447\\
1457	-21.5481445330254\\
1458	-24.2980196034539\\
1459	-26.7613477294226\\
1460	-28.9102742457385\\
1461	-27.2892923625611\\
1462	-36.9509373827509\\
1463	-45.982903097093\\
1464	-53.9398895944455\\
1465	-33.7601108890472\\
1466	-23.006520445768\\
1467	-18.1608729456766\\
1468	-16.2139606005696\\
1469	-17.2251561863482\\
1470	-16.2314796545022\\
1471	-24.2479052403801\\
1472	-22.7299707712793\\
1473	-20.6775681178051\\
1476	-30.0596196507959\\
1477	-31.950233520862\\
1478	-44.946059728728\\
1479	-43.7310021217763\\
1480	-32.2181931505374\\
1481	-24.1552226056642\\
1482	-25.271008044813\\
1483	-25.8274310359311\\
1484	-30.5103299694026\\
1485	-22.3507178312104\\
1486	-23.6187678179845\\
1487	-22.7612424592453\\
1488	-15.9569883350066\\
1489	-20.8303156029315\\
1490	-29.0859476804594\\
1491	-21.9783911039253\\
1492	-23.1193005049395\\
1493	-39.4194812533431\\
1494	-31.0072684403228\\
1495	-28.9595453453589\\
1496	-40.9462046468589\\
1497	-37.3363928623007\\
1498	-24.8015039671636\\
1499	-17.9099340009279\\
1500	-20.2324816435812\\
1501	-27.571303837723\\
1502	-45.0636253193188\\
1503	-44.5984264457672\\
1504	-43.6279290233615\\
1505	-34.8826820149195\\
1506	-24.9143515720305\\
1507	-23.1709859726352\\
1508	-18.5650185577267\\
1509	-17.7248879149936\\
1510	-17.2220833192573\\
1511	-19.5689490751433\\
1512	-21.8420294498474\\
1513	-15.5184839825165\\
1514	-18.8820166208584\\
1515	-25.3288481614657\\
1516	-27.4651159747441\\
1517	-32.7028195570629\\
1518	-41.5624729909714\\
1519	-52.374488608443\\
1520	-40.1712629952965\\
1521	-34.1290381353765\\
1522	-36.0330225656874\\
1523	-31.1133906886907\\
1524	-23.7959550481726\\
1525	-25.6297538789988\\
1526	-39.2282414282151\\
1527	-44.7377837642628\\
1528	-25.8624168080198\\
1529	-18.2011998184423\\
1530	-15.6050367965095\\
1531	-15.0289608839819\\
1532	-14.2597860654632\\
1533	-14.3350183442681\\
1534	-16.6052566721378\\
1535	-21.4472665062119\\
1536	-19.1858862097379\\
1537	-15.8626631968536\\
1538	-16.6052487733525\\
1539	-14.9284888951781\\
1540	-15.2928156912631\\
1541	-15.6184225908282\\
1542	-19.4796918750346\\
1543	-29.9458318723832\\
1544	-31.9482481596772\\
1545	-28.9236418578403\\
1546	-33.3409223179985\\
1547	-40.3367020822823\\
1548	-42.4346089253515\\
1549	-45.9217764036216\\
1550	-45.0650447775811\\
1551	-32.7629286676995\\
1552	-26.0250912962615\\
1553	-24.6273855525608\\
1554	-36.4205887285502\\
1555	-40.7440566825526\\
1556	-29.4957856908391\\
1558	-14.5592265791922\\
1559	-15.883867651685\\
1560	-14.1149785698565\\
1561	-12.4397027035832\\
1562	-13.4811140255117\\
1563	-17.7996876702202\\
1564	-17.2736517632295\\
1565	-16.5390486479503\\
1566	-13.8230378317073\\
1567	-13.814245020993\\
1568	-13.2009611074891\\
1569	-13.2437538032868\\
1570	-14.4772808584503\\
1571	-12.8061964241247\\
1572	-12.8376309922289\\
1573	-14.21043248776\\
1574	-32.6434729334599\\
1575	-42.5892718000905\\
1576	-41.8590550719496\\
1577	-31.8386147397878\\
1578	-39.1433119428229\\
1579	-59.0613983125111\\
1580	-51.0658890128122\\
1581	-50.2330910554065\\
1582	-37.7344470552448\\
1583	-39.2400197501031\\
1584	-41.3878747957672\\
1585	-29.7515713070648\\
1586	-27.0872367299958\\
1587	-17.3349142829195\\
1588	-17.5222741417349\\
1589	-27.4760565744991\\
1590	-19.8366386000239\\
1591	-21.3162918074927\\
1592	-24.3243191444687\\
1593	-21.8729118541255\\
1594	-22.7469372131543\\
1595	-24.1792261508733\\
1596	-25.5087856482739\\
1597	-26.669725981988\\
1598	-24.950119140876\\
1599	-19.9843619174053\\
1600	-24.2394484035271\\
1601	-14.4130596366927\\
1602	-15.5789459732323\\
1603	-13.7594983890842\\
1604	-15.207873046885\\
1605	-11.5977282480849\\
1606	-15.7461227671024\\
1607	-11.8384629788818\\
1608	-12.5899973326791\\
1609	-17.4899772329202\\
1610	-20.5842663715946\\
1611	-18.3239957164217\\
1612	-25.5225154860725\\
1613	-24.4276732018639\\
1614	-19.1357009059809\\
1615	-22.9038505448862\\
1616	-15.8624179060114\\
1617	-16.0521391429131\\
1618	-14.1296124955888\\
1619	-14.1925101470056\\
1620	-12.223669783428\\
1621	-11.2734117674408\\
1622	-13.5416134756751\\
1623	-20.6776012130474\\
1624	-20.1468898892324\\
1625	-17.1783223194177\\
1626	-17.4768946829174\\
1627	-15.7691909093785\\
1628	-17.9076832916976\\
1629	-15.0121104116215\\
1630	-14.9820996135411\\
1631	-14.077378869378\\
1632	-12.9870124756962\\
1633	-13.5566431942852\\
1634	-12.6044109893051\\
1635	-15.8576286631214\\
1636	-15.9237270051465\\
1637	-14.5935202769888\\
1638	-14.6188673631862\\
1639	-13.078320463266\\
1640	-13.5157839129211\\
1641	-21.011155030199\\
1642	-32.4082221395943\\
1643	-22.9062060723916\\
1644	-22.4099842725866\\
1645	-17.5748827459861\\
1646	-23.9232318091304\\
1647	-24.2617616939801\\
1648	-17.4212661497354\\
1649	-20.26215578046\\
1650	-15.8374320041073\\
1651	-19.2321392401195\\
1652	-14.3937751299939\\
1653	-15.2652273207732\\
1654	-16.559885975325\\
1655	-18.7990106146769\\
1656	-15.9888487047333\\
1657	-16.5783151256671\\
1658	-17.0795533340711\\
1659	-23.2494128992751\\
1660	-21.3056799029382\\
1661	-29.7201072080779\\
1662	-40.5066525649816\\
1663	-32.8407144840155\\
1664	-24.685492446692\\
1665	-20.1712192368125\\
1666	-25.8635650368396\\
1667	-32.6130922448115\\
1668	-24.9467295021498\\
1669	-27.5026637872027\\
1670	-19.794861598853\\
1671	-15.5745599466507\\
1672	-15.4897257654366\\
1673	-22.1376694819351\\
1674	-23.2341507246633\\
1675	-20.0163877384541\\
1676	-27.0031070559764\\
1677	-25.6021093367101\\
1678	-23.9777928175279\\
1679	-19.1849099073124\\
1680	-21.5552079530737\\
1681	-27.1487932552407\\
1682	-22.9243263075041\\
1683	-23.0741330682345\\
1684	-21.3462278462916\\
1685	-18.4602762628826\\
1686	-19.5851812900214\\
1687	-16.4744816930711\\
1688	-17.4329754654127\\
1689	-24.4016165373223\\
1690	-27.9730008436657\\
1691	-27.534921590697\\
1692	-25.6103928947311\\
1693	-20.5417032158625\\
1694	-17.2747863913423\\
1695	-18.0462988784352\\
1696	-20.6733560982598\\
1698	-27.7872522557875\\
1699	-32.5360292298794\\
1700	-27.3108320364036\\
1701	-18.7580374464592\\
1702	-27.1832992689599\\
1703	-38.8057171996652\\
1704	-28.2480480327933\\
1705	-27.1898042747409\\
1706	-31.4892554515448\\
1707	-26.0043846788831\\
1708	-21.8668654220307\\
1709	-24.9294175751809\\
1710	-23.3736938337991\\
1711	-24.0493572331634\\
1712	-29.1684633685632\\
1713	-23.1563394905816\\
1714	-25.9047443248026\\
1715	-24.9397487501035\\
1716	-25.4507763996535\\
1717	-21.0309208406964\\
1718	-24.4012283295503\\
1719	-19.4612714579455\\
1720	-20.3372409449153\\
1721	-21.6989973514212\\
1722	-22.2454552450793\\
1723	-14.5305480206864\\
1724	-14.7530734317158\\
1725	-14.2223430056968\\
1726	-12.5428028055655\\
1727	-22.9977856971118\\
1728	-35.8287726558151\\
1729	-30.8908038909558\\
1730	-23.6581827323214\\
1731	-24.8330856799214\\
1732	-24.2757777026684\\
1733	-21.3935118087538\\
1734	-17.0416204200669\\
1735	-18.9213696736292\\
1736	-18.3645288208852\\
1737	-18.6617425445868\\
1738	-20.0330956735156\\
1739	-18.0810627919388\\
1740	-17.7340761599901\\
1741	-14.438726310744\\
1742	-15.7675435511401\\
1743	-24.5888850837623\\
1744	-19.1844309423968\\
1745	-22.0720292504006\\
1746	-20.5169919359512\\
1747	-24.6031131425648\\
1748	-25.1046835633297\\
1749	-30.728276266959\\
1750	-24.7247504919135\\
1751	-25.3714268702349\\
1752	-27.4950740218424\\
1753	-17.0464274774436\\
1754	-24.2947707170802\\
1755	-18.5901865082881\\
1756	-21.9699844031684\\
1757	-22.931215132636\\
1758	-26.8699049225104\\
1759	-23.1300171248597\\
1760	-26.6678314620328\\
1761	-33.4478326739779\\
1762	-26.8765287441297\\
1763	-25.5582517458006\\
1764	-21.8311508850904\\
1765	-23.7522907878856\\
1766	-21.6841814738468\\
1767	-20.4567677894822\\
1768	-18.4929988780734\\
1769	-19.5900260122116\\
1770	-18.8387134406992\\
1771	-29.8830302320591\\
1772	-39.358488542345\\
1773	-33.4673411541444\\
1774	-32.0872629539542\\
1775	-30.8468142694817\\
1776	-20.7335458342861\\
1777	-20.0635277499446\\
1778	-17.8205029769611\\
1779	-16.3458669832687\\
1780	-16.9937478454017\\
1781	-22.6160166114512\\
1782	-27.4594842385277\\
1783	-29.7497323479579\\
1784	-25.6113760418611\\
1785	-19.7779296452775\\
1786	-28.7588096840682\\
1787	-35.3046133711282\\
1788	-37.5085639718077\\
1789	-30.8044551292303\\
1790	-47.2236858487663\\
1791	-38.53304021489\\
1792	-22.7024202604366\\
1793	-19.3703874675052\\
1794	-17.5973104373832\\
1796	-29.5019188757378\\
1797	-37.9783819779564\\
1798	-30.3841657333501\\
1799	-24.9021125051754\\
1800	-17.8266750804851\\
1801	-17.5819848314293\\
1802	-18.7962660106839\\
1803	-18.9544547859141\\
1804	-15.0646951378587\\
1805	-16.2646809755402\\
};
\addlegendentry{MPO prediction}

\end{axis}

\begin{axis}[%
width=6.159cm,
height=1.831cm,
at={(8.104cm,10.169cm)},
scale only axis,
xmin=1000,
xmax=2000,
xlabel style={font=\color{white!15!black}},
xlabel={Sample index},
ymin=-60.4542102409581,
ymax=0,
ylabel style={font=\color{white!15!black}},
ylabel={$y(t)$},
axis background/.style={fill=white},
title style={font=\bfseries},
title={C2: RMSE(OSA) = 2.6945, RMSE(MPO) = 5.5293},
legend style={legend cell align=left, align=left, draw=white!15!black}
]
\addplot [color=mycolor1, line width=2.0pt]
  table[row sep=crcr]{%
1006	-24.414\\
1007	-28.076\\
1008	-23.193\\
1009	-13.4280000000001\\
1010	-17.0899999999999\\
1011	-17.0899999999999\\
1012	-18.3109999999999\\
1013	-15.8689999999999\\
1014	-8.54500000000007\\
1015	-6.10400000000004\\
1016	-7.32400000000007\\
1017	-9.76600000000008\\
1018	-21.973\\
1019	-23.193\\
1020	-23.193\\
1021	-19.5309999999999\\
1022	-10.9860000000001\\
1023	-17.0899999999999\\
1024	-19.5309999999999\\
1025	-13.4280000000001\\
1026	-18.3109999999999\\
1027	-19.5309999999999\\
1028	-14.6479999999999\\
1029	-24.414\\
1030	-21.973\\
1031	-15.8689999999999\\
1032	-23.193\\
1033	-25.635\\
1034	-19.5309999999999\\
1036	-14.6479999999999\\
1037	-17.0899999999999\\
1038	-20.752\\
1039	-18.3109999999999\\
1040	-18.3109999999999\\
1042	-20.752\\
1043	-28.076\\
1044	-25.635\\
1045	-17.0899999999999\\
1046	-15.8689999999999\\
1047	-12.2070000000001\\
1048	-10.9860000000001\\
1049	-15.8689999999999\\
1050	-12.2070000000001\\
1051	-12.2070000000001\\
1052	-15.8689999999999\\
1053	-18.3109999999999\\
1054	-15.8689999999999\\
1055	-25.635\\
1056	-21.973\\
1057	-12.2070000000001\\
1058	-9.76600000000008\\
1059	-13.4280000000001\\
1060	-15.8689999999999\\
1061	-10.9860000000001\\
1062	-10.9860000000001\\
1063	-12.2070000000001\\
1064	-9.76600000000008\\
1065	-8.54500000000007\\
1066	-12.2070000000001\\
1068	-17.0899999999999\\
1069	-18.3109999999999\\
1070	-23.193\\
1071	-21.973\\
1072	-21.973\\
1073	-17.0899999999999\\
1075	-17.0899999999999\\
1076	-15.8689999999999\\
1077	-17.0899999999999\\
1078	-24.414\\
1079	-35.4000000000001\\
1080	-36.6210000000001\\
1081	-34.1800000000001\\
1082	-24.414\\
1083	-29.297\\
1084	-36.6210000000001\\
1085	-40.2829999999999\\
1086	-31.7380000000001\\
1087	-37.8420000000001\\
1088	-48.828\\
1089	-39.0630000000001\\
1090	-30.518\\
1091	-25.635\\
1092	-19.5309999999999\\
1093	-14.6479999999999\\
1094	-19.5309999999999\\
1095	-15.8689999999999\\
1096	-9.76600000000008\\
1097	-9.76600000000008\\
1098	-12.2070000000001\\
1099	-17.0899999999999\\
1100	-15.8689999999999\\
1101	-15.8689999999999\\
1102	-19.5309999999999\\
1103	-19.5309999999999\\
1104	-28.076\\
1105	-23.193\\
1106	-25.635\\
1107	-23.193\\
1108	-18.3109999999999\\
1109	-20.752\\
1110	-17.0899999999999\\
1111	-14.6479999999999\\
1112	-18.3109999999999\\
1114	-15.8689999999999\\
1116	-10.9860000000001\\
1118	-13.4280000000001\\
1119	-10.9860000000001\\
1120	-9.76600000000008\\
1122	-19.5309999999999\\
1123	-23.193\\
1124	-23.193\\
1125	-17.0899999999999\\
1126	-18.3109999999999\\
1127	-30.518\\
1128	-23.193\\
1129	-24.414\\
1130	-26.855\\
1131	-17.0899999999999\\
1132	-10.9860000000001\\
1133	-17.0899999999999\\
1134	-18.3109999999999\\
1135	-26.855\\
1136	-34.1800000000001\\
1137	-31.7380000000001\\
1138	-25.635\\
1140	-23.193\\
1141	-23.193\\
1142	-21.973\\
1144	-14.6479999999999\\
1146	-12.2070000000001\\
1147	-13.4280000000001\\
1149	-25.635\\
1150	-21.973\\
1151	-14.6479999999999\\
1152	-13.4280000000001\\
1153	-14.6479999999999\\
1154	-12.2070000000001\\
1155	-8.54500000000007\\
1156	-8.54500000000007\\
1157	-13.4280000000001\\
1158	-12.2070000000001\\
1161	-12.2070000000001\\
1162	-8.54500000000007\\
1163	-7.32400000000007\\
1164	-4.88300000000004\\
1165	-7.32400000000007\\
1166	-17.0899999999999\\
1167	-24.414\\
1168	-24.414\\
1169	-19.5309999999999\\
1170	-12.2070000000001\\
1171	-12.2070000000001\\
1172	-8.54500000000007\\
1173	-10.9860000000001\\
1174	-14.6479999999999\\
1175	-14.6479999999999\\
1177	-34.1800000000001\\
1178	-40.2829999999999\\
1179	-32.9590000000001\\
1180	-31.7380000000001\\
1181	-23.193\\
1182	-25.635\\
1184	-28.076\\
1185	-21.973\\
1187	-19.5309999999999\\
1188	-19.5309999999999\\
1189	-21.973\\
1190	-19.5309999999999\\
1191	-23.193\\
1192	-29.297\\
1193	-24.414\\
1194	-14.6479999999999\\
1195	-15.8689999999999\\
1196	-25.635\\
1197	-32.9590000000001\\
1198	-35.4000000000001\\
1199	-39.0630000000001\\
1200	-37.8420000000001\\
1201	-30.518\\
1202	-26.855\\
1203	-30.518\\
1204	-35.4000000000001\\
1205	-25.635\\
1206	-14.6479999999999\\
1207	-19.5309999999999\\
1208	-20.752\\
1209	-12.2070000000001\\
1210	-13.4280000000001\\
1211	-18.3109999999999\\
1212	-14.6479999999999\\
1213	-12.2070000000001\\
1214	-18.3109999999999\\
1215	-10.9860000000001\\
1216	-14.6479999999999\\
1217	-24.414\\
1218	-21.973\\
1219	-14.6479999999999\\
1220	-14.6479999999999\\
1221	-20.752\\
1222	-34.1800000000001\\
1223	-28.076\\
1224	-15.8689999999999\\
1225	-14.6479999999999\\
1226	-14.6479999999999\\
1227	-12.2070000000001\\
1228	-10.9860000000001\\
1229	-10.9860000000001\\
1230	-13.4280000000001\\
1231	-17.0899999999999\\
1232	-19.5309999999999\\
1233	-14.6479999999999\\
1234	-19.5309999999999\\
1235	-26.855\\
1236	-23.193\\
1237	-17.0899999999999\\
1238	-21.973\\
1239	-25.635\\
1240	-21.973\\
1241	-8.54500000000007\\
1242	-12.2070000000001\\
1243	-13.4280000000001\\
1244	-15.8689999999999\\
1245	-15.8689999999999\\
1246	-10.9860000000001\\
1247	-17.0899999999999\\
1248	-15.8689999999999\\
1249	-10.9860000000001\\
1250	-14.6479999999999\\
1251	-14.6479999999999\\
1252	-12.2070000000001\\
1253	-14.6479999999999\\
1254	-10.9860000000001\\
1255	-9.76600000000008\\
1256	-12.2070000000001\\
1257	-8.54500000000007\\
1258	-17.0899999999999\\
1259	-18.3109999999999\\
1260	-20.752\\
1261	-29.297\\
1262	-20.752\\
1263	-10.9860000000001\\
1264	-10.9860000000001\\
1265	-8.54500000000007\\
1266	-13.4280000000001\\
1267	-12.2070000000001\\
1268	-9.76600000000008\\
1269	-15.8689999999999\\
1270	-19.5309999999999\\
1271	-20.752\\
1272	-23.193\\
1273	-23.193\\
1274	-17.0899999999999\\
1275	-15.8689999999999\\
1276	-10.9860000000001\\
1277	-10.9860000000001\\
1278	-7.32400000000007\\
1279	-8.54500000000007\\
1280	-10.9860000000001\\
1281	-17.0899999999999\\
1282	-18.3109999999999\\
1283	-17.0899999999999\\
1284	-24.414\\
1285	-25.635\\
1286	-15.8689999999999\\
1289	-26.855\\
1290	-25.635\\
1291	-17.0899999999999\\
1292	-10.9860000000001\\
1293	-7.32400000000007\\
1294	-6.10400000000004\\
1295	-8.54500000000007\\
1296	-9.76600000000008\\
1297	-8.54500000000007\\
1298	-10.9860000000001\\
1299	-17.0899999999999\\
1300	-15.8689999999999\\
1301	-13.4280000000001\\
1302	-15.8689999999999\\
1303	-12.2070000000001\\
1304	-6.10400000000004\\
1305	-9.76600000000008\\
1306	-20.752\\
1307	-17.0899999999999\\
1308	-18.3109999999999\\
1309	-15.8689999999999\\
1310	-10.9860000000001\\
1312	-23.193\\
1313	-23.193\\
1314	-17.0899999999999\\
1315	-26.855\\
1316	-25.635\\
1317	-18.3109999999999\\
1318	-23.193\\
1319	-25.635\\
1320	-19.5309999999999\\
1321	-15.8689999999999\\
1322	-24.414\\
1323	-35.4000000000001\\
1324	-29.297\\
1325	-17.0899999999999\\
1326	-17.0899999999999\\
1327	-18.3109999999999\\
1329	-23.193\\
1330	-19.5309999999999\\
1331	-17.0899999999999\\
1332	-24.414\\
1333	-25.635\\
1334	-18.3109999999999\\
1335	-15.8689999999999\\
1336	-18.3109999999999\\
1337	-28.076\\
1338	-26.855\\
1339	-24.414\\
1340	-19.5309999999999\\
1341	-17.0899999999999\\
1342	-18.3109999999999\\
1343	-13.4280000000001\\
1344	-6.10400000000004\\
1346	-6.10400000000004\\
1347	-9.76600000000008\\
1348	-19.5309999999999\\
1349	-15.8689999999999\\
1351	-13.4280000000001\\
1352	-17.0899999999999\\
1354	-9.76600000000008\\
1355	-12.2070000000001\\
1356	-9.76600000000008\\
1357	-8.54500000000007\\
1358	-17.0899999999999\\
1359	-20.752\\
1360	-17.0899999999999\\
1361	-10.9860000000001\\
1362	-10.9860000000001\\
1363	-13.4280000000001\\
1364	-9.76600000000008\\
1365	-10.9860000000001\\
1366	-28.076\\
1367	-20.752\\
1368	-25.635\\
1369	-35.4000000000001\\
1370	-30.518\\
1372	-18.3109999999999\\
1373	-18.3109999999999\\
1374	-19.5309999999999\\
1375	-21.973\\
1376	-25.635\\
1377	-25.635\\
1378	-31.7380000000001\\
1379	-34.1800000000001\\
1380	-41.5039999999999\\
1381	-32.9590000000001\\
1382	-28.076\\
1383	-35.4000000000001\\
1384	-26.855\\
1385	-29.297\\
1386	-28.076\\
1387	-17.0899999999999\\
1388	-10.9860000000001\\
1389	-12.2070000000001\\
1390	-17.0899999999999\\
1391	-14.6479999999999\\
1392	-9.76600000000008\\
1393	-13.4280000000001\\
1394	-13.4280000000001\\
1395	-7.32400000000007\\
1396	-8.54500000000007\\
1397	-12.2070000000001\\
1398	-7.32400000000007\\
1399	-14.6479999999999\\
1400	-20.752\\
1401	-13.4280000000001\\
1403	-30.518\\
1404	-29.297\\
1405	-35.4000000000001\\
1406	-29.297\\
1407	-28.076\\
1408	-23.193\\
1409	-12.2070000000001\\
1410	-14.6479999999999\\
1411	-15.8689999999999\\
1412	-10.9860000000001\\
1414	-13.4280000000001\\
1415	-3.66200000000003\\
1416	-8.54500000000007\\
1417	-17.0899999999999\\
1418	-19.5309999999999\\
1419	-20.752\\
1420	-23.193\\
1421	-20.752\\
1422	-23.193\\
1423	-18.3109999999999\\
1424	-15.8689999999999\\
1425	-15.8689999999999\\
1426	-12.2070000000001\\
1427	-18.3109999999999\\
1428	-17.0899999999999\\
1429	-12.2070000000001\\
1430	-17.0899999999999\\
1431	-23.193\\
1432	-17.0899999999999\\
1433	-13.4280000000001\\
1434	-12.2070000000001\\
1435	-14.6479999999999\\
1436	-7.32400000000007\\
1437	-8.54500000000007\\
1438	-12.2070000000001\\
1439	-17.0899999999999\\
1440	-17.0899999999999\\
1441	-14.6479999999999\\
1442	-15.8689999999999\\
1443	-18.3109999999999\\
1444	-12.2070000000001\\
1445	-9.76600000000008\\
1447	-26.855\\
1448	-26.855\\
1449	-21.973\\
1452	-14.6479999999999\\
1453	-13.4280000000001\\
1454	-17.0899999999999\\
1455	-17.0899999999999\\
1456	-10.9860000000001\\
1458	-18.3109999999999\\
1459	-19.5309999999999\\
1460	-21.973\\
1461	-21.973\\
1462	-29.297\\
1463	-35.4000000000001\\
1464	-36.6210000000001\\
1465	-26.855\\
1466	-14.6479999999999\\
1468	-7.32400000000007\\
1469	-10.9860000000001\\
1470	-10.9860000000001\\
1471	-14.6479999999999\\
1473	-12.2070000000001\\
1474	-15.8689999999999\\
1475	-20.752\\
1476	-20.752\\
1477	-24.414\\
1478	-31.7380000000001\\
1480	-26.855\\
1481	-18.3109999999999\\
1482	-17.0899999999999\\
1483	-20.752\\
1484	-20.752\\
1485	-17.0899999999999\\
1486	-14.6479999999999\\
1487	-18.3109999999999\\
1488	-12.2070000000001\\
1489	-13.4280000000001\\
1490	-20.752\\
1491	-17.0899999999999\\
1492	-18.3109999999999\\
1493	-32.9590000000001\\
1494	-25.635\\
1495	-20.752\\
1496	-28.076\\
1497	-28.076\\
1498	-19.5309999999999\\
1499	-13.4280000000001\\
1500	-13.4280000000001\\
1501	-19.5309999999999\\
1502	-30.518\\
1503	-34.1800000000001\\
1504	-31.7380000000001\\
1505	-25.635\\
1506	-17.0899999999999\\
1507	-14.6479999999999\\
1508	-13.4280000000001\\
1510	-8.54500000000007\\
1511	-12.2070000000001\\
1512	-14.6479999999999\\
1513	-10.9860000000001\\
1514	-13.4280000000001\\
1515	-19.5309999999999\\
1517	-21.973\\
1518	-29.297\\
1519	-34.1800000000001\\
1520	-25.635\\
1521	-23.193\\
1522	-24.414\\
1523	-20.752\\
1524	-14.6479999999999\\
1525	-18.3109999999999\\
1526	-29.297\\
1527	-31.7380000000001\\
1528	-19.5309999999999\\
1529	-12.2070000000001\\
1530	-8.54500000000007\\
1531	-7.32400000000007\\
1532	-9.76600000000008\\
1533	-8.54500000000007\\
1534	-12.2070000000001\\
1535	-14.6479999999999\\
1536	-13.4280000000001\\
1537	-9.76600000000008\\
1538	-10.9860000000001\\
1539	-10.9860000000001\\
1540	-9.76600000000008\\
1541	-10.9860000000001\\
1542	-14.6479999999999\\
1543	-21.973\\
1544	-18.3109999999999\\
1545	-20.752\\
1546	-21.973\\
1547	-30.518\\
1548	-31.7380000000001\\
1549	-34.1800000000001\\
1550	-34.1800000000001\\
1551	-24.414\\
1552	-17.0899999999999\\
1553	-15.8689999999999\\
1554	-26.855\\
1555	-31.7380000000001\\
1556	-23.193\\
1557	-18.3109999999999\\
1558	-9.76600000000008\\
1559	-4.88300000000004\\
1560	-6.10400000000004\\
1561	-6.10400000000004\\
1562	-7.32400000000007\\
1563	-14.6479999999999\\
1564	-10.9860000000001\\
1565	-12.2070000000001\\
1567	-4.88300000000004\\
1568	-8.54500000000007\\
1570	-8.54500000000007\\
1571	-9.76600000000008\\
1572	-7.32400000000007\\
1573	-10.9860000000001\\
1574	-24.414\\
1575	-26.855\\
1576	-25.635\\
1577	-21.973\\
1578	-30.518\\
1579	-40.2829999999999\\
1580	-34.1800000000001\\
1581	-31.7380000000001\\
1582	-26.855\\
1584	-29.297\\
1585	-20.752\\
1586	-17.0899999999999\\
1588	-12.2070000000001\\
1589	-20.752\\
1591	-15.8689999999999\\
1592	-18.3109999999999\\
1593	-18.3109999999999\\
1594	-14.6479999999999\\
1595	-17.0899999999999\\
1596	-17.0899999999999\\
1597	-20.752\\
1598	-19.5309999999999\\
1599	-13.4280000000001\\
1600	-17.0899999999999\\
1601	-9.76600000000008\\
1602	-4.88300000000004\\
1603	-10.9860000000001\\
1604	-13.4280000000001\\
1605	-8.54500000000007\\
1606	-2.44100000000003\\
1607	-8.54500000000007\\
1608	-12.2070000000001\\
1609	-12.2070000000001\\
1610	-15.8689999999999\\
1611	-13.4280000000001\\
1612	-18.3109999999999\\
1613	-17.0899999999999\\
1614	-10.9860000000001\\
1615	-17.0899999999999\\
1616	-14.6479999999999\\
1617	-9.76600000000008\\
1618	-7.32400000000007\\
1619	-6.10400000000004\\
1621	-8.54500000000007\\
1622	-6.10400000000004\\
1623	-13.4280000000001\\
1624	-17.0899999999999\\
1625	-10.9860000000001\\
1626	-14.6479999999999\\
1627	-10.9860000000001\\
1628	-14.6479999999999\\
1629	-10.9860000000001\\
1630	-9.76600000000008\\
1631	-9.76600000000008\\
1632	-7.32400000000007\\
1633	-8.54500000000007\\
1634	-10.9860000000001\\
1635	-14.6479999999999\\
1636	-12.2070000000001\\
1637	-8.54500000000007\\
1639	-8.54500000000007\\
1640	-10.9860000000001\\
1641	-21.973\\
1642	-24.414\\
1643	-18.3109999999999\\
1644	-17.0899999999999\\
1645	-13.4280000000001\\
1646	-19.5309999999999\\
1647	-17.0899999999999\\
1648	-9.76600000000008\\
1649	-14.6479999999999\\
1650	-13.4280000000001\\
1651	-15.8689999999999\\
1652	-12.2070000000001\\
1653	-14.6479999999999\\
1654	-12.2070000000001\\
1655	-15.8689999999999\\
1656	-12.2070000000001\\
1658	-14.6479999999999\\
1659	-19.5309999999999\\
1660	-18.3109999999999\\
1661	-21.973\\
1662	-30.518\\
1663	-24.414\\
1664	-14.6479999999999\\
1665	-14.6479999999999\\
1666	-15.8689999999999\\
1667	-23.193\\
1668	-18.3109999999999\\
1669	-23.193\\
1670	-17.0899999999999\\
1671	-8.54500000000007\\
1672	-9.76600000000008\\
1673	-19.5309999999999\\
1674	-14.6479999999999\\
1675	-13.4280000000001\\
1676	-20.752\\
1677	-21.973\\
1679	-14.6479999999999\\
1680	-18.3109999999999\\
1681	-20.752\\
1683	-15.8689999999999\\
1684	-15.8689999999999\\
1685	-12.2070000000001\\
1686	-14.6479999999999\\
1687	-12.2070000000001\\
1688	-13.4280000000001\\
1689	-19.5309999999999\\
1690	-20.752\\
1691	-19.5309999999999\\
1692	-19.5309999999999\\
1693	-14.6479999999999\\
1694	-10.9860000000001\\
1695	-13.4280000000001\\
1696	-14.6479999999999\\
1697	-18.3109999999999\\
1698	-20.752\\
1699	-24.414\\
1701	-12.2070000000001\\
1702	-21.973\\
1703	-30.518\\
1704	-20.752\\
1705	-20.752\\
1706	-25.635\\
1707	-18.3109999999999\\
1708	-17.0899999999999\\
1709	-18.3109999999999\\
1710	-15.8689999999999\\
1711	-18.3109999999999\\
1712	-21.973\\
1713	-17.0899999999999\\
1714	-20.752\\
1715	-20.752\\
1716	-18.3109999999999\\
1717	-14.6479999999999\\
1718	-21.973\\
1719	-14.6479999999999\\
1720	-17.0899999999999\\
1722	-17.0899999999999\\
1723	-10.9860000000001\\
1724	-6.10400000000004\\
1725	-4.88300000000004\\
1726	-10.9860000000001\\
1727	-20.752\\
1728	-21.973\\
1729	-19.5309999999999\\
1730	-13.4280000000001\\
1731	-17.0899999999999\\
1732	-14.6479999999999\\
1733	-14.6479999999999\\
1734	-9.76600000000008\\
1735	-15.8689999999999\\
1736	-14.6479999999999\\
1737	-12.2070000000001\\
1738	-13.4280000000001\\
1739	-12.2070000000001\\
1741	-7.32400000000007\\
1742	-8.54500000000007\\
1743	-18.3109999999999\\
1744	-13.4280000000001\\
1745	-18.3109999999999\\
1746	-15.8689999999999\\
1747	-21.973\\
1748	-19.5309999999999\\
1749	-25.635\\
1750	-15.8689999999999\\
1751	-23.193\\
1752	-21.973\\
1753	-13.4280000000001\\
1754	-17.0899999999999\\
1755	-14.6479999999999\\
1756	-19.5309999999999\\
1757	-20.752\\
1758	-24.414\\
1759	-17.0899999999999\\
1760	-24.414\\
1761	-25.635\\
1762	-23.193\\
1763	-18.3109999999999\\
1764	-14.6479999999999\\
1765	-19.5309999999999\\
1768	-12.2070000000001\\
1769	-13.4280000000001\\
1770	-12.2070000000001\\
1771	-25.635\\
1772	-30.518\\
1773	-25.635\\
1774	-23.193\\
1775	-25.635\\
1776	-14.6479999999999\\
1777	-13.4280000000001\\
1778	-9.76600000000008\\
1779	-9.76600000000008\\
1780	-12.2070000000001\\
1781	-18.3109999999999\\
1782	-19.5309999999999\\
1783	-23.193\\
1784	-18.3109999999999\\
1785	-15.8689999999999\\
1786	-19.5309999999999\\
1787	-28.076\\
1788	-28.076\\
1789	-23.193\\
1790	-37.8420000000001\\
1791	-25.635\\
1792	-15.8689999999999\\
1793	-13.4280000000001\\
1794	-12.2070000000001\\
1795	-19.5309999999999\\
1796	-23.193\\
1797	-28.076\\
1798	-24.414\\
1799	-18.3109999999999\\
1800	-10.9860000000001\\
1801	-14.6479999999999\\
1802	-10.9860000000001\\
1803	-13.4280000000001\\
1804	-8.54500000000007\\
1805	-10.9860000000001\\
};
\addlegendentry{True output}

\addplot [color=mycolor2, dashed, line width=2.0pt]
  table[row sep=crcr]{%
1006	-25.7061036614996\\
1007	-28.0556469959067\\
1008	-24.2828081013954\\
1009	-15.3073418931401\\
1010	-17.2179207345996\\
1011	-18.3035331647829\\
1012	-19.0108879578029\\
1013	-18.0730805047065\\
1014	-11.8362512594515\\
1015	-14.8373294630364\\
1016	-9.98567388227889\\
1017	-10.371271255666\\
1018	-26.2413412597543\\
1019	-24.4256428208673\\
1020	-23.4086378695351\\
1021	-18.5793969238809\\
1022	-16.3028971698375\\
1023	-21.9669446496684\\
1024	-15.8913263518259\\
1025	-15.7440314001115\\
1026	-19.4091670775554\\
1027	-18.7357255089578\\
1028	-17.9544294370398\\
1029	-22.5434985642535\\
1030	-25.1285084192473\\
1031	-18.1343966402014\\
1032	-23.8619501674345\\
1033	-24.0002183624567\\
1034	-20.8570750384292\\
1035	-18.3545462012376\\
1036	-16.3880205555167\\
1037	-17.7254150149608\\
1038	-20.4433857409981\\
1039	-19.855220725731\\
1040	-20.2123751339288\\
1041	-19.5917669657704\\
1042	-21.468989187432\\
1043	-28.0800862021715\\
1044	-26.2614197290636\\
1045	-19.4001553012949\\
1046	-18.6176840779319\\
1047	-13.6734426666055\\
1048	-14.0179370617179\\
1049	-14.6848476115506\\
1050	-13.9327051666583\\
1051	-13.8018062543213\\
1052	-16.4517744027901\\
1053	-18.9080799466924\\
1054	-18.0684220763701\\
1055	-24.255000303709\\
1056	-21.6679047666514\\
1057	-16.1858634133484\\
1058	-14.2350382493321\\
1059	-14.2468817834567\\
1060	-15.7485283839685\\
1061	-12.3853350728564\\
1062	-12.457409648541\\
1063	-13.2612577032905\\
1064	-12.8470258840116\\
1065	-11.353721862258\\
1066	-11.5974224991664\\
1067	-13.4844501188973\\
1068	-19.6425692637083\\
1069	-19.388101841548\\
1070	-20.848263870158\\
1071	-21.67564107312\\
1072	-24.9865421424222\\
1073	-21.05298256573\\
1074	-18.4206542740603\\
1075	-17.7861807614072\\
1076	-17.6371989179822\\
1077	-16.8597592590784\\
1078	-25.9180952363392\\
1079	-39.6538971267182\\
1080	-38.4496344320239\\
1081	-36.9778784406972\\
1082	-22.9595713962342\\
1083	-33.3242143887755\\
1084	-41.0899999458436\\
1085	-39.1640586361887\\
1086	-31.143899303499\\
1087	-40.1026984143746\\
1088	-55.858466347114\\
1089	-37.1621721873532\\
1090	-33.1486621025872\\
1091	-21.4836441636653\\
1092	-20.5325514912661\\
1093	-18.4722182481289\\
1094	-19.759351165897\\
1095	-17.517527746445\\
1096	-12.407061244598\\
1097	-12.0088352311552\\
1098	-12.6432524041088\\
1099	-17.812562048628\\
1100	-16.9827098550063\\
1101	-17.1806522065808\\
1102	-20.8868459521375\\
1103	-22.6044883730915\\
1104	-27.7370468585102\\
1105	-25.3367357579107\\
1106	-27.5228609241369\\
1107	-23.1327937215538\\
1108	-20.581298051068\\
1109	-22.5945948607584\\
1110	-18.2668442152224\\
1111	-17.1080964337209\\
1112	-19.0257040106396\\
1113	-18.909032860661\\
1114	-15.4551795431066\\
1115	-16.3117903118202\\
1116	-13.385211033682\\
1117	-13.8078559526641\\
1118	-13.7834638449137\\
1119	-12.3790939283467\\
1120	-13.1533380654803\\
1121	-14.3148207422341\\
1122	-21.6467865754148\\
1123	-22.2173729935018\\
1124	-23.9880336651202\\
1125	-17.0690845156048\\
1126	-21.7924107427334\\
1127	-31.1974843416299\\
1128	-25.879162179865\\
1129	-25.996471208744\\
1130	-27.9698442238428\\
1131	-17.1094693465393\\
1132	-16.2688547723062\\
1133	-16.070487035038\\
1134	-19.7941571183842\\
1135	-30.3049834347344\\
1136	-35.803863574494\\
1137	-34.4718543570834\\
1138	-26.3690727146397\\
1139	-26.6976996870785\\
1140	-24.1523578716267\\
1141	-24.3790363518801\\
1142	-24.3476058568683\\
1143	-17.7471632587949\\
1144	-16.9551170654104\\
1145	-15.5958323455154\\
1146	-14.3685462519068\\
1147	-15.3067530385442\\
1148	-19.3116160903171\\
1149	-29.9091690700575\\
1150	-22.7707531129822\\
1151	-17.4564256203059\\
1152	-16.2491924532408\\
1153	-15.2723939383893\\
1154	-12.4617189795056\\
1155	-12.6483191197167\\
1156	-10.4974948061595\\
1157	-13.378270023381\\
1158	-12.7641333705214\\
1159	-13.6436247190748\\
1160	-13.4559640578241\\
1161	-13.6595189311431\\
1162	-12.0699111424703\\
1163	-10.9241852468401\\
1164	-10.2331983291208\\
1165	-7.5489826954481\\
1166	-16.4812499094251\\
1167	-28.4186097215027\\
1168	-27.7563255695725\\
1169	-19.3949253589601\\
1170	-16.4753997742514\\
1171	-13.8774935818851\\
1172	-13.021738342376\\
1173	-13.3630591896242\\
1174	-13.6325513578061\\
1175	-18.8162532564077\\
1176	-22.6745212635949\\
1177	-41.3115591724738\\
1178	-47.7289341263506\\
1179	-35.9179477913183\\
1180	-34.3078748139403\\
1181	-21.1366273445144\\
1182	-27.4287990598036\\
1183	-27.3291670237584\\
1184	-29.7527118415708\\
1185	-24.1374957937603\\
1186	-21.2904086401297\\
1187	-22.3932089105238\\
1188	-19.8270513000309\\
1189	-22.8390277827591\\
1190	-18.4398846251167\\
1191	-28.4641153383689\\
1192	-33.8823964194603\\
1193	-22.4540079440701\\
1194	-16.937000876625\\
1195	-17.0282290539196\\
1196	-27.7940689973627\\
1197	-33.7341246445676\\
1198	-39.3011202831512\\
1199	-40.4357154579002\\
1200	-40.4008831397637\\
1201	-30.5340328390498\\
1202	-29.3110260466483\\
1203	-32.6187453170503\\
1204	-37.7724263248465\\
1205	-23.6387704386316\\
1206	-16.6789121876297\\
1207	-22.3360028585619\\
1208	-19.8504231236561\\
1209	-15.1977117439515\\
1210	-16.2590637838189\\
1211	-18.8152512183556\\
1212	-16.2757732783637\\
1213	-14.1100475343674\\
1214	-15.3895865959605\\
1215	-16.7631900563424\\
1216	-16.8004403119623\\
1217	-23.9078759900324\\
1218	-22.6278784330136\\
1219	-17.4777919406226\\
1220	-16.8374766778118\\
1221	-22.899431571044\\
1222	-41.0109907717067\\
1223	-25.7932539598282\\
1224	-18.6838133257561\\
1225	-15.6331960359769\\
1226	-18.5185478986173\\
1227	-16.1763985110285\\
1228	-12.9356997618606\\
1229	-13.1650685061065\\
1230	-14.6975512162153\\
1232	-20.6169461053241\\
1233	-15.0353753869479\\
1234	-22.051586525698\\
1235	-26.7400167276767\\
1236	-23.9531759269428\\
1237	-21.5930958416764\\
1238	-20.3413645678254\\
1239	-29.1129131959888\\
1240	-18.8199625705308\\
1241	-14.5477127470463\\
1242	-13.4681954526209\\
1243	-15.022956640245\\
1244	-18.8700259100356\\
1245	-16.4283817912415\\
1246	-14.4189064948248\\
1247	-17.7079081261063\\
1248	-18.2697883494216\\
1249	-14.6418321761546\\
1250	-15.0940128510056\\
1251	-15.6035134402396\\
1252	-14.2284370036298\\
1253	-15.2778319230447\\
1254	-12.2862527273614\\
1255	-13.1355302087827\\
1256	-10.7513425380546\\
1257	-12.5035239628567\\
1258	-14.4989440138138\\
1260	-24.0920672440491\\
1261	-32.6624354674282\\
1262	-22.0294512519831\\
1263	-15.8962588706081\\
1264	-13.5943723787516\\
1265	-12.8847045399505\\
1266	-13.9986205687651\\
1267	-10.9130386098775\\
1268	-12.952798445614\\
1269	-13.1528548866188\\
1270	-24.9163530574476\\
1271	-21.3114202702548\\
1272	-26.7984065942765\\
1273	-19.0939118221995\\
1274	-20.0103497857401\\
1275	-14.0926498697743\\
1276	-15.4253429535393\\
1277	-12.6104864202205\\
1278	-11.8466468617994\\
1279	-10.311658832811\\
1280	-13.2683318166526\\
1281	-16.1352927024529\\
1282	-16.7335661403852\\
1283	-18.6324932800933\\
1284	-28.5668002466687\\
1285	-25.9098368300274\\
1286	-20.1879683346297\\
1287	-21.2140898260318\\
1288	-22.2067957605284\\
1289	-29.5552740595897\\
1290	-22.5000543875792\\
1291	-19.0169548664494\\
1292	-14.6725929610732\\
1293	-14.4158958761091\\
1294	-10.5807889787066\\
1295	-8.99537804824763\\
1296	-9.83755749434317\\
1297	-10.1406179542014\\
1298	-10.6731000290931\\
1299	-15.9345476541689\\
1300	-16.2927841593698\\
1301	-16.6936205006757\\
1302	-16.7961557315382\\
1303	-13.6195946044222\\
1304	-13.6155583509387\\
1305	-9.9808979083939\\
1306	-17.671189916759\\
1307	-19.047008966729\\
1308	-19.1703585528101\\
1309	-17.944330286419\\
1310	-15.5822282786103\\
1311	-16.8300317696342\\
1312	-22.825692450589\\
1313	-24.0406403255492\\
1314	-17.3314139626495\\
1315	-26.0321628360475\\
1316	-24.790496744462\\
1317	-22.4399732489569\\
1318	-24.259549546932\\
1319	-24.326645336542\\
1320	-21.1951739320268\\
1321	-18.4963379060207\\
1322	-26.9918689334986\\
1323	-37.8172086109043\\
1324	-28.4548528196017\\
1325	-18.6420430471685\\
1326	-19.0223793701282\\
1327	-19.2216486665602\\
1328	-22.4427523423421\\
1329	-24.7994708809924\\
1330	-19.1637552765887\\
1331	-20.714553186632\\
1332	-24.3476941318299\\
1333	-27.0847091570836\\
1334	-19.4957484056526\\
1335	-16.8332877637465\\
1336	-20.1653612412515\\
1337	-32.945606140308\\
1338	-26.7622738433581\\
1339	-26.0922744918846\\
1340	-18.4303829306718\\
1341	-17.8823473126115\\
1342	-17.9940080553097\\
1343	-14.825483685712\\
1344	-13.581724562305\\
1345	-10.9486124825305\\
1346	-9.2905728844687\\
1347	-9.05916240173406\\
1349	-21.4447244750156\\
1350	-14.8184216663046\\
1351	-16.5416472403476\\
1352	-17.5160663897689\\
1353	-13.3231311289926\\
1354	-14.3649789496897\\
1355	-12.3819992268145\\
1356	-12.0512201059289\\
1357	-12.4099023192514\\
1358	-16.4791288748345\\
1359	-22.1763178030965\\
1360	-15.2134329274554\\
1361	-15.0837063940191\\
1362	-13.1286765887501\\
1363	-13.0653906683146\\
1364	-12.4209800295891\\
1365	-15.6156690102839\\
1366	-30.2186572150724\\
1367	-22.2236744052257\\
1368	-26.7308073982786\\
1369	-32.5277657210852\\
1370	-34.4094621602756\\
1371	-25.8081034219827\\
1372	-20.4752288790189\\
1373	-20.6234682352949\\
1374	-20.5344615400581\\
1375	-22.8689273834439\\
1376	-25.2728350769773\\
1377	-25.4214566016617\\
1378	-33.2755446367369\\
1379	-39.3680657675752\\
1380	-43.0135509582319\\
1381	-30.6019745137723\\
1382	-32.4144516621614\\
1383	-36.1494228399736\\
1384	-29.9522370877903\\
1385	-31.6498495202261\\
1386	-28.0262603224382\\
1387	-16.8976167048704\\
1388	-14.4448289520371\\
1389	-13.656061078153\\
1390	-17.4797932249221\\
1391	-15.612507033856\\
1392	-12.9222208509254\\
1393	-14.2184637570008\\
1394	-12.6732642086099\\
1395	-12.6906484874062\\
1396	-11.0427669199155\\
1397	-11.6029836705698\\
1398	-11.6856924421725\\
1399	-13.0304254305038\\
1400	-20.3267552271252\\
1401	-16.1536910658303\\
1402	-24.4584276760493\\
1403	-39.4425377213672\\
1404	-29.6628657295728\\
1405	-36.023265943199\\
1406	-28.177033338462\\
1407	-29.0498957388484\\
1408	-22.6513778973797\\
1409	-16.8007440565709\\
1410	-17.3544143202703\\
1411	-14.8137644839755\\
1412	-14.3883881839508\\
1413	-14.4029190554377\\
1414	-10.6214514228845\\
1415	-13.4336212042579\\
1416	-8.4843078957324\\
1417	-13.6046009690672\\
1418	-21.0318133691026\\
1419	-23.0389406112042\\
1420	-22.863273647675\\
1421	-22.5585193112347\\
1422	-23.9253696128001\\
1423	-21.4849185002254\\
1424	-17.7155731208559\\
1425	-18.3458484702508\\
1426	-15.6851132184365\\
1427	-19.6817759435926\\
1428	-14.91910316249\\
1429	-15.5348700399002\\
1430	-16.9079488723312\\
1431	-24.4437848960463\\
1432	-19.2661459721676\\
1433	-16.0132380767106\\
1434	-14.8172822006873\\
1435	-15.5708357853937\\
1436	-13.0907562820655\\
1437	-11.0305469611703\\
1438	-13.0748665438366\\
1439	-15.5879846637076\\
1440	-17.6155197987491\\
1441	-15.6142533739062\\
1442	-18.487792110586\\
1443	-17.822111230583\\
1444	-13.9441310703278\\
1445	-14.0478133760751\\
1446	-19.6056956909695\\
1447	-29.1452663915782\\
1448	-25.5676868200683\\
1449	-22.9078558388314\\
1450	-22.0921153922577\\
1451	-19.487228061929\\
1452	-17.7443668345143\\
1453	-15.5508583226283\\
1454	-17.5501515351837\\
1455	-17.3224111087952\\
1456	-14.1766698868371\\
1458	-18.157968758044\\
1459	-20.9076064980563\\
1460	-22.3444253174375\\
1461	-22.4166557933438\\
1462	-29.3000958600364\\
1463	-38.9912715385542\\
1464	-43.026560250627\\
1465	-23.0857003969725\\
1466	-19.4031276698745\\
1467	-14.4198538423357\\
1468	-13.2416690950395\\
1469	-11.3158015444619\\
1470	-11.3388839052957\\
1471	-17.6815775947507\\
1472	-16.0931120480941\\
1473	-14.5694167281479\\
1474	-17.0691029886689\\
1475	-18.9273540996994\\
1476	-23.7569771640631\\
1477	-24.243322523665\\
1478	-35.944262544255\\
1479	-32.7541607023375\\
1480	-23.3837118186334\\
1481	-20.8011128299866\\
1482	-19.7008111146185\\
1483	-20.01525547495\\
1484	-23.9362499350027\\
1485	-18.5239494996101\\
1486	-17.4885892900379\\
1487	-17.1193570417702\\
1488	-13.6208566958771\\
1489	-17.2524049605145\\
1490	-23.4188734119825\\
1491	-17.4691118136002\\
1492	-17.8045358768866\\
1493	-31.5647811287413\\
1494	-26.1713901061762\\
1495	-23.5563719195688\\
1496	-30.6950186482384\\
1497	-29.8386601682712\\
1498	-19.0836067628049\\
1499	-16.0718945084936\\
1500	-15.5877427612409\\
1501	-21.7521065711583\\
1502	-34.9058671295302\\
1503	-33.9469173275947\\
1504	-33.5735457640465\\
1505	-27.4346881618396\\
1506	-19.1183675516727\\
1507	-17.6054833486639\\
1508	-14.488661500747\\
1509	-13.8475688674605\\
1510	-12.5836652502967\\
1511	-13.2405666919994\\
1512	-14.9838989396533\\
1513	-12.2337369555221\\
1514	-13.8390434703745\\
1515	-19.5332040561932\\
1516	-21.1574632228003\\
1517	-27.4197067513849\\
1518	-32.9277982638991\\
1519	-38.9702463037072\\
1520	-26.5545742910217\\
1521	-23.352643223787\\
1522	-26.4416780450865\\
1523	-23.1831080715219\\
1524	-17.8028842446972\\
1525	-18.5242958775686\\
1526	-31.1720647682946\\
1527	-34.291487056606\\
1528	-19.9972462274425\\
1529	-14.1858483787673\\
1530	-13.5946890790692\\
1531	-12.8588497886096\\
1532	-10.729663406195\\
1533	-9.43034492526385\\
1534	-11.3784201493688\\
1535	-16.3756811065593\\
1536	-14.0808253550051\\
1537	-13.06640103526\\
1538	-12.1408955055076\\
1539	-11.7467562702261\\
1540	-11.7733053786014\\
1541	-11.3590101005411\\
1542	-14.7155316144012\\
1543	-23.5879559238151\\
1544	-24.4186769056869\\
1545	-21.2827726037167\\
1546	-24.4006319348425\\
1547	-30.7470052356218\\
1548	-31.6602059395414\\
1549	-36.7319843174541\\
1550	-35.1812576945188\\
1551	-25.7322626160503\\
1552	-21.2098388257766\\
1553	-18.2658951337382\\
1554	-29.469013062788\\
1555	-32.5973177575077\\
1556	-21.3507907845067\\
1557	-17.9058793011045\\
1558	-14.0843700062933\\
1559	-14.360864361925\\
1560	-9.33045436499583\\
1561	-9.29595540646278\\
1562	-7.27480389796938\\
1563	-10.76417256996\\
1564	-13.9200654198946\\
1565	-12.426172383615\\
1566	-11.6269574673972\\
1567	-11.5223404519452\\
1568	-7.88616306054541\\
1569	-9.35934792765124\\
1570	-9.96477914137654\\
1571	-9.36721806134528\\
1572	-10.2811781057846\\
1573	-9.46956622960238\\
1574	-24.6258074360728\\
1575	-35.1030988232826\\
1576	-30.2417692258821\\
1577	-21.4878923143795\\
1578	-28.739406977397\\
1579	-44.4581685749361\\
1580	-39.2585095221709\\
1581	-36.9109628662279\\
1582	-28.5601586121611\\
1583	-26.7549883505358\\
1584	-30.6867281743296\\
1585	-20.8020555172966\\
1586	-20.2473531112355\\
1587	-14.277226814439\\
1588	-14.7240397483297\\
1589	-20.4011418388866\\
1590	-16.7912697046563\\
1591	-18.2873414404833\\
1592	-19.2841088114221\\
1593	-18.619785837046\\
1594	-19.2379310284498\\
1595	-18.2450416997417\\
1596	-20.5551859193315\\
1597	-21.146275668126\\
1598	-18.5358690240976\\
1599	-16.9509240652974\\
1600	-19.2127996286595\\
1601	-11.7476662339764\\
1602	-14.6677622640734\\
1603	-8.56208133366317\\
1604	-11.1759985786657\\
1605	-10.1557884352571\\
1606	-14.9753662428618\\
1607	-5.2951103580283\\
1608	-9.06876669702069\\
1609	-13.3270654449343\\
1610	-15.2456830165199\\
1611	-15.0405577936428\\
1612	-19.5241440187845\\
1613	-19.8693014625271\\
1614	-14.6148563608708\\
1615	-16.7315359714214\\
1616	-12.3876703843716\\
1617	-14.636585615481\\
1618	-11.5239375476121\\
1619	-11.9235852490594\\
1620	-7.18132725717624\\
1621	-7.76940109791417\\
1622	-9.55421269253702\\
1623	-14.1024247561691\\
1624	-15.1728910075544\\
1625	-13.8704786862654\\
1626	-12.8950237048766\\
1627	-13.0753204668947\\
1628	-13.7687396199451\\
1629	-12.9484827576282\\
1630	-12.5390867749438\\
1631	-11.2284020970242\\
1632	-10.8086781454936\\
1633	-9.63268201837263\\
1634	-9.4079379719692\\
1635	-12.723441198749\\
1636	-13.5839009493286\\
1637	-12.7228563939343\\
1638	-11.4039092960522\\
1639	-10.3076357231005\\
1640	-10.1138953293785\\
1641	-16.3814857818372\\
1642	-29.0416171666063\\
1643	-17.9344422457759\\
1644	-18.5891748719553\\
1645	-15.6848841297401\\
1646	-18.8135698680521\\
1647	-20.3757598391958\\
1648	-14.2793337471458\\
1649	-14.4185235505427\\
1650	-13.3832765693005\\
1651	-16.1489792246064\\
1652	-12.6508961156749\\
1653	-13.3234294869821\\
1654	-14.1476582691973\\
1655	-15.5992926198385\\
1656	-14.5151578903317\\
1657	-14.1368772053306\\
1658	-14.0929065002601\\
1659	-18.5982216254681\\
1660	-18.6813248435894\\
1661	-24.1960422873456\\
1662	-34.1030522727215\\
1663	-26.5286825536452\\
1664	-18.6934214336534\\
1665	-15.5919391371724\\
1666	-20.4884938965342\\
1667	-24.8672264972183\\
1668	-18.0933424386892\\
1669	-21.4828746905564\\
1670	-17.0611793762198\\
1671	-14.6042297054014\\
1672	-12.4602064530111\\
1673	-16.0029726615974\\
1674	-18.7547250704695\\
1675	-15.4009706793308\\
1676	-21.132241418491\\
1677	-20.0881203402337\\
1678	-19.8062780403179\\
1679	-15.9654459920766\\
1680	-17.9908522671699\\
1681	-22.3417483726796\\
1682	-18.3511464438466\\
1683	-19.4243693603676\\
1684	-17.5108289182883\\
1685	-15.1137309441935\\
1686	-14.827154681049\\
1687	-13.0631656481182\\
1688	-13.6982343639859\\
1689	-19.4155764380873\\
1690	-22.264707256483\\
1691	-22.6993375750912\\
1692	-20.0670677207643\\
1693	-16.8481257966771\\
1694	-14.428308643118\\
1695	-14.0136672704098\\
1696	-15.6051485056878\\
1697	-18.5941553963619\\
1698	-21.7344113900699\\
1699	-25.7991435294487\\
1700	-20.9214764253957\\
1701	-15.1933671339002\\
1702	-20.105859288808\\
1703	-31.5126256901988\\
1704	-21.8144706563924\\
1705	-21.9800337733318\\
1706	-24.5037986498619\\
1707	-22.430966546367\\
1708	-17.2726082249519\\
1709	-20.2015008527851\\
1710	-18.1970657629581\\
1711	-18.9463894064891\\
1712	-23.1469633435697\\
1713	-18.5637214876811\\
1714	-20.2813218022447\\
1715	-20.4796314701305\\
1716	-21.1939203392876\\
1717	-18.2214220043095\\
1718	-19.579939322063\\
1719	-17.4872175500673\\
1720	-16.7343109104772\\
1721	-19.1639043071023\\
1722	-17.9185180187462\\
1723	-13.0854662999252\\
1724	-13.889984393403\\
1725	-11.4294863204291\\
1726	-7.20095511698332\\
1727	-17.0855224757477\\
1728	-28.6173015363865\\
1729	-23.7171052900601\\
1730	-17.4030640139488\\
1731	-17.6347296686508\\
1732	-17.5819380658886\\
1733	-15.0308914278557\\
1734	-13.0957978741988\\
1735	-12.9283956737752\\
1736	-14.64841157245\\
1738	-15.2715471357944\\
1739	-13.8047302396988\\
1740	-13.5332185871632\\
1741	-10.9207846585705\\
1742	-10.7360752776424\\
1743	-16.7382042077595\\
1744	-14.4865736523393\\
1745	-16.2172329995033\\
1746	-16.7375880575169\\
1747	-19.7302895408648\\
1748	-20.8760365823819\\
1749	-24.9810639668035\\
1750	-21.8566836154353\\
1751	-20.5189740652206\\
1752	-22.3580573423189\\
1753	-16.0401513666645\\
1754	-19.8785519171361\\
1755	-16.1975629599531\\
1756	-17.6140996878041\\
1757	-19.6719327593378\\
1758	-23.1518119624855\\
1759	-19.8092257539818\\
1760	-21.9372143752357\\
1761	-29.7386115242837\\
1762	-23.389171175354\\
1764	-18.2352873499219\\
1765	-19.3966523801885\\
1766	-18.8286311411991\\
1767	-16.112440871635\\
1768	-14.9666913483709\\
1769	-15.5516213284695\\
1770	-14.6986335315919\\
1771	-22.007502671692\\
1772	-32.355175166864\\
1773	-27.7878991261402\\
1774	-25.1078890279816\\
1775	-24.6263416902184\\
1776	-17.6762293763122\\
1777	-16.0281571383903\\
1778	-14.0382916951469\\
1779	-12.4300693290738\\
1780	-11.8941941059236\\
1781	-17.0727273749073\\
1782	-21.826031703509\\
1783	-22.440725055822\\
1784	-21.3288555025404\\
1785	-16.2639060250865\\
1786	-22.4864761252056\\
1787	-27.7594661747405\\
1788	-29.1599178888916\\
1789	-24.21029849823\\
1790	-37.1730146997984\\
1791	-32.6750452855531\\
1792	-16.2734777145249\\
1793	-15.5781793809363\\
1794	-14.5267943717404\\
1795	-18.737793378657\\
1796	-24.902145575843\\
1797	-31.6895658060441\\
1798	-21.6560133612752\\
1799	-21.1722401607499\\
1800	-14.5175720784684\\
1801	-14.1035606059006\\
1802	-15.0165008292045\\
1803	-14.4623719446611\\
1804	-12.0779099413166\\
1805	-12.5232740369374\\
};
\addlegendentry{OSA predition}

\addplot [color=mycolor3, dotted, line width=2.0pt]
  table[row sep=crcr]{%
1006	-24.414\\
1007	-28.076\\
1008	-23.193\\
1009	-13.4280000000001\\
1010	-17.2179207345996\\
1012	-19.4944403743407\\
1013	-18.6819305957893\\
1014	-13.087561983567\\
1015	-17.0009924965887\\
1016	-14.7381673482973\\
1017	-15.1814481159433\\
1018	-31.5486432766359\\
1019	-31.0264268340795\\
1020	-28.9984899026015\\
1021	-23.1874067824081\\
1022	-19.7631271859211\\
1023	-27.1594085768534\\
1024	-21.6438546389111\\
1025	-18.8459351390479\\
1026	-23.7058744265898\\
1027	-22.6137007115183\\
1028	-20.5385859615517\\
1029	-26.3491294614437\\
1030	-27.509753117163\\
1031	-21.3512946949729\\
1032	-27.414759438225\\
1033	-27.1182089524655\\
1034	-23.0692330171769\\
1035	-20.7546251339577\\
1036	-18.714198637218\\
1037	-20.1615603435478\\
1038	-22.8524829447522\\
1039	-21.8377763742046\\
1040	-22.5358527663473\\
1041	-22.1255116217885\\
1042	-23.5732182145255\\
1043	-30.4100911535406\\
1044	-28.242414346757\\
1045	-21.1870821086382\\
1046	-20.940148141565\\
1047	-16.426176654619\\
1048	-16.7003428903461\\
1049	-18.1885684429963\\
1050	-16.3450272543498\\
1051	-16.5597746803753\\
1052	-19.3801049105359\\
1053	-21.4747438861152\\
1054	-20.5878549163692\\
1055	-27.3296180143518\\
1056	-23.6018823463919\\
1057	-17.6624970870425\\
1058	-17.0096013453326\\
1059	-17.9108019250668\\
1060	-19.0017122680886\\
1061	-15.13943982605\\
1062	-15.3713604320678\\
1063	-16.1051323055858\\
1064	-15.4915127804863\\
1065	-14.6788979660012\\
1066	-15.3334055528653\\
1067	-16.3297583562191\\
1068	-22.0706306220316\\
1069	-22.4571815733193\\
1070	-23.5057014555673\\
1071	-23.0603965316805\\
1072	-26.4572693873506\\
1073	-23.2873121425559\\
1074	-21.4161297228518\\
1075	-20.6844264966498\\
1076	-20.5224448380718\\
1077	-20.0494632370269\\
1078	-28.5317117849177\\
1079	-42.7807953531105\\
1080	-42.7117407793226\\
1081	-41.17025478956\\
1082	-27.4865864263063\\
1083	-36.8951764691426\\
1084	-45.9704929438308\\
1085	-44.8252767861593\\
1086	-35.3899010478544\\
1087	-44.1869187423779\\
1088	-60.454210240958\\
1089	-43.331859934289\\
1090	-37.4943323945422\\
1091	-26.0675395088008\\
1092	-23.0288880466812\\
1093	-20.6193925169605\\
1094	-23.1579945416586\\
1095	-19.9941981446696\\
1096	-15.0633050270699\\
1097	-15.2607305720962\\
1098	-16.0067549255944\\
1099	-20.9404398433855\\
1100	-20.0553359981573\\
1101	-20.1787830321948\\
1102	-23.8605500490155\\
1103	-25.6688037204108\\
1104	-31.6292141271701\\
1105	-28.4291428530566\\
1106	-31.2001052703113\\
1107	-26.8691271003825\\
1108	-23.5029214609449\\
1109	-26.0988324392224\\
1110	-21.7898635425986\\
1111	-20.3426903722177\\
1112	-22.7956435148565\\
1113	-22.3093058389147\\
1114	-18.9263316451595\\
1115	-19.0527936174376\\
1116	-16.6570741264386\\
1117	-17.370577570995\\
1118	-17.1927310889453\\
1119	-15.3776621731631\\
1120	-16.2503601344233\\
1121	-18.1488739802055\\
1122	-24.7236814585542\\
1123	-25.9494602507261\\
1124	-26.7714512888017\\
1125	-19.5493862423159\\
1126	-23.9664443932511\\
1127	-34.4122758971121\\
1128	-28.7162930437069\\
1129	-29.5600804138505\\
1130	-31.6483068872424\\
1131	-20.3533045428278\\
1132	-18.9188401937015\\
1133	-20.2982066580009\\
1134	-22.7975074285941\\
1135	-34.0202956473631\\
1136	-40.3724192670368\\
1137	-38.7564327449702\\
1138	-31.1224209852826\\
1139	-31.0528690350986\\
1140	-28.641537435888\\
1141	-28.5482830967958\\
1142	-28.326826928497\\
1143	-21.9159461958855\\
1144	-20.0466074472349\\
1146	-18.0192494737073\\
1147	-19.021353276534\\
1148	-23.3251656404761\\
1149	-33.5250495451294\\
1151	-21.4279484183885\\
1152	-20.4527884996187\\
1153	-19.9335626146924\\
1154	-16.2075266177603\\
1155	-15.69646584412\\
1156	-14.5933075778755\\
1157	-17.4133112916193\\
1158	-16.0721757166298\\
1159	-16.8831584285867\\
1160	-16.6181053373484\\
1161	-16.6058632320801\\
1162	-14.9506377353805\\
1163	-14.5194083674883\\
1164	-14.2529092424372\\
1165	-12.7649964225798\\
1166	-21.2598676924345\\
1167	-33.0982391751111\\
1168	-33.3954064036307\\
1169	-24.8193455049332\\
1170	-20.5789434158844\\
1171	-18.8718259373809\\
1172	-17.5631101524943\\
1173	-18.8767177489813\\
1174	-19.1534584065814\\
1175	-23.3365432826283\\
1176	-28.518475531162\\
1177	-45.5722594988711\\
1178	-54.8141133200998\\
1179	-44.3253823579651\\
1180	-42.5260072762346\\
1181	-28.9537961052822\\
1182	-33.5990059115081\\
1183	-33.3170007976787\\
1184	-35.0092891584043\\
1185	-29.0202611907348\\
1186	-26.2316986959147\\
1187	-26.6820834218215\\
1188	-24.5493231444784\\
1189	-26.9497935447419\\
1190	-22.1501648980982\\
1191	-31.4839405685823\\
1192	-38.5805807035063\\
1193	-27.7801664969863\\
1194	-20.2831824949792\\
1195	-20.9990812202009\\
1196	-31.7741769347972\\
1197	-37.9218436372973\\
1198	-43.4336485697215\\
1199	-45.6523738106355\\
1200	-45.3235459652428\\
1201	-35.6682351461154\\
1202	-33.7431553387839\\
1203	-37.4182450042617\\
1204	-42.7509249885306\\
1205	-28.5419589509072\\
1206	-19.8179748757896\\
1207	-25.962240993023\\
1208	-23.9015148048329\\
1209	-17.8662563824284\\
1210	-19.834054338841\\
1211	-22.8789380759799\\
1212	-19.6522609991039\\
1213	-17.6346526859788\\
1214	-19.0789056491942\\
1215	-18.6369233547111\\
1216	-20.8398299023497\\
1217	-27.7842130620538\\
1218	-25.7569466799184\\
1219	-20.6834265362165\\
1220	-20.455625815704\\
1221	-26.6113097741641\\
1222	-45.4457502761738\\
1223	-32.1969695781131\\
1224	-22.7860838994056\\
1225	-19.8982992367266\\
1226	-22.7928854704805\\
1227	-20.8706212136867\\
1228	-18.2701794282364\\
1229	-18.1379833433612\\
1230	-19.912948864818\\
1231	-22.6481350170504\\
1232	-25.1503590256382\\
1233	-19.1358441712644\\
1234	-25.775493686707\\
1235	-30.960293554502\\
1236	-27.4077385803284\\
1237	-24.8270689260371\\
1238	-24.7852275972991\\
1239	-32.1307291047058\\
1240	-22.7884303921423\\
1241	-16.534708929363\\
1242	-17.1209766507984\\
1243	-18.5607498443896\\
1244	-22.4557418405063\\
1245	-20.8496863352216\\
1246	-18.0145044544856\\
1247	-22.1965103877326\\
1248	-22.2352272927956\\
1249	-18.7926284833554\\
1250	-19.9678684441992\\
1251	-19.6511051741484\\
1252	-18.0545767980202\\
1253	-19.3574021706631\\
1254	-15.616217017747\\
1255	-16.3818552237765\\
1256	-14.6351072813854\\
1257	-15.0119052759026\\
1258	-18.4140641485521\\
1259	-21.5488128811301\\
1260	-26.7214071709216\\
1261	-36.1296224248676\\
1262	-25.85060548761\\
1263	-19.3794929100386\\
1264	-18.2348303442323\\
1265	-17.564565122917\\
1266	-19.5515965048874\\
1267	-15.6355914854935\\
1268	-16.4298394648292\\
1269	-17.5109224909163\\
1270	-27.4455385718993\\
1271	-25.7705366029313\\
1272	-30.5134124869198\\
1273	-23.4697573579469\\
1274	-22.2768065244836\\
1275	-16.8405732665133\\
1276	-17.1627118351776\\
1277	-15.2300337025549\\
1278	-14.7113190676457\\
1279	-14.1312834752516\\
1280	-17.2854318535628\\
1281	-20.5422167472134\\
1282	-20.1782465747451\\
1283	-21.0591812198754\\
1284	-31.4402434856363\\
1285	-29.6099773649501\\
1286	-23.1568177866463\\
1287	-25.584391974865\\
1288	-26.5066691465681\\
1289	-33.014453316955\\
1290	-26.6451379508824\\
1291	-21.0451952157875\\
1292	-16.7363372628106\\
1293	-17.6643122927676\\
1294	-15.3701183284588\\
1295	-14.4242922683857\\
1296	-14.5324270930275\\
1297	-14.3749203777352\\
1298	-14.8599657690852\\
1299	-19.2415280240944\\
1300	-18.7289635622774\\
1301	-19.0361998103517\\
1302	-19.8349449027514\\
1303	-16.1922788096015\\
1304	-16.0549212272913\\
1305	-15.0208479819453\\
1306	-21.6966438763668\\
1307	-21.796850611609\\
1308	-22.7131582904055\\
1309	-20.7557131968224\\
1310	-18.4207250005988\\
1311	-21.1140864619847\\
1312	-26.3102243425253\\
1313	-27.2295617192565\\
1314	-20.337091469612\\
1315	-28.5598570205275\\
1316	-26.6188525472162\\
1317	-23.7810774434849\\
1318	-26.9340554895452\\
1319	-26.6952315294834\\
1320	-22.7200863113819\\
1321	-20.664512178048\\
1322	-29.6836598992402\\
1323	-41.0382828394891\\
1324	-32.2538420787255\\
1325	-21.3824040309278\\
1326	-21.9076367816517\\
1327	-22.3511450356648\\
1328	-25.3332894476339\\
1329	-28.0720720238132\\
1330	-22.4572876106954\\
1331	-23.3029112773188\\
1332	-28.0772471817822\\
1333	-30.1077623330971\\
1334	-22.5627605460688\\
1335	-19.87302339692\\
1336	-22.9867272258962\\
1337	-36.2990632094486\\
1338	-31.4152519272088\\
1339	-29.8697381765021\\
1340	-22.2026531934318\\
1341	-20.6975910901547\\
1342	-20.5313056377563\\
1343	-16.7959234569912\\
1344	-15.4950625648794\\
1345	-15.256678437117\\
1346	-14.1486698513152\\
1347	-14.3742026267851\\
1348	-20.0038283503429\\
1349	-24.3267117011299\\
1350	-19.3375330743568\\
1351	-19.7806856177117\\
1352	-21.4082956855545\\
1353	-16.8670628582081\\
1354	-17.117476513303\\
1355	-16.4788035060953\\
1356	-15.1661855831471\\
1357	-15.932512941708\\
1358	-21.1245943591036\\
1359	-25.7767930271007\\
1360	-18.9026580652417\\
1361	-17.3727205431278\\
1362	-16.411261847461\\
1363	-16.477279519105\\
1364	-14.8004980850387\\
1365	-19.1444848995436\\
1366	-34.9207594012664\\
1367	-26.8002237677047\\
1368	-31.6234572445953\\
1369	-37.3630031149253\\
1370	-37.4986254062835\\
1371	-29.9613675661237\\
1372	-24.1960582967006\\
1373	-24.2952017796115\\
1374	-24.7173467079613\\
1375	-26.7898371082238\\
1376	-29.0839615606383\\
1377	-28.57050690347\\
1378	-36.057902359873\\
1379	-42.3990108981357\\
1380	-47.4940434240173\\
1381	-34.7799260840404\\
1382	-35.3268896076365\\
1383	-40.6648279731351\\
1384	-33.6880878165841\\
1385	-36.0271451454121\\
1386	-32.7667760998336\\
1387	-20.5103572188989\\
1388	-17.2082837769972\\
1389	-17.2678384430064\\
1390	-20.9349439358612\\
1391	-18.617446284959\\
1392	-15.8595447242405\\
1393	-17.8191885096742\\
1394	-15.7329056963849\\
1395	-14.8742354181538\\
1396	-15.0610994243787\\
1397	-15.5548340695234\\
1398	-14.5542829568894\\
1399	-17.5997071232086\\
1400	-23.3793337314171\\
1401	-18.6519125376117\\
1402	-27.8806183640006\\
1403	-43.0611872870447\\
1404	-36.1997132119232\\
1405	-41.7173154310033\\
1406	-33.5498022113279\\
1407	-33.5094545121226\\
1409	-19.5501057593317\\
1410	-21.3077644166983\\
1411	-18.9926189664266\\
1412	-17.2229316324342\\
1413	-18.4220181739736\\
1414	-14.5220180411347\\
1415	-15.1575134867576\\
1416	-13.7791418207246\\
1417	-17.6287875224352\\
1418	-23.6914844006578\\
1419	-26.6349663305823\\
1420	-26.120108037376\\
1421	-24.9322859182653\\
1422	-26.9787714070333\\
1423	-24.3011000156803\\
1424	-21.069876748775\\
1425	-21.7953336235128\\
1426	-19.3979542281652\\
1427	-24.3029367747499\\
1428	-19.0928435878543\\
1429	-18.2023617956181\\
1430	-20.6923892189791\\
1431	-27.3773976064115\\
1432	-22.181454889364\\
1433	-19.3321692379836\\
1434	-18.3114583668753\\
1435	-19.4325706507238\\
1436	-16.5569075066485\\
1437	-16.0536395836186\\
1438	-18.0538747833741\\
1439	-20.2071268307523\\
1440	-21.3219957161662\\
1441	-18.8977448944756\\
1442	-21.5074891014779\\
1443	-21.2007305005816\\
1444	-16.4224627375411\\
1445	-16.790383179871\\
1446	-23.6419365525489\\
1447	-32.9583220445127\\
1448	-30.0199224266821\\
1449	-26.1948412864042\\
1450	-25.2152028949713\\
1451	-23.0136014676782\\
1453	-19.6481103181623\\
1454	-21.8588234984345\\
1455	-21.1358762799744\\
1456	-17.4156383079885\\
1457	-20.1566409346303\\
1458	-21.9125486236533\\
1459	-24.1543371231051\\
1460	-25.8695115216813\\
1461	-25.4260219200639\\
1462	-32.1126391264281\\
1463	-41.602242748826\\
1464	-46.7132863283655\\
1465	-28.304696907386\\
1466	-21.94501095177\\
1467	-18.0919253397035\\
1468	-17.8350055062276\\
1469	-16.6828110677429\\
1470	-15.9800664267875\\
1471	-22.3534160742909\\
1472	-21.2752546464237\\
1473	-19.498600520971\\
1474	-22.1205646046976\\
1475	-23.8433871942414\\
1476	-27.4656773732847\\
1477	-28.7053352993432\\
1478	-39.5223742324201\\
1479	-37.5592979977621\\
1480	-28.6185629617132\\
1481	-23.580199124247\\
1482	-23.2493102992159\\
1483	-23.9518971745888\\
1484	-26.7124182790544\\
1485	-22.2922019290752\\
1486	-21.0747367626207\\
1487	-21.0267545145953\\
1488	-16.329192364464\\
1489	-20.2933110026022\\
1490	-27.4731741036073\\
1491	-21.5570227211888\\
1492	-21.4525522462363\\
1493	-35.0278266512948\\
1494	-28.5926749310506\\
1495	-25.738734157617\\
1496	-33.5199004801202\\
1497	-33.1058272688833\\
1498	-22.4104239970691\\
1499	-18.56499513938\\
1500	-18.6761353866762\\
1501	-25.1739534040771\\
1502	-38.8478643520957\\
1503	-39.1954563904835\\
1504	-37.9398335030455\\
1505	-31.9657746319233\\
1506	-23.4713566527705\\
1507	-21.6791441570638\\
1508	-18.8594805802816\\
1509	-17.7534577892641\\
1510	-16.8788663316011\\
1511	-18.3627911088374\\
1512	-19.5471310334829\\
1513	-16.1568526119295\\
1514	-17.7613079637688\\
1515	-22.9896816115836\\
1516	-24.143964847121\\
1517	-30.2980604910711\\
1518	-37.4512285146109\\
1519	-44.005195616026\\
1520	-32.7292321221591\\
1521	-28.9584488573132\\
1522	-31.3700328766818\\
1523	-28.1968459206628\\
1524	-22.6512172705015\\
1525	-23.6695113622657\\
1526	-35.980526427451\\
1527	-39.5191303570741\\
1528	-25.1476378380223\\
1529	-18.1230324094081\\
1530	-17.327688748494\\
1531	-17.5772983692714\\
1532	-16.2857399000602\\
1533	-14.354717232666\\
1534	-16.2346622335954\\
1535	-20.4331158327248\\
1536	-17.9734925550465\\
1537	-16.3309518304175\\
1538	-15.963372041602\\
1539	-15.2706922155066\\
1540	-14.9037383766654\\
1541	-14.8294998356703\\
1542	-17.725585349106\\
1543	-26.4224521727103\\
1544	-27.5462968810245\\
1545	-26.0500117818183\\
1546	-28.4087831845454\\
1547	-35.6619516224703\\
1548	-36.0958856378611\\
1549	-40.5884969601022\\
1550	-39.5241870673153\\
1551	-29.5359175746473\\
1552	-24.7447492367296\\
1553	-22.7712816250398\\
1554	-34.2484324377344\\
1555	-37.9256279982599\\
1556	-26.1810784885938\\
1557	-21.0501958896643\\
1558	-16.2929043043657\\
1559	-17.7489245493821\\
1560	-15.2988668541434\\
1561	-14.6765824798554\\
1562	-13.2523009546107\\
1563	-16.2279388027566\\
1564	-17.1252076972858\\
1565	-16.3496490961186\\
1566	-14.5464946509574\\
1567	-14.6728574507956\\
1568	-13.0949483468287\\
1569	-13.0531400339487\\
1570	-13.682700498149\\
1571	-13.0770854582724\\
1572	-12.8451087673445\\
1573	-12.8182770032413\\
1574	-27.016334798971\\
1575	-37.7538772603821\\
1576	-35.471998358727\\
1577	-26.9740000651511\\
1578	-33.5668605021467\\
1579	-49.0210294812041\\
1580	-44.7100308178353\\
1581	-42.9343545883719\\
1582	-35.4561622435726\\
1583	-33.4181908426763\\
1584	-36.1758014812344\\
1585	-25.9027311743966\\
1586	-24.3616317787273\\
1587	-18.5256788263023\\
1588	-18.0510082130058\\
1589	-24.3964316019137\\
1590	-19.9732770330425\\
1591	-20.3773347004892\\
1592	-22.129570705973\\
1593	-21.1264119841335\\
1594	-21.4130242287615\\
1595	-21.9601493261339\\
1596	-23.913683034647\\
1597	-25.4174493460491\\
1598	-22.3091320597114\\
1599	-19.6403357319921\\
1600	-23.0303896634987\\
1601	-15.3479388734661\\
1602	-17.7913106807221\\
1603	-14.8118039728649\\
1604	-15.2051280660619\\
1605	-12.6666093899182\\
1606	-18.0591163326426\\
1607	-12.0160216198658\\
1608	-12.6945619089277\\
1609	-16.3910039320162\\
1610	-18.8644587636859\\
1611	-17.0076831956601\\
1612	-21.8362730681717\\
1613	-22.3923703557118\\
1614	-17.6172996540113\\
1615	-20.5714284051735\\
1616	-15.158251932804\\
1617	-16.1415346590277\\
1618	-14.7074568434059\\
1619	-15.7583763025227\\
1620	-12.0748302587722\\
1621	-11.6928258282362\\
1622	-13.0525171368843\\
1623	-18.6476786722023\\
1624	-18.7636863615762\\
1625	-16.0895437692238\\
1626	-16.1236011715632\\
1627	-14.849074154703\\
1628	-16.0356117200315\\
1629	-14.5262571694202\\
1630	-14.4573946957391\\
1631	-13.8570325802691\\
1632	-13.214828237001\\
1633	-13.0045225289728\\
1634	-12.4943752451093\\
1635	-14.87626516261\\
1636	-14.8981702753345\\
1637	-14.321389758911\\
1638	-14.0198548165433\\
1639	-13.2036600401632\\
1640	-13.1758968235406\\
1641	-19.0514663927572\\
1642	-29.5738278295403\\
1643	-20.271540862758\\
1644	-19.8369416410212\\
1645	-16.9814592289663\\
1646	-21.1801577496387\\
1647	-21.9292541631614\\
1648	-16.840618253952\\
1649	-18.1252084377797\\
1650	-16.0454404144259\\
1651	-18.7719693465133\\
1652	-14.9876610831118\\
1653	-15.2101087612782\\
1654	-15.1871625558144\\
1655	-17.3547253244728\\
1656	-15.7324977985827\\
1657	-15.954964047241\\
1658	-15.8712765890305\\
1659	-19.9363196070892\\
1660	-19.6232971660672\\
1661	-25.1585854266057\\
1662	-35.6325719212655\\
1663	-29.0466595787705\\
1664	-21.4788189369881\\
1665	-19.3281302952541\\
1666	-24.0612853570778\\
1667	-29.8826552713419\\
1668	-22.7930881350553\\
1669	-25.4444089992182\\
1670	-19.8017843801367\\
1671	-16.6804804698802\\
1672	-16.3229294790867\\
1673	-19.9560866626066\\
1674	-20.9636096365766\\
1675	-19.3708839238302\\
1676	-24.78796141671\\
1677	-23.1118521280855\\
1678	-22.0192387870406\\
1679	-18.3752537750497\\
1680	-20.3346668519778\\
1681	-24.1122679108428\\
1682	-20.6108149357074\\
1683	-21.2731129378385\\
1684	-20.3238134853575\\
1685	-17.945958974374\\
1686	-18.1981292144849\\
1687	-15.9152419429447\\
1688	-16.4088942505839\\
1689	-21.9785139291182\\
1690	-24.3935109639831\\
1691	-25.1310835182594\\
1692	-23.170092823533\\
1693	-19.44584568955\\
1694	-17.3522188846362\\
1695	-17.7911612502307\\
1696	-18.8168152921601\\
1697	-21.9283783900528\\
1698	-24.7981443613844\\
1699	-28.792988369808\\
1700	-23.8850250621244\\
1701	-18.4513256571192\\
1702	-24.0058423830887\\
1703	-34.3225294305905\\
1704	-24.8288886497166\\
1705	-24.8119761641294\\
1706	-27.1637741515042\\
1707	-24.3105323299314\\
1708	-20.4346483624333\\
1709	-22.7581420889755\\
1710	-20.9808823561414\\
1711	-22.3486083616756\\
1712	-26.1816089501708\\
1713	-21.6090192887457\\
1714	-23.4461696365104\\
1715	-22.8879964780094\\
1716	-23.23553239219\\
1717	-21.020792210081\\
1718	-23.062105429645\\
1719	-19.3421843833271\\
1720	-19.6865778535657\\
1721	-21.4488465652009\\
1722	-20.4523933157882\\
1723	-15.4850480011726\\
1724	-16.3100803482125\\
1725	-16.2278085514895\\
1726	-13.1745750608907\\
1727	-21.1476567140157\\
1728	-32.1580645546162\\
1729	-29.024640466512\\
1730	-22.3840215245823\\
1731	-23.1729592957122\\
1732	-22.8348686586965\\
1733	-20.5326398504205\\
1734	-17.5374250770583\\
1735	-17.7524554656256\\
1736	-17.5894197269433\\
1737	-17.5944466191356\\
1738	-18.5718988286815\\
1739	-16.8021235000524\\
1740	-16.6109559725105\\
1741	-14.8572428380585\\
1742	-15.211350861551\\
1743	-21.4671359846072\\
1744	-17.9678759214714\\
1745	-19.8302838597137\\
1746	-18.8128370612226\\
1747	-21.8626106612958\\
1748	-21.8105846295628\\
1749	-26.3769044086223\\
1750	-22.6838396112412\\
1751	-23.3919766546924\\
1752	-23.5940370681433\\
1753	-17.1925045438411\\
1754	-22.1593220081625\\
1755	-18.7123326261781\\
1756	-20.307120299263\\
1757	-21.3877953685128\\
1758	-24.4969735412196\\
1759	-20.3940600569454\\
1760	-23.323556281669\\
1761	-29.823493467358\\
1762	-25.2301814086652\\
1763	-22.2751223144828\\
1764	-20.1999090705763\\
1765	-22.5703154260682\\
1766	-21.2485022827098\\
1767	-18.9182757160747\\
1768	-17.8520003282276\\
1769	-18.8533552720555\\
1770	-18.2070003858969\\
1771	-26.1241627204706\\
1772	-34.7441216221318\\
1773	-30.8949274234878\\
1774	-28.2350954614492\\
1775	-27.6992572823799\\
1776	-19.8577430810165\\
1777	-19.033368830251\\
1778	-17.3943266033909\\
1779	-16.4867866099628\\
1780	-16.2616869445023\\
1781	-20.8611176870177\\
1782	-24.9601309804536\\
1783	-26.0101649676669\\
1784	-23.7441904502743\\
1785	-19.3350683463127\\
1787	-31.3937582897356\\
1788	-32.1726151434145\\
1789	-27.218336190921\\
1790	-40.3014021553374\\
1791	-35.0043539274027\\
1792	-20.7770997859045\\
1793	-19.1481832717898\\
1794	-17.8809953343427\\
1796	-28.1744243792418\\
1797	-35.5188935673086\\
1798	-26.0317319244673\\
1799	-23.5555097648798\\
1800	-17.4737393478611\\
1801	-17.984645557653\\
1802	-17.6052206791769\\
1803	-18.4573133491583\\
1804	-15.6222887479719\\
1805	-16.6766187784908\\
};
\addlegendentry{MPO prediction}

\end{axis}

\begin{axis}[%
width=6.159cm,
height=1.831cm,
at={(0cm,7.627cm)},
scale only axis,
xmin=1000,
xmax=2000,
xlabel style={font=\color{white!15!black}},
xlabel={Sample index},
ymin=-50.8795298561816,
ymax=1.221,
ylabel style={font=\color{white!15!black}},
ylabel={$y(t)$},
axis background/.style={fill=white},
title style={font=\bfseries},
title={C3: RMSE(OSA) = 2.9683, RMSE(MPO) = 6.8292},
legend style={legend cell align=left, align=left, draw=white!15!black}
]
\addplot [color=mycolor1, line width=2.0pt]
  table[row sep=crcr]{%
1006	-18.3109999999999\\
1007	-23.193\\
1008	-18.3109999999999\\
1009	-9.76600000000008\\
1010	-15.8689999999999\\
1011	-12.2070000000001\\
1012	-14.6479999999999\\
1013	-10.9860000000001\\
1014	-3.66200000000003\\
1015	-2.44100000000003\\
1016	-4.88300000000004\\
1017	-6.10400000000004\\
1018	-14.6479999999999\\
1019	-17.0899999999999\\
1020	-17.0899999999999\\
1022	-9.76600000000008\\
1023	-18.3109999999999\\
1025	-10.9860000000001\\
1026	-13.4280000000001\\
1028	-10.9860000000001\\
1029	-20.752\\
1030	-19.5309999999999\\
1031	-12.2070000000001\\
1032	-20.752\\
1033	-20.752\\
1034	-15.8689999999999\\
1035	-13.4280000000001\\
1036	-9.76600000000008\\
1037	-14.6479999999999\\
1041	-14.6479999999999\\
1042	-15.8689999999999\\
1043	-21.973\\
1044	-20.752\\
1045	-13.4280000000001\\
1046	-13.4280000000001\\
1047	-8.54500000000007\\
1048	-12.2070000000001\\
1049	-12.2070000000001\\
1050	-13.4280000000001\\
1051	-10.9860000000001\\
1052	-13.4280000000001\\
1053	-14.6479999999999\\
1054	-13.4280000000001\\
1055	-20.752\\
1056	-13.4280000000001\\
1057	-9.76600000000008\\
1058	-7.32400000000007\\
1059	-8.54500000000007\\
1060	-10.9860000000001\\
1061	-6.10400000000004\\
1062	-9.76600000000008\\
1063	-10.9860000000001\\
1064	-6.10400000000004\\
1065	-6.10400000000004\\
1066	-9.76600000000008\\
1067	-9.76600000000008\\
1068	-15.8689999999999\\
1069	-14.6479999999999\\
1070	-17.0899999999999\\
1071	-15.8689999999999\\
1072	-18.3109999999999\\
1073	-13.4280000000001\\
1074	-14.6479999999999\\
1075	-12.2070000000001\\
1076	-10.9860000000001\\
1077	-12.2070000000001\\
1078	-23.193\\
1079	-28.076\\
1080	-30.518\\
1081	-29.297\\
1082	-18.3109999999999\\
1083	-28.076\\
1084	-32.9590000000001\\
1085	-31.7380000000001\\
1086	-20.752\\
1087	-34.1800000000001\\
1088	-40.2829999999999\\
1089	-29.297\\
1091	-19.5309999999999\\
1093	-12.2070000000001\\
1094	-14.6479999999999\\
1095	-9.76600000000008\\
1096	-7.32400000000007\\
1097	-7.32400000000007\\
1098	-10.9860000000001\\
1099	-13.4280000000001\\
1100	-12.2070000000001\\
1101	-12.2070000000001\\
1102	-15.8689999999999\\
1103	-18.3109999999999\\
1104	-19.5309999999999\\
1105	-15.8689999999999\\
1106	-21.973\\
1107	-15.8689999999999\\
1108	-15.8689999999999\\
1109	-17.0899999999999\\
1110	-12.2070000000001\\
1111	-12.2070000000001\\
1112	-15.8689999999999\\
1113	-14.6479999999999\\
1114	-8.54500000000007\\
1115	-12.2070000000001\\
1116	-8.54500000000007\\
1117	-9.76600000000008\\
1118	-9.76600000000008\\
1119	-7.32400000000007\\
1121	-12.2070000000001\\
1122	-17.0899999999999\\
1124	-17.0899999999999\\
1125	-10.9860000000001\\
1126	-12.2070000000001\\
1127	-23.193\\
1128	-15.8689999999999\\
1129	-20.752\\
1130	-21.973\\
1131	-12.2070000000001\\
1132	-9.76600000000008\\
1133	-12.2070000000001\\
1134	-13.4280000000001\\
1135	-21.973\\
1136	-25.635\\
1137	-26.855\\
1138	-19.5309999999999\\
1139	-20.752\\
1140	-18.3109999999999\\
1141	-17.0899999999999\\
1142	-18.3109999999999\\
1143	-13.4280000000001\\
1144	-12.2070000000001\\
1145	-9.76600000000008\\
1146	-9.76600000000008\\
1147	-10.9860000000001\\
1148	-15.8689999999999\\
1149	-23.193\\
1150	-17.0899999999999\\
1151	-13.4280000000001\\
1152	-10.9860000000001\\
1153	-10.9860000000001\\
1154	-7.32400000000007\\
1155	-6.10400000000004\\
1156	-6.10400000000004\\
1157	-12.2070000000001\\
1158	-7.32400000000007\\
1159	-8.54500000000007\\
1161	-8.54500000000007\\
1162	-7.32400000000007\\
1163	-4.88300000000004\\
1164	-3.66200000000003\\
1165	-8.54500000000007\\
1166	-15.8689999999999\\
1168	-20.752\\
1170	-10.9860000000001\\
1171	-8.54500000000007\\
1172	-4.88300000000004\\
1173	-9.76600000000008\\
1174	-8.54500000000007\\
1175	-15.8689999999999\\
1176	-17.0899999999999\\
1177	-29.297\\
1178	-32.9590000000001\\
1179	-25.635\\
1180	-25.635\\
1181	-18.3109999999999\\
1182	-23.193\\
1183	-21.973\\
1184	-24.414\\
1185	-15.8689999999999\\
1186	-17.0899999999999\\
1187	-17.0899999999999\\
1188	-15.8689999999999\\
1189	-15.8689999999999\\
1190	-13.4280000000001\\
1191	-23.193\\
1192	-25.635\\
1194	-13.4280000000001\\
1195	-14.6479999999999\\
1196	-23.193\\
1197	-24.414\\
1198	-28.076\\
1199	-30.518\\
1200	-29.297\\
1201	-23.193\\
1202	-21.973\\
1203	-25.635\\
1204	-28.076\\
1205	-18.3109999999999\\
1206	-12.2070000000001\\
1207	-19.5309999999999\\
1209	-9.76600000000008\\
1210	-13.4280000000001\\
1211	-14.6479999999999\\
1212	-10.9860000000001\\
1214	-10.9860000000001\\
1215	-8.54500000000007\\
1216	-10.9860000000001\\
1217	-18.3109999999999\\
1218	-17.0899999999999\\
1219	-12.2070000000001\\
1220	-12.2070000000001\\
1221	-18.3109999999999\\
1222	-26.855\\
1223	-17.0899999999999\\
1224	-14.6479999999999\\
1225	-10.9860000000001\\
1226	-13.4280000000001\\
1228	-8.54500000000007\\
1229	-8.54500000000007\\
1231	-13.4280000000001\\
1232	-14.6479999999999\\
1233	-12.2070000000001\\
1234	-15.8689999999999\\
1235	-20.752\\
1236	-17.0899999999999\\
1237	-12.2070000000001\\
1238	-15.8689999999999\\
1239	-20.752\\
1240	-13.4280000000001\\
1241	-7.32400000000007\\
1242	-13.4280000000001\\
1243	-10.9860000000001\\
1244	-12.2070000000001\\
1245	-9.76600000000008\\
1246	-8.54500000000007\\
1247	-13.4280000000001\\
1248	-13.4280000000001\\
1249	-9.76600000000008\\
1250	-12.2070000000001\\
1252	-7.32400000000007\\
1253	-12.2070000000001\\
1254	-8.54500000000007\\
1255	-9.76600000000008\\
1257	-4.88300000000004\\
1258	-12.2070000000001\\
1259	-15.8689999999999\\
1260	-17.0899999999999\\
1261	-23.193\\
1262	-15.8689999999999\\
1263	-9.76600000000008\\
1264	-8.54500000000007\\
1265	-8.54500000000007\\
1266	-10.9860000000001\\
1267	-8.54500000000007\\
1268	-4.88300000000004\\
1269	-10.9860000000001\\
1270	-15.8689999999999\\
1271	-13.4280000000001\\
1272	-20.752\\
1273	-15.8689999999999\\
1274	-14.6479999999999\\
1275	-8.54500000000007\\
1276	-10.9860000000001\\
1277	-4.88300000000004\\
1278	-3.66200000000003\\
1279	-4.88300000000004\\
1280	-9.76600000000008\\
1283	-13.4280000000001\\
1284	-20.752\\
1285	-17.0899999999999\\
1286	-14.6479999999999\\
1288	-19.5309999999999\\
1289	-20.752\\
1291	-13.4280000000001\\
1292	-7.32400000000007\\
1294	-4.88300000000004\\
1295	-7.32400000000007\\
1296	-7.32400000000007\\
1297	-4.88300000000004\\
1299	-12.2070000000001\\
1300	-10.9860000000001\\
1301	-12.2070000000001\\
1302	-12.2070000000001\\
1304	-4.88300000000004\\
1305	-9.76600000000008\\
1306	-15.8689999999999\\
1307	-13.4280000000001\\
1308	-15.8689999999999\\
1309	-13.4280000000001\\
1310	-9.76600000000008\\
1311	-14.6479999999999\\
1312	-18.3109999999999\\
1313	-18.3109999999999\\
1314	-13.4280000000001\\
1315	-20.752\\
1316	-15.8689999999999\\
1319	-19.5309999999999\\
1320	-13.4280000000001\\
1321	-13.4280000000001\\
1322	-18.3109999999999\\
1323	-26.855\\
1324	-21.973\\
1325	-13.4280000000001\\
1327	-13.4280000000001\\
1328	-15.8689999999999\\
1329	-17.0899999999999\\
1330	-12.2070000000001\\
1331	-14.6479999999999\\
1332	-18.3109999999999\\
1333	-20.752\\
1334	-13.4280000000001\\
1335	-12.2070000000001\\
1336	-15.8689999999999\\
1337	-23.193\\
1338	-20.752\\
1339	-21.973\\
1340	-14.6479999999999\\
1341	-14.6479999999999\\
1343	-9.76600000000008\\
1344	-4.88300000000004\\
1345	-3.66200000000003\\
1346	-3.66200000000003\\
1347	-7.32400000000007\\
1348	-13.4280000000001\\
1349	-14.6479999999999\\
1350	-9.76600000000008\\
1351	-12.2070000000001\\
1352	-13.4280000000001\\
1353	-9.76600000000008\\
1354	-8.54500000000007\\
1355	-8.54500000000007\\
1356	-6.10400000000004\\
1357	-9.76600000000008\\
1358	-14.6479999999999\\
1359	-15.8689999999999\\
1360	-10.9860000000001\\
1361	-8.54500000000007\\
1363	-8.54500000000007\\
1364	-7.32400000000007\\
1365	-10.9860000000001\\
1366	-20.752\\
1367	-18.3109999999999\\
1368	-18.3109999999999\\
1369	-24.414\\
1370	-23.193\\
1371	-18.3109999999999\\
1372	-14.6479999999999\\
1374	-14.6479999999999\\
1376	-19.5309999999999\\
1377	-19.5309999999999\\
1378	-25.635\\
1379	-26.855\\
1380	-31.7380000000001\\
1381	-24.414\\
1383	-26.855\\
1384	-21.973\\
1385	-24.414\\
1386	-20.752\\
1387	-13.4280000000001\\
1388	-9.76600000000008\\
1389	-9.76600000000008\\
1390	-12.2070000000001\\
1392	-7.32400000000007\\
1393	-10.9860000000001\\
1395	-6.10400000000004\\
1396	-7.32400000000007\\
1397	-9.76600000000008\\
1398	-6.10400000000004\\
1399	-12.2070000000001\\
1400	-14.6479999999999\\
1401	-10.9860000000001\\
1402	-17.0899999999999\\
1403	-24.414\\
1404	-24.414\\
1405	-28.076\\
1406	-23.193\\
1407	-21.973\\
1408	-17.0899999999999\\
1409	-9.76600000000008\\
1410	-10.9860000000001\\
1411	-10.9860000000001\\
1412	-7.32400000000007\\
1413	-10.9860000000001\\
1414	-9.76600000000008\\
1415	-2.44100000000003\\
1417	-9.76600000000008\\
1420	-17.0899999999999\\
1421	-14.6479999999999\\
1422	-17.0899999999999\\
1423	-14.6479999999999\\
1426	-10.9860000000001\\
1427	-14.6479999999999\\
1428	-13.4280000000001\\
1429	-10.9860000000001\\
1430	-13.4280000000001\\
1431	-18.3109999999999\\
1432	-13.4280000000001\\
1433	-10.9860000000001\\
1435	-10.9860000000001\\
1436	-8.54500000000007\\
1437	-7.32400000000007\\
1438	-10.9860000000001\\
1439	-13.4280000000001\\
1440	-13.4280000000001\\
1441	-10.9860000000001\\
1442	-13.4280000000001\\
1443	-13.4280000000001\\
1444	-8.54500000000007\\
1445	-7.32400000000007\\
1446	-15.8689999999999\\
1447	-19.5309999999999\\
1448	-18.3109999999999\\
1449	-18.3109999999999\\
1452	-10.9860000000001\\
1453	-9.76600000000008\\
1454	-13.4280000000001\\
1455	-13.4280000000001\\
1456	-8.54500000000007\\
1457	-12.2070000000001\\
1458	-14.6479999999999\\
1459	-14.6479999999999\\
1460	-17.0899999999999\\
1461	-17.0899999999999\\
1462	-20.752\\
1463	-28.076\\
1464	-30.518\\
1465	-20.752\\
1466	-13.4280000000001\\
1467	-8.54500000000007\\
1468	-6.10400000000004\\
1469	-8.54500000000007\\
1470	-7.32400000000007\\
1471	-10.9860000000001\\
1472	-12.2070000000001\\
1473	-12.2070000000001\\
1474	-13.4280000000001\\
1475	-15.8689999999999\\
1476	-19.5309999999999\\
1477	-19.5309999999999\\
1478	-28.076\\
1480	-18.3109999999999\\
1481	-14.6479999999999\\
1483	-14.6479999999999\\
1484	-17.0899999999999\\
1486	-12.2070000000001\\
1487	-13.4280000000001\\
1488	-8.54500000000007\\
1489	-7.32400000000007\\
1490	-17.0899999999999\\
1492	-12.2070000000001\\
1493	-23.193\\
1494	-18.3109999999999\\
1495	-15.8689999999999\\
1496	-21.973\\
1497	-23.193\\
1498	-14.6479999999999\\
1499	-8.54500000000007\\
1500	-9.76600000000008\\
1501	-15.8689999999999\\
1502	-23.193\\
1503	-25.635\\
1504	-25.635\\
1505	-20.752\\
1506	-13.4280000000001\\
1507	-10.9860000000001\\
1508	-9.76600000000008\\
1509	-7.32400000000007\\
1510	-9.76600000000008\\
1511	-9.76600000000008\\
1512	-10.9860000000001\\
1513	-8.54500000000007\\
1514	-9.76600000000008\\
1515	-15.8689999999999\\
1516	-17.0899999999999\\
1517	-17.0899999999999\\
1518	-24.414\\
1519	-29.297\\
1520	-18.3109999999999\\
1521	-18.3109999999999\\
1522	-19.5309999999999\\
1523	-17.0899999999999\\
1524	-13.4280000000001\\
1525	-13.4280000000001\\
1526	-23.193\\
1527	-25.635\\
1529	-8.54500000000007\\
1531	-3.66200000000003\\
1532	-2.44100000000003\\
1533	-6.10400000000004\\
1534	-8.54500000000007\\
1535	-9.76600000000008\\
1536	-9.76600000000008\\
1537	-7.32400000000007\\
1538	-7.32400000000007\\
1539	-8.54500000000007\\
1540	-6.10400000000004\\
1541	-8.54500000000007\\
1543	-15.8689999999999\\
1544	-17.0899999999999\\
1545	-17.0899999999999\\
1546	-19.5309999999999\\
1547	-23.193\\
1548	-24.414\\
1549	-26.855\\
1550	-28.076\\
1551	-20.752\\
1552	-15.8689999999999\\
1553	-13.4280000000001\\
1554	-20.752\\
1555	-24.414\\
1556	-17.0899999999999\\
1557	-13.4280000000001\\
1558	-7.32400000000007\\
1559	-4.88300000000004\\
1560	-4.88300000000004\\
1561	-6.10400000000004\\
1562	-2.44100000000003\\
1563	-12.2070000000001\\
1564	-9.76600000000008\\
1566	-7.32400000000007\\
1567	-4.88300000000004\\
1568	-7.32400000000007\\
1569	-8.54500000000007\\
1570	-7.32400000000007\\
1571	-4.88300000000004\\
1572	-3.66200000000003\\
1573	-7.32400000000007\\
1574	-15.8689999999999\\
1575	-21.973\\
1576	-23.193\\
1577	-17.0899999999999\\
1578	-19.5309999999999\\
1579	-30.518\\
1580	-30.518\\
1581	-26.855\\
1583	-21.973\\
1584	-23.193\\
1585	-15.8689999999999\\
1587	-10.9860000000001\\
1588	-7.32400000000007\\
1589	-14.6479999999999\\
1590	-17.0899999999999\\
1591	-12.2070000000001\\
1593	-14.6479999999999\\
1594	-10.9860000000001\\
1595	-13.4280000000001\\
1596	-14.6479999999999\\
1597	-18.3109999999999\\
1599	-10.9860000000001\\
1600	-15.8689999999999\\
1601	-10.9860000000001\\
1602	-2.44100000000003\\
1603	-7.32400000000007\\
1604	-8.54500000000007\\
1605	-7.32400000000007\\
1606	1.221\\
1607	-2.44100000000003\\
1608	-8.54500000000007\\
1609	-8.54500000000007\\
1610	-12.2070000000001\\
1611	-10.9860000000001\\
1612	-14.6479999999999\\
1613	-17.0899999999999\\
1614	-8.54500000000007\\
1615	-12.2070000000001\\
1616	-14.6479999999999\\
1617	-4.88300000000004\\
1618	-6.10400000000004\\
1619	-3.66200000000003\\
1620	-9.76600000000008\\
1621	-6.10400000000004\\
1622	-3.66200000000003\\
1623	-13.4280000000001\\
1624	-13.4280000000001\\
1625	-7.32400000000007\\
1626	-8.54500000000007\\
1627	-7.32400000000007\\
1631	-7.32400000000007\\
1632	-4.88300000000004\\
1633	-4.88300000000004\\
1634	-6.10400000000004\\
1635	-8.54500000000007\\
1636	-9.76600000000008\\
1637	-6.10400000000004\\
1640	-6.10400000000004\\
1641	-14.6479999999999\\
1642	-21.973\\
1643	-14.6479999999999\\
1644	-13.4280000000001\\
1645	-10.9860000000001\\
1646	-10.9860000000001\\
1647	-15.8689999999999\\
1648	-8.54500000000007\\
1649	-10.9860000000001\\
1650	-12.2070000000001\\
1651	-10.9860000000001\\
1652	-12.2070000000001\\
1653	-6.10400000000004\\
1654	-10.9860000000001\\
1655	-12.2070000000001\\
1656	-9.76600000000008\\
1657	-8.54500000000007\\
1659	-13.4280000000001\\
1660	-13.4280000000001\\
1661	-15.8689999999999\\
1662	-25.635\\
1663	-19.5309999999999\\
1664	-12.2070000000001\\
1665	-10.9860000000001\\
1666	-13.4280000000001\\
1667	-19.5309999999999\\
1668	-13.4280000000001\\
1669	-15.8689999999999\\
1670	-14.6479999999999\\
1671	-6.10400000000004\\
1672	-4.88300000000004\\
1673	-13.4280000000001\\
1674	-12.2070000000001\\
1675	-12.2070000000001\\
1676	-18.3109999999999\\
1677	-18.3109999999999\\
1678	-14.6479999999999\\
1680	-12.2070000000001\\
1681	-18.3109999999999\\
1682	-13.4280000000001\\
1683	-12.2070000000001\\
1684	-13.4280000000001\\
1685	-10.9860000000001\\
1687	-10.9860000000001\\
1688	-9.76600000000008\\
1689	-12.2070000000001\\
1690	-18.3109999999999\\
1692	-15.8689999999999\\
1693	-12.2070000000001\\
1694	-7.32400000000007\\
1695	-8.54500000000007\\
1697	-13.4280000000001\\
1698	-18.3109999999999\\
1699	-20.752\\
1700	-17.0899999999999\\
1701	-10.9860000000001\\
1702	-15.8689999999999\\
1703	-25.635\\
1704	-15.8689999999999\\
1705	-14.6479999999999\\
1706	-20.752\\
1707	-15.8689999999999\\
1708	-13.4280000000001\\
1709	-15.8689999999999\\
1711	-13.4280000000001\\
1712	-17.0899999999999\\
1713	-14.6479999999999\\
1714	-14.6479999999999\\
1715	-15.8689999999999\\
1716	-13.4280000000001\\
1717	-13.4280000000001\\
1718	-17.0899999999999\\
1719	-13.4280000000001\\
1720	-12.2070000000001\\
1721	-14.6479999999999\\
1722	-13.4280000000001\\
1723	-8.54500000000007\\
1724	-2.44100000000003\\
1725	-3.66200000000003\\
1726	-6.10400000000004\\
1727	-14.6479999999999\\
1728	-19.5309999999999\\
1729	-15.8689999999999\\
1730	-9.76600000000008\\
1731	-13.4280000000001\\
1732	-12.2070000000001\\
1733	-12.2070000000001\\
1734	-8.54500000000007\\
1735	-8.54500000000007\\
1736	-10.9860000000001\\
1737	-8.54500000000007\\
1738	-10.9860000000001\\
1739	-10.9860000000001\\
1740	-9.76600000000008\\
1741	-7.32400000000007\\
1742	-9.76600000000008\\
1743	-13.4280000000001\\
1744	-14.6479999999999\\
1745	-10.9860000000001\\
1746	-15.8689999999999\\
1747	-14.6479999999999\\
1748	-14.6479999999999\\
1749	-18.3109999999999\\
1750	-17.0899999999999\\
1751	-14.6479999999999\\
1752	-18.3109999999999\\
1753	-10.9860000000001\\
1754	-19.5309999999999\\
1755	-13.4280000000001\\
1756	-10.9860000000001\\
1757	-15.8689999999999\\
1758	-17.0899999999999\\
1759	-14.6479999999999\\
1760	-14.6479999999999\\
1761	-21.973\\
1762	-18.3109999999999\\
1764	-13.4280000000001\\
1765	-14.6479999999999\\
1766	-12.2070000000001\\
1768	-9.76600000000008\\
1769	-10.9860000000001\\
1770	-10.9860000000001\\
1771	-17.0899999999999\\
1772	-26.855\\
1773	-20.752\\
1774	-21.973\\
1775	-20.752\\
1776	-14.6479999999999\\
1777	-9.76600000000008\\
1778	-9.76600000000008\\
1779	-6.10400000000004\\
1780	-8.54500000000007\\
1781	-13.4280000000001\\
1783	-18.3109999999999\\
1784	-17.0899999999999\\
1785	-12.2070000000001\\
1786	-19.5309999999999\\
1787	-23.193\\
1788	-23.193\\
1789	-21.973\\
1790	-28.076\\
1791	-28.076\\
1792	-13.4280000000001\\
1793	-9.76600000000008\\
1794	-9.76600000000008\\
1795	-13.4280000000001\\
1796	-20.752\\
1797	-20.752\\
1798	-14.6479999999999\\
1799	-14.6479999999999\\
1800	-8.54500000000007\\
1801	-8.54500000000007\\
1803	-10.9860000000001\\
1804	-7.32400000000007\\
1805	-7.32400000000007\\
};
\addlegendentry{True output}

\addplot [color=mycolor2, dashed, line width=2.0pt]
  table[row sep=crcr]{%
1006	-21.3779612002118\\
1007	-22.3046674829764\\
1008	-19.3186969381456\\
1009	-13.5959107476563\\
1010	-13.5371175914393\\
1011	-15.6844200354617\\
1012	-16.1267090963283\\
1013	-14.6420167838705\\
1014	-11.1704728381135\\
1015	-12.5812677930414\\
1016	-9.31675392815373\\
1017	-4.71169086319287\\
1018	-19.5865712801367\\
1019	-18.2410257080669\\
1020	-18.1741582485433\\
1021	-14.4073489614873\\
1022	-13.0884658004711\\
1023	-16.206044638772\\
1024	-15.0804444604701\\
1025	-13.9167616926445\\
1026	-15.0342654641877\\
1027	-16.458816536554\\
1028	-12.9573512902537\\
1029	-17.5067780124834\\
1030	-19.8570995014913\\
1031	-16.1534411819332\\
1032	-19.2885852518982\\
1033	-21.8280067331568\\
1034	-17.8143847948854\\
1035	-15.6980856397788\\
1036	-14.4622968590174\\
1037	-13.7419132164271\\
1038	-16.9148020940959\\
1039	-16.1509196220711\\
1040	-15.478468839734\\
1041	-16.4697226656567\\
1042	-17.0886484237774\\
1043	-22.7209603059939\\
1044	-22.1800366969067\\
1045	-16.004793950136\\
1046	-15.0693954529886\\
1047	-12.6394183712582\\
1048	-11.728506774271\\
1049	-12.7824096134423\\
1050	-12.4007270672091\\
1051	-13.2260385704712\\
1052	-14.4248080664249\\
1053	-16.4459341967597\\
1054	-15.5233608945055\\
1055	-20.8182454584605\\
1056	-18.4950802084622\\
1057	-12.3497021659271\\
1058	-12.0248564027197\\
1059	-11.1868300281678\\
1060	-12.2193547986301\\
1061	-10.134813124602\\
1062	-9.51800238071678\\
1063	-10.164637532074\\
1064	-11.0391064681237\\
1065	-9.33495574674203\\
1066	-9.13719377059783\\
1067	-11.0059512905552\\
1068	-14.7807570513005\\
1069	-16.611585726471\\
1070	-17.5868810605423\\
1071	-17.0476795508712\\
1072	-19.1625068314768\\
1073	-17.288223990962\\
1074	-15.3410030023329\\
1075	-15.1301928789421\\
1076	-14.4174714353437\\
1077	-13.3558851389621\\
1078	-20.1740458399211\\
1079	-35.0884242276604\\
1080	-32.1955418478078\\
1081	-30.2030845567713\\
1082	-19.4156480477343\\
1083	-27.2769757600518\\
1084	-36.3300260177855\\
1085	-34.3741978458997\\
1086	-26.1179076275048\\
1087	-31.9584942446002\\
1088	-46.1632077168604\\
1089	-32.9020378238306\\
1090	-25.2031021648988\\
1091	-18.1709030061568\\
1092	-17.2984270111258\\
1093	-15.4404912835648\\
1094	-16.1804466519063\\
1095	-14.4365473169346\\
1096	-10.1782318117985\\
1097	-9.98680551273583\\
1098	-9.96801571018455\\
1099	-14.559940068382\\
1100	-14.6871073079058\\
1101	-14.2109528065723\\
1102	-16.6019374946416\\
1103	-17.9583698942838\\
1104	-25.7090254983161\\
1105	-20.555549762028\\
1106	-21.2678761517241\\
1107	-18.6528048545231\\
1108	-15.7031953056701\\
1109	-19.0898838522473\\
1110	-15.6965800375572\\
1111	-13.4569742200083\\
1112	-15.7050395502913\\
1113	-16.7137331106769\\
1114	-13.9308431828988\\
1115	-12.2154472752311\\
1116	-12.1410324527394\\
1117	-11.0882227797297\\
1118	-12.1120893407485\\
1119	-10.0685105438554\\
1120	-10.1411664056702\\
1121	-12.3686285522192\\
1122	-18.0990121205612\\
1123	-18.7307052362539\\
1124	-18.8598426991346\\
1125	-14.3290397147896\\
1126	-16.4456112682108\\
1127	-25.1694468104542\\
1128	-18.729489920838\\
1129	-19.9054804186162\\
1130	-21.1277166622517\\
1131	-16.2257034367128\\
1132	-13.1410292277612\\
1133	-13.5807352628581\\
1134	-15.8418059875569\\
1135	-23.3789914093336\\
1136	-30.4823849982686\\
1137	-26.3968192263278\\
1138	-21.8608611150212\\
1139	-21.4047519665587\\
1140	-21.6750350583509\\
1141	-19.781440595742\\
1142	-18.9874909621572\\
1143	-15.1108151808187\\
1144	-13.8165797016663\\
1145	-13.4784803823663\\
1146	-12.2822649351767\\
1147	-12.4080016462747\\
1148	-16.3688786448304\\
1149	-25.2006889678016\\
1150	-21.0024386519651\\
1151	-14.25525045262\\
1152	-13.9548390865664\\
1153	-13.6319574540598\\
1154	-10.9329812939891\\
1155	-10.8055200853046\\
1156	-8.49928076861897\\
1157	-10.0967464345888\\
1158	-11.1537602739145\\
1159	-10.2676990421146\\
1160	-10.7047199226693\\
1161	-10.5953370742968\\
1162	-9.80137761397646\\
1163	-9.94947103917139\\
1164	-9.13417871673596\\
1165	-5.59421789724206\\
1166	-13.2442400023413\\
1167	-24.2634404386977\\
1168	-23.805579924848\\
1169	-15.8658622401854\\
1170	-14.9914930113628\\
1171	-12.7957326128567\\
1172	-12.2370866247973\\
1173	-8.80544928098288\\
1174	-11.8696357804424\\
1175	-14.1680906829536\\
1176	-19.9917917029982\\
1177	-33.4828367464604\\
1178	-41.1259625968573\\
1179	-28.6013329407774\\
1180	-27.5359330735284\\
1181	-20.6803759928866\\
1182	-20.8398069294101\\
1183	-24.4288006509391\\
1184	-25.0723051774251\\
1185	-20.3699890408684\\
1186	-18.1057012551059\\
1187	-18.8042455535326\\
1188	-17.3221772567219\\
1189	-19.8029063393888\\
1190	-15.1392522996773\\
1191	-21.2810863116956\\
1192	-29.1872685907774\\
1193	-19.3430382910137\\
1194	-14.7541129438391\\
1195	-15.8530158609919\\
1196	-23.6424103395163\\
1197	-29.7247190146893\\
1198	-32.6124537863209\\
1199	-33.6853053698187\\
1200	-32.7536625493444\\
1201	-24.1712410180112\\
1202	-23.218902312251\\
1203	-26.668721582842\\
1204	-30.4058144513647\\
1205	-19.3881862281605\\
1206	-13.961221974856\\
1207	-17.7220815423598\\
1208	-18.1185935210253\\
1209	-12.6766182530398\\
1210	-13.8373040380268\\
1211	-16.9790943444743\\
1212	-13.9739155805739\\
1213	-11.7266919109488\\
1214	-13.5521082244568\\
1215	-12.8226929439491\\
1216	-13.896846879562\\
1217	-19.341117223833\\
1218	-17.3923568876753\\
1219	-14.2139400698729\\
1220	-13.9285527530212\\
1221	-19.4606549967784\\
1222	-33.4019423422778\\
1223	-23.4780144277431\\
1224	-13.2852516492771\\
1225	-13.9908042634456\\
1226	-13.9999692378165\\
1227	-14.6319469907789\\
1228	-11.5069896238292\\
1229	-11.5804578046605\\
1230	-12.1237846410459\\
1231	-15.2161173877296\\
1232	-16.5070900951678\\
1233	-12.1525365165764\\
1234	-16.346757150922\\
1235	-21.3102857378628\\
1236	-20.573843865594\\
1237	-16.6984686376359\\
1238	-16.514982834165\\
1239	-22.7230187635298\\
1240	-15.4853911982077\\
1241	-11.9486794453685\\
1242	-10.8685104074164\\
1243	-13.4222084237431\\
1244	-15.9192236822114\\
1245	-14.0808270148996\\
1246	-11.5844305207124\\
1247	-13.3613174048712\\
1248	-14.6648636717016\\
1249	-12.3558645229027\\
1250	-13.0625464903762\\
1251	-13.1643648605866\\
1252	-12.1538211243862\\
1253	-11.4415386955711\\
1254	-10.8523583484973\\
1255	-10.5534993836875\\
1256	-10.0664022072219\\
1257	-9.87536167719827\\
1258	-11.5178543236484\\
1259	-15.6325159979897\\
1260	-19.9327435793812\\
1261	-26.4789321791388\\
1262	-17.5867095265028\\
1263	-13.8703487563871\\
1264	-12.2210503437132\\
1265	-11.2363223937939\\
1266	-11.9571793553314\\
1267	-10.1566335400632\\
1268	-10.5382530702295\\
1269	-9.71569986487884\\
1270	-17.1388103178128\\
1271	-17.2520879568435\\
1272	-19.6820874352227\\
1273	-16.271913686573\\
1274	-15.6056187400841\\
1275	-12.9080506770042\\
1276	-12.1424037452248\\
1277	-12.0167589163329\\
1278	-9.10088851509295\\
1279	-6.5556456970794\\
1280	-10.3830366987988\\
1281	-11.624912067643\\
1282	-13.4085385963863\\
1283	-14.0728767060684\\
1284	-22.2533241592096\\
1285	-22.1484703419674\\
1286	-14.7666397435507\\
1287	-17.6445589495081\\
1288	-18.9319145665243\\
1289	-24.764406883116\\
1290	-19.6507977573924\\
1291	-14.7740587243939\\
1292	-13.0904480047577\\
1293	-12.4270107729581\\
1294	-9.14918922736615\\
1295	-7.67092272932177\\
1296	-8.37772388442977\\
1297	-8.59559686256785\\
1298	-7.9683897821792\\
1299	-12.1549827942179\\
1300	-13.0851888556854\\
1301	-12.4525996167331\\
1302	-14.2809026356529\\
1303	-11.5323476408341\\
1304	-12.1744610120713\\
1305	-7.60482825998361\\
1306	-14.5177862473954\\
1307	-16.090997559395\\
1308	-15.3559289448835\\
1309	-15.4912715241292\\
1310	-13.5699205546548\\
1311	-13.8364668065635\\
1312	-19.7713775312941\\
1313	-20.1604006064397\\
1314	-15.1365947989214\\
1315	-21.8346065700982\\
1316	-19.279850918472\\
1317	-17.1300949124704\\
1318	-20.1244604080894\\
1319	-19.3174643620348\\
1321	-14.712101762075\\
1322	-19.7280881353568\\
1323	-31.3735532881635\\
1324	-21.5316480264719\\
1325	-15.2274529520669\\
1326	-15.0054201572734\\
1327	-16.3157456756869\\
1328	-17.8736181158054\\
1329	-19.9677519430227\\
1330	-15.2467722537285\\
1331	-15.7708077287784\\
1332	-19.8632869772196\\
1333	-21.5811687879529\\
1334	-15.9692314379995\\
1335	-14.1530168102984\\
1336	-16.3073405258101\\
1337	-26.9386361608451\\
1338	-22.4894730020917\\
1339	-21.9126762870376\\
1340	-15.4568705147162\\
1341	-15.484140156273\\
1342	-15.6992666256183\\
1343	-11.8807668071242\\
1344	-11.9983284519947\\
1345	-10.1447603910285\\
1346	-8.67084914823818\\
1347	-5.28577764962051\\
1348	-12.2966594423419\\
1349	-15.9494352569309\\
1350	-12.8358689526349\\
1351	-12.6712098548087\\
1352	-14.1975901618471\\
1353	-12.3349112895403\\
1354	-11.580712815851\\
1355	-10.8631759963496\\
1356	-10.2414143112421\\
1357	-9.21578683127359\\
1358	-12.7435400217169\\
1359	-18.2909324397383\\
1360	-12.7918283960846\\
1361	-12.5230901237071\\
1362	-10.9439708025648\\
1363	-10.6967964121195\\
1364	-9.70622126436206\\
1365	-11.87759910258\\
1366	-24.4676338160982\\
1367	-17.8093537038501\\
1368	-22.7311979394033\\
1369	-27.4516497897723\\
1370	-26.0381561725562\\
1371	-20.1517793504379\\
1372	-15.8861050588403\\
1373	-16.4865209432678\\
1374	-16.8731120849952\\
1375	-18.4116306252715\\
1376	-20.5762689062699\\
1377	-21.1715491471825\\
1378	-25.2482431753351\\
1379	-30.7860073503975\\
1380	-35.5318912926928\\
1381	-24.2272195788862\\
1382	-25.3615519333384\\
1383	-31.3236319058301\\
1384	-24.2305693823882\\
1385	-25.1891639370679\\
1386	-23.5996168846218\\
1387	-13.321657947042\\
1388	-13.3247876471455\\
1389	-12.3254332504218\\
1390	-15.1795860606699\\
1391	-13.0194603484576\\
1392	-10.463171083951\\
1393	-10.9418882172038\\
1394	-10.5890749246903\\
1395	-10.2108609883071\\
1396	-8.9081871738988\\
1397	-9.80140854484193\\
1398	-9.91680902239045\\
1399	-10.4550591579075\\
1400	-17.0557107313957\\
1401	-13.4563232437858\\
1402	-17.4688448447992\\
1403	-29.6251287360537\\
1404	-25.4891138110681\\
1405	-30.5400427166483\\
1406	-22.4406804931182\\
1407	-23.5656911100762\\
1408	-18.6608247227227\\
1409	-14.5119201075004\\
1410	-14.4237109945263\\
1411	-12.1403948272123\\
1412	-11.5368382222937\\
1413	-10.9682742139455\\
1414	-9.84567675886456\\
1415	-12.2939109561885\\
1416	-6.75969490009606\\
1417	-9.0916614003454\\
1418	-15.6088952228033\\
1419	-17.3331326474374\\
1420	-17.0976500399649\\
1421	-16.9660464435124\\
1422	-18.0069078432152\\
1423	-16.6766698976114\\
1424	-14.7519266050883\\
1425	-15.1741681784035\\
1426	-13.2060690987421\\
1427	-16.4336372258326\\
1428	-12.9944060312948\\
1429	-12.9537256882036\\
1430	-14.395351183784\\
1431	-20.9904881486084\\
1432	-16.8339292633914\\
1433	-13.2017313100564\\
1434	-12.4522146934228\\
1435	-13.2711470029112\\
1436	-11.076357013789\\
1437	-10.6585730956069\\
1438	-10.6798003958354\\
1439	-13.3530622900753\\
1440	-15.5608977929051\\
1441	-13.2202709401492\\
1442	-14.5662710850638\\
1443	-14.8688790915737\\
1444	-12.019700321564\\
1445	-11.1616901203054\\
1446	-13.9024158092745\\
1447	-24.0689022506965\\
1448	-20.8440775426184\\
1449	-17.5433039816169\\
1450	-18.5332492680798\\
1451	-15.6336287958566\\
1452	-14.9581923341059\\
1453	-12.650324923512\\
1454	-14.2489129462097\\
1455	-14.2142458523056\\
1456	-11.6791599699743\\
1457	-12.4038652855675\\
1458	-14.8320017114656\\
1459	-18.0915024521644\\
1460	-17.6659730850697\\
1461	-18.222214399533\\
1462	-24.1108035663312\\
1463	-30.2587452950133\\
1464	-34.7089713816572\\
1465	-21.3788620819632\\
1466	-15.1992179199217\\
1467	-13.9982526865449\\
1468	-12.412016900598\\
1469	-9.67161292932542\\
1470	-10.2148479723255\\
1471	-13.3270052436899\\
1472	-13.1947434491478\\
1473	-12.0991810317794\\
1474	-14.6246064668458\\
1475	-16.9433825912406\\
1476	-19.4184273028179\\
1477	-21.32583388906\\
1478	-31.8922241816188\\
1479	-27.9757639265342\\
1480	-20.5228946894545\\
1481	-15.7369128430803\\
1482	-15.9294485250095\\
1483	-17.0512365294953\\
1484	-19.0243941303881\\
1485	-15.1569139782328\\
1486	-15.2756283479207\\
1487	-14.5476959401674\\
1488	-11.6452802622257\\
1489	-12.5827258943391\\
1490	-17.0263439607952\\
1491	-14.0218852198\\
1492	-14.8639751215997\\
1493	-23.8466714343999\\
1494	-20.3560311902691\\
1495	-18.2387534760915\\
1496	-23.8423991218715\\
1497	-23.8127048967995\\
1498	-16.9995634699553\\
1499	-13.0966693565908\\
1500	-12.5024168779917\\
1501	-16.5120954716779\\
1502	-30.2305746983295\\
1503	-27.8555276748771\\
1504	-24.9153909632682\\
1505	-22.8833027325663\\
1506	-15.3010050956216\\
1507	-15.4550984581469\\
1508	-11.8729110971972\\
1509	-11.8122207267761\\
1510	-9.86056310738695\\
1511	-12.1481321861886\\
1512	-12.7873428555563\\
1513	-10.4364014420228\\
1514	-11.3213300555308\\
1515	-15.3995612009012\\
1516	-17.4564746143415\\
1517	-22.4460767608784\\
1518	-26.1877249393149\\
1519	-35.8068820139038\\
1520	-23.6301655758987\\
1521	-18.8976426674401\\
1522	-21.3078375658658\\
1523	-18.6055845652684\\
1524	-13.9287879821638\\
1525	-16.1778616619599\\
1526	-25.6288174710162\\
1527	-28.8871373137281\\
1528	-15.8001659980289\\
1529	-13.7194077480001\\
1530	-12.0674356643931\\
1531	-13.3551025906045\\
1532	-9.12291265712906\\
1533	-4.49458797620923\\
1534	-8.19319669026891\\
1535	-11.7898660254432\\
1536	-10.1381909323648\\
1537	-10.6003900108835\\
1538	-9.54349247686037\\
1539	-9.26390358203457\\
1540	-9.65380146600819\\
1541	-8.76046112401377\\
1542	-10.9442578163801\\
1543	-18.5147488947462\\
1544	-19.8933646535309\\
1545	-17.702501685488\\
1546	-20.0396824186714\\
1547	-26.9302591472072\\
1548	-26.4085042677082\\
1549	-31.1031168827865\\
1550	-27.8733404241254\\
1551	-21.857629673387\\
1552	-17.0164064582088\\
1553	-17.2094370279051\\
1554	-24.5881905871859\\
1555	-28.7021218162265\\
1556	-18.045882417556\\
1557	-14.335725802961\\
1558	-12.3403934529326\\
1559	-13.2213251691014\\
1560	-8.49011259538838\\
1561	-9.48954746220784\\
1562	-5.54847093253034\\
1563	-7.92833447626708\\
1564	-11.619003528235\\
1565	-10.634758119427\\
1566	-9.96927322463421\\
1567	-10.4425110114134\\
1568	-7.00851062069614\\
1569	-8.09702147765029\\
1570	-9.08360407064288\\
1571	-8.79930063036613\\
1572	-8.42688964840386\\
1573	-6.18924609822488\\
1574	-20.0079384878136\\
1575	-26.1321133496112\\
1576	-23.9955426647539\\
1577	-19.2038046226264\\
1578	-23.8193533113083\\
1579	-36.5390628968219\\
1580	-30.5161900714063\\
1581	-30.2394002691401\\
1582	-22.6887765262795\\
1583	-24.0658472870468\\
1584	-25.2348312308968\\
1585	-16.4338161939486\\
1586	-17.4947355196834\\
1587	-11.6153283278561\\
1588	-12.7303657528269\\
1589	-16.2374055882633\\
1590	-12.267004564203\\
1591	-15.345088233049\\
1592	-15.8223790529578\\
1593	-14.7054312070786\\
1594	-15.8465875241518\\
1595	-15.0247534540338\\
1596	-16.2142650231322\\
1597	-17.0173520390949\\
1598	-16.9600118245191\\
1599	-14.1613545328303\\
1600	-14.7606469687778\\
1601	-12.3895910680828\\
1602	-16.3144851327788\\
1603	-7.20124676649084\\
1604	-8.4315213633854\\
1605	-8.75512519478184\\
1606	-13.8261814386981\\
1607	-2.85854588129632\\
1608	-4.55600573470883\\
1609	-9.48619624406047\\
1610	-11.0172554246797\\
1611	-11.7257936213932\\
1612	-15.4982547814864\\
1613	-15.9475593763459\\
1614	-13.9069399802395\\
1615	-13.908729025238\\
1616	-10.5785451157103\\
1617	-14.202239497944\\
1618	-9.89856281767925\\
1619	-10.5183303960532\\
1620	-5.82316274364871\\
1621	-8.154365869618\\
1622	-6.90568490287137\\
1623	-11.1005700171913\\
1624	-13.2803726804866\\
1625	-12.0820246997309\\
1626	-10.2070623893408\\
1627	-10.3471638098608\\
1628	-10.0731073011334\\
1629	-8.84249344038244\\
1630	-9.18861767355429\\
1631	-8.46181019799133\\
1632	-8.98198158165951\\
1633	-7.23889574962186\\
1634	-7.18844490912647\\
1635	-8.27989216711148\\
1636	-9.73779214013075\\
1637	-10.0041809911636\\
1638	-8.78632618023357\\
1639	-8.51035942412273\\
1640	-8.01961310402908\\
1641	-11.48372284635\\
1642	-22.7181903316916\\
1643	-15.6569554365035\\
1644	-15.0832239276692\\
1645	-13.4476917907723\\
1646	-15.4635893978757\\
1647	-14.8380637230732\\
1648	-12.9549645157042\\
1649	-11.8036536593115\\
1650	-11.2118678048973\\
1651	-13.7222848835183\\
1652	-10.605585047161\\
1653	-12.2761216833958\\
1654	-9.89969446186092\\
1655	-13.2431071598462\\
1656	-11.7362055674889\\
1657	-11.2318919346915\\
1658	-10.8735076668879\\
1659	-14.7377333772458\\
1660	-14.0758840116944\\
1661	-19.0508219407375\\
1662	-27.3546065307473\\
1663	-21.9746776175982\\
1664	-15.8165539680433\\
1665	-13.148360274253\\
1666	-16.2226610390132\\
1667	-20.1006344464465\\
1668	-15.5563388184491\\
1669	-18.0357002410622\\
1670	-12.3619641144664\\
1671	-14.5204325362745\\
1672	-10.5467585604629\\
1673	-12.0545805014156\\
1674	-15.7198093219743\\
1675	-12.2040250010127\\
1676	-17.8299607952333\\
1677	-17.7903209612437\\
1678	-16.800178986339\\
1679	-13.7901919077037\\
1680	-15.5006655151244\\
1681	-17.8642794766663\\
1682	-16.4517594100439\\
1683	-15.4116241336353\\
1684	-13.9195954975996\\
1685	-12.979707827313\\
1686	-12.7952853969239\\
1687	-11.4183535953287\\
1688	-11.9930802892347\\
1689	-15.2601814461427\\
1690	-17.1775579127782\\
1691	-18.8275048619785\\
1692	-16.8688757257175\\
1693	-14.394888028582\\
1694	-13.1222215514169\\
1695	-11.2091758318445\\
1696	-12.1679291792309\\
1697	-14.6338519238186\\
1698	-17.5736916503106\\
1699	-21.6882650714838\\
1700	-17.3676899342584\\
1701	-14.339864480103\\
1702	-16.9843629065942\\
1703	-27.167878366118\\
1704	-18.094838155889\\
1705	-18.2514734485358\\
1706	-19.5282234885906\\
1707	-18.0790989943519\\
1708	-15.1895763816285\\
1709	-16.1225941487014\\
1710	-15.9116451370753\\
1711	-16.9924863246165\\
1712	-17.9945166907289\\
1713	-16.5997470317175\\
1714	-17.1196008378788\\
1715	-15.9415799506705\\
1716	-16.5122462989307\\
1717	-14.0643753967674\\
1718	-15.9780717716901\\
1719	-14.9310503913935\\
1720	-14.5548534813267\\
1721	-15.5417313052076\\
1722	-15.6994721440342\\
1723	-11.8068027961433\\
1724	-13.2540408050334\\
1725	-10.6746622273074\\
1726	-5.243197265954\\
1727	-11.7804733542444\\
1728	-21.8815077252468\\
1729	-19.2957825117981\\
1730	-15.0781837936968\\
1731	-14.0375005480121\\
1732	-14.3962939843043\\
1733	-12.8034083835701\\
1734	-11.7391898346193\\
1735	-11.0363673161662\\
1736	-11.3104732847589\\
1737	-11.5224669510715\\
1738	-11.4803471719365\\
1739	-11.3337608647068\\
1740	-11.7096893645421\\
1741	-10.6429855791482\\
1742	-9.85126032164226\\
1743	-15.3581530775903\\
1744	-12.7149717316913\\
1745	-15.0660361238677\\
1746	-12.9979940347782\\
1747	-17.4967451796299\\
1748	-17.1835541577982\\
1749	-19.6717737760018\\
1750	-17.2788547467151\\
1751	-17.8232809093984\\
1752	-17.3308737031975\\
1753	-13.5858116733712\\
1754	-14.8098472261061\\
1755	-15.5307096335612\\
1756	-15.5482972926711\\
1757	-15.4657225766707\\
1758	-18.8107418338711\\
1759	-16.1093214004713\\
1760	-17.5826713027859\\
1761	-22.4791885629666\\
1762	-18.5968773675766\\
1763	-17.2590307981923\\
1764	-15.3699065667504\\
1765	-16.9534985930504\\
1766	-14.8521167183887\\
1767	-13.7871163112484\\
1768	-12.0227993921308\\
1769	-12.8481143170927\\
1770	-11.6879284602987\\
1771	-17.9943496897906\\
1772	-26.6201191564812\\
1773	-22.9146175719952\\
1774	-20.5068747780194\\
1775	-22.028154993824\\
1776	-15.5613295431185\\
1777	-15.3540605722164\\
1778	-12.5042193201293\\
1779	-11.7432168931459\\
1780	-9.89697527744079\\
1781	-12.7846756033687\\
1782	-17.7739441057456\\
1783	-18.8258442922076\\
1784	-17.4407102259247\\
1785	-14.3842979064743\\
1786	-20.2341257068563\\
1787	-26.3055220489014\\
1788	-24.998650393045\\
1789	-20.8514714460632\\
1790	-31.9811228446968\\
1791	-27.0506536936937\\
1792	-16.2024481712538\\
1793	-15.0004726376978\\
1794	-12.4162524261073\\
1795	-15.3739782438474\\
1796	-19.6562724394928\\
1797	-26.5960532156384\\
1798	-19.5254339316107\\
1799	-14.5212660905488\\
1800	-12.7500628571545\\
1801	-11.6798899352998\\
1802	-11.0765522077247\\
1803	-11.9566847083834\\
1804	-10.7485182938929\\
1805	-10.1603826450321\\
};
\addlegendentry{OSA predition}

\addplot [color=mycolor3, dotted, line width=2.0pt]
  table[row sep=crcr]{%
1006	-18.3109999999999\\
1007	-23.193\\
1008	-18.3109999999999\\
1009	-9.76600000000008\\
1010	-13.5371175914393\\
1011	-14.8247892207007\\
1012	-16.7998718633642\\
1013	-15.5155427822235\\
1014	-12.9984157496485\\
1015	-16.872336396437\\
1016	-16.0509823372786\\
1017	-11.9557184709174\\
1018	-26.397422103053\\
1019	-26.6176711966964\\
1020	-25.0269465607194\\
1021	-20.2629159154794\\
1022	-18.3204018143599\\
1023	-22.0765786894121\\
1024	-18.961752971237\\
1025	-17.3979428452315\\
1026	-19.1671488675499\\
1027	-20.2137324339456\\
1028	-17.6718875704769\\
1029	-22.3416825911409\\
1030	-22.8374940171411\\
1031	-19.0648054718499\\
1032	-23.0807011514996\\
1033	-24.1196518043491\\
1034	-20.4227513780552\\
1035	-18.6782477514228\\
1036	-17.4646319724618\\
1037	-17.9802738377546\\
1038	-20.1409513441574\\
1039	-19.9957174998522\\
1040	-19.2428750639099\\
1041	-19.7593399826333\\
1042	-20.6901135697713\\
1043	-26.3172468762818\\
1044	-25.5335648555995\\
1045	-19.2924819846837\\
1046	-18.6761864734333\\
1047	-15.9825692496436\\
1048	-15.9305311204964\\
1049	-16.1120010799168\\
1050	-15.4693245021458\\
1051	-15.5218788092059\\
1052	-17.1403026057906\\
1053	-18.9846219725453\\
1054	-18.2980558537772\\
1055	-24.1263407357462\\
1056	-21.2290420369909\\
1057	-16.4313127064681\\
1058	-16.1278214606614\\
1059	-16.2027314443851\\
1060	-17.5603655561392\\
1061	-14.8148998295792\\
1062	-14.8695257593083\\
1063	-14.4259977091315\\
1064	-14.3236396262444\\
1065	-14.0307867369856\\
1066	-13.8327831860518\\
1067	-14.6054389387436\\
1068	-18.9367314156223\\
1069	-19.6202635831983\\
1070	-20.8430565857684\\
1071	-19.8657056841894\\
1072	-22.0219958611187\\
1073	-20.0417098238627\\
1074	-19.0071784846596\\
1075	-18.3017287291023\\
1076	-18.1961842809444\\
1077	-17.8330704023786\\
1078	-24.4264189189273\\
1079	-38.1824645319173\\
1080	-37.8307782477773\\
1081	-34.935207263263\\
1082	-23.5474374414978\\
1083	-31.8697377835808\\
1084	-39.9885000902391\\
1085	-38.8739146957425\\
1086	-30.6996155736592\\
1087	-37.9936265510587\\
1088	-50.8795298561815\\
1089	-39.5556577750351\\
1090	-31.8909870529394\\
1091	-23.5328856447127\\
1092	-21.4713991290346\\
1093	-19.5085101970419\\
1094	-20.7040197985843\\
1095	-18.5096752710624\\
1096	-15.1589944569689\\
1097	-15.0179205524717\\
1098	-15.0424302454057\\
1099	-18.7760860645342\\
1100	-18.8612879008394\\
1101	-18.5326720546063\\
1102	-20.8107926292078\\
1103	-21.9146373463007\\
1104	-29.3054379781779\\
1105	-25.9040665542741\\
1106	-27.1906141833326\\
1107	-23.2899283640645\\
1108	-20.9907491420527\\
1109	-23.4911860896073\\
1110	-19.9115186352462\\
1111	-18.2791220565796\\
1112	-20.0845214428634\\
1113	-20.4902315162876\\
1114	-17.9497783726638\\
1115	-17.38621315709\\
1116	-16.0921178542681\\
1117	-15.8355999368712\\
1118	-16.6974779567479\\
1119	-14.4250038381517\\
1120	-14.863874946227\\
1121	-16.425373047304\\
1122	-21.9551108636199\\
1123	-22.4440287492862\\
1124	-22.4927775805238\\
1125	-17.8430132459739\\
1126	-20.7203172395737\\
1127	-30.5169778770144\\
1128	-23.9190841951734\\
1129	-25.6350128182655\\
1130	-25.6758743374473\\
1131	-19.628764922242\\
1132	-17.4331295227175\\
1133	-18.1404209966472\\
1134	-20.1342192393074\\
1135	-28.5178301250912\\
1136	-35.5303374578541\\
1137	-32.5492071397\\
1138	-26.714940981517\\
1140	-26.3317054340566\\
1141	-24.8267917817657\\
1142	-24.2266666724092\\
1143	-19.5433251508659\\
1144	-18.1967372303966\\
1145	-17.6805846628222\\
1146	-17.0303098450154\\
1147	-17.2320074761849\\
1148	-21.1471551354227\\
1149	-29.8754446965586\\
1150	-25.7349327154209\\
1151	-19.4148272953835\\
1152	-18.2815225992431\\
1153	-18.3995183253635\\
1154	-15.7570537517513\\
1155	-15.7378630017793\\
1156	-14.2118647704265\\
1157	-15.8584863193457\\
1158	-15.2539128408355\\
1159	-15.3907652802636\\
1160	-15.3098235892669\\
1161	-14.995371449673\\
1162	-14.2921775860027\\
1163	-14.3028362544303\\
1164	-14.3616701174697\\
1165	-11.776400186488\\
1166	-17.6642811113136\\
1167	-28.1843600496627\\
1168	-29.2818745220977\\
1169	-20.710667705474\\
1170	-18.743381051062\\
1171	-17.6051277849942\\
1172	-17.5209619063214\\
1173	-15.7902806161903\\
1174	-17.302367314797\\
1175	-20.7826509591134\\
1176	-24.9728439920743\\
1177	-39.3214460777699\\
1178	-47.6772054355217\\
1179	-37.1240560614715\\
1180	-35.9230780274763\\
1181	-28.4216637997795\\
1182	-28.5383778290663\\
1183	-30.0478983892351\\
1184	-31.057029463821\\
1185	-25.4638251643953\\
1186	-23.8663886085774\\
1187	-24.0422533779467\\
1188	-22.3742090483586\\
1189	-24.8503500636998\\
1190	-20.6425542712423\\
1191	-26.6575346746195\\
1192	-33.4177331042376\\
1193	-24.4379164991651\\
1194	-18.5531269851017\\
1195	-19.3150141278106\\
1196	-27.4115258162713\\
1197	-33.1754802590503\\
1198	-37.7229527772304\\
1199	-39.5690711090197\\
1200	-39.1930067303242\\
1201	-31.0823563263893\\
1202	-29.4158618061263\\
1203	-32.56867851181\\
1204	-36.0469317086399\\
1205	-24.8235082673305\\
1206	-18.6061221797777\\
1207	-22.3430799068465\\
1208	-21.372475869061\\
1209	-16.6323228098413\\
1210	-18.119067730215\\
1211	-20.5883083340009\\
1212	-18.1091599858082\\
1213	-16.1515713462486\\
1214	-17.3750096793192\\
1215	-17.068468688072\\
1216	-19.1221795836552\\
1217	-24.8167152297215\\
1218	-22.5392308202643\\
1219	-18.6912812816317\\
1220	-18.3770491617781\\
1221	-23.8688339046957\\
1222	-37.9313325463838\\
1223	-29.8604993654296\\
1224	-20.561254014058\\
1225	-19.1175363544908\\
1226	-19.7608987231761\\
1227	-19.790695185698\\
1228	-16.7725780999392\\
1229	-17.064802626338\\
1230	-17.7615274083864\\
1231	-20.6307224012216\\
1232	-21.9036860576487\\
1233	-17.1842334322316\\
1234	-20.6010329026267\\
1235	-25.3462892521159\\
1236	-24.2343636644282\\
1237	-20.9240639729878\\
1238	-21.5970699453462\\
1239	-27.3618681952325\\
1240	-20.2469089096792\\
1241	-16.4079819754238\\
1242	-16.012779449441\\
1243	-16.6982859268276\\
1244	-20.1375420419706\\
1245	-18.8703201906021\\
1246	-16.6864339568845\\
1247	-18.9286468786572\\
1248	-19.4739788874974\\
1249	-16.9140662483301\\
1250	-17.7953578432539\\
1251	-17.2116822341302\\
1252	-16.8556470031535\\
1253	-17.1191412852879\\
1254	-14.916818831307\\
1255	-14.9900760371572\\
1256	-14.0137262161093\\
1257	-13.9468206243962\\
1259	-19.838520437137\\
1260	-24.0378041110037\\
1261	-31.2120018533737\\
1262	-22.353363814379\\
1263	-18.1693312915791\\
1264	-17.1884196803755\\
1265	-16.6222794244966\\
1266	-17.439513822221\\
1267	-14.9937657348441\\
1268	-15.2148550520078\\
1269	-15.7119692992401\\
1270	-21.6460484212096\\
1271	-22.0380291832398\\
1272	-25.274748723514\\
1273	-20.0976232114451\\
1274	-19.1026424023296\\
1275	-16.1416116161442\\
1276	-16.1552688245072\\
1277	-15.4180641345292\\
1278	-14.4229026541695\\
1279	-12.8687690560039\\
1280	-16.4201553921191\\
1281	-17.3925702073584\\
1282	-18.6620193340536\\
1283	-18.7921882263965\\
1284	-26.6739869167486\\
1285	-26.5661277346696\\
1286	-20.2070546425359\\
1287	-22.0728537240204\\
1288	-23.0870855301785\\
1289	-28.4276160263471\\
1290	-24.0868641650397\\
1291	-19.1299167567745\\
1292	-16.741093296909\\
1293	-17.5123771935816\\
1294	-15.4408679558633\\
1295	-14.1213028638808\\
1297	-13.8240708766787\\
1298	-13.7078919095779\\
1299	-16.5835876505184\\
1300	-17.0513502034075\\
1301	-16.6579018918674\\
1302	-17.6465448350893\\
1303	-14.9798914078674\\
1304	-15.976385803986\\
1305	-13.3752915068901\\
1306	-18.4813258416632\\
1307	-19.6486877411978\\
1308	-19.4960535920427\\
1309	-18.2100661612135\\
1310	-16.5534189384368\\
1311	-17.9010255487422\\
1312	-22.7882084537928\\
1313	-23.6245028392007\\
1314	-18.6422888491504\\
1315	-25.3659575671641\\
1316	-22.6908989834324\\
1317	-21.3399992185243\\
1318	-23.6222548510111\\
1319	-23.0501203467065\\
1320	-20.0719357337716\\
1321	-18.4803837798827\\
1322	-23.3834615528763\\
1323	-35.3543356755072\\
1324	-26.6753314389105\\
1325	-18.9826662058169\\
1326	-18.732942126077\\
1327	-20.2170418154949\\
1328	-22.1403627031489\\
1329	-24.3778444794955\\
1330	-19.9701059732038\\
1331	-20.9028460223378\\
1332	-24.6790803625413\\
1333	-26.4891246995562\\
1334	-20.3078772870906\\
1335	-18.5661556799212\\
1336	-20.7286941448426\\
1337	-31.1538515448356\\
1338	-27.6345608430297\\
1339	-26.7583752078313\\
1340	-19.2490355587881\\
1341	-19.0789520518751\\
1342	-18.9832273457207\\
1343	-15.6628867067757\\
1344	-15.6687718855414\\
1345	-15.4768797405488\\
1346	-15.178491900632\\
1347	-12.5265835387511\\
1348	-18.1559260689478\\
1349	-21.2120348090014\\
1350	-17.5137979635026\\
1351	-17.3031614148715\\
1352	-18.1562886865195\\
1353	-15.9597490230083\\
1354	-15.5964719888866\\
1355	-15.0928311018329\\
1356	-14.412781782009\\
1357	-14.2703278491463\\
1358	-16.8443314343751\\
1359	-21.4714373593299\\
1360	-16.40495305731\\
1361	-15.7989950724866\\
1362	-14.8344947499577\\
1363	-14.8785080142075\\
1364	-13.7956848037495\\
1365	-16.5231507605683\\
1366	-28.9713457493451\\
1367	-23.0883340029973\\
1368	-26.9889025528555\\
1369	-33.0747360869441\\
1370	-31.8500441727581\\
1371	-25.9721341048355\\
1372	-21.4125667139472\\
1373	-21.5536714180741\\
1374	-21.8651188589597\\
1375	-23.4901196393473\\
1376	-25.4126803349779\\
1377	-25.7790662026791\\
1378	-29.9462014941814\\
1379	-34.7377348444543\\
1380	-40.665922678043\\
1381	-29.7394958951224\\
1382	-29.9203847682318\\
1383	-35.5871193278601\\
1384	-29.530491879832\\
1385	-30.2222291742523\\
1386	-28.1833024254995\\
1387	-18.180057642054\\
1388	-17.0474167004074\\
1389	-16.5859597542808\\
1390	-19.8336670679439\\
1391	-17.8926940361939\\
1392	-15.6544438197946\\
1393	-16.3844479936758\\
1394	-14.9595094143649\\
1395	-14.577279403906\\
1396	-14.0437022671638\\
1397	-14.3945433306733\\
1398	-13.72600846571\\
1399	-15.4094322470237\\
1400	-20.4251015960319\\
1401	-17.3295406972211\\
1402	-21.6919297192767\\
1403	-33.4128620312028\\
1404	-30.9827375798207\\
1405	-35.5090334857082\\
1406	-27.5926611888328\\
1407	-27.8206148388194\\
1408	-22.683020631076\\
1409	-18.3112540631821\\
1410	-19.1962581891705\\
1411	-17.2384183566908\\
1412	-16.0054182610268\\
1413	-16.5226611257419\\
1414	-14.1278603928445\\
1415	-15.5523508353017\\
1416	-13.2620258725547\\
1417	-14.2626759208154\\
1418	-20.4514608656832\\
1419	-23.2996643048964\\
1420	-22.5924337161259\\
1421	-21.4973931305017\\
1422	-23.0752259216185\\
1423	-21.1697843447985\\
1424	-19.0772115543057\\
1425	-19.3189809167518\\
1426	-17.6721195765165\\
1427	-21.0964898689588\\
1428	-17.4192170685988\\
1429	-16.461018907235\\
1430	-18.22847084923\\
1431	-24.6165802102282\\
1432	-20.822441139903\\
1433	-17.6860833519536\\
1434	-16.7902905850488\\
1435	-17.5094528491411\\
1436	-15.4013088730385\\
1437	-15.0491739281181\\
1438	-15.5415374950267\\
1440	-19.2163795916338\\
1441	-17.0569893434845\\
1442	-18.3949769917308\\
1443	-18.444977636568\\
1444	-15.4968704057242\\
1445	-15.2897128868283\\
1446	-18.7604755041216\\
1447	-27.6974732261992\\
1448	-26.139331654241\\
1449	-22.5690928059994\\
1450	-22.2696329130617\\
1451	-20.0242795482368\\
1452	-19.3455459987001\\
1453	-17.4843692601387\\
1454	-19.4594724661029\\
1455	-18.9063854731812\\
1456	-15.8441563884696\\
1457	-17.1172893646299\\
1458	-18.7162208351183\\
1459	-21.7012882048223\\
1460	-22.1172867962159\\
1461	-21.9329339418412\\
1462	-27.9666834658356\\
1463	-35.0569156467179\\
1464	-39.5664009805123\\
1465	-27.011577756527\\
1466	-19.8813864628587\\
1467	-18.1875803908927\\
1468	-17.8102753041887\\
1469	-16.2253733270586\\
1470	-15.8945518175101\\
1471	-19.8618839340859\\
1472	-19.5606494548188\\
1473	-17.5143603041015\\
1474	-19.3294422178265\\
1475	-21.531249610681\\
1476	-23.6685235701827\\
1477	-24.8816984635503\\
1478	-35.9546074447442\\
1479	-32.7578433557242\\
1480	-26.1386048502698\\
1481	-21.040030873196\\
1482	-20.8019847313801\\
1483	-21.8148277923174\\
1484	-23.9760226286187\\
1485	-19.8559570354823\\
1486	-19.3702306760495\\
1487	-19.1956358987352\\
1488	-15.6642816890276\\
1489	-17.1197432295899\\
1490	-22.9640443582466\\
1491	-18.7566897495569\\
1492	-18.8924094889194\\
1493	-28.6685578721185\\
1494	-24.32203656296\\
1495	-22.3687670033751\\
1496	-28.41378512087\\
1497	-28.3617798237863\\
1498	-20.982899995219\\
1499	-17.0765039335379\\
1500	-17.425561773967\\
1501	-21.6231799185546\\
1502	-35.4455668598744\\
1503	-35.2939909236875\\
1504	-31.6776861936844\\
1505	-28.4469558870862\\
1506	-20.823307673656\\
1507	-20.571873644137\\
1508	-17.352845761422\\
1509	-17.0629095753316\\
1510	-15.7620688446323\\
1511	-17.249843630332\\
1512	-18.0905769638478\\
1513	-15.4072975497645\\
1514	-16.0167305872985\\
1515	-20.1452316543043\\
1516	-21.3907455419553\\
1517	-26.1961595030702\\
1518	-31.382988248648\\
1519	-40.7476752453574\\
1520	-30.4853395725868\\
1521	-26.5806415493114\\
1522	-27.9109872778502\\
1523	-25.1087340290642\\
1524	-19.8091340816077\\
1525	-21.1620264161593\\
1526	-31.2793522343329\\
1527	-34.6838133931392\\
1528	-21.6731875061168\\
1529	-17.8041830668683\\
1530	-17.0848837587887\\
1531	-19.4647441709235\\
1532	-16.9684614140117\\
1533	-13.5044572154686\\
1534	-15.2943143389195\\
1535	-18.6696917788825\\
1536	-16.3999044095929\\
1537	-15.3680840398706\\
1538	-14.6442687958079\\
1539	-14.2948983434023\\
1540	-13.9180889380668\\
1541	-13.7523194071337\\
1542	-15.1169630579827\\
1543	-21.9569520203913\\
1544	-24.0511570281965\\
1545	-21.8933822247623\\
1546	-23.756112215005\\
1547	-30.74489580513\\
1548	-31.1331941921778\\
1549	-35.7146009196322\\
1550	-33.5339183028943\\
1551	-26.5347101847303\\
1552	-21.1867832291418\\
1553	-21.2166134318495\\
1554	-29.4438661216529\\
1555	-34.3120438349258\\
1556	-24.3086263815699\\
1557	-19.7647906341269\\
1558	-16.6724929055154\\
1559	-18.5285535466896\\
1560	-15.6217746393477\\
1561	-15.9546301857304\\
1562	-12.5200530713303\\
1563	-15.2905631358772\\
1564	-16.1233579942746\\
1565	-15.3540955013314\\
1566	-14.3854794252684\\
1567	-14.4676199946234\\
1568	-12.4421054380998\\
1569	-12.3124337353524\\
1570	-12.6687820736695\\
1571	-12.5263201316789\\
1572	-12.5917015871826\\
1573	-11.2294816268468\\
1574	-24.3256708852184\\
1575	-32.2087744381784\\
1576	-30.3216787174474\\
1577	-24.5608213625419\\
1578	-29.6999860555386\\
1579	-43.501214956096\\
1580	-38.5500264618749\\
1581	-37.2428243544055\\
1582	-30.057270165539\\
1583	-29.6904736339791\\
1584	-30.8184742532167\\
1585	-21.7544533293767\\
1586	-21.914974525062\\
1587	-16.6454907916636\\
1588	-17.0050323095022\\
1589	-22.1083674598469\\
1590	-17.6078522710293\\
1591	-18.0569720845701\\
1592	-19.7937058440098\\
1593	-18.5540893628149\\
1594	-18.7612356031659\\
1596	-20.5468103806688\\
1597	-21.2871411668514\\
1598	-20.2444428164329\\
1599	-17.7333885795395\\
1600	-18.8224734482658\\
1601	-14.9485515659073\\
1602	-18.756062079415\\
1603	-14.4646114325399\\
1604	-13.6476186750733\\
1605	-13.0101200206518\\
1606	-18.3236486557389\\
1607	-11.6768613946092\\
1608	-11.2019174611532\\
1609	-14.5274509571873\\
1610	-16.7079544647418\\
1611	-15.2468244801789\\
1612	-18.6795607832871\\
1613	-19.0773992082516\\
1614	-15.9397496874358\\
1615	-17.7411696568952\\
1616	-14.0385815316586\\
1617	-15.2795090989227\\
1618	-14.5716355103177\\
1619	-15.3764754472911\\
1620	-11.720280264034\\
1621	-11.7307824145601\\
1622	-11.2229972219063\\
1623	-15.9080173307052\\
1624	-15.9571554070967\\
1625	-14.6291897565197\\
1626	-14.1483053779359\\
1627	-13.7066323974832\\
1628	-14.0228891182339\\
1629	-13.1690690013195\\
1630	-13.136216798327\\
1631	-12.4056924354513\\
1632	-12.5418016050148\\
1633	-11.6764088612924\\
1634	-11.5056513326117\\
1635	-12.4006383544297\\
1636	-13.2517254924273\\
1637	-12.8721480026802\\
1638	-12.5224649929476\\
1639	-12.2580466490372\\
1640	-11.8805846226487\\
1641	-15.8563442684549\\
1642	-25.5051771503006\\
1643	-18.4881627583727\\
1644	-17.6030558218572\\
1645	-15.815273824835\\
1646	-18.5151268473091\\
1647	-19.0431154256703\\
1648	-15.8500771480781\\
1649	-16.0266111685903\\
1650	-14.9265249943896\\
1651	-16.4107978569534\\
1652	-14.0084926783181\\
1653	-14.2755643167056\\
1654	-13.8532331343708\\
1655	-15.9841551148738\\
1656	-14.4292089033088\\
1657	-14.4446883199614\\
1658	-14.2163916441482\\
1659	-17.5722314958923\\
1660	-17.1300020353385\\
1661	-21.9549769823955\\
1662	-31.0707581170223\\
1663	-25.6661195595527\\
1664	-19.7227938293461\\
1665	-17.6151857849616\\
1666	-20.7680176842941\\
1667	-25.207465348004\\
1668	-20.0402591829659\\
1669	-22.7535798268266\\
1670	-16.8832912463511\\
1671	-17.0066147327859\\
1672	-15.7315052821716\\
1673	-18.3252877111597\\
1674	-20.4466408620585\\
1675	-18.1711168584495\\
1676	-22.7577090815737\\
1677	-21.6353080546871\\
1678	-20.0488952245487\\
1679	-17.1174400061097\\
1680	-18.2810905404124\\
1681	-21.5263373593186\\
1682	-19.3156336906159\\
1683	-18.9992991590307\\
1684	-18.0513804254897\\
1685	-16.2914779624577\\
1686	-16.4891348258318\\
1687	-15.0988660664038\\
1688	-15.0630842724465\\
1689	-18.9133559387344\\
1690	-21.3732503435531\\
1691	-21.8971464203394\\
1692	-20.3195168062109\\
1693	-17.5501294545281\\
1694	-16.2498747916377\\
1695	-15.9617279073238\\
1696	-16.9479820088457\\
1697	-19.3395334275874\\
1698	-22.3717097411463\\
1699	-25.5129085781452\\
1700	-20.9000466968052\\
1701	-17.2110924259296\\
1702	-20.6737123573305\\
1703	-30.7634497397535\\
1704	-21.733833905844\\
1705	-22.2370212991564\\
1706	-24.1343306430572\\
1707	-21.3978683611942\\
1708	-18.9050432582624\\
1709	-19.9162276996765\\
1710	-18.94988452907\\
1711	-20.234567396735\\
1712	-22.0668007870959\\
1713	-20.1577134649388\\
1714	-20.9797521839064\\
1715	-20.1569763585776\\
1716	-19.957990194496\\
1717	-18.1316574014731\\
1718	-19.6041923210018\\
1719	-17.4586942292665\\
1720	-17.4334428711302\\
1721	-18.6928955819208\\
1722	-18.5109092273897\\
1723	-14.9136117222699\\
1724	-16.841808318298\\
1725	-17.2308305519327\\
1726	-12.7955454695127\\
1727	-18.4878176380557\\
1728	-27.8727012053751\\
1729	-25.2354690438297\\
1730	-20.6074410719439\\
1731	-20.3856648428457\\
1732	-19.9851873532248\\
1733	-18.3751412469173\\
1734	-16.4624414403916\\
1735	-15.9878480673476\\
1736	-16.3353404978413\\
1737	-15.7107238188717\\
1738	-16.3083813120663\\
1739	-15.3575382958256\\
1740	-15.1159243718328\\
1741	-14.1854715228146\\
1742	-13.9009594194886\\
1743	-18.8281106400766\\
1744	-16.5028549973531\\
1745	-17.5485464223639\\
1746	-16.5697173456313\\
1747	-19.2940642971832\\
1748	-19.9569912297495\\
1749	-22.9224814322417\\
1750	-20.2391195654891\\
1751	-20.614147438909\\
1752	-20.9467400943722\\
1753	-15.9641557810021\\
1754	-17.8682376040642\\
1755	-16.337176908389\\
1756	-17.1869597757532\\
1757	-18.4053660477587\\
1758	-20.7089629954551\\
1759	-18.7348638875574\\
1760	-20.4404567038027\\
1761	-25.8816525802285\\
1762	-21.6018069983918\\
1763	-19.9752955569948\\
1764	-18.1902237598988\\
1765	-19.943007447024\\
1766	-18.1060226273637\\
1767	-17.4649647149545\\
1768	-16.0258835339669\\
1769	-17.0392807148132\\
1770	-15.8263172727097\\
1771	-22.0282519928498\\
1772	-30.6505574593148\\
1773	-26.2425001212282\\
1774	-24.1099187585814\\
1775	-24.4371610932267\\
1776	-17.9344349017483\\
1777	-17.7133556943377\\
1778	-16.2830513426941\\
1779	-15.712617504222\\
1780	-15.1857370753912\\
1781	-17.9177227766493\\
1782	-22.2108731104802\\
1783	-23.6116452040176\\
1784	-21.3729199167892\\
1785	-17.5937640679724\\
1786	-24.0702268029791\\
1787	-29.8040071952294\\
1788	-29.2635292932137\\
1789	-25.0263398118539\\
1790	-35.3839081745223\\
1791	-31.620491991953\\
1792	-19.2968828203173\\
1793	-18.2863323045653\\
1794	-17.193075264245\\
1795	-20.1402601193336\\
1796	-24.7819739555089\\
1797	-30.9886072665447\\
1798	-25.4479638273519\\
1799	-20.8591850199607\\
1800	-17.3522174701443\\
1801	-17.2934756158877\\
1802	-16.9353486224384\\
1803	-17.1114090686581\\
1804	-15.4406026673448\\
1805	-15.3521481183593\\
};
\addlegendentry{MPO prediction}

\end{axis}

\begin{axis}[%
width=6.159cm,
height=1.831cm,
at={(8.104cm,7.627cm)},
scale only axis,
xmin=1000,
xmax=2000,
xlabel style={font=\color{white!15!black}},
xlabel={Sample index},
ymin=-40.7725560622456,
ymax=1.221,
ylabel style={font=\color{white!15!black}},
ylabel={$y(t)$},
axis background/.style={fill=white},
title style={font=\bfseries},
title={C4: RMSE(OSA) = 2.9177, RMSE(MPO) = 5.9764},
legend style={legend cell align=left, align=left, draw=white!15!black}
]
\addplot [color=mycolor1, line width=2.0pt]
  table[row sep=crcr]{%
1006	-14.6479999999999\\
1007	-17.0899999999999\\
1008	-10.9860000000001\\
1009	-7.32400000000007\\
1010	-14.6479999999999\\
1011	-8.54500000000007\\
1012	-9.76600000000008\\
1013	-6.10400000000004\\
1014	-3.66200000000003\\
1016	-3.66200000000003\\
1017	-2.44100000000003\\
1018	-13.4280000000001\\
1020	-13.4280000000001\\
1021	-9.76600000000008\\
1022	-7.32400000000007\\
1023	-10.9860000000001\\
1024	-12.2070000000001\\
1025	-6.10400000000004\\
1027	-13.4280000000001\\
1028	-7.32400000000007\\
1029	-15.8689999999999\\
1031	-10.9860000000001\\
1032	-17.0899999999999\\
1033	-13.4280000000001\\
1035	-8.54500000000007\\
1036	-8.54500000000007\\
1037	-10.9860000000001\\
1038	-12.2070000000001\\
1040	-9.76600000000008\\
1041	-10.9860000000001\\
1042	-10.9860000000001\\
1043	-15.8689999999999\\
1044	-14.6479999999999\\
1045	-10.9860000000001\\
1046	-10.9860000000001\\
1047	-6.10400000000004\\
1048	-4.88300000000004\\
1049	-8.54500000000007\\
1050	-7.32400000000007\\
1051	-7.32400000000007\\
1052	-10.9860000000001\\
1053	-9.76600000000008\\
1054	-9.76600000000008\\
1055	-15.8689999999999\\
1056	-8.54500000000007\\
1057	-6.10400000000004\\
1060	-9.76600000000008\\
1061	-4.88300000000004\\
1062	-7.32400000000007\\
1064	-7.32400000000007\\
1065	-6.10400000000004\\
1069	-10.9860000000001\\
1070	-10.9860000000001\\
1072	-13.4280000000001\\
1073	-10.9860000000001\\
1074	-12.2070000000001\\
1075	-9.76600000000008\\
1077	-9.76600000000008\\
1078	-17.0899999999999\\
1079	-20.752\\
1080	-19.5309999999999\\
1081	-19.5309999999999\\
1082	-15.8689999999999\\
1083	-20.752\\
1084	-23.193\\
1085	-23.193\\
1086	-14.6479999999999\\
1087	-23.193\\
1088	-26.855\\
1089	-21.973\\
1090	-15.8689999999999\\
1091	-14.6479999999999\\
1092	-10.9860000000001\\
1093	-8.54500000000007\\
1094	-10.9860000000001\\
1096	-6.10400000000004\\
1097	-6.10400000000004\\
1098	-8.54500000000007\\
1100	-10.9860000000001\\
1101	-9.76600000000008\\
1102	-12.2070000000001\\
1103	-10.9860000000001\\
1104	-15.8689999999999\\
1105	-13.4280000000001\\
1106	-18.3109999999999\\
1107	-12.2070000000001\\
1108	-10.9860000000001\\
1109	-12.2070000000001\\
1110	-9.76600000000008\\
1111	-8.54500000000007\\
1112	-10.9860000000001\\
1113	-10.9860000000001\\
1114	-8.54500000000007\\
1115	-8.54500000000007\\
1116	-7.32400000000007\\
1118	-7.32400000000007\\
1119	-4.88300000000004\\
1120	-6.10400000000004\\
1121	-8.54500000000007\\
1122	-13.4280000000001\\
1123	-12.2070000000001\\
1124	-13.4280000000001\\
1125	-8.54500000000007\\
1126	-15.8689999999999\\
1127	-17.0899999999999\\
1128	-13.4280000000001\\
1129	-14.6479999999999\\
1130	-14.6479999999999\\
1132	-7.32400000000007\\
1133	-8.54500000000007\\
1134	-8.54500000000007\\
1135	-17.0899999999999\\
1136	-20.752\\
1137	-19.5309999999999\\
1138	-14.6479999999999\\
1139	-15.8689999999999\\
1140	-13.4280000000001\\
1141	-12.2070000000001\\
1142	-12.2070000000001\\
1143	-9.76600000000008\\
1144	-8.54500000000007\\
1145	-8.54500000000007\\
1146	-9.76600000000008\\
1147	-8.54500000000007\\
1148	-13.4280000000001\\
1149	-15.8689999999999\\
1150	-10.9860000000001\\
1151	-8.54500000000007\\
1152	-9.76600000000008\\
1153	-8.54500000000007\\
1155	-3.66200000000003\\
1156	-4.88300000000004\\
1157	-8.54500000000007\\
1158	-9.76600000000008\\
1159	-4.88300000000004\\
1160	-7.32400000000007\\
1161	-7.32400000000007\\
1162	-3.66200000000003\\
1163	-3.66200000000003\\
1164	-2.44100000000003\\
1165	-4.88300000000004\\
1166	-10.9860000000001\\
1167	-12.2070000000001\\
1168	-14.6479999999999\\
1169	-13.4280000000001\\
1170	-8.54500000000007\\
1171	-7.32400000000007\\
1172	-4.88300000000004\\
1173	-7.32400000000007\\
1174	-6.10400000000004\\
1175	-10.9860000000001\\
1176	-12.2070000000001\\
1177	-21.973\\
1178	-23.193\\
1179	-21.973\\
1180	-18.3109999999999\\
1181	-13.4280000000001\\
1182	-17.0899999999999\\
1183	-15.8689999999999\\
1184	-19.5309999999999\\
1185	-12.2070000000001\\
1186	-13.4280000000001\\
1187	-12.2070000000001\\
1188	-13.4280000000001\\
1189	-13.4280000000001\\
1190	-10.9860000000001\\
1191	-18.3109999999999\\
1192	-17.0899999999999\\
1193	-13.4280000000001\\
1194	-7.32400000000007\\
1195	-12.2070000000001\\
1196	-15.8689999999999\\
1197	-17.0899999999999\\
1199	-21.973\\
1200	-20.752\\
1201	-15.8689999999999\\
1202	-14.6479999999999\\
1203	-18.3109999999999\\
1204	-19.5309999999999\\
1205	-13.4280000000001\\
1206	-9.76600000000008\\
1207	-13.4280000000001\\
1208	-9.76600000000008\\
1209	-7.32400000000007\\
1210	-9.76600000000008\\
1211	-10.9860000000001\\
1212	-7.32400000000007\\
1213	-7.32400000000007\\
1214	-9.76600000000008\\
1215	-6.10400000000004\\
1216	-10.9860000000001\\
1217	-13.4280000000001\\
1218	-13.4280000000001\\
1219	-8.54500000000007\\
1220	-9.76600000000008\\
1221	-12.2070000000001\\
1222	-18.3109999999999\\
1223	-15.8689999999999\\
1224	-9.76600000000008\\
1225	-8.54500000000007\\
1226	-9.76600000000008\\
1227	-7.32400000000007\\
1228	-8.54500000000007\\
1229	-7.32400000000007\\
1230	-8.54500000000007\\
1231	-10.9860000000001\\
1232	-10.9860000000001\\
1233	-6.10400000000004\\
1234	-10.9860000000001\\
1235	-13.4280000000001\\
1236	-13.4280000000001\\
1237	-9.76600000000008\\
1238	-12.2070000000001\\
1239	-13.4280000000001\\
1240	-10.9860000000001\\
1241	-4.88300000000004\\
1242	-10.9860000000001\\
1243	-8.54500000000007\\
1244	-8.54500000000007\\
1245	-6.10400000000004\\
1246	-6.10400000000004\\
1247	-10.9860000000001\\
1248	-10.9860000000001\\
1249	-6.10400000000004\\
1250	-8.54500000000007\\
1251	-8.54500000000007\\
1252	-4.88300000000004\\
1253	-8.54500000000007\\
1254	-4.88300000000004\\
1255	-7.32400000000007\\
1256	-4.88300000000004\\
1257	-4.88300000000004\\
1258	-10.9860000000001\\
1259	-12.2070000000001\\
1260	-12.2070000000001\\
1261	-15.8689999999999\\
1263	-6.10400000000004\\
1264	-7.32400000000007\\
1265	-4.88300000000004\\
1266	-7.32400000000007\\
1267	-6.10400000000004\\
1268	-3.66200000000003\\
1269	-8.54500000000007\\
1270	-9.76600000000008\\
1271	-9.76600000000008\\
1272	-17.0899999999999\\
1273	-10.9860000000001\\
1274	-13.4280000000001\\
1275	-7.32400000000007\\
1276	-8.54500000000007\\
1277	-3.66200000000003\\
1278	-1.221\\
1279	-6.10400000000004\\
1280	-9.76600000000008\\
1281	-8.54500000000007\\
1282	-9.76600000000008\\
1283	-9.76600000000008\\
1284	-17.0899999999999\\
1285	-10.9860000000001\\
1287	-10.9860000000001\\
1289	-15.8689999999999\\
1290	-15.8689999999999\\
1292	-6.10400000000004\\
1293	-3.66200000000003\\
1295	-6.10400000000004\\
1296	-4.88300000000004\\
1297	-6.10400000000004\\
1298	-6.10400000000004\\
1299	-9.76600000000008\\
1300	-9.76600000000008\\
1301	-8.54500000000007\\
1302	-9.76600000000008\\
1303	-6.10400000000004\\
1304	-3.66200000000003\\
1306	-10.9860000000001\\
1307	-8.54500000000007\\
1308	-12.2070000000001\\
1309	-8.54500000000007\\
1310	-7.32400000000007\\
1311	-8.54500000000007\\
1312	-12.2070000000001\\
1313	-14.6479999999999\\
1314	-10.9860000000001\\
1315	-17.0899999999999\\
1316	-9.76600000000008\\
1317	-12.2070000000001\\
1318	-13.4280000000001\\
1319	-13.4280000000001\\
1320	-9.76600000000008\\
1321	-9.76600000000008\\
1322	-12.2070000000001\\
1323	-18.3109999999999\\
1324	-15.8689999999999\\
1325	-8.54500000000007\\
1326	-12.2070000000001\\
1327	-9.76600000000008\\
1328	-12.2070000000001\\
1329	-13.4280000000001\\
1330	-9.76600000000008\\
1331	-9.76600000000008\\
1332	-14.6479999999999\\
1334	-12.2070000000001\\
1335	-8.54500000000007\\
1336	-9.76600000000008\\
1337	-15.8689999999999\\
1338	-14.6479999999999\\
1339	-15.8689999999999\\
1340	-12.2070000000001\\
1343	-8.54500000000007\\
1344	-4.88300000000004\\
1345	-3.66200000000003\\
1346	-3.66200000000003\\
1348	-10.9860000000001\\
1350	-8.54500000000007\\
1351	-10.9860000000001\\
1353	-8.54500000000007\\
1354	-3.66200000000003\\
1355	-6.10400000000004\\
1357	-6.10400000000004\\
1359	-10.9860000000001\\
1360	-7.32400000000007\\
1361	-6.10400000000004\\
1362	-7.32400000000007\\
1363	-7.32400000000007\\
1364	-6.10400000000004\\
1365	-8.54500000000007\\
1366	-15.8689999999999\\
1367	-14.6479999999999\\
1368	-17.0899999999999\\
1370	-17.0899999999999\\
1371	-13.4280000000001\\
1372	-10.9860000000001\\
1373	-9.76600000000008\\
1375	-12.2070000000001\\
1376	-14.6479999999999\\
1377	-14.6479999999999\\
1380	-21.973\\
1381	-18.3109999999999\\
1382	-21.973\\
1383	-19.5309999999999\\
1384	-14.6479999999999\\
1385	-17.0899999999999\\
1386	-14.6479999999999\\
1387	-10.9860000000001\\
1388	-8.54500000000007\\
1389	-8.54500000000007\\
1390	-10.9860000000001\\
1391	-7.32400000000007\\
1394	-7.32400000000007\\
1395	-3.66200000000003\\
1396	-7.32400000000007\\
1397	-7.32400000000007\\
1398	-6.10400000000004\\
1399	-8.54500000000007\\
1400	-12.2070000000001\\
1401	-8.54500000000007\\
1402	-12.2070000000001\\
1403	-18.3109999999999\\
1404	-17.0899999999999\\
1405	-19.5309999999999\\
1407	-14.6479999999999\\
1408	-14.6479999999999\\
1409	-7.32400000000007\\
1410	-9.76600000000008\\
1411	-8.54500000000007\\
1412	-6.10400000000004\\
1414	-8.54500000000007\\
1415	-2.44100000000003\\
1416	-4.88300000000004\\
1417	-8.54500000000007\\
1418	-9.76600000000008\\
1419	-12.2070000000001\\
1420	-12.2070000000001\\
1421	-10.9860000000001\\
1422	-12.2070000000001\\
1423	-14.6479999999999\\
1424	-10.9860000000001\\
1425	-9.76600000000008\\
1426	-7.32400000000007\\
1427	-10.9860000000001\\
1428	-10.9860000000001\\
1429	-7.32400000000007\\
1430	-10.9860000000001\\
1431	-13.4280000000001\\
1433	-8.54500000000007\\
1435	-8.54500000000007\\
1436	-6.10400000000004\\
1437	-7.32400000000007\\
1438	-6.10400000000004\\
1439	-9.76600000000008\\
1440	-8.54500000000007\\
1441	-10.9860000000001\\
1442	-10.9860000000001\\
1444	-6.10400000000004\\
1445	-6.10400000000004\\
1446	-10.9860000000001\\
1447	-14.6479999999999\\
1448	-14.6479999999999\\
1449	-12.2070000000001\\
1452	-8.54500000000007\\
1453	-8.54500000000007\\
1454	-9.76600000000008\\
1455	-12.2070000000001\\
1456	-7.32400000000007\\
1457	-7.32400000000007\\
1458	-10.9860000000001\\
1459	-10.9860000000001\\
1460	-13.4280000000001\\
1461	-13.4280000000001\\
1462	-14.6479999999999\\
1463	-20.752\\
1464	-21.973\\
1465	-14.6479999999999\\
1466	-9.76600000000008\\
1467	-7.32400000000007\\
1468	-6.10400000000004\\
1469	-7.32400000000007\\
1470	-4.88300000000004\\
1471	-7.32400000000007\\
1473	-7.32400000000007\\
1474	-10.9860000000001\\
1476	-13.4280000000001\\
1477	-13.4280000000001\\
1478	-18.3109999999999\\
1479	-17.0899999999999\\
1480	-12.2070000000001\\
1481	-12.2070000000001\\
1482	-9.76600000000008\\
1483	-9.76600000000008\\
1484	-13.4280000000001\\
1485	-9.76600000000008\\
1486	-8.54500000000007\\
1487	-10.9860000000001\\
1488	-6.10400000000004\\
1489	-10.9860000000001\\
1490	-14.6479999999999\\
1491	-8.54500000000007\\
1492	-8.54500000000007\\
1493	-15.8689999999999\\
1495	-10.9860000000001\\
1496	-17.0899999999999\\
1497	-17.0899999999999\\
1498	-10.9860000000001\\
1499	-8.54500000000007\\
1500	-9.76600000000008\\
1501	-12.2070000000001\\
1502	-18.3109999999999\\
1503	-19.5309999999999\\
1504	-18.3109999999999\\
1505	-15.8689999999999\\
1506	-9.76600000000008\\
1507	-9.76600000000008\\
1508	-7.32400000000007\\
1509	-6.10400000000004\\
1510	-6.10400000000004\\
1511	-7.32400000000007\\
1512	-9.76600000000008\\
1513	-8.54500000000007\\
1514	-6.10400000000004\\
1515	-12.2070000000001\\
1516	-10.9860000000001\\
1517	-13.4280000000001\\
1518	-17.0899999999999\\
1519	-19.5309999999999\\
1521	-12.2070000000001\\
1522	-15.8689999999999\\
1523	-12.2070000000001\\
1524	-9.76600000000008\\
1525	-10.9860000000001\\
1526	-15.8689999999999\\
1527	-18.3109999999999\\
1528	-12.2070000000001\\
1530	-4.88300000000004\\
1532	0\\
1533	-6.10400000000004\\
1534	-8.54500000000007\\
1535	-8.54500000000007\\
1537	-6.10400000000004\\
1538	-6.10400000000004\\
1539	-7.32400000000007\\
1540	-4.88300000000004\\
1541	-6.10400000000004\\
1544	-13.4280000000001\\
1545	-10.9860000000001\\
1546	-12.2070000000001\\
1547	-17.0899999999999\\
1548	-17.0899999999999\\
1549	-19.5309999999999\\
1550	-20.752\\
1551	-13.4280000000001\\
1552	-10.9860000000001\\
1553	-10.9860000000001\\
1554	-14.6479999999999\\
1555	-17.0899999999999\\
1557	-9.76600000000008\\
1558	-7.32400000000007\\
1559	-3.66200000000003\\
1561	-6.10400000000004\\
1562	-1.221\\
1563	-9.76600000000008\\
1564	-7.32400000000007\\
1567	-3.66200000000003\\
1568	-6.10400000000004\\
1569	-6.10400000000004\\
1571	-3.66200000000003\\
1572	-3.66200000000003\\
1573	-4.88300000000004\\
1574	-12.2070000000001\\
1575	-13.4280000000001\\
1576	-15.8689999999999\\
1577	-14.6479999999999\\
1578	-14.6479999999999\\
1579	-23.193\\
1581	-18.3109999999999\\
1582	-18.3109999999999\\
1583	-15.8689999999999\\
1584	-15.8689999999999\\
1585	-14.6479999999999\\
1586	-10.9860000000001\\
1587	-8.54500000000007\\
1588	-8.54500000000007\\
1589	-12.2070000000001\\
1590	-14.6479999999999\\
1591	-8.54500000000007\\
1592	-9.76600000000008\\
1594	-9.76600000000008\\
1595	-10.9860000000001\\
1596	-10.9860000000001\\
1597	-12.2070000000001\\
1599	-9.76600000000008\\
1600	-9.76600000000008\\
1601	-10.9860000000001\\
1602	-2.44100000000003\\
1603	-4.88300000000004\\
1604	-8.54500000000007\\
1605	-7.32400000000007\\
1606	1.221\\
1607	-2.44100000000003\\
1608	-7.32400000000007\\
1609	-7.32400000000007\\
1610	-8.54500000000007\\
1612	-8.54500000000007\\
1613	-10.9860000000001\\
1614	-7.32400000000007\\
1615	-8.54500000000007\\
1616	-10.9860000000001\\
1617	-4.88300000000004\\
1618	-4.88300000000004\\
1619	-3.66200000000003\\
1620	-3.66200000000003\\
1621	-2.44100000000003\\
1622	-3.66200000000003\\
1623	-10.9860000000001\\
1624	-12.2070000000001\\
1625	-8.54500000000007\\
1626	-8.54500000000007\\
1628	-6.10400000000004\\
1629	-8.54500000000007\\
1630	-6.10400000000004\\
1632	-6.10400000000004\\
1633	-3.66200000000003\\
1634	-4.88300000000004\\
1635	-7.32400000000007\\
1636	-7.32400000000007\\
1637	-3.66200000000003\\
1639	-6.10400000000004\\
1640	-6.10400000000004\\
1642	-13.4280000000001\\
1643	-6.10400000000004\\
1644	-7.32400000000007\\
1645	-9.76600000000008\\
1646	-8.54500000000007\\
1647	-12.2070000000001\\
1648	-7.32400000000007\\
1649	-7.32400000000007\\
1650	-10.9860000000001\\
1651	-7.32400000000007\\
1652	-9.76600000000008\\
1653	-6.10400000000004\\
1655	-8.54500000000007\\
1656	-8.54500000000007\\
1657	-7.32400000000007\\
1658	-8.54500000000007\\
1659	-8.54500000000007\\
1660	-10.9860000000001\\
1661	-10.9860000000001\\
1662	-18.3109999999999\\
1663	-13.4280000000001\\
1664	-10.9860000000001\\
1665	-9.76600000000008\\
1666	-9.76600000000008\\
1667	-13.4280000000001\\
1668	-9.76600000000008\\
1669	-10.9860000000001\\
1670	-14.6479999999999\\
1671	-4.88300000000004\\
1672	-3.66200000000003\\
1673	-10.9860000000001\\
1674	-7.32400000000007\\
1675	-9.76600000000008\\
1677	-12.2070000000001\\
1680	-8.54500000000007\\
1681	-12.2070000000001\\
1682	-8.54500000000007\\
1683	-10.9860000000001\\
1684	-10.9860000000001\\
1685	-7.32400000000007\\
1686	-8.54500000000007\\
1687	-4.88300000000004\\
1688	-6.10400000000004\\
1689	-10.9860000000001\\
1690	-13.4280000000001\\
1691	-10.9860000000001\\
1692	-13.4280000000001\\
1693	-9.76600000000008\\
1694	-7.32400000000007\\
1696	-7.32400000000007\\
1698	-12.2070000000001\\
1699	-13.4280000000001\\
1700	-12.2070000000001\\
1701	-7.32400000000007\\
1702	-8.54500000000007\\
1703	-19.5309999999999\\
1704	-12.2070000000001\\
1705	-13.4280000000001\\
1706	-15.8689999999999\\
1707	-10.9860000000001\\
1708	-9.76600000000008\\
1710	-9.76600000000008\\
1711	-8.54500000000007\\
1712	-13.4280000000001\\
1713	-10.9860000000001\\
1714	-9.76600000000008\\
1715	-10.9860000000001\\
1716	-10.9860000000001\\
1717	-8.54500000000007\\
1718	-9.76600000000008\\
1719	-4.88300000000004\\
1720	-9.76600000000008\\
1721	-12.2070000000001\\
1723	-7.32400000000007\\
1724	-3.66200000000003\\
1725	-2.44100000000003\\
1726	-3.66200000000003\\
1727	-9.76600000000008\\
1728	-13.4280000000001\\
1729	-10.9860000000001\\
1730	-10.9860000000001\\
1731	-13.4280000000001\\
1732	-12.2070000000001\\
1733	-7.32400000000007\\
1734	-6.10400000000004\\
1735	-3.66200000000003\\
1736	-8.54500000000007\\
1737	-4.88300000000004\\
1738	-8.54500000000007\\
1739	-7.32400000000007\\
1740	-7.32400000000007\\
1742	-4.88300000000004\\
1743	-9.76600000000008\\
1744	-10.9860000000001\\
1745	-6.10400000000004\\
1746	-10.9860000000001\\
1747	-8.54500000000007\\
1748	-9.76600000000008\\
1749	-14.6479999999999\\
1750	-8.54500000000007\\
1751	-12.2070000000001\\
1752	-12.2070000000001\\
1753	-3.66200000000003\\
1754	-12.2070000000001\\
1755	-13.4280000000001\\
1756	-9.76600000000008\\
1757	-15.8689999999999\\
1758	-13.4280000000001\\
1759	-13.4280000000001\\
1760	-10.9860000000001\\
1761	-15.8689999999999\\
1762	-13.4280000000001\\
1764	-10.9860000000001\\
1765	-8.54500000000007\\
1766	-9.76600000000008\\
1767	-8.54500000000007\\
1768	-6.10400000000004\\
1769	-8.54500000000007\\
1770	-7.32400000000007\\
1771	-10.9860000000001\\
1772	-15.8689999999999\\
1773	-13.4280000000001\\
1774	-14.6479999999999\\
1775	-14.6479999999999\\
1777	-7.32400000000007\\
1778	-7.32400000000007\\
1779	-4.88300000000004\\
1780	-6.10400000000004\\
1781	-9.76600000000008\\
1782	-12.2070000000001\\
1784	-12.2070000000001\\
1785	-8.54500000000007\\
1786	-9.76600000000008\\
1787	-17.0899999999999\\
1788	-14.6479999999999\\
1789	-13.4280000000001\\
1790	-17.0899999999999\\
1791	-23.193\\
1792	-7.32400000000007\\
1793	-7.32400000000007\\
1794	-8.54500000000007\\
1795	-8.54500000000007\\
1796	-12.2070000000001\\
1797	-17.0899999999999\\
1798	-14.6479999999999\\
1799	-10.9860000000001\\
1800	-9.76600000000008\\
1801	-6.10400000000004\\
1802	-6.10400000000004\\
1803	-8.54500000000007\\
1805	-3.66200000000003\\
};
\addlegendentry{True output}

\addplot [color=mycolor2, dashed, line width=2.0pt]
  table[row sep=crcr]{%
1006	-16.3165927036027\\
1007	-16.5210027330381\\
1008	-14.9450419802497\\
1009	-9.48396082256772\\
1010	-10.0316139263655\\
1011	-12.8525164531904\\
1012	-12.4405732540645\\
1013	-10.5929132477559\\
1014	-9.19980593080527\\
1015	-11.1870082720559\\
1016	-8.78619373448146\\
1017	-2.78591565695069\\
1018	-15.9460718653306\\
1019	-15.1744387797601\\
1020	-13.5701809914849\\
1021	-11.5304567570495\\
1022	-10.5464995004595\\
1023	-11.7717596579823\\
1024	-10.4479539085953\\
1025	-10.8251619796467\\
1026	-10.8918524381406\\
1027	-12.2280971196485\\
1028	-11.7017118265503\\
1029	-13.1484473338849\\
1030	-15.3490010063927\\
1031	-12.1143434027515\\
1032	-14.8972590062635\\
1033	-16.930953634135\\
1034	-13.2566802722909\\
1035	-11.3276626579197\\
1036	-10.5097026352626\\
1037	-10.4043533609436\\
1038	-12.7369036170132\\
1039	-12.3711793166226\\
1040	-12.5376564158719\\
1041	-12.2870773814504\\
1042	-12.6841168096216\\
1043	-16.69642307711\\
1044	-15.7986997907799\\
1045	-11.5879169160839\\
1046	-11.5042484947794\\
1047	-10.8268609610543\\
1048	-9.2716017171249\\
1049	-8.18307474169001\\
1050	-8.86643390819722\\
1051	-8.41117836460739\\
1052	-9.42484855486737\\
1053	-12.2508420267936\\
1054	-10.9993280083754\\
1055	-15.2299958598865\\
1056	-14.0461405942744\\
1057	-9.27118360705231\\
1058	-9.00415803576379\\
1059	-8.79724949635215\\
1060	-9.73487842508598\\
1061	-9.08168216065451\\
1062	-8.2773521587128\\
1063	-7.77796314292527\\
1064	-8.39278600229159\\
1065	-8.38167902732926\\
1066	-7.41414289823842\\
1067	-8.55245937199811\\
1068	-12.9271691129563\\
1069	-12.1369061550645\\
1070	-13.6570019754008\\
1071	-12.678229962281\\
1072	-14.1148778193549\\
1073	-12.5633632582762\\
1074	-12.0126503529943\\
1075	-12.165694586625\\
1076	-11.06842619638\\
1077	-11.1586184764042\\
1078	-15.6609129679039\\
1079	-26.9696908117091\\
1080	-25.7579468552735\\
1081	-21.3361613128225\\
1082	-13.252929411276\\
1084	-28.1369150329085\\
1085	-24.8678686902783\\
1086	-20.4705665670097\\
1087	-22.2360078739748\\
1088	-35.7180036908064\\
1089	-19.6322510725183\\
1090	-18.7166045270542\\
1091	-11.7529468661075\\
1092	-11.8966200450957\\
1093	-11.4834163563487\\
1094	-12.7313813432224\\
1095	-10.3128639723182\\
1096	-8.97456493005552\\
1097	-8.33729872580761\\
1098	-8.12026606974655\\
1099	-11.3323109746491\\
1100	-11.2911886559486\\
1101	-11.1232356283072\\
1102	-13.5228324818729\\
1103	-13.8165060498927\\
1104	-18.2474971041554\\
1105	-15.0436857998682\\
1106	-17.1794594077198\\
1107	-14.2700453621785\\
1108	-12.8592455989219\\
1109	-14.009066281938\\
1110	-11.5794539464785\\
1111	-10.7941150461022\\
1112	-11.726289314561\\
1113	-12.4848403962762\\
1114	-9.85539074320832\\
1115	-10.1155796998387\\
1116	-9.20119684870383\\
1117	-9.04249944062303\\
1118	-9.08715381391812\\
1119	-8.39280890592522\\
1120	-7.55615618356137\\
1121	-8.47152715676339\\
1122	-13.1785498379429\\
1123	-15.30587787168\\
1124	-14.6372500567304\\
1125	-10.2870371752413\\
1126	-11.6860626901509\\
1127	-22.0245749735489\\
1128	-15.5760784081776\\
1129	-15.2678920925403\\
1130	-17.9824583840689\\
1131	-10.6764099121715\\
1132	-11.1609523287068\\
1133	-10.5023399787822\\
1134	-12.2676515238427\\
1135	-17.2496075963961\\
1136	-24.6484500529295\\
1137	-21.4936060909467\\
1138	-14.1939426969445\\
1139	-16.3797804011017\\
1140	-16.535668134115\\
1141	-14.8564598808587\\
1142	-14.0970132123105\\
1143	-10.574076330834\\
1144	-10.3142939211634\\
1145	-9.93761524435399\\
1146	-9.74028996705215\\
1147	-10.591101138361\\
1148	-12.6143273507432\\
1149	-20.3800836101607\\
1150	-15.0797762803072\\
1151	-10.365032218481\\
1152	-10.1512773987779\\
1153	-10.6886072776804\\
1154	-9.1232946765283\\
1155	-9.68376084740657\\
1156	-6.61968862391996\\
1157	-7.1254106972865\\
1159	-9.03713595576119\\
1160	-7.77532731712677\\
1161	-8.68427175583179\\
1162	-8.55074209394593\\
1163	-7.72728449269357\\
1164	-8.19156519040598\\
1165	-3.53320352325522\\
1166	-8.46542382891971\\
1167	-19.0280954417503\\
1168	-17.4405000455615\\
1169	-11.7550257895275\\
1170	-11.7278888287278\\
1171	-10.5885597758045\\
1172	-10.5481581348056\\
1173	-6.87365455588542\\
1174	-9.88581812193479\\
1175	-10.7409116405217\\
1176	-14.5927808404711\\
1177	-25.8463193619723\\
1178	-33.4594872403436\\
1179	-21.6086634511478\\
1180	-21.3419431174489\\
1181	-13.3301339672278\\
1182	-16.1781316601075\\
1183	-19.5440879965581\\
1184	-17.6535885371632\\
1185	-15.45029502487\\
1186	-13.6704302096496\\
1187	-15.0407429464012\\
1188	-12.8074800624224\\
1189	-14.9977430492916\\
1190	-12.671669913565\\
1191	-17.4644497236857\\
1192	-24.0964124217251\\
1193	-13.2914831171383\\
1194	-10.55830444971\\
1195	-10.7589490573175\\
1196	-18.5209031031989\\
1197	-21.8804849191567\\
1198	-25.402653218782\\
1199	-24.1798043619449\\
1200	-23.817859533327\\
1201	-16.7936103387526\\
1202	-16.1932211978128\\
1203	-19.123874245066\\
1204	-22.4546276284461\\
1205	-13.8493853283335\\
1206	-9.51215660148614\\
1207	-13.5365255175318\\
1208	-13.5103677897375\\
1209	-9.21161812332639\\
1210	-9.9543765015062\\
1211	-12.6227866054223\\
1212	-10.2088601664341\\
1213	-9.06097952823984\\
1214	-9.92502818236812\\
1215	-10.1477729379183\\
1216	-10.2603961274147\\
1217	-15.9404624714568\\
1218	-14.2027508279305\\
1219	-10.8180391077669\\
1220	-10.6618920188814\\
1221	-14.8914527196337\\
1222	-28.0270035700532\\
1223	-17.1474699600644\\
1224	-9.92369484181836\\
1225	-10.4534730239031\\
1226	-11.0303596873816\\
1227	-11.2389727628283\\
1228	-8.59263670631071\\
1229	-9.34758946414286\\
1230	-9.9723091008882\\
1231	-11.7361914731973\\
1232	-12.8145242538562\\
1233	-10.3512008058892\\
1234	-11.6777272476277\\
1235	-16.3492597105369\\
1236	-13.942800180437\\
1237	-12.1043340660028\\
1238	-12.8685413984429\\
1239	-16.6778274400706\\
1240	-11.1672103625845\\
1241	-10.1973978732626\\
1242	-8.22872626547746\\
1243	-10.4292454929341\\
1244	-11.9482979351494\\
1245	-10.4543041247332\\
1246	-8.42592865329493\\
1247	-9.44415150216946\\
1248	-11.3725793650613\\
1249	-9.88133973886033\\
1250	-9.38553578635219\\
1251	-9.8968080599434\\
1252	-9.64686167323748\\
1253	-8.5691768567965\\
1254	-8.41896274725741\\
1255	-7.58274295847923\\
1256	-7.85674905699625\\
1257	-7.36112379458268\\
1258	-8.07093135327932\\
1259	-12.2313324768686\\
1260	-15.6886565420216\\
1261	-20.6623235116267\\
1262	-12.6757097918053\\
1263	-10.3721864233385\\
1264	-9.31271405359485\\
1265	-8.98644981765983\\
1266	-8.48676395502321\\
1267	-7.66533743228956\\
1268	-7.88843267202742\\
1269	-7.06746809062361\\
1270	-14.3796629702153\\
1271	-12.5552832663295\\
1272	-14.9999101810424\\
1273	-12.2280761852976\\
1274	-11.4232495969227\\
1275	-11.2776214833609\\
1276	-10.5515958543208\\
1277	-10.0591569239452\\
1278	-7.7182474770907\\
1279	-4.73091915581722\\
1280	-7.5593497223424\\
1281	-9.93041553677722\\
1282	-10.1917683634285\\
1283	-11.2469351829927\\
1284	-17.4843171015768\\
1285	-17.066094436658\\
1286	-10.975404549808\\
1287	-13.408129635221\\
1288	-14.0312356613213\\
1289	-16.9942159668587\\
1290	-14.4986485078014\\
1291	-12.3056304754341\\
1292	-12.127381745579\\
1293	-11.8899617876871\\
1294	-7.54504569504843\\
1295	-6.28849349887355\\
1296	-6.59809478949319\\
1297	-6.49936769279839\\
1298	-6.77542621897737\\
1299	-9.17161574435954\\
1300	-10.1071953812097\\
1301	-10.0989457238745\\
1302	-10.5014049178053\\
1303	-9.49255708038845\\
1304	-10.8045124737966\\
1305	-4.80446412274136\\
1306	-11.3563375549686\\
1307	-11.5350142588295\\
1308	-11.0116742035423\\
1309	-11.8519310196491\\
1310	-9.60561692780448\\
1311	-9.64017655377279\\
1312	-15.3392666061698\\
1313	-13.7712751335634\\
1314	-11.7115084452205\\
1315	-15.5429238408633\\
1316	-15.824562736703\\
1317	-12.3828628286187\\
1318	-14.4675961145633\\
1319	-14.8332448449457\\
1320	-12.0744809918344\\
1321	-11.1563681003038\\
1322	-14.8291728319705\\
1323	-22.5737308165722\\
1324	-15.841855116515\\
1325	-10.8695074218861\\
1326	-10.8031290687459\\
1327	-12.4791614005965\\
1328	-13.8642170483679\\
1329	-15.0378809529288\\
1330	-12.0066882013102\\
1331	-12.0862633395282\\
1332	-14.3373823140878\\
1333	-16.6971520856162\\
1334	-10.5256708744341\\
1335	-11.486636963818\\
1336	-13.1849542789219\\
1337	-19.598595599417\\
1338	-17.1293928630155\\
1339	-16.1286699666155\\
1340	-10.5627318634138\\
1341	-12.0360292595469\\
1342	-12.2736932698663\\
1343	-9.84450926888371\\
1344	-10.9086990064989\\
1345	-9.85294430925364\\
1346	-7.88438832642669\\
1347	-3.98079313183302\\
1348	-10.6087045292645\\
1349	-12.9791906412386\\
1350	-9.70997410831774\\
1351	-9.93051452071336\\
1352	-11.6812260069485\\
1353	-9.55169565140068\\
1354	-9.45023563885843\\
1355	-7.54774291099943\\
1356	-7.89390739669261\\
1357	-7.1610705424248\\
1358	-9.34983431964747\\
1359	-13.6413116755466\\
1360	-9.1979561409089\\
1361	-9.07937230717494\\
1362	-8.35420142122848\\
1363	-8.14782367725525\\
1364	-8.31398312747797\\
1365	-8.40890380856513\\
1366	-20.0877905562122\\
1367	-15.0057421915456\\
1368	-17.4803301625336\\
1369	-21.9802480044439\\
1370	-20.5858504764817\\
1371	-14.5165840597742\\
1372	-12.084811963982\\
1373	-12.288025235305\\
1374	-12.4016721546575\\
1375	-13.846099292582\\
1376	-15.3538886383601\\
1377	-15.9047435598916\\
1378	-21.4717804963416\\
1379	-23.9158383487079\\
1380	-27.7691248920364\\
1381	-16.7084191671959\\
1382	-18.3998355669612\\
1383	-24.2330431738028\\
1384	-15.8508295210124\\
1385	-19.3369128180229\\
1386	-17.3220429964092\\
1387	-9.89381354016996\\
1388	-10.2593934980312\\
1389	-10.0893230172135\\
1390	-11.8162837053005\\
1391	-11.0695556472508\\
1392	-8.72072699054252\\
1393	-9.09868921169641\\
1394	-8.65751349782545\\
1395	-8.60280375584239\\
1396	-6.70824518337736\\
1397	-8.00381867279589\\
1398	-8.1428890772404\\
1399	-8.4122227718558\\
1400	-12.829905962612\\
1401	-10.7866091046312\\
1402	-14.5612882178816\\
1403	-23.8397115242963\\
1404	-19.9341416759162\\
1405	-23.0837915242619\\
1406	-15.2082522713299\\
1407	-17.5632428689755\\
1408	-12.1523138434877\\
1409	-11.8338798967484\\
1410	-10.8742686906637\\
1411	-10.5773501239714\\
1412	-9.51013280225266\\
1413	-8.78082988383653\\
1414	-7.92167304291843\\
1415	-11.1985619971133\\
1416	-5.69297766527734\\
1417	-7.32020235572622\\
1418	-12.9294228713188\\
1419	-13.9142509709009\\
1420	-13.2546190230573\\
1421	-12.8947382245951\\
1422	-14.2161195564888\\
1423	-12.6458648025186\\
1424	-12.234677060055\\
1425	-11.9725247928961\\
1426	-10.9770027542884\\
1427	-12.6842682719353\\
1428	-9.85420212444023\\
1429	-10.4600983566404\\
1430	-10.2987989197293\\
1431	-16.3934414789308\\
1432	-12.2819894797742\\
1433	-10.5797187957389\\
1434	-9.95676951609562\\
1435	-10.3295529966165\\
1436	-9.14054185048531\\
1437	-8.404154577084\\
1438	-8.57888824573365\\
1439	-9.90240808174667\\
1440	-11.4090576338353\\
1441	-9.2702505869745\\
1442	-11.206035766892\\
1443	-11.7628618430701\\
1444	-9.19298465708926\\
1445	-8.51820519110402\\
1446	-10.7488112721637\\
1447	-18.681380594662\\
1448	-15.9257920639802\\
1449	-13.3321947244199\\
1450	-13.0116363934555\\
1451	-11.7627611343705\\
1452	-10.9058826791359\\
1453	-9.89047171114817\\
1454	-11.1296997341717\\
1455	-11.3620105156303\\
1456	-10.0329057525307\\
1457	-9.89772535498355\\
1458	-11.919819257356\\
1459	-13.024203516346\\
1460	-13.1238746769602\\
1461	-14.1620148930665\\
1462	-19.0471885613435\\
1463	-23.6874450971379\\
1464	-27.0706611098112\\
1465	-15.8033331638424\\
1466	-9.82112797708191\\
1467	-11.4991935278333\\
1468	-11.370236597653\\
1469	-7.67776389080177\\
1470	-8.39350356786349\\
1471	-9.91817355852845\\
1472	-9.9007448670277\\
1473	-8.69509050687316\\
1474	-9.93537877271478\\
1475	-12.8254603969592\\
1476	-14.5731721071988\\
1477	-15.7223861292853\\
1478	-23.392057451113\\
1479	-19.8876575433389\\
1480	-13.7526353712085\\
1481	-11.0702571458908\\
1482	-12.3253592287954\\
1483	-12.5948010310096\\
1484	-13.855210793457\\
1485	-11.2349605286322\\
1486	-10.8693728833011\\
1487	-10.8129765242934\\
1488	-9.70052321085086\\
1489	-8.50586112693554\\
1490	-15.1727628079061\\
1491	-12.0293319257182\\
1492	-11.1924865240494\\
1493	-19.1524444736601\\
1494	-15.5773213012667\\
1495	-13.0411833502601\\
1496	-17.0147802678321\\
1497	-18.0841586499796\\
1498	-12.8802947534091\\
1499	-10.2592407393122\\
1500	-10.4942316300612\\
1501	-14.1541256198975\\
1502	-23.8211191287496\\
1503	-23.5125033869738\\
1504	-20.0092520871626\\
1505	-15.6874704294341\\
1506	-12.0622375600133\\
1507	-11.4258760171701\\
1508	-10.2406024230538\\
1509	-9.53667592000534\\
1510	-8.17186483265891\\
1511	-8.51524494924274\\
1512	-9.76769597308999\\
1513	-8.64722286848678\\
1514	-9.18850262434603\\
1515	-11.3541136146632\\
1516	-14.5583195524669\\
1517	-15.7158899222241\\
1519	-25.3975147238668\\
1520	-16.2931885327594\\
1521	-14.1502803475023\\
1522	-15.2614462815618\\
1523	-14.7900407967638\\
1524	-10.7307632644838\\
1525	-11.9241001111957\\
1526	-19.3331351419763\\
1527	-22.4051847831404\\
1528	-11.5065275633081\\
1529	-10.3551668190164\\
1530	-11.3657924486538\\
1531	-11.8106590596694\\
1532	-7.71605625311622\\
1533	-2.65314005092887\\
1534	-6.63217109126026\\
1535	-8.91564119274722\\
1536	-8.79980742452608\\
1537	-8.73960437909318\\
1538	-7.87169732130747\\
1539	-7.61582247827732\\
1540	-7.94406999526495\\
1541	-7.00201516816151\\
1542	-8.13348987031668\\
1543	-13.906264355235\\
1544	-14.2782904086462\\
1545	-13.117650701696\\
1546	-15.0992987098427\\
1547	-19.3629378975775\\
1548	-18.795405431883\\
1549	-22.3827033246394\\
1550	-20.286724032756\\
1551	-15.7424394243417\\
1552	-12.2013009158286\\
1553	-12.4141928275321\\
1554	-20.1868831026406\\
1555	-20.8137767532407\\
1556	-12.2045943278599\\
1557	-10.9061067158545\\
1558	-10.8764518730695\\
1559	-12.5091840253201\\
1560	-7.40875596682918\\
1561	-8.9992175899622\\
1562	-4.13398509869171\\
1563	-6.2424203934363\\
1564	-9.04939558006959\\
1565	-8.32511371752048\\
1566	-8.29750341934687\\
1567	-8.5787263841064\\
1568	-5.11050228280078\\
1569	-6.22486148141843\\
1570	-6.76467259856054\\
1571	-6.75679376020412\\
1572	-6.58607912386537\\
1573	-4.51748541911184\\
1574	-16.792491707605\\
1575	-21.5814987662525\\
1576	-18.1713275762552\\
1577	-12.8412177558052\\
1578	-17.6232066907719\\
1579	-29.4344523658256\\
1580	-24.9896841392979\\
1581	-20.6794582031168\\
1582	-15.0338555599105\\
1583	-17.9008535764403\\
1584	-18.7053668449651\\
1585	-12.3410006956667\\
1586	-13.2629552900244\\
1587	-10.3249093403865\\
1588	-10.07339613115\\
1589	-13.1006067210901\\
1590	-10.8348184370391\\
1591	-12.7264447461657\\
1593	-11.6380908518536\\
1594	-11.6711875106496\\
1595	-11.15268450641\\
1596	-12.8336295841148\\
1597	-13.5707821708111\\
1598	-11.896330611874\\
1599	-10.7821446249116\\
1600	-11.6510825254456\\
1601	-9.21869607136045\\
1602	-15.3041605298997\\
1603	-6.80922708625189\\
1604	-5.61623378760373\\
1605	-8.53528474923223\\
1606	-14.0124866424296\\
1607	-1.39857938075579\\
1608	-3.26262531069119\\
1609	-7.38649445712394\\
1610	-8.34539752687238\\
1611	-8.83861227582634\\
1612	-11.667716072194\\
1613	-10.9985535219987\\
1614	-9.98844718624559\\
1615	-9.80492673939739\\
1616	-8.48382672816865\\
1617	-11.4752779646399\\
1618	-9.66689319103011\\
1619	-9.76074839200714\\
1620	-3.43280974703725\\
1621	-5.30058043064969\\
1622	-3.98929663724084\\
1623	-8.03612248834111\\
1624	-10.0070024606352\\
1625	-10.0357318054462\\
1626	-9.29198685775464\\
1627	-9.54075888934858\\
1628	-8.89282901294177\\
1629	-8.05713698363797\\
1630	-8.78464963010015\\
1631	-7.67798678422082\\
1632	-8.06072839929152\\
1633	-6.48747766979272\\
1634	-6.2345728758437\\
1635	-5.96454857488948\\
1636	-7.95528194121425\\
1637	-7.95270728935816\\
1638	-6.54366774000732\\
1639	-7.47761745933099\\
1640	-6.55949488232318\\
1641	-8.60863283478716\\
1642	-17.3385463559912\\
1643	-11.9992693346358\\
1644	-9.22123820427987\\
1645	-8.71786743411167\\
1646	-10.9793318156633\\
1647	-10.8239331319621\\
1648	-10.0717988570502\\
1649	-9.36161764101712\\
1650	-8.69074817528372\\
1651	-10.955368925353\\
1652	-8.451460777085\\
1653	-9.79933893938596\\
1654	-7.99304452882143\\
1655	-9.46296787096298\\
1656	-9.03652210637961\\
1657	-8.94800378458808\\
1658	-8.58856764677694\\
1659	-11.320646058716\\
1660	-10.18117384121\\
1661	-14.1077649756935\\
1662	-22.2632652415812\\
1663	-16.2416506959216\\
1664	-10.5104054427188\\
1665	-11.1238424125982\\
1666	-12.7510998439195\\
1667	-15.7906578493585\\
1668	-12.4450401148711\\
1669	-12.8582871537612\\
1670	-9.3280497891385\\
1671	-13.2725422885753\\
1672	-9.28945530000851\\
1673	-8.05630550184128\\
1674	-12.5428872740329\\
1675	-8.82402382412329\\
1676	-12.044982289955\\
1677	-13.5264233233286\\
1678	-11.2614484333451\\
1679	-10.6648068367463\\
1680	-10.8835095248089\\
1681	-13.1953630576531\\
1682	-11.4598406277971\\
1683	-10.8790568268316\\
1684	-11.1528194345838\\
1685	-10.6005669028382\\
1686	-9.59292572221307\\
1687	-9.41484634385301\\
1688	-8.17702361658212\\
1689	-10.3824407395336\\
1690	-13.9069674649184\\
1691	-13.5565569459411\\
1692	-12.1394102495315\\
1693	-11.4847190099322\\
1694	-10.7133322275622\\
1695	-9.35980865458714\\
1696	-10.0284016650908\\
1697	-10.8478298046357\\
1698	-13.3182047342041\\
1699	-15.9213500822957\\
1700	-11.9934809751858\\
1701	-10.5850524668786\\
1702	-11.9205839429126\\
1703	-18.6052360427489\\
1704	-13.6370367005659\\
1705	-12.8561826390614\\
1706	-15.7874468047896\\
1707	-13.7991690151205\\
1708	-11.5219769060316\\
1709	-12.3894192323019\\
1710	-11.3745045535463\\
1711	-12.0139417992248\\
1712	-13.1982505759179\\
1713	-11.5938716918738\\
1714	-12.4822361220006\\
1715	-12.3002541690371\\
1716	-12.2178381061233\\
1717	-10.9632101070347\\
1718	-11.802398309864\\
1719	-10.2358810958146\\
1720	-8.80097589391244\\
1721	-10.7401121957014\\
1722	-11.740085957726\\
1723	-9.07679197193374\\
1724	-12.0532113137899\\
1725	-10.6746919352383\\
1726	-3.73010160207718\\
1727	-8.61238070492982\\
1728	-17.7571966263426\\
1729	-13.4770398914804\\
1730	-9.94603770788353\\
1731	-11.2670718190461\\
1732	-12.3603862496022\\
1733	-11.7705058764675\\
1734	-9.83894180070138\\
1735	-8.41099083163726\\
1736	-7.50845205120095\\
1737	-8.61730128679551\\
1738	-7.82632121516349\\
1739	-8.64829983675031\\
1740	-8.59950822256064\\
1741	-8.59645345436752\\
1742	-7.29282591993228\\
1743	-10.5650758029533\\
1744	-9.192325050916\\
1745	-10.735586609904\\
1746	-9.04349856283216\\
1747	-12.5108276932258\\
1748	-11.1389553707345\\
1749	-14.6849902978402\\
1750	-13.5106088307741\\
1751	-11.4451681798296\\
1752	-13.060138951003\\
1753	-10.1804488467676\\
1754	-9.96450263693669\\
1755	-10.5113741234884\\
1756	-12.3817561432979\\
1757	-11.7248763220634\\
1758	-16.1919358256721\\
1759	-13.1664516537803\\
1760	-14.3912713235165\\
1761	-17.3426655307508\\
1762	-15.1391048227379\\
1763	-12.7218203397172\\
1764	-11.8172593582244\\
1765	-13.1133921239616\\
1766	-10.663658048524\\
1767	-10.3683926957922\\
1769	-8.83161825833554\\
1770	-9.29719273648061\\
1771	-13.5607707942852\\
1772	-20.6143635060264\\
1773	-14.8546541994021\\
1774	-14.0850281338501\\
1775	-14.5746160775416\\
1776	-11.2248893512217\\
1777	-11.3969584289921\\
1778	-9.71808280646769\\
1779	-9.22340941789821\\
1780	-7.76141874107634\\
1781	-9.66800139416932\\
1782	-13.2763253434891\\
1783	-14.3277624945283\\
1784	-12.5090782427778\\
1785	-10.8353498608553\\
1786	-13.5489692683179\\
1787	-17.6872082518494\\
1788	-17.814358055299\\
1789	-14.3439396542103\\
1790	-22.3509865895144\\
1791	-17.5060584988173\\
1792	-12.379968477554\\
1793	-10.7282560605631\\
1794	-9.56428600328536\\
1795	-12.1227170600928\\
1796	-15.4668555074416\\
1797	-16.7533371424686\\
1799	-12.2423090397492\\
1800	-10.7469452999671\\
1801	-10.9687513463582\\
1802	-9.29487927119112\\
1803	-8.89187665135023\\
1804	-8.8692360600171\\
1805	-7.85595846504134\\
};
\addlegendentry{OSA predition}

\addplot [color=mycolor3, dotted, line width=2.0pt]
  table[row sep=crcr]{%
1006	-14.6479999999999\\
1007	-17.0899999999999\\
1008	-10.9860000000001\\
1009	-7.32400000000007\\
1010	-10.0316139263653\\
1011	-11.1567920087077\\
1012	-12.8257251027273\\
1013	-11.6262460595835\\
1014	-11.2726974986422\\
1015	-14.9770851856208\\
1016	-14.2681202290998\\
1017	-9.26725110859752\\
1018	-22.0825408336441\\
1019	-21.8422744034292\\
1020	-19.5912841920056\\
1021	-16.277594950576\\
1022	-15.0167584114299\\
1023	-16.8578704661311\\
1024	-14.6898801522968\\
1025	-13.643897033885\\
1026	-15.3143072719936\\
1027	-16.0096550122769\\
1028	-14.2821959160367\\
1029	-17.386831547901\\
1030	-17.5875328556413\\
1031	-14.7223723335035\\
1032	-17.6339874799735\\
1033	-18.1886025896249\\
1034	-15.8395067346023\\
1035	-14.1211605443991\\
1036	-13.5540450142603\\
1037	-13.7887294540651\\
1038	-15.409950350476\\
1039	-14.9243836199748\\
1040	-15.1461320730191\\
1041	-15.3376746350984\\
1042	-15.6539729788285\\
1043	-20.0005804414923\\
1044	-18.9212713672887\\
1045	-14.4983361845987\\
1046	-14.0925072750213\\
1047	-13.0028643263529\\
1048	-12.796840989557\\
1049	-12.5294878670531\\
1050	-12.1115927803496\\
1051	-11.9446706126446\\
1052	-12.8449355035946\\
1053	-14.3756168411749\\
1054	-13.8501982784333\\
1055	-17.9678670096373\\
1056	-15.9841590861354\\
1057	-12.990197231024\\
1058	-12.9749508401894\\
1059	-12.415001786055\\
1060	-13.16223616201\\
1061	-11.8306462981445\\
1062	-12.0093328789064\\
1063	-10.9823478350665\\
1064	-11.1496213847167\\
1065	-11.2174574220724\\
1066	-10.4806227397726\\
1067	-10.9979809635338\\
1068	-15.2143760685876\\
1069	-15.2892735095998\\
1070	-16.4874995517537\\
1071	-16.0419612568508\\
1072	-17.2046760111029\\
1073	-15.3550292819691\\
1074	-14.9649006443176\\
1075	-14.4132314444525\\
1076	-13.8329455719761\\
1077	-13.8898930878113\\
1078	-18.5010940663115\\
1079	-29.0883688850383\\
1080	-30.0543618060642\\
1081	-26.8638948029816\\
1082	-18.2159395777471\\
1083	-24.5570792949791\\
1084	-31.8276294485001\\
1085	-29.5788195387011\\
1086	-24.6619705664129\\
1087	-28.2146871814334\\
1088	-40.7725560622457\\
1089	-27.2871975760129\\
1090	-24.1241176543824\\
1091	-16.9169861962444\\
1092	-15.5063544409893\\
1093	-14.3946912374745\\
1094	-16.4746687572394\\
1095	-13.6534829094603\\
1096	-12.3051397002248\\
1097	-12.1102342309177\\
1098	-11.9781729759263\\
1099	-14.4653333816425\\
1100	-14.6678011937665\\
1101	-13.9631226678132\\
1102	-16.374249946376\\
1103	-16.7155701007362\\
1104	-21.7924399993185\\
1105	-18.8180106319414\\
1106	-21.0227552657873\\
1107	-17.0671339775438\\
1108	-16.0239300534404\\
1109	-17.2677629464285\\
1110	-14.7298710815273\\
1111	-14.1307925960696\\
1112	-15.3582487417718\\
1113	-15.7629503697185\\
1114	-13.0638709636262\\
1115	-13.265088448017\\
1116	-12.2152661497817\\
1117	-12.2059136211233\\
1118	-12.2984375871335\\
1119	-11.60302947276\\
1120	-11.4873765329341\\
1121	-12.1701983026192\\
1122	-16.4657767832955\\
1123	-18.1895798382054\\
1124	-18.1744061980319\\
1125	-13.3557984577533\\
1126	-14.8898760006218\\
1127	-23.4646436554328\\
1128	-18.8195894916942\\
1129	-18.3941067573833\\
1130	-20.6706195066524\\
1131	-14.3776678448996\\
1132	-13.9129723843969\\
1133	-14.1622277809922\\
1134	-16.0644174681311\\
1135	-21.929522692041\\
1136	-28.8704177550842\\
1137	-26.7290209296748\\
1138	-19.1180887245689\\
1139	-20.2308653544837\\
1140	-20.1748350419984\\
1141	-19.0311992610691\\
1142	-18.3430653854557\\
1143	-14.6276280190978\\
1144	-13.9926132904925\\
1145	-13.6320914084181\\
1146	-13.2368394821367\\
1147	-13.4306718440678\\
1148	-15.9087912258726\\
1149	-22.8689280637466\\
1150	-18.8868073540407\\
1151	-14.83351017582\\
1152	-14.1920479054272\\
1153	-14.3491533568329\\
1154	-12.8957687269069\\
1155	-13.6721656932493\\
1156	-11.8417081896096\\
1157	-12.0571146200825\\
1158	-11.6766886756536\\
1159	-11.7111977447148\\
1160	-11.4771261410447\\
1161	-11.5073474477113\\
1162	-11.2484863008804\\
1163	-11.8747528451552\\
1164	-12.6571629984144\\
1165	-9.13256247211166\\
1166	-12.9051006798904\\
1167	-22.6397569501239\\
1168	-23.1552338755689\\
1169	-16.7376284355646\\
1170	-14.9057427298235\\
1171	-14.7220501886056\\
1172	-14.9933220997691\\
1173	-12.3153876375677\\
1174	-14.2440767298031\\
1175	-16.2467021433902\\
1176	-19.1070021788955\\
1177	-30.8793656370606\\
1178	-39.3321619068913\\
1179	-30.2278119253776\\
1180	-28.325021173433\\
1181	-20.2409407705072\\
1182	-22.3524061102062\\
1183	-24.2170622422507\\
1184	-23.1402630659022\\
1185	-19.0503073960942\\
1186	-17.8975741882814\\
1187	-18.7368666904763\\
1188	-16.730713697593\\
1189	-18.1780971161841\\
1190	-15.8456862027392\\
1191	-20.8919462968843\\
1192	-26.6424432930576\\
1193	-18.1180620577818\\
1194	-14.1831648475493\\
1195	-14.7765018789498\\
1196	-21.8208752074547\\
1197	-25.7922070944978\\
1198	-30.4242831556721\\
1199	-30.4447953262559\\
1200	-30.1673070122897\\
1201	-23.3996779612205\\
1202	-22.1227188087544\\
1203	-24.7176684648457\\
1204	-27.659351480529\\
1205	-19.1530967548561\\
1206	-13.7997396335859\\
1207	-17.074433827378\\
1208	-16.6331552853901\\
1209	-13.0341282945717\\
1210	-13.6206285583003\\
1211	-15.7635896613945\\
1212	-13.5992077906008\\
1213	-12.813586050489\\
1214	-13.5112430418426\\
1215	-13.1849741478668\\
1216	-14.4843461848175\\
1217	-19.094571634301\\
1218	-17.9089095830175\\
1219	-14.1611854361965\\
1220	-14.0746269140268\\
1221	-18.2179624286457\\
1222	-32.1210429539374\\
1223	-24.1169154023535\\
1224	-15.6640673525524\\
1225	-15.0771533848067\\
1226	-16.0919956806511\\
1227	-15.6728173568147\\
1228	-13.4890467163118\\
1229	-13.3154759143356\\
1230	-14.1265759336302\\
1231	-15.847509197301\\
1232	-16.4210091270425\\
1233	-14.0230927705722\\
1234	-16.3224039043862\\
1235	-20.4285124689825\\
1236	-18.6246420475177\\
1237	-16.1826178518668\\
1238	-17.0829015181341\\
1239	-20.5592407696822\\
1240	-15.484170468387\\
1241	-13.5733418245268\\
1242	-12.8193012276299\\
1243	-13.2052053302903\\
1244	-15.2347527295335\\
1245	-14.4909184188214\\
1246	-12.853090522914\\
1247	-14.0595762700698\\
1248	-14.7790789483179\\
1249	-13.0233656356247\\
1250	-13.3179274905478\\
1251	-13.1117805452914\\
1252	-12.8501549895354\\
1253	-13.1486041509997\\
1254	-11.84208824149\\
1255	-11.7102307652392\\
1256	-11.3525694263244\\
1257	-11.1525800206678\\
1258	-12.2704477258469\\
1259	-14.5883598803082\\
1260	-18.2079388615066\\
1261	-23.948180719734\\
1262	-16.7628560234998\\
1263	-14.1292939736907\\
1264	-13.9331895390296\\
1265	-13.5038100056677\\
1266	-13.7292410061141\\
1267	-12.3078838256108\\
1268	-12.2660157171654\\
1269	-12.3427999778746\\
1270	-18.2294679244965\\
1271	-17.7790375482714\\
1272	-20.3059839485013\\
1273	-15.5702850460859\\
1274	-14.9902771819727\\
1275	-13.3465622618421\\
1276	-13.4780766189956\\
1277	-13.1189964289542\\
1278	-12.2293818981334\\
1279	-10.7861230425926\\
1280	-11.9990654702819\\
1281	-13.4063764705963\\
1282	-13.8268590276004\\
1283	-13.9947828954043\\
1284	-20.4146409967179\\
1285	-19.7824353879087\\
1286	-15.4241455740146\\
1287	-16.8958118445628\\
1288	-17.952452077398\\
1289	-20.788051812259\\
1290	-17.9276357892918\\
1291	-14.6140759727191\\
1292	-14.28874586394\\
1293	-15.8749643315443\\
1294	-13.3747788463891\\
1295	-11.6890438375806\\
1296	-11.3442781979481\\
1297	-11.3253624410622\\
1298	-10.738775162645\\
1299	-12.7263850975\\
1300	-12.8993863632538\\
1301	-12.5538370270292\\
1302	-13.0843458461113\\
1303	-11.6855261999967\\
1304	-13.712755258384\\
1305	-9.80664094255712\\
1306	-14.3830176303243\\
1307	-14.8042951933692\\
1308	-14.9587962583273\\
1309	-14.186335644582\\
1310	-12.7849206217434\\
1311	-13.1848878266896\\
1312	-18.6601744565226\\
1313	-17.9029223865168\\
1314	-14.6650995902644\\
1315	-18.4932180660358\\
1316	-17.7611527541146\\
1317	-16.1767011041902\\
1318	-17.4485515097251\\
1319	-17.7624666498043\\
1320	-15.3013601646521\\
1321	-14.4480429026635\\
1322	-18.1102961287561\\
1323	-26.5407080770142\\
1324	-20.7259990866919\\
1325	-14.6273768555711\\
1326	-14.7856772567341\\
1327	-15.3671246117019\\
1328	-17.3338787181954\\
1329	-18.5021990592131\\
1330	-15.2916660579713\\
1331	-15.7864428906041\\
1332	-18.2850640139434\\
1333	-19.9305607726169\\
1334	-14.4211422250341\\
1335	-13.9806905987857\\
1336	-16.4220015903479\\
1337	-23.6510969057688\\
1338	-21.7408940277089\\
1339	-21.0713479614687\\
1340	-14.6095421535131\\
1341	-14.81063875365\\
1342	-15.0529493792183\\
1343	-12.9070686161986\\
1344	-13.5547939172093\\
1345	-14.0879399649157\\
1346	-13.4117406209859\\
1347	-9.94134765549029\\
1348	-14.7382020962925\\
1349	-16.9708451426727\\
1350	-13.9909803103108\\
1351	-13.3732251316094\\
1352	-14.2467949246404\\
1353	-12.5297380996046\\
1354	-12.1471304708575\\
1355	-11.6882165528109\\
1356	-11.6310113689174\\
1357	-10.8888740473628\\
1358	-13.1779719209912\\
1359	-17.1506168468718\\
1360	-12.9960549134555\\
1361	-12.6854581013549\\
1362	-12.2347816669906\\
1363	-11.7298240919165\\
1364	-11.4519759215473\\
1365	-12.0603462387355\\
1366	-23.14057268496\\
1367	-19.2482086362695\\
1368	-21.0453781219883\\
1369	-25.3262152203322\\
1370	-25.4236018145793\\
1371	-19.4951451903999\\
1372	-16.4007514117229\\
1373	-16.4792377060799\\
1374	-16.8833184636735\\
1375	-18.0091970064314\\
1376	-19.5196954882763\\
1377	-19.7573777709058\\
1378	-25.3345063401109\\
1379	-28.8624146609616\\
1380	-33.5463849591126\\
1381	-23.6643264128427\\
1382	-23.7598241767632\\
1383	-27.8570584596623\\
1384	-20.7767147410339\\
1385	-23.4427098533874\\
1386	-21.4923547910537\\
1387	-14.4801638200861\\
1388	-13.3051662689884\\
1389	-13.209775204378\\
1390	-15.1069898218229\\
1391	-13.921510680075\\
1392	-12.442855980888\\
1393	-12.5818052663565\\
1394	-12.0751519299008\\
1395	-11.9337262940669\\
1396	-11.1765042448778\\
1397	-11.2878783189731\\
1398	-11.1392393884269\\
1399	-11.9139506292279\\
1400	-15.5014885102514\\
1401	-13.2886342643101\\
1402	-17.6260013162384\\
1403	-27.2383337734143\\
1404	-24.8906695892092\\
1405	-28.3549983962773\\
1406	-20.9406743814598\\
1407	-21.8383197250105\\
1408	-16.6922511085795\\
1409	-14.5826978064897\\
1410	-14.6617476415329\\
1411	-14.1530924874746\\
1412	-12.8318930539469\\
1413	-13.0636348465762\\
1414	-11.6430509264469\\
1415	-13.8295917541968\\
1416	-11.1774663708802\\
1417	-11.8366564211924\\
1418	-16.612830248893\\
1419	-18.7874494410419\\
1420	-17.4856197166603\\
1421	-16.6827999655338\\
1422	-18.3448707462114\\
1423	-16.7588515846226\\
1424	-14.7521251467931\\
1425	-14.6676196665169\\
1426	-14.005480729332\\
1427	-16.3498729574474\\
1428	-13.4156208637094\\
1429	-12.9843955245497\\
1430	-13.7996626202778\\
1431	-18.9354298325939\\
1432	-15.482428296679\\
1433	-13.6899804876173\\
1434	-13.0474967111886\\
1435	-13.5099383136983\\
1436	-12.2846120104141\\
1437	-12.0625899852917\\
1438	-11.915995344498\\
1439	-13.7055906788978\\
1440	-14.6622298974896\\
1441	-12.9670908044673\\
1442	-13.6161301696086\\
1443	-13.910992279564\\
1444	-12.194554940176\\
1445	-11.7771150159688\\
1446	-14.3884821370987\\
1447	-21.9888650744065\\
1448	-20.3959169743762\\
1449	-17.2669421205665\\
1450	-16.6058119741579\\
1451	-15.6409194922851\\
1453	-13.7926745673426\\
1454	-14.9147116611348\\
1455	-15.0551802207381\\
1456	-12.6718535945008\\
1457	-13.174114918743\\
1458	-15.5135360474512\\
1459	-16.2696030589768\\
1460	-16.7965899217625\\
1461	-17.1130253627123\\
1462	-21.9462246789631\\
1463	-27.8378634922697\\
1464	-31.5087711985134\\
1465	-21.4234827144812\\
1466	-14.6703095404671\\
1467	-15.1661618555452\\
1468	-15.9048011029852\\
1469	-13.2050023689562\\
1470	-12.7622223366275\\
1471	-15.3836176202526\\
1472	-15.3587036918068\\
1473	-13.9159933274022\\
1474	-14.9082478935318\\
1475	-16.7142467556964\\
1476	-18.268963864567\\
1477	-19.1626447859924\\
1478	-27.1537795558208\\
1479	-24.9472879846157\\
1480	-18.8448827480477\\
1481	-15.7072366918028\\
1482	-15.870676936644\\
1483	-16.5990407933762\\
1484	-18.1501701969037\\
1485	-14.6979733424212\\
1486	-14.4392940607215\\
1487	-14.6592180203315\\
1488	-12.4938613669708\\
1489	-12.2467007235055\\
1490	-17.4113415010204\\
1491	-14.1961704605542\\
1492	-14.3467846401443\\
1493	-22.4857591391628\\
1494	-19.6746164762314\\
1495	-17.3480680504695\\
1496	-21.4973522940218\\
1497	-21.8838727330983\\
1498	-16.3947255848796\\
1499	-13.718802344187\\
1500	-13.7997626924878\\
1501	-17.2966697843449\\
1502	-27.4894098356083\\
1503	-28.6456617863071\\
1504	-25.6232158791845\\
1506	-16.4239576610569\\
1507	-15.8749557254232\\
1508	-14.3222190273634\\
1509	-13.7952544716604\\
1510	-12.8859254963807\\
1511	-13.1249827889869\\
1512	-14.1330984759998\\
1513	-12.1731392632223\\
1514	-12.2054651827757\\
1515	-15.0535894199452\\
1516	-17.190789935443\\
1517	-19.4922968155031\\
1518	-24.515055919825\\
1519	-30.1282261545921\\
1520	-22.4821223208328\\
1521	-19.2819016959709\\
1522	-20.4481916998302\\
1523	-19.0240706188611\\
1524	-14.9641899875915\\
1525	-15.7873575735691\\
1526	-23.040920625006\\
1527	-27.0495332934661\\
1528	-16.6220188760265\\
1529	-13.9467133261346\\
1530	-14.8023393196652\\
1531	-17.0306860683963\\
1532	-14.8710179241627\\
1533	-11.0803706684605\\
1534	-12.4310270194633\\
1535	-14.020348897538\\
1536	-13.1725903279055\\
1537	-12.2840059584066\\
1538	-11.5888486277436\\
1539	-11.2874482129455\\
1540	-11.0551333096655\\
1541	-10.7393706053822\\
1542	-11.475788954428\\
1543	-16.7094992076782\\
1544	-17.9340369256593\\
1545	-16.2388462385477\\
1546	-18.5463243242011\\
1547	-23.4967193847808\\
1548	-23.0633378585012\\
1549	-26.7978028205673\\
1550	-25.1501559253722\\
1551	-19.5236630069408\\
1552	-16.1022710927086\\
1553	-16.115116821707\\
1554	-23.9028217693917\\
1555	-26.1499573318649\\
1556	-17.7572205161869\\
1557	-14.8202476023941\\
1558	-14.3655520090383\\
1559	-16.681386186413\\
1560	-13.6580292421011\\
1561	-14.4645289183668\\
1562	-9.93246153001405\\
1563	-12.5204833239832\\
1564	-12.748101439161\\
1565	-12.2552765960352\\
1566	-12.1760101024879\\
1567	-12.4913269401061\\
1568	-10.0901109997765\\
1569	-9.82784560629261\\
1571	-10.1853930848745\\
1572	-10.1671050335042\\
1573	-8.41929455935065\\
1574	-20.3704780522714\\
1575	-26.7262966761098\\
1576	-25.200954937832\\
1577	-19.1649359625048\\
1578	-22.819583563821\\
1579	-35.7351141839474\\
1580	-32.2274062364509\\
1581	-28.0869827371253\\
1582	-22.1740090437834\\
1583	-22.9541811925112\\
1584	-23.9294072657453\\
1585	-17.5136870664203\\
1586	-16.3097311510903\\
1587	-13.7503988992487\\
1588	-13.5933407618174\\
1589	-16.4447977830748\\
1590	-13.9903196045709\\
1591	-13.9944719138291\\
1592	-14.9765698536526\\
1593	-14.5805023549101\\
1595	-14.513075739161\\
1596	-15.7170866829458\\
1597	-16.7239479603822\\
1598	-14.9404675878761\\
1599	-13.5094903655659\\
1600	-14.376613578577\\
1601	-12.0214763434119\\
1602	-16.698852586165\\
1603	-12.6800825965186\\
1604	-10.707054208033\\
1605	-11.162992646174\\
1606	-17.2842216129907\\
1607	-9.4921767125079\\
1608	-8.43995383014612\\
1609	-10.9595037329705\\
1610	-12.4349836717502\\
1611	-11.4600803570438\\
1612	-13.7355605378148\\
1613	-14.0125997069651\\
1614	-12.3221982053178\\
1615	-12.7802049863556\\
1616	-11.3259703034855\\
1617	-12.6525322541886\\
1618	-13.1721091858501\\
1619	-14.1360847411268\\
1620	-8.75858773658774\\
1621	-9.61178580458295\\
1622	-9.02736837445286\\
1623	-12.416722742023\\
1624	-12.4428478986438\\
1625	-11.4532714602531\\
1626	-11.0149644246935\\
1627	-10.9467456041652\\
1628	-10.7758782725402\\
1629	-10.6485494205735\\
1630	-10.5863414514788\\
1631	-10.2205026806225\\
1632	-10.6546599746082\\
1633	-9.15526357989984\\
1634	-9.4713728827121\\
1635	-9.10104413049498\\
1636	-10.074134347883\\
1637	-10.0666756696949\\
1638	-9.78499317275964\\
1639	-10.3375622813592\\
1640	-9.35708228778367\\
1641	-11.4882894763639\\
1642	-19.4330341299108\\
1643	-15.2178872206732\\
1644	-13.7586082232829\\
1645	-12.7411562770865\\
1646	-14.3255860466281\\
1647	-14.8101291757021\\
1648	-12.5605471850456\\
1649	-12.3476910744218\\
1650	-11.9074427923479\\
1651	-12.6374701627158\\
1652	-11.3337466812866\\
1653	-11.5468594774136\\
1654	-10.6649912167029\\
1655	-11.9887666314453\\
1656	-11.2944788027989\\
1657	-11.1464056834864\\
1658	-10.9158965563813\\
1659	-13.2722967031209\\
1660	-12.8441033430713\\
1661	-16.0137401284924\\
1662	-25.281440296078\\
1663	-20.1094615546774\\
1664	-14.4227858066035\\
1665	-14.1274360647606\\
1666	-16.0274966857496\\
1667	-19.6748145720405\\
1668	-16.2976843896615\\
1669	-17.1309175747626\\
1670	-13.4409503167851\\
1671	-14.4368923353838\\
1672	-13.4828964504252\\
1673	-13.3248051914991\\
1674	-15.4783960355678\\
1675	-14.022334742355\\
1676	-15.7986217906725\\
1677	-16.9293101973556\\
1678	-14.7759236301783\\
1679	-13.3988142458497\\
1680	-13.5280349087921\\
1681	-16.3275280574708\\
1682	-14.2877630051396\\
1683	-14.323562883465\\
1684	-13.9261646546977\\
1685	-12.907434540173\\
1686	-12.8265973064333\\
1687	-12.2019818529848\\
1688	-12.1024503903591\\
1689	-14.4681482397732\\
1690	-17.1973787298116\\
1691	-16.7793300895971\\
1692	-15.6674492810523\\
1693	-13.6556981203687\\
1694	-13.1141855627131\\
1695	-12.6873728488658\\
1696	-13.2549288938894\\
1697	-14.6514038354369\\
1698	-17.0221115265442\\
1699	-19.5612407840977\\
1700	-15.9135284378876\\
1701	-13.4996746995475\\
1702	-15.7217428637462\\
1703	-23.0830881123707\\
1704	-16.8642078082444\\
1705	-16.3298075304147\\
1706	-18.5092182859341\\
1707	-15.8882653592941\\
1708	-14.3340234490524\\
1709	-15.206730449994\\
1710	-14.6179449685819\\
1711	-15.3992206762287\\
1712	-17.3568838940171\\
1713	-14.8710106127485\\
1714	-15.5469525543385\\
1715	-15.9523352814183\\
1716	-15.518757385812\\
1717	-14.1137249738629\\
1718	-15.4558126465079\\
1719	-13.8797983262014\\
1720	-13.72871737639\\
1721	-14.3786336501803\\
1722	-14.4106969848885\\
1723	-12.1385543579036\\
1724	-14.8386434758681\\
1725	-15.6531570514026\\
1726	-10.6118159201333\\
1727	-14.5146033608455\\
1728	-23.4311930661852\\
1729	-19.9483502252576\\
1730	-15.5325984138235\\
1731	-15.312963699246\\
1732	-15.2697384707121\\
1733	-14.3447425956165\\
1735	-12.3254269721531\\
1736	-12.3948363807085\\
1737	-12.3437908695555\\
1738	-12.5167120627364\\
1740	-11.7937602840348\\
1741	-11.7573881330138\\
1742	-10.6425558125693\\
1743	-14.3266174097703\\
1744	-12.522517351432\\
1745	-12.9781194005834\\
1746	-12.7038194844847\\
1747	-14.5716605445803\\
1748	-14.3985801451388\\
1749	-17.9941342268014\\
1750	-16.1109770833336\\
1751	-15.6930051663549\\
1752	-16.0698007357073\\
1753	-12.8270477201352\\
1754	-14.895758587762\\
1755	-13.2948705010288\\
1756	-13.9118537899747\\
1757	-14.3637164642812\\
1758	-16.3963730925107\\
1759	-14.4510747588552\\
1760	-15.3833693353388\\
1761	-19.2897756289622\\
1762	-17.3431215494713\\
1763	-15.0956300281232\\
1764	-14.0611518179539\\
1765	-15.2858679063195\\
1766	-14.143636572051\\
1767	-13.341616665009\\
1768	-12.6926067744387\\
1769	-12.843817093559\\
1770	-12.4813173039229\\
1771	-17.2594916174198\\
1772	-24.8134791124987\\
1773	-19.9926148219697\\
1774	-18.830938607681\\
1775	-18.4307880164483\\
1776	-14.4353417957946\\
1777	-14.0331506541652\\
1778	-13.237655682188\\
1779	-12.7959841993647\\
1780	-12.204391899543\\
1781	-14.1129144652846\\
1782	-17.125496999166\\
1783	-18.1881684697696\\
1784	-16.3631182989971\\
1785	-13.9057039464274\\
1786	-17.1556892942986\\
1787	-22.1703281615942\\
1788	-21.6996832198315\\
1789	-18.9184550739562\\
1790	-26.6922758467149\\
1791	-23.0656564488834\\
1792	-14.6597932273833\\
1793	-14.3203552651621\\
1794	-14.0413084835134\\
1795	-15.6859097252479\\
1796	-20.337469135666\\
1797	-22.0346809473658\\
1798	-18.6679124797299\\
1799	-15.7228073265778\\
1800	-14.0166010594194\\
1801	-13.8113292077874\\
1802	-13.3877409539618\\
1803	-13.3119824109153\\
1804	-12.4603385603655\\
1805	-12.0637430931281\\
};
\addlegendentry{MPO prediction}

\end{axis}

\begin{axis}[%
width=6.159cm,
height=1.831cm,
at={(0cm,5.085cm)},
scale only axis,
xmin=1000,
xmax=2000,
xlabel style={font=\color{white!15!black}},
xlabel={Sample index},
ymin=-38.4123338858692,
ymax=2.441,
ylabel style={font=\color{white!15!black}},
ylabel={$y(t)$},
axis background/.style={fill=white},
title style={font=\bfseries},
title={C5: RMSE(OSA) = 2.757, RMSE(MPO) = 5.0346},
legend style={legend cell align=left, align=left, draw=white!15!black}
]
\addplot [color=mycolor1, line width=2.0pt]
  table[row sep=crcr]{%
1006	-15.8689999999999\\
1007	-14.6479999999999\\
1008	-14.6479999999999\\
1009	-7.32400000000007\\
1010	-8.54500000000007\\
1011	-10.9860000000001\\
1012	-9.76600000000008\\
1013	-9.76600000000008\\
1015	-2.44100000000003\\
1016	-3.66200000000003\\
1017	-2.44100000000003\\
1018	-12.2070000000001\\
1019	-13.4280000000001\\
1021	-10.9860000000001\\
1022	-6.10400000000004\\
1023	-6.10400000000004\\
1024	-1.221\\
1026	-10.9860000000001\\
1027	-12.2070000000001\\
1028	-8.54500000000007\\
1029	-14.6479999999999\\
1030	-14.6479999999999\\
1031	-9.76600000000008\\
1032	-13.4280000000001\\
1033	-12.2070000000001\\
1034	-9.76600000000008\\
1035	-9.76600000000008\\
1036	-8.54500000000007\\
1037	-8.54500000000007\\
1038	-12.2070000000001\\
1039	-10.9860000000001\\
1042	-10.9860000000001\\
1043	-14.6479999999999\\
1044	-15.8689999999999\\
1045	-9.76600000000008\\
1046	-8.54500000000007\\
1047	-8.54500000000007\\
1048	-6.10400000000004\\
1049	-7.32400000000007\\
1050	-7.32400000000007\\
1051	-4.88300000000004\\
1052	-9.76600000000008\\
1053	-10.9860000000001\\
1054	-8.54500000000007\\
1055	-13.4280000000001\\
1056	-15.8689999999999\\
1057	-8.54500000000007\\
1058	-6.10400000000004\\
1059	-6.10400000000004\\
1060	-9.76600000000008\\
1061	-6.10400000000004\\
1062	-4.88300000000004\\
1063	-7.32400000000007\\
1064	-7.32400000000007\\
1065	-3.66200000000003\\
1068	-10.9860000000001\\
1069	-10.9860000000001\\
1070	-9.76600000000008\\
1071	-13.4280000000001\\
1072	-10.9860000000001\\
1073	-12.2070000000001\\
1074	-10.9860000000001\\
1075	-10.9860000000001\\
1076	-7.32400000000007\\
1077	-8.54500000000007\\
1079	-20.752\\
1080	-20.752\\
1081	-19.5309999999999\\
1082	-13.4280000000001\\
1083	-15.8689999999999\\
1084	-24.414\\
1085	-23.193\\
1086	-15.8689999999999\\
1088	-28.076\\
1089	-23.193\\
1090	-17.0899999999999\\
1091	-14.6479999999999\\
1092	-13.4280000000001\\
1093	-9.76600000000008\\
1095	-12.2070000000001\\
1096	-4.88300000000004\\
1097	-4.88300000000004\\
1098	-7.32400000000007\\
1099	-10.9860000000001\\
1100	-9.76600000000008\\
1101	-9.76600000000008\\
1102	-10.9860000000001\\
1103	-13.4280000000001\\
1105	-15.8689999999999\\
1106	-14.6479999999999\\
1107	-17.0899999999999\\
1108	-12.2070000000001\\
1109	-12.2070000000001\\
1110	-10.9860000000001\\
1111	-8.54500000000007\\
1112	-9.76600000000008\\
1113	-12.2070000000001\\
1114	-8.54500000000007\\
1115	-9.76600000000008\\
1116	-7.32400000000007\\
1117	-6.10400000000004\\
1118	-8.54500000000007\\
1119	-6.10400000000004\\
1120	-4.88300000000004\\
1121	-8.54500000000007\\
1122	-13.4280000000001\\
1124	-10.9860000000001\\
1125	-7.32400000000007\\
1126	-8.54500000000007\\
1127	-19.5309999999999\\
1128	-14.6479999999999\\
1129	-10.9860000000001\\
1130	-17.0899999999999\\
1131	-12.2070000000001\\
1132	-6.10400000000004\\
1133	-9.76600000000008\\
1134	-8.54500000000007\\
1135	-14.6479999999999\\
1136	-15.8689999999999\\
1137	-19.5309999999999\\
1138	-14.6479999999999\\
1139	-14.6479999999999\\
1140	-13.4280000000001\\
1141	-13.4280000000001\\
1142	-14.6479999999999\\
1144	-7.32400000000007\\
1145	-7.32400000000007\\
1146	-6.10400000000004\\
1148	-10.9860000000001\\
1149	-15.8689999999999\\
1150	-14.6479999999999\\
1151	-8.54500000000007\\
1152	-8.54500000000007\\
1153	-9.76600000000008\\
1155	-2.44100000000003\\
1157	-7.32400000000007\\
1158	-8.54500000000007\\
1159	-6.10400000000004\\
1160	-7.32400000000007\\
1162	-4.88300000000004\\
1163	-4.88300000000004\\
1164	-1.221\\
1165	-3.66200000000003\\
1166	-12.2070000000001\\
1167	-14.6479999999999\\
1168	-15.8689999999999\\
1169	-12.2070000000001\\
1170	-7.32400000000007\\
1171	-6.10400000000004\\
1172	-3.66200000000003\\
1173	-7.32400000000007\\
1175	-9.76600000000008\\
1176	-12.2070000000001\\
1177	-18.3109999999999\\
1178	-18.3109999999999\\
1179	-19.5309999999999\\
1180	-18.3109999999999\\
1181	-15.8689999999999\\
1182	-15.8689999999999\\
1184	-18.3109999999999\\
1185	-12.2070000000001\\
1186	-10.9860000000001\\
1187	-12.2070000000001\\
1188	-10.9860000000001\\
1189	-12.2070000000001\\
1190	-9.76600000000008\\
1191	-15.8689999999999\\
1192	-18.3109999999999\\
1193	-12.2070000000001\\
1194	-7.32400000000007\\
1195	-9.76600000000008\\
1196	-17.0899999999999\\
1197	-18.3109999999999\\
1198	-18.3109999999999\\
1199	-21.973\\
1200	-20.752\\
1201	-17.0899999999999\\
1202	-15.8689999999999\\
1204	-20.752\\
1205	-17.0899999999999\\
1206	-8.54500000000007\\
1207	-14.6479999999999\\
1208	-12.2070000000001\\
1209	-7.32400000000007\\
1210	-10.9860000000001\\
1213	-7.32400000000007\\
1214	-8.54500000000007\\
1215	-8.54500000000007\\
1216	-9.76600000000008\\
1217	-13.4280000000001\\
1218	-14.6479999999999\\
1219	-7.32400000000007\\
1220	-8.54500000000007\\
1221	-12.2070000000001\\
1222	-17.0899999999999\\
1223	-15.8689999999999\\
1224	-8.54500000000007\\
1225	-8.54500000000007\\
1226	-9.76600000000008\\
1229	-6.10400000000004\\
1230	-8.54500000000007\\
1231	-9.76600000000008\\
1233	-9.76600000000008\\
1235	-14.6479999999999\\
1236	-15.8689999999999\\
1237	-10.9860000000001\\
1238	-13.4280000000001\\
1239	-17.0899999999999\\
1240	-12.2070000000001\\
1241	-3.66200000000003\\
1242	-9.76600000000008\\
1243	-8.54500000000007\\
1244	-9.76600000000008\\
1245	-8.54500000000007\\
1247	-8.54500000000007\\
1248	-10.9860000000001\\
1249	-7.32400000000007\\
1250	-7.32400000000007\\
1251	-9.76600000000008\\
1252	-6.10400000000004\\
1253	-8.54500000000007\\
1255	-3.66200000000003\\
1256	-6.10400000000004\\
1257	-4.88300000000004\\
1258	-10.9860000000001\\
1259	-12.2070000000001\\
1260	-12.2070000000001\\
1261	-18.3109999999999\\
1262	-10.9860000000001\\
1263	-4.88300000000004\\
1264	-6.10400000000004\\
1265	-6.10400000000004\\
1266	-8.54500000000007\\
1267	-6.10400000000004\\
1268	-8.54500000000007\\
1269	-7.32400000000007\\
1270	-13.4280000000001\\
1271	-9.76600000000008\\
1272	-14.6479999999999\\
1273	-10.9860000000001\\
1274	-9.76600000000008\\
1275	-6.10400000000004\\
1276	-3.66200000000003\\
1277	-3.66200000000003\\
1278	-1.221\\
1279	-2.44100000000003\\
1280	-8.54500000000007\\
1281	-8.54500000000007\\
1282	-7.32400000000007\\
1283	-9.76600000000008\\
1284	-15.8689999999999\\
1285	-10.9860000000001\\
1286	-9.76600000000008\\
1287	-9.76600000000008\\
1288	-13.4280000000001\\
1289	-14.6479999999999\\
1290	-12.2070000000001\\
1291	-12.2070000000001\\
1292	-6.10400000000004\\
1293	-2.44100000000003\\
1294	-4.88300000000004\\
1295	-4.88300000000004\\
1296	-6.10400000000004\\
1297	-3.66200000000003\\
1298	-4.88300000000004\\
1299	-9.76600000000008\\
1300	-7.32400000000007\\
1301	-7.32400000000007\\
1302	-10.9860000000001\\
1303	-7.32400000000007\\
1304	-4.88300000000004\\
1305	-9.76600000000008\\
1306	-12.2070000000001\\
1307	-9.76600000000008\\
1308	-10.9860000000001\\
1309	-7.32400000000007\\
1310	-7.32400000000007\\
1312	-14.6479999999999\\
1313	-13.4280000000001\\
1314	-8.54500000000007\\
1315	-14.6479999999999\\
1316	-13.4280000000001\\
1319	-13.4280000000001\\
1320	-10.9860000000001\\
1321	-10.9860000000001\\
1322	-12.2070000000001\\
1323	-18.3109999999999\\
1324	-17.0899999999999\\
1325	-10.9860000000001\\
1327	-10.9860000000001\\
1328	-13.4280000000001\\
1329	-13.4280000000001\\
1330	-9.76600000000008\\
1332	-14.6479999999999\\
1333	-14.6479999999999\\
1334	-9.76600000000008\\
1335	-8.54500000000007\\
1336	-10.9860000000001\\
1337	-18.3109999999999\\
1338	-14.6479999999999\\
1339	-13.4280000000001\\
1340	-9.76600000000008\\
1341	-10.9860000000001\\
1342	-9.76600000000008\\
1343	-6.10400000000004\\
1345	-3.66200000000003\\
1346	-3.66200000000003\\
1348	-10.9860000000001\\
1349	-9.76600000000008\\
1350	-6.10400000000004\\
1351	-7.32400000000007\\
1352	-9.76600000000008\\
1353	-6.10400000000004\\
1354	-6.10400000000004\\
1355	-9.76600000000008\\
1356	-6.10400000000004\\
1357	-7.32400000000007\\
1358	-12.2070000000001\\
1359	-12.2070000000001\\
1361	-4.88300000000004\\
1363	-7.32400000000007\\
1364	-6.10400000000004\\
1365	-7.32400000000007\\
1366	-14.6479999999999\\
1367	-9.76600000000008\\
1368	-17.0899999999999\\
1369	-20.752\\
1370	-18.3109999999999\\
1371	-13.4280000000001\\
1372	-10.9860000000001\\
1373	-10.9860000000001\\
1377	-15.8689999999999\\
1378	-20.752\\
1379	-20.752\\
1380	-24.414\\
1381	-15.8689999999999\\
1382	-20.752\\
1383	-20.752\\
1384	-14.6479999999999\\
1385	-17.0899999999999\\
1386	-18.3109999999999\\
1387	-9.76600000000008\\
1388	-7.32400000000007\\
1390	-9.76600000000008\\
1391	-7.32400000000007\\
1392	-6.10400000000004\\
1393	-7.32400000000007\\
1394	-6.10400000000004\\
1395	-3.66200000000003\\
1396	-4.88300000000004\\
1397	-7.32400000000007\\
1398	-2.44100000000003\\
1399	-10.9860000000001\\
1400	-8.54500000000007\\
1401	-7.32400000000007\\
1402	-15.8689999999999\\
1403	-19.5309999999999\\
1405	-19.5309999999999\\
1406	-14.6479999999999\\
1407	-17.0899999999999\\
1409	-7.32400000000007\\
1410	-10.9860000000001\\
1412	-3.66200000000003\\
1413	-6.10400000000004\\
1414	-4.88300000000004\\
1415	-2.44100000000003\\
1416	-4.88300000000004\\
1417	-10.9860000000001\\
1418	-9.76600000000008\\
1419	-10.9860000000001\\
1420	-10.9860000000001\\
1421	-12.2070000000001\\
1422	-14.6479999999999\\
1423	-10.9860000000001\\
1424	-10.9860000000001\\
1425	-8.54500000000007\\
1426	-7.32400000000007\\
1427	-13.4280000000001\\
1429	-6.10400000000004\\
1430	-10.9860000000001\\
1431	-13.4280000000001\\
1432	-10.9860000000001\\
1433	-7.32400000000007\\
1435	-7.32400000000007\\
1436	-4.88300000000004\\
1438	-7.32400000000007\\
1439	-10.9860000000001\\
1440	-8.54500000000007\\
1441	-7.32400000000007\\
1442	-10.9860000000001\\
1444	-6.10400000000004\\
1445	-7.32400000000007\\
1446	-13.4280000000001\\
1447	-14.6479999999999\\
1449	-12.2070000000001\\
1450	-12.2070000000001\\
1451	-8.54500000000007\\
1452	-9.76600000000008\\
1453	-8.54500000000007\\
1454	-12.2070000000001\\
1455	-7.32400000000007\\
1456	-4.88300000000004\\
1457	-9.76600000000008\\
1458	-9.76600000000008\\
1460	-12.2070000000001\\
1461	-10.9860000000001\\
1462	-17.0899999999999\\
1463	-20.752\\
1464	-23.193\\
1465	-15.8689999999999\\
1466	-9.76600000000008\\
1467	-6.10400000000004\\
1468	-6.10400000000004\\
1469	-9.76600000000008\\
1470	-6.10400000000004\\
1471	-9.76600000000008\\
1472	-6.10400000000004\\
1474	-8.54500000000007\\
1476	-13.4280000000001\\
1477	-14.6479999999999\\
1478	-19.5309999999999\\
1479	-14.6479999999999\\
1480	-13.4280000000001\\
1481	-10.9860000000001\\
1483	-10.9860000000001\\
1484	-12.2070000000001\\
1485	-9.76600000000008\\
1486	-10.9860000000001\\
1487	-9.76600000000008\\
1488	-7.32400000000007\\
1489	-12.2070000000001\\
1490	-13.4280000000001\\
1491	-8.54500000000007\\
1492	-9.76600000000008\\
1493	-18.3109999999999\\
1494	-12.2070000000001\\
1495	-12.2070000000001\\
1496	-17.0899999999999\\
1497	-14.6479999999999\\
1499	-7.32400000000007\\
1500	-7.32400000000007\\
1502	-17.0899999999999\\
1503	-20.752\\
1504	-18.3109999999999\\
1505	-13.4280000000001\\
1506	-9.76600000000008\\
1507	-8.54500000000007\\
1508	-6.10400000000004\\
1509	-7.32400000000007\\
1510	-4.88300000000004\\
1511	-8.54500000000007\\
1512	-8.54500000000007\\
1513	-6.10400000000004\\
1514	-9.76600000000008\\
1516	-12.2070000000001\\
1519	-19.5309999999999\\
1520	-18.3109999999999\\
1521	-13.4280000000001\\
1522	-14.6479999999999\\
1523	-12.2070000000001\\
1524	-7.32400000000007\\
1526	-19.5309999999999\\
1527	-18.3109999999999\\
1529	-6.10400000000004\\
1530	-4.88300000000004\\
1532	2.44100000000003\\
1533	-4.88300000000004\\
1534	-6.10400000000004\\
1535	-8.54500000000007\\
1536	-6.10400000000004\\
1537	-4.88300000000004\\
1538	-4.88300000000004\\
1539	-7.32400000000007\\
1540	-4.88300000000004\\
1541	-4.88300000000004\\
1543	-12.2070000000001\\
1544	-13.4280000000001\\
1545	-10.9860000000001\\
1546	-14.6479999999999\\
1547	-17.0899999999999\\
1548	-17.0899999999999\\
1549	-20.752\\
1550	-20.752\\
1551	-14.6479999999999\\
1552	-10.9860000000001\\
1553	-10.9860000000001\\
1554	-18.3109999999999\\
1555	-18.3109999999999\\
1556	-12.2070000000001\\
1558	-4.88300000000004\\
1560	-4.88300000000004\\
1561	-2.44100000000003\\
1562	-6.10400000000004\\
1563	-8.54500000000007\\
1564	-7.32400000000007\\
1565	-7.32400000000007\\
1566	-6.10400000000004\\
1567	-3.66200000000003\\
1568	-4.88300000000004\\
1569	-4.88300000000004\\
1570	-6.10400000000004\\
1571	-2.44100000000003\\
1572	-2.44100000000003\\
1573	-7.32400000000007\\
1574	-9.76600000000008\\
1575	-13.4280000000001\\
1576	-15.8689999999999\\
1577	-10.9860000000001\\
1578	-18.3109999999999\\
1579	-23.193\\
1580	-21.973\\
1581	-19.5309999999999\\
1582	-14.6479999999999\\
1583	-14.6479999999999\\
1584	-17.0899999999999\\
1586	-9.76600000000008\\
1587	-7.32400000000007\\
1588	-7.32400000000007\\
1589	-13.4280000000001\\
1590	-8.54500000000007\\
1591	-12.2070000000001\\
1592	-10.9860000000001\\
1593	-8.54500000000007\\
1594	-8.54500000000007\\
1595	-10.9860000000001\\
1596	-9.76600000000008\\
1597	-12.2070000000001\\
1598	-10.9860000000001\\
1599	-8.54500000000007\\
1600	-12.2070000000001\\
1601	-6.10400000000004\\
1602	-2.44100000000003\\
1603	-7.32400000000007\\
1604	-7.32400000000007\\
1605	-2.44100000000003\\
1606	-1.221\\
1607	-2.44100000000003\\
1608	-6.10400000000004\\
1609	-6.10400000000004\\
1610	-8.54500000000007\\
1611	-7.32400000000007\\
1612	-13.4280000000001\\
1613	-8.54500000000007\\
1614	-6.10400000000004\\
1615	-10.9860000000001\\
1616	-7.32400000000007\\
1617	-4.88300000000004\\
1618	-4.88300000000004\\
1619	-2.44100000000003\\
1620	-3.66200000000003\\
1621	-3.66200000000003\\
1622	-6.10400000000004\\
1623	-9.76600000000008\\
1624	-8.54500000000007\\
1625	-4.88300000000004\\
1626	-7.32400000000007\\
1628	-7.32400000000007\\
1629	-6.10400000000004\\
1630	-2.44100000000003\\
1631	-7.32400000000007\\
1632	-3.66200000000003\\
1633	-4.88300000000004\\
1634	-4.88300000000004\\
1635	-7.32400000000007\\
1638	-3.66200000000003\\
1639	-4.88300000000004\\
1640	-4.88300000000004\\
1641	-13.4280000000001\\
1642	-13.4280000000001\\
1643	-7.32400000000007\\
1644	-12.2070000000001\\
1645	-7.32400000000007\\
1646	-12.2070000000001\\
1647	-8.54500000000007\\
1648	-7.32400000000007\\
1649	-8.54500000000007\\
1650	-7.32400000000007\\
1651	-4.88300000000004\\
1652	-6.10400000000004\\
1653	-9.76600000000008\\
1654	-7.32400000000007\\
1655	-12.2070000000001\\
1656	-6.10400000000004\\
1657	-8.54500000000007\\
1658	-6.10400000000004\\
1659	-12.2070000000001\\
1660	-7.32400000000007\\
1661	-14.6479999999999\\
1662	-17.0899999999999\\
1664	-9.76600000000008\\
1665	-9.76600000000008\\
1666	-12.2070000000001\\
1667	-13.4280000000001\\
1668	-8.54500000000007\\
1669	-13.4280000000001\\
1670	-10.9860000000001\\
1671	-4.88300000000004\\
1672	-3.66200000000003\\
1673	-12.2070000000001\\
1674	-10.9860000000001\\
1675	-8.54500000000007\\
1676	-10.9860000000001\\
1677	-12.2070000000001\\
1679	-7.32400000000007\\
1681	-12.2070000000001\\
1682	-8.54500000000007\\
1683	-10.9860000000001\\
1684	-9.76600000000008\\
1685	-6.10400000000004\\
1686	-8.54500000000007\\
1687	-4.88300000000004\\
1688	-8.54500000000007\\
1689	-10.9860000000001\\
1690	-14.6479999999999\\
1691	-13.4280000000001\\
1692	-13.4280000000001\\
1693	-8.54500000000007\\
1694	-6.10400000000004\\
1696	-8.54500000000007\\
1697	-12.2070000000001\\
1698	-12.2070000000001\\
1699	-14.6479999999999\\
1700	-10.9860000000001\\
1701	-8.54500000000007\\
1702	-12.2070000000001\\
1703	-18.3109999999999\\
1704	-12.2070000000001\\
1705	-14.6479999999999\\
1706	-14.6479999999999\\
1707	-9.76600000000008\\
1708	-9.76600000000008\\
1709	-12.2070000000001\\
1710	-9.76600000000008\\
1711	-10.9860000000001\\
1712	-14.6479999999999\\
1713	-9.76600000000008\\
1714	-12.2070000000001\\
1715	-10.9860000000001\\
1716	-10.9860000000001\\
1717	-6.10400000000004\\
1718	-9.76600000000008\\
1719	-8.54500000000007\\
1720	-9.76600000000008\\
1722	-9.76600000000008\\
1723	-6.10400000000004\\
1724	-3.66200000000003\\
1725	-2.44100000000003\\
1726	-6.10400000000004\\
1727	-10.9860000000001\\
1728	-12.2070000000001\\
1729	-10.9860000000001\\
1730	-8.54500000000007\\
1731	-12.2070000000001\\
1732	-8.54500000000007\\
1733	-9.76600000000008\\
1734	-6.10400000000004\\
1735	-8.54500000000007\\
1736	-7.32400000000007\\
1739	-7.32400000000007\\
1740	-6.10400000000004\\
1741	-6.10400000000004\\
1742	-4.88300000000004\\
1743	-10.9860000000001\\
1744	-7.32400000000007\\
1745	-9.76600000000008\\
1746	-7.32400000000007\\
1747	-12.2070000000001\\
1748	-9.76600000000008\\
1749	-14.6479999999999\\
1750	-8.54500000000007\\
1751	-13.4280000000001\\
1752	-12.2070000000001\\
1753	-7.32400000000007\\
1754	-10.9860000000001\\
1755	-9.76600000000008\\
1758	-13.4280000000001\\
1759	-7.32400000000007\\
1760	-13.4280000000001\\
1761	-14.6479999999999\\
1762	-14.6479999999999\\
1763	-9.76600000000008\\
1764	-9.76600000000008\\
1765	-10.9860000000001\\
1766	-7.32400000000007\\
1767	-9.76600000000008\\
1768	-7.32400000000007\\
1769	-8.54500000000007\\
1770	-7.32400000000007\\
1771	-13.4280000000001\\
1772	-15.8689999999999\\
1773	-15.8689999999999\\
1774	-14.6479999999999\\
1775	-15.8689999999999\\
1776	-8.54500000000007\\
1779	-4.88300000000004\\
1780	-4.88300000000004\\
1781	-8.54500000000007\\
1783	-13.4280000000001\\
1784	-12.2070000000001\\
1785	-7.32400000000007\\
1786	-12.2070000000001\\
1787	-15.8689999999999\\
1788	-15.8689999999999\\
1789	-13.4280000000001\\
1790	-20.752\\
1791	-17.0899999999999\\
1792	-10.9860000000001\\
1793	-6.10400000000004\\
1794	-7.32400000000007\\
1795	-10.9860000000001\\
1796	-13.4280000000001\\
1797	-17.0899999999999\\
1798	-12.2070000000001\\
1799	-10.9860000000001\\
1800	-6.10400000000004\\
1801	-9.76600000000008\\
1802	-7.32400000000007\\
1803	-7.32400000000007\\
1804	-4.88300000000004\\
1805	-4.88300000000004\\
};
\addlegendentry{True output}

\addplot [color=mycolor2, dashed, line width=2.0pt]
  table[row sep=crcr]{%
1006	-14.7114842952997\\
1007	-17.5445951906447\\
1008	-13.3844042393337\\
1009	-10.4430158982946\\
1010	-10.30741003042\\
1011	-10.7206674570182\\
1012	-12.1487209751747\\
1013	-10.0621907030395\\
1014	-10.0755820678819\\
1015	-13.4811124636426\\
1016	-8.33318843671714\\
1017	-3.09461132545948\\
1018	-15.8059015517833\\
1019	-14.0376657699996\\
1020	-12.9764877487653\\
1021	-10.6526596596684\\
1022	-10.5915758885014\\
1023	-10.6638584165623\\
1024	-9.68090985702725\\
1025	-5.57455780120085\\
1026	-8.90497187049232\\
1027	-10.164210093373\\
1028	-10.0427043193208\\
1029	-13.1006444039972\\
1030	-14.6039607987823\\
1031	-12.1145918726531\\
1032	-14.1471403146795\\
1033	-14.9982847267709\\
1034	-12.0187661174539\\
1035	-10.3211197795545\\
1036	-10.0437555265546\\
1037	-10.1087328681681\\
1038	-11.1455139128275\\
1039	-11.7559829854074\\
1040	-11.8406895501628\\
1041	-12.0814058801309\\
1042	-13.1671195146191\\
1043	-16.2018001807319\\
1044	-14.9835520232664\\
1045	-11.6022217010875\\
1046	-11.087977770592\\
1047	-9.4678466362775\\
1048	-9.73428703208879\\
1049	-8.47618454831013\\
1051	-8.40982396231084\\
1052	-8.17677472207538\\
1053	-10.952664209233\\
1054	-10.7179428526558\\
1055	-14.023555499236\\
1056	-12.6040082463714\\
1057	-11.4630936184083\\
1058	-10.7011844817271\\
1059	-8.65453954934105\\
1060	-9.41959489336318\\
1061	-8.8638621004161\\
1062	-8.14066362072117\\
1063	-6.74590536025039\\
1064	-7.95025648381738\\
1065	-7.96603223401598\\
1066	-6.27493894309328\\
1067	-7.57123872516627\\
1068	-10.4414610852816\\
1069	-11.9105776080692\\
1070	-12.6178279359488\\
1071	-12.1020715772579\\
1072	-13.9429798817778\\
1073	-11.3847188223447\\
1074	-11.755523897288\\
1075	-11.477125150958\\
1076	-11.3549410691037\\
1077	-10.0295803857573\\
1078	-13.6725236346856\\
1079	-24.913684981562\\
1080	-23.0987536333337\\
1081	-20.3350814880378\\
1082	-13.5076660765642\\
1083	-19.4704396377856\\
1084	-25.1955927428253\\
1085	-24.4676995741195\\
1086	-17.765099054217\\
1087	-22.9152670533424\\
1088	-34.8615850084409\\
1089	-21.630802932269\\
1090	-16.6614549145377\\
1091	-13.4114743460564\\
1092	-12.2423215138701\\
1093	-12.7420586669816\\
1094	-12.5429746393431\\
1095	-10.7877831896581\\
1096	-10.2736644451554\\
1097	-8.4652714418055\\
1098	-7.62176940192603\\
1099	-10.571651882331\\
1100	-10.9677346943558\\
1101	-10.5075801227897\\
1102	-12.4074750256666\\
1103	-13.6629943814235\\
1104	-18.0714923536461\\
1105	-14.9042852435352\\
1106	-16.5814307047769\\
1107	-13.9870573239907\\
1108	-13.6135152923325\\
1109	-14.6332367772213\\
1110	-11.8257798403956\\
1111	-11.375235968181\\
1112	-11.3248161196345\\
1113	-11.6589637418977\\
1114	-10.2598221471735\\
1115	-10.0011949772456\\
1116	-9.54884054654235\\
1117	-9.25450171077159\\
1118	-8.56039625690551\\
1119	-8.65464119456396\\
1120	-7.85441344708397\\
1121	-8.24962849883627\\
1122	-12.1321964416286\\
1123	-14.3732010910405\\
1124	-14.3726885788233\\
1125	-9.29692485432906\\
1126	-10.8661474386356\\
1127	-17.8568619809405\\
1128	-14.7358625982474\\
1129	-15.6067763558517\\
1130	-15.0437787712015\\
1131	-11.3837978777592\\
1132	-11.7159510108718\\
1133	-9.83857848469597\\
1134	-12.1780013380437\\
1135	-17.0402769359264\\
1136	-22.2541819031894\\
1137	-17.7838147693712\\
1138	-14.0659646122213\\
1139	-14.9746320955589\\
1140	-15.7219904021258\\
1141	-13.8810278552844\\
1142	-14.4563418739124\\
1143	-11.5795838930148\\
1144	-11.1557710599343\\
1145	-9.65449601221076\\
1146	-9.14364793213053\\
1147	-8.80611691317768\\
1148	-11.9384702840116\\
1149	-18.6475413925073\\
1150	-14.0588999574288\\
1151	-10.8827745591877\\
1152	-10.2000791523651\\
1153	-10.2773003765822\\
1154	-9.66372419726213\\
1155	-9.69634330963072\\
1156	-6.04117439856213\\
1157	-6.89140996021888\\
1158	-7.61165428476284\\
1159	-8.20154110498333\\
1160	-7.95755638462288\\
1161	-8.24039553043372\\
1162	-7.92499894172056\\
1163	-7.98483615513942\\
1164	-7.92809033175627\\
1165	-3.58377830988138\\
1166	-7.87424035878507\\
1167	-18.1523147506136\\
1168	-18.0693494784105\\
1169	-11.2621918142186\\
1170	-11.7065902613165\\
1171	-10.0875178203592\\
1172	-10.2374631150515\\
1173	-5.68453064862024\\
1174	-8.99422144854134\\
1175	-10.7505332182209\\
1176	-14.1107384770664\\
1177	-25.44994408868\\
1178	-29.8940324006737\\
1179	-17.3854810344135\\
1180	-18.4669817666006\\
1181	-13.2458247732091\\
1182	-15.7478663087022\\
1183	-17.9211656023986\\
1184	-18.1686856863951\\
1185	-15.4705152206602\\
1186	-12.8227780991508\\
1187	-13.2253621358807\\
1188	-13.2939752064287\\
1189	-13.3258323151558\\
1190	-10.3000431735222\\
1191	-15.0746717331629\\
1192	-22.0725897571137\\
1193	-13.4446753865179\\
1194	-10.0831735749796\\
1195	-9.97969311271254\\
1196	-16.3725555722299\\
1197	-22.5121596799729\\
1198	-25.0848220365358\\
1199	-23.0718948825383\\
1200	-22.2706281925366\\
1201	-17.0753862074373\\
1202	-15.5530226250469\\
1203	-19.8061404453208\\
1204	-23.155308024928\\
1205	-12.8593910538541\\
1206	-11.8146285189846\\
1207	-12.9471196081624\\
1208	-14.8134375768543\\
1209	-9.78116285830697\\
1210	-10.0691834509901\\
1211	-13.1431640242777\\
1212	-10.2318515485838\\
1213	-8.9223612905937\\
1214	-9.86633101742723\\
1215	-9.43073799367403\\
1216	-10.8249143923131\\
1217	-14.5862019579986\\
1218	-13.6291081856129\\
1219	-11.4012190254498\\
1220	-10.258565967644\\
1221	-13.8525769561941\\
1222	-23.9168681122151\\
1223	-14.689104435316\\
1224	-10.1898596801534\\
1225	-9.92496909217698\\
1226	-10.7889066576038\\
1227	-10.8995779848246\\
1228	-8.72328909777798\\
1229	-8.82561427324367\\
1230	-9.34878072595961\\
1231	-11.6212407727348\\
1232	-11.8230208951959\\
1233	-8.88532370715075\\
1234	-13.8095864776155\\
1235	-17.6213381475452\\
1236	-14.1303001490401\\
1237	-13.2658830780365\\
1238	-13.2209441037367\\
1239	-17.1865608323346\\
1240	-12.6916547839969\\
1241	-10.9957468702096\\
1242	-8.07793311414275\\
1243	-10.1712437132092\\
1244	-11.5326791544626\\
1245	-10.3395633895811\\
1246	-9.19301204425847\\
1247	-10.5805923150674\\
1248	-11.250912846539\\
1249	-9.5237169659099\\
1250	-9.61486549411961\\
1251	-9.13168829316874\\
1252	-9.69441090444138\\
1253	-8.85903671638084\\
1254	-8.32627298856187\\
1255	-8.05940159952024\\
1256	-6.61120265567888\\
1257	-7.25446834924583\\
1258	-7.98995077955351\\
1259	-12.1276758527817\\
1260	-15.2004408227392\\
1261	-20.0021734265608\\
1262	-13.1178980621337\\
1263	-10.3783408785009\\
1264	-8.90967582950202\\
1265	-8.47710378897591\\
1266	-8.32693554035677\\
1267	-7.89999060963578\\
1268	-7.77332651008214\\
1269	-8.68877715524468\\
1270	-12.9024017788693\\
1271	-14.3631410268747\\
1272	-15.004923577173\\
1273	-11.6461527315375\\
1274	-11.2540067929958\\
1275	-10.1836575052057\\
1276	-9.30600076783458\\
1277	-7.9917437088618\\
1278	-6.40943528785579\\
1279	-4.00154262538149\\
1280	-5.28154786235473\\
1281	-9.36656982824047\\
1282	-9.27319900569682\\
1283	-9.78025510074485\\
1284	-15.7176161068057\\
1285	-16.1753925838702\\
1286	-10.524316109382\\
1287	-11.9682874512164\\
1288	-13.1852216577599\\
1289	-16.2645939641102\\
1290	-15.1004495658076\\
1291	-9.35547871197787\\
1292	-11.8807039801586\\
1293	-11.6072497176251\\
1294	-7.08411910581844\\
1295	-6.0471778392955\\
1296	-6.07601783222003\\
1297	-6.54098886889096\\
1298	-5.75272586793267\\
1299	-8.27339373209747\\
1300	-9.58908448495981\\
1301	-8.79502028519687\\
1302	-9.34135203461915\\
1303	-9.50688381200689\\
1304	-10.9550629213923\\
1305	-5.30056425092152\\
1306	-12.1107477419614\\
1307	-12.47323769834\\
1308	-11.724849461124\\
1309	-11.0683242594375\\
1310	-9.23795858581275\\
1311	-9.49009380694565\\
1312	-14.7377083316594\\
1313	-14.688778740476\\
1314	-11.4055649655843\\
1315	-15.0385502380987\\
1316	-14.4401510604337\\
1317	-12.4189519301731\\
1318	-14.8426719721142\\
1319	-14.3577351520576\\
1320	-12.578293390415\\
1321	-11.2419469608187\\
1322	-16.9631434427965\\
1323	-23.9260804073929\\
1324	-15.1429580064755\\
1325	-10.689047357198\\
1326	-11.4496323610192\\
1327	-12.4638634035794\\
1328	-14.0391706749963\\
1329	-15.3633613661355\\
1330	-12.1846276373724\\
1331	-11.4851738806733\\
1332	-15.7600145461718\\
1333	-15.6870608129834\\
1334	-12.8531632422244\\
1335	-9.79217892221232\\
1336	-11.7352164336753\\
1337	-19.2636463584374\\
1338	-18.1430682079044\\
1339	-15.0239529622936\\
1340	-10.5555679643651\\
1341	-10.6854122823083\\
1342	-11.5745881385851\\
1343	-9.30484444068065\\
1344	-10.4021051179084\\
1345	-9.11782561546784\\
1346	-7.68717636119982\\
1347	-3.53405988251029\\
1348	-9.69974660729099\\
1349	-12.3111208230721\\
1350	-9.79790094131158\\
1351	-8.80716177669592\\
1352	-9.79627779179305\\
1353	-8.81464014152289\\
1354	-7.98713679690582\\
1355	-7.60882538748683\\
1356	-9.13723566038993\\
1357	-7.48447130904151\\
1358	-9.97286439925983\\
1359	-14.8546185122866\\
1360	-9.85324475603738\\
1361	-9.81709147446259\\
1362	-8.2455476557036\\
1363	-7.71250704169393\\
1364	-8.04188176315324\\
1365	-7.78804442615933\\
1366	-19.5470175812011\\
1367	-13.3290558087042\\
1368	-14.7187559266595\\
1369	-19.5976567879011\\
1370	-21.1498097796266\\
1371	-15.0959188143624\\
1372	-12.2286643409091\\
1373	-12.2872057002226\\
1374	-12.6472775739692\\
1375	-13.6193319214501\\
1376	-15.424889800594\\
1377	-15.7183559815721\\
1378	-20.5601348975683\\
1379	-24.2497671057956\\
1380	-27.5707017823149\\
1381	-17.7646758552455\\
1382	-17.7130970848671\\
1383	-22.9351530624092\\
1384	-17.91942171347\\
1385	-18.339133506581\\
1386	-17.6398478046558\\
1387	-11.0225482987191\\
1388	-10.2396634369215\\
1389	-9.55617283944503\\
1390	-11.6090860229499\\
1391	-10.0933926931839\\
1392	-8.3075448260729\\
1393	-8.40012993024402\\
1395	-8.03120015860532\\
1396	-6.16350060701166\\
1398	-7.55821053235491\\
1399	-6.30021634077093\\
1400	-12.4281634502515\\
1401	-9.02259809920633\\
1402	-12.4137403837931\\
1403	-24.0736280270257\\
1404	-19.565802735151\\
1405	-23.5524457186796\\
1406	-15.5576823011133\\
1407	-16.5583163577016\\
1408	-13.198124338991\\
1409	-10.9391573203534\\
1410	-10.2178114615297\\
1411	-10.6722690164804\\
1412	-8.98996233721755\\
1413	-7.43169222035817\\
1414	-7.29793435661782\\
1415	-9.36315910556755\\
1416	-4.90463360460808\\
1417	-6.21313557584426\\
1418	-13.0097155358642\\
1419	-13.5294062366102\\
1420	-13.485476632623\\
1421	-12.3020459642885\\
1422	-13.2618005192717\\
1423	-13.39952807945\\
1424	-10.8806202570083\\
1425	-11.583026276892\\
1426	-10.2745095704099\\
1427	-11.0759278658354\\
1428	-10.9767884379323\\
1429	-9.88161669204442\\
1430	-9.69406770089563\\
1431	-14.8643031910717\\
1432	-11.2661035173367\\
1434	-9.49704566027162\\
1435	-9.47608396122905\\
1436	-8.44207728010679\\
1437	-7.49811346076581\\
1438	-7.50547764725366\\
1439	-9.73629486662253\\
1440	-11.385263288432\\
1441	-9.34025551877926\\
1442	-9.94962996055006\\
1443	-11.2227271911979\\
1444	-8.86039102879295\\
1445	-8.23480317389271\\
1446	-11.2046248710531\\
1447	-18.4801050929768\\
1448	-15.7967832218671\\
1449	-12.8079965402503\\
1450	-12.7952207967951\\
1451	-11.6063219104153\\
1452	-10.4592444695102\\
1453	-9.91105356664343\\
1454	-10.9486174725334\\
1455	-11.263328149779\\
1456	-8.75790722089164\\
1457	-8.30814671009261\\
1458	-11.8412999807638\\
1459	-11.6485088545255\\
1460	-13.3765273572349\\
1461	-12.7176122371854\\
1462	-16.8691156386521\\
1463	-23.9213540197536\\
1464	-24.2061216943391\\
1465	-15.2183683672997\\
1466	-11.501336999856\\
1467	-11.7105714978331\\
1468	-10.9971031996388\\
1469	-7.73117433345828\\
1470	-9.07944471154269\\
1471	-10.2912391236721\\
1472	-10.8252940962923\\
1473	-8.58675688036806\\
1474	-9.54829007420244\\
1475	-11.4457830211131\\
1476	-14.119774053537\\
1477	-14.5501506306707\\
1478	-22.5412799957862\\
1479	-19.6967891069246\\
1480	-13.2621999886082\\
1481	-10.8413419705098\\
1482	-12.0627120343181\\
1483	-12.2524582671633\\
1484	-14.4627697578326\\
1485	-10.6085411020754\\
1486	-10.7314721016178\\
1487	-11.5351009890805\\
1488	-9.2900681500787\\
1489	-9.53668525070429\\
1490	-16.2583441663012\\
1491	-11.5057186779854\\
1492	-10.3118024225598\\
1493	-18.4899548915214\\
1494	-16.0480524091729\\
1495	-12.2246459414414\\
1496	-17.6103204399658\\
1497	-18.0495472712018\\
1498	-10.8399972673094\\
1499	-10.201232965698\\
1500	-9.77595964003285\\
1501	-12.3742112682462\\
1502	-23.1664790681371\\
1503	-20.0366458450067\\
1504	-19.8702468716931\\
1505	-15.8768100335669\\
1506	-10.9299370376787\\
1507	-11.1083317749453\\
1508	-9.60792074945971\\
1509	-8.54020600029207\\
1510	-8.30625789249143\\
1511	-7.85216057444836\\
1512	-10.0899156473884\\
1513	-8.0138739686131\\
1514	-7.92993623508232\\
1515	-12.3026010598276\\
1516	-12.8831453559187\\
1517	-17.0454422388859\\
1518	-20.6035000905308\\
1519	-24.9722317896606\\
1520	-13.8420429974369\\
1521	-15.2215731484157\\
1522	-15.6132943270625\\
1523	-14.3874951752646\\
1524	-11.0340025555599\\
1525	-10.5931427626301\\
1526	-19.5441525228366\\
1527	-23.1530378985378\\
1528	-11.1356779767398\\
1529	-10.9260783395575\\
1530	-10.5282015820546\\
1531	-12.4910239730607\\
1532	-7.39080382588986\\
1533	-1.00236078963758\\
1534	-5.67485906800289\\
1535	-7.17152480577784\\
1536	-7.83084421641911\\
1537	-7.70372309670006\\
1538	-7.00684734997185\\
1539	-6.80222386728406\\
1540	-7.40947253743957\\
1541	-6.59229179197791\\
1542	-7.54093794860887\\
1543	-13.0368406107452\\
1544	-14.5005704771706\\
1545	-13.2197833834489\\
1546	-14.1785653158431\\
1547	-19.4698930058719\\
1548	-19.2304223872327\\
1549	-22.2923103431808\\
1550	-20.4798029501635\\
1551	-15.5131863857907\\
1552	-12.7856540131727\\
1553	-11.9937133988183\\
1554	-18.150398722034\\
1555	-21.8424986312727\\
1556	-13.8289281869868\\
1557	-10.5542008444577\\
1558	-10.462756447766\\
1559	-11.844122658079\\
1560	-7.10557144533323\\
1561	-8.73091391778075\\
1562	-2.90233958139538\\
1563	-7.10370433281219\\
1564	-8.50056187818222\\
1565	-8.03177162031693\\
1566	-8.80013191483431\\
1567	-9.19061515305589\\
1568	-5.02814475944911\\
1569	-6.00563102518595\\
1570	-5.90925192392206\\
1571	-6.75166559601098\\
1572	-6.17721776254507\\
1573	-3.65813552647273\\
1574	-12.5958746757019\\
1575	-18.699697019533\\
1576	-17.3352955926571\\
1577	-12.4289207380953\\
1578	-15.5673074048561\\
1579	-27.7511635677463\\
1581	-21.2297407852809\\
1582	-16.3745621775402\\
1583	-17.1294911087859\\
1584	-16.9684605509474\\
1585	-12.3488218971111\\
1586	-12.6479396155241\\
1587	-9.37096328597477\\
1588	-9.54757588454913\\
1589	-12.1284639448716\\
1590	-10.6675953024412\\
1591	-10.3197304185078\\
1592	-12.5913492185186\\
1593	-11.8957509272386\\
1594	-10.5399431711573\\
1595	-10.6569397654309\\
1596	-12.3133965410368\\
1597	-12.4926522319765\\
1598	-12.0600324025697\\
1599	-10.413450464504\\
1600	-10.7379753490497\\
1601	-9.86607781102566\\
1602	-14.0354286704248\\
1603	-5.71898022444634\\
1604	-6.45859358381676\\
1605	-7.62494976227572\\
1606	-12.4620152831747\\
1607	-1.40429614473533\\
1608	-3.14089006077393\\
1609	-6.81404042341251\\
1610	-7.77295925993258\\
1611	-8.32573692370602\\
1612	-10.387601076221\\
1613	-12.8802486690652\\
1614	-9.0140718825603\\
1615	-8.96408387090514\\
1617	-9.97615224730998\\
1618	-8.11998540932041\\
1619	-9.78425132067559\\
1620	-3.1815657339871\\
1621	-5.03601639594353\\
1622	-4.12711821578569\\
1623	-7.8089169628297\\
1624	-9.68371080255793\\
1625	-8.58733094504851\\
1626	-7.41587230921687\\
1627	-8.16982306854857\\
1628	-7.6968789110615\\
1629	-7.84913139982109\\
1630	-7.58142136033234\\
1631	-5.96010103558797\\
1632	-7.96462689558257\\
1633	-5.04243929666495\\
1634	-5.83686078820369\\
1635	-5.95365387437846\\
1636	-7.71822246052193\\
1637	-7.49904314880723\\
1638	-6.58430578125785\\
1639	-6.59119784507743\\
1640	-5.89084793606639\\
1641	-8.01144854702784\\
1642	-17.2039593609938\\
1643	-10.573416212334\\
1644	-9.57590246962991\\
1645	-10.8368360157222\\
1646	-10.2756113472185\\
1647	-12.2900167299433\\
1648	-9.21354801080838\\
1649	-8.70602404269289\\
1650	-9.15212533404952\\
1651	-9.2283208996746\\
1652	-7.04507454177065\\
1653	-7.86121310228464\\
1654	-8.33177716799355\\
1655	-8.94313819091462\\
1656	-10.0755150850657\\
1657	-8.59768066896254\\
1658	-8.94999939987815\\
1659	-10.0904898024205\\
1660	-11.0253671513651\\
1661	-12.4197094183542\\
1662	-20.9741041863274\\
1663	-15.3520679857709\\
1664	-10.4066847208085\\
1665	-10.856858902918\\
1666	-12.1584836498046\\
1667	-15.9970468181511\\
1668	-11.8223919016523\\
1669	-12.0241500792004\\
1670	-10.5911648532453\\
1671	-12.865835372799\\
1672	-8.35349618715441\\
1673	-7.3421706036338\\
1674	-11.8897803162683\\
1675	-9.91428843457948\\
1676	-11.9448693906204\\
1677	-13.039074930794\\
1678	-11.4368769825153\\
1679	-9.9748826510604\\
1680	-9.69710381112759\\
1681	-12.487881745569\\
1682	-11.2325878022996\\
1683	-10.2297079098403\\
1684	-10.9290884055949\\
1685	-9.95992929696763\\
1686	-8.61867064373564\\
1687	-8.98120759268227\\
1688	-7.65719917262345\\
1689	-10.7821927031416\\
1690	-13.2879434231415\\
1691	-14.0814257607481\\
1692	-13.0108372225327\\
1693	-11.6750664672945\\
1694	-10.4215829214065\\
1695	-8.77427843023884\\
1696	-9.66262125058597\\
1697	-10.8018967591909\\
1698	-14.2881614533258\\
1699	-14.964828116269\\
1700	-12.7950410693645\\
1701	-10.1894843911957\\
1702	-11.7213719359734\\
1703	-19.9426387973301\\
1704	-13.2377411890004\\
1705	-12.5252659696687\\
1706	-15.7812177868539\\
1707	-13.4755806803585\\
1708	-10.9948294606349\\
1709	-11.5468459896715\\
1710	-12.0110726161518\\
1711	-11.9000652281438\\
1712	-13.95265295124\\
1713	-12.2715944498093\\
1714	-12.0883647758578\\
1715	-12.2584952590062\\
1716	-12.3692444330959\\
1717	-11.00787906391\\
1718	-11.2868630828521\\
1719	-9.4147215485923\\
1720	-9.52625952828271\\
1721	-10.7953119197255\\
1722	-10.914854060559\\
1723	-8.98584099676759\\
1724	-11.8529871066112\\
1725	-10.4008347242584\\
1726	-3.56917117215426\\
1727	-9.58474822451876\\
1728	-17.6555606166032\\
1729	-12.9156160086343\\
1730	-10.2167010802812\\
1731	-10.5121379786501\\
1732	-11.478312493851\\
1733	-9.78422328877264\\
1734	-9.75699365363266\\
1735	-8.27788720256854\\
1736	-8.94591383777743\\
1737	-8.77102218441269\\
1739	-8.58849441183952\\
1741	-8.07288143171809\\
1742	-7.05900125091193\\
1743	-9.83023347272069\\
1744	-9.45554969218165\\
1745	-9.17816248384588\\
1746	-9.75106240207379\\
1747	-10.6622124665098\\
1748	-12.2844854518059\\
1749	-13.0380419050182\\
1750	-12.4950971505846\\
1751	-11.7549098039731\\
1752	-13.0875065363684\\
1753	-10.3369944972742\\
1754	-10.0736405981329\\
1755	-10.5725103430395\\
1756	-10.7498307994424\\
1757	-11.9155140739756\\
1758	-14.4725963726698\\
1759	-12.1416403421651\\
1760	-11.4689293083318\\
1761	-16.4129496417568\\
1762	-13.446300196631\\
1763	-12.3001698337748\\
1764	-11.191811198335\\
1765	-11.7362784796387\\
1766	-11.5847182302746\\
1767	-9.01995349249637\\
1768	-9.64123420179249\\
1769	-9.21546397637735\\
1770	-9.36659498760196\\
1771	-13.4893339663793\\
1772	-20.0199252698678\\
1773	-14.508754466804\\
1774	-15.2033276619277\\
1775	-14.8164749410371\\
1776	-11.877366799911\\
1777	-10.6431299509129\\
1778	-9.35682147946955\\
1779	-8.62858440142509\\
1780	-7.20123957612032\\
1781	-9.57352446071377\\
1782	-11.6917584013263\\
1783	-12.9708695128259\\
1784	-11.7505486811397\\
1785	-10.479264575167\\
1786	-12.9611069635696\\
1787	-17.4971316462847\\
1788	-18.1551512201004\\
1789	-13.3847648712792\\
1790	-22.0893319364932\\
1791	-18.3380437028902\\
1792	-10.5738552358675\\
1793	-11.3163592088983\\
1794	-9.65530186360206\\
1795	-10.8868794635484\\
1796	-15.1925713489195\\
1797	-18.3813377020722\\
1798	-13.961171084127\\
1799	-11.7432534750226\\
1800	-10.1145897670915\\
1801	-9.44727283285488\\
1802	-9.62204157640167\\
1803	-9.53895757051123\\
1804	-8.1186219422068\\
1805	-7.51177687301083\\
};
\addlegendentry{OSA predition}

\addplot [color=mycolor3, dotted, line width=2.0pt]
  table[row sep=crcr]{%
1006	-15.8689999999999\\
1007	-14.6479999999999\\
1008	-14.6479999999999\\
1009	-7.32400000000007\\
1010	-10.30741003042\\
1011	-11.3687055459736\\
1012	-12.5100303394918\\
1013	-11.2577265542154\\
1014	-11.1480847942689\\
1015	-15.5885485736151\\
1016	-14.077448355798\\
1017	-9.03250945447508\\
1018	-21.6958739624845\\
1019	-20.9798599539563\\
1020	-18.6428745810604\\
1021	-15.3568481526497\\
1022	-14.2764926125994\\
1023	-15.5243201075639\\
1024	-15.1414818114249\\
1025	-12.8113625580695\\
1026	-14.7915619135592\\
1027	-14.4770114118974\\
1028	-13.2438710916424\\
1029	-16.1408885209071\\
1030	-16.3246309543601\\
1031	-13.5226372316681\\
1032	-16.3402023760539\\
1033	-16.8487596529096\\
1034	-14.5795700797005\\
1035	-13.2531649238767\\
1036	-12.4976378213091\\
1037	-12.7859332926062\\
1038	-13.9590540655975\\
1039	-13.6004331277595\\
1040	-13.7772123888276\\
1041	-13.9940569323455\\
1042	-15.033756320468\\
1043	-18.6269458861132\\
1044	-17.5350437938844\\
1045	-13.31313200164\\
1046	-13.2611462955479\\
1047	-12.1061415556487\\
1048	-11.975595482643\\
1049	-11.7406042864632\\
1051	-11.231944327026\\
1052	-11.9543964202983\\
1053	-13.331226766237\\
1054	-12.8203678154944\\
1055	-16.7399887693823\\
1056	-14.7784889065595\\
1057	-11.9485627742024\\
1058	-12.3602125956472\\
1059	-11.6378003167447\\
1060	-12.3718029436466\\
1061	-11.1875028827019\\
1062	-11.2257882071553\\
1063	-10.4087622534751\\
1064	-10.4160245173659\\
1065	-10.3751696378818\\
1066	-9.93867751883477\\
1067	-10.3017900253506\\
1068	-12.5241645468043\\
1069	-13.7265564728418\\
1070	-14.3624442976377\\
1071	-14.4426940215074\\
1072	-15.2968748385097\\
1073	-13.7401656154852\\
1074	-13.3816059380613\\
1075	-12.960329940744\\
1076	-12.9111768871294\\
1077	-12.658047023825\\
1078	-16.3131430966762\\
1079	-27.0135506754914\\
1080	-26.7278450847589\\
1081	-23.9337969384367\\
1082	-16.5744674335383\\
1083	-22.4239324577272\\
1084	-29.1267752276465\\
1085	-27.8916909222446\\
1086	-21.1015983350851\\
1087	-26.6483807603274\\
1088	-38.4123338858692\\
1089	-27.1454002139949\\
1090	-20.4934662979852\\
1091	-16.30480233656\\
1092	-14.5144921425331\\
1093	-13.9296642051572\\
1094	-14.6646108667967\\
1095	-12.919552447888\\
1096	-11.3073098119964\\
1097	-11.4789242505863\\
1098	-11.2588659942664\\
1099	-13.4803383938704\\
1100	-13.6058514432732\\
1101	-13.2162263041621\\
1102	-14.7685677908339\\
1103	-16.1046890388839\\
1104	-20.3079509100151\\
1105	-18.0520822112776\\
1106	-18.7527635092672\\
1107	-16.5261399002582\\
1108	-14.6473159050106\\
1109	-16.0068338691804\\
1110	-13.8966511856017\\
1111	-13.0457676555131\\
1112	-13.885210270254\\
1113	-14.3533250879691\\
1114	-12.1567424479472\\
1115	-12.3304729497329\\
1116	-11.4661788312835\\
1117	-11.5461758022002\\
1118	-11.6046135956028\\
1119	-10.9422045010156\\
1120	-10.7795625864821\\
1121	-11.7819582158781\\
1122	-14.8312256408101\\
1123	-16.4069353472403\\
1124	-17.0166779623767\\
1125	-12.44794866296\\
1126	-13.9986158745908\\
1127	-21.6075139966181\\
1128	-17.2905377419352\\
1129	-17.9236358417988\\
1130	-18.6658473652953\\
1131	-13.2327321672894\\
1132	-12.8742474008302\\
1133	-13.2326137799239\\
1134	-14.545444300426\\
1135	-20.6095590734935\\
1136	-26.3184740834856\\
1137	-23.4746689416918\\
1138	-18.0155898532719\\
1139	-18.1899301251474\\
1140	-18.8157509100583\\
1141	-16.9878653936105\\
1142	-17.1096967880558\\
1143	-13.6502015391989\\
1144	-13.1981193300592\\
1145	-12.6490517011414\\
1146	-12.249892944222\\
1147	-12.4229010169313\\
1148	-15.211243296154\\
1149	-21.9776345785012\\
1150	-17.8469390358271\\
1151	-13.5234347736932\\
1152	-13.2086021859861\\
1153	-13.4214625899558\\
1154	-12.1290332439587\\
1155	-13.0001598154543\\
1156	-11.2728935222019\\
1157	-11.3459413858859\\
1158	-11.2533888258843\\
1159	-11.1914083094871\\
1160	-11.079860564932\\
1161	-10.8011336368179\\
1162	-10.6712229024154\\
1163	-11.3208801940998\\
1164	-11.5146159719798\\
1165	-8.9038411554227\\
1166	-12.2285827545022\\
1167	-20.8221754999727\\
1168	-22.1022185357945\\
1169	-14.7942720956203\\
1170	-13.8216250888288\\
1171	-13.6350791548025\\
1172	-14.4921807195224\\
1173	-11.303912009608\\
1174	-12.9986109219385\\
1175	-14.7458327920112\\
1176	-17.9785691836544\\
1177	-29.2371559196695\\
1178	-35.8629337467751\\
1179	-26.4645754428425\\
1180	-25.1192663253137\\
1181	-19.0104567223791\\
1183	-21.1614544708684\\
1184	-21.1923645468596\\
1185	-17.7433636166959\\
1186	-15.9606321117099\\
1187	-16.414497490571\\
1188	-16.2262437119398\\
1189	-16.7856232660804\\
1190	-13.4456579302309\\
1191	-17.9614078122881\\
1192	-24.3741587406644\\
1193	-16.7500982940742\\
1194	-13.0355509730398\\
1195	-13.2203233958712\\
1196	-19.4281328529758\\
1197	-24.9682778164226\\
1198	-28.8815005084866\\
1199	-28.4722832231796\\
1200	-27.1079908584632\\
1201	-21.9506977947049\\
1202	-19.7748523972596\\
1203	-23.150767852055\\
1204	-26.6273186428989\\
1205	-16.4690004327065\\
1206	-13.0198636786317\\
1207	-15.3600916837645\\
1208	-16.1978068352494\\
1209	-11.5326164166443\\
1210	-12.6173995767113\\
1211	-14.6829408560898\\
1212	-12.9670209307631\\
1213	-11.6762485599409\\
1214	-12.5597853096649\\
1215	-12.245460502972\\
1216	-13.5038672055234\\
1217	-17.2732622103238\\
1218	-16.2736924417804\\
1219	-13.1330173719641\\
1220	-13.241452773108\\
1221	-16.8492625589267\\
1222	-27.1690010201924\\
1223	-20.085128178712\\
1224	-13.828727445971\\
1225	-13.3346689691382\\
1226	-14.4832199302421\\
1227	-14.0301261121929\\
1228	-12.1331038881522\\
1229	-12.0554844959286\\
1230	-13.0237435222355\\
1231	-14.9936150113811\\
1232	-15.3544285349865\\
1233	-12.4944575967565\\
1234	-16.4945990526035\\
1235	-20.6749472682627\\
1236	-17.7114718368996\\
1237	-15.3482402249308\\
1238	-15.9778976728085\\
1239	-19.4576831402533\\
1240	-14.4128227503209\\
1241	-12.6202807439131\\
1242	-12.013281220291\\
1243	-12.5056324266202\\
1244	-14.327266382985\\
1245	-13.5583270822324\\
1246	-12.0919936101986\\
1247	-13.2827440833628\\
1248	-14.3153369406446\\
1249	-12.066886003214\\
1250	-12.4901670174531\\
1251	-12.3189983000386\\
1252	-11.9288381900333\\
1253	-12.1796433603445\\
1254	-11.0198275194509\\
1255	-10.9410712257356\\
1256	-10.6416229643455\\
1257	-10.439740140506\\
1258	-11.7291068965251\\
1259	-14.2260246308572\\
1260	-17.1529971150021\\
1261	-22.699284344566\\
1262	-15.6091346668825\\
1263	-13.1135360528551\\
1264	-13.1714056769792\\
1265	-12.7631414795817\\
1266	-12.7032488862512\\
1267	-11.4533340471276\\
1268	-11.3419122278979\\
1269	-11.3231835457491\\
1270	-15.6966561752949\\
1271	-16.4985596561257\\
1272	-18.506142426965\\
1273	-14.4808258377063\\
1274	-13.7414781470875\\
1275	-12.8548495226955\\
1276	-12.7941438915275\\
1277	-12.6354897804044\\
1278	-11.4788635846512\\
1279	-10.0234522496078\\
1280	-10.9383577911933\\
1281	-13.0382107897517\\
1282	-12.9621521719882\\
1283	-13.3441258832445\\
1284	-18.4793478338461\\
1285	-18.6483128295843\\
1286	-14.528415582123\\
1287	-15.2618491104568\\
1288	-16.7470773629998\\
1289	-19.4313520910739\\
1290	-18.3047549525993\\
1291	-12.9593378117281\\
1292	-13.4111187918445\\
1293	-14.9314215264553\\
1294	-13.2059338508557\\
1295	-11.0655414217383\\
1296	-10.846532246959\\
1297	-10.7572640236133\\
1298	-10.1791108782959\\
1299	-12.1052471747009\\
1300	-12.1897973193527\\
1301	-12.0302136804989\\
1302	-12.3666861447075\\
1303	-11.1086040002858\\
1304	-13.1388657365599\\
1305	-9.33892413288231\\
1306	-13.4288774996448\\
1307	-14.1396570759232\\
1308	-14.2258105011745\\
1309	-12.8472592612495\\
1310	-12.0705873561108\\
1311	-12.594185701988\\
1312	-16.8053813902518\\
1313	-16.7226538515779\\
1314	-13.4745319593362\\
1315	-17.6506385908704\\
1316	-16.6835807641942\\
1317	-14.7441306237856\\
1318	-16.4985787855003\\
1319	-16.2328662567309\\
1320	-14.4215643271716\\
1321	-13.2039686034734\\
1322	-18.8140802332189\\
1323	-27.3662416080874\\
1324	-19.8897642683964\\
1325	-13.5574244429663\\
1326	-13.8930732055551\\
1327	-14.9270994373098\\
1328	-16.3594039650927\\
1329	-17.4813839701526\\
1330	-14.6660536319789\\
1331	-14.4223072041891\\
1332	-17.8864287234057\\
1333	-18.042353363154\\
1334	-15.1819630213511\\
1335	-12.6292263536277\\
1336	-14.5215665075505\\
1337	-22.0331031014937\\
1338	-20.9609723037634\\
1339	-18.5934720669038\\
1340	-13.8778012169753\\
1341	-13.5857368850916\\
1342	-14.0162355738933\\
1343	-11.8812759558366\\
1344	-13.5286943667522\\
1345	-13.3678833396527\\
1346	-12.8674226426383\\
1347	-9.20885607479408\\
1348	-13.2871610705149\\
1349	-15.398605575803\\
1350	-13.1986838869291\\
1351	-12.4506230076972\\
1352	-13.2381546685147\\
1353	-11.6747881266076\\
1354	-11.4560979224211\\
1355	-10.9622439389345\\
1356	-10.8680934164713\\
1357	-10.2926239930916\\
1358	-12.2535109633625\\
1359	-15.8808819345645\\
1360	-11.9241166825941\\
1361	-11.7903270134759\\
1362	-11.3828146382489\\
1363	-11.0468472790478\\
1364	-10.742931108186\\
1365	-11.0916555486274\\
1366	-22.4998080039911\\
1367	-17.5775479382503\\
1368	-19.4493479312641\\
1369	-22.7314175827446\\
1370	-23.8676739205282\\
1371	-18.3716085724941\\
1372	-15.143100021204\\
1373	-15.0154719670529\\
1374	-15.5708602018469\\
1375	-16.2139733278245\\
1376	-17.681040158187\\
1377	-17.9170879430808\\
1378	-22.356705549332\\
1379	-25.7542558511004\\
1380	-30.1954714508577\\
1381	-20.9199890452785\\
1382	-20.9917436895346\\
1383	-24.7897734339183\\
1384	-20.4571156995262\\
1385	-21.5352104910537\\
1386	-20.4700243193845\\
1387	-13.1555693317223\\
1388	-12.4580533598264\\
1389	-12.4097651922293\\
1390	-14.1575199133654\\
1391	-12.9206241989873\\
1392	-11.6418966634869\\
1393	-11.8277357389516\\
1394	-11.3287149829573\\
1395	-11.3283148665519\\
1396	-10.4296712168482\\
1397	-10.6176448754425\\
1398	-10.4312986521636\\
1399	-10.8587084851556\\
1400	-14.200815463146\\
1401	-12.1363214236994\\
1402	-15.6125344305697\\
1403	-25.3289071985005\\
1404	-22.8441447188347\\
1405	-25.9300767436985\\
1406	-18.8625219836904\\
1407	-19.8294234833834\\
1408	-15.512473548968\\
1409	-13.3085247207819\\
1410	-13.4252295602932\\
1411	-12.7989849693224\\
1412	-11.960673189305\\
1413	-11.9169886898228\\
1414	-10.9664053549025\\
1415	-13.107528091379\\
1416	-10.6132561916377\\
1417	-10.6575116294198\\
1418	-15.3559520411859\\
1419	-17.3003833302046\\
1420	-16.9769856973646\\
1421	-15.7886408713618\\
1422	-16.4633568541262\\
1423	-15.6418036866066\\
1424	-13.6537974034532\\
1425	-13.6329438879716\\
1426	-12.9703464548784\\
1427	-14.4722464965357\\
1428	-12.6987512653957\\
1429	-11.9416456095435\\
1430	-12.8363807584574\\
1431	-16.6665609325771\\
1432	-13.4914461086696\\
1433	-12.4054347598517\\
1434	-12.0613377922239\\
1435	-12.3380275682146\\
1436	-11.4149140803574\\
1437	-11.257965603751\\
1438	-11.0298825831519\\
1439	-12.7877296552945\\
1440	-13.6313208665197\\
1441	-12.2303865268398\\
1442	-12.8863732255209\\
1443	-13.1206444941647\\
1444	-11.5829787552682\\
1445	-11.3940096342546\\
1446	-14.0038469792771\\
1447	-20.2936270948835\\
1448	-18.973423031076\\
1449	-15.9936597911808\\
1450	-15.4188054122249\\
1451	-14.1993635452836\\
1452	-13.7456854700361\\
1453	-12.6568417128397\\
1454	-13.7345342742478\\
1455	-13.1936065776983\\
1456	-11.7412227363784\\
1457	-12.0720977436172\\
1458	-14.20910455376\\
1459	-14.8136091164438\\
1460	-16.2311034790494\\
1461	-15.3150498238119\\
1462	-19.8361733308077\\
1463	-26.3785686277383\\
1464	-27.6034026051277\\
1465	-18.2587567402866\\
1466	-13.5030950016424\\
1467	-14.0575758117516\\
1468	-14.8779709221867\\
1469	-12.3794558957361\\
1470	-11.9713717583663\\
1471	-14.4264912430547\\
1472	-14.3242413312355\\
1473	-12.9122173012649\\
1474	-13.5786662350899\\
1475	-15.2073241306687\\
1476	-17.6956132473429\\
1477	-17.7183153638175\\
1478	-25.2568373411359\\
1479	-23.1078278561235\\
1480	-17.8171809797975\\
1481	-14.217644713465\\
1482	-14.908175844801\\
1483	-15.2476805183153\\
1484	-17.263821303231\\
1485	-13.6238439851863\\
1486	-13.4938344212712\\
1487	-13.7392908906074\\
1488	-11.7563532852871\\
1489	-12.2288560305217\\
1490	-17.4767741271398\\
1491	-13.7913439756576\\
1492	-13.1660602632473\\
1493	-20.8863238559607\\
1494	-18.3641278006012\\
1495	-15.6469831393235\\
1496	-20.2802836076924\\
1497	-20.5545232149602\\
1498	-14.2852084301737\\
1499	-12.6632437686424\\
1500	-12.7583429540744\\
1501	-15.9176197675481\\
1502	-26.2517034240066\\
1503	-25.1215473939758\\
1504	-23.6255154117573\\
1505	-19.6291942293881\\
1506	-15.0700917540403\\
1507	-14.5973269788917\\
1508	-13.3119402211469\\
1509	-12.8407312196068\\
1510	-12.08209347306\\
1511	-12.2982819806759\\
1512	-13.5279422094111\\
1513	-11.2801606055534\\
1514	-11.4292635634438\\
1515	-14.3779895093153\\
1516	-15.3548528396013\\
1518	-23.305365047541\\
1519	-28.6189844993964\\
1520	-18.7604104235706\\
1521	-17.5673200935087\\
1522	-18.5155414181593\\
1523	-17.2743566701358\\
1524	-13.7118384851356\\
1525	-14.2817203835039\\
1526	-21.6227796680744\\
1527	-25.3373445047109\\
1528	-14.677329345948\\
1529	-12.9357984125129\\
1530	-13.706036404272\\
1531	-17.3036902201052\\
1532	-14.7729919998869\\
1533	-10.4380791167891\\
1534	-11.8766087514477\\
1535	-13.2321655971164\\
1537	-11.4748062254496\\
1538	-11.0145621986283\\
1539	-10.6709331047809\\
1540	-10.3829071985936\\
1541	-10.0672246645322\\
1542	-10.9790147718961\\
1543	-15.4937908842244\\
1544	-17.1520939786221\\
1545	-15.7416704874975\\
1546	-16.9638552007218\\
1547	-21.6731375780842\\
1548	-22.1257083794758\\
1549	-25.485604349373\\
1550	-23.6239554097217\\
1551	-18.0827432892559\\
1552	-15.2221926159816\\
1553	-14.6358568614266\\
1554	-20.6786846570624\\
1555	-24.0157733811088\\
1556	-17.0016498191048\\
1557	-13.5724853382894\\
1558	-13.2809953323085\\
1559	-16.1772862171701\\
1560	-12.9338129154364\\
1561	-13.738193550299\\
1562	-9.56059965351665\\
1563	-11.4448728581899\\
1564	-11.8051371469869\\
1565	-11.3886939301051\\
1566	-11.2958531584998\\
1567	-12.0322484185524\\
1568	-9.3662866194029\\
1569	-9.32004274380211\\
1570	-9.17807560852566\\
1571	-9.47000038599754\\
1572	-9.77920793248836\\
1573	-7.79899846642388\\
1574	-14.7810293742295\\
1575	-22.3685684991656\\
1576	-21.9625447527549\\
1577	-16.3038298928218\\
1578	-19.7247112976975\\
1579	-30.6346156847248\\
1580	-28.7230111930226\\
1581	-25.3664325407321\\
1582	-20.2312116034968\\
1583	-21.2938864638863\\
1584	-21.3490414311555\\
1585	-15.7704998010076\\
1586	-15.0758367568055\\
1587	-12.4592807299648\\
1588	-12.6419681658679\\
1589	-15.4558721116946\\
1590	-12.9938818426522\\
1591	-13.1753162751036\\
1592	-14.1949492847821\\
1593	-13.7642255960347\\
1594	-13.3676679556838\\
1595	-13.4385635490687\\
1596	-14.5972584321305\\
1597	-15.5401768575766\\
1598	-14.5742004341591\\
1599	-12.7369786528507\\
1600	-13.4605772526804\\
1601	-11.3486492672866\\
1602	-16.5083044082608\\
1603	-11.986254579783\\
1604	-10.3281692459734\\
1605	-10.5796539353335\\
1606	-17.2610001052349\\
1607	-8.81892553161947\\
1608	-8.18884636601251\\
1609	-10.6950420101937\\
1610	-11.9181485805404\\
1611	-10.8762283401529\\
1612	-12.7742601761845\\
1613	-13.836247133029\\
1614	-11.4685991504887\\
1615	-11.8665070071279\\
1616	-10.6584608227813\\
1617	-12.0564749860591\\
1618	-11.7089777835329\\
1619	-13.3047274579897\\
1620	-8.67460760831159\\
1621	-9.15429796524904\\
1622	-8.23415598223119\\
1623	-10.7739619389429\\
1624	-11.3406022881181\\
1625	-10.4298861139328\\
1626	-10.056629539087\\
1627	-10.0851548082808\\
1628	-9.70085698492517\\
1629	-9.74429662691409\\
1630	-9.63132100942858\\
1631	-9.42886846162855\\
1632	-9.96897567161432\\
1633	-8.41363476262882\\
1634	-8.66675209715936\\
1635	-8.51251764454287\\
1636	-9.48964021887127\\
1637	-9.48129775139455\\
1638	-9.10449976346626\\
1639	-9.4095646527478\\
1640	-8.77882699289989\\
1641	-10.9977936484761\\
1642	-17.8635793569745\\
1643	-12.7802476883703\\
1644	-12.2875994677411\\
1645	-11.534062018301\\
1646	-12.7323319922978\\
1647	-13.5272470176142\\
1648	-11.4003590536022\\
1649	-11.2546843504856\\
1650	-10.9966905578112\\
1651	-11.7289691076419\\
1652	-10.5544372057432\\
1653	-10.7816668391408\\
1654	-10.0595134590487\\
1655	-11.1020074802016\\
1656	-10.4889882609016\\
1657	-10.3818562740867\\
1658	-10.3302928223209\\
1659	-12.1788923435461\\
1660	-12.099549353243\\
1661	-14.8226627588394\\
1662	-22.0198161462495\\
1663	-17.7219476148916\\
1664	-13.0738237475387\\
1665	-12.8785583194394\\
1666	-14.5320060985164\\
1667	-17.9912624399587\\
1668	-14.3891468971074\\
1669	-15.2598001234223\\
1670	-12.5190074073132\\
1671	-14.2334455941709\\
1672	-12.6063126588926\\
1673	-12.0834608226373\\
1674	-13.9342375104525\\
1675	-12.6896651659179\\
1676	-14.6696172790707\\
1677	-15.1677485533776\\
1678	-13.6196631794987\\
1679	-12.401237680245\\
1680	-12.6131454510178\\
1681	-14.8023231548193\\
1682	-13.3392199209225\\
1684	-12.7501333411467\\
1685	-11.8641716029904\\
1686	-11.7063253026179\\
1687	-11.2279773685591\\
1688	-11.0348695233017\\
1689	-13.3265353371412\\
1690	-15.4206525971358\\
1691	-15.5268449015923\\
1692	-14.3316626548865\\
1693	-12.5475293000429\\
1694	-12.1854025142982\\
1695	-11.769525321735\\
1696	-12.4399844829263\\
1697	-13.7282511828614\\
1698	-16.3909830309442\\
1699	-17.5511801323617\\
1700	-14.894032760049\\
1701	-12.3647126538622\\
1702	-14.2784573128056\\
1703	-21.9141860114362\\
1704	-15.5771073291444\\
1705	-14.7667514109971\\
1706	-16.7811318749939\\
1707	-14.8876637072165\\
1708	-13.4828200579536\\
1709	-13.741739753581\\
1710	-13.6407918197235\\
1711	-14.3352242950155\\
1712	-16.1784662368493\\
1713	-13.7241436432066\\
1714	-14.4051355702925\\
1715	-14.0155902302743\\
1716	-14.2424611165052\\
1717	-13.1359861915359\\
1718	-14.7683726157784\\
1719	-12.6698104375766\\
1720	-12.4997321546059\\
1721	-13.4025768208007\\
1722	-13.4150758668134\\
1723	-11.3105381683777\\
1724	-14.5588301607395\\
1725	-15.4445149309174\\
1726	-10.2332810226028\\
1727	-14.3367147989829\\
1728	-22.24201153938\\
1729	-18.7647244713448\\
1730	-14.9592883881912\\
1731	-14.9360714975946\\
1732	-14.8170734605969\\
1733	-13.6209113112436\\
1734	-12.6981963197884\\
1735	-11.8303104641341\\
1736	-11.8197781251249\\
1737	-11.6873075752749\\
1738	-11.7136016218844\\
1739	-11.3728630807691\\
1740	-11.0908544973709\\
1741	-11.0359511971515\\
1742	-10.1427383065034\\
1743	-13.2526566356205\\
1744	-11.8222301452774\\
1745	-12.0649113055847\\
1746	-11.8239993383581\\
1747	-13.2258713467429\\
1748	-13.8521590920311\\
1749	-15.3509678180724\\
1750	-13.7791757810919\\
1751	-14.2299965265986\\
1752	-14.4957632706055\\
1753	-11.6247137778096\\
1754	-12.5122491237148\\
1755	-11.9038138308822\\
1756	-12.2965779901381\\
1757	-13.2079899298828\\
1758	-15.3645898569507\\
1759	-13.2965562559621\\
1760	-14.1224325470732\\
1761	-17.7187769721538\\
1762	-15.4089453403544\\
1763	-13.5972878892376\\
1764	-12.9703487966922\\
1765	-13.75805965009\\
1766	-13.3771846076245\\
1767	-12.1829985626291\\
1768	-11.7707550982007\\
1769	-11.8353985752333\\
1770	-11.8836819958726\\
1771	-16.2933398442017\\
1772	-22.4836405321123\\
1773	-18.1324131731035\\
1774	-17.6018136202824\\
1775	-17.0462974883465\\
1776	-13.4669705115498\\
1777	-12.9663410373587\\
1778	-12.4095794392003\\
1779	-12.0050696819349\\
1780	-11.4352035794489\\
1781	-13.9963439016524\\
1782	-15.8453159497446\\
1783	-16.8370423428498\\
1784	-14.7329995803332\\
1785	-12.6719197646132\\
1786	-16.0482970855112\\
1787	-20.1875738359288\\
1788	-21.122040514766\\
1789	-16.7479348351562\\
1790	-24.9004069110415\\
1791	-21.2587039957639\\
1792	-13.3681566156745\\
1793	-13.2305007114514\\
1794	-13.1395538318741\\
1795	-14.5298592635963\\
1796	-18.2047863048538\\
1797	-22.0053564878131\\
1798	-17.2869470683825\\
1799	-14.9625182860179\\
1800	-12.8760481478525\\
1801	-13.1359713191466\\
1802	-12.4468343361032\\
1803	-12.7406826269319\\
1804	-11.5895261024052\\
1805	-11.3062682043735\\
};
\addlegendentry{MPO prediction}

\end{axis}

\begin{axis}[%
width=6.159cm,
height=1.831cm,
at={(8.104cm,5.085cm)},
scale only axis,
xmin=1000,
xmax=2000,
xlabel style={font=\color{white!15!black}},
xlabel={Sample index},
ymin=-404.053,
ymax=3.71940897576542,
ylabel style={font=\color{white!15!black}},
ylabel={$y(t)$},
axis background/.style={fill=white},
title style={font=\bfseries},
title={C6: RMSE(OSA) = 8.3465, RMSE(MPO) = 10.0061},
legend style={legend cell align=left, align=left, draw=white!15!black}
]
\addplot [color=mycolor1, line width=2.0pt]
  table[row sep=crcr]{%
1006	-185.547\\
1007	-219.727\\
1008	-169.678\\
1009	-86.6700000000001\\
1010	-107.422\\
1011	-102.539\\
1012	-126.953\\
1013	-103.76\\
1014	-41.5039999999999\\
1015	-31.7380000000001\\
1016	-28.076\\
1018	-157.471\\
1019	-167.236\\
1020	-174.561\\
1022	-74.463\\
1023	-158.691\\
1024	-111.084\\
1025	-83.008\\
1026	-142.822\\
1028	-103.76\\
1029	-173.34\\
1030	-164.795\\
1031	-124.512\\
1032	-180.664\\
1033	-179.443\\
1034	-142.822\\
1036	-89.1110000000001\\
1037	-108.643\\
1038	-137.939\\
1039	-130.615\\
1040	-136.719\\
1041	-140.381\\
1042	-151.367\\
1043	-213.623\\
1044	-191.65\\
1045	-126.953\\
1046	-115.967\\
1047	-64.6970000000001\\
1048	-69.5799999999999\\
1049	-96.4359999999999\\
1050	-74.463\\
1051	-78.125\\
1052	-111.084\\
1053	-128.174\\
1054	-119.629\\
1055	-190.43\\
1057	-81.787\\
1058	-63.4770000000001\\
1059	-85.4490000000001\\
1060	-101.318\\
1061	-51.27\\
1062	-58.5940000000001\\
1063	-81.787\\
1064	-65.9180000000001\\
1065	-56.152\\
1066	-74.463\\
1067	-86.6700000000001\\
1068	-139.16\\
1069	-130.615\\
1070	-163.574\\
1071	-146.484\\
1072	-170.898\\
1073	-139.16\\
1074	-133.057\\
1075	-115.967\\
1076	-111.084\\
1077	-111.084\\
1079	-280.762\\
1080	-291.748\\
1081	-280.762\\
1082	-175.781\\
1084	-303.955\\
1085	-317.383\\
1086	-239.258\\
1087	-313.721\\
1088	-404.053\\
1089	-289.307\\
1090	-238.037\\
1091	-158.691\\
1092	-122.07\\
1093	-102.539\\
1094	-129.395\\
1096	-61.0350000000001\\
1097	-59.8140000000001\\
1098	-75.684\\
1099	-118.408\\
1100	-107.422\\
1101	-111.084\\
1102	-146.484\\
1103	-152.588\\
1104	-207.52\\
1105	-167.236\\
1106	-208.74\\
1107	-148.926\\
1108	-130.615\\
1109	-151.367\\
1110	-108.643\\
1111	-98.877\\
1112	-122.07\\
1113	-123.291\\
1114	-85.4490000000001\\
1115	-95.2149999999999\\
1116	-67.1389999999999\\
1117	-76.904\\
1118	-81.787\\
1119	-57.373\\
1121	-93.9939999999999\\
1122	-151.367\\
1123	-152.588\\
1124	-170.898\\
1125	-97.6559999999999\\
1126	-140.381\\
1127	-211.182\\
1128	-161.133\\
1129	-186.768\\
1130	-191.65\\
1131	-112.305\\
1132	-75.684\\
1133	-107.422\\
1134	-123.291\\
1135	-205.078\\
1136	-252.686\\
1137	-252.686\\
1138	-184.326\\
1139	-189.209\\
1140	-167.236\\
1141	-163.574\\
1142	-162.354\\
1143	-108.643\\
1144	-96.4359999999999\\
1146	-78.125\\
1147	-87.8910000000001\\
1148	-133.057\\
1149	-203.857\\
1150	-161.133\\
1151	-109.863\\
1152	-92.7729999999999\\
1153	-89.1110000000001\\
1154	-52.49\\
1155	-41.5039999999999\\
1156	-47.607\\
1157	-90.3320000000001\\
1158	-67.1389999999999\\
1159	-78.125\\
1160	-85.4490000000001\\
1161	-80.566\\
1162	-54.932\\
1163	-37.8420000000001\\
1164	-30.518\\
1165	-54.932\\
1166	-119.629\\
1167	-172.119\\
1168	-194.092\\
1169	-136.719\\
1170	-91.5530000000001\\
1171	-64.6970000000001\\
1172	-43.9449999999999\\
1173	-98.877\\
1174	-91.5530000000001\\
1175	-137.939\\
1176	-168.457\\
1177	-275.879\\
1178	-316.162\\
1179	-260.01\\
1180	-249.023\\
1181	-159.912\\
1182	-185.547\\
1183	-195.313\\
1184	-212.402\\
1185	-158.691\\
1186	-144.043\\
1187	-142.822\\
1188	-129.395\\
1189	-156.25\\
1190	-109.863\\
1191	-195.313\\
1192	-234.375\\
1193	-175.781\\
1194	-104.98\\
1195	-111.084\\
1196	-192.871\\
1197	-234.375\\
1198	-289.307\\
1199	-303.955\\
1200	-302.734\\
1201	-228.271\\
1202	-205.078\\
1203	-235.596\\
1204	-275.879\\
1205	-170.898\\
1206	-104.98\\
1207	-152.588\\
1208	-122.07\\
1209	-79.346\\
1210	-109.863\\
1211	-122.07\\
1212	-95.2149999999999\\
1213	-75.684\\
1214	-101.318\\
1215	-80.566\\
1216	-115.967\\
1217	-162.354\\
1218	-147.705\\
1219	-100.098\\
1220	-101.318\\
1221	-157.471\\
1222	-261.23\\
1224	-118.408\\
1225	-85.4490000000001\\
1226	-101.318\\
1227	-90.3320000000001\\
1228	-67.1389999999999\\
1229	-76.904\\
1230	-92.7729999999999\\
1231	-120.85\\
1232	-134.277\\
1233	-89.1110000000001\\
1234	-156.25\\
1235	-179.443\\
1236	-170.898\\
1237	-140.381\\
1238	-150.146\\
1239	-202.637\\
1241	-53.711\\
1242	-78.125\\
1243	-85.4490000000001\\
1244	-119.629\\
1245	-101.318\\
1246	-74.463\\
1247	-118.408\\
1248	-112.305\\
1249	-79.346\\
1250	-102.539\\
1251	-90.3320000000001\\
1252	-80.566\\
1253	-95.2149999999999\\
1254	-54.932\\
1255	-68.3589999999999\\
1256	-47.607\\
1257	-51.27\\
1258	-106.201\\
1259	-122.07\\
1260	-167.236\\
1261	-219.727\\
1262	-142.822\\
1263	-83.008\\
1264	-63.4770000000001\\
1265	-62.2560000000001\\
1266	-91.5530000000001\\
1267	-54.932\\
1269	-85.4490000000001\\
1270	-150.146\\
1271	-133.057\\
1272	-190.43\\
1273	-124.512\\
1274	-117.188\\
1275	-56.152\\
1276	-62.2560000000001\\
1277	-34.1800000000001\\
1278	-46.3869999999999\\
1279	-51.27\\
1280	-95.2149999999999\\
1281	-100.098\\
1282	-117.188\\
1283	-123.291\\
1284	-197.754\\
1285	-170.898\\
1286	-128.174\\
1287	-153.809\\
1288	-163.574\\
1289	-213.623\\
1292	-53.711\\
1293	-40.2829999999999\\
1294	-41.5039999999999\\
1295	-52.49\\
1296	-54.932\\
1297	-51.27\\
1298	-70.8009999999999\\
1299	-108.643\\
1300	-98.877\\
1301	-103.76\\
1302	-118.408\\
1303	-69.5799999999999\\
1304	-36.6210000000001\\
1306	-125.732\\
1307	-125.732\\
1308	-142.822\\
1309	-118.408\\
1310	-87.8910000000001\\
1311	-123.291\\
1312	-172.119\\
1313	-172.119\\
1314	-120.85\\
1315	-207.52\\
1316	-155.029\\
1317	-151.367\\
1318	-173.34\\
1319	-172.119\\
1320	-130.615\\
1321	-112.305\\
1323	-266.113\\
1324	-202.637\\
1325	-124.512\\
1326	-118.408\\
1327	-115.967\\
1328	-150.146\\
1329	-173.34\\
1330	-125.732\\
1331	-134.277\\
1332	-177.002\\
1333	-192.871\\
1334	-120.85\\
1335	-97.6559999999999\\
1336	-130.615\\
1337	-218.506\\
1338	-195.313\\
1339	-203.857\\
1340	-115.967\\
1342	-101.318\\
1343	-63.4770000000001\\
1344	-41.5039999999999\\
1345	-34.1800000000001\\
1346	-29.297\\
1347	-78.125\\
1348	-108.643\\
1349	-131.836\\
1350	-93.9939999999999\\
1352	-119.629\\
1353	-72.021\\
1354	-79.346\\
1355	-73.242\\
1356	-54.932\\
1357	-75.684\\
1358	-120.85\\
1359	-156.25\\
1360	-95.2149999999999\\
1361	-73.242\\
1362	-58.5940000000001\\
1363	-74.463\\
1364	-45.1659999999999\\
1366	-195.313\\
1367	-163.574\\
1368	-216.064\\
1369	-241.699\\
1370	-249.023\\
1371	-181.885\\
1372	-131.836\\
1374	-129.395\\
1376	-184.326\\
1377	-181.885\\
1378	-247.803\\
1379	-270.996\\
1380	-328.369\\
1381	-217.285\\
1382	-238.037\\
1383	-264.893\\
1384	-205.078\\
1385	-239.258\\
1386	-200.195\\
1387	-113.525\\
1388	-74.463\\
1389	-79.346\\
1390	-117.188\\
1391	-97.6559999999999\\
1392	-64.6970000000001\\
1393	-90.3320000000001\\
1394	-63.4770000000001\\
1395	-48.828\\
1396	-64.6970000000001\\
1397	-72.021\\
1398	-48.828\\
1399	-100.098\\
1400	-135.498\\
1401	-97.6559999999999\\
1403	-241.699\\
1404	-220.947\\
1405	-279.541\\
1406	-187.988\\
1407	-206.299\\
1408	-134.277\\
1409	-85.4490000000001\\
1410	-108.643\\
1411	-87.8910000000001\\
1412	-63.4770000000001\\
1413	-92.7729999999999\\
1414	-50.049\\
1415	-32.9590000000001\\
1416	-43.9449999999999\\
1417	-98.877\\
1418	-125.732\\
1419	-157.471\\
1420	-152.588\\
1421	-146.484\\
1422	-168.457\\
1423	-135.498\\
1424	-108.643\\
1425	-112.305\\
1426	-89.1110000000001\\
1427	-146.484\\
1428	-96.4359999999999\\
1429	-80.566\\
1431	-164.795\\
1432	-123.291\\
1433	-93.9939999999999\\
1434	-80.566\\
1435	-91.5530000000001\\
1436	-61.0350000000001\\
1437	-61.0350000000001\\
1438	-87.8910000000001\\
1439	-109.863\\
1440	-124.512\\
1441	-96.4359999999999\\
1442	-128.174\\
1443	-114.746\\
1444	-61.0350000000001\\
1445	-69.5799999999999\\
1446	-135.498\\
1447	-190.43\\
1448	-186.768\\
1449	-157.471\\
1450	-152.588\\
1451	-114.746\\
1452	-109.863\\
1453	-87.8910000000001\\
1454	-125.732\\
1455	-117.188\\
1456	-70.8009999999999\\
1457	-107.422\\
1458	-126.953\\
1459	-150.146\\
1460	-164.795\\
1461	-163.574\\
1462	-222.168\\
1463	-267.334\\
1464	-302.734\\
1465	-190.43\\
1466	-106.201\\
1467	-68.3589999999999\\
1468	-45.1659999999999\\
1469	-70.8009999999999\\
1470	-64.6970000000001\\
1471	-119.629\\
1472	-102.539\\
1473	-93.9939999999999\\
1474	-124.512\\
1475	-145.264\\
1476	-172.119\\
1477	-183.105\\
1478	-258.789\\
1479	-231.934\\
1481	-120.85\\
1482	-120.85\\
1483	-123.291\\
1484	-157.471\\
1485	-108.643\\
1486	-113.525\\
1487	-107.422\\
1488	-58.5940000000001\\
1490	-152.588\\
1491	-107.422\\
1492	-115.967\\
1493	-216.064\\
1494	-159.912\\
1495	-156.25\\
1496	-219.727\\
1497	-206.299\\
1498	-129.395\\
1499	-85.4490000000001\\
1500	-85.4490000000001\\
1501	-146.484\\
1502	-236.816\\
1503	-246.582\\
1504	-251.465\\
1505	-192.871\\
1506	-122.07\\
1507	-103.76\\
1508	-72.021\\
1509	-69.5799999999999\\
1510	-58.5940000000001\\
1511	-89.1110000000001\\
1512	-102.539\\
1513	-58.5940000000001\\
1515	-126.953\\
1516	-145.264\\
1517	-185.547\\
1519	-281.982\\
1520	-197.754\\
1521	-172.119\\
1522	-178.223\\
1523	-156.25\\
1524	-103.76\\
1525	-128.174\\
1526	-214.844\\
1527	-246.582\\
1528	-141.602\\
1529	-79.346\\
1531	-28.076\\
1532	-35.4000000000001\\
1533	-64.6970000000001\\
1534	-74.463\\
1535	-102.539\\
1536	-81.787\\
1537	-64.6970000000001\\
1538	-62.2560000000001\\
1539	-61.0350000000001\\
1540	-54.932\\
1541	-65.9180000000001\\
1542	-102.539\\
1543	-148.926\\
1544	-157.471\\
1545	-155.029\\
1546	-183.105\\
1547	-225.83\\
1548	-225.83\\
1549	-270.996\\
1550	-247.803\\
1552	-125.732\\
1553	-122.07\\
1554	-205.078\\
1555	-236.816\\
1556	-166.016\\
1558	-53.711\\
1559	-40.2829999999999\\
1560	-42.7249999999999\\
1561	-25.635\\
1562	-61.0350000000001\\
1563	-84.229\\
1564	-89.1110000000001\\
1565	-78.125\\
1566	-47.607\\
1567	-35.4000000000001\\
1568	-52.49\\
1569	-56.152\\
1570	-62.2560000000001\\
1571	-47.607\\
1572	-37.8420000000001\\
1573	-69.5799999999999\\
1574	-164.795\\
1576	-217.285\\
1577	-147.705\\
1578	-208.74\\
1579	-291.748\\
1580	-261.23\\
1581	-273.438\\
1582	-200.195\\
1583	-202.637\\
1584	-212.402\\
1585	-139.16\\
1586	-133.057\\
1587	-80.566\\
1588	-75.684\\
1589	-135.498\\
1590	-93.9939999999999\\
1592	-115.967\\
1593	-111.084\\
1594	-113.525\\
1595	-120.85\\
1596	-136.719\\
1597	-147.705\\
1598	-125.732\\
1599	-92.7729999999999\\
1600	-120.85\\
1601	-57.373\\
1602	-26.855\\
1603	-46.3869999999999\\
1604	-73.242\\
1605	-37.8420000000001\\
1606	-17.0899999999999\\
1607	-39.0630000000001\\
1608	-70.8009999999999\\
1609	-84.229\\
1610	-104.98\\
1611	-87.8910000000001\\
1612	-142.822\\
1613	-123.291\\
1614	-83.008\\
1615	-125.732\\
1616	-70.8009999999999\\
1617	-56.152\\
1619	-24.414\\
1620	-48.828\\
1621	-36.6210000000001\\
1622	-65.9180000000001\\
1623	-111.084\\
1624	-106.201\\
1625	-76.904\\
1626	-86.6700000000001\\
1627	-64.6970000000001\\
1628	-92.7729999999999\\
1629	-67.1389999999999\\
1630	-63.4770000000001\\
1631	-58.5940000000001\\
1632	-43.9449999999999\\
1633	-57.373\\
1634	-51.27\\
1635	-90.3320000000001\\
1636	-83.008\\
1637	-67.1389999999999\\
1638	-64.6970000000001\\
1639	-51.27\\
1640	-61.0350000000001\\
1641	-135.498\\
1642	-179.443\\
1643	-123.291\\
1644	-114.746\\
1645	-80.566\\
1646	-136.719\\
1647	-125.732\\
1648	-80.566\\
1649	-102.539\\
1650	-76.904\\
1651	-108.643\\
1652	-64.6970000000001\\
1653	-64.6970000000001\\
1654	-80.566\\
1655	-104.98\\
1656	-74.463\\
1657	-81.787\\
1658	-85.4490000000001\\
1659	-135.498\\
1660	-115.967\\
1662	-231.934\\
1663	-183.105\\
1664	-117.188\\
1665	-86.6700000000001\\
1666	-135.498\\
1667	-173.34\\
1668	-133.057\\
1669	-163.574\\
1670	-98.877\\
1671	-50.049\\
1672	-52.49\\
1673	-119.629\\
1674	-107.422\\
1675	-100.098\\
1676	-156.25\\
1677	-146.484\\
1678	-133.057\\
1679	-90.3320000000001\\
1680	-114.746\\
1681	-148.926\\
1682	-120.85\\
1683	-125.732\\
1684	-112.305\\
1685	-80.566\\
1686	-97.6559999999999\\
1687	-70.8009999999999\\
1688	-79.346\\
1689	-135.498\\
1690	-156.25\\
1691	-164.795\\
1692	-146.484\\
1693	-104.98\\
1694	-72.021\\
1695	-85.4490000000001\\
1696	-101.318\\
1698	-159.912\\
1699	-190.43\\
1700	-146.484\\
1701	-84.229\\
1703	-222.168\\
1704	-155.029\\
1705	-155.029\\
1706	-181.885\\
1707	-137.939\\
1708	-113.525\\
1709	-129.395\\
1710	-115.967\\
1711	-131.836\\
1712	-167.236\\
1713	-125.732\\
1714	-146.484\\
1715	-137.939\\
1716	-141.602\\
1717	-109.863\\
1718	-144.043\\
1719	-102.539\\
1720	-106.201\\
1721	-119.629\\
1722	-115.967\\
1723	-58.5940000000001\\
1724	-35.4000000000001\\
1725	-28.076\\
1726	-47.607\\
1727	-122.07\\
1728	-161.133\\
1729	-161.133\\
1730	-101.318\\
1731	-119.629\\
1732	-104.98\\
1733	-92.7729999999999\\
1734	-58.5940000000001\\
1735	-81.787\\
1736	-81.787\\
1737	-80.566\\
1738	-95.2149999999999\\
1739	-79.346\\
1740	-75.684\\
1741	-50.049\\
1742	-72.021\\
1743	-139.16\\
1744	-96.4359999999999\\
1745	-119.629\\
1746	-107.422\\
1747	-145.264\\
1748	-135.498\\
1749	-186.768\\
1750	-137.939\\
1751	-151.367\\
1752	-155.029\\
1753	-72.021\\
1754	-131.836\\
1755	-95.2149999999999\\
1756	-113.525\\
1757	-128.174\\
1758	-166.016\\
1759	-123.291\\
1760	-158.691\\
1761	-201.416\\
1762	-164.795\\
1763	-142.822\\
1764	-113.525\\
1765	-136.719\\
1766	-113.525\\
1768	-81.787\\
1769	-96.4359999999999\\
1770	-83.008\\
1771	-172.119\\
1772	-225.83\\
1773	-191.65\\
1774	-187.988\\
1775	-181.885\\
1776	-101.318\\
1777	-89.1110000000001\\
1778	-72.021\\
1779	-62.2560000000001\\
1780	-68.3589999999999\\
1781	-115.967\\
1782	-146.484\\
1783	-167.236\\
1784	-141.602\\
1785	-93.9939999999999\\
1786	-169.678\\
1787	-201.416\\
1788	-224.609\\
1789	-178.223\\
1790	-288.086\\
1791	-209.961\\
1792	-117.188\\
1793	-84.229\\
1794	-69.5799999999999\\
1795	-125.732\\
1796	-162.354\\
1797	-218.506\\
1798	-166.016\\
1799	-126.953\\
1800	-69.5799999999999\\
1801	-72.021\\
1802	-78.125\\
1803	-87.8910000000001\\
1804	-56.152\\
1805	-70.8009999999999\\
};
\addlegendentry{True output}

\addplot [color=mycolor2, dashed, line width=2.0pt]
  table[row sep=crcr]{%
1006	-171.60756735679\\
1007	-207.372972410161\\
1008	-174.960700080454\\
1009	-86.1823724162059\\
1010	-110.995349893054\\
1011	-112.023314091797\\
1012	-130.286760047934\\
1013	-105.521457372687\\
1014	-31.3009211090216\\
1015	-14.2758473006229\\
1016	-17.9044403620314\\
1017	-102.248068986451\\
1018	-163.036122603209\\
1019	-161.191278483162\\
1020	-160.928268060267\\
1021	-131.985295260273\\
1022	-72.9806388441045\\
1023	-164.063749985325\\
1024	-114.542853108525\\
1025	-84.3001681052083\\
1026	-143.819111393111\\
1027	-119.111689982027\\
1028	-110.593550378328\\
1029	-162.632863195516\\
1030	-159.86140893243\\
1031	-123.280081345\\
1032	-173.069205017179\\
1033	-171.181174413626\\
1036	-96.1676893660672\\
1037	-107.842602842546\\
1038	-132.123227286532\\
1039	-125.008812171101\\
1040	-134.004555266186\\
1042	-148.069464793165\\
1043	-190.914077541743\\
1044	-184.20765536164\\
1045	-133.838571011551\\
1046	-122.769688606768\\
1047	-63.1680338191484\\
1048	-74.1423583827795\\
1049	-97.3808794557792\\
1050	-77.2013846939462\\
1051	-78.3115139394336\\
1052	-107.107139669424\\
1053	-127.475584347494\\
1054	-117.772307925938\\
1055	-173.023027508628\\
1056	-147.574105845098\\
1057	-86.3437201342206\\
1058	-66.3005347209328\\
1059	-89.0935902400213\\
1060	-102.862631529143\\
1061	-54.101214748802\\
1062	-52.0856294605719\\
1063	-89.0150319778511\\
1064	-66.7441862989297\\
1065	-56.9302768571335\\
1066	-79.4786112198153\\
1067	-86.6834765090168\\
1068	-128.573201333524\\
1069	-136.736464748564\\
1070	-147.208130303548\\
1071	-143.109015556528\\
1072	-166.171113989256\\
1073	-135.820405668215\\
1074	-141.248350723683\\
1075	-115.834481763208\\
1076	-115.288495826598\\
1077	-114.506334953746\\
1079	-256.647343694504\\
1080	-268.158224847341\\
1081	-262.256214482121\\
1082	-189.264294819265\\
1083	-233.524326057332\\
1084	-287.20326615306\\
1085	-290.159692962746\\
1086	-249.0635120674\\
1087	-281.692727180445\\
1088	-374.43012309209\\
1089	-319.707954776764\\
1090	-255.577067846775\\
1091	-156.741932613529\\
1092	-126.384580931613\\
1093	-110.302479075114\\
1094	-125.717550459032\\
1095	-102.323907884601\\
1096	-57.7341314747478\\
1097	-48.792317424078\\
1098	-80.4611846824305\\
1099	-117.587282370857\\
1100	-113.113300221172\\
1101	-113.364287678256\\
1102	-143.975618763065\\
1103	-149.359754213597\\
1104	-199.511201146069\\
1105	-177.494870303361\\
1106	-191.182351886397\\
1108	-134.600757275002\\
1109	-157.789366867725\\
1110	-114.683241206739\\
1111	-110.833571107961\\
1112	-120.246901337074\\
1113	-131.689759748769\\
1114	-90.6544538106116\\
1115	-102.303524389067\\
1116	-70.1999240658456\\
1117	-85.5980283767328\\
1118	-85.9671000733904\\
1119	-62.3082795925441\\
1120	-71.80369230372\\
1121	-97.0699024356288\\
1122	-147.506708994321\\
1123	-152.152079226968\\
1124	-163.681604523938\\
1125	-102.167338168934\\
1126	-136.641486064431\\
1127	-206.438065115638\\
1128	-165.169515409515\\
1129	-185.172468570194\\
1130	-183.18224555349\\
1131	-122.252563560329\\
1132	-83.1607033895111\\
1133	-107.643760490254\\
1134	-126.425747902242\\
1135	-191.776231936178\\
1136	-234.866982126499\\
1137	-238.707484955929\\
1138	-194.358952944519\\
1139	-188.06721222856\\
1140	-179.931143715754\\
1141	-173.725105141008\\
1142	-162.609828824815\\
1143	-113.668715307522\\
1144	-97.1568934606894\\
1145	-93.4406057260273\\
1146	-87.3739708476257\\
1147	-92.8675967833324\\
1148	-129.071628209816\\
1149	-196.662784146023\\
1150	-162.613478820537\\
1151	-115.456991879055\\
1152	-102.700696610724\\
1153	-94.3079414154411\\
1154	-52.6908917222938\\
1155	-34.3910191427019\\
1156	-44.0025802660605\\
1157	-96.1551967948546\\
1158	-73.983765498675\\
1159	-80.9045384756546\\
1160	-82.0479932759977\\
1161	-81.6501954425814\\
1162	-62.5939007144398\\
1163	-35.8717457526664\\
1164	-23.6236836028502\\
1165	-57.4937967023991\\
1166	-125.910471965547\\
1167	-172.639998011784\\
1168	-186.78510141708\\
1169	-145.574739591317\\
1170	-93.902502583454\\
1171	-66.9874576530995\\
1172	-35.3391658088319\\
1173	-102.77727009567\\
1174	-93.9035494247942\\
1175	-126.899955467471\\
1176	-172.554592118505\\
1177	-256.746498719573\\
1178	-318.617592517548\\
1179	-265.71707881409\\
1180	-252.83524829252\\
1181	-170.217673255881\\
1182	-176.427008910613\\
1183	-204.426900363193\\
1184	-199.913322723565\\
1185	-169.555426664912\\
1186	-146.462859294929\\
1187	-154.446183664432\\
1188	-134.504029793194\\
1189	-155.078054454768\\
1190	-120.668986436425\\
1191	-187.635786618039\\
1192	-215.762541435262\\
1193	-185.740691305797\\
1194	-109.7813352417\\
1195	-112.89133060463\\
1196	-192.244848294263\\
1197	-220.996325396744\\
1199	-286.610575916957\\
1200	-292.569005705594\\
1201	-238.801354632643\\
1202	-218.131503416166\\
1203	-229.959349272988\\
1204	-264.523977004674\\
1206	-103.84624290981\\
1207	-154.680799964568\\
1208	-128.065735161669\\
1209	-91.08777023467\\
1210	-104.817906257511\\
1211	-129.940234100648\\
1212	-102.933749851241\\
1213	-82.9835964507872\\
1214	-103.869962873596\\
1215	-91.3043894057328\\
1216	-117.649579755576\\
1217	-168.066805209796\\
1219	-114.326628027773\\
1220	-109.424411316551\\
1221	-156.013568514267\\
1222	-245.308731992177\\
1223	-195.305372266798\\
1224	-125.786278815001\\
1225	-84.9002379706615\\
1226	-106.404757355543\\
1227	-107.656759716062\\
1228	-71.308864455126\\
1229	-79.060473013431\\
1230	-101.859418183679\\
1232	-134.342440155234\\
1233	-92.9652612405623\\
1234	-157.836614296345\\
1235	-177.667338505431\\
1236	-162.393212003096\\
1237	-155.329390715426\\
1238	-144.834849341862\\
1239	-200.049554479291\\
1240	-130.702813420889\\
1241	-48.5188436646533\\
1242	-75.6237665504646\\
1243	-92.0762322369712\\
1244	-123.776764701007\\
1245	-107.435363853692\\
1246	-83.0435980332591\\
1247	-121.029323843135\\
1248	-116.668577215531\\
1249	-89.0592977961815\\
1250	-101.697968586044\\
1251	-94.6318533533026\\
1252	-91.4619067851654\\
1253	-99.7532086283559\\
1254	-55.976001769917\\
1255	-75.722676208844\\
1256	-48.2296209226556\\
1257	-58.8437225240191\\
1258	-113.204717424319\\
1259	-114.993874944\\
1260	-165.280913016054\\
1261	-205.962399976436\\
1262	-158.186773136123\\
1263	-89.9563547483629\\
1264	-54.8320047503692\\
1266	-95.3720010852107\\
1267	-56.0018107095029\\
1268	-68.5101991595425\\
1269	-89.5938637143925\\
1270	-149.319324622138\\
1271	-136.652236190324\\
1272	-179.965058482017\\
1273	-125.917125044516\\
1274	-126.078524265512\\
1275	-48.6912146082666\\
1276	-61.643669907045\\
1277	-31.2789154635775\\
1278	-45.0793940327628\\
1279	-50.1020521770292\\
1280	-99.1486735298968\\
1281	-101.507125901656\\
1283	-126.033914563285\\
1284	-186.354799632089\\
1285	-188.487568812567\\
1286	-133.138892793903\\
1287	-150.092378411877\\
1288	-161.419373043033\\
1289	-198.394779321282\\
1290	-165.331941256843\\
1291	-116.992365678863\\
1292	-40.8417857679776\\
1293	-28.2705411783743\\
1294	-38.3119466310359\\
1295	-54.4691402988522\\
1296	-58.7042309192384\\
1297	-50.1315909199043\\
1298	-71.860486956125\\
1299	-113.616150347283\\
1300	-104.078476997485\\
1301	-97.281962952987\\
1302	-118.891048604297\\
1304	-32.3368281983921\\
1305	-90.1339571281092\\
1306	-128.585212771559\\
1307	-115.39053016495\\
1308	-134.228843072427\\
1309	-126.339512762541\\
1310	-90.1546688026867\\
1311	-121.744618149766\\
1312	-167.262785248922\\
1313	-162.391433481384\\
1314	-128.718290149912\\
1315	-192.304106322933\\
1316	-161.513861114372\\
1317	-152.721053874013\\
1318	-171.069342924754\\
1319	-163.368848430852\\
1320	-139.059768528284\\
1321	-120.911491434619\\
1322	-175.585510626661\\
1323	-248.997591582228\\
1324	-206.287844067891\\
1325	-131.474717495517\\
1326	-120.869184689813\\
1327	-123.661834576401\\
1328	-148.94894537851\\
1329	-162.006710467626\\
1330	-136.884981417465\\
1331	-136.836503634947\\
1332	-170.641232004077\\
1333	-179.368997630635\\
1334	-132.071447282708\\
1335	-106.426457186634\\
1336	-135.910018615418\\
1337	-197.768071566046\\
1338	-196.361556133985\\
1339	-191.569249374759\\
1340	-122.855750364011\\
1341	-112.013394957334\\
1342	-116.97631527707\\
1343	-62.247558955695\\
1344	-32.3337304641125\\
1345	-23.4505327859472\\
1346	-23.2920969601932\\
1347	-82.4735962821476\\
1348	-111.550247289127\\
1349	-125.886267582702\\
1350	-102.819807762831\\
1351	-106.483284841543\\
1352	-117.387857258674\\
1353	-77.4834811705718\\
1354	-78.9396804119399\\
1355	-79.1744455326284\\
1356	-56.7381688655967\\
1357	-80.1115920251534\\
1359	-149.46214202716\\
1360	-98.0665562529359\\
1361	-82.838207661568\\
1362	-57.7932443171744\\
1363	-79.2883278913898\\
1364	-49.2543121577592\\
1366	-186.663653047807\\
1367	-159.724948172568\\
1369	-235.054265813664\\
1370	-238.182734694308\\
1371	-196.569353811115\\
1372	-141.448122863608\\
1373	-129.614985731915\\
1374	-139.200718721464\\
1375	-152.442314057459\\
1376	-170.007250263561\\
1377	-174.019087960666\\
1378	-224.786907203761\\
1379	-257.997251243094\\
1380	-304.158222948087\\
1381	-232.038820582253\\
1382	-238.883031330399\\
1383	-262.813860845776\\
1384	-209.608353073976\\
1385	-236.856163810697\\
1386	-198.304433960595\\
1388	-58.4849060422439\\
1389	-75.7572822223381\\
1390	-129.102314531604\\
1391	-103.06411412279\\
1392	-69.6945355384041\\
1393	-90.8020239017301\\
1395	-46.6558360982895\\
1396	-71.8447551334366\\
1397	-69.5978780839876\\
1398	-59.420560745905\\
1400	-140.645285687259\\
1401	-99.0844374856724\\
1402	-163.6313198654\\
1403	-241.882453896661\\
1404	-221.673669381069\\
1405	-266.843033154963\\
1406	-200.602236366809\\
1407	-203.468631831995\\
1408	-146.855894840801\\
1409	-78.5537638561395\\
1410	-111.752306681607\\
1411	-96.9702364069183\\
1412	-64.556885475738\\
1413	-96.3836220786325\\
1414	-47.9842150956117\\
1415	-28.4391877461519\\
1416	-37.0322449434502\\
1417	-108.851097548332\\
1418	-125.548155293032\\
1419	-153.110415964829\\
1420	-154.358434172048\\
1421	-148.468990037377\\
1422	-156.972280174365\\
1423	-143.930287990298\\
1424	-112.680191158512\\
1425	-115.644896330711\\
1426	-100.606777020107\\
1427	-140.99748620282\\
1428	-99.034121714059\\
1429	-85.06932983915\\
1430	-119.181061454035\\
1431	-161.855767233555\\
1432	-122.165161367068\\
1434	-86.5558365230038\\
1435	-96.4703879082231\\
1436	-72.7290912516237\\
1437	-56.5010909056664\\
1438	-91.3246635811288\\
1439	-107.92940953624\\
1440	-115.750679902097\\
1441	-110.471795378409\\
1442	-117.078707638738\\
1443	-122.110669773986\\
1444	-59.5027744735696\\
1445	-73.1989602405322\\
1446	-142.259510673237\\
1447	-179.442946704287\\
1449	-162.958399906291\\
1450	-153.2270198293\\
1451	-120.095354783296\\
1452	-116.967129377747\\
1453	-95.101612812877\\
1454	-124.966387869322\\
1455	-119.379974774395\\
1456	-74.2284222436642\\
1457	-117.050111034389\\
1458	-113.243520544356\\
1459	-147.710995883867\\
1460	-155.345165064858\\
1461	-157.973001530932\\
1464	-296.70056413222\\
1466	-114.134659773428\\
1467	-47.2861036667969\\
1468	-34.4196749340365\\
1469	-79.4150680912044\\
1470	-61.9958359766058\\
1471	-122.037145063034\\
1472	-105.913363505203\\
1473	-100.679503544843\\
1474	-124.400294166356\\
1475	-140.877707827376\\
1476	-166.79385884389\\
1477	-177.408980466373\\
1478	-246.743884228179\\
1479	-237.0663373856\\
1481	-127.780318158598\\
1482	-123.461437763226\\
1483	-130.242307107138\\
1484	-157.99938424945\\
1485	-117.895232777355\\
1486	-110.886345540873\\
1487	-116.51725239163\\
1488	-45.7254673566433\\
1489	-109.504605525919\\
1490	-148.969174304607\\
1491	-117.030861847922\\
1492	-114.978302933916\\
1493	-207.06981936384\\
1494	-164.999393756582\\
1495	-157.586031212647\\
1496	-204.180727663906\\
1497	-210.452729346202\\
1498	-134.990258356643\\
1499	-88.5365191610138\\
1500	-88.1767857855316\\
1502	-220.482783250664\\
1504	-239.782827260157\\
1505	-198.335591525767\\
1506	-124.848406638447\\
1507	-115.583419784407\\
1508	-73.5140752414509\\
1509	-76.8663577191646\\
1510	-59.6209269244798\\
1511	-96.7094511841408\\
1512	-102.312218166409\\
1513	-61.6821985961033\\
1515	-126.746732872514\\
1516	-140.553627514754\\
1518	-217.10027598038\\
1519	-277.204030707942\\
1520	-204.482317974963\\
1521	-180.964556070369\\
1522	-185.066763424233\\
1523	-169.389007480088\\
1524	-110.745430697401\\
1525	-126.495269707488\\
1526	-204.702792931656\\
1527	-223.307684298803\\
1529	-80.7103577503542\\
1531	-2.40589883018629\\
1533	-65.9119293064671\\
1534	-71.7793517267994\\
1535	-108.120600602799\\
1536	-88.1471441050785\\
1537	-62.5020671761065\\
1538	-70.4428646301644\\
1539	-59.4238852599178\\
1540	-66.994019530274\\
1541	-65.287121102202\\
1542	-99.7184409560666\\
1543	-146.88965482459\\
1544	-153.244045746714\\
1545	-156.201666674994\\
1546	-171.433961881505\\
1547	-219.373797403722\\
1548	-228.457834856338\\
1549	-254.909132953458\\
1550	-247.107678702122\\
1551	-200.83904856529\\
1552	-136.00815259104\\
1553	-124.348562317082\\
1554	-202.081610190082\\
1555	-218.481194677814\\
1557	-116.605564106341\\
1558	-31.6406653532852\\
1559	-22.9983349862637\\
1560	-44.9878636080805\\
1561	-12.0617102054132\\
1562	-66.7665726866996\\
1563	-88.0722447350079\\
1564	-86.9681454737499\\
1565	-82.0147012278851\\
1566	-42.816676554329\\
1567	-36.073174960798\\
1568	-51.9577986779186\\
1569	-59.4269547474169\\
1570	-62.9649598363521\\
1571	-46.1562996541559\\
1572	-39.2900091121312\\
1573	-75.9944203821726\\
1574	-172.592247320596\\
1575	-205.737637277609\\
1576	-194.683379007028\\
1577	-165.418644607196\\
1578	-202.196934061609\\
1579	-270.789435238416\\
1580	-266.467150328795\\
1581	-260.813900339197\\
1582	-209.592302773478\\
1583	-215.560907502821\\
1584	-201.548750063892\\
1585	-158.595473737212\\
1586	-131.448957388516\\
1587	-73.591818747487\\
1588	-69.8683472809146\\
1589	-145.956495260838\\
1590	-94.1743813119481\\
1592	-118.496518703989\\
1593	-116.443886875717\\
1594	-111.844416258739\\
1595	-125.879709178539\\
1596	-136.058206072074\\
1597	-142.885527017608\\
1598	-130.650881157905\\
1599	-101.98251284279\\
1600	-126.959857786229\\
1601	-53.0265709835598\\
1602	-9.71854915659742\\
1603	-35.0096628505928\\
1604	-81.9972696957257\\
1605	-28.855893253862\\
1606	-7.54144079534422\\
1607	-37.3575719529824\\
1608	-75.5331047793247\\
1609	-80.4097514028672\\
1610	-107.594884647822\\
1611	-89.9471840579497\\
1612	-146.595917779035\\
1613	-130.029317806485\\
1614	-81.3832835924816\\
1615	-130.457092417608\\
1616	-55.9182897589133\\
1617	-64.4146519679734\\
1618	-27.1221813402069\\
1619	-20.0187262680167\\
1620	-50.3980938034797\\
1621	-26.198683567151\\
1623	-112.12263345842\\
1624	-104.61707785814\\
1625	-80.8611997535318\\
1626	-85.0556351421978\\
1627	-65.9970697162203\\
1628	-95.1608630037258\\
1629	-59.473254731183\\
1630	-63.3917798887751\\
1631	-59.141587058112\\
1632	-48.172495722343\\
1633	-53.6299948007857\\
1634	-50.7100372205591\\
1635	-94.8179406758848\\
1636	-79.1159533408461\\
1638	-60.0683477990892\\
1639	-51.6971299986883\\
1640	-61.9493506406727\\
1641	-136.246714223892\\
1642	-178.807347260791\\
1643	-123.682462915286\\
1644	-117.284240166506\\
1645	-74.6650064539026\\
1646	-145.411674261479\\
1647	-122.11572408629\\
1648	-82.2320437457558\\
1649	-100.921330234516\\
1650	-76.0791875587634\\
1651	-108.705650508026\\
1652	-59.3707428134046\\
1653	-61.6287461493591\\
1654	-85.3791944574634\\
1655	-97.1440723636174\\
1656	-80.9412814100585\\
1657	-76.6204621781103\\
1658	-88.0196578256621\\
1659	-129.366864912269\\
1660	-115.09538509835\\
1662	-226.893288802622\\
1663	-190.489098677657\\
1664	-120.951669782991\\
1665	-80.0564716853785\\
1666	-140.147836321836\\
1667	-162.196722318017\\
1668	-131.879844906438\\
1669	-157.52538067543\\
1670	-96.1677325774642\\
1671	-45.2911475535002\\
1672	-47.9811742304373\\
1673	-127.682954728919\\
1674	-110.923767334309\\
1675	-97.4529933199908\\
1676	-146.738598924817\\
1677	-144.631754590723\\
1678	-127.682496784908\\
1679	-96.1839808414632\\
1680	-107.815571762234\\
1681	-148.078309025635\\
1682	-115.972902123401\\
1683	-132.263656483075\\
1685	-91.2365252371528\\
1686	-101.381553326329\\
1687	-68.6631287074047\\
1688	-83.7435655719964\\
1689	-125.202521237502\\
1690	-152.074232510418\\
1691	-154.986571104358\\
1692	-141.163223676032\\
1694	-75.9441748362624\\
1695	-86.4851914930623\\
1697	-121.01006733609\\
1699	-178.035473185592\\
1700	-149.749419490699\\
1701	-92.3557604215089\\
1703	-212.139686374269\\
1704	-156.613384425416\\
1705	-152.664724499956\\
1706	-173.454344821357\\
1707	-146.082955329448\\
1708	-114.287367914884\\
1709	-135.185095909484\\
1710	-116.466225219178\\
1711	-130.99752843648\\
1712	-158.315698995845\\
1713	-127.708480451293\\
1714	-144.443886581992\\
1715	-136.263352823333\\
1716	-142.110890192354\\
1717	-116.732111367337\\
1718	-140.974638462889\\
1719	-110.543953394173\\
1720	-107.495701378705\\
1721	-123.846008076907\\
1722	-113.446011120521\\
1723	-58.5836062234002\\
1724	-19.230722234113\\
1725	-12.9219271235443\\
1726	-50.5621965390997\\
1727	-133.818954295853\\
1728	-160.290350543724\\
1729	-154.53798305964\\
1730	-104.791365953509\\
1731	-119.793230248833\\
1732	-111.879120565535\\
1733	-99.9967368661271\\
1734	-55.9029213069637\\
1735	-87.4000933227785\\
1736	-80.4536971574892\\
1737	-81.8877150012215\\
1738	-95.3490023508054\\
1739	-81.0775451602162\\
1740	-79.2710457797473\\
1741	-55.178213245915\\
1742	-71.3860647022425\\
1743	-134.485372530812\\
1744	-97.8189711044706\\
1745	-118.286465854341\\
1746	-101.759226225685\\
1747	-146.73190557208\\
1748	-130.089731619173\\
1749	-172.809978241695\\
1750	-144.427470765847\\
1751	-144.895107538812\\
1752	-152.454166340822\\
1753	-74.2079621417176\\
1754	-139.629023880668\\
1755	-87.6569822885556\\
1756	-120.06022485249\\
1757	-118.80743993465\\
1758	-153.40502446019\\
1759	-126.768360353407\\
1761	-181.640893778087\\
1762	-170.437693853442\\
1763	-138.720023644465\\
1764	-124.216118756028\\
1765	-133.550691002028\\
1766	-114.899934723004\\
1767	-108.766442132003\\
1768	-81.2999498747131\\
1769	-105.893614301944\\
1770	-86.6484977223133\\
1771	-160.190555386243\\
1772	-202.671580102085\\
1774	-176.495180648755\\
1775	-186.435869190943\\
1776	-103.849234907339\\
1777	-97.9308867314876\\
1778	-69.9356187429767\\
1779	-65.1973371852691\\
1780	-78.2803059593102\\
1781	-118.621860340461\\
1782	-135.45062230059\\
1783	-157.196177473886\\
1784	-137.99680742193\\
1785	-103.274222541544\\
1786	-168.076924017269\\
1788	-210.955497032556\\
1789	-173.706804625807\\
1790	-272.394014733168\\
1791	-224.728629576137\\
1792	-125.763781132402\\
1793	-78.0547283396434\\
1794	-66.559288880758\\
1795	-133.356137610284\\
1796	-155.136801484478\\
1797	-203.425269410272\\
1798	-172.502424979007\\
1799	-131.992934089358\\
1800	-60.8873250655156\\
1801	-73.6290027426701\\
1803	-93.3329960627152\\
1804	-54.6116043238849\\
1805	-80.3090614263533\\
};
\addlegendentry{OSA predition}

\addplot [color=mycolor3, dotted, line width=2.0pt]
  table[row sep=crcr]{%
1006	-185.547\\
1007	-219.727\\
1008	-169.678\\
1009	-86.6700000000001\\
1010	-110.995349893054\\
1011	-113.48015936147\\
1012	-135.19327379848\\
1013	-110.360946472996\\
1014	-34.8172796545821\\
1015	-12.7660538094156\\
1016	-9.65849234555185\\
1017	-92.7438606403211\\
1018	-158.777941637327\\
1020	-155.55723983941\\
1021	-123.397786203265\\
1022	-70.5790363408887\\
1023	-160.232197873451\\
1024	-112.768851437712\\
1025	-84.9968505149641\\
1026	-144.543637685234\\
1027	-120.278561299612\\
1028	-109.823766628659\\
1029	-165.061806387555\\
1030	-157.200723171348\\
1031	-119.097894005191\\
1032	-169.969627169353\\
1033	-164.884288578847\\
1034	-136.93681106948\\
1036	-94.3296238556434\\
1037	-108.386459061253\\
1038	-132.061345973268\\
1039	-122.795101762155\\
1042	-144.987563571806\\
1043	-186.591636008599\\
1044	-171.552699409156\\
1045	-121.793716770385\\
1046	-117.382712461922\\
1047	-61.3617515859144\\
1048	-71.0846890973312\\
1049	-97.0969194975905\\
1050	-77.7171953450575\\
1051	-79.4321511093226\\
1052	-108.214999741604\\
1053	-126.768712571956\\
1054	-117.268519029459\\
1055	-171.732138212703\\
1056	-139.293646642306\\
1057	-83.9653427485123\\
1058	-67.5663048608785\\
1059	-89.1814629127123\\
1060	-104.983705149856\\
1061	-56.2687732641559\\
1062	-54.9440333895845\\
1063	-88.6123462529536\\
1064	-69.6021405326696\\
1065	-59.6268948915574\\
1066	-81.2118219435376\\
1067	-90.7005418767144\\
1068	-131.698036712644\\
1069	-135.132184658992\\
1070	-149.213618083935\\
1071	-137.752569292308\\
1072	-160.409434913501\\
1073	-130.397866925905\\
1074	-134.620869648923\\
1075	-114.336991028697\\
1076	-113.605579419946\\
1077	-114.503982773202\\
1079	-252.709425635659\\
1080	-256.100195234047\\
1081	-243.321067938582\\
1082	-167.846929090957\\
1084	-271.501809266627\\
1085	-269.835862016719\\
1086	-223.466043397849\\
1087	-264.697290837415\\
1088	-343.705053589925\\
1089	-282.446590473256\\
1090	-241.074931688262\\
1091	-154.624147103811\\
1092	-119.575679685124\\
1093	-107.907632054288\\
1094	-127.867241734112\\
1095	-101.992882400741\\
1096	-60.8914813638912\\
1097	-50.5074886380255\\
1098	-76.5055911478364\\
1099	-117.694450426636\\
1100	-112.227957160808\\
1101	-114.367986787761\\
1102	-145.934751460476\\
1103	-149.80255903684\\
1104	-199.13722509029\\
1105	-174.012522914705\\
1106	-192.615888642202\\
1107	-154.494309719521\\
1108	-133.985708880866\\
1109	-160.046217382638\\
1110	-117.086019165295\\
1111	-115.980182043542\\
1112	-128.865890340022\\
1113	-137.860394652278\\
1114	-98.5446884083733\\
1115	-111.319464692393\\
1116	-78.9686551400971\\
1117	-94.0740168090688\\
1118	-96.1378868847739\\
1119	-71.9370317764954\\
1120	-80.9870618245516\\
1121	-103.318988342816\\
1122	-154.858309128\\
1123	-156.671104906495\\
1124	-167.182376762871\\
1125	-102.125617214529\\
1126	-138.471185938093\\
1127	-206.298810579124\\
1128	-162.891798818806\\
1129	-185.567992510351\\
1130	-182.474501888118\\
1131	-117.86819812411\\
1132	-84.4504347069512\\
1133	-111.849825875277\\
1134	-128.637894166269\\
1135	-196.084166664454\\
1136	-233.3501233701\\
1137	-230.788200197408\\
1138	-183.004017563308\\
1139	-183.186662725417\\
1140	-174.957399238873\\
1141	-173.860156907509\\
1142	-167.490514202969\\
1143	-116.464793522313\\
1144	-101.608603627606\\
1146	-92.9794572410171\\
1147	-101.179648669014\\
1148	-137.546412571466\\
1149	-203.067339092969\\
1150	-165.497699265968\\
1151	-118.562528576525\\
1152	-107.27019552469\\
1153	-101.227855018227\\
1154	-59.6171916755654\\
1155	-39.0060038005088\\
1156	-44.9720947242135\\
1157	-96.3625826094931\\
1158	-76.7150090856637\\
1159	-85.0901700923714\\
1160	-85.9944545436288\\
1161	-83.6074286614437\\
1162	-64.9324175384934\\
1163	-40.9888270884699\\
1164	-26.0296488858671\\
1165	-56.4807375331598\\
1166	-127.38313011177\\
1167	-175.511058284223\\
1168	-188.839480223654\\
1170	-97.198219940946\\
1171	-70.8489428575774\\
1172	-37.5446485485013\\
1173	-101.889012706432\\
1174	-95.2353911865328\\
1175	-128.714348014696\\
1176	-168.942203915952\\
1177	-256.781930745541\\
1178	-309.495105810918\\
1179	-259.730767857054\\
1180	-250.585047172704\\
1181	-169.025458544714\\
1182	-179.696761887039\\
1183	-203.314095107743\\
1184	-203.033633709121\\
1185	-167.510594758444\\
1186	-148.297495921108\\
1187	-157.981224174138\\
1188	-140.454316933763\\
1189	-162.62488821973\\
1190	-125.356590046052\\
1191	-197.160870814102\\
1192	-220.227370495071\\
1193	-181.882173974361\\
1194	-111.925982697545\\
1195	-116.996894105499\\
1196	-194.545562133969\\
1198	-253.321180569697\\
1199	-271.334559888572\\
1200	-273.644236493625\\
1201	-220.166996686381\\
1202	-206.692707399118\\
1203	-225.314671915146\\
1204	-257.110743641904\\
1206	-104.947710910914\\
1207	-154.725134680761\\
1208	-127.0427598572\\
1209	-94.1274850203706\\
1210	-111.508201189256\\
1211	-132.908430402233\\
1212	-108.49968929726\\
1213	-91.239360632401\\
1214	-112.326805527275\\
1215	-99.0781977135923\\
1216	-128.760203161413\\
1217	-178.193282174354\\
1219	-119.300077490081\\
1220	-118.945633812569\\
1221	-167.343493199659\\
1222	-254.474370077357\\
1224	-131.659425078793\\
1225	-93.1696148678352\\
1226	-110.42980982086\\
1227	-113.722225889628\\
1228	-83.3205729684353\\
1229	-89.4424536616682\\
1231	-131.011329106772\\
1232	-143.256871195944\\
1233	-99.6005624580482\\
1234	-165.777795035362\\
1235	-184.646267007019\\
1236	-167.402895719238\\
1237	-156.069857497589\\
1238	-151.876944702146\\
1239	-203.038277205571\\
1241	-52.1924440096814\\
1242	-75.8601830102975\\
1243	-90.993161399675\\
1244	-126.183174633084\\
1245	-110.480168093163\\
1246	-87.4871303330906\\
1247	-128.35993394542\\
1248	-123.422188871953\\
1249	-96.0751800345415\\
1250	-111.509532228151\\
1251	-101.590589092212\\
1252	-98.7122095375021\\
1253	-110.541965195355\\
1254	-65.2684111602432\\
1255	-83.1871193586269\\
1256	-57.1831147003327\\
1257	-66.1445069589224\\
1258	-122.58737699298\\
1259	-125.726044277162\\
1261	-211.812944222738\\
1263	-95.1931191004796\\
1264	-62.576764050674\\
1265	-74.8030979880789\\
1266	-102.186494533485\\
1267	-63.0296079625696\\
1268	-73.2492107400956\\
1269	-93.2110425831274\\
1270	-154.707013424026\\
1271	-140.459150898716\\
1272	-184.688441558762\\
1273	-125.43666010532\\
1274	-126.375378113186\\
1275	-53.579784762991\\
1276	-61.1487999144256\\
1277	-30.1268752013273\\
1278	-44.2437051578838\\
1279	-48.4893969800523\\
1280	-97.3335315466434\\
1281	-101.464501573417\\
1282	-114.958317992413\\
1283	-125.17399899783\\
1284	-187.229905549205\\
1285	-184.322952402596\\
1286	-137.200516776599\\
1287	-155.236760757795\\
1288	-162.677416418589\\
1289	-200.319145122401\\
1291	-115.280857165873\\
1292	-45.8586680999897\\
1293	-25.1147950538464\\
1294	-30.5410735240039\\
1295	-49.2099551494898\\
1296	-54.8168101246517\\
1297	-47.8640255406297\\
1298	-69.20806570796\\
1299	-112.245021930481\\
1300	-105.067316242943\\
1301	-99.7753169209739\\
1302	-118.056414309156\\
1303	-72.4606265799728\\
1304	-34.2472086356108\\
1305	-89.1109647409958\\
1306	-131.502822059699\\
1307	-118.61707095479\\
1308	-132.404283204588\\
1309	-122.244420660296\\
1310	-90.9392370888143\\
1311	-122.339090629224\\
1312	-166.537575593008\\
1313	-160.656128516536\\
1314	-123.625749848343\\
1315	-191.440978782938\\
1316	-154.226044966168\\
1317	-149.384117599934\\
1318	-169.607836149202\\
1319	-159.96304017563\\
1320	-133.497399502222\\
1321	-120.29243097863\\
1323	-243.133837507078\\
1324	-196.088354911589\\
1325	-126.521710548268\\
1326	-119.786349637094\\
1327	-122.257484930297\\
1328	-151.132820604995\\
1329	-163.497082763436\\
1330	-133.391082491102\\
1331	-139.364191699968\\
1332	-173.219758827865\\
1333	-178.032546694327\\
1334	-126.424983213975\\
1335	-107.304464593157\\
1336	-139.838883766983\\
1337	-201.808979815136\\
1338	-192.20026790968\\
1339	-189.894708461745\\
1340	-117.743358822717\\
1341	-109.298496484061\\
1342	-117.20031591442\\
1343	-67.3690349092235\\
1344	-36.0501260139683\\
1345	-20.7561882498294\\
1346	-18.5461925176749\\
1347	-76.639312108591\\
1348	-107.794463573477\\
1349	-122.689864156151\\
1350	-97.8263570221779\\
1352	-117.100154062181\\
1353	-75.4918732557219\\
1354	-80.4229929948292\\
1355	-79.9441256701443\\
1356	-59.3510388548946\\
1357	-83.0817786348957\\
1359	-150.700368062312\\
1360	-96.3990002487928\\
1361	-83.3664358217347\\
1362	-62.005698703153\\
1363	-81.370001613459\\
1364	-52.5570062952297\\
1366	-192.711253547816\\
1367	-161.340511392172\\
1368	-199.852178679743\\
1369	-228.83315427262\\
1370	-230.401348888691\\
1372	-139.267168197548\\
1373	-132.181911620758\\
1374	-139.018966472279\\
1376	-173.022333593456\\
1377	-170.48365375595\\
1378	-219.866827781739\\
1379	-244.184359373487\\
1380	-287.016103134227\\
1381	-209.082299253106\\
1382	-225.969550548494\\
1383	-251.892978000681\\
1384	-198.349043458402\\
1385	-230.873572962284\\
1386	-192.490747649133\\
1387	-119.848529287689\\
1388	-59.9984546272819\\
1389	-70.3009346599645\\
1390	-122.95373670799\\
1391	-104.323289545051\\
1392	-71.5712629149168\\
1393	-93.6499323507533\\
1395	-51.4184740940452\\
1396	-74.7590992359292\\
1397	-74.5617817797852\\
1398	-62.6074501313544\\
1400	-145.393066584891\\
1401	-104.804388589965\\
1403	-243.068842561898\\
1404	-223.656116788186\\
1405	-268.331594598311\\
1406	-196.072605421755\\
1407	-205.50462007241\\
1409	-82.4678639190538\\
1410	-113.181126585173\\
1411	-98.2467441819022\\
1412	-70.7329483494877\\
1413	-100.45193899889\\
1414	-52.2729146722108\\
1415	-31.0294214052449\\
1416	-37.0271684570625\\
1417	-106.720836625809\\
1418	-127.963364287519\\
1419	-153.752802710967\\
1420	-153.190639289618\\
1421	-149.186029234218\\
1422	-158.157772486123\\
1423	-139.725265500493\\
1424	-113.192870610955\\
1425	-118.14573260246\\
1426	-102.575515635868\\
1427	-147.916281154035\\
1428	-101.877291768973\\
1429	-88.1727341091114\\
1431	-164.824407940039\\
1432	-123.584973765956\\
1433	-105.965991429126\\
1434	-91.8213629084973\\
1435	-102.564808761641\\
1436	-78.6866930158517\\
1437	-66.1895057842455\\
1438	-97.0578198446765\\
1439	-114.461819083953\\
1440	-120.842794572804\\
1441	-110.525707525639\\
1442	-123.525803303377\\
1443	-122.231167766686\\
1444	-61.3959435710251\\
1445	-75.6857725121488\\
1446	-144.858593154789\\
1447	-184.770305574069\\
1449	-156.975392916746\\
1450	-151.702260724142\\
1451	-119.036832890748\\
1452	-117.011361936425\\
1453	-98.5147775138603\\
1454	-130.28194147065\\
1455	-123.289933761513\\
1456	-78.0125072508592\\
1457	-121.983208361814\\
1458	-120.778015148778\\
1459	-148.141293643682\\
1460	-155.626237657072\\
1461	-155.017552860762\\
1463	-242.67312253545\\
1464	-281.722156351427\\
1466	-110.466376994711\\
1467	-48.7630729511172\\
1468	-24.8358989976675\\
1469	-68.6299530170179\\
1470	-58.9361258499107\\
1471	-115.657949113317\\
1472	-101.450848137918\\
1473	-99.2239707420308\\
1474	-125.586117412517\\
1475	-141.484537546206\\
1476	-165.758477572652\\
1477	-174.962464398205\\
1478	-242.253060491117\\
1479	-228.272521169602\\
1481	-127.823131264074\\
1482	-125.085295134772\\
1483	-132.546602252132\\
1484	-162.965315709038\\
1485	-122.121567667129\\
1486	-117.691277648927\\
1487	-120.969950603312\\
1488	-51.7620574045532\\
1489	-109.860318958032\\
1490	-152.495960898817\\
1491	-118.766434683824\\
1492	-119.285692196981\\
1493	-210.567487385024\\
1494	-163.974264878427\\
1495	-159.896151949835\\
1496	-206.123864145329\\
1497	-205.125750175049\\
1498	-133.221756388709\\
1499	-90.800038917562\\
1500	-89.5570608704368\\
1502	-224.827491711511\\
1503	-226.810745124581\\
1504	-231.363687068009\\
1505	-187.705259908497\\
1506	-119.419006319111\\
1507	-112.469817980158\\
1508	-74.9652102806276\\
1509	-78.7273941355136\\
1510	-63.3042979924915\\
1511	-100.726441610737\\
1512	-108.421135410228\\
1513	-66.4454792879048\\
1515	-133.267632321669\\
1516	-145.596265735836\\
1518	-217.051718188387\\
1519	-270.753471695916\\
1520	-197.572862514098\\
1521	-178.536759046202\\
1522	-185.84308626213\\
1523	-172.238002712188\\
1524	-118.186887057834\\
1525	-135.183165543245\\
1526	-211.498824371521\\
1527	-226.009745456101\\
1528	-143.914134300377\\
1529	-79.7869138513724\\
1530	-43.0018558619013\\
1531	3.71940897576542\\
1532	-18.5413309828018\\
1533	-55.1283246930745\\
1534	-63.1245539531751\\
1535	-97.540338408246\\
1536	-82.5247865977826\\
1537	-61.0171125219192\\
1538	-67.8351239059868\\
1539	-60.7731084434397\\
1540	-67.5070252852631\\
1541	-70.2047270301546\\
1542	-103.439650588156\\
1543	-149.18088618574\\
1544	-155.192689996941\\
1545	-155.812136156626\\
1546	-171.582642529035\\
1547	-214.570114494524\\
1548	-222.269388798872\\
1549	-250.824656986575\\
1550	-236.385188661907\\
1551	-192.272905425838\\
1552	-137.733964324411\\
1553	-128.562611692331\\
1554	-205.682509067561\\
1555	-221.494288218977\\
1557	-114.655020666315\\
1558	-35.8930041403153\\
1559	-15.3119067249452\\
1560	-32.0032519512749\\
1561	-6.39498688534104\\
1562	-55.0308733068603\\
1563	-80.1798859151797\\
1564	-81.8960152516702\\
1565	-76.3487549519971\\
1566	-40.6849196578755\\
1567	-32.8275584172268\\
1568	-49.3811669751501\\
1569	-57.4739282309115\\
1570	-62.3457365985785\\
1571	-46.0064003967768\\
1572	-38.3465473526126\\
1573	-76.1556743237422\\
1574	-175.082284707694\\
1575	-210.589243023739\\
1576	-205.179080153966\\
1577	-164.281934896043\\
1578	-209.821724891764\\
1579	-274.018452957888\\
1580	-260.004404854101\\
1581	-259.744880674325\\
1582	-203.495703925144\\
1583	-213.242918523729\\
1584	-205.817988235256\\
1585	-155.98776390428\\
1586	-137.683749832145\\
1587	-79.1263557576797\\
1588	-68.4652363658774\\
1589	-144.716781723043\\
1590	-97.4543263098133\\
1592	-120.154618897142\\
1593	-119.688705354891\\
1594	-116.371562079482\\
1596	-140.507846340443\\
1597	-146.431121275206\\
1598	-131.218935877415\\
1599	-104.877133618097\\
1600	-132.921272267032\\
1601	-59.1823259337509\\
1602	-11.4579369384091\\
1603	-29.5049005031606\\
1604	-74.517663475624\\
1605	-27.5143959310053\\
1606	-2.55687034031826\\
1607	-28.5629144324291\\
1608	-69.1463126213202\\
1609	-76.2058172010875\\
1610	-101.519957205873\\
1611	-86.493837701207\\
1612	-144.606136980466\\
1613	-129.526399899475\\
1614	-83.7558787152345\\
1615	-131.474386431832\\
1616	-58.3985121799001\\
1617	-60.6503088634167\\
1618	-27.2564007894964\\
1619	-16.931722427554\\
1620	-44.8162097528252\\
1621	-24.1722718827698\\
1623	-109.044254380261\\
1624	-101.918173251549\\
1625	-77.5815603564572\\
1626	-84.637358051812\\
1627	-65.0237286712311\\
1628	-94.6655030851996\\
1629	-60.3216457683109\\
1630	-60.632388144375\\
1631	-57.2378446856476\\
1632	-47.4540541903675\\
1633	-54.1723573850184\\
1634	-49.4925876876262\\
1635	-93.8747747226953\\
1637	-67.9496625083532\\
1638	-59.9899162606287\\
1639	-50.2235441634184\\
1640	-60.7550409197111\\
1641	-135.571838830798\\
1642	-178.175633792418\\
1643	-123.043186593823\\
1644	-117.007859182734\\
1645	-75.5896778606927\\
1646	-143.521983077267\\
1647	-124.376827802893\\
1648	-82.6189970550856\\
1649	-101.254903755818\\
1650	-76.486519420585\\
1651	-108.344822466927\\
1652	-59.4035513249733\\
1653	-59.3085861306079\\
1654	-82.4991548319117\\
1655	-96.9824039161017\\
1656	-77.142627805822\\
1657	-76.3057938776183\\
1658	-86.0331615387672\\
1659	-128.387523997971\\
1660	-112.153994136065\\
1662	-220.451635156299\\
1663	-183.224452892807\\
1664	-119.008616070849\\
1665	-80.0816990119029\\
1666	-136.562311677659\\
1667	-161.940694555677\\
1668	-127.169286101076\\
1669	-153.126010664913\\
1670	-91.3315991590937\\
1671	-40.1249808947307\\
1672	-42.9290139553971\\
1673	-121.108202016344\\
1675	-96.5834691386012\\
1676	-144.65584252095\\
1677	-139.722907522033\\
1678	-123.488216114443\\
1679	-91.0370679484604\\
1680	-105.657327037889\\
1681	-143.216857651435\\
1682	-111.62416338341\\
1683	-127.383042747583\\
1685	-89.288118716144\\
1686	-103.601072339297\\
1687	-72.3315067491058\\
1688	-84.8888866751879\\
1689	-128.910439175861\\
1690	-150.784042204629\\
1691	-152.66302885164\\
1692	-135.727905742337\\
1694	-73.5109955285561\\
1695	-85.6758468589439\\
1697	-122.13159169083\\
1699	-172.979922231163\\
1700	-141.140351942604\\
1701	-87.3666517251818\\
1703	-208.732010440957\\
1704	-150.36743158274\\
1705	-149.211882552193\\
1706	-169.531713326464\\
1708	-112.585183466665\\
1709	-134.05086095195\\
1710	-117.036334319076\\
1711	-132.063286344634\\
1712	-158.75997936465\\
1713	-124.783378763573\\
1714	-143.113550966566\\
1715	-134.509008522499\\
1716	-139.407117463593\\
1717	-115.240129235648\\
1718	-142.401720870981\\
1719	-110.17368637229\\
1720	-110.305721302346\\
1721	-126.953477329094\\
1722	-117.042068039509\\
1723	-60.6067745037994\\
1724	-20.0627075100888\\
1725	-7.55635433304792\\
1726	-40.4535422850649\\
1727	-126.937496461069\\
1728	-156.521931595342\\
1729	-150.126322015866\\
1730	-99.23367077085\\
1731	-117.946212869116\\
1732	-110.284586266651\\
1733	-100.813668678545\\
1734	-59.6494710783829\\
1735	-88.7282439410053\\
1736	-84.1157329066887\\
1737	-84.486930258231\\
1738	-97.6671352652747\\
1739	-83.3013058680724\\
1740	-81.5141647272219\\
1741	-58.4116972824017\\
1742	-75.8113601975449\\
1743	-137.944243989215\\
1744	-98.8731220891984\\
1745	-120.253750339119\\
1746	-102.645930038029\\
1747	-144.916412707557\\
1748	-129.646177916534\\
1749	-170.006444865264\\
1750	-136.379611109052\\
1752	-147.092836456599\\
1753	-68.1608024507268\\
1754	-136.515638926453\\
1755	-88.1276424519406\\
1756	-116.759068220153\\
1757	-119.081512831084\\
1758	-150.011873843234\\
1759	-118.579898893774\\
1760	-151.220654401967\\
1761	-175.867307535309\\
1762	-157.208000625357\\
1763	-131.712162475851\\
1764	-117.800798854048\\
1765	-131.157458406576\\
1766	-112.219115348509\\
1767	-106.523191406658\\
1768	-85.6704120344518\\
1769	-108.139062670644\\
1770	-91.9797998543909\\
1771	-166.651246994857\\
1772	-203.394140728446\\
1773	-181.24360298119\\
1774	-169.978717603356\\
1775	-176.229783739313\\
1776	-97.2228512823901\\
1777	-94.8981269742981\\
1778	-70.1795355230183\\
1779	-64.6940429242088\\
1780	-78.7205102063383\\
1781	-123.581900434756\\
1783	-156.696989914251\\
1784	-134.363399689335\\
1785	-99.7080368936579\\
1786	-168.154006305237\\
1788	-204.605159321083\\
1789	-164.143673953452\\
1790	-262.225643022105\\
1791	-209.58613547228\\
1792	-120.423919367915\\
1793	-79.4936038834514\\
1794	-62.5248366866076\\
1795	-129.948818351003\\
1796	-155.900843844142\\
1797	-199.884596345352\\
1799	-128.970455073577\\
1800	-61.8084940740246\\
1801	-68.9746064755602\\
1803	-94.3023901370616\\
1804	-56.8488968032398\\
1805	-81.1508170353807\\
};
\addlegendentry{MPO prediction}

\end{axis}

\begin{axis}[%
width=6.159cm,
height=1.831cm,
at={(0cm,2.542cm)},
scale only axis,
xmin=1000,
xmax=2000,
xlabel style={font=\color{white!15!black}},
xlabel={Sample index},
ymin=-400,
ymax=0,
ylabel style={font=\color{white!15!black}},
ylabel={$y(t)$},
axis background/.style={fill=white},
title style={font=\bfseries},
title={C7: RMSE(OSA) = 6.82, RMSE(MPO) = 9.0988},
legend style={legend cell align=left, align=left, draw=white!15!black}
]
\addplot [color=mycolor1, line width=2.0pt]
  table[row sep=crcr]{%
1006	-147.705\\
1007	-175.781\\
1008	-134.277\\
1009	-68.3589999999999\\
1010	-80.566\\
1011	-80.566\\
1012	-100.098\\
1013	-81.787\\
1014	-34.1800000000001\\
1015	-24.414\\
1016	-23.193\\
1018	-128.174\\
1020	-137.939\\
1021	-92.7729999999999\\
1022	-56.152\\
1023	-125.732\\
1024	-86.6700000000001\\
1025	-68.3589999999999\\
1026	-114.746\\
1028	-85.4490000000001\\
1029	-142.822\\
1030	-133.057\\
1031	-102.539\\
1032	-150.146\\
1033	-145.264\\
1034	-113.525\\
1036	-72.021\\
1037	-86.6700000000001\\
1038	-112.305\\
1039	-103.76\\
1040	-109.863\\
1041	-112.305\\
1042	-122.07\\
1043	-175.781\\
1044	-155.029\\
1045	-102.539\\
1046	-92.7729999999999\\
1047	-53.711\\
1048	-58.5940000000001\\
1049	-79.346\\
1050	-59.8140000000001\\
1051	-62.2560000000001\\
1052	-89.1110000000001\\
1053	-102.539\\
1054	-95.2149999999999\\
1055	-151.367\\
1057	-64.6970000000001\\
1058	-51.27\\
1059	-69.5799999999999\\
1060	-81.787\\
1061	-43.9449999999999\\
1062	-45.1659999999999\\
1063	-67.1389999999999\\
1064	-51.27\\
1065	-42.7249999999999\\
1066	-61.0350000000001\\
1067	-70.8009999999999\\
1068	-109.863\\
1069	-104.98\\
1070	-129.395\\
1071	-120.85\\
1072	-137.939\\
1073	-109.863\\
1074	-107.422\\
1075	-92.7729999999999\\
1076	-91.5530000000001\\
1077	-87.8910000000001\\
1079	-224.609\\
1080	-231.934\\
1081	-225.83\\
1082	-140.381\\
1083	-205.078\\
1084	-252.686\\
1085	-261.23\\
1086	-200.195\\
1088	-335.693\\
1089	-235.596\\
1090	-191.65\\
1091	-128.174\\
1092	-98.877\\
1093	-84.229\\
1094	-104.98\\
1095	-74.463\\
1096	-50.049\\
1097	-48.828\\
1098	-63.4770000000001\\
1099	-95.2149999999999\\
1100	-86.6700000000001\\
1101	-89.1110000000001\\
1102	-119.629\\
1103	-123.291\\
1104	-167.236\\
1105	-134.277\\
1106	-169.678\\
1107	-122.07\\
1108	-108.643\\
1109	-125.732\\
1110	-89.1110000000001\\
1111	-80.566\\
1112	-97.6559999999999\\
1113	-98.877\\
1114	-68.3589999999999\\
1115	-73.242\\
1116	-54.932\\
1117	-61.0350000000001\\
1118	-64.6970000000001\\
1119	-45.1659999999999\\
1121	-72.021\\
1122	-117.188\\
1123	-122.07\\
1124	-136.719\\
1125	-76.904\\
1126	-109.863\\
1127	-173.34\\
1128	-131.836\\
1129	-157.471\\
1130	-157.471\\
1131	-85.4490000000001\\
1132	-59.8140000000001\\
1133	-85.4490000000001\\
1134	-98.877\\
1135	-161.133\\
1136	-202.637\\
1137	-198.975\\
1138	-145.264\\
1139	-150.146\\
1140	-135.498\\
1141	-131.836\\
1142	-126.953\\
1143	-85.4490000000001\\
1145	-69.5799999999999\\
1146	-62.2560000000001\\
1147	-70.8009999999999\\
1148	-107.422\\
1149	-164.795\\
1151	-86.6700000000001\\
1152	-73.242\\
1153	-70.8009999999999\\
1154	-42.7249999999999\\
1155	-31.7380000000001\\
1156	-39.0630000000001\\
1157	-72.021\\
1158	-57.373\\
1159	-59.8140000000001\\
1160	-67.1389999999999\\
1161	-63.4770000000001\\
1162	-43.9449999999999\\
1163	-29.297\\
1164	-25.635\\
1165	-43.9449999999999\\
1166	-96.4359999999999\\
1167	-140.381\\
1168	-155.029\\
1169	-101.318\\
1170	-69.5799999999999\\
1171	-50.049\\
1172	-34.1800000000001\\
1173	-83.008\\
1174	-81.787\\
1175	-119.629\\
1176	-145.264\\
1177	-230.713\\
1178	-262.451\\
1179	-211.182\\
1180	-202.637\\
1181	-136.719\\
1182	-151.367\\
1183	-159.912\\
1184	-172.119\\
1185	-129.395\\
1186	-118.408\\
1187	-115.967\\
1188	-103.76\\
1189	-123.291\\
1190	-87.8910000000001\\
1191	-153.809\\
1192	-189.209\\
1194	-78.125\\
1195	-85.4490000000001\\
1196	-146.484\\
1197	-181.885\\
1198	-229.492\\
1199	-241.699\\
1200	-240.479\\
1201	-180.664\\
1202	-163.574\\
1203	-187.988\\
1204	-222.168\\
1205	-139.16\\
1206	-81.787\\
1207	-122.07\\
1208	-102.539\\
1209	-65.9180000000001\\
1210	-87.8910000000001\\
1211	-98.877\\
1212	-76.904\\
1213	-58.5940000000001\\
1214	-80.566\\
1215	-65.9180000000001\\
1216	-91.5530000000001\\
1217	-133.057\\
1218	-122.07\\
1219	-79.346\\
1220	-83.008\\
1221	-126.953\\
1222	-209.961\\
1224	-92.7729999999999\\
1225	-69.5799999999999\\
1226	-83.008\\
1227	-74.463\\
1228	-56.152\\
1229	-62.2560000000001\\
1230	-74.463\\
1231	-96.4359999999999\\
1232	-107.422\\
1233	-72.021\\
1234	-122.07\\
1235	-144.043\\
1236	-137.939\\
1237	-106.201\\
1238	-118.408\\
1239	-158.691\\
1241	-41.5039999999999\\
1242	-58.5940000000001\\
1243	-65.9180000000001\\
1244	-91.5530000000001\\
1245	-81.787\\
1246	-59.8140000000001\\
1247	-96.4359999999999\\
1248	-91.5530000000001\\
1249	-65.9180000000001\\
1250	-83.008\\
1251	-76.904\\
1252	-67.1389999999999\\
1253	-79.346\\
1254	-46.3869999999999\\
1255	-53.711\\
1256	-40.2829999999999\\
1257	-43.9449999999999\\
1258	-86.6700000000001\\
1259	-100.098\\
1261	-177.002\\
1262	-115.967\\
1263	-68.3589999999999\\
1264	-48.828\\
1265	-48.828\\
1266	-73.242\\
1267	-46.3869999999999\\
1268	-52.49\\
1269	-70.8009999999999\\
1270	-119.629\\
1271	-107.422\\
1272	-150.146\\
1273	-102.539\\
1274	-91.5530000000001\\
1275	-47.607\\
1276	-47.607\\
1277	-31.7380000000001\\
1278	-34.1800000000001\\
1279	-41.5039999999999\\
1280	-75.684\\
1281	-83.008\\
1282	-92.7729999999999\\
1283	-97.6559999999999\\
1284	-152.588\\
1285	-130.615\\
1286	-97.6559999999999\\
1287	-119.629\\
1288	-129.395\\
1289	-167.236\\
1290	-133.057\\
1291	-84.229\\
1292	-43.9449999999999\\
1293	-31.7380000000001\\
1294	-34.1800000000001\\
1295	-41.5039999999999\\
1296	-43.9449999999999\\
1297	-40.2829999999999\\
1298	-56.152\\
1299	-87.8910000000001\\
1300	-79.346\\
1301	-81.787\\
1302	-96.4359999999999\\
1303	-58.5940000000001\\
1304	-29.297\\
1306	-97.6559999999999\\
1307	-96.4359999999999\\
1308	-109.863\\
1309	-93.9939999999999\\
1310	-69.5799999999999\\
1311	-97.6559999999999\\
1312	-137.939\\
1313	-139.16\\
1314	-97.6559999999999\\
1315	-162.354\\
1316	-128.174\\
1317	-118.408\\
1318	-137.939\\
1319	-137.939\\
1320	-103.76\\
1321	-90.3320000000001\\
1323	-214.844\\
1324	-164.795\\
1325	-92.7729999999999\\
1326	-92.7729999999999\\
1327	-91.5530000000001\\
1329	-136.719\\
1330	-100.098\\
1331	-104.98\\
1332	-140.381\\
1333	-153.809\\
1334	-103.76\\
1335	-79.346\\
1336	-104.98\\
1337	-175.781\\
1338	-159.912\\
1339	-162.354\\
1340	-95.2149999999999\\
1341	-84.229\\
1342	-83.008\\
1343	-52.49\\
1344	-34.1800000000001\\
1345	-28.076\\
1346	-23.193\\
1347	-59.8140000000001\\
1348	-86.6700000000001\\
1349	-104.98\\
1350	-76.904\\
1352	-97.6559999999999\\
1353	-59.8140000000001\\
1354	-63.4770000000001\\
1355	-61.0350000000001\\
1356	-45.1659999999999\\
1357	-57.373\\
1358	-97.6559999999999\\
1359	-124.512\\
1360	-78.125\\
1361	-56.152\\
1362	-46.3869999999999\\
1363	-59.8140000000001\\
1364	-41.5039999999999\\
1365	-91.5530000000001\\
1366	-158.691\\
1367	-122.07\\
1368	-167.236\\
1369	-194.092\\
1370	-197.754\\
1371	-140.381\\
1372	-106.201\\
1373	-106.201\\
1374	-104.98\\
1375	-120.85\\
1376	-150.146\\
1377	-147.705\\
1378	-200.195\\
1379	-220.947\\
1380	-263.672\\
1381	-178.223\\
1382	-190.43\\
1383	-212.402\\
1384	-163.574\\
1385	-189.209\\
1386	-163.574\\
1387	-91.5530000000001\\
1388	-58.5940000000001\\
1389	-62.2560000000001\\
1390	-95.2149999999999\\
1391	-75.684\\
1392	-51.27\\
1393	-72.021\\
1394	-52.49\\
1395	-40.2829999999999\\
1397	-56.152\\
1398	-40.2829999999999\\
1400	-109.863\\
1401	-78.125\\
1403	-195.313\\
1404	-178.223\\
1405	-222.168\\
1406	-150.146\\
1407	-161.133\\
1409	-67.1389999999999\\
1410	-85.4490000000001\\
1412	-48.828\\
1413	-72.021\\
1414	-41.5039999999999\\
1415	-25.635\\
1416	-36.6210000000001\\
1417	-79.346\\
1419	-128.174\\
1420	-124.512\\
1421	-117.188\\
1422	-136.719\\
1423	-111.084\\
1424	-90.3320000000001\\
1425	-90.3320000000001\\
1426	-73.242\\
1427	-115.967\\
1428	-78.125\\
1429	-67.1389999999999\\
1431	-137.939\\
1432	-103.76\\
1433	-76.904\\
1434	-64.6970000000001\\
1435	-74.463\\
1436	-51.27\\
1437	-50.049\\
1438	-72.021\\
1439	-86.6700000000001\\
1440	-98.877\\
1441	-79.346\\
1442	-101.318\\
1443	-92.7729999999999\\
1444	-52.49\\
1445	-54.932\\
1446	-112.305\\
1447	-158.691\\
1448	-153.809\\
1449	-129.395\\
1450	-122.07\\
1451	-92.7729999999999\\
1452	-89.1110000000001\\
1453	-70.8009999999999\\
1454	-102.539\\
1455	-90.3320000000001\\
1456	-54.932\\
1457	-84.229\\
1459	-118.408\\
1460	-129.395\\
1461	-129.395\\
1463	-216.064\\
1464	-244.141\\
1465	-153.809\\
1466	-85.4490000000001\\
1467	-54.932\\
1468	-37.8420000000001\\
1469	-57.373\\
1470	-53.711\\
1471	-95.2149999999999\\
1473	-75.684\\
1474	-102.539\\
1475	-118.408\\
1476	-141.602\\
1477	-148.926\\
1478	-208.74\\
1479	-181.885\\
1481	-90.3320000000001\\
1483	-95.2149999999999\\
1484	-123.291\\
1485	-87.8910000000001\\
1486	-89.1110000000001\\
1487	-87.8910000000001\\
1488	-47.607\\
1489	-81.787\\
1490	-128.174\\
1491	-90.3320000000001\\
1492	-96.4359999999999\\
1493	-170.898\\
1494	-129.395\\
1495	-122.07\\
1496	-177.002\\
1497	-166.016\\
1498	-102.539\\
1499	-64.6970000000001\\
1500	-65.9180000000001\\
1501	-114.746\\
1502	-190.43\\
1503	-197.754\\
1504	-206.299\\
1506	-97.6559999999999\\
1507	-83.008\\
1508	-58.5940000000001\\
1509	-56.152\\
1510	-47.607\\
1511	-68.3589999999999\\
1512	-81.787\\
1513	-46.3869999999999\\
1514	-70.8009999999999\\
1515	-103.76\\
1516	-114.746\\
1517	-145.264\\
1519	-223.389\\
1520	-161.133\\
1521	-137.939\\
1522	-144.043\\
1523	-120.85\\
1524	-84.229\\
1525	-103.76\\
1526	-167.236\\
1527	-194.092\\
1528	-115.967\\
1529	-62.2560000000001\\
1531	-23.193\\
1532	-31.7380000000001\\
1533	-51.27\\
1534	-61.0350000000001\\
1535	-80.566\\
1536	-67.1389999999999\\
1537	-51.27\\
1539	-48.828\\
1540	-45.1659999999999\\
1541	-52.49\\
1542	-81.787\\
1543	-123.291\\
1544	-133.057\\
1545	-125.732\\
1547	-181.885\\
1548	-183.105\\
1549	-223.389\\
1550	-201.416\\
1551	-146.484\\
1552	-101.318\\
1553	-96.4359999999999\\
1554	-163.574\\
1555	-191.65\\
1556	-131.836\\
1558	-45.1659999999999\\
1559	-31.7380000000001\\
1560	-35.4000000000001\\
1561	-21.973\\
1563	-68.3589999999999\\
1564	-72.021\\
1565	-62.2560000000001\\
1566	-36.6210000000001\\
1567	-28.076\\
1568	-41.5039999999999\\
1569	-43.9449999999999\\
1570	-48.828\\
1571	-36.6210000000001\\
1572	-30.518\\
1573	-54.932\\
1574	-128.174\\
1575	-153.809\\
1576	-170.898\\
1577	-124.512\\
1578	-172.119\\
1579	-241.699\\
1580	-222.168\\
1581	-224.609\\
1582	-164.795\\
1583	-163.574\\
1584	-172.119\\
1585	-113.525\\
1586	-106.201\\
1587	-65.9180000000001\\
1588	-59.8140000000001\\
1589	-117.188\\
1590	-79.346\\
1591	-81.787\\
1592	-96.4359999999999\\
1593	-86.6700000000001\\
1594	-87.8910000000001\\
1595	-93.9939999999999\\
1597	-119.629\\
1598	-103.76\\
1599	-74.463\\
1600	-96.4359999999999\\
1601	-48.828\\
1602	-20.752\\
1603	-34.1800000000001\\
1604	-58.5940000000001\\
1605	-30.518\\
1606	-13.4280000000001\\
1607	-32.9590000000001\\
1608	-58.5940000000001\\
1609	-69.5799999999999\\
1610	-83.008\\
1611	-72.021\\
1612	-117.188\\
1613	-104.98\\
1614	-70.8009999999999\\
1615	-107.422\\
1616	-59.8140000000001\\
1618	-32.9590000000001\\
1619	-21.973\\
1620	-41.5039999999999\\
1621	-34.1800000000001\\
1622	-53.711\\
1623	-91.5530000000001\\
1624	-87.8910000000001\\
1625	-59.8140000000001\\
1626	-68.3589999999999\\
1627	-52.49\\
1628	-74.463\\
1629	-53.711\\
1631	-46.3869999999999\\
1632	-36.6210000000001\\
1633	-45.1659999999999\\
1634	-41.5039999999999\\
1635	-75.684\\
1636	-73.242\\
1637	-54.932\\
1638	-51.27\\
1639	-40.2829999999999\\
1640	-48.828\\
1641	-108.643\\
1642	-144.043\\
1643	-96.4359999999999\\
1644	-90.3320000000001\\
1645	-63.4770000000001\\
1646	-111.084\\
1647	-101.318\\
1648	-67.1389999999999\\
1649	-83.008\\
1650	-64.6970000000001\\
1651	-89.1110000000001\\
1652	-52.49\\
1653	-54.932\\
1654	-65.9180000000001\\
1655	-84.229\\
1656	-62.2560000000001\\
1657	-65.9180000000001\\
1658	-68.3589999999999\\
1659	-112.305\\
1660	-95.2149999999999\\
1662	-190.43\\
1663	-151.367\\
1664	-95.2149999999999\\
1665	-72.021\\
1666	-113.525\\
1667	-141.602\\
1668	-108.643\\
1669	-129.395\\
1670	-81.787\\
1671	-42.7249999999999\\
1672	-42.7249999999999\\
1673	-98.877\\
1675	-79.346\\
1676	-124.512\\
1677	-118.408\\
1678	-103.76\\
1679	-67.1389999999999\\
1680	-84.229\\
1681	-115.967\\
1682	-95.2149999999999\\
1683	-97.6559999999999\\
1684	-89.1110000000001\\
1685	-62.2560000000001\\
1686	-75.684\\
1687	-54.932\\
1688	-62.2560000000001\\
1689	-106.201\\
1690	-123.291\\
1691	-131.836\\
1692	-117.188\\
1693	-83.008\\
1694	-58.5940000000001\\
1695	-67.1389999999999\\
1696	-80.566\\
1699	-155.029\\
1701	-69.5799999999999\\
1703	-178.223\\
1704	-128.174\\
1705	-125.732\\
1706	-147.705\\
1707	-111.084\\
1708	-90.3320000000001\\
1709	-100.098\\
1710	-92.7729999999999\\
1711	-103.76\\
1712	-131.836\\
1713	-101.318\\
1714	-112.305\\
1715	-106.201\\
1716	-108.643\\
1717	-86.6700000000001\\
1718	-114.746\\
1719	-81.787\\
1720	-86.6700000000001\\
1721	-96.4359999999999\\
1722	-93.9939999999999\\
1723	-42.7249999999999\\
1724	-26.855\\
1725	-18.3109999999999\\
1726	-39.0630000000001\\
1727	-96.4359999999999\\
1728	-125.732\\
1729	-125.732\\
1730	-83.008\\
1731	-97.6559999999999\\
1732	-85.4490000000001\\
1733	-75.684\\
1734	-46.3869999999999\\
1735	-65.9180000000001\\
1736	-65.9180000000001\\
1737	-64.6970000000001\\
1738	-76.904\\
1739	-65.9180000000001\\
1740	-59.8140000000001\\
1741	-40.2829999999999\\
1742	-58.5940000000001\\
1743	-109.863\\
1744	-78.125\\
1745	-92.7729999999999\\
1746	-83.008\\
1747	-118.408\\
1748	-109.863\\
1749	-152.588\\
1750	-111.084\\
1751	-124.512\\
1752	-124.512\\
1753	-63.4770000000001\\
1754	-113.525\\
1755	-76.904\\
1756	-96.4359999999999\\
1757	-102.539\\
1758	-129.395\\
1759	-97.6559999999999\\
1760	-125.732\\
1761	-158.691\\
1762	-131.836\\
1763	-114.746\\
1764	-90.3320000000001\\
1765	-107.422\\
1766	-89.1110000000001\\
1767	-76.904\\
1768	-68.3589999999999\\
1769	-81.787\\
1770	-69.5799999999999\\
1771	-141.602\\
1772	-181.885\\
1773	-157.471\\
1774	-150.146\\
1775	-147.705\\
1776	-81.787\\
1777	-72.021\\
1778	-54.932\\
1779	-50.049\\
1780	-54.932\\
1781	-93.9939999999999\\
1782	-118.408\\
1783	-134.277\\
1784	-114.746\\
1785	-76.904\\
1786	-140.381\\
1787	-163.574\\
1788	-183.105\\
1789	-150.146\\
1790	-236.816\\
1791	-158.691\\
1792	-93.9939999999999\\
1793	-68.3589999999999\\
1794	-56.152\\
1795	-101.318\\
1796	-130.615\\
1797	-172.119\\
1798	-129.395\\
1799	-102.539\\
1800	-53.711\\
1801	-53.711\\
1802	-64.6970000000001\\
1803	-70.8009999999999\\
1804	-43.9449999999999\\
1805	-61.0350000000001\\
};
\addlegendentry{True output}

\addplot [color=mycolor2, dashed, line width=2.0pt]
  table[row sep=crcr]{%
1006	-151.11723943552\\
1007	-161.983221323869\\
1008	-137.995196727072\\
1009	-72.7626727761606\\
1010	-85.4686037505749\\
1011	-89.5987934740049\\
1012	-102.898894109405\\
1013	-87.9967473673973\\
1014	-23.8307185378378\\
1015	-9.67533439215072\\
1016	-16.1864695024394\\
1017	-87.0982779529395\\
1018	-135.643161541276\\
1019	-125.15707727804\\
1020	-143.418943774374\\
1021	-92.1394583394563\\
1022	-63.1600405179497\\
1023	-133.158208289984\\
1024	-87.1931484357422\\
1025	-70.5712429881601\\
1026	-112.767241521516\\
1027	-101.340474173738\\
1028	-93.3101722676577\\
1029	-131.519639493086\\
1030	-137.737798969717\\
1031	-100.07498836992\\
1032	-145.228362976789\\
1033	-143.076318707906\\
1034	-119.110267954399\\
1035	-100.161380317569\\
1036	-74.7501867136048\\
1037	-92.7513238581882\\
1038	-107.270623918935\\
1039	-99.1063468155119\\
1040	-109.047248655299\\
1041	-109.840347269106\\
1042	-119.960954714404\\
1043	-164.243041575465\\
1044	-152.251230514262\\
1045	-115.620452565116\\
1046	-97.9590431005943\\
1047	-49.7292327357416\\
1048	-61.4041342746966\\
1049	-80.4889935367025\\
1050	-67.3015006495593\\
1051	-61.3883820285098\\
1052	-92.5283029850461\\
1053	-100.102956007118\\
1054	-96.4499830268971\\
1055	-145.158656235709\\
1056	-114.922593665844\\
1057	-70.43319243822\\
1058	-54.276641957407\\
1059	-73.0552689081753\\
1060	-88.0302811128113\\
1061	-43.9069444456695\\
1062	-45.3546419180313\\
1063	-69.9641284111883\\
1064	-55.0530575697971\\
1065	-46.0432034720193\\
1066	-60.7880006448361\\
1067	-70.7567064927293\\
1068	-112.253803278054\\
1069	-102.853962816952\\
1070	-125.294858217704\\
1071	-118.904254989641\\
1072	-133.148050355062\\
1073	-116.548415000917\\
1074	-105.540527313905\\
1075	-104.193815539815\\
1076	-92.6366903790745\\
1077	-92.2519443095835\\
1079	-209.822644358566\\
1080	-217.849891269364\\
1081	-211.566701849692\\
1082	-153.391535734028\\
1084	-241.245223278482\\
1085	-242.317872234928\\
1086	-213.26322134249\\
1087	-245.282425215254\\
1088	-323.195508377903\\
1089	-256.727340541164\\
1090	-212.088830407031\\
1091	-129.527625974475\\
1092	-105.090779045958\\
1093	-87.1458515885047\\
1094	-112.984033129884\\
1096	-48.5444204321127\\
1097	-44.3726265114065\\
1098	-63.5094015675097\\
1099	-94.9543774647227\\
1100	-92.1382351888944\\
1101	-88.1705346127071\\
1102	-121.74009892591\\
1103	-117.096617264499\\
1104	-173.405606127332\\
1105	-134.193839582948\\
1106	-167.044629411646\\
1107	-130.604471349407\\
1108	-117.338125798235\\
1109	-127.977138920087\\
1110	-101.146071292675\\
1111	-87.8078518346974\\
1112	-100.931967316156\\
1113	-103.222179606532\\
1114	-73.6259018290248\\
1115	-78.9890154295074\\
1116	-59.5127843391342\\
1117	-65.4611995661785\\
1118	-70.2043279253023\\
1119	-52.3221081674385\\
1120	-58.2443421333717\\
1121	-78.9137784709569\\
1122	-119.01521223404\\
1123	-116.82738587693\\
1124	-131.366591762202\\
1125	-80.9890599924342\\
1126	-118.962816059881\\
1127	-169.136832094591\\
1128	-136.457950037825\\
1129	-157.684924880299\\
1130	-149.504678118703\\
1131	-98.0349507461724\\
1132	-66.5818280923593\\
1133	-83.4522543937835\\
1134	-106.324036373426\\
1135	-155.667978578261\\
1136	-191.236930968256\\
1137	-187.342310244252\\
1138	-154.893484232252\\
1140	-145.033737751978\\
1141	-134.99767580854\\
1142	-141.399175478924\\
1143	-89.2570147359261\\
1144	-79.8932073108936\\
1145	-76.1735163550127\\
1146	-66.8338939302112\\
1147	-78.8510560102859\\
1148	-104.741631385721\\
1149	-162.196778886423\\
1151	-96.4144920442061\\
1152	-78.2802583683822\\
1153	-77.0547592642918\\
1154	-44.76499460632\\
1155	-25.653485028593\\
1156	-39.2200475941047\\
1157	-81.6222019401309\\
1158	-53.7297318052154\\
1159	-64.7262535940488\\
1160	-71.5954220969495\\
1161	-64.7714626980026\\
1163	-32.4882764903286\\
1164	-21.5307700916583\\
1165	-45.5904029383985\\
1166	-107.919569851239\\
1167	-138.483991592694\\
1168	-154.77033156541\\
1170	-73.2789566326626\\
1172	-28.8280786008449\\
1173	-92.2599574627191\\
1174	-82.1105002322374\\
1175	-115.48557065217\\
1176	-141.755103221621\\
1177	-221.464965880117\\
1178	-263.255484002109\\
1179	-222.236870010612\\
1180	-203.084769959425\\
1181	-146.697399910002\\
1182	-159.736396149544\\
1183	-162.262546302067\\
1184	-172.431176309208\\
1185	-138.892616522109\\
1186	-127.442464622286\\
1187	-119.646433839042\\
1188	-116.630375599647\\
1189	-121.97112444744\\
1190	-98.261077858471\\
1192	-184.894844093207\\
1193	-132.560180864139\\
1194	-92.0177743861095\\
1195	-77.222059330353\\
1196	-162.289027132403\\
1197	-166.538326254517\\
1198	-214.851057141577\\
1199	-222.433778890919\\
1200	-236.13332586401\\
1201	-195.782659584486\\
1202	-170.285840544235\\
1203	-199.301715661468\\
1204	-202.218777858797\\
1206	-92.8752900843965\\
1207	-120.572509184559\\
1208	-119.465380217355\\
1209	-66.0422382547547\\
1210	-93.9450911278238\\
1211	-100.951038641667\\
1213	-67.4086351152494\\
1214	-84.1728457476922\\
1215	-73.6731645298755\\
1216	-96.3001011784906\\
1217	-132.165288799423\\
1218	-127.217962705065\\
1219	-80.9269882209019\\
1220	-95.8918524145158\\
1221	-121.399374780818\\
1222	-212.661022286512\\
1224	-103.890004625111\\
1225	-70.4268189014592\\
1226	-87.7851516439537\\
1227	-84.8798954441261\\
1228	-62.4558196206949\\
1229	-61.9719460283341\\
1230	-84.7148265578585\\
1231	-100.836710310483\\
1232	-101.490618817484\\
1233	-79.6952425633551\\
1234	-132.431977249603\\
1235	-140.441171507532\\
1236	-131.903351354189\\
1237	-116.353926510348\\
1238	-123.459659086731\\
1239	-154.829391130548\\
1240	-104.629057255707\\
1241	-34.6202571465055\\
1242	-61.8112373681786\\
1243	-75.1158990994556\\
1244	-98.4049015360501\\
1245	-84.8741080811419\\
1246	-66.2634691803948\\
1247	-102.043590633894\\
1248	-90.6170728131585\\
1249	-73.3895725644281\\
1250	-87.2376837351246\\
1251	-81.7069726665418\\
1252	-73.7371152605556\\
1253	-84.2012845186184\\
1254	-49.9317923560145\\
1255	-57.338237619352\\
1256	-47.1221352295984\\
1257	-48.242891227387\\
1258	-96.3956923644218\\
1259	-95.8802564526316\\
1260	-138.059195070432\\
1261	-171.820816416924\\
1262	-124.493267851625\\
1263	-66.1853992898623\\
1264	-54.3719588237427\\
1265	-55.8920041576744\\
1266	-81.779943693422\\
1267	-44.4221766330497\\
1269	-71.8288474244227\\
1270	-127.386058094289\\
1271	-107.122571136299\\
1272	-143.258454654687\\
1273	-109.028999235571\\
1274	-94.4553665322935\\
1275	-45.6410463519139\\
1276	-49.9814421916858\\
1277	-26.8309807321364\\
1278	-38.456440064982\\
1279	-46.659184913796\\
1280	-82.5308020270234\\
1281	-78.0596081319609\\
1282	-98.6438090099748\\
1283	-98.0723482562737\\
1284	-148.348362759229\\
1285	-139.535796951948\\
1286	-106.859081138705\\
1287	-120.719488012673\\
1288	-123.846305644005\\
1289	-158.170916151251\\
1290	-139.124098920907\\
1291	-94.9870525173558\\
1292	-33.3339910304649\\
1293	-28.0481767149745\\
1294	-34.3027666756859\\
1295	-42.8329299555867\\
1296	-45.8252960836501\\
1297	-43.3778531663429\\
1298	-58.0595616087503\\
1299	-93.3431708863088\\
1300	-81.3432279794224\\
1301	-82.7041166393224\\
1302	-99.8715902755916\\
1303	-56.7295568308555\\
1304	-27.2302248767378\\
1306	-102.032647990875\\
1307	-93.5842088629076\\
1308	-109.722117687422\\
1309	-96.1046440044372\\
1310	-72.248596280534\\
1311	-100.421622463213\\
1312	-134.933765101069\\
1313	-136.033960401415\\
1314	-100.153525055106\\
1315	-157.486541589797\\
1316	-132.969115197636\\
1317	-124.787547153505\\
1318	-135.028774900119\\
1319	-131.018379096665\\
1320	-116.012571572406\\
1321	-93.2350009496454\\
1322	-153.705022016753\\
1323	-198.019612662049\\
1324	-168.544161887842\\
1325	-101.779346783535\\
1326	-92.0631172109181\\
1327	-102.068387823065\\
1328	-118.665533949999\\
1329	-132.693300660786\\
1330	-103.849967304143\\
1331	-110.628173283864\\
1332	-135.741623131085\\
1333	-143.610799252852\\
1334	-109.755204889944\\
1335	-84.9757478332269\\
1336	-117.252249074205\\
1337	-167.313666590329\\
1338	-161.525247519645\\
1339	-153.350931084431\\
1340	-99.7265918495139\\
1341	-87.1179205535993\\
1342	-99.5317067763363\\
1343	-45.8081549759977\\
1345	-16.9786157926785\\
1346	-21.0031933360519\\
1347	-66.2543104455544\\
1348	-97.4445889135773\\
1349	-99.1578757891771\\
1350	-80.8082088100655\\
1352	-100.123350339352\\
1353	-64.5223129099386\\
1354	-63.6077562788169\\
1355	-62.1416347447639\\
1356	-49.9710324357388\\
1357	-62.6954300786826\\
1358	-96.6288740959994\\
1359	-121.920664839877\\
1360	-83.4796156561742\\
1361	-57.8611922520636\\
1362	-58.222299723451\\
1363	-58.9968691056617\\
1364	-42.4329382228152\\
1365	-97.2088141045133\\
1366	-161.592703461824\\
1367	-121.258210049495\\
1368	-159.978295152543\\
1369	-185.134229429822\\
1370	-196.800613902859\\
1371	-146.536868523144\\
1372	-117.22357437187\\
1373	-113.094373854753\\
1374	-115.234907782929\\
1375	-121.813108650407\\
1376	-137.348025340792\\
1377	-146.128170602716\\
1380	-248.936318538679\\
1381	-200.168546226462\\
1382	-186.425922440186\\
1383	-217.612372039376\\
1384	-167.165737055849\\
1385	-192.895083089467\\
1386	-165.794211046131\\
1387	-96.8744653360529\\
1388	-50.7891040320105\\
1389	-61.3843116528926\\
1390	-100.776738459086\\
1392	-54.7096231140463\\
1393	-71.9391057271407\\
1394	-63.2131875393347\\
1395	-34.6890319612\\
1396	-57.0605385377814\\
1397	-57.8248947627042\\
1398	-43.2169228382297\\
1399	-86.8821649890692\\
1400	-104.441900254321\\
1401	-84.6730877131831\\
1403	-191.901543836301\\
1404	-179.831343200195\\
1405	-203.663394806748\\
1406	-164.917630230929\\
1407	-167.273830004609\\
1408	-124.092002972129\\
1409	-65.0298143759528\\
1410	-92.7776053565963\\
1411	-69.4454754225476\\
1412	-58.0390595464789\\
1413	-72.1885548387768\\
1414	-35.3482304195129\\
1415	-23.6725652617351\\
1416	-32.2902446963924\\
1417	-95.7471969464934\\
1418	-108.796650855723\\
1419	-125.560574705019\\
1420	-124.294916600536\\
1421	-117.082394057727\\
1422	-136.547323265747\\
1423	-112.902827766254\\
1424	-98.7931114849382\\
1425	-100.550567420432\\
1426	-80.1611734283558\\
1427	-113.565109843263\\
1429	-65.0128936840165\\
1430	-105.511717971018\\
1431	-130.246207425763\\
1432	-111.638557949629\\
1433	-80.6754190338231\\
1434	-75.1785710369295\\
1435	-74.8616875400583\\
1436	-60.4135634480394\\
1437	-50.1567517750557\\
1438	-72.5739436568649\\
1439	-89.802244465395\\
1440	-97.8779226487584\\
1441	-81.8829444629087\\
1442	-100.707477117174\\
1443	-97.4703447047473\\
1444	-55.82823704357\\
1445	-62.7481509203903\\
1446	-115.617017454842\\
1447	-150.013915992626\\
1448	-147.317470901811\\
1449	-131.12562251367\\
1450	-129.260419576501\\
1451	-102.575354978645\\
1452	-92.9035264309298\\
1453	-80.0002832130986\\
1454	-97.9469108008375\\
1455	-98.1544350495985\\
1456	-58.1222082591387\\
1457	-86.9872914005464\\
1458	-102.937888455678\\
1459	-109.669265925073\\
1460	-131.404047898966\\
1461	-120.600441565501\\
1462	-171.244498509571\\
1464	-236.136449865088\\
1465	-176.49380142577\\
1466	-82.6024227464652\\
1468	-28.0274688320521\\
1469	-61.9997457388242\\
1470	-54.6930540368933\\
1471	-96.755385103055\\
1472	-95.4214519495711\\
1473	-79.4968299112211\\
1474	-104.570931770457\\
1475	-116.262949381394\\
1476	-133.656724873857\\
1477	-146.507961667717\\
1478	-202.798668956662\\
1479	-193.846775094304\\
1481	-97.1929842855075\\
1482	-97.3495744355509\\
1483	-101.761314489986\\
1484	-120.455792909437\\
1485	-91.7163389359903\\
1486	-96.3419722504266\\
1487	-89.3107242868273\\
1488	-52.9950906631816\\
1490	-121.860101847493\\
1491	-97.3441468903761\\
1492	-95.8069998500978\\
1493	-173.795916497962\\
1494	-134.434205977568\\
1495	-124.331930733344\\
1496	-166.098005816097\\
1497	-164.086481947059\\
1498	-118.49942936905\\
1499	-62.3492878506984\\
1500	-72.9440851274558\\
1502	-178.848343275708\\
1503	-192.609245446009\\
1504	-189.035749243761\\
1505	-157.926871531038\\
1506	-109.103842831946\\
1507	-84.5392253833274\\
1508	-66.6106205532938\\
1509	-57.453786608547\\
1510	-56.4503755286164\\
1511	-71.6649041778376\\
1512	-83.1895588459847\\
1513	-47.7217033450797\\
1514	-80.903100259183\\
1516	-114.991364205161\\
1517	-143.311098355966\\
1519	-219.196728411464\\
1520	-174.662765043856\\
1521	-146.822757418025\\
1522	-149.12266235735\\
1523	-132.356228741605\\
1524	-87.3712149599398\\
1525	-110.862558311093\\
1526	-153.523129445167\\
1527	-184.460995152376\\
1529	-63.5044182570023\\
1530	-36.3364427941826\\
1531	-2.4709507559171\\
1532	-30.3444455257056\\
1533	-51.8644559746708\\
1534	-62.2579671484486\\
1535	-85.4249558969293\\
1536	-71.982037461117\\
1537	-53.9580680467991\\
1538	-57.9242927438377\\
1539	-52.3099306979861\\
1540	-49.1804583341298\\
1541	-54.4324848222811\\
1543	-121.274764018879\\
1544	-133.211921882614\\
1545	-127.201540525137\\
1546	-148.286478209233\\
1547	-173.319266764375\\
1548	-182.906480942369\\
1549	-212.420282123998\\
1550	-215.408760011563\\
1551	-155.421445583992\\
1552	-108.193307024137\\
1553	-100.328734361177\\
1554	-168.350582526255\\
1555	-177.147142119666\\
1556	-142.238855883815\\
1557	-89.9598526813249\\
1558	-28.0724816974252\\
1559	-24.1783512337963\\
1560	-34.7242980305073\\
1561	-11.5925658406188\\
1562	-52.2785230748023\\
1563	-74.232792898106\\
1564	-67.2811975201025\\
1565	-67.5918449720746\\
1566	-34.4657685559746\\
1567	-32.4378736176793\\
1568	-42.5076216915809\\
1569	-48.8550394874776\\
1570	-48.1369600961134\\
1571	-37.5372219519488\\
1572	-31.1444730706464\\
1573	-61.5939458605276\\
1574	-140.480052591605\\
1575	-161.61028804959\\
1576	-166.794068647675\\
1577	-136.221299935249\\
1578	-164.47233241759\\
1579	-237.326612708478\\
1580	-220.861205011653\\
1581	-216.813759860506\\
1582	-181.67569778438\\
1583	-168.153751267672\\
1584	-173.624919799643\\
1585	-122.89188370037\\
1586	-114.243429256006\\
1587	-59.7279424942255\\
1588	-62.6810015828614\\
1589	-116.842675840091\\
1590	-80.4296152653992\\
1591	-85.6703869366227\\
1592	-101.151497123296\\
1593	-92.2716319082108\\
1594	-91.5628858119439\\
1595	-96.0633919756945\\
1596	-108.811335457291\\
1597	-117.58540704756\\
1598	-110.000311354392\\
1599	-85.0812984126765\\
1600	-96.7760157197927\\
1602	-6.02155906256576\\
1603	-31.518606802083\\
1604	-61.871205717993\\
1605	-23.563528605833\\
1606	-9.73925109996981\\
1607	-28.4240529511198\\
1608	-62.2997549901945\\
1609	-68.9776481983213\\
1610	-93.0333765621147\\
1611	-74.4631741009953\\
1612	-122.923333247456\\
1613	-108.049846201927\\
1614	-72.6310729346678\\
1615	-111.464629474083\\
1616	-53.361070181865\\
1617	-50.2490382169299\\
1618	-28.6887728716194\\
1619	-17.9085056232207\\
1620	-41.8599199522316\\
1621	-28.7342150052173\\
1623	-92.9873573769496\\
1624	-87.5460643949639\\
1625	-67.9305327216507\\
1626	-66.8642542219077\\
1627	-54.0660686397575\\
1628	-74.3495373456813\\
1629	-51.6494809063454\\
1630	-51.2332132508134\\
1631	-49.1311887886679\\
1632	-36.7558871562694\\
1633	-50.0814389582038\\
1634	-43.7292194885215\\
1635	-77.0477469105563\\
1636	-70.7766034148733\\
1637	-56.4624060656954\\
1638	-52.1946060054079\\
1639	-42.9748627975973\\
1640	-51.5817429749243\\
1641	-105.070920229715\\
1642	-145.629728149747\\
1643	-98.4054867313889\\
1644	-98.2609889602768\\
1645	-60.5755499246065\\
1646	-110.78473663893\\
1647	-106.768572266423\\
1648	-64.8576959571271\\
1649	-85.3244602741147\\
1650	-62.9944237381553\\
1651	-90.3114959845407\\
1652	-53.2557239325147\\
1653	-51.3512513702196\\
1654	-73.26360232876\\
1655	-80.8875837758164\\
1656	-63.4470677720956\\
1657	-64.4363163645892\\
1658	-75.6291258632093\\
1659	-106.832238108993\\
1660	-100.846056776882\\
1661	-136.625910457362\\
1662	-190.062948008012\\
1663	-151.842551011614\\
1664	-105.306804564792\\
1665	-71.5367718293021\\
1666	-114.569733509368\\
1667	-136.784975200353\\
1668	-112.586089394\\
1669	-127.169507971154\\
1670	-88.174019796433\\
1671	-33.9316647034241\\
1672	-44.4090340057683\\
1673	-102.353523716835\\
1674	-92.710146542315\\
1675	-77.4316551476263\\
1676	-124.882993876781\\
1677	-113.199300266322\\
1678	-109.100241363351\\
1679	-69.1367577852084\\
1680	-86.0814390951555\\
1681	-119.174440272398\\
1682	-94.7098787049442\\
1683	-98.7531073355046\\
1684	-91.9700298450571\\
1685	-67.9209628311391\\
1686	-78.1483052612373\\
1687	-56.7702454839825\\
1688	-69.30088425237\\
1689	-101.961940256909\\
1690	-118.568458569538\\
1691	-119.927216749009\\
1692	-120.993857367661\\
1693	-87.6786477696849\\
1694	-63.2837729758794\\
1695	-71.3112190177737\\
1696	-85.3099688281186\\
1697	-102.414268974657\\
1698	-124.503632040554\\
1699	-143.094019069712\\
1700	-117.931771084312\\
1701	-77.8801026338181\\
1702	-125.225985985148\\
1703	-181.078347171007\\
1704	-126.651572480675\\
1705	-122.735864115802\\
1706	-143.29386195501\\
1707	-123.397375497553\\
1708	-93.9737925849709\\
1709	-107.536711721088\\
1710	-93.3596668421208\\
1711	-105.365487698898\\
1712	-128.040892884316\\
1713	-101.698385888866\\
1714	-113.086231407315\\
1715	-112.716922540471\\
1716	-107.571518943528\\
1717	-96.8549890899096\\
1718	-112.266470917846\\
1719	-85.2195399616492\\
1720	-90.2426787534262\\
1721	-100.876197871885\\
1722	-93.0453878109076\\
1723	-41.0272909692217\\
1724	-22.1235855928576\\
1725	-7.30680888400434\\
1726	-40.5386970134409\\
1727	-103.942614576183\\
1728	-133.143137049338\\
1729	-122.655213569867\\
1730	-91.3978582373982\\
1731	-98.6256580154857\\
1733	-81.963681956955\\
1734	-49.5463105804067\\
1735	-65.2163668445676\\
1736	-66.5451920679757\\
1737	-65.2278431537297\\
1738	-77.4630191953013\\
1739	-69.5730459114914\\
1740	-65.6408604831383\\
1741	-39.8169301861794\\
1742	-61.0352795116221\\
1743	-112.788561280926\\
1744	-74.9472714709102\\
1745	-97.0061223871207\\
1746	-84.7475585950012\\
1747	-113.808663691861\\
1748	-114.87071992382\\
1749	-136.665633518564\\
1750	-119.968613052142\\
1751	-117.964376400702\\
1752	-126.199775677046\\
1753	-68.1155618270964\\
1754	-110.311978885582\\
1755	-85.5072562940486\\
1756	-94.9759897713882\\
1757	-98.3997243807416\\
1758	-121.535805148937\\
1759	-100.51454216192\\
1761	-146.390886331062\\
1762	-140.239189870437\\
1763	-111.812236682974\\
1764	-102.307928252149\\
1765	-108.497060061745\\
1766	-92.4226622083233\\
1768	-73.5635716490294\\
1769	-83.0833252101904\\
1770	-77.7418893308106\\
1772	-174.427882112191\\
1773	-149.385981672279\\
1774	-151.785358361117\\
1775	-147.236710654958\\
1776	-85.79702752652\\
1777	-76.6385707057163\\
1779	-54.9659580643836\\
1780	-59.0199209809796\\
1781	-93.8420194311184\\
1782	-113.097314423909\\
1783	-127.889493618698\\
1784	-117.378962348783\\
1785	-80.7481692860817\\
1786	-134.479411476903\\
1787	-164.43452953136\\
1788	-167.785502882258\\
1789	-154.59613790062\\
1790	-217.137248625196\\
1791	-176.118513462075\\
1792	-100.975173035974\\
1793	-62.8463207293983\\
1794	-61.2201849998698\\
1795	-103.86382075351\\
1797	-160.548009350228\\
1799	-108.212898376918\\
1800	-57.3989378924671\\
1801	-53.6328071054504\\
1802	-71.2891839837334\\
1803	-74.9243221500983\\
1804	-48.5423790903035\\
1805	-58.3919941835481\\
};
\addlegendentry{OSA predition}

\addplot [color=mycolor3, dotted, line width=2.0pt]
  table[row sep=crcr]{%
1006	-147.705\\
1007	-175.781\\
1008	-134.277\\
1009	-68.3589999999999\\
1010	-85.4686037505746\\
1011	-91.6047705115723\\
1012	-108.027257417993\\
1013	-92.8567005136913\\
1014	-29.4787978305205\\
1015	-9.65065853579017\\
1016	-9.65601966726103\\
1017	-80.5662900232217\\
1018	-134.218821102751\\
1019	-124.71863850667\\
1020	-140.311894734878\\
1021	-93.0218308250635\\
1022	-64.2554583511944\\
1023	-135.84533894183\\
1024	-92.5811203992041\\
1025	-74.4938168879769\\
1026	-117.457135870964\\
1027	-104.534738888153\\
1028	-96.3946435329667\\
1029	-137.389133869682\\
1030	-137.516094707514\\
1031	-102.217499616648\\
1032	-146.317336833964\\
1033	-141.236231088326\\
1034	-117.406666053946\\
1035	-101.104238140385\\
1036	-78.2894051922392\\
1038	-112.7924640166\\
1039	-101.65817957799\\
1040	-109.47198137959\\
1041	-110.376680299447\\
1042	-119.034252423239\\
1043	-162.511015239933\\
1044	-146.234453087022\\
1045	-110.031326719743\\
1046	-99.3858683527071\\
1047	-52.4080330590375\\
1048	-60.7548669179421\\
1049	-82.1602676344057\\
1050	-69.3158642104854\\
1051	-65.5669647077632\\
1052	-95.4270829150851\\
1053	-104.114798979683\\
1054	-99.0305732132258\\
1055	-147.940853728755\\
1056	-114.721012212928\\
1057	-71.7046193150491\\
1058	-58.0451645461296\\
1060	-92.4073927228053\\
1061	-49.8822151212419\\
1062	-49.7825236879337\\
1063	-73.6476539826897\\
1064	-59.5242388024083\\
1065	-50.9003867591412\\
1066	-65.8315263624886\\
1067	-74.7879199783238\\
1068	-116.170233857737\\
1069	-107.16490078462\\
1070	-127.818299000124\\
1071	-119.38229592331\\
1072	-133.148796875553\\
1073	-114.538894116459\\
1074	-106.639083753073\\
1075	-104.146467440436\\
1076	-96.8732509325853\\
1077	-96.2749666863904\\
1079	-211.938461741868\\
1080	-214.237235857703\\
1081	-203.290900441504\\
1082	-141.261486525997\\
1083	-191.585410843642\\
1084	-231.876372792619\\
1085	-229.889164533849\\
1086	-196.719761113574\\
1087	-236.407536070844\\
1088	-302.426703258745\\
1089	-234.957327219621\\
1090	-205.191544265953\\
1091	-132.395222574953\\
1092	-105.481879018029\\
1093	-89.9859676841279\\
1094	-117.672627907726\\
1096	-55.24204424441\\
1097	-48.2268436263316\\
1098	-65.3117973834549\\
1099	-97.0833585201487\\
1100	-93.5882999337771\\
1101	-91.2892047476346\\
1102	-123.82068762579\\
1103	-119.783953532074\\
1104	-173.443567970009\\
1105	-136.828108350636\\
1106	-168.954230401223\\
1107	-130.56002864417\\
1108	-121.41304345593\\
1109	-134.540328276548\\
1110	-106.548839030864\\
1111	-97.2131805145191\\
1112	-111.601992157591\\
1113	-112.795087460492\\
1114	-83.2369896427463\\
1115	-88.876017843183\\
1116	-69.2827446358872\\
1117	-75.0139503561354\\
1118	-79.7469512389862\\
1119	-62.0576409644846\\
1120	-68.8582971146327\\
1121	-87.3774749237416\\
1122	-129.868040952789\\
1123	-126.594977555435\\
1124	-137.316331399899\\
1125	-83.6032706056524\\
1126	-123.278323875847\\
1127	-175.989790010638\\
1128	-139.843355928618\\
1129	-163.25507138857\\
1130	-154.411934298005\\
1131	-97.8661574754112\\
1132	-71.9279295730171\\
1134	-110.301127843215\\
1135	-163.721408583301\\
1136	-195.667138762571\\
1137	-186.745496658829\\
1138	-150.271287897101\\
1139	-150.446281314915\\
1140	-144.720783916804\\
1141	-137.805126816111\\
1142	-145.5671605497\\
1143	-97.7904074744993\\
1144	-88.3046887579419\\
1145	-83.1045897545282\\
1146	-75.7979873220438\\
1147	-87.7444313694837\\
1148	-115.496949777659\\
1149	-170.738091671067\\
1150	-133.252844902431\\
1151	-102.23139106541\\
1152	-86.4342280036908\\
1153	-85.0978219326723\\
1154	-52.8100221311754\\
1155	-32.6730406391232\\
1156	-42.0225360517713\\
1157	-85.0313363397388\\
1158	-60.4428335061632\\
1159	-67.735980716109\\
1160	-76.173910697594\\
1161	-70.8266214193475\\
1163	-38.0120846841512\\
1164	-26.9817186790156\\
1165	-48.1205868509653\\
1166	-111.79130705908\\
1167	-146.315062956166\\
1168	-159.741717438628\\
1169	-116.954820848254\\
1170	-81.8228756020769\\
1171	-60.0092275304748\\
1172	-34.1909683358444\\
1173	-95.725306039822\\
1174	-89.2307199161271\\
1175	-120.962787486691\\
1176	-144.711405692196\\
1177	-223.979056843691\\
1178	-261.417677937934\\
1179	-221.019674047526\\
1180	-206.418783172139\\
1181	-148.940622977912\\
1182	-165.418010868716\\
1183	-170.831671539526\\
1184	-180.02379616507\\
1185	-145.417324955775\\
1186	-136.795453263469\\
1187	-130.620914258518\\
1188	-126.379986981066\\
1189	-135.68339552366\\
1190	-108.538094091439\\
1191	-156.395875767613\\
1192	-192.242109463641\\
1193	-136.699103701596\\
1194	-95.9637349460629\\
1195	-85.2193531152861\\
1196	-165.138721443571\\
1197	-176.523968764176\\
1198	-216.720351174206\\
1199	-218.771208915486\\
1200	-226.327983263188\\
1201	-185.645648668569\\
1202	-168.408141226146\\
1203	-198.893851152826\\
1204	-206.320224061865\\
1205	-142.566258359785\\
1206	-92.3070951871255\\
1207	-126.249236628219\\
1208	-121.120088962143\\
1209	-74.5592939087821\\
1210	-101.059265556851\\
1211	-108.660899028457\\
1213	-76.2473671720902\\
1214	-95.068579452553\\
1215	-83.3046605193463\\
1216	-107.961178133366\\
1217	-144.255680625751\\
1218	-136.899311866925\\
1219	-90.6764701422444\\
1220	-104.385125357734\\
1221	-133.610059654144\\
1222	-221.327619946146\\
1224	-114.418815171157\\
1225	-81.8709082612691\\
1226	-96.7950078597885\\
1227	-94.2354400565598\\
1228	-74.6504661856272\\
1229	-73.5134953796644\\
1231	-113.593184151233\\
1232	-113.620015214967\\
1233	-86.2596328992524\\
1234	-142.465752401784\\
1235	-152.860935122959\\
1236	-139.90394603898\\
1237	-120.701025455324\\
1238	-132.084856319116\\
1239	-163.33261848046\\
1240	-108.999879956698\\
1241	-39.124584653661\\
1242	-63.0863575498852\\
1243	-77.2260076726686\\
1244	-104.129184549593\\
1245	-91.455445382242\\
1246	-72.6010533929959\\
1247	-110.425501263298\\
1248	-99.7046348648748\\
1249	-79.8827095282488\\
1250	-95.8815118761283\\
1251	-90.4170540078476\\
1252	-82.1930001670444\\
1253	-94.0500872081595\\
1254	-59.2105550992487\\
1255	-66.0176840380182\\
1256	-55.3583382886147\\
1257	-57.636259994802\\
1258	-106.157433776608\\
1259	-108.14146262221\\
1261	-180.569570534273\\
1263	-72.9296273767034\\
1264	-58.7856693840797\\
1265	-60.6682042397006\\
1266	-89.4109010916434\\
1267	-53.128393012325\\
1268	-63.4770419941165\\
1269	-79.0905989821133\\
1270	-134.477622163534\\
1271	-115.967809704117\\
1272	-150.608328770823\\
1273	-111.915864812322\\
1274	-100.087857875543\\
1275	-51.1633682965007\\
1276	-52.3473484914064\\
1277	-29.8533853555905\\
1279	-48.9455376157009\\
1280	-86.4228888348368\\
1281	-83.5585836400367\\
1282	-101.324096453653\\
1283	-103.237708768236\\
1284	-152.778554292585\\
1285	-141.198508271015\\
1286	-112.387279895537\\
1287	-128.650062349794\\
1288	-130.004875840347\\
1289	-162.003671961967\\
1290	-139.157963176215\\
1291	-97.7234805362932\\
1292	-39.9848604460738\\
1293	-27.2499487027387\\
1294	-32.1041635958179\\
1295	-42.7342391127661\\
1296	-45.5200383520498\\
1297	-43.6298842968301\\
1298	-59.4733506380983\\
1299	-95.359854537229\\
1300	-85.2237345731473\\
1301	-86.5440759829919\\
1302	-103.392309945473\\
1303	-60.9964443218526\\
1304	-29.481652132361\\
1306	-104.067464661087\\
1307	-96.7697962604066\\
1308	-110.888756647637\\
1309	-97.2186585983559\\
1310	-74.3340263764494\\
1311	-102.848963162292\\
1312	-138.04617126224\\
1313	-137.517662857851\\
1314	-100.243708281599\\
1315	-158.862323462649\\
1316	-131.779654254144\\
1317	-125.82729671\\
1318	-138.569600634872\\
1319	-132.078513143968\\
1321	-97.5692237078729\\
1322	-157.596555850099\\
1323	-201.302535156389\\
1324	-165.283108660214\\
1325	-101.00974311065\\
1326	-96.0000049311811\\
1327	-103.151482382495\\
1328	-124.347731724666\\
1329	-139.669202871685\\
1330	-107.329262485936\\
1331	-115.636352475229\\
1332	-142.306215311941\\
1333	-146.627024998731\\
1334	-108.229819962204\\
1335	-86.946504094579\\
1336	-120.832326686338\\
1337	-174.543102493447\\
1338	-164.285035715383\\
1339	-157.157155758977\\
1340	-99.7298735506627\\
1341	-88.111775067561\\
1342	-102.068765178198\\
1343	-53.5090092991281\\
1345	-17.0062078762746\\
1346	-18.4732000542047\\
1347	-63.2171046798585\\
1348	-97.1136200538156\\
1349	-102.085390474884\\
1350	-80.6004546604984\\
1352	-103.15854540047\\
1353	-67.0839304308627\\
1354	-67.6997588594249\\
1355	-65.2678354741852\\
1356	-52.8486504038406\\
1357	-67.2263488374078\\
1358	-102.344592268619\\
1359	-126.314188477284\\
1360	-86.1950781446701\\
1361	-62.4150645459645\\
1362	-62.2499674225703\\
1363	-66.580583006443\\
1364	-47.9529451547312\\
1365	-102.647737187988\\
1366	-169.517842341099\\
1367	-128.424495489548\\
1368	-165.903019608553\\
1369	-187.634038106004\\
1370	-195.756965827951\\
1371	-145.442141486642\\
1372	-118.822085461991\\
1373	-118.274874161837\\
1374	-121.794829054061\\
1375	-131.484053397151\\
1376	-146.029743939117\\
1377	-148.318263839049\\
1378	-182.374423172499\\
1379	-205.15453260821\\
1380	-239.824375666554\\
1381	-187.046715289805\\
1382	-184.181530116077\\
1383	-213.106313519371\\
1384	-164.5686560897\\
1385	-193.841555962699\\
1386	-167.238541599867\\
1387	-99.0582388265825\\
1388	-54.1915717556835\\
1389	-60.9181671364529\\
1390	-100.296842857275\\
1391	-81.5514138066587\\
1392	-57.2659629590967\\
1393	-75.1395078623898\\
1394	-65.8648687635366\\
1395	-41.1267420573568\\
1396	-59.6880272580552\\
1397	-63.1515773807826\\
1398	-48.6876662908096\\
1399	-92.1035495985368\\
1400	-113.432188303965\\
1401	-89.2383758912795\\
1403	-199.046703013571\\
1404	-184.315985368434\\
1405	-208.893066641537\\
1406	-161.399990876133\\
1407	-170.80757636587\\
1408	-129.970051547803\\
1409	-72.3413347065205\\
1410	-98.5656374385098\\
1411	-76.4218233371055\\
1412	-65.3881249030426\\
1413	-81.0767622252704\\
1414	-42.32805735944\\
1415	-25.5856820581755\\
1416	-33.9053398195997\\
1417	-95.5986286811853\\
1418	-115.217061973091\\
1419	-131.158819339393\\
1420	-128.122740429105\\
1421	-121.225790263112\\
1422	-139.883390489676\\
1423	-115.169447441689\\
1424	-101.348684642178\\
1425	-106.085865452844\\
1426	-88.4022718245612\\
1427	-122.892334407663\\
1428	-94.4855048665036\\
1429	-74.4321386670329\\
1430	-112.43702859405\\
1431	-138.358348287812\\
1432	-115.495696145196\\
1433	-86.6344790979774\\
1434	-81.8755262342743\\
1435	-83.3866556443411\\
1436	-67.4980707769028\\
1437	-59.0642127018309\\
1438	-80.3142204202277\\
1439	-96.3115957337184\\
1440	-105.193358536218\\
1441	-86.9831352662293\\
1442	-106.042855325077\\
1443	-101.6386591744\\
1444	-60.7277589165312\\
1445	-68.0253391538045\\
1446	-123.165259157016\\
1447	-157.996573006865\\
1448	-150.807845597908\\
1449	-131.93951375\\
1450	-130.955843885811\\
1451	-106.406605455135\\
1452	-99.3778102982042\\
1453	-86.3165326586013\\
1454	-107.103987783564\\
1455	-103.847012598497\\
1456	-65.4517216441086\\
1457	-94.6968881546143\\
1458	-109.813873530908\\
1459	-116.626920392125\\
1460	-133.737880676677\\
1461	-123.793337902353\\
1462	-170.144962260344\\
1463	-202.266037188648\\
1464	-229.398610018034\\
1465	-167.713841203445\\
1466	-86.1664065736006\\
1468	-26.9398378377582\\
1469	-59.3412619413714\\
1470	-54.2995781345203\\
1471	-96.5603827524781\\
1472	-95.1362070385919\\
1473	-83.8520267284898\\
1474	-109.245606223134\\
1475	-120.898111016488\\
1476	-137.241870268561\\
1477	-146.475263383136\\
1478	-202.243103901877\\
1479	-190.639450624871\\
1481	-103.499300802691\\
1482	-103.817293603741\\
1483	-109.146664707821\\
1484	-129.690031659257\\
1485	-97.8304814412013\\
1486	-102.882349191968\\
1487	-97.7977023096157\\
1488	-59.3674577083148\\
1490	-130.933571209137\\
1491	-101.886523429186\\
1492	-102.823769985178\\
1493	-179.494115563346\\
1494	-139.983268382308\\
1495	-131.32517629978\\
1496	-172.604941933744\\
1497	-165.146115419852\\
1499	-69.9881923200955\\
1500	-76.8647498628577\\
1502	-188.744277685394\\
1503	-195.969170960677\\
1504	-190.963797710269\\
1505	-152.818985022547\\
1506	-107.969822484255\\
1508	-68.9585833011088\\
1509	-62.6517770658018\\
1510	-61.470056733667\\
1511	-79.1887009662082\\
1512	-90.7454900432735\\
1513	-53.6539331458014\\
1514	-86.9132642306531\\
1515	-107.613480129708\\
1516	-120.083528388404\\
1517	-148.565327082258\\
1519	-220.174077409775\\
1520	-173.974973723369\\
1521	-151.913835124236\\
1522	-156.326509305249\\
1523	-139.319014877946\\
1524	-97.8500863103925\\
1525	-120.619082033154\\
1526	-164.933294759423\\
1527	-189.136985430906\\
1528	-120.601803583157\\
1529	-66.4818394682304\\
1530	-38.7838218565917\\
1531	-0.955754744427395\\
1532	-21.1861002777919\\
1533	-45.2491360461327\\
1534	-56.7279563412376\\
1535	-79.6886111704237\\
1536	-69.6468651103696\\
1537	-54.1318084450913\\
1538	-58.9407352062935\\
1540	-53.5879183765826\\
1541	-59.3858326625318\\
1543	-128.138703003307\\
1544	-138.129091809071\\
1545	-131.580914640203\\
1547	-175.131557704816\\
1548	-181.273917319333\\
1549	-211.407454295146\\
1550	-209.603770544869\\
1551	-156.435058899319\\
1552	-113.263422996408\\
1553	-105.29758852229\\
1554	-175.131788375771\\
1555	-185.288233392823\\
1557	-94.5679993540266\\
1558	-34.9006260549947\\
1559	-19.6474438680461\\
1560	-29.1979732132384\\
1561	-8.91437417523116\\
1562	-44.7344999431946\\
1563	-70.3672211567618\\
1564	-66.1197421262702\\
1565	-63.9962216321414\\
1566	-34.3389755643859\\
1567	-31.9547009733039\\
1568	-43.2736027011124\\
1569	-50.0886306251216\\
1570	-50.8330714936499\\
1571	-39.4787986868823\\
1572	-32.8009981646212\\
1573	-63.5741954737789\\
1574	-144.929702254778\\
1575	-170.127253334415\\
1576	-177.071478963723\\
1577	-142.998454044305\\
1578	-175.838485790895\\
1579	-243.686489601513\\
1580	-225.044348443695\\
1581	-220.468012107115\\
1582	-180.829066273095\\
1583	-174.439995356246\\
1584	-180.367226487178\\
1585	-127.510007581545\\
1586	-122.818826704467\\
1587	-69.2190093802565\\
1588	-66.7278139314881\\
1589	-122.434198110027\\
1590	-85.2152286081607\\
1591	-89.3989949897568\\
1592	-105.970149727999\\
1593	-97.9261166897188\\
1594	-98.3278437209528\\
1595	-102.978595759398\\
1596	-115.460591300068\\
1597	-123.881771312617\\
1598	-114.323188265503\\
1599	-91.0622894525027\\
1600	-105.872755174616\\
1602	-10.3783945038799\\
1603	-29.5825298180521\\
1604	-59.9398979241043\\
1605	-24.2057431332853\\
1606	-6.3769721854178\\
1607	-24.3627627096801\\
1608	-57.7112400074902\\
1609	-66.3605103067343\\
1610	-89.8574392830667\\
1611	-75.9895034657661\\
1612	-124.862879258595\\
1613	-111.937953942146\\
1614	-77.2180263144674\\
1615	-115.836217763559\\
1616	-58.3552043611646\\
1617	-51.3262011702593\\
1618	-30.9164942739394\\
1619	-18.5440901299542\\
1620	-40.1718252957419\\
1621	-28.2585786149339\\
1622	-56.9123776450801\\
1623	-93.1607910436421\\
1624	-87.7743327046633\\
1625	-67.6781599716271\\
1626	-70.4466889713929\\
1627	-56.0074716514896\\
1628	-76.4172923354199\\
1629	-53.6918124176043\\
1630	-51.7487577278607\\
1631	-50.2157915082769\\
1632	-38.7099895933595\\
1633	-51.4296484150659\\
1634	-46.8301419795487\\
1635	-80.4794550372244\\
1636	-74.0181560176634\\
1637	-58.1974604734812\\
1638	-54.2750699249657\\
1639	-45.066384879213\\
1640	-54.0637931811759\\
1641	-108.40340856019\\
1642	-147.157042931782\\
1643	-100.647365244528\\
1644	-100.856880744175\\
1645	-65.373029904315\\
1646	-113.412882792665\\
1647	-109.074370638774\\
1648	-69.406646593827\\
1649	-87.474964900126\\
1650	-65.4194441098455\\
1651	-92.0381112749228\\
1652	-54.8392380323669\\
1653	-53.1100113273783\\
1654	-72.9862736865407\\
1655	-84.0077359712172\\
1656	-64.3099587857184\\
1657	-65.3437400351147\\
1658	-76.326474282811\\
1659	-110.191679220987\\
1660	-101.062575777933\\
1661	-139.636597614291\\
1662	-188.457510071935\\
1663	-150.569656762099\\
1664	-105.100805115948\\
1665	-74.8430935364615\\
1666	-116.790088541223\\
1667	-139.186079754978\\
1668	-113.171942645168\\
1669	-129.239534396363\\
1670	-88.9651271877433\\
1671	-36.4091562075118\\
1672	-43.1897263869587\\
1673	-102.169796446999\\
1674	-94.2865607965953\\
1675	-79.336197834828\\
1676	-125.856342945761\\
1677	-114.453456031805\\
1678	-108.155225874337\\
1679	-70.4560674107545\\
1680	-88.021329528659\\
1681	-121.057624302162\\
1682	-97.8889956654559\\
1683	-101.021898294514\\
1684	-94.2551889564552\\
1685	-70.9834174176092\\
1686	-82.722780273971\\
1687	-61.1125337101114\\
1688	-73.4985775209127\\
1689	-108.733924167913\\
1690	-122.262255495584\\
1691	-121.511424728258\\
1692	-117.936602853533\\
1693	-86.922124374727\\
1694	-64.9319993560662\\
1695	-73.3826375961889\\
1696	-88.8897091097379\\
1699	-143.755052066172\\
1700	-113.952298523697\\
1701	-76.3673658217022\\
1703	-180.751310471157\\
1704	-128.188785754113\\
1705	-123.732550439436\\
1706	-142.617389583831\\
1708	-97.4631162470621\\
1709	-111.398583521693\\
1710	-98.6964956072497\\
1711	-110.599755905017\\
1712	-133.163964744018\\
1713	-104.510671947577\\
1714	-115.532169994074\\
1715	-115.184635417487\\
1716	-111.945535862186\\
1717	-99.7773253298492\\
1718	-118.876265265792\\
1719	-89.5452830989407\\
1720	-94.7493195963868\\
1721	-106.657582494455\\
1722	-99.0162543629142\\
1723	-45.0108232649118\\
1724	-23.9662430902499\\
1725	-7.07073686337662\\
1726	-36.1001644068567\\
1727	-101.34761463222\\
1728	-132.609617902361\\
1729	-124.45527072749\\
1730	-91.6906702276322\\
1731	-102.787990688485\\
1732	-93.7503536058289\\
1733	-86.1553959550274\\
1734	-55.7074275876821\\
1735	-70.9927640925257\\
1736	-70.90376601875\\
1737	-69.305178806339\\
1738	-81.1539710300881\\
1739	-72.5875690413775\\
1740	-69.5535532179917\\
1741	-45.0716232566042\\
1742	-64.7797173423614\\
1743	-117.534679980679\\
1744	-80.106858696513\\
1745	-99.6911472399722\\
1746	-88.7855379595644\\
1747	-117.920285118162\\
1748	-116.096794555158\\
1749	-140.409688961801\\
1750	-116.294188919952\\
1751	-118.885425114622\\
1752	-124.506747168456\\
1753	-66.1767763303153\\
1754	-111.659046121415\\
1755	-84.5283806938646\\
1756	-98.0197610440816\\
1757	-100.152530189679\\
1758	-121.160581403804\\
1759	-97.6909507024161\\
1760	-123.833849991173\\
1761	-144.665747941135\\
1762	-133.453863661572\\
1763	-110.629335404296\\
1764	-100.316618888021\\
1765	-110.727818749449\\
1766	-95.0661136782132\\
1767	-85.9752298441947\\
1768	-79.0776409589241\\
1769	-89.1507013599494\\
1770	-83.0209344128953\\
1771	-137.622212883315\\
1772	-176.511321258626\\
1773	-148.888464028964\\
1774	-148.612885903819\\
1775	-144.815335063003\\
1776	-83.948486211068\\
1777	-76.3183411451769\\
1778	-67.3814005671504\\
1779	-60.313911954938\\
1780	-65.0559457900256\\
1781	-100.568285446563\\
1782	-119.268125075199\\
1783	-131.161274997786\\
1784	-117.713241331246\\
1785	-82.2407605129031\\
1786	-136.905733128809\\
1787	-163.676521271138\\
1788	-168.323253492348\\
1789	-148.812662227728\\
1790	-214.336305438218\\
1791	-165.773603700777\\
1792	-99.4689762458092\\
1793	-66.8926003782681\\
1794	-58.8423182143449\\
1795	-105.721494813303\\
1797	-162.744628126298\\
1798	-130.963192715724\\
1799	-108.221848137398\\
1800	-60.5446271369192\\
1801	-56.1950364779473\\
1802	-73.4000820953311\\
1803	-79.6513401100169\\
1804	-53.9385721640506\\
1805	-64.0595804565251\\
};
\addlegendentry{MPO prediction}

\end{axis}

\begin{axis}[%
width=6.159cm,
height=1.831cm,
at={(8.104cm,2.542cm)},
scale only axis,
xmin=1000,
xmax=2000,
xlabel style={font=\color{white!15!black}},
xlabel={Sample index},
ymin=-279.541,
ymax=0,
ylabel style={font=\color{white!15!black}},
ylabel={$y(t)$},
axis background/.style={fill=white},
title style={font=\bfseries},
title={C8: RMSE(OSA) = 6.0607, RMSE(MPO) = 7.4879},
legend style={legend cell align=left, align=left, draw=white!15!black}
]
\addplot [color=mycolor1, line width=2.0pt]
  table[row sep=crcr]{%
1006	-125.732\\
1007	-153.809\\
1008	-115.967\\
1009	-61.0350000000001\\
1010	-74.463\\
1011	-73.242\\
1012	-86.6700000000001\\
1013	-73.242\\
1014	-34.1800000000001\\
1015	-20.752\\
1016	-23.193\\
1017	-58.5940000000001\\
1018	-100.098\\
1019	-109.863\\
1020	-114.746\\
1022	-52.49\\
1023	-104.98\\
1025	-59.8140000000001\\
1026	-97.6559999999999\\
1027	-83.008\\
1028	-72.021\\
1029	-118.408\\
1030	-107.422\\
1031	-83.008\\
1032	-119.629\\
1033	-123.291\\
1034	-95.2149999999999\\
1035	-80.566\\
1036	-62.2560000000001\\
1037	-75.684\\
1038	-95.2149999999999\\
1039	-87.8910000000001\\
1040	-91.5530000000001\\
1041	-96.4359999999999\\
1042	-102.539\\
1043	-141.602\\
1044	-128.174\\
1045	-85.4490000000001\\
1046	-79.346\\
1047	-47.607\\
1048	-48.828\\
1049	-68.3589999999999\\
1050	-52.49\\
1051	-57.373\\
1052	-75.684\\
1053	-89.1110000000001\\
1054	-81.787\\
1055	-128.174\\
1056	-98.877\\
1057	-57.373\\
1058	-45.1659999999999\\
1059	-62.2560000000001\\
1060	-72.021\\
1061	-40.2829999999999\\
1062	-42.7249999999999\\
1063	-56.152\\
1064	-46.3869999999999\\
1065	-40.2829999999999\\
1066	-52.49\\
1067	-62.2560000000001\\
1068	-92.7729999999999\\
1069	-90.3320000000001\\
1070	-107.422\\
1071	-98.877\\
1072	-115.967\\
1073	-91.5530000000001\\
1074	-91.5530000000001\\
1075	-79.346\\
1076	-75.684\\
1077	-76.904\\
1079	-189.209\\
1080	-194.092\\
1081	-189.209\\
1082	-119.629\\
1083	-170.898\\
1084	-213.623\\
1085	-219.727\\
1086	-169.678\\
1087	-219.727\\
1088	-279.541\\
1089	-197.754\\
1090	-161.133\\
1091	-109.863\\
1092	-85.4490000000001\\
1093	-73.242\\
1094	-91.5530000000001\\
1095	-62.2560000000001\\
1096	-46.3869999999999\\
1097	-42.7249999999999\\
1098	-54.932\\
1099	-79.346\\
1100	-74.463\\
1101	-75.684\\
1102	-100.098\\
1103	-103.76\\
1104	-141.602\\
1105	-114.746\\
1106	-142.822\\
1107	-104.98\\
1108	-90.3320000000001\\
1109	-103.76\\
1110	-76.904\\
1111	-69.5799999999999\\
1112	-84.229\\
1113	-86.6700000000001\\
1114	-59.8140000000001\\
1115	-64.6970000000001\\
1116	-48.828\\
1117	-53.711\\
1118	-57.373\\
1119	-41.5039999999999\\
1120	-52.49\\
1121	-65.9180000000001\\
1122	-101.318\\
1123	-104.98\\
1124	-114.746\\
1125	-68.3589999999999\\
1126	-93.9939999999999\\
1127	-150.146\\
1128	-114.746\\
1129	-125.732\\
1130	-128.174\\
1131	-80.566\\
1132	-53.711\\
1133	-75.684\\
1134	-85.4490000000001\\
1135	-137.939\\
1136	-172.119\\
1137	-170.898\\
1138	-123.291\\
1139	-130.615\\
1140	-115.967\\
1142	-111.084\\
1143	-76.904\\
1144	-68.3589999999999\\
1145	-61.0350000000001\\
1146	-57.373\\
1147	-62.2560000000001\\
1148	-91.5530000000001\\
1149	-140.381\\
1151	-79.346\\
1152	-64.6970000000001\\
1153	-62.2560000000001\\
1154	-40.2829999999999\\
1155	-29.297\\
1156	-36.6210000000001\\
1157	-63.4770000000001\\
1158	-51.27\\
1159	-52.49\\
1160	-58.5940000000001\\
1161	-54.932\\
1162	-37.8420000000001\\
1163	-28.076\\
1164	-23.193\\
1165	-39.0630000000001\\
1166	-83.008\\
1167	-117.188\\
1168	-130.615\\
1169	-86.6700000000001\\
1170	-58.5940000000001\\
1172	-32.9590000000001\\
1173	-67.1389999999999\\
1174	-65.9180000000001\\
1176	-117.188\\
1177	-186.768\\
1178	-216.064\\
1179	-177.002\\
1180	-168.457\\
1181	-109.863\\
1182	-122.07\\
1183	-131.836\\
1184	-146.484\\
1185	-108.643\\
1186	-100.098\\
1187	-98.877\\
1188	-89.1110000000001\\
1189	-106.201\\
1190	-78.125\\
1191	-130.615\\
1192	-159.912\\
1194	-70.8009999999999\\
1195	-73.242\\
1196	-123.291\\
1197	-153.809\\
1198	-189.209\\
1199	-202.637\\
1200	-202.637\\
1201	-153.809\\
1202	-137.939\\
1203	-158.691\\
1204	-187.988\\
1205	-109.863\\
1206	-72.021\\
1207	-101.318\\
1208	-86.6700000000001\\
1209	-53.711\\
1210	-74.463\\
1211	-84.229\\
1213	-51.27\\
1214	-70.8009999999999\\
1215	-58.5940000000001\\
1216	-79.346\\
1217	-107.422\\
1218	-100.098\\
1219	-69.5799999999999\\
1220	-70.8009999999999\\
1221	-107.422\\
1222	-177.002\\
1224	-79.346\\
1225	-61.0350000000001\\
1226	-70.8009999999999\\
1227	-64.6970000000001\\
1228	-48.828\\
1229	-53.711\\
1230	-65.9180000000001\\
1231	-83.008\\
1232	-91.5530000000001\\
1233	-63.4770000000001\\
1234	-104.98\\
1235	-124.512\\
1236	-118.408\\
1237	-91.5530000000001\\
1238	-100.098\\
1239	-135.498\\
1241	-42.7249999999999\\
1242	-57.373\\
1243	-62.2560000000001\\
1244	-81.787\\
1245	-72.021\\
1246	-52.49\\
1247	-79.346\\
1248	-80.566\\
1249	-57.373\\
1250	-70.8009999999999\\
1252	-56.152\\
1253	-67.1389999999999\\
1254	-41.5039999999999\\
1255	-48.828\\
1256	-36.6210000000001\\
1257	-40.2829999999999\\
1258	-75.684\\
1259	-84.229\\
1260	-113.525\\
1261	-146.484\\
1262	-98.877\\
1263	-58.5940000000001\\
1264	-46.3869999999999\\
1265	-43.9449999999999\\
1266	-63.4770000000001\\
1267	-42.7249999999999\\
1268	-47.607\\
1269	-61.0350000000001\\
1270	-102.539\\
1271	-93.9939999999999\\
1272	-134.277\\
1273	-89.1110000000001\\
1274	-81.787\\
1275	-46.3869999999999\\
1276	-45.1659999999999\\
1277	-28.076\\
1278	-35.4000000000001\\
1279	-37.8420000000001\\
1280	-67.1389999999999\\
1281	-70.8009999999999\\
1282	-79.346\\
1283	-85.4490000000001\\
1284	-125.732\\
1285	-112.305\\
1286	-84.229\\
1287	-102.539\\
1288	-109.863\\
1289	-144.043\\
1290	-115.967\\
1291	-75.684\\
1292	-40.2829999999999\\
1293	-29.297\\
1294	-29.297\\
1295	-36.6210000000001\\
1296	-39.0630000000001\\
1297	-37.8420000000001\\
1298	-50.049\\
1299	-74.463\\
1300	-68.3589999999999\\
1301	-70.8009999999999\\
1302	-83.008\\
1303	-51.27\\
1304	-30.518\\
1305	-58.5940000000001\\
1306	-90.3320000000001\\
1307	-87.8910000000001\\
1308	-97.6559999999999\\
1309	-83.008\\
1310	-61.0350000000001\\
1311	-85.4490000000001\\
1312	-118.408\\
1313	-118.408\\
1314	-84.229\\
1315	-136.719\\
1316	-100.098\\
1317	-101.318\\
1318	-117.188\\
1319	-115.967\\
1320	-86.6700000000001\\
1321	-76.904\\
1323	-178.223\\
1324	-139.16\\
1325	-79.346\\
1326	-76.904\\
1327	-79.346\\
1328	-98.877\\
1329	-114.746\\
1330	-85.4490000000001\\
1331	-90.3320000000001\\
1332	-120.85\\
1333	-131.836\\
1334	-87.8910000000001\\
1335	-68.3589999999999\\
1336	-91.5530000000001\\
1337	-156.25\\
1338	-137.939\\
1339	-140.381\\
1340	-84.229\\
1341	-76.904\\
1342	-72.021\\
1343	-47.607\\
1344	-30.518\\
1346	-23.193\\
1347	-54.932\\
1349	-87.8910000000001\\
1350	-65.9180000000001\\
1351	-73.242\\
1352	-83.008\\
1353	-53.711\\
1354	-56.152\\
1355	-53.711\\
1356	-40.2829999999999\\
1357	-51.27\\
1358	-84.229\\
1359	-106.201\\
1360	-63.4770000000001\\
1361	-48.828\\
1362	-40.2829999999999\\
1363	-50.049\\
1364	-35.4000000000001\\
1365	-76.904\\
1366	-130.615\\
1367	-104.98\\
1368	-139.16\\
1369	-162.354\\
1370	-164.795\\
1371	-124.512\\
1372	-90.3320000000001\\
1373	-89.1110000000001\\
1374	-89.1110000000001\\
1375	-106.201\\
1376	-125.732\\
1377	-125.732\\
1378	-168.457\\
1379	-183.105\\
1380	-219.727\\
1381	-147.705\\
1383	-184.326\\
1384	-140.381\\
1385	-162.354\\
1386	-139.16\\
1387	-80.566\\
1388	-52.49\\
1389	-56.152\\
1390	-81.787\\
1391	-65.9180000000001\\
1392	-43.9449999999999\\
1393	-62.2560000000001\\
1394	-47.607\\
1395	-36.6210000000001\\
1396	-46.3869999999999\\
1397	-52.49\\
1398	-36.6210000000001\\
1399	-70.8009999999999\\
1400	-92.7729999999999\\
1401	-68.3589999999999\\
1403	-164.795\\
1404	-150.146\\
1405	-196.533\\
1406	-131.836\\
1407	-144.043\\
1408	-96.4359999999999\\
1409	-63.4770000000001\\
1410	-76.904\\
1411	-59.8140000000001\\
1412	-45.1659999999999\\
1413	-64.6970000000001\\
1414	-36.6210000000001\\
1415	-28.076\\
1416	-34.1800000000001\\
1417	-70.8009999999999\\
1418	-86.6700000000001\\
1419	-107.422\\
1421	-100.098\\
1422	-114.746\\
1423	-93.9939999999999\\
1424	-76.904\\
1425	-76.904\\
1426	-62.2560000000001\\
1427	-97.6559999999999\\
1428	-68.3589999999999\\
1429	-56.152\\
1430	-81.787\\
1431	-113.525\\
1432	-86.6700000000001\\
1433	-65.9180000000001\\
1434	-56.152\\
1435	-65.9180000000001\\
1436	-45.1659999999999\\
1437	-45.1659999999999\\
1439	-75.684\\
1440	-84.229\\
1441	-65.9180000000001\\
1442	-86.6700000000001\\
1443	-79.346\\
1444	-47.607\\
1445	-50.049\\
1446	-95.2149999999999\\
1447	-131.836\\
1448	-131.836\\
1449	-107.422\\
1450	-103.76\\
1451	-79.346\\
1452	-76.904\\
1453	-61.0350000000001\\
1454	-89.1110000000001\\
1455	-75.684\\
1456	-50.049\\
1457	-74.463\\
1458	-85.4490000000001\\
1459	-101.318\\
1460	-109.863\\
1461	-109.863\\
1462	-148.926\\
1463	-181.885\\
1464	-205.078\\
1465	-129.395\\
1466	-73.242\\
1467	-47.607\\
1468	-34.1800000000001\\
1469	-54.932\\
1470	-46.3869999999999\\
1471	-80.566\\
1472	-72.021\\
1473	-64.6970000000001\\
1474	-85.4490000000001\\
1475	-98.877\\
1476	-117.188\\
1477	-123.291\\
1478	-175.781\\
1479	-150.146\\
1480	-117.188\\
1481	-80.566\\
1482	-80.566\\
1483	-83.008\\
1484	-106.201\\
1485	-75.684\\
1486	-79.346\\
1487	-74.463\\
1488	-42.7249999999999\\
1490	-107.422\\
1491	-73.242\\
1492	-80.566\\
1493	-147.705\\
1494	-109.863\\
1495	-106.201\\
1496	-147.705\\
1497	-140.381\\
1498	-87.8910000000001\\
1499	-56.152\\
1500	-58.5940000000001\\
1501	-97.6559999999999\\
1502	-151.367\\
1503	-162.354\\
1504	-167.236\\
1505	-129.395\\
1506	-83.008\\
1507	-72.021\\
1508	-53.711\\
1509	-50.049\\
1510	-42.7249999999999\\
1511	-61.0350000000001\\
1512	-72.021\\
1513	-41.5039999999999\\
1515	-89.1110000000001\\
1516	-101.318\\
1517	-128.174\\
1519	-192.871\\
1520	-136.719\\
1521	-118.408\\
1522	-124.512\\
1523	-102.539\\
1524	-73.242\\
1525	-89.1110000000001\\
1526	-136.719\\
1527	-162.354\\
1528	-93.9939999999999\\
1529	-54.932\\
1530	-37.8420000000001\\
1531	-23.193\\
1532	-32.9590000000001\\
1533	-45.1659999999999\\
1534	-52.49\\
1535	-73.242\\
1537	-45.1659999999999\\
1538	-45.1659999999999\\
1539	-43.9449999999999\\
1540	-40.2829999999999\\
1541	-47.607\\
1542	-70.8009999999999\\
1543	-104.98\\
1544	-111.084\\
1545	-107.422\\
1546	-123.291\\
1547	-152.588\\
1548	-155.029\\
1549	-183.105\\
1550	-167.236\\
1552	-85.4490000000001\\
1553	-83.008\\
1554	-141.602\\
1555	-158.691\\
1557	-76.904\\
1558	-39.0630000000001\\
1559	-30.518\\
1560	-29.297\\
1561	-19.5309999999999\\
1562	-45.1659999999999\\
1563	-57.373\\
1564	-62.2560000000001\\
1565	-53.711\\
1566	-34.1800000000001\\
1567	-26.855\\
1568	-36.6210000000001\\
1569	-41.5039999999999\\
1570	-43.9449999999999\\
1571	-35.4000000000001\\
1572	-28.076\\
1573	-50.049\\
1574	-112.305\\
1576	-145.264\\
1577	-101.318\\
1578	-133.057\\
1579	-195.313\\
1580	-174.561\\
1581	-184.326\\
1582	-135.498\\
1583	-137.939\\
1584	-145.264\\
1585	-95.2149999999999\\
1586	-90.3320000000001\\
1587	-56.152\\
1588	-54.932\\
1589	-102.539\\
1590	-68.3589999999999\\
1591	-79.346\\
1592	-84.229\\
1593	-79.346\\
1594	-79.346\\
1595	-84.229\\
1596	-93.9939999999999\\
1597	-102.539\\
1598	-89.1110000000001\\
1599	-67.1389999999999\\
1600	-84.229\\
1601	-39.0630000000001\\
1602	-20.752\\
1604	-51.27\\
1605	-26.855\\
1606	-13.4280000000001\\
1608	-50.049\\
1609	-58.5940000000001\\
1610	-72.021\\
1611	-62.2560000000001\\
1612	-91.5530000000001\\
1613	-80.566\\
1614	-57.373\\
1615	-86.6700000000001\\
1616	-48.828\\
1617	-41.5039999999999\\
1618	-28.076\\
1619	-21.973\\
1620	-37.8420000000001\\
1621	-28.076\\
1622	-50.049\\
1623	-75.684\\
1624	-73.242\\
1625	-54.932\\
1626	-61.0350000000001\\
1627	-45.1659999999999\\
1628	-65.9180000000001\\
1629	-47.607\\
1630	-46.3869999999999\\
1631	-42.7249999999999\\
1632	-32.9590000000001\\
1633	-42.7249999999999\\
1634	-37.8420000000001\\
1635	-64.6970000000001\\
1636	-63.4770000000001\\
1637	-50.049\\
1638	-46.3869999999999\\
1639	-36.6210000000001\\
1640	-43.9449999999999\\
1641	-93.9939999999999\\
1642	-120.85\\
1643	-81.787\\
1644	-80.566\\
1645	-54.932\\
1646	-93.9939999999999\\
1647	-81.787\\
1648	-54.932\\
1649	-70.8009999999999\\
1650	-52.49\\
1651	-75.684\\
1652	-42.7249999999999\\
1654	-57.373\\
1655	-74.463\\
1656	-58.5940000000001\\
1657	-61.0350000000001\\
1658	-61.0350000000001\\
1659	-95.2149999999999\\
1660	-79.346\\
1661	-125.732\\
1662	-157.471\\
1663	-125.732\\
1664	-81.787\\
1665	-62.2560000000001\\
1666	-89.1110000000001\\
1667	-112.305\\
1668	-87.8910000000001\\
1669	-107.422\\
1670	-64.6970000000001\\
1671	-34.1800000000001\\
1672	-37.8420000000001\\
1673	-84.229\\
1674	-70.8009999999999\\
1675	-69.5799999999999\\
1676	-104.98\\
1677	-97.6559999999999\\
1678	-87.8910000000001\\
1679	-62.2560000000001\\
1680	-73.242\\
1681	-100.098\\
1682	-81.787\\
1683	-87.8910000000001\\
1684	-78.125\\
1685	-58.5940000000001\\
1686	-67.1389999999999\\
1687	-48.828\\
1688	-56.152\\
1689	-95.2149999999999\\
1690	-109.863\\
1691	-115.967\\
1692	-102.539\\
1693	-73.242\\
1694	-51.27\\
1695	-62.2560000000001\\
1696	-70.8009999999999\\
1699	-131.836\\
1700	-107.422\\
1701	-62.2560000000001\\
1702	-109.863\\
1703	-151.367\\
1704	-107.422\\
1705	-108.643\\
1706	-124.512\\
1707	-93.9939999999999\\
1708	-79.346\\
1709	-89.1110000000001\\
1710	-78.125\\
1711	-91.5530000000001\\
1712	-114.746\\
1713	-86.6700000000001\\
1714	-102.539\\
1715	-92.7729999999999\\
1716	-96.4359999999999\\
1717	-76.904\\
1718	-100.098\\
1719	-69.5799999999999\\
1721	-84.229\\
1722	-80.566\\
1723	-41.5039999999999\\
1724	-28.076\\
1725	-21.973\\
1726	-37.8420000000001\\
1727	-84.229\\
1728	-108.643\\
1729	-111.084\\
1730	-74.463\\
1731	-86.6700000000001\\
1732	-72.021\\
1733	-65.9180000000001\\
1734	-42.7249999999999\\
1735	-61.0350000000001\\
1736	-58.5940000000001\\
1737	-57.373\\
1738	-67.1389999999999\\
1739	-57.373\\
1740	-53.711\\
1741	-37.8420000000001\\
1742	-50.049\\
1743	-91.5530000000001\\
1744	-70.8009999999999\\
1745	-86.6700000000001\\
1746	-75.684\\
1747	-101.318\\
1748	-92.7729999999999\\
1749	-129.395\\
1750	-91.5530000000001\\
1751	-106.201\\
1752	-103.76\\
1753	-54.932\\
1754	-96.4359999999999\\
1755	-67.1389999999999\\
1756	-84.229\\
1757	-90.3320000000001\\
1758	-115.967\\
1759	-85.4490000000001\\
1761	-136.719\\
1762	-111.084\\
1763	-95.2149999999999\\
1764	-75.684\\
1765	-91.5530000000001\\
1767	-69.5799999999999\\
1768	-56.152\\
1769	-70.8009999999999\\
1770	-58.5940000000001\\
1771	-119.629\\
1772	-151.367\\
1773	-130.615\\
1774	-126.953\\
1775	-124.512\\
1776	-69.5799999999999\\
1777	-63.4770000000001\\
1778	-50.049\\
1779	-42.7249999999999\\
1780	-48.828\\
1781	-80.566\\
1782	-100.098\\
1783	-114.746\\
1784	-96.4359999999999\\
1785	-67.1389999999999\\
1786	-117.188\\
1787	-137.939\\
1788	-152.588\\
1789	-118.408\\
1790	-187.988\\
1792	-78.125\\
1793	-57.373\\
1794	-48.828\\
1795	-87.8910000000001\\
1796	-111.084\\
1797	-147.705\\
1798	-109.863\\
1799	-87.8910000000001\\
1800	-47.607\\
1801	-53.711\\
1802	-56.152\\
1803	-63.4770000000001\\
1804	-40.2829999999999\\
1805	-52.49\\
};
\addlegendentry{True output}

\addplot [color=mycolor2, dashed, line width=2.0pt]
  table[row sep=crcr]{%
1006	-115.168180771827\\
1007	-142.584446657916\\
1008	-120.611546293092\\
1009	-63.8143766301507\\
1010	-75.4486911407591\\
1011	-74.8694342537567\\
1012	-91.0410285618675\\
1013	-71.6653354872767\\
1014	-27.3605442009346\\
1015	-17.8569413719265\\
1016	-17.1070988521847\\
1017	-66.8991971135169\\
1018	-104.40611887042\\
1019	-104.409756520188\\
1020	-107.400278209422\\
1021	-87.5660173828328\\
1022	-52.2755822961888\\
1023	-108.485880208089\\
1024	-76.1431449397344\\
1025	-63.157024417186\\
1026	-98.3335530853026\\
1028	-72.7167854020056\\
1029	-110.073451784364\\
1030	-103.73720333007\\
1031	-85.0934784360618\\
1032	-110.989289201236\\
1033	-114.001952562031\\
1034	-100.418733536963\\
1035	-83.0917532790033\\
1036	-68.1883476732701\\
1038	-86.5671440278552\\
1039	-89.0672254000258\\
1040	-86.8316603626934\\
1041	-92.0750628801459\\
1042	-95.5928796979549\\
1043	-129.203836838746\\
1044	-125.150865194128\\
1045	-94.6177849862588\\
1046	-82.1592866092653\\
1047	-48.1299436869294\\
1048	-50.5261640306055\\
1049	-69.0627772505075\\
1050	-53.5673356128755\\
1051	-55.4179262183861\\
1052	-77.0685986794288\\
1053	-83.5799219217811\\
1054	-81.3805944546566\\
1055	-119.289990482762\\
1056	-96.3994009822891\\
1057	-63.479720992445\\
1058	-49.906805765366\\
1060	-74.4897826494969\\
1061	-37.9358544902911\\
1062	-43.2064596227087\\
1063	-59.4800657038281\\
1064	-47.5372743010951\\
1065	-41.0037077996858\\
1066	-55.0736982254191\\
1067	-56.9363705412823\\
1068	-96.2791651851098\\
1069	-82.2146517350966\\
1070	-104.220918621385\\
1071	-97.7338997926233\\
1072	-104.908440267729\\
1073	-96.594260020408\\
1075	-83.5717724308488\\
1076	-81.5316949242585\\
1077	-75.6141348649116\\
1078	-115.853444610257\\
1079	-168.802760649183\\
1080	-180.369948559761\\
1081	-178.473842822427\\
1082	-130.35574561343\\
1083	-163.75432122183\\
1084	-192.973082551627\\
1085	-203.62029222734\\
1086	-176.383892664682\\
1087	-198.546476899494\\
1088	-261.352492464363\\
1089	-214.640602717118\\
1090	-176.27390476006\\
1091	-113.034185743401\\
1092	-91.4503219264545\\
1093	-75.7661623027957\\
1094	-90.3092781126284\\
1096	-39.4157057542907\\
1097	-39.4287627731255\\
1098	-57.6883504845855\\
1099	-80.2673959709805\\
1100	-73.5719697376051\\
1101	-76.6767621087861\\
1102	-95.6005876903444\\
1103	-103.395097417128\\
1104	-133.249635434272\\
1105	-124.384984455313\\
1106	-129.315198416773\\
1108	-94.1230605798914\\
1109	-105.093767241302\\
1110	-83.0821472136611\\
1111	-73.4630943771194\\
1112	-86.9994800496088\\
1113	-85.736238370446\\
1114	-67.5731479291376\\
1115	-67.220429132559\\
1116	-50.2661527488474\\
1117	-53.4429790040692\\
1118	-61.6426905957069\\
1119	-45.2780394740532\\
1120	-53.4332988977076\\
1121	-66.8730043609121\\
1122	-100.608586254654\\
1123	-94.7999299917378\\
1124	-113.443207856619\\
1125	-69.360921864054\\
1126	-97.0741248009601\\
1127	-138.797500903872\\
1128	-118.350704469595\\
1129	-124.11268967917\\
1130	-121.96304809171\\
1132	-55.4190424355361\\
1133	-78.7146644532918\\
1134	-85.1236201774757\\
1135	-129.493832281641\\
1136	-157.173005565468\\
1137	-159.83083649986\\
1138	-126.97942676309\\
1139	-135.527740577161\\
1140	-119.419273972405\\
1141	-115.642413936071\\
1142	-114.234909288149\\
1143	-78.3292530105975\\
1144	-73.1798632328962\\
1145	-63.3918738903299\\
1146	-58.7309998739829\\
1147	-67.1605936132221\\
1148	-90.505529692608\\
1149	-125.788514361261\\
1150	-118.140927985982\\
1151	-78.9245327793315\\
1152	-72.4610305235285\\
1153	-66.9388599028275\\
1154	-38.777538349719\\
1155	-27.0906768189293\\
1156	-37.0445347366074\\
1157	-62.7891545739594\\
1158	-50.2355532089719\\
1159	-56.9943236923791\\
1160	-58.7747488175769\\
1161	-58.7848970624696\\
1162	-40.2959126275668\\
1164	-19.8413993781921\\
1165	-38.3795945524178\\
1166	-85.927834178136\\
1167	-116.715379930576\\
1168	-122.604165635155\\
1169	-94.7094902365725\\
1170	-63.328425772334\\
1171	-47.0031618959024\\
1172	-33.187105338284\\
1173	-68.4739901034297\\
1174	-64.1847099077547\\
1175	-86.4473103659732\\
1176	-111.517526889704\\
1177	-176.247077280754\\
1178	-208.103932826402\\
1179	-183.062621864288\\
1180	-166.002216583385\\
1181	-119.440809656983\\
1182	-124.547892841315\\
1183	-133.077579856399\\
1184	-137.386306143064\\
1185	-118.640320393961\\
1186	-97.2009151282195\\
1187	-106.893937250298\\
1188	-95.4224309188426\\
1189	-101.535230316042\\
1190	-83.0479996255142\\
1191	-121.477359750848\\
1192	-152.386985515019\\
1194	-79.8535209024144\\
1195	-72.9545634227768\\
1196	-118.674440326911\\
1197	-144.292546518471\\
1199	-185.721888490663\\
1200	-196.407268649021\\
1201	-159.291397657118\\
1202	-147.985315606196\\
1203	-154.807116668283\\
1204	-175.862753481701\\
1205	-117.80438533671\\
1206	-75.2154846858743\\
1207	-106.975275691696\\
1208	-90.3588428223461\\
1209	-56.7393430184025\\
1210	-74.5333425278504\\
1211	-85.0731531452604\\
1212	-70.0545881285052\\
1213	-60.8600832877371\\
1214	-70.1517101305672\\
1215	-63.3642009272182\\
1216	-73.7321839752076\\
1217	-111.138362379853\\
1218	-97.4396968430594\\
1219	-77.2651972388931\\
1220	-73.409265814857\\
1221	-104.09583284125\\
1222	-162.562549300479\\
1223	-138.512324273562\\
1224	-82.9704352912458\\
1225	-67.6005981857804\\
1226	-74.0745022478889\\
1227	-71.8366072098324\\
1228	-47.7015783438649\\
1229	-58.1548975735641\\
1230	-65.7585835853117\\
1231	-80.92989805246\\
1232	-93.0061950778681\\
1233	-66.8111146913106\\
1234	-100.590135723977\\
1235	-122.972869468598\\
1236	-112.098535359748\\
1237	-96.4590695565726\\
1238	-100.610653710991\\
1239	-126.246934569731\\
1240	-102.812445142566\\
1241	-36.0240435202702\\
1242	-54.5660524825596\\
1243	-68.0356948845931\\
1244	-84.6529326388227\\
1245	-74.8643913927763\\
1246	-59.2011595922127\\
1247	-83.1433299307644\\
1248	-75.7333966180456\\
1249	-59.8731663381459\\
1250	-76.1310182752172\\
1251	-65.0690070215903\\
1252	-59.2560837203976\\
1253	-71.1604174345905\\
1254	-42.5467495081382\\
1255	-51.9817112485114\\
1256	-36.87525816979\\
1257	-42.4579386451071\\
1258	-77.8485016077391\\
1259	-79.5195832338957\\
1260	-105.078019482067\\
1261	-140.992370761641\\
1263	-65.0944543121568\\
1264	-46.7350413171091\\
1265	-48.4702308983751\\
1266	-70.5096800611377\\
1267	-41.4895551433178\\
1268	-48.072495887774\\
1269	-61.5194596449026\\
1270	-100.26194206785\\
1271	-92.6147681415453\\
1272	-125.437573517272\\
1273	-96.4688132724548\\
1274	-80.0963235072129\\
1275	-44.7063052615956\\
1276	-47.2834986296182\\
1277	-28.1994501528955\\
1278	-34.8203414992283\\
1279	-40.297115710978\\
1280	-66.1279502840976\\
1281	-71.4220258056735\\
1282	-76.1059208252943\\
1283	-82.5361519355097\\
1284	-120.15353081871\\
1285	-121.533841819505\\
1286	-86.7628766181228\\
1287	-98.8415164837711\\
1288	-109.527601112322\\
1289	-130.578046977216\\
1290	-114.738712466346\\
1291	-83.1408287436263\\
1292	-34.9799198474172\\
1293	-27.8346795716943\\
1294	-32.0360329302455\\
1295	-38.1933286930528\\
1296	-39.0760513908315\\
1297	-38.1342281524471\\
1298	-47.0925498325739\\
1299	-79.5418725891309\\
1300	-70.4430049865291\\
1301	-67.5744117997931\\
1302	-80.7303622170944\\
1304	-30.9072089604695\\
1306	-88.4092284841663\\
1307	-83.0026014582636\\
1308	-96.2696143081296\\
1309	-82.0054183384411\\
1310	-65.5467564236876\\
1311	-82.4073156790369\\
1312	-109.40954907741\\
1313	-113.108130572197\\
1314	-88.0038562362329\\
1315	-126.903263179808\\
1316	-108.137631796599\\
1317	-95.9656504044056\\
1318	-117.269753559046\\
1319	-111.694559020459\\
1320	-94.1748546357073\\
1321	-82.6745350763902\\
1322	-117.546802971225\\
1323	-164.974524931576\\
1324	-136.213462254218\\
1325	-87.5565859362364\\
1326	-75.9529207516425\\
1327	-88.0008324270623\\
1328	-93.9069218331881\\
1329	-106.061641406436\\
1330	-89.1673301280978\\
1331	-94.4789960438752\\
1332	-113.238351881704\\
1333	-119.892005135046\\
1334	-93.2928242569485\\
1335	-79.7735583343706\\
1336	-91.5313925542953\\
1337	-147.026851868399\\
1338	-132.268381029231\\
1339	-130.540782171862\\
1340	-91.1637718854593\\
1341	-77.9942257437071\\
1342	-77.4103825207942\\
1343	-48.9949484249189\\
1344	-27.523977370837\\
1345	-22.255560594939\\
1346	-20.768748964249\\
1347	-53.4358187636692\\
1348	-75.1954584726388\\
1349	-86.172785961774\\
1350	-69.3071742022942\\
1351	-70.6834572186374\\
1352	-82.2449824825082\\
1353	-54.1735661679625\\
1354	-57.6756690815982\\
1355	-56.3522047755127\\
1356	-42.3511105795744\\
1357	-53.9726267118338\\
1358	-79.5566847546254\\
1359	-100.763109735315\\
1360	-65.1569783538823\\
1361	-52.5075875890927\\
1362	-46.4167774102627\\
1363	-55.160342208374\\
1364	-33.3824028274785\\
1366	-125.113510311896\\
1367	-105.053358945649\\
1368	-126.397861242185\\
1369	-151.812811972421\\
1370	-159.419568890393\\
1371	-125.911517648542\\
1372	-97.8974827847505\\
1373	-96.6315624064989\\
1374	-91.1656503481138\\
1376	-116.977468105433\\
1377	-117.026344843024\\
1378	-152.093851030648\\
1379	-173.637025559469\\
1380	-198.312403258321\\
1381	-160.431325923915\\
1382	-165.668042195733\\
1383	-171.833870897764\\
1384	-153.457809714303\\
1385	-155.487663674642\\
1386	-142.41641866984\\
1387	-88.1506398955416\\
1388	-48.5347529944588\\
1389	-55.9692862428992\\
1390	-81.5859671107046\\
1391	-67.2073076749364\\
1392	-47.6226625384952\\
1393	-66.1105008423692\\
1394	-48.5954440595017\\
1395	-38.6540212640466\\
1396	-48.3420820538342\\
1397	-49.4518658690756\\
1398	-42.9590644255386\\
1400	-92.4437509681047\\
1401	-70.9853781250526\\
1402	-113.247629043849\\
1403	-163.380071070643\\
1404	-149.526956213526\\
1405	-178.908645123228\\
1406	-139.421666776716\\
1407	-145.372412999727\\
1408	-101.409578007704\\
1409	-64.7796067260504\\
1410	-77.3344125874744\\
1411	-66.7236265210881\\
1412	-45.7876784695011\\
1413	-62.8854325290943\\
1414	-43.4916547168959\\
1415	-18.7000746897018\\
1416	-33.7345248583747\\
1417	-70.864288950901\\
1418	-90.3503000308335\\
1419	-100.98214552166\\
1420	-105.36592088072\\
1421	-96.6056364884673\\
1422	-109.515499311778\\
1423	-97.418029025539\\
1424	-82.2383721979854\\
1425	-81.5099372156849\\
1426	-67.5419206453337\\
1427	-95.2300394147042\\
1428	-66.9119791366518\\
1429	-59.6569893216631\\
1430	-84.4387262794546\\
1431	-105.155931896595\\
1433	-70.0689781436517\\
1434	-62.5906211275765\\
1435	-69.4777029013287\\
1436	-46.8799628101679\\
1437	-45.0544158681357\\
1438	-62.4060209371189\\
1439	-70.5766459866954\\
1440	-81.3102630577298\\
1441	-71.4217573960627\\
1442	-82.3686745718198\\
1443	-78.9055671236304\\
1444	-52.0617121223356\\
1445	-53.7668555209912\\
1447	-129.868925880792\\
1448	-120.655015226016\\
1449	-107.776019626\\
1450	-103.291107306524\\
1451	-89.4506561132162\\
1452	-78.0242439090605\\
1453	-68.8740925565576\\
1454	-81.2812570559997\\
1455	-82.5620210569584\\
1456	-50.2647511575949\\
1457	-76.1149221165149\\
1458	-80.1571129333513\\
1460	-107.600790277165\\
1461	-101.151170565749\\
1462	-144.426632483355\\
1463	-169.440398566456\\
1464	-190.458916356296\\
1465	-147.228872045074\\
1466	-76.7580773265142\\
1467	-43.6500943017543\\
1468	-35.087455156837\\
1469	-53.4502807807621\\
1470	-47.0587796917414\\
1471	-79.9649398338217\\
1472	-72.2535536757589\\
1473	-66.3670893041406\\
1474	-83.8023549000193\\
1475	-93.6714180307392\\
1476	-111.390146953928\\
1477	-118.344250595616\\
1478	-163.834963557768\\
1479	-158.346773301583\\
1480	-121.078160633321\\
1481	-91.3448899050929\\
1482	-81.2791758600204\\
1484	-100.204315658497\\
1485	-79.3424111577335\\
1486	-78.1113316384062\\
1487	-83.6700074699977\\
1488	-37.8524398034101\\
1489	-73.4875026296693\\
1490	-102.980623752006\\
1491	-77.5476493672109\\
1492	-75.5812601557648\\
1493	-146.067830115025\\
1494	-107.313862868227\\
1495	-107.804888944024\\
1496	-137.209012476735\\
1497	-136.442865972202\\
1499	-60.0484034785736\\
1500	-60.1754506764494\\
1502	-135.999443686608\\
1503	-150.986186215754\\
1504	-153.600815322603\\
1505	-140.563796470621\\
1506	-85.8323085014558\\
1507	-79.1885287346809\\
1508	-56.6929816913841\\
1509	-54.6500586518582\\
1510	-44.0198575844222\\
1511	-63.3366148499615\\
1512	-69.0876938208726\\
1513	-43.4910120129016\\
1514	-70.4358992916757\\
1515	-87.7014466137682\\
1516	-93.8177710263012\\
1517	-121.025419467664\\
1518	-155.247051298006\\
1519	-179.930719138045\\
1520	-145.029762038945\\
1521	-128.079839307811\\
1522	-126.075786740821\\
1523	-109.713846494595\\
1524	-75.3799964930397\\
1525	-85.2469735486536\\
1526	-132.678914240665\\
1527	-148.094576837354\\
1529	-55.4290662037822\\
1530	-35.4079889983946\\
1531	-11.9952827177806\\
1532	-24.8468061532201\\
1533	-47.5121622880704\\
1534	-49.4948719171023\\
1535	-79.3783002867146\\
1537	-47.175826092071\\
1538	-48.119672625133\\
1539	-45.8318331099811\\
1540	-42.8265813961175\\
1541	-49.1146301850977\\
1542	-70.2237278569703\\
1543	-103.704903487805\\
1544	-106.681494281585\\
1545	-105.629377461336\\
1546	-114.166680221678\\
1547	-151.428790758962\\
1548	-144.524812324485\\
1549	-171.279138163777\\
1550	-173.630064911743\\
1552	-93.279704275511\\
1553	-85.9378104311747\\
1554	-133.541424636663\\
1555	-150.355252370807\\
1556	-116.416782306548\\
1557	-87.0936487910567\\
1558	-20.6619707131931\\
1559	-27.867491027029\\
1560	-32.2960363461\\
1561	-11.5881447863699\\
1562	-44.2535087979854\\
1563	-61.683655887254\\
1564	-58.6958937321106\\
1565	-57.2794508953748\\
1566	-33.0123382329693\\
1567	-30.7456490421796\\
1568	-35.3637789740214\\
1569	-43.8598633272875\\
1570	-42.0173770834263\\
1572	-31.8987072672539\\
1573	-50.8665393948572\\
1574	-116.516801872041\\
1575	-135.067263980491\\
1576	-133.739228268035\\
1577	-110.728107467833\\
1578	-124.56357388411\\
1579	-193.82691155425\\
1580	-165.11193412139\\
1581	-179.423086496774\\
1582	-140.102068633263\\
1583	-142.152445433982\\
1584	-145.709669089539\\
1585	-103.946219752198\\
1586	-92.1702253584726\\
1587	-53.4498749225115\\
1588	-50.529509953242\\
1589	-105.348352246118\\
1590	-68.5316379689932\\
1591	-73.9727658342579\\
1592	-85.3694001804367\\
1593	-79.4541148906515\\
1594	-85.3034205162423\\
1595	-80.8191617765194\\
1596	-94.4459966507468\\
1597	-99.517913467243\\
1598	-89.9658646613213\\
1599	-73.2787941169122\\
1600	-88.6411795111192\\
1601	-32.3073756979584\\
1602	-15.9032262620581\\
1603	-28.1049511636536\\
1604	-55.9474362564115\\
1605	-23.3304784137995\\
1606	-12.2653633720261\\
1607	-27.3236547898844\\
1608	-49.2002866448938\\
1609	-59.9520980325881\\
1610	-73.1807279281907\\
1611	-63.5984358509052\\
1612	-89.8753876564806\\
1613	-88.0703389341004\\
1614	-58.5499906007869\\
1615	-85.0691175172735\\
1616	-44.9478926710881\\
1617	-41.4204010122189\\
1619	-16.509271622112\\
1620	-37.2152779494729\\
1621	-27.6327417104058\\
1622	-48.7344038573494\\
1623	-77.0073462719999\\
1624	-71.9176427168375\\
1625	-56.9008907674286\\
1626	-60.5016468029798\\
1627	-50.5631976287161\\
1628	-60.3999324158531\\
1629	-49.3615257018839\\
1630	-44.9866722901572\\
1631	-44.8126372334932\\
1632	-32.6598512601513\\
1633	-43.5191099900296\\
1634	-37.116992004321\\
1635	-63.8815321515174\\
1636	-58.9777889024767\\
1637	-49.2155331183108\\
1638	-50.807807437074\\
1639	-36.9288555910714\\
1640	-43.8034451309265\\
1641	-87.9358984709561\\
1642	-119.865292451861\\
1643	-83.1844778236607\\
1644	-79.4264802258633\\
1645	-55.4465317610279\\
1646	-90.0912240407431\\
1647	-85.1414253969083\\
1648	-54.6291066288352\\
1649	-74.1802067545318\\
1650	-53.6461562459299\\
1651	-71.013432977132\\
1652	-46.0271682316918\\
1653	-42.8852257207732\\
1654	-60.8657173508516\\
1655	-69.8805695122091\\
1656	-57.6326371770583\\
1657	-60.6847383263521\\
1658	-59.0441301803655\\
1659	-89.62230095271\\
1660	-83.0118736285801\\
1661	-110.990553902597\\
1662	-155.301735279085\\
1663	-128.118509954627\\
1664	-84.9322232275256\\
1665	-64.3032908553678\\
1666	-90.1832270801769\\
1667	-111.863974112723\\
1668	-85.4047995370354\\
1669	-101.117492611213\\
1671	-33.3431262109623\\
1672	-34.4997252989001\\
1673	-83.4226318903686\\
1675	-65.3867615395154\\
1676	-98.0032371275406\\
1677	-96.6616057590793\\
1678	-84.9792786161504\\
1679	-66.2444390951161\\
1680	-76.2921236623029\\
1681	-93.8472736804181\\
1682	-83.9999853957813\\
1683	-82.1039113122788\\
1684	-76.8492488082686\\
1685	-63.1231926231371\\
1686	-70.4381494173272\\
1687	-49.261903984657\\
1688	-60.0190668273628\\
1689	-91.3281285108062\\
1690	-97.8998837022491\\
1691	-107.234869981819\\
1692	-100.559378859788\\
1694	-57.2663461447494\\
1695	-62.1035978124128\\
1696	-70.8262095176722\\
1697	-85.9329789960098\\
1699	-122.460008977504\\
1700	-107.15432542564\\
1701	-66.5446755389369\\
1703	-144.944097035099\\
1704	-106.311713526629\\
1705	-108.916814936375\\
1706	-116.621211494112\\
1707	-101.100709130266\\
1708	-82.1791812691151\\
1709	-90.2196603928246\\
1710	-80.9246085814361\\
1711	-87.5245320855358\\
1712	-110.945601182235\\
1713	-83.8338615047494\\
1714	-98.6384687785987\\
1715	-93.7260893645932\\
1716	-95.9624023401007\\
1717	-80.1344704185378\\
1718	-96.007773184986\\
1719	-73.5831907258967\\
1720	-76.1802354037161\\
1721	-82.0819550058584\\
1722	-82.033594874129\\
1723	-42.8151809120116\\
1724	-25.0693926922486\\
1725	-16.1628343829559\\
1726	-33.9374666094545\\
1727	-93.4696951945273\\
1728	-109.31657527724\\
1729	-102.162386982884\\
1730	-80.1660901818132\\
1731	-84.8454857775212\\
1732	-78.595103813652\\
1733	-67.2942872264548\\
1734	-42.532895594182\\
1735	-58.2722797139247\\
1736	-60.2673001962771\\
1737	-59.9347965795905\\
1738	-64.4815907881343\\
1739	-58.2627812373303\\
1740	-56.0999659164822\\
1741	-38.9211031347165\\
1742	-50.6166325098623\\
1743	-92.3442352242855\\
1744	-68.3971082677849\\
1745	-82.7845340409913\\
1746	-73.5757134144624\\
1747	-97.4094304066291\\
1748	-90.4519357412996\\
1749	-116.372600241674\\
1750	-99.768535175928\\
1751	-97.7810557921946\\
1752	-105.131458741233\\
1753	-59.3566273441618\\
1754	-93.9049263334628\\
1755	-66.6538832658498\\
1756	-85.7311454137732\\
1757	-85.7113547369145\\
1758	-103.842938172871\\
1759	-88.9936584990271\\
1760	-102.886970950752\\
1761	-128.538529355068\\
1762	-111.471318729006\\
1763	-99.1480668805016\\
1764	-81.7003760459543\\
1765	-87.0899789090922\\
1766	-86.8417112086854\\
1767	-68.5565887633668\\
1768	-63.5128824422554\\
1769	-68.6053711737534\\
1770	-62.5018939167769\\
1771	-110.836083936304\\
1772	-138.506685955202\\
1773	-123.669228480981\\
1774	-119.675798261542\\
1775	-121.797713942757\\
1776	-77.1820614353717\\
1777	-65.9181989809426\\
1778	-52.2977612997447\\
1779	-48.7804688078513\\
1780	-50.8833826137877\\
1781	-79.8497798776327\\
1782	-90.3159384876581\\
1783	-104.801708229842\\
1784	-97.2930521527751\\
1785	-72.002807021312\\
1786	-114.641887831185\\
1787	-127.573872075798\\
1788	-137.826309691896\\
1789	-125.67663582191\\
1790	-163.59397734801\\
1791	-153.159947994354\\
1792	-82.7040424750103\\
1793	-55.1103510204905\\
1794	-55.1659738676033\\
1795	-84.319467598104\\
1796	-104.923274612831\\
1797	-130.600109419687\\
1798	-115.037779546829\\
1799	-91.630508581319\\
1800	-50.106269600471\\
1801	-53.6752737352504\\
1802	-59.9079950883588\\
1803	-61.7803952514175\\
1804	-45.9057838879417\\
1805	-53.2042553551562\\
};
\addlegendentry{OSA predition}

\addplot [color=mycolor3, dotted, line width=2.0pt]
  table[row sep=crcr]{%
1006	-125.732\\
1007	-153.809\\
1008	-115.967\\
1009	-61.0350000000001\\
1010	-75.4486911407587\\
1011	-75.273590954039\\
1012	-91.9974010179676\\
1013	-74.1635841232301\\
1014	-28.4763016825591\\
1015	-15.58547829847\\
1016	-14.7683590136419\\
1017	-62.4713099701394\\
1018	-103.667431219579\\
1019	-104.344399154594\\
1020	-105.224179373229\\
1021	-83.597439350508\\
1022	-51.6357112808241\\
1023	-106.881138583492\\
1024	-75.9129118022586\\
1025	-60.9912777972481\\
1026	-98.1603753765639\\
1028	-73.316897834477\\
1029	-111.085348151297\\
1030	-101.166651630305\\
1031	-81.9688028132518\\
1032	-109.276466465623\\
1033	-108.386676692763\\
1034	-92.3257764728487\\
1036	-65.9719623304993\\
1037	-77.5295980122701\\
1038	-87.4211853601189\\
1039	-86.2857129238785\\
1040	-85.7233434826896\\
1041	-89.0945839753892\\
1042	-90.9694301116033\\
1043	-122.738827699012\\
1044	-114.658020110243\\
1045	-85.2950322590611\\
1046	-78.5364698262665\\
1047	-46.0274065174576\\
1048	-48.4438364579642\\
1049	-68.6511964146134\\
1050	-53.4822584525032\\
1051	-55.6812158471344\\
1052	-76.4796328190937\\
1053	-83.8260558468794\\
1054	-79.2568491370664\\
1055	-117.543407495924\\
1056	-91.3727582765403\\
1057	-58.4971217448817\\
1058	-49.1359699652821\\
1059	-62.7105269184447\\
1060	-74.4979937413391\\
1061	-39.2411934668271\\
1062	-43.4778923861945\\
1063	-59.8370392283803\\
1064	-49.3250058141862\\
1065	-42.5634353333783\\
1066	-56.6473407821527\\
1067	-59.46948930385\\
1068	-96.1864308030404\\
1069	-83.9432298699728\\
1070	-102.091389469646\\
1071	-94.7838784783\\
1072	-102.352004364914\\
1073	-89.5214032301737\\
1074	-86.4523096098119\\
1075	-80.1339236723306\\
1076	-79.7657140608048\\
1077	-76.9018851093701\\
1078	-115.940048961137\\
1079	-162.515993446343\\
1080	-167.72377067455\\
1081	-162.000987193822\\
1082	-112.645570834156\\
1083	-152.454243556152\\
1084	-178.889649612391\\
1085	-183.345402653564\\
1086	-154.402392326466\\
1087	-181.955873265499\\
1088	-235.158961468324\\
1089	-186.046771155007\\
1090	-161.100646994791\\
1091	-107.428069228702\\
1092	-86.2052416088698\\
1093	-74.5832329412076\\
1094	-90.8619315207216\\
1096	-40.180702493479\\
1097	-37.2615920925862\\
1098	-54.6669263549356\\
1099	-79.1172037872041\\
1100	-72.3330959033433\\
1101	-75.2156760863722\\
1102	-95.0995242525805\\
1103	-101.059253038564\\
1104	-131.256162594272\\
1105	-119.238018431611\\
1106	-129.071746045583\\
1107	-105.80944924117\\
1108	-91.8358506459015\\
1109	-105.445845307298\\
1110	-82.6494137999211\\
1111	-76.1788422405316\\
1112	-90.6944580840193\\
1113	-89.7837232360055\\
1114	-70.5594469644982\\
1115	-72.9803145757519\\
1116	-55.7350174799467\\
1117	-57.9929006055579\\
1118	-65.7510532867184\\
1119	-50.3253356896964\\
1120	-58.9703978829784\\
1121	-71.6186280466759\\
1122	-105.619763242556\\
1123	-98.8417015717621\\
1124	-112.741293457566\\
1125	-68.5095784458738\\
1126	-96.8908335842948\\
1127	-139.169313577469\\
1128	-114.027362990482\\
1129	-122.758726990353\\
1130	-120.081241401326\\
1131	-82.3179613138382\\
1132	-55.0203484337183\\
1133	-78.916541263691\\
1134	-85.9294008674094\\
1135	-130.43743898075\\
1136	-154.790323173866\\
1137	-152.175731592669\\
1138	-116.533561136175\\
1139	-128.343335983389\\
1140	-114.88836759428\\
1141	-112.638680902423\\
1142	-112.913671509028\\
1143	-78.7066465573696\\
1144	-74.0286449861328\\
1145	-65.9045285605007\\
1146	-61.6846177987995\\
1147	-70.0173656193501\\
1148	-95.2319754543214\\
1149	-129.318433703848\\
1150	-115.393025294482\\
1151	-80.2718770976901\\
1152	-73.4779844077211\\
1153	-69.6710376769622\\
1154	-43.3574284859137\\
1155	-29.25526533375\\
1156	-38.3604000083767\\
1157	-64.451467200651\\
1158	-51.133939057775\\
1159	-57.177317848071\\
1160	-60.9160077815375\\
1161	-60.3749617359501\\
1162	-42.9561500358179\\
1163	-33.8011391288715\\
1164	-22.9931086488775\\
1165	-39.6881509281816\\
1166	-87.3887193835046\\
1167	-119.108179093813\\
1168	-123.963337618849\\
1169	-92.6738693283869\\
1170	-65.4338818353328\\
1172	-35.603688803923\\
1173	-71.2591919664935\\
1174	-67.2878591900389\\
1176	-111.146551489857\\
1177	-174.046074402157\\
1178	-201.654461122775\\
1179	-174.354909978007\\
1180	-161.215640138268\\
1181	-114.476839848732\\
1183	-133.738710569879\\
1184	-138.103169826281\\
1185	-116.037667731992\\
1186	-99.3836820471392\\
1187	-107.335286373486\\
1188	-98.3558836279842\\
1189	-107.103095572619\\
1190	-84.9619881124661\\
1191	-126.13466376395\\
1192	-152.46043472629\\
1193	-112.34747347989\\
1194	-78.9316696321052\\
1195	-75.5896556527073\\
1196	-119.88679175409\\
1197	-144.117037648972\\
1198	-160.73329400265\\
1199	-172.546292691823\\
1200	-178.747094212628\\
1201	-142.377924754599\\
1202	-135.507060037219\\
1203	-147.638368550844\\
1204	-167.511000195128\\
1205	-106.672709097517\\
1206	-70.9027643820662\\
1207	-104.229418329334\\
1208	-89.2527491667527\\
1209	-57.8308429468489\\
1210	-76.5758164265012\\
1211	-86.9202253046096\\
1212	-71.9279775984617\\
1213	-63.662582572714\\
1214	-76.1975743222642\\
1215	-67.6365683356973\\
1216	-79.4835199938166\\
1217	-114.144121028888\\
1218	-101.651966004836\\
1219	-79.632726507401\\
1220	-78.0356820223833\\
1221	-109.140288625375\\
1222	-165.89486624236\\
1223	-136.095499599988\\
1224	-85.6007546857622\\
1225	-71.3440116716381\\
1226	-78.1937756881969\\
1227	-77.2469025056573\\
1228	-54.6891231411521\\
1230	-71.9471857841493\\
1231	-86.3425795122469\\
1232	-96.6819754863122\\
1233	-70.6555245669697\\
1234	-105.075871045894\\
1235	-124.808976717152\\
1236	-113.488032889171\\
1237	-95.1928374922281\\
1238	-101.492208150445\\
1239	-126.86996145354\\
1240	-99.222643169865\\
1241	-39.9090029413026\\
1242	-55.0273357296605\\
1243	-66.2209239254103\\
1244	-86.8512001741403\\
1245	-76.933359611324\\
1246	-61.811227485289\\
1247	-88.3545927120006\\
1248	-81.4232869399591\\
1249	-62.2698988805782\\
1250	-79.7457713841613\\
1251	-70.2764305526603\\
1252	-63.4660162430566\\
1253	-76.0632894539458\\
1254	-48.1775018760654\\
1255	-56.7820980194467\\
1256	-41.8626901365044\\
1257	-46.6799800602562\\
1258	-82.519252014084\\
1259	-84.4102561348648\\
1260	-107.316619341977\\
1261	-140.106483381208\\
1263	-65.2334407750313\\
1264	-49.2651798901911\\
1265	-49.8755062352316\\
1266	-73.9693299790983\\
1267	-47.276477726326\\
1268	-51.7500723933806\\
1269	-65.0050564456756\\
1270	-103.968726528934\\
1271	-94.5909295513413\\
1272	-126.812729625668\\
1273	-93.9842537022585\\
1274	-81.3304757306298\\
1275	-45.2677154366706\\
1276	-46.0142661651741\\
1277	-28.8937380214786\\
1278	-35.1942727506359\\
1279	-40.2329378872159\\
1280	-67.2473240900754\\
1282	-76.755875716884\\
1283	-81.8357512403716\\
1284	-118.498172532877\\
1285	-117.91974830043\\
1286	-87.6615192806917\\
1287	-100.10771471481\\
1288	-108.673475935107\\
1289	-130.718261765623\\
1290	-109.253951960346\\
1291	-78.4047486999204\\
1292	-35.510526282663\\
1293	-25.2259745201429\\
1294	-29.0632708003575\\
1295	-37.7373423847459\\
1296	-38.8131168367113\\
1297	-37.7424683493339\\
1298	-47.0744796369663\\
1299	-78.3932918045814\\
1300	-71.685052695075\\
1301	-69.1151095962771\\
1302	-80.4613047956641\\
1303	-54.3713888167063\\
1304	-32.2029187385458\\
1306	-89.377016120942\\
1307	-83.3035638790993\\
1308	-94.6382351560774\\
1309	-80.3146739121664\\
1310	-63.7969128361253\\
1311	-82.5675249900742\\
1312	-107.999764517376\\
1313	-108.538834274783\\
1314	-82.7573609027124\\
1315	-123.714031097531\\
1316	-100.8776302186\\
1317	-93.5896091811264\\
1318	-113.005117184177\\
1319	-107.770827430575\\
1320	-89.8871579661945\\
1321	-81.996519262229\\
1322	-118.906759155454\\
1323	-161.476153645646\\
1324	-129.092742426408\\
1325	-81.4168003979487\\
1326	-74.3948556014743\\
1328	-95.1727284329834\\
1329	-105.253252780483\\
1330	-84.9627924522506\\
1331	-93.4054988288178\\
1332	-113.239611335371\\
1333	-116.178908286981\\
1334	-86.045770354966\\
1335	-77.0230736932895\\
1336	-93.2314621773965\\
1337	-147.038934157991\\
1338	-129.567348459838\\
1339	-126.976534242021\\
1340	-84.4230176177423\\
1341	-75.2623529449513\\
1342	-75.6098478989945\\
1343	-48.9363792459599\\
1344	-28.5646662478066\\
1345	-21.372927084135\\
1346	-18.6197849782691\\
1347	-50.9247045261761\\
1348	-72.2131625234035\\
1349	-85.1377400801271\\
1350	-67.5240981955537\\
1351	-70.7166646027247\\
1352	-81.347658604057\\
1353	-52.9758829502493\\
1354	-57.22889453777\\
1355	-56.3899608964375\\
1356	-43.3915925076856\\
1357	-55.5482776581682\\
1358	-82.0375355793162\\
1359	-101.019540416127\\
1360	-63.3965300339478\\
1361	-52.2073564565885\\
1362	-47.5105626946167\\
1363	-58.0187808111111\\
1364	-37.5849429265556\\
1366	-129.107118025643\\
1367	-106.017496604623\\
1369	-147.385827121934\\
1370	-151.768251996196\\
1371	-117.866787620378\\
1372	-91.8422060757648\\
1373	-94.4702047622518\\
1374	-91.987830382523\\
1376	-117.463597115828\\
1377	-114.263658178791\\
1378	-146.457383363334\\
1379	-161.947168879911\\
1380	-184.312100493266\\
1381	-140.099861261112\\
1382	-153.831818458573\\
1383	-160.821115756034\\
1384	-138.147245074551\\
1385	-149.618132558744\\
1386	-134.172205116665\\
1387	-82.0297406725135\\
1388	-48.5987703486126\\
1389	-53.378464464904\\
1390	-79.5242439369322\\
1391	-65.8461264555665\\
1392	-46.8333175214204\\
1393	-66.8903093372724\\
1394	-50.6272438595756\\
1395	-40.522440917432\\
1396	-50.7832848720807\\
1397	-52.3305537897422\\
1398	-43.8884319314807\\
1399	-70.8938508308413\\
1400	-93.7504901507455\\
1401	-72.0281662522907\\
1402	-115.595846965114\\
1403	-164.345931314174\\
1404	-150.08262837334\\
1405	-179.315691319192\\
1406	-132.404322405801\\
1407	-143.249957897843\\
1408	-100.697510243435\\
1409	-64.7192065165568\\
1410	-78.4557539810821\\
1411	-67.5448474382647\\
1412	-49.5973732822317\\
1413	-65.8640729223737\\
1415	-22.9569459369252\\
1416	-33.0818121733678\\
1417	-70.6128236434015\\
1418	-90.2823813240364\\
1419	-101.778382129934\\
1420	-103.347614814603\\
1421	-96.0302701105038\\
1422	-107.671285328245\\
1423	-93.4209662031087\\
1424	-80.8710296074039\\
1425	-82.3155666385508\\
1426	-69.5433209031096\\
1427	-99.1717218886338\\
1428	-69.1094098667891\\
1429	-60.976338946748\\
1430	-87.3336527750528\\
1431	-108.252004506255\\
1432	-86.2578171357852\\
1433	-70.0021745352644\\
1434	-64.6048278287799\\
1435	-72.9175585475673\\
1436	-50.8683491108116\\
1437	-48.8966490103326\\
1438	-65.8659718285853\\
1440	-82.5257162330286\\
1441	-71.4124811350764\\
1442	-84.9291264780154\\
1443	-78.6492234031762\\
1444	-51.5584222750942\\
1445	-55.7892963918714\\
1447	-131.109108552859\\
1448	-121.561100990684\\
1449	-104.047890875796\\
1450	-100.552222631746\\
1451	-87.2051596387448\\
1452	-79.6897138850679\\
1453	-70.5746736032775\\
1454	-85.7197906801166\\
1455	-83.3232549059774\\
1456	-53.5652741318431\\
1457	-79.2318693699622\\
1458	-82.7481030637553\\
1460	-105.400330251025\\
1461	-98.7368036995465\\
1462	-138.223247406281\\
1463	-162.364254462904\\
1464	-178.961634765957\\
1465	-131.961163314807\\
1466	-73.2829367165316\\
1467	-43.3226683631644\\
1468	-30.6387588537207\\
1469	-51.703994221295\\
1470	-45.1472662418689\\
1471	-78.2952848477883\\
1472	-70.5415866746398\\
1473	-65.083534391187\\
1474	-83.4727659267571\\
1475	-92.5516542003904\\
1476	-108.455254775803\\
1477	-113.810077536755\\
1478	-157.747450547442\\
1479	-148.050268645153\\
1480	-116.308228055216\\
1481	-89.2906373136041\\
1482	-82.9258153593573\\
1483	-92.8838187703545\\
1484	-105.156628185123\\
1485	-81.1598326629887\\
1486	-80.9906904070385\\
1487	-85.960219258418\\
1488	-42.7014355373979\\
1489	-75.5460252367591\\
1490	-104.684502022924\\
1491	-77.6857282944272\\
1492	-77.1432804723702\\
1493	-144.974396518632\\
1494	-105.918362658012\\
1495	-105.97497122001\\
1496	-135.999576409288\\
1497	-130.975117335179\\
1498	-93.9601367696316\\
1499	-61.2566231412966\\
1500	-61.8892441007386\\
1502	-138.886310010987\\
1503	-147.228745665937\\
1504	-146.52185465099\\
1505	-129.328250233598\\
1506	-81.5498843655064\\
1507	-76.9020379966939\\
1508	-56.4598309819407\\
1509	-56.3751808268273\\
1510	-46.9063052200661\\
1511	-66.4713245583346\\
1512	-72.5824615958409\\
1513	-45.167996168148\\
1514	-72.809156396135\\
1515	-91.5618643214109\\
1516	-96.0990723720613\\
1517	-120.421884769502\\
1518	-152.453466141458\\
1519	-174.824007814952\\
1520	-135.232888270401\\
1521	-123.761807401656\\
1522	-126.219901099541\\
1523	-109.285550350098\\
1524	-78.3427112263776\\
1525	-88.8238119346772\\
1526	-134.07610641539\\
1527	-148.316255206038\\
1528	-95.8125311677352\\
1529	-54.3151861544218\\
1530	-35.6899117986291\\
1531	-9.52466468958482\\
1532	-19.171744912677\\
1533	-40.0820477622922\\
1534	-44.598744697465\\
1535	-72.5210862406875\\
1537	-46.6482759644584\\
1538	-48.0598252943962\\
1539	-47.1662100688454\\
1540	-44.6612838550448\\
1541	-51.5963981355374\\
1542	-72.9457420714102\\
1543	-105.961023880811\\
1544	-108.300961511743\\
1545	-105.236102790534\\
1546	-113.231181665007\\
1547	-146.785208362149\\
1548	-140.351754424976\\
1549	-163.003268568451\\
1550	-161.827714283396\\
1552	-91.1598703800526\\
1553	-86.3694023206717\\
1554	-135.240749143761\\
1555	-148.986366983082\\
1556	-112.493834201813\\
1557	-84.5054305255712\\
1558	-23.2483432252595\\
1559	-21.6361075361372\\
1560	-26.113209618497\\
1561	-10.3141190321587\\
1562	-38.065421089686\\
1563	-56.5651337185623\\
1564	-56.3155381803001\\
1565	-53.2664929870464\\
1566	-31.4256825999264\\
1567	-29.4320209729265\\
1568	-35.5029589009227\\
1569	-43.4893218827453\\
1570	-42.6081140264648\\
1571	-36.8086969157619\\
1572	-32.2855554042142\\
1573	-52.8523732747917\\
1574	-118.235355221572\\
1575	-138.757297937004\\
1576	-139.648634425264\\
1577	-110.537976924458\\
1578	-129.097867300357\\
1579	-193.631377583735\\
1580	-164.526099393517\\
1581	-175.495856207471\\
1582	-134.628295356833\\
1584	-144.754817412768\\
1585	-103.088096304437\\
1586	-95.3073355475512\\
1587	-56.7405328710424\\
1588	-51.3296575301238\\
1589	-105.134799532819\\
1590	-69.673983828849\\
1591	-74.618707531333\\
1592	-83.4118615404259\\
1593	-78.8676553001681\\
1594	-84.7823685983815\\
1595	-82.46811940152\\
1596	-94.280592087071\\
1597	-99.8424226440375\\
1598	-89.3034594249766\\
1599	-72.8577098254507\\
1600	-90.8665516544756\\
1601	-35.4846452075476\\
1602	-14.9974938419907\\
1603	-25.7300022047373\\
1604	-51.2592045865026\\
1605	-21.55386208706\\
1606	-9.59210165915488\\
1607	-24.0734116702395\\
1608	-45.2464376033536\\
1609	-56.2256735114529\\
1610	-70.3617673511908\\
1611	-61.4727494635658\\
1612	-88.6329601216105\\
1613	-86.3900143812762\\
1614	-60.3903528979679\\
1615	-86.7017967807201\\
1616	-45.1493155251251\\
1617	-40.5274620757914\\
1619	-16.7627519349617\\
1620	-34.761532696924\\
1621	-25.7216536867988\\
1622	-47.0618117364443\\
1623	-74.6374349614962\\
1624	-70.5913406997802\\
1625	-55.2441552852026\\
1626	-59.9682284970261\\
1627	-49.9610465426326\\
1628	-62.0010301417428\\
1629	-48.2930606770838\\
1630	-44.91123447021\\
1631	-44.6053261071688\\
1632	-32.8916670899537\\
1633	-43.7008526437012\\
1634	-37.4446377435145\\
1635	-64.0054481415832\\
1636	-58.7259304790616\\
1637	-47.2623139452462\\
1638	-49.0469598971129\\
1639	-37.5045224670398\\
1640	-43.8800038686909\\
1641	-88.0436841732492\\
1642	-117.821635030521\\
1643	-81.3575105647678\\
1644	-78.4733305105178\\
1645	-53.9065733700522\\
1646	-89.0849223529283\\
1647	-82.8029510753245\\
1648	-54.2136271118134\\
1649	-73.6128064745456\\
1650	-54.2608827618278\\
1651	-72.1224067776259\\
1652	-44.7967294428424\\
1653	-43.6397294375145\\
1654	-58.4704499043166\\
1655	-69.3985330970897\\
1656	-55.4456739954551\\
1657	-58.2403843202578\\
1658	-57.3698448654727\\
1659	-86.9839329300851\\
1660	-78.645928024974\\
1661	-109.008353107459\\
1662	-146.765519322945\\
1663	-120.802125727686\\
1664	-80.7218121005908\\
1665	-61.5698292190777\\
1666	-88.4951532134348\\
1667	-110.938757347275\\
1668	-84.6099842383728\\
1669	-99.482693459074\\
1671	-32.401187346522\\
1672	-33.6106903005887\\
1673	-80.7095924848838\\
1675	-65.1397865293688\\
1676	-95.6143228809892\\
1677	-92.0907095004895\\
1678	-81.3971143535089\\
1679	-62.1129320244049\\
1680	-74.2310738064605\\
1681	-93.156385150214\\
1682	-80.6227753140554\\
1683	-80.8427116089183\\
1684	-73.5433148032712\\
1685	-59.6497102879305\\
1686	-69.8284928516564\\
1687	-49.6672694183537\\
1688	-60.2891861434393\\
1689	-93.4953531853128\\
1691	-102.743802492358\\
1692	-93.9519205359268\\
1693	-72.59512035227\\
1694	-54.5816702614779\\
1695	-61.8348555734794\\
1696	-70.1801444403216\\
1699	-117.637861219742\\
1700	-99.5574462076327\\
1701	-60.739195389936\\
1703	-138.556159428799\\
1704	-98.9849852078235\\
1705	-102.94428422854\\
1706	-111.357850243762\\
1707	-93.2394369260662\\
1708	-79.1161965985257\\
1709	-88.7769474361844\\
1710	-79.4253809195864\\
1711	-87.9256152252115\\
1712	-109.566891529764\\
1713	-81.3087235758715\\
1714	-95.6429232235919\\
1715	-89.5405808141682\\
1716	-92.8797541885185\\
1717	-77.3837788975193\\
1718	-94.8425394091414\\
1719	-71.009714896127\\
1720	-75.7864265945198\\
1721	-81.5318977426296\\
1722	-80.3663312730037\\
1723	-42.5235903987657\\
1724	-25.4128988972498\\
1725	-14.9395379790967\\
1726	-30.7302994763663\\
1727	-89.5048208173589\\
1728	-109.227703359876\\
1729	-101.201238507129\\
1730	-75.8586643710387\\
1731	-84.736841183565\\
1732	-77.5659946987207\\
1733	-68.421429371105\\
1734	-44.3528383022858\\
1735	-59.2656052742877\\
1736	-60.4132366787892\\
1737	-60.8271223514589\\
1738	-66.1442318532227\\
1739	-58.1921336249186\\
1740	-56.6775250015417\\
1741	-40.5508026130979\\
1742	-52.0168156549971\\
1743	-93.8797299681216\\
1744	-70.1161545039536\\
1745	-83.2342074207404\\
1746	-72.4207811082179\\
1747	-95.8231378499079\\
1748	-87.4603253293694\\
1749	-112.831327098046\\
1750	-91.5804253448787\\
1751	-94.8050569949041\\
1752	-98.7700790984736\\
1753	-54.2448783146642\\
1754	-92.2253621912619\\
1755	-63.5395615404916\\
1756	-83.2004627750866\\
1757	-84.4653789155584\\
1758	-100.547666553981\\
1759	-81.5230789870477\\
1760	-98.7724522223875\\
1761	-120.825788948112\\
1762	-101.567185162082\\
1763	-92.0529682147919\\
1764	-77.3935230487568\\
1765	-85.3466491886395\\
1766	-83.3946567412625\\
1767	-68.709452991556\\
1768	-63.4363489783104\\
1769	-71.0162384614316\\
1770	-63.7203088733008\\
1771	-113.754195149718\\
1772	-137.470386728632\\
1773	-117.907119431351\\
1774	-112.705994343065\\
1775	-112.789315497032\\
1776	-68.9381403266925\\
1777	-62.7321783252512\\
1778	-50.4847316648672\\
1779	-47.7012817181551\\
1780	-52.8840027910294\\
1781	-82.0755805096444\\
1783	-102.643816753876\\
1784	-91.8788750206902\\
1785	-68.4361924603375\\
1786	-112.704537163105\\
1787	-124.214619485128\\
1788	-131.363457200892\\
1789	-114.961374980003\\
1790	-157.363481743212\\
1791	-136.850465622811\\
1792	-78.0879286563609\\
1793	-55.7762819859674\\
1794	-51.4339606349486\\
1795	-85.5552043382572\\
1796	-104.109321394116\\
1797	-127.672848355037\\
1798	-105.983300566087\\
1799	-86.8744072696584\\
1800	-48.6452170590969\\
1801	-51.9971251600396\\
1802	-58.9259452037725\\
1803	-62.6680587330841\\
1804	-45.9363010158418\\
1805	-55.3976765030961\\
};
\addlegendentry{MPO prediction}

\end{axis}

\begin{axis}[%
width=6.159cm,
height=1.831cm,
at={(0cm,0cm)},
scale only axis,
xmin=1000,
xmax=2000,
xlabel style={font=\color{white!15!black}},
xlabel={Sample index},
ymin=-200,
ymax=0,
ylabel style={font=\color{white!15!black}},
ylabel={$y(t)$},
axis background/.style={fill=white},
title style={font=\bfseries},
title={C9: RMSE(OSA) = 3.9697, RMSE(MPO) = 6.995},
legend style={legend cell align=left, align=left, draw=white!15!black}
]
\addplot [color=mycolor1, line width=2.0pt]
  table[row sep=crcr]{%
1006	-86.6700000000001\\
1007	-103.76\\
1008	-79.346\\
1009	-45.1659999999999\\
1010	-57.373\\
1011	-51.27\\
1012	-63.4770000000001\\
1013	-48.828\\
1014	-24.414\\
1015	-17.0899999999999\\
1016	-17.0899999999999\\
1017	-43.9449999999999\\
1018	-73.242\\
1019	-76.904\\
1020	-79.346\\
1022	-36.6210000000001\\
1023	-76.904\\
1024	-53.711\\
1025	-43.9449999999999\\
1026	-69.5799999999999\\
1027	-61.0350000000001\\
1028	-48.828\\
1029	-84.229\\
1030	-79.346\\
1031	-59.8140000000001\\
1032	-87.8910000000001\\
1033	-85.4490000000001\\
1034	-67.1389999999999\\
1035	-54.932\\
1036	-46.3869999999999\\
1037	-56.152\\
1038	-64.6970000000001\\
1039	-64.6970000000001\\
1040	-67.1389999999999\\
1041	-68.3589999999999\\
1042	-73.242\\
1043	-100.098\\
1044	-87.8910000000001\\
1045	-61.0350000000001\\
1046	-56.152\\
1047	-34.1800000000001\\
1048	-34.1800000000001\\
1049	-48.828\\
1050	-37.8420000000001\\
1051	-39.0630000000001\\
1052	-56.152\\
1053	-62.2560000000001\\
1054	-58.5940000000001\\
1055	-90.3320000000001\\
1056	-64.6970000000001\\
1057	-41.5039999999999\\
1058	-34.1800000000001\\
1059	-45.1659999999999\\
1060	-51.27\\
1061	-28.076\\
1062	-25.635\\
1063	-41.5039999999999\\
1064	-34.1800000000001\\
1065	-29.297\\
1066	-40.2829999999999\\
1067	-43.9449999999999\\
1068	-68.3589999999999\\
1069	-63.4770000000001\\
1070	-79.346\\
1071	-69.5799999999999\\
1072	-79.346\\
1073	-63.4770000000001\\
1074	-62.2560000000001\\
1075	-54.932\\
1076	-53.711\\
1077	-53.711\\
1079	-126.953\\
1080	-131.836\\
1081	-129.395\\
1082	-83.008\\
1083	-114.746\\
1084	-141.602\\
1085	-147.705\\
1086	-111.084\\
1087	-146.484\\
1088	-185.547\\
1089	-131.836\\
1090	-112.305\\
1091	-75.684\\
1092	-61.0350000000001\\
1093	-52.49\\
1094	-64.6970000000001\\
1095	-46.3869999999999\\
1096	-31.7380000000001\\
1097	-31.7380000000001\\
1098	-40.2829999999999\\
1099	-57.373\\
1100	-52.49\\
1101	-53.711\\
1102	-72.021\\
1103	-72.021\\
1104	-97.6559999999999\\
1105	-79.346\\
1106	-101.318\\
1107	-72.021\\
1108	-64.6970000000001\\
1109	-73.242\\
1110	-54.932\\
1111	-51.27\\
1112	-59.8140000000001\\
1113	-62.2560000000001\\
1114	-43.9449999999999\\
1115	-47.607\\
1116	-34.1800000000001\\
1117	-40.2829999999999\\
1118	-42.7249999999999\\
1119	-30.518\\
1121	-47.607\\
1122	-72.021\\
1123	-74.463\\
1124	-81.787\\
1125	-48.828\\
1126	-70.8009999999999\\
1127	-101.318\\
1128	-84.229\\
1129	-93.9939999999999\\
1130	-91.5530000000001\\
1131	-54.932\\
1132	-40.2829999999999\\
1133	-56.152\\
1134	-61.0350000000001\\
1135	-97.6559999999999\\
1136	-119.629\\
1137	-117.188\\
1138	-89.1110000000001\\
1139	-92.7729999999999\\
1140	-81.787\\
1142	-79.346\\
1143	-54.932\\
1144	-48.828\\
1145	-45.1659999999999\\
1146	-40.2829999999999\\
1147	-45.1659999999999\\
1148	-64.6970000000001\\
1149	-96.4359999999999\\
1151	-52.49\\
1152	-45.1659999999999\\
1153	-46.3869999999999\\
1154	-29.297\\
1155	-23.193\\
1156	-26.855\\
1157	-46.3869999999999\\
1158	-35.4000000000001\\
1159	-41.5039999999999\\
1161	-39.0630000000001\\
1162	-28.076\\
1163	-21.973\\
1164	-17.0899999999999\\
1165	-29.297\\
1166	-58.5940000000001\\
1167	-80.566\\
1168	-91.5530000000001\\
1169	-59.8140000000001\\
1170	-43.9449999999999\\
1171	-32.9590000000001\\
1172	-23.193\\
1173	-47.607\\
1174	-46.3869999999999\\
1175	-67.1389999999999\\
1176	-81.787\\
1177	-131.836\\
1178	-147.705\\
1179	-120.85\\
1180	-117.188\\
1181	-78.125\\
1182	-81.787\\
1183	-91.5530000000001\\
1184	-98.877\\
1185	-73.242\\
1186	-68.3589999999999\\
1187	-69.5799999999999\\
1188	-62.2560000000001\\
1189	-75.684\\
1190	-53.711\\
1191	-93.9939999999999\\
1192	-108.643\\
1194	-51.27\\
1195	-54.932\\
1196	-92.7729999999999\\
1197	-109.863\\
1198	-134.277\\
1199	-140.381\\
1200	-140.381\\
1201	-106.201\\
1202	-96.4359999999999\\
1203	-111.084\\
1204	-129.395\\
1205	-78.125\\
1206	-50.049\\
1207	-73.242\\
1208	-57.373\\
1209	-39.0630000000001\\
1210	-56.152\\
1211	-61.0350000000001\\
1212	-46.3869999999999\\
1213	-37.8420000000001\\
1214	-50.049\\
1215	-39.0630000000001\\
1216	-53.711\\
1217	-76.904\\
1218	-72.021\\
1219	-51.27\\
1220	-51.27\\
1221	-78.125\\
1222	-123.291\\
1223	-86.6700000000001\\
1224	-56.152\\
1225	-43.9449999999999\\
1226	-48.828\\
1227	-43.9449999999999\\
1228	-34.1800000000001\\
1229	-37.8420000000001\\
1230	-47.607\\
1231	-59.8140000000001\\
1232	-64.6970000000001\\
1233	-46.3869999999999\\
1234	-74.463\\
1235	-85.4490000000001\\
1236	-83.008\\
1237	-65.9180000000001\\
1238	-73.242\\
1239	-96.4359999999999\\
1240	-61.0350000000001\\
1241	-29.297\\
1242	-42.7249999999999\\
1243	-43.9449999999999\\
1244	-58.5940000000001\\
1245	-50.049\\
1246	-40.2829999999999\\
1247	-58.5940000000001\\
1248	-54.932\\
1249	-39.0630000000001\\
1250	-48.828\\
1252	-39.0630000000001\\
1253	-50.049\\
1254	-29.297\\
1255	-36.6210000000001\\
1256	-26.855\\
1257	-31.7380000000001\\
1258	-53.711\\
1259	-61.0350000000001\\
1260	-76.904\\
1261	-101.318\\
1262	-67.1389999999999\\
1263	-42.7249999999999\\
1264	-32.9590000000001\\
1265	-31.7380000000001\\
1266	-47.607\\
1267	-30.518\\
1268	-36.6210000000001\\
1269	-43.9449999999999\\
1270	-65.9180000000001\\
1271	-62.2560000000001\\
1272	-86.6700000000001\\
1273	-59.8140000000001\\
1274	-56.152\\
1275	-31.7380000000001\\
1276	-32.9590000000001\\
1277	-20.752\\
1278	-24.414\\
1279	-26.855\\
1280	-47.607\\
1281	-47.607\\
1282	-56.152\\
1283	-59.8140000000001\\
1284	-93.9939999999999\\
1285	-79.346\\
1286	-62.2560000000001\\
1287	-73.242\\
1288	-79.346\\
1289	-101.318\\
1290	-80.566\\
1291	-52.49\\
1292	-30.518\\
1293	-21.973\\
1294	-23.193\\
1295	-29.297\\
1296	-29.297\\
1297	-26.855\\
1298	-36.6210000000001\\
1299	-53.711\\
1300	-47.607\\
1301	-51.27\\
1302	-57.373\\
1303	-37.8420000000001\\
1304	-20.752\\
1306	-63.4770000000001\\
1307	-61.0350000000001\\
1308	-69.5799999999999\\
1309	-58.5940000000001\\
1310	-45.1659999999999\\
1311	-59.8140000000001\\
1312	-84.229\\
1313	-85.4490000000001\\
1314	-59.8140000000001\\
1315	-102.539\\
1316	-76.904\\
1317	-74.463\\
1318	-84.229\\
1319	-83.008\\
1320	-63.4770000000001\\
1321	-54.932\\
1323	-124.512\\
1324	-95.2149999999999\\
1325	-57.373\\
1326	-53.711\\
1327	-54.932\\
1328	-72.021\\
1329	-83.008\\
1330	-59.8140000000001\\
1331	-64.6970000000001\\
1332	-83.008\\
1333	-90.3320000000001\\
1334	-61.0350000000001\\
1335	-50.049\\
1336	-63.4770000000001\\
1337	-104.98\\
1338	-92.7729999999999\\
1339	-95.2149999999999\\
1340	-58.5940000000001\\
1341	-53.711\\
1342	-51.27\\
1343	-34.1800000000001\\
1344	-23.193\\
1345	-20.752\\
1346	-17.0899999999999\\
1347	-40.2829999999999\\
1348	-54.932\\
1349	-62.2560000000001\\
1350	-47.607\\
1352	-57.373\\
1353	-37.8420000000001\\
1354	-39.0630000000001\\
1355	-39.0630000000001\\
1356	-28.076\\
1357	-39.0630000000001\\
1358	-57.373\\
1359	-73.242\\
1360	-48.828\\
1361	-35.4000000000001\\
1362	-31.7380000000001\\
1363	-36.6210000000001\\
1364	-26.855\\
1365	-54.932\\
1366	-93.9939999999999\\
1367	-72.021\\
1368	-97.6559999999999\\
1369	-113.525\\
1370	-114.746\\
1371	-85.4490000000001\\
1372	-63.4770000000001\\
1374	-63.4770000000001\\
1375	-75.684\\
1376	-89.1110000000001\\
1377	-86.6700000000001\\
1378	-115.967\\
1379	-125.732\\
1380	-150.146\\
1381	-96.4359999999999\\
1382	-109.863\\
1383	-122.07\\
1384	-92.7729999999999\\
1385	-107.422\\
1386	-92.7729999999999\\
1387	-54.932\\
1388	-39.0630000000001\\
1389	-40.2829999999999\\
1390	-57.373\\
1391	-42.7249999999999\\
1392	-32.9590000000001\\
1393	-45.1659999999999\\
1394	-34.1800000000001\\
1395	-26.855\\
1396	-34.1800000000001\\
1397	-37.8420000000001\\
1398	-25.635\\
1399	-48.828\\
1400	-64.6970000000001\\
1401	-46.3869999999999\\
1403	-115.967\\
1404	-106.201\\
1405	-133.057\\
1406	-92.7729999999999\\
1407	-97.6559999999999\\
1408	-69.5799999999999\\
1409	-43.9449999999999\\
1410	-53.711\\
1411	-42.7249999999999\\
1412	-34.1800000000001\\
1413	-45.1659999999999\\
1414	-26.855\\
1415	-20.752\\
1416	-25.635\\
1417	-50.049\\
1418	-61.0350000000001\\
1419	-78.125\\
1420	-74.463\\
1421	-69.5799999999999\\
1422	-80.566\\
1423	-64.6970000000001\\
1424	-53.711\\
1425	-54.932\\
1426	-43.9449999999999\\
1427	-69.5799999999999\\
1428	-50.049\\
1429	-40.2829999999999\\
1430	-58.5940000000001\\
1431	-80.566\\
1432	-59.8140000000001\\
1433	-46.3869999999999\\
1434	-41.5039999999999\\
1435	-47.607\\
1436	-32.9590000000001\\
1437	-31.7380000000001\\
1438	-43.9449999999999\\
1439	-53.711\\
1440	-59.8140000000001\\
1441	-48.828\\
1442	-61.0350000000001\\
1443	-57.373\\
1444	-34.1800000000001\\
1445	-37.8420000000001\\
1446	-65.9180000000001\\
1447	-91.5530000000001\\
1448	-90.3320000000001\\
1449	-75.684\\
1450	-73.242\\
1451	-57.373\\
1452	-53.711\\
1453	-43.9449999999999\\
1454	-61.0350000000001\\
1455	-54.932\\
1456	-36.6210000000001\\
1457	-52.49\\
1458	-61.0350000000001\\
1459	-70.8009999999999\\
1460	-76.904\\
1461	-76.904\\
1462	-102.539\\
1463	-124.512\\
1464	-140.381\\
1465	-91.5530000000001\\
1466	-52.49\\
1467	-34.1800000000001\\
1468	-25.635\\
1469	-36.6210000000001\\
1470	-32.9590000000001\\
1471	-58.5940000000001\\
1472	-53.711\\
1473	-46.3869999999999\\
1474	-62.2560000000001\\
1476	-81.787\\
1477	-86.6700000000001\\
1478	-122.07\\
1481	-58.5940000000001\\
1482	-58.5940000000001\\
1483	-59.8140000000001\\
1484	-75.684\\
1485	-54.932\\
1486	-56.152\\
1487	-53.711\\
1488	-30.518\\
1489	-54.932\\
1490	-70.8009999999999\\
1491	-50.049\\
1492	-53.711\\
1493	-100.098\\
1494	-75.684\\
1495	-72.021\\
1496	-102.539\\
1497	-98.877\\
1498	-62.2560000000001\\
1499	-40.2829999999999\\
1500	-42.7249999999999\\
1501	-70.8009999999999\\
1502	-107.422\\
1503	-112.305\\
1504	-115.967\\
1505	-90.3320000000001\\
1506	-59.8140000000001\\
1507	-52.49\\
1508	-39.0630000000001\\
1510	-31.7380000000001\\
1511	-45.1659999999999\\
1512	-50.049\\
1513	-30.518\\
1514	-42.7249999999999\\
1515	-61.0350000000001\\
1516	-68.3589999999999\\
1517	-84.229\\
1518	-107.422\\
1519	-128.174\\
1520	-95.2149999999999\\
1521	-81.787\\
1522	-85.4490000000001\\
1523	-72.021\\
1524	-52.49\\
1525	-63.4770000000001\\
1526	-102.539\\
1527	-115.967\\
1528	-69.5799999999999\\
1529	-40.2829999999999\\
1530	-29.297\\
1531	-17.0899999999999\\
1532	-21.973\\
1533	-32.9590000000001\\
1534	-36.6210000000001\\
1535	-52.49\\
1536	-43.9449999999999\\
1537	-31.7380000000001\\
1538	-34.1800000000001\\
1539	-31.7380000000001\\
1540	-30.518\\
1541	-34.1800000000001\\
1542	-51.27\\
1543	-75.684\\
1544	-76.904\\
1545	-75.684\\
1546	-86.6700000000001\\
1547	-107.422\\
1548	-107.422\\
1549	-128.174\\
1550	-115.967\\
1551	-86.6700000000001\\
1552	-61.0350000000001\\
1553	-58.5940000000001\\
1554	-98.877\\
1555	-111.084\\
1556	-79.346\\
1558	-29.297\\
1559	-20.752\\
1560	-24.414\\
1561	-18.3109999999999\\
1562	-29.297\\
1563	-43.9449999999999\\
1564	-42.7249999999999\\
1565	-39.0630000000001\\
1566	-26.855\\
1567	-20.752\\
1568	-29.297\\
1569	-29.297\\
1570	-31.7380000000001\\
1571	-24.414\\
1572	-21.973\\
1573	-34.1800000000001\\
1574	-78.125\\
1575	-91.5530000000001\\
1576	-100.098\\
1577	-70.8009999999999\\
1578	-97.6559999999999\\
1579	-140.381\\
1580	-128.174\\
1581	-128.174\\
1582	-97.6559999999999\\
1583	-95.2149999999999\\
1584	-102.539\\
1585	-68.3589999999999\\
1586	-64.6970000000001\\
1587	-42.7249999999999\\
1588	-41.5039999999999\\
1589	-69.5799999999999\\
1590	-53.711\\
1591	-51.27\\
1592	-57.373\\
1593	-54.932\\
1594	-54.932\\
1595	-59.8140000000001\\
1597	-72.021\\
1598	-61.0350000000001\\
1599	-45.1659999999999\\
1600	-59.8140000000001\\
1601	-26.855\\
1602	-18.3109999999999\\
1603	-23.193\\
1604	-37.8420000000001\\
1605	-23.193\\
1606	-10.9860000000001\\
1607	-21.973\\
1608	-36.6210000000001\\
1609	-42.7249999999999\\
1610	-51.27\\
1611	-43.9449999999999\\
1612	-68.3589999999999\\
1613	-57.373\\
1614	-40.2829999999999\\
1615	-58.5940000000001\\
1616	-41.5039999999999\\
1617	-29.297\\
1618	-23.193\\
1619	-15.8689999999999\\
1620	-25.635\\
1621	-19.5309999999999\\
1622	-35.4000000000001\\
1623	-54.932\\
1624	-53.711\\
1625	-36.6210000000001\\
1626	-43.9449999999999\\
1627	-34.1800000000001\\
1628	-43.9449999999999\\
1629	-36.6210000000001\\
1630	-31.7380000000001\\
1631	-31.7380000000001\\
1632	-25.635\\
1633	-31.7380000000001\\
1634	-28.076\\
1635	-43.9449999999999\\
1636	-43.9449999999999\\
1637	-34.1800000000001\\
1638	-32.9590000000001\\
1639	-26.855\\
1640	-30.518\\
1641	-64.6970000000001\\
1642	-84.229\\
1643	-56.152\\
1644	-56.152\\
1645	-37.8420000000001\\
1646	-65.9180000000001\\
1647	-62.2560000000001\\
1648	-39.0630000000001\\
1649	-48.828\\
1650	-41.5039999999999\\
1651	-51.27\\
1652	-36.6210000000001\\
1653	-32.9590000000001\\
1654	-43.9449999999999\\
1655	-52.49\\
1656	-37.8420000000001\\
1657	-42.7249999999999\\
1658	-45.1659999999999\\
1659	-64.6970000000001\\
1660	-58.5940000000001\\
1661	-81.787\\
1662	-111.084\\
1663	-86.6700000000001\\
1664	-57.373\\
1665	-43.9449999999999\\
1666	-67.1389999999999\\
1667	-80.566\\
1668	-63.4770000000001\\
1669	-75.684\\
1671	-25.635\\
1672	-26.855\\
1673	-57.373\\
1674	-53.711\\
1675	-48.828\\
1676	-74.463\\
1677	-68.3589999999999\\
1678	-61.0350000000001\\
1679	-45.1659999999999\\
1680	-54.932\\
1681	-73.242\\
1682	-59.8140000000001\\
1683	-59.8140000000001\\
1684	-56.152\\
1685	-42.7249999999999\\
1686	-46.3869999999999\\
1687	-36.6210000000001\\
1688	-40.2829999999999\\
1689	-67.1389999999999\\
1690	-78.125\\
1691	-80.566\\
1692	-69.5799999999999\\
1693	-52.49\\
1694	-37.8420000000001\\
1695	-42.7249999999999\\
1696	-51.27\\
1697	-63.4770000000001\\
1699	-90.3320000000001\\
1700	-72.021\\
1701	-42.7249999999999\\
1702	-73.242\\
1703	-106.201\\
1704	-75.684\\
1705	-72.021\\
1706	-86.6700000000001\\
1707	-65.9180000000001\\
1708	-54.932\\
1709	-63.4770000000001\\
1710	-57.373\\
1711	-64.6970000000001\\
1712	-79.346\\
1713	-62.2560000000001\\
1714	-70.8009999999999\\
1715	-67.1389999999999\\
1716	-67.1389999999999\\
1717	-54.932\\
1718	-69.5799999999999\\
1719	-52.49\\
1720	-52.49\\
1721	-59.8140000000001\\
1722	-57.373\\
1723	-31.7380000000001\\
1724	-20.752\\
1725	-15.8689999999999\\
1726	-25.635\\
1727	-59.8140000000001\\
1728	-78.125\\
1729	-75.684\\
1730	-52.49\\
1731	-62.2560000000001\\
1732	-57.373\\
1733	-47.607\\
1734	-34.1800000000001\\
1735	-42.7249999999999\\
1736	-43.9449999999999\\
1737	-41.5039999999999\\
1738	-46.3869999999999\\
1739	-40.2829999999999\\
1740	-37.8420000000001\\
1741	-29.297\\
1742	-39.0630000000001\\
1743	-63.4770000000001\\
1744	-47.607\\
1745	-54.932\\
1746	-54.932\\
1747	-67.1389999999999\\
1748	-64.6970000000001\\
1749	-85.4490000000001\\
1750	-67.1389999999999\\
1751	-72.021\\
1752	-73.242\\
1753	-41.5039999999999\\
1754	-64.6970000000001\\
1755	-42.7249999999999\\
1756	-58.5940000000001\\
1757	-63.4770000000001\\
1758	-76.904\\
1759	-59.8140000000001\\
1760	-75.684\\
1761	-93.9939999999999\\
1763	-64.6970000000001\\
1764	-56.152\\
1765	-63.4770000000001\\
1766	-56.152\\
1767	-47.607\\
1768	-41.5039999999999\\
1769	-48.828\\
1770	-43.9449999999999\\
1771	-79.346\\
1772	-107.422\\
1773	-91.5530000000001\\
1774	-87.8910000000001\\
1775	-86.6700000000001\\
1776	-50.049\\
1777	-46.3869999999999\\
1778	-37.8420000000001\\
1779	-32.9590000000001\\
1780	-35.4000000000001\\
1781	-57.373\\
1782	-73.242\\
1783	-80.566\\
1784	-68.3589999999999\\
1785	-47.607\\
1786	-84.229\\
1788	-103.76\\
1789	-84.229\\
1790	-134.277\\
1791	-101.318\\
1792	-56.152\\
1793	-42.7249999999999\\
1794	-36.6210000000001\\
1795	-61.0350000000001\\
1796	-79.346\\
1797	-100.098\\
1798	-79.346\\
1799	-63.4770000000001\\
1800	-35.4000000000001\\
1801	-36.6210000000001\\
1802	-39.0630000000001\\
1803	-43.9449999999999\\
1804	-30.518\\
1805	-36.6210000000001\\
};
\addlegendentry{True output}

\addplot [color=mycolor2, dashed, line width=2.0pt]
  table[row sep=crcr]{%
1006	-88.0285879115352\\
1007	-102.215711035965\\
1008	-81.6831289873805\\
1009	-44.4920181216498\\
1010	-54.985306145418\\
1011	-56.4484542579223\\
1012	-64.5118884980077\\
1013	-57.9839263317706\\
1014	-15.3954170179704\\
1015	-14.134123304558\\
1016	-15.0184584540866\\
1017	-52.9679495002224\\
1018	-79.9739725675174\\
1019	-80.7112588203793\\
1020	-78.9805772924817\\
1021	-62.5924597068915\\
1022	-37.2633624061154\\
1023	-80.3173608188365\\
1024	-55.2767804427144\\
1025	-43.8943133300031\\
1026	-70.9058074400448\\
1027	-62.1622231295964\\
1028	-54.8703675468307\\
1029	-84.6958816751428\\
1030	-79.3475972429915\\
1031	-60.5499688864111\\
1032	-83.543437228959\\
1033	-88.0062185179556\\
1034	-66.8904322958708\\
1035	-63.3578313779424\\
1036	-45.2458956911898\\
1037	-57.9837803079092\\
1038	-64.3173400806147\\
1039	-64.1325785430283\\
1040	-64.4937360726706\\
1041	-68.6029765182579\\
1042	-75.3440551683345\\
1043	-94.7097413133313\\
1044	-92.822450051226\\
1045	-60.1677385859389\\
1046	-64.8565941483398\\
1047	-30.7670965701113\\
1048	-37.9171489639798\\
1049	-48.3410616706255\\
1050	-38.7814124809793\\
1051	-44.6216654885179\\
1052	-52.5688349231084\\
1053	-65.674709617547\\
1054	-58.4728392821621\\
1055	-87.0264710373924\\
1056	-72.0287160996199\\
1057	-44.2040641636693\\
1058	-36.9531152536574\\
1059	-45.2589305291085\\
1060	-52.9261926880365\\
1061	-28.9114240055046\\
1062	-33.4936966288146\\
1063	-41.099549468745\\
1064	-35.3171522943915\\
1065	-30.4828537260851\\
1066	-41.2019679074913\\
1067	-45.7849303524986\\
1068	-68.6534598731182\\
1069	-67.0778155985163\\
1070	-74.4460082106423\\
1071	-73.8007122818619\\
1072	-76.7434704679131\\
1073	-66.145620320927\\
1074	-64.4597041673103\\
1075	-58.5765627136689\\
1076	-57.8885953262118\\
1077	-56.8517050632317\\
1078	-88.5924541764355\\
1079	-126.841186343893\\
1080	-126.793506303385\\
1081	-125.786298417855\\
1082	-88.0450261745218\\
1084	-141.231457589399\\
1085	-137.971730374414\\
1086	-113.23217427389\\
1087	-143.799593485876\\
1088	-179.112564427322\\
1089	-148.840168720271\\
1091	-80.7548784944224\\
1092	-61.8164131242606\\
1093	-56.4945495446345\\
1094	-65.764968386057\\
1095	-45.758830168103\\
1096	-28.6855709551546\\
1097	-30.7994283379639\\
1098	-39.1720566762408\\
1099	-63.1364450832671\\
1100	-53.7189856288587\\
1101	-59.4439681575623\\
1102	-68.890908266754\\
1103	-74.1716657860939\\
1104	-97.4251887370201\\
1105	-91.481510677623\\
1106	-93.2564174116239\\
1107	-83.3150065701034\\
1108	-63.359709850246\\
1109	-80.4355065266595\\
1110	-55.9135815171687\\
1111	-53.5366342533653\\
1112	-64.4399950566783\\
1113	-62.7596422382162\\
1114	-44.0568106735677\\
1115	-52.2108899649932\\
1116	-38.2361727628274\\
1118	-44.9250858249509\\
1119	-32.6137614950287\\
1120	-39.879192232376\\
1121	-50.1758846971734\\
1122	-75.847469257825\\
1123	-75.7258123532113\\
1124	-78.3152065913346\\
1125	-55.4764170742947\\
1126	-68.5760370759867\\
1127	-109.816351102771\\
1128	-80.3816442708276\\
1129	-95.2964227082816\\
1130	-91.2658572348821\\
1131	-59.5543480046686\\
1132	-41.5834270197906\\
1133	-59.1339002953912\\
1134	-64.5568894975361\\
1135	-95.2532168648877\\
1136	-117.810036776633\\
1137	-111.732428271931\\
1138	-98.3383492533032\\
1139	-92.1238143515361\\
1140	-91.0498051866114\\
1141	-82.3628874194176\\
1142	-80.7765547877084\\
1143	-55.610329919004\\
1144	-55.0191894253658\\
1145	-46.4118724301356\\
1146	-43.8627966506917\\
1147	-49.9326264675267\\
1148	-63.0601857906647\\
1149	-96.9460034446645\\
1151	-56.8473502213412\\
1152	-49.3404150668575\\
1153	-49.9879118633232\\
1154	-27.6532674561656\\
1155	-20.678471833975\\
1156	-27.0550891996945\\
1157	-49.5567753830769\\
1158	-35.149882108471\\
1159	-42.6713636673044\\
1160	-42.6007404153261\\
1161	-42.9084052116521\\
1162	-34.3305470947375\\
1163	-21.7491725852469\\
1164	-17.5065602343291\\
1165	-30.4442818999403\\
1166	-64.9166702150192\\
1167	-89.899943137314\\
1168	-90.4022695370861\\
1169	-64.7763726968433\\
1170	-46.9334668186007\\
1171	-34.5851308531676\\
1172	-25.4573126078676\\
1173	-51.02920958753\\
1174	-47.5806744448539\\
1175	-64.5485712800919\\
1176	-84.1940594167752\\
1177	-133.107958664201\\
1178	-155.39452382523\\
1179	-125.654575621333\\
1180	-123.971144783733\\
1181	-78.7963637543378\\
1182	-90.4078300892377\\
1183	-90.6949578396279\\
1184	-97.5486261762956\\
1185	-81.5933547477209\\
1186	-69.4250052956027\\
1187	-72.6772338530095\\
1188	-67.8764416168758\\
1189	-75.0924001703027\\
1190	-59.0154345776018\\
1191	-89.7911939816038\\
1192	-109.86653171068\\
1193	-82.6712415270667\\
1194	-52.1711464557804\\
1195	-59.3189878067574\\
1196	-92.5100524989336\\
1197	-108.739546852764\\
1198	-131.066166677327\\
1199	-134.847747373575\\
1200	-140.188607915595\\
1201	-115.294815259751\\
1202	-97.2877002867399\\
1203	-115.47128843651\\
1204	-122.621256166135\\
1205	-80.2909481368183\\
1206	-54.2443287114208\\
1207	-72.8507757858545\\
1208	-68.0519799851349\\
1209	-39.2360468545833\\
1210	-55.7164346579821\\
1211	-61.8468283118912\\
1212	-53.6946443043319\\
1213	-38.8834860484969\\
1214	-55.3856895127451\\
1215	-45.139420136928\\
1216	-53.7265308404665\\
1217	-78.898897781602\\
1218	-70.5106769417475\\
1219	-54.7636333166286\\
1220	-56.5159461889755\\
1221	-78.1436885154371\\
1222	-128.016659727485\\
1223	-96.6034725974375\\
1224	-56.5213816694479\\
1225	-43.1629452476939\\
1226	-56.0663444984154\\
1227	-47.399024797953\\
1228	-37.8047914856286\\
1229	-41.7740470348238\\
1230	-48.9306879813503\\
1231	-60.0538703584934\\
1232	-66.5958095274698\\
1233	-47.895236598283\\
1234	-80.1480360713015\\
1235	-88.9064046577673\\
1236	-81.293995974057\\
1237	-71.0615143121836\\
1238	-71.0098650907833\\
1239	-97.4795751136326\\
1240	-67.5274957428715\\
1241	-25.3165361553615\\
1242	-41.2723621761431\\
1243	-48.3522711316996\\
1244	-65.236262383452\\
1245	-52.1125110142571\\
1246	-40.9436562820583\\
1247	-61.7086651090569\\
1248	-57.9341428752709\\
1249	-42.2139942769188\\
1250	-55.9318936910422\\
1251	-46.5027930207264\\
1252	-46.2539916279727\\
1253	-47.4973510063176\\
1254	-32.3039150626544\\
1255	-37.4783547078125\\
1256	-27.8596384759633\\
1257	-33.7631383274745\\
1258	-57.8842108752945\\
1259	-62.3822225896406\\
1260	-79.2078156574555\\
1261	-105.011506316376\\
1262	-72.9542368116613\\
1263	-44.5970740181237\\
1264	-34.321971850401\\
1265	-35.445592693015\\
1266	-50.0429496709532\\
1267	-31.0769067153899\\
1268	-37.4187282235105\\
1269	-47.1829018856631\\
1270	-71.5323790399918\\
1271	-65.1194828703537\\
1272	-86.3962086467134\\
1273	-61.3097648281357\\
1274	-57.9584099058316\\
1275	-29.8272806371017\\
1276	-32.1182189821159\\
1277	-22.4797206551364\\
1278	-24.0619315359629\\
1279	-31.0671600066203\\
1280	-48.5665930938112\\
1281	-55.2377570101964\\
1282	-54.8991441051862\\
1283	-63.0980632206824\\
1284	-95.1665198547043\\
1285	-89.1885337175017\\
1286	-63.6563499748186\\
1287	-76.6645632945458\\
1288	-76.1972745847061\\
1289	-97.7248127245248\\
1290	-84.8490253285302\\
1292	-25.4487171090907\\
1293	-23.1289064266309\\
1294	-24.5230790222022\\
1295	-28.3729104782369\\
1296	-31.1634903364111\\
1297	-29.4953474474469\\
1298	-37.8092428682621\\
1299	-59.1172281011748\\
1300	-50.9490471940624\\
1301	-51.3004837071874\\
1302	-59.8528371647274\\
1303	-39.5264881236017\\
1304	-23.3482678043422\\
1306	-66.6596044013495\\
1307	-61.1549359294845\\
1308	-68.2646794989512\\
1309	-61.9549976007377\\
1310	-46.6050107654498\\
1311	-64.3721926849737\\
1312	-84.6534732331484\\
1313	-81.5835323657298\\
1314	-62.1406993076673\\
1315	-103.011274976048\\
1316	-80.8364824991265\\
1317	-77.5278342311076\\
1318	-84.5607591126545\\
1319	-84.2644071305572\\
1320	-65.4515693310286\\
1321	-61.4945052789865\\
1322	-88.0646991415645\\
1323	-124.124411939208\\
1324	-94.2128698976351\\
1325	-59.7251382585757\\
1326	-57.3455323298817\\
1327	-58.8836257583087\\
1328	-73.3414946660159\\
1329	-81.1800386383736\\
1330	-60.4222486457616\\
1332	-83.6210890337081\\
1333	-84.4673434528195\\
1334	-64.7670628694514\\
1335	-51.1215718336587\\
1336	-69.4125817493939\\
1337	-104.471022337675\\
1338	-96.4368016505243\\
1339	-91.022615944521\\
1340	-62.0418612150374\\
1341	-54.2282726110691\\
1342	-60.3308395067681\\
1343	-31.4906598855071\\
1344	-21.3381964963864\\
1345	-17.3631066851053\\
1346	-16.7842640925958\\
1347	-42.4334188615953\\
1348	-58.3647469563321\\
1349	-65.4505703715822\\
1350	-51.4676621563892\\
1351	-52.4494996251208\\
1352	-61.5069965888295\\
1353	-38.1891023000392\\
1354	-42.4019879188766\\
1355	-41.1598698760481\\
1356	-32.8767990451302\\
1357	-36.4734110065585\\
1358	-62.8281244024843\\
1359	-73.9772370645753\\
1360	-48.1272194543358\\
1361	-40.0054357503734\\
1362	-33.6054983803444\\
1363	-42.2154412837037\\
1364	-25.218828631791\\
1366	-100.237801985772\\
1367	-72.9216561215148\\
1368	-96.4440432029032\\
1369	-111.199224087713\\
1370	-113.012705007757\\
1371	-91.6760225608218\\
1372	-67.8507827878352\\
1373	-67.6076709944373\\
1374	-68.5811136665691\\
1375	-75.5899135883706\\
1376	-84.1080658379797\\
1377	-88.2887133073361\\
1378	-111.62395664271\\
1379	-123.360514678596\\
1380	-143.068182171462\\
1381	-106.049741304573\\
1382	-107.688158921769\\
1383	-121.529446164638\\
1384	-101.798126459501\\
1385	-110.587346242849\\
1386	-93.1703824717913\\
1387	-57.0050647485918\\
1388	-31.9285330457294\\
1389	-42.7314806554293\\
1390	-59.4147952325002\\
1391	-50.8099093592496\\
1392	-34.4891186248351\\
1393	-45.636130637007\\
1394	-38.0634035426253\\
1395	-25.3659259689832\\
1396	-36.9525252359974\\
1397	-38.5309418663958\\
1398	-31.8415397434806\\
1400	-68.9307862475118\\
1401	-53.1742790919288\\
1402	-84.5589504039622\\
1403	-123.948066779581\\
1404	-105.500606464663\\
1405	-124.069393242502\\
1406	-98.9575513192815\\
1407	-98.975282522186\\
1408	-73.5811804994423\\
1409	-43.6885741227534\\
1410	-61.8306912370817\\
1411	-42.158828474179\\
1412	-33.8735605622999\\
1413	-50.6573806709373\\
1414	-26.8089135375558\\
1415	-18.3891582880356\\
1416	-25.405904586177\\
1417	-52.9539212183709\\
1418	-71.2151304019255\\
1419	-75.6403610205693\\
1420	-76.5781355356371\\
1421	-73.5883471793484\\
1422	-80.0123102865875\\
1423	-65.485208862533\\
1424	-60.8945265739267\\
1425	-60.3567584058123\\
1426	-46.8807185199373\\
1427	-70.8011754882029\\
1428	-47.4239629706838\\
1429	-45.9615167897955\\
1430	-58.7397984652378\\
1431	-81.3428114339397\\
1432	-62.2014507650715\\
1433	-50.3671309824983\\
1434	-45.5057130606895\\
1435	-51.5736342149055\\
1436	-34.0409260673839\\
1437	-33.6634006217707\\
1438	-46.6088991410381\\
1439	-55.3597820580587\\
1440	-60.9295922043277\\
1441	-50.4010221604919\\
1442	-61.9735368300276\\
1443	-59.7164443280819\\
1444	-33.402233050409\\
1445	-42.7150765644537\\
1447	-98.1498875603704\\
1448	-84.9150411494263\\
1449	-78.9398873722043\\
1450	-73.5874117369169\\
1451	-61.858754835094\\
1452	-61.3571420522958\\
1453	-46.8642046970783\\
1454	-63.434070006381\\
1455	-54.2596954139626\\
1456	-36.5959110288923\\
1457	-58.1150552948659\\
1458	-61.5317596950358\\
1459	-70.4238331831648\\
1461	-78.3947331502843\\
1462	-100.029508193404\\
1463	-127.70776765095\\
1464	-136.56423643696\\
1466	-55.1907478645328\\
1467	-35.5919547077767\\
1468	-22.2279734470428\\
1469	-42.5285250085599\\
1470	-36.8062545071093\\
1471	-61.4172156966388\\
1472	-55.1673511372387\\
1473	-47.6141700350827\\
1474	-63.3166427764384\\
1475	-70.3189160192105\\
1476	-84.0059042674379\\
1477	-87.5296841410425\\
1478	-119.151823331266\\
1479	-117.778779906445\\
1480	-83.0474718524902\\
1481	-59.0910821723571\\
1482	-64.0996664374472\\
1483	-63.1040289332\\
1484	-75.102950201652\\
1485	-56.7598010923621\\
1486	-57.5100179303613\\
1487	-58.919491247543\\
1488	-26.6751705959978\\
1489	-51.9251143902495\\
1490	-79.5676572951459\\
1491	-47.7574417978612\\
1492	-63.6666792771875\\
1493	-101.708622010086\\
1494	-75.4774835751391\\
1495	-74.9158206884269\\
1496	-101.860854022534\\
1497	-97.0246255762329\\
1498	-66.9114417930389\\
1499	-42.5256464806944\\
1500	-48.0614731122134\\
1501	-69.2361408398674\\
1502	-116.356294181352\\
1503	-109.396485155378\\
1504	-110.928727364589\\
1505	-92.69931154712\\
1506	-64.6070611596722\\
1508	-42.2369650566293\\
1509	-39.6467056390604\\
1510	-32.2506146825872\\
1511	-45.5956093052057\\
1512	-51.8332616438024\\
1513	-28.511151635161\\
1514	-50.2120524930474\\
1515	-62.0496172884129\\
1516	-68.8224316938138\\
1517	-89.8682721426919\\
1518	-104.750992114102\\
1519	-130.759602047643\\
1520	-99.6901483056911\\
1521	-86.0910402209402\\
1522	-92.5599818332144\\
1523	-75.3083806955567\\
1524	-60.7667177673957\\
1525	-59.6906756303742\\
1526	-104.031847943533\\
1527	-112.682046114503\\
1528	-66.5828343140677\\
1529	-43.1374648464466\\
1531	-9.58708478978474\\
1532	-21.3328446963367\\
1533	-36.6564687341045\\
1534	-40.2439542936395\\
1535	-57.1204408437175\\
1536	-44.8610618114699\\
1537	-37.0486656925623\\
1538	-33.841701521882\\
1539	-37.1135699549727\\
1540	-30.3841790661161\\
1541	-38.0339067327495\\
1542	-52.9064307523026\\
1543	-79.1382825950795\\
1544	-82.8882194239609\\
1545	-74.4709088040506\\
1546	-87.1567041521428\\
1547	-108.511183658024\\
1548	-108.794375681202\\
1549	-124.696165733878\\
1550	-126.402872483139\\
1551	-85.6019348567443\\
1552	-67.9434942033217\\
1553	-60.8607679212066\\
1554	-105.737917449366\\
1555	-104.296600299321\\
1556	-83.0532247052233\\
1557	-51.2398046268374\\
1558	-25.524621290145\\
1559	-18.2294265207856\\
1560	-24.2183129897035\\
1561	-11.4724226325598\\
1562	-35.4928358670595\\
1563	-44.7630803587679\\
1564	-45.793143201437\\
1565	-42.1332252298055\\
1566	-24.4611577920625\\
1567	-23.6524713660606\\
1568	-30.2798473108946\\
1569	-30.8266318521069\\
1570	-34.4607543934335\\
1572	-23.093835053721\\
1573	-37.6593357796503\\
1574	-90.2934729058493\\
1575	-101.850624642375\\
1576	-99.7553547803727\\
1577	-78.2210655550919\\
1578	-93.4979572506418\\
1579	-147.242547185352\\
1580	-128.40329012497\\
1581	-127.186502706852\\
1582	-105.222365257344\\
1583	-99.4533939374999\\
1584	-103.020585507872\\
1585	-71.7963053541914\\
1586	-68.877889828103\\
1587	-37.7081528881047\\
1588	-40.3324045581014\\
1589	-73.3808908535577\\
1590	-48.5423831106993\\
1591	-57.1772622940609\\
1592	-64.2948511704103\\
1593	-56.6008896721667\\
1594	-60.0879879964193\\
1595	-57.6858878132145\\
1596	-68.4845487451471\\
1597	-72.4546789101646\\
1598	-61.9849488993848\\
1599	-54.105305270537\\
1600	-60.7297258389322\\
1601	-22.3490584253875\\
1602	-12.4386294311721\\
1603	-23.9567741836095\\
1604	-40.6215985542258\\
1605	-15.9527626718209\\
1606	-13.8818814903516\\
1607	-23.3666457762595\\
1608	-37.551447092035\\
1609	-45.2224399434608\\
1610	-55.8651434580454\\
1611	-46.8464134417104\\
1612	-69.7077819791439\\
1613	-64.2518578779295\\
1614	-41.2454118827786\\
1615	-62.528984161213\\
1616	-32.4359655689332\\
1617	-32.9933571791012\\
1618	-21.9773228688728\\
1619	-15.5717976134099\\
1620	-29.9554486492048\\
1621	-17.8188279501092\\
1622	-33.7823522659678\\
1623	-59.8847584128387\\
1624	-54.7496525026838\\
1625	-43.160730744921\\
1626	-42.7300145282356\\
1627	-37.2832019891423\\
1628	-47.0016816069356\\
1629	-33.1364370891033\\
1630	-37.9234598639741\\
1631	-31.2472607724631\\
1632	-26.8586317893532\\
1633	-32.1955433774604\\
1634	-28.8558167678193\\
1635	-48.6649153374967\\
1636	-41.3767716789453\\
1637	-35.7517605471874\\
1638	-36.7033162502025\\
1639	-28.441220422656\\
1640	-32.8062347708435\\
1641	-66.4179886200611\\
1642	-90.0023760638455\\
1643	-61.5681890655706\\
1644	-57.3077992583837\\
1645	-39.8818040780432\\
1646	-64.0420348190062\\
1647	-64.2960267205901\\
1648	-39.9231296534674\\
1649	-51.9045678160926\\
1650	-40.5302712243601\\
1651	-55.8535966440477\\
1652	-29.1018800936406\\
1653	-38.0250752885574\\
1654	-42.7483176984549\\
1655	-53.9568704192905\\
1656	-42.4260546104936\\
1657	-41.3367984263994\\
1658	-46.9765438910003\\
1659	-67.2669258672584\\
1660	-58.1725773504322\\
1661	-83.5979491295595\\
1662	-112.708471210604\\
1664	-61.6271100892764\\
1665	-42.5269470583005\\
1666	-68.7913303849964\\
1667	-83.6763536539074\\
1668	-62.4726800988155\\
1669	-75.2771294031147\\
1670	-49.3472206133954\\
1671	-26.1310987246625\\
1672	-29.3185964401696\\
1673	-63.543890255595\\
1675	-47.2524839773207\\
1676	-73.0757638678933\\
1677	-70.3148518030746\\
1678	-64.1642492749118\\
1679	-44.9144082314276\\
1680	-59.2202991022843\\
1681	-70.5124897545984\\
1682	-60.3855592914324\\
1683	-63.8914590870299\\
1684	-55.4934121957031\\
1685	-44.201500952156\\
1686	-52.5990398762549\\
1687	-34.6023563138656\\
1688	-45.121940622015\\
1689	-68.8724064980845\\
1690	-72.4443949406648\\
1691	-79.1618592222881\\
1692	-69.7855836427411\\
1693	-57.2144855719637\\
1694	-40.7433922681062\\
1695	-46.5484038622385\\
1696	-53.6693133304379\\
1697	-59.4847327909474\\
1698	-79.020253663301\\
1699	-88.7876144698082\\
1700	-70.7680492715056\\
1701	-45.1586027729165\\
1702	-78.7813186416674\\
1703	-105.487163022612\\
1704	-71.9523175769361\\
1705	-77.9904029598338\\
1706	-81.8352719891973\\
1707	-73.62601458872\\
1708	-55.550822924409\\
1709	-65.2313518107435\\
1710	-60.0637437033674\\
1712	-74.3868794313166\\
1713	-67.5762224133571\\
1714	-68.6679539767345\\
1715	-68.8999878510742\\
1716	-72.4038742196092\\
1717	-54.9163619722542\\
1718	-71.0075269917666\\
1719	-51.8250869092114\\
1720	-54.2375966475236\\
1721	-64.550014103721\\
1722	-56.800550401756\\
1723	-31.5732746259366\\
1724	-18.1089066926327\\
1725	-14.1926963502558\\
1726	-26.3642264095076\\
1727	-68.6816716031037\\
1728	-83.1854190170325\\
1729	-77.7874989480142\\
1730	-56.3809351717218\\
1731	-62.312076520247\\
1732	-60.0407545525677\\
1733	-51.0624309319178\\
1734	-32.4462338857784\\
1735	-45.5624938184073\\
1736	-47.6483624487132\\
1737	-41.5536169521722\\
1738	-49.2793205174135\\
1739	-41.540227450874\\
1740	-43.3119335624949\\
1741	-26.534126241402\\
1742	-38.9695825723734\\
1743	-66.0266476867446\\
1744	-47.0371741585238\\
1745	-58.940927384017\\
1746	-54.5013275048711\\
1747	-68.4041350476489\\
1748	-66.1128433755739\\
1749	-83.6528580241261\\
1750	-66.6713195092495\\
1751	-72.8085691969902\\
1752	-74.4644742761436\\
1753	-40.7929620547363\\
1754	-75.208954460868\\
1755	-38.8490426754015\\
1756	-61.1158143252235\\
1757	-57.3093841148602\\
1758	-79.3738089839997\\
1759	-60.49482992346\\
1760	-75.013148215148\\
1761	-92.3373494764771\\
1762	-77.8859332746788\\
1763	-70.77119993256\\
1764	-60.0913209900184\\
1765	-66.6826790555181\\
1766	-58.2789814033515\\
1767	-51.4972271978042\\
1768	-47.0011589867402\\
1769	-46.5789958408018\\
1770	-47.6686757410823\\
1771	-80.6539042112538\\
1772	-101.400106164962\\
1773	-88.2005150744524\\
1774	-91.8746032787092\\
1775	-84.1806340991229\\
1776	-59.4898112879116\\
1777	-47.4062247507859\\
1778	-44.1245608102111\\
1779	-32.6714661266394\\
1780	-40.5715751181187\\
1782	-70.6168668944827\\
1783	-78.6028152263254\\
1784	-71.3943813619389\\
1785	-49.4703668405057\\
1786	-82.4100257781897\\
1787	-100.14382449023\\
1788	-96.2018538807063\\
1789	-86.5980134209233\\
1790	-129.391606303591\\
1791	-106.884201275107\\
1792	-61.0165024250202\\
1793	-43.4518724433453\\
1794	-37.1644814614701\\
1795	-67.2374748146965\\
1796	-77.6022842927121\\
1797	-98.9554803530682\\
1798	-78.48670952467\\
1799	-65.6652213965067\\
1800	-38.1812331318513\\
1801	-37.779983090883\\
1802	-46.5213961698248\\
1803	-42.5221749576385\\
1804	-33.4584395914771\\
1805	-37.5433385370898\\
};
\addlegendentry{OSA predition}

\addplot [color=mycolor3, dotted, line width=2.0pt]
  table[row sep=crcr]{%
1006	-86.6700000000001\\
1007	-103.76\\
1008	-79.346\\
1009	-45.1659999999999\\
1010	-54.985306145418\\
1011	-55.4723037098422\\
1012	-65.9310471815597\\
1013	-59.3187067273664\\
1014	-19.7738936413107\\
1015	-13.8877055822079\\
1016	-13.1087466889533\\
1017	-51.5662156054195\\
1018	-81.6744259845425\\
1019	-83.9720117462523\\
1020	-83.4706478623773\\
1021	-66.5339544065932\\
1022	-42.0103416319337\\
1023	-84.5950035389658\\
1024	-60.0473926037491\\
1025	-48.5097871076734\\
1026	-74.769172394585\\
1027	-66.1127204620491\\
1028	-58.5496360324389\\
1029	-90.3343503207998\\
1030	-84.0900200048302\\
1031	-64.5002863120983\\
1032	-87.5956773260118\\
1033	-89.5381263265219\\
1034	-69.3507643315261\\
1035	-65.2252723651466\\
1036	-49.8464071027258\\
1037	-61.236677837717\\
1038	-67.8776848261393\\
1039	-67.377165825668\\
1040	-66.7593325070368\\
1041	-69.5462635629835\\
1042	-76.3404780665473\\
1043	-96.3424663503281\\
1044	-91.770255507389\\
1045	-61.5882149064362\\
1046	-65.6864851657565\\
1047	-34.3884232570856\\
1048	-39.5999797699262\\
1049	-51.180545490115\\
1050	-41.34285884271\\
1051	-46.7495970736563\\
1053	-67.4725081611748\\
1054	-61.7582449910374\\
1055	-89.7924437617896\\
1056	-72.7552381826167\\
1057	-48.0977806358167\\
1058	-40.9687624970613\\
1059	-49.1465394308495\\
1060	-56.6612523221925\\
1061	-32.530655877332\\
1062	-36.7963258914949\\
1063	-46.9262894870365\\
1064	-39.6454104015229\\
1065	-34.385889876736\\
1066	-45.3916511089792\\
1067	-49.4370469972262\\
1068	-72.6499522774479\\
1069	-70.5557267964639\\
1070	-78.9899701728189\\
1071	-75.466765516227\\
1072	-80.1990110713855\\
1073	-67.8979131570218\\
1074	-66.7407105619452\\
1075	-61.5961107806356\\
1076	-61.5447659042454\\
1077	-61.6301229961382\\
1078	-94.043087163873\\
1079	-130.80636604649\\
1080	-130.764437774528\\
1081	-126.961217008852\\
1082	-87.3957500289514\\
1083	-115.840064908424\\
1084	-141.875757441045\\
1085	-138.581759562654\\
1086	-110.074167394769\\
1088	-175.948257884421\\
1089	-143.320387469622\\
1090	-116.7923881165\\
1091	-83.5557815674599\\
1092	-64.8326372029721\\
1093	-60.3605010243391\\
1094	-70.3744223486322\\
1095	-50.0526651112996\\
1096	-31.426012392262\\
1097	-32.0106096393488\\
1098	-39.9587535867402\\
1099	-63.3652548251655\\
1100	-56.1264291305865\\
1101	-61.638743039385\\
1102	-73.143585677486\\
1103	-76.5641069888304\\
1104	-100.816869974129\\
1105	-94.2509457300728\\
1106	-100.378598183261\\
1108	-69.9041733151878\\
1109	-85.7405516282358\\
1110	-62.3112049310462\\
1111	-59.7677811451049\\
1112	-70.2590991829913\\
1113	-69.9210202134825\\
1114	-49.6874037944922\\
1115	-56.9855766121643\\
1116	-44.1333013486833\\
1117	-48.0128805228405\\
1118	-50.4744830077439\\
1119	-38.1181543571267\\
1120	-45.37129971385\\
1121	-55.042120594477\\
1122	-81.3590085983312\\
1123	-81.9718540953622\\
1124	-84.0045121523926\\
1125	-58.6267235730718\\
1126	-74.3422014252969\\
1127	-113.535135390311\\
1128	-87.0970879454328\\
1129	-99.3719372888572\\
1130	-95.2862652601273\\
1131	-63.1132814285843\\
1132	-45.7820630384904\\
1133	-63.1610100770602\\
1134	-69.1158075563396\\
1135	-100.981822161234\\
1136	-121.823412678296\\
1137	-114.872659564717\\
1138	-98.7962739674401\\
1140	-93.8308617074235\\
1141	-88.2359615631694\\
1142	-86.7250156417385\\
1143	-60.5044326163782\\
1144	-59.7817465430428\\
1145	-52.657185913675\\
1146	-49.3000859294127\\
1147	-55.8218940477066\\
1148	-70.3135785815175\\
1149	-102.572928567921\\
1151	-62.4285042867614\\
1152	-55.2663648508951\\
1153	-56.3655097402559\\
1154	-34.0196730743482\\
1155	-24.8717169608205\\
1156	-29.5299517667986\\
1157	-52.2208847681216\\
1158	-38.4334850754951\\
1159	-44.9621123215077\\
1160	-45.0367362064642\\
1161	-46.0437457132775\\
1162	-38.2321930546923\\
1163	-27.1949555259848\\
1164	-21.3816881530242\\
1165	-33.992591238637\\
1166	-68.9850611459115\\
1167	-95.9339527279108\\
1168	-99.0616755503197\\
1169	-71.1324213373734\\
1170	-54.1213879372251\\
1171	-41.7828153035891\\
1172	-31.0935095037134\\
1173	-57.1698083024146\\
1174	-54.1656684250054\\
1175	-70.643477510285\\
1176	-88.5853873771234\\
1177	-138.698846209218\\
1178	-160.488635490564\\
1179	-133.096066769908\\
1180	-132.109904260244\\
1181	-87.7893875505024\\
1182	-98.3691776731296\\
1183	-101.070830728057\\
1184	-106.039147700348\\
1185	-88.0503745206363\\
1186	-78.6558955468795\\
1187	-80.2586362752575\\
1188	-75.2708172297037\\
1189	-84.1402934669029\\
1190	-65.8539988132736\\
1191	-98.3131766912086\\
1192	-115.575725632599\\
1193	-88.1110774378592\\
1194	-57.5346137323133\\
1195	-63.5752676997824\\
1196	-98.2814042952155\\
1197	-113.694766228207\\
1198	-135.374094324226\\
1199	-137.451182614884\\
1200	-140.177774911753\\
1201	-115.281234828344\\
1202	-100.89440719741\\
1203	-118.257723099809\\
1204	-127.101323651908\\
1205	-81.4910947915564\\
1206	-55.876248134814\\
1207	-76.2966226463811\\
1208	-70.0514504915018\\
1209	-45.3797812517676\\
1210	-60.6975790583938\\
1211	-65.7406461328424\\
1212	-57.9038943776638\\
1213	-45.0313702270259\\
1214	-60.5532670097436\\
1215	-51.5068133503523\\
1216	-61.8412602555757\\
1217	-85.6862631669321\\
1218	-77.4574713343768\\
1219	-59.8895033080298\\
1220	-62.0572778304336\\
1221	-85.0262930307451\\
1222	-134.156349623851\\
1223	-104.202364178877\\
1224	-66.5684275737831\\
1225	-50.5825632345438\\
1226	-61.8387302591666\\
1227	-55.8178277303155\\
1228	-45.6289743675338\\
1229	-49.314824950279\\
1230	-57.1790948817659\\
1231	-67.6409668367048\\
1232	-73.2988007982708\\
1233	-54.1720045384443\\
1234	-86.1832443604396\\
1235	-96.4576230296382\\
1236	-88.9771433841231\\
1237	-76.6247386688999\\
1238	-78.0944092431635\\
1239	-102.562913232764\\
1240	-72.0294098973573\\
1241	-31.6624455195104\\
1242	-44.2527159365388\\
1243	-50.3534872155715\\
1244	-69.3645005778681\\
1246	-45.8912592037882\\
1247	-66.4582676993166\\
1248	-63.5170208974412\\
1249	-47.7718998291562\\
1250	-61.6969541264573\\
1251	-54.161425660446\\
1252	-53.436951813937\\
1253	-56.4003083799696\\
1254	-38.3380583647534\\
1255	-43.6767122726512\\
1256	-33.4701048408112\\
1257	-38.3933338589893\\
1258	-62.9851929051849\\
1259	-68.4085355911336\\
1260	-84.8642498175234\\
1261	-111.227336132376\\
1262	-79.6310337457774\\
1263	-51.9830211762132\\
1264	-40.5604406354082\\
1265	-40.9126361393032\\
1266	-56.5892676700537\\
1267	-37.1486766411235\\
1268	-42.4972225687657\\
1269	-51.9420191539818\\
1270	-77.1783660964584\\
1271	-71.9939447663487\\
1272	-93.3706770989234\\
1273	-66.943525846893\\
1274	-63.4372680949734\\
1275	-34.7328394243243\\
1276	-34.9936372545319\\
1277	-24.4112092077876\\
1278	-26.5867304760839\\
1279	-32.78150477482\\
1280	-51.8033665858763\\
1281	-58.2235623813674\\
1282	-60.5874300074245\\
1283	-67.1659086716154\\
1284	-100.608409532237\\
1285	-94.3889385319301\\
1286	-71.6837305524666\\
1287	-83.6948096800268\\
1288	-83.5261969441603\\
1289	-103.329504719683\\
1290	-88.1372463898022\\
1291	-60.2978196609763\\
1292	-30.0125371371\\
1293	-23.9409691940034\\
1294	-26.0172663665267\\
1295	-30.5073992753998\\
1296	-32.0522885170083\\
1297	-31.129193346844\\
1298	-40.2380567350287\\
1299	-61.5111098338548\\
1300	-55.2615649072081\\
1301	-56.0810117434942\\
1302	-63.7833691100525\\
1303	-43.8957984013146\\
1304	-27.4065630114058\\
1306	-71.1007359727237\\
1307	-66.3193294250318\\
1308	-72.6268504290242\\
1309	-65.0373134302608\\
1310	-50.6344545913696\\
1311	-68.1580957413605\\
1312	-89.7717373592241\\
1313	-86.1401210754216\\
1314	-64.2881608211962\\
1315	-106.407622851297\\
1316	-83.5936907680468\\
1317	-81.2192891395496\\
1318	-88.963340596227\\
1319	-88.0343782629429\\
1320	-69.217301618032\\
1321	-65.4174132502703\\
1322	-94.0494199100574\\
1323	-128.333605357335\\
1324	-98.068554819577\\
1325	-62.6045427873971\\
1326	-60.3372052119958\\
1327	-62.7785396632762\\
1328	-78.1263045455735\\
1329	-85.8910373068447\\
1330	-63.6180525850468\\
1331	-75.4487661975891\\
1332	-89.3840697433197\\
1333	-89.1168330046203\\
1334	-66.2114568686786\\
1335	-54.3942533446941\\
1336	-72.4070646367832\\
1337	-109.226157500349\\
1338	-100.362630743748\\
1339	-96.1120210794372\\
1340	-64.4922060657711\\
1341	-57.4074482378701\\
1342	-63.4080927417215\\
1343	-37.1414365372866\\
1344	-24.6552830147446\\
1345	-18.5800672780417\\
1346	-17.2668771910292\\
1347	-42.6618153690556\\
1348	-59.2344816060477\\
1349	-67.2806704969435\\
1350	-54.2592652484464\\
1351	-56.3260130556371\\
1352	-64.7132801560344\\
1353	-42.4689661908569\\
1354	-46.0300939106401\\
1355	-45.3219822302419\\
1356	-37.1944953222596\\
1357	-41.8004531256911\\
1358	-66.3511778645263\\
1359	-79.7724692167892\\
1360	-52.9695114565639\\
1361	-43.1528634705078\\
1362	-38.3307894088127\\
1363	-46.7081535545603\\
1364	-30.8187455787399\\
1365	-66.6558730572131\\
1366	-107.803486429204\\
1367	-81.4466391315134\\
1368	-103.981995306269\\
1369	-117.873859702508\\
1370	-118.138200372721\\
1372	-73.0713541315299\\
1373	-73.4005699778968\\
1374	-74.808818288577\\
1375	-83.2833341592404\\
1376	-90.6267120035841\\
1377	-92.0388911303769\\
1378	-116.108880881911\\
1379	-125.063262608116\\
1380	-143.762375642831\\
1381	-103.890565680041\\
1382	-109.826337311036\\
1383	-121.897156229133\\
1384	-101.692187352297\\
1385	-114.938563485686\\
1386	-97.4453883868875\\
1387	-60.1842590787562\\
1388	-35.504147132365\\
1389	-42.7641955009581\\
1390	-60.7612710500318\\
1391	-52.7450436596084\\
1392	-38.7828311479423\\
1393	-49.7109269424016\\
1394	-41.4739594156722\\
1395	-30.0814400777692\\
1396	-39.9619039771012\\
1397	-42.1829132350456\\
1398	-35.1958749861692\\
1399	-56.2105749979953\\
1400	-74.159161218987\\
1401	-59.2871969210025\\
1402	-92.8851541101687\\
1403	-132.370337223955\\
1404	-116.418659475398\\
1405	-133.215430845209\\
1406	-102.854701929626\\
1407	-105.557664071072\\
1408	-79.1917962875114\\
1409	-48.7736565236876\\
1410	-66.419940553061\\
1411	-49.0891119795729\\
1412	-39.1707608246436\\
1413	-54.7758199819746\\
1414	-32.8220904405898\\
1415	-22.6829687754584\\
1416	-27.6384995031078\\
1417	-55.4874825450438\\
1418	-74.519140112249\\
1419	-82.2533473926021\\
1420	-80.7131490005265\\
1421	-78.471851830277\\
1422	-86.1490987058967\\
1423	-69.7366877751017\\
1424	-64.8223966081075\\
1425	-66.7692725922582\\
1426	-53.9233065337264\\
1427	-77.8923579596221\\
1428	-53.888921296247\\
1429	-50.2717845915713\\
1430	-64.8981829094412\\
1431	-86.4020372449011\\
1432	-66.6727507336402\\
1433	-55.2691020731688\\
1434	-50.8741856276238\\
1435	-57.496552755875\\
1436	-40.284454234132\\
1437	-39.1252018601274\\
1438	-52.1105178740722\\
1439	-61.2330111180199\\
1440	-66.4759432444305\\
1441	-55.4058630486686\\
1442	-66.9302030441313\\
1443	-64.2013216649061\\
1444	-37.8070716717384\\
1445	-45.9465895301032\\
1447	-103.914598076152\\
1448	-92.4978736138648\\
1449	-83.0314166481749\\
1450	-78.6210187021779\\
1451	-66.3353032475977\\
1452	-66.3536449258174\\
1453	-54.144740279942\\
1454	-70.4281012244148\\
1455	-61.1595663649441\\
1456	-41.9799607635712\\
1457	-62.7666293185409\\
1458	-67.8525176288906\\
1459	-75.646259717566\\
1460	-78.8491737279012\\
1461	-81.4845535963891\\
1462	-103.41048644405\\
1463	-129.410586941684\\
1464	-139.585530180318\\
1465	-95.7848017473557\\
1466	-56.9752475428936\\
1467	-38.5039881018586\\
1468	-24.3870518219119\\
1469	-43.1366201360968\\
1470	-39.9963367280955\\
1471	-65.3365067082725\\
1472	-59.2889564524246\\
1473	-51.8972239712525\\
1474	-67.5453627799302\\
1475	-74.3632997984782\\
1476	-86.8118242415453\\
1477	-91.032675881693\\
1478	-122.42709160428\\
1479	-119.381038475062\\
1480	-91.3607904009773\\
1481	-65.8522143395287\\
1482	-69.0711977917472\\
1483	-70.6510390054398\\
1484	-82.4732362404638\\
1485	-62.3441783350138\\
1486	-63.1459017331035\\
1487	-64.2303089021364\\
1488	-32.6644082678747\\
1489	-55.2649315149768\\
1490	-81.7669850592347\\
1491	-53.4925413167316\\
1492	-66.642224743159\\
1493	-108.916300709542\\
1494	-82.1268548573696\\
1495	-80.2301638274407\\
1496	-108.297272312422\\
1497	-101.988526375593\\
1498	-70.1008103670156\\
1499	-47.0427525156122\\
1500	-52.5072768218622\\
1501	-75.0314401131695\\
1502	-121.152928608619\\
1503	-117.76611508984\\
1504	-116.438105197263\\
1505	-95.1720540569452\\
1506	-68.1923043096501\\
1507	-58.1097049695197\\
1508	-45.766962825704\\
1509	-43.8279169651503\\
1510	-37.544601377024\\
1511	-49.9967305625116\\
1512	-55.8779968972867\\
1513	-32.5752066974326\\
1514	-52.6839940558152\\
1515	-67.4044289315864\\
1516	-73.4137242489905\\
1517	-94.2732820974879\\
1518	-111.374414776181\\
1519	-135.072431140591\\
1520	-104.624080540579\\
1521	-92.13099233252\\
1522	-98.9513820229263\\
1524	-68.6767421727498\\
1525	-69.5643526668468\\
1526	-111.310289509171\\
1527	-120.456501580856\\
1528	-71.6104170200899\\
1529	-45.2748554626644\\
1530	-29.8474532127998\\
1531	-10.8027178403852\\
1532	-18.9807634122878\\
1533	-35.2029980963546\\
1534	-40.3012973872274\\
1535	-57.876466567848\\
1536	-47.3923691491671\\
1537	-39.5260793117127\\
1538	-38.0409962836375\\
1539	-40.3236642768111\\
1540	-35.0775025838529\\
1541	-41.9120208802665\\
1542	-57.8093867790183\\
1543	-84.181748252812\\
1544	-88.6709269130804\\
1545	-81.8079061377962\\
1546	-92.6768371976823\\
1547	-114.055224971812\\
1548	-114.195667962896\\
1549	-129.741510555362\\
1550	-129.338164529677\\
1551	-92.4121492406857\\
1552	-72.8184964711436\\
1553	-66.9979093374659\\
1554	-112.668314485467\\
1555	-113.080477601831\\
1556	-87.622378945324\\
1557	-56.1785589349904\\
1558	-28.4536248446766\\
1559	-17.9230182829692\\
1560	-23.7628742660231\\
1561	-10.9940595499966\\
1562	-32.1641364458549\\
1563	-44.7710937962022\\
1564	-45.5843785709801\\
1565	-42.9255791557462\\
1566	-26.68087295333\\
1567	-24.1647183830194\\
1568	-32.1015270140006\\
1569	-32.7258628832139\\
1570	-36.4146976754264\\
1572	-27.0395055346187\\
1573	-41.2341807579196\\
1574	-95.4189217817723\\
1575	-111.387619837437\\
1576	-111.554138934663\\
1577	-87.7624703280524\\
1578	-105.551397559072\\
1579	-156.389187273586\\
1580	-139.551786482746\\
1581	-136.626707214005\\
1582	-112.26038913929\\
1583	-108.948233743207\\
1584	-112.430381662526\\
1585	-79.2492537826117\\
1586	-76.7909607171259\\
1587	-45.5680742194259\\
1588	-44.2398588319691\\
1589	-76.9426687972882\\
1590	-53.2764078839232\\
1591	-58.3978152732261\\
1592	-68.0261476785183\\
1593	-62.5318339155244\\
1594	-65.0609548504378\\
1595	-64.2874651399579\\
1596	-73.3048633003666\\
1597	-77.7566237377466\\
1598	-66.6005274301399\\
1599	-57.9821158249711\\
1600	-67.8080021058254\\
1601	-27.8548784596046\\
1602	-13.9679155099241\\
1603	-23.5560616021421\\
1604	-41.1037945852154\\
1605	-17.2276952482553\\
1606	-11.3767300935301\\
1607	-22.9840091819156\\
1608	-37.8348022894991\\
1609	-45.2836997913666\\
1610	-57.2062166710912\\
1611	-49.7698041421672\\
1612	-73.1867326223828\\
1613	-67.7559216757293\\
1614	-46.9967037396386\\
1615	-67.4967564528602\\
1616	-37.7167168007768\\
1617	-33.699874725155\\
1618	-24.0312786311852\\
1619	-17.3212063008559\\
1620	-30.4859207431371\\
1621	-20.4285945755098\\
1622	-34.8656622105495\\
1623	-60.5128164166833\\
1624	-57.5605268509451\\
1625	-45.3870703161315\\
1626	-47.0627604891108\\
1627	-40.3168864369072\\
1628	-50.8956632955833\\
1629	-37.7171600968475\\
1630	-39.8300296043881\\
1631	-35.6188493033596\\
1632	-30.0971005567244\\
1633	-35.0073522071114\\
1634	-31.6934910310435\\
1635	-51.3218311486296\\
1636	-45.5201104284697\\
1637	-37.78095311177\\
1638	-39.1267141461017\\
1639	-32.1669242826549\\
1640	-36.0305584097855\\
1641	-70.377957950421\\
1642	-94.4052827528922\\
1643	-67.5526668905313\\
1644	-64.1776763879466\\
1645	-45.4482325434026\\
1646	-70.018366468415\\
1647	-68.7190442453334\\
1648	-44.266012582646\\
1649	-55.8276663548331\\
1650	-44.7245192750941\\
1651	-59.0625270406063\\
1652	-33.4607587950102\\
1653	-38.4865348084872\\
1654	-45.4109756472178\\
1655	-55.6798886631252\\
1656	-44.0062457872259\\
1657	-44.8683260634541\\
1658	-48.942813527924\\
1659	-70.0341423383964\\
1660	-61.6100948676717\\
1661	-86.1517047978932\\
1662	-116.042848935147\\
1663	-91.4990432533609\\
1664	-64.7859494595457\\
1665	-46.6805313730426\\
1666	-71.6465260579523\\
1667	-87.1037871310982\\
1668	-66.6698642522836\\
1669	-78.089127711585\\
1670	-51.4707520950237\\
1671	-27.1481842137225\\
1672	-30.2852886015123\\
1673	-65.3862077813703\\
1674	-59.1602744254837\\
1675	-50.8606127828932\\
1676	-75.7539525372974\\
1677	-72.4038226483954\\
1678	-66.6926135178562\\
1679	-47.9882932750729\\
1681	-74.4461583142124\\
1682	-62.4633619275796\\
1683	-65.9530889295513\\
1684	-59.0527623655662\\
1685	-46.3842381797187\\
1686	-55.0879574299429\\
1687	-39.3274276461214\\
1688	-47.784463571875\\
1689	-73.5459639523267\\
1691	-80.6888445918378\\
1692	-71.0112066684801\\
1693	-58.3873440648804\\
1694	-43.2407054747023\\
1695	-49.5694273208458\\
1696	-57.6851432404101\\
1697	-64.0111502053389\\
1698	-81.356472876347\\
1699	-92.1645312957733\\
1700	-72.7419118447058\\
1701	-45.8890094920735\\
1702	-80.7518827157817\\
1703	-109.230375178908\\
1704	-74.4992872032108\\
1705	-78.8582998563297\\
1706	-85.4528262877513\\
1707	-74.2168745731806\\
1708	-59.0363825476009\\
1709	-68.5691508460689\\
1710	-63.0787785522041\\
1711	-71.4106439696848\\
1712	-78.7396984983875\\
1713	-69.1406909855007\\
1714	-72.5250055980098\\
1715	-71.0925255511424\\
1716	-74.7558051596227\\
1717	-59.3124622103664\\
1718	-74.2569038113099\\
1719	-55.107043597704\\
1720	-56.9092108880382\\
1721	-67.4783382536857\\
1722	-61.1378316373043\\
1723	-34.4526230575277\\
1724	-20.0358572721543\\
1725	-14.898932381974\\
1726	-26.324637628755\\
1727	-69.0255257965375\\
1728	-86.7882182544108\\
1729	-82.3504315072792\\
1730	-60.9642769082348\\
1731	-68.1046857927108\\
1732	-64.7694687824821\\
1733	-55.9663279879626\\
1734	-37.7737936017622\\
1735	-49.0021276276175\\
1736	-51.9287103513761\\
1737	-46.6270297768974\\
1738	-53.1643513480876\\
1739	-46.0418831160716\\
1740	-47.5909150531779\\
1741	-31.9451423746389\\
1742	-42.1876459236621\\
1743	-69.2287110070324\\
1744	-50.9432112395793\\
1745	-61.5380925818513\\
1746	-58.3505219716276\\
1747	-71.4831693337237\\
1748	-69.2454636970954\\
1749	-87.083037558961\\
1750	-68.5974874155736\\
1751	-74.4214550246825\\
1752	-76.208884415767\\
1753	-42.5132859045818\\
1754	-76.3417379279233\\
1755	-44.093885343894\\
1756	-63.4588139517371\\
1757	-60.4264475095256\\
1758	-80.0104034685917\\
1759	-61.9287199867113\\
1760	-76.4437454778458\\
1761	-92.9582771853072\\
1762	-78.061405101261\\
1764	-62.2694435212425\\
1765	-69.72134748336\\
1766	-61.8488000913669\\
1767	-55.530571655634\\
1768	-51.8884703964666\\
1769	-52.7546697199205\\
1770	-51.6080585359257\\
1771	-86.293816559564\\
1772	-106.775004479619\\
1773	-90.131216957164\\
1774	-92.6910811200519\\
1775	-86.5887003141252\\
1776	-59.9582839372908\\
1777	-51.4440861594674\\
1778	-47.946533098778\\
1779	-37.6960355974174\\
1780	-44.9861506495411\\
1781	-62.0559884141985\\
1782	-75.2232115251009\\
1783	-81.6372860210072\\
1784	-73.478100324133\\
1785	-52.2584163048421\\
1786	-85.2550381127135\\
1787	-101.872775170537\\
1788	-100.663149569231\\
1789	-86.8837106128519\\
1790	-131.236588585345\\
1791	-106.443279830367\\
1792	-62.5295332264452\\
1793	-46.9982025215008\\
1794	-39.4562009121196\\
1795	-69.9195556505513\\
1796	-82.6995456407751\\
1797	-102.242130392465\\
1798	-80.9989376870562\\
1799	-67.7096600581906\\
1800	-40.4824442246349\\
1801	-40.6402936565776\\
1802	-49.1661651686659\\
1803	-47.8835422670102\\
1804	-36.986808076362\\
1805	-41.6190395187846\\
};
\addlegendentry{MPO prediction}

\end{axis}

\begin{axis}[%
width=6.159cm,
height=1.831cm,
at={(8.104cm,0cm)},
scale only axis,
xmin=1000,
xmax=2000,
xlabel style={font=\color{white!15!black}},
xlabel={Sample index},
ymin=-158.691,
ymax=-9.766,
ylabel style={font=\color{white!15!black}},
ylabel={$y(t)$},
axis background/.style={fill=white},
title style={font=\bfseries},
title={C10: RMSE(OSA) = 3.4069, RMSE(MPO) = 5.9801},
legend style={legend cell align=left, align=left, draw=white!15!black}
]
\addplot [color=mycolor1, line width=2.0pt]
  table[row sep=crcr]{%
1006	-76.904\\
1007	-90.3320000000001\\
1008	-68.3589999999999\\
1009	-40.2829999999999\\
1010	-48.828\\
1011	-46.3869999999999\\
1012	-54.932\\
1013	-45.1659999999999\\
1014	-21.973\\
1015	-17.0899999999999\\
1016	-18.3109999999999\\
1018	-62.2560000000001\\
1019	-67.1389999999999\\
1020	-69.5799999999999\\
1021	-52.49\\
1022	-34.1800000000001\\
1023	-64.6970000000001\\
1024	-52.49\\
1025	-39.0630000000001\\
1026	-61.0350000000001\\
1027	-53.711\\
1028	-45.1659999999999\\
1029	-73.242\\
1030	-67.1389999999999\\
1031	-52.49\\
1032	-75.684\\
1033	-74.463\\
1034	-58.5940000000001\\
1036	-41.5039999999999\\
1037	-47.607\\
1038	-58.5940000000001\\
1039	-54.932\\
1040	-59.8140000000001\\
1041	-59.8140000000001\\
1042	-65.9180000000001\\
1043	-85.4490000000001\\
1044	-76.904\\
1045	-54.932\\
1046	-50.049\\
1047	-35.4000000000001\\
1048	-34.1800000000001\\
1049	-45.1659999999999\\
1050	-34.1800000000001\\
1051	-36.6210000000001\\
1052	-51.27\\
1053	-54.932\\
1054	-51.27\\
1055	-78.125\\
1056	-62.2560000000001\\
1057	-36.6210000000001\\
1058	-30.518\\
1059	-40.2829999999999\\
1060	-46.3869999999999\\
1061	-29.297\\
1062	-29.297\\
1063	-36.6210000000001\\
1065	-26.855\\
1066	-35.4000000000001\\
1067	-40.2829999999999\\
1068	-58.5940000000001\\
1069	-56.152\\
1070	-65.9180000000001\\
1071	-61.0350000000001\\
1072	-70.8009999999999\\
1073	-57.373\\
1074	-58.5940000000001\\
1075	-51.27\\
1077	-48.828\\
1079	-111.084\\
1080	-114.746\\
1081	-112.305\\
1082	-74.463\\
1084	-125.732\\
1085	-128.174\\
1086	-96.4359999999999\\
1087	-124.512\\
1088	-158.691\\
1089	-115.967\\
1090	-97.6559999999999\\
1091	-68.3589999999999\\
1092	-53.711\\
1093	-47.607\\
1094	-56.152\\
1095	-45.1659999999999\\
1096	-30.518\\
1097	-30.518\\
1098	-36.6210000000001\\
1099	-50.049\\
1101	-47.607\\
1102	-62.2560000000001\\
1103	-64.6970000000001\\
1104	-85.4490000000001\\
1105	-72.021\\
1106	-85.4490000000001\\
1107	-63.4770000000001\\
1108	-54.932\\
1109	-64.6970000000001\\
1110	-48.828\\
1111	-43.9449999999999\\
1112	-53.711\\
1113	-54.932\\
1114	-40.2829999999999\\
1115	-43.9449999999999\\
1116	-32.9590000000001\\
1118	-40.2829999999999\\
1119	-30.518\\
1120	-34.1800000000001\\
1121	-43.9449999999999\\
1122	-63.4770000000001\\
1123	-65.9180000000001\\
1124	-72.021\\
1125	-45.1659999999999\\
1126	-56.152\\
1127	-89.1110000000001\\
1128	-70.8009999999999\\
1129	-81.787\\
1130	-80.566\\
1131	-50.049\\
1132	-36.6210000000001\\
1133	-47.607\\
1134	-52.49\\
1135	-81.787\\
1136	-103.76\\
1137	-102.539\\
1138	-76.904\\
1139	-80.566\\
1140	-72.021\\
1141	-69.5799999999999\\
1142	-68.3589999999999\\
1143	-51.27\\
1145	-39.0630000000001\\
1146	-37.8420000000001\\
1147	-41.5039999999999\\
1148	-56.152\\
1149	-84.229\\
1150	-70.8009999999999\\
1151	-50.049\\
1152	-42.7249999999999\\
1153	-40.2829999999999\\
1154	-28.076\\
1155	-23.193\\
1156	-24.414\\
1157	-41.5039999999999\\
1158	-32.9590000000001\\
1159	-34.1800000000001\\
1160	-37.8420000000001\\
1161	-35.4000000000001\\
1162	-26.855\\
1163	-21.973\\
1164	-18.3109999999999\\
1165	-25.635\\
1166	-51.27\\
1167	-73.242\\
1168	-78.125\\
1169	-54.932\\
1170	-39.0630000000001\\
1172	-24.414\\
1173	-41.5039999999999\\
1174	-42.7249999999999\\
1175	-54.932\\
1176	-73.242\\
1177	-108.643\\
1178	-125.732\\
1179	-103.76\\
1180	-101.318\\
1181	-67.1389999999999\\
1182	-75.684\\
1183	-79.346\\
1184	-86.6700000000001\\
1185	-69.5799999999999\\
1186	-62.2560000000001\\
1187	-61.0350000000001\\
1188	-56.152\\
1189	-67.1389999999999\\
1190	-52.49\\
1191	-76.904\\
1192	-97.6559999999999\\
1194	-45.1659999999999\\
1195	-47.607\\
1196	-78.125\\
1197	-93.9939999999999\\
1198	-112.305\\
1199	-119.629\\
1200	-119.629\\
1201	-91.5530000000001\\
1202	-84.229\\
1203	-96.4359999999999\\
1204	-111.084\\
1205	-74.463\\
1206	-46.3869999999999\\
1207	-62.2560000000001\\
1208	-54.932\\
1209	-34.1800000000001\\
1210	-47.607\\
1211	-53.711\\
1212	-42.7249999999999\\
1213	-34.1800000000001\\
1214	-43.9449999999999\\
1215	-40.2829999999999\\
1216	-52.49\\
1217	-65.9180000000001\\
1218	-62.2560000000001\\
1219	-45.1659999999999\\
1220	-45.1659999999999\\
1221	-67.1389999999999\\
1222	-104.98\\
1223	-79.346\\
1224	-51.27\\
1225	-40.2829999999999\\
1226	-47.607\\
1227	-40.2829999999999\\
1228	-31.7380000000001\\
1229	-35.4000000000001\\
1230	-42.7249999999999\\
1231	-51.27\\
1232	-57.373\\
1233	-42.7249999999999\\
1234	-63.4770000000001\\
1235	-75.684\\
1236	-70.8009999999999\\
1237	-57.373\\
1238	-62.2560000000001\\
1239	-81.787\\
1240	-52.49\\
1241	-28.076\\
1242	-36.6210000000001\\
1243	-42.7249999999999\\
1244	-51.27\\
1245	-46.3869999999999\\
1246	-36.6210000000001\\
1247	-52.49\\
1248	-52.49\\
1249	-37.8420000000001\\
1250	-47.607\\
1251	-41.5039999999999\\
1252	-37.8420000000001\\
1253	-45.1659999999999\\
1254	-29.297\\
1255	-31.7380000000001\\
1257	-26.855\\
1258	-47.607\\
1259	-53.711\\
1260	-69.5799999999999\\
1261	-89.1110000000001\\
1263	-36.6210000000001\\
1264	-31.7380000000001\\
1265	-29.297\\
1266	-40.2829999999999\\
1267	-31.7380000000001\\
1268	-30.518\\
1269	-39.0630000000001\\
1270	-61.0350000000001\\
1271	-53.711\\
1272	-79.346\\
1273	-54.932\\
1274	-48.828\\
1275	-32.9590000000001\\
1276	-29.297\\
1277	-23.193\\
1278	-23.193\\
1279	-25.635\\
1280	-40.2829999999999\\
1281	-46.3869999999999\\
1283	-51.27\\
1284	-79.346\\
1285	-73.242\\
1286	-53.711\\
1287	-63.4770000000001\\
1288	-69.5799999999999\\
1289	-85.4490000000001\\
1290	-72.021\\
1291	-47.607\\
1292	-28.076\\
1293	-21.973\\
1294	-21.973\\
1295	-26.855\\
1296	-26.855\\
1297	-25.635\\
1298	-32.9590000000001\\
1299	-47.607\\
1300	-43.9449999999999\\
1301	-45.1659999999999\\
1302	-52.49\\
1303	-35.4000000000001\\
1304	-20.752\\
1305	-35.4000000000001\\
1306	-56.152\\
1307	-54.932\\
1308	-59.8140000000001\\
1309	-53.711\\
1310	-40.2829999999999\\
1311	-51.27\\
1312	-72.021\\
1313	-70.8009999999999\\
1314	-51.27\\
1315	-81.787\\
1316	-62.2560000000001\\
1317	-63.4770000000001\\
1318	-73.242\\
1319	-70.8009999999999\\
1320	-54.932\\
1321	-47.607\\
1323	-108.643\\
1324	-85.4490000000001\\
1325	-50.049\\
1326	-51.27\\
1327	-51.27\\
1328	-61.0350000000001\\
1329	-72.021\\
1330	-54.932\\
1331	-56.152\\
1332	-73.242\\
1333	-79.346\\
1334	-56.152\\
1335	-45.1659999999999\\
1336	-57.373\\
1337	-87.8910000000001\\
1338	-78.125\\
1339	-80.566\\
1340	-54.932\\
1341	-45.1659999999999\\
1342	-47.607\\
1343	-31.7380000000001\\
1344	-20.752\\
1345	-20.752\\
1346	-17.0899999999999\\
1347	-31.7380000000001\\
1348	-48.828\\
1349	-54.932\\
1350	-40.2829999999999\\
1351	-45.1659999999999\\
1352	-52.49\\
1353	-35.4000000000001\\
1355	-35.4000000000001\\
1356	-29.297\\
1357	-31.7380000000001\\
1358	-51.27\\
1359	-64.6970000000001\\
1360	-41.5039999999999\\
1361	-32.9590000000001\\
1362	-28.076\\
1363	-32.9590000000001\\
1364	-26.855\\
1365	-45.1659999999999\\
1366	-80.566\\
1367	-63.4770000000001\\
1368	-81.787\\
1369	-98.877\\
1370	-97.6559999999999\\
1371	-76.904\\
1372	-58.5940000000001\\
1373	-54.932\\
1374	-57.373\\
1375	-65.9180000000001\\
1376	-76.904\\
1377	-76.904\\
1378	-100.098\\
1379	-108.643\\
1380	-130.615\\
1381	-91.5530000000001\\
1382	-98.877\\
1383	-108.643\\
1384	-85.4490000000001\\
1385	-96.4359999999999\\
1386	-85.4490000000001\\
1387	-51.27\\
1388	-34.1800000000001\\
1389	-36.6210000000001\\
1390	-51.27\\
1391	-43.9449999999999\\
1392	-31.7380000000001\\
1393	-41.5039999999999\\
1394	-32.9590000000001\\
1395	-23.193\\
1396	-31.7380000000001\\
1397	-34.1800000000001\\
1398	-25.635\\
1399	-42.7249999999999\\
1400	-58.5940000000001\\
1401	-45.1659999999999\\
1402	-68.3589999999999\\
1403	-100.098\\
1404	-87.8910000000001\\
1405	-109.863\\
1406	-81.787\\
1407	-81.787\\
1408	-62.2560000000001\\
1409	-37.8420000000001\\
1410	-46.3869999999999\\
1411	-42.7249999999999\\
1412	-29.297\\
1413	-40.2829999999999\\
1414	-21.973\\
1415	-19.5309999999999\\
1416	-24.414\\
1417	-45.1659999999999\\
1418	-57.373\\
1419	-65.9180000000001\\
1420	-65.9180000000001\\
1421	-62.2560000000001\\
1422	-69.5799999999999\\
1423	-57.373\\
1424	-48.828\\
1425	-50.049\\
1426	-43.9449999999999\\
1427	-59.8140000000001\\
1428	-46.3869999999999\\
1429	-36.6210000000001\\
1430	-51.27\\
1431	-69.5799999999999\\
1432	-54.932\\
1433	-41.5039999999999\\
1434	-37.8420000000001\\
1435	-41.5039999999999\\
1436	-32.9590000000001\\
1437	-30.518\\
1438	-41.5039999999999\\
1439	-47.607\\
1440	-52.49\\
1441	-42.7249999999999\\
1442	-53.711\\
1443	-51.27\\
1444	-32.9590000000001\\
1445	-32.9590000000001\\
1446	-58.5940000000001\\
1447	-80.566\\
1448	-78.125\\
1449	-67.1389999999999\\
1450	-62.2560000000001\\
1451	-50.049\\
1452	-48.828\\
1453	-41.5039999999999\\
1454	-54.932\\
1455	-50.049\\
1456	-32.9590000000001\\
1457	-45.1659999999999\\
1458	-52.49\\
1459	-61.0350000000001\\
1460	-68.3589999999999\\
1461	-68.3589999999999\\
1463	-107.422\\
1464	-119.629\\
1465	-80.566\\
1466	-46.3869999999999\\
1467	-32.9590000000001\\
1468	-24.414\\
1469	-34.1800000000001\\
1470	-29.297\\
1471	-51.27\\
1472	-48.828\\
1473	-40.2829999999999\\
1474	-52.49\\
1475	-63.4770000000001\\
1476	-70.8009999999999\\
1477	-75.684\\
1478	-103.76\\
1479	-92.7729999999999\\
1481	-51.27\\
1482	-52.49\\
1483	-54.932\\
1484	-65.9180000000001\\
1485	-51.27\\
1487	-48.828\\
1488	-30.518\\
1489	-45.1659999999999\\
1490	-68.3589999999999\\
1491	-48.828\\
1492	-52.49\\
1493	-86.6700000000001\\
1494	-63.4770000000001\\
1495	-63.4770000000001\\
1496	-85.4490000000001\\
1497	-81.787\\
1498	-53.711\\
1499	-35.4000000000001\\
1500	-36.6210000000001\\
1501	-61.0350000000001\\
1502	-90.3320000000001\\
1503	-97.6559999999999\\
1504	-100.098\\
1505	-79.346\\
1506	-53.711\\
1507	-46.3869999999999\\
1508	-37.8420000000001\\
1509	-34.1800000000001\\
1510	-29.297\\
1511	-40.2829999999999\\
1512	-42.7249999999999\\
1513	-30.518\\
1514	-42.7249999999999\\
1515	-56.152\\
1516	-62.2560000000001\\
1517	-78.125\\
1519	-114.746\\
1520	-84.229\\
1521	-73.242\\
1522	-76.904\\
1523	-63.4770000000001\\
1524	-46.3869999999999\\
1525	-57.373\\
1526	-90.3320000000001\\
1527	-101.318\\
1528	-61.0350000000001\\
1529	-36.6210000000001\\
1530	-26.855\\
1531	-18.3109999999999\\
1532	-21.973\\
1533	-30.518\\
1534	-31.7380000000001\\
1535	-42.7249999999999\\
1536	-37.8420000000001\\
1537	-30.518\\
1538	-29.297\\
1539	-29.297\\
1540	-26.855\\
1541	-31.7380000000001\\
1542	-43.9449999999999\\
1543	-63.4770000000001\\
1544	-68.3589999999999\\
1545	-65.9180000000001\\
1546	-74.463\\
1547	-91.5530000000001\\
1548	-92.7729999999999\\
1549	-106.201\\
1550	-100.098\\
1551	-74.463\\
1552	-54.932\\
1553	-51.27\\
1554	-84.229\\
1555	-97.6559999999999\\
1556	-69.5799999999999\\
1557	-46.3869999999999\\
1558	-26.855\\
1559	-21.973\\
1560	-20.752\\
1561	-14.6479999999999\\
1562	-25.635\\
1563	-40.2829999999999\\
1564	-40.2829999999999\\
1565	-35.4000000000001\\
1566	-24.414\\
1567	-19.5309999999999\\
1568	-28.076\\
1569	-28.076\\
1570	-30.518\\
1571	-24.414\\
1572	-21.973\\
1573	-32.9590000000001\\
1574	-69.5799999999999\\
1575	-79.346\\
1576	-86.6700000000001\\
1577	-63.4770000000001\\
1578	-80.566\\
1579	-114.746\\
1580	-103.76\\
1581	-106.201\\
1582	-81.787\\
1583	-81.787\\
1584	-86.6700000000001\\
1585	-59.8140000000001\\
1586	-56.152\\
1587	-40.2829999999999\\
1588	-35.4000000000001\\
1589	-62.2560000000001\\
1590	-47.607\\
1591	-48.828\\
1592	-53.711\\
1593	-50.049\\
1594	-50.049\\
1595	-52.49\\
1596	-57.373\\
1597	-64.6970000000001\\
1598	-56.152\\
1599	-41.5039999999999\\
1600	-52.49\\
1601	-31.7380000000001\\
1602	-18.3109999999999\\
1603	-25.635\\
1604	-34.1800000000001\\
1606	-9.76600000000008\\
1607	-21.973\\
1608	-32.9590000000001\\
1609	-37.8420000000001\\
1610	-43.9449999999999\\
1611	-39.0630000000001\\
1612	-58.5940000000001\\
1613	-48.828\\
1614	-36.6210000000001\\
1615	-54.932\\
1616	-39.0630000000001\\
1617	-26.855\\
1618	-23.193\\
1619	-17.0899999999999\\
1620	-25.635\\
1621	-19.5309999999999\\
1622	-29.297\\
1623	-43.9449999999999\\
1624	-42.7249999999999\\
1625	-32.9590000000001\\
1626	-40.2829999999999\\
1627	-29.297\\
1628	-43.9449999999999\\
1629	-31.7380000000001\\
1630	-31.7380000000001\\
1631	-30.518\\
1632	-24.414\\
1633	-26.855\\
1634	-26.855\\
1635	-41.5039999999999\\
1636	-36.6210000000001\\
1637	-30.518\\
1638	-32.9590000000001\\
1639	-26.855\\
1640	-31.7380000000001\\
1641	-58.5940000000001\\
1642	-75.684\\
1643	-51.27\\
1644	-48.828\\
1645	-39.0630000000001\\
1646	-58.5940000000001\\
1647	-58.5940000000001\\
1648	-37.8420000000001\\
1649	-46.3869999999999\\
1650	-39.0630000000001\\
1651	-51.27\\
1652	-32.9590000000001\\
1654	-37.8420000000001\\
1655	-47.607\\
1656	-36.6210000000001\\
1657	-41.5039999999999\\
1658	-41.5039999999999\\
1659	-57.373\\
1660	-51.27\\
1661	-72.021\\
1662	-93.9939999999999\\
1663	-78.125\\
1664	-50.049\\
1665	-40.2829999999999\\
1667	-69.5799999999999\\
1668	-56.152\\
1669	-63.4770000000001\\
1670	-43.9449999999999\\
1671	-23.193\\
1672	-24.414\\
1673	-52.49\\
1674	-45.1659999999999\\
1675	-42.7249999999999\\
1676	-64.6970000000001\\
1677	-61.0350000000001\\
1678	-56.152\\
1679	-39.0630000000001\\
1680	-53.711\\
1681	-65.9180000000001\\
1682	-52.49\\
1683	-53.711\\
1684	-48.828\\
1685	-37.8420000000001\\
1686	-42.7249999999999\\
1687	-35.4000000000001\\
1688	-36.6210000000001\\
1689	-57.373\\
1690	-67.1389999999999\\
1691	-70.8009999999999\\
1692	-62.2560000000001\\
1693	-45.1659999999999\\
1694	-35.4000000000001\\
1695	-39.0630000000001\\
1696	-46.3869999999999\\
1699	-79.346\\
1700	-65.9180000000001\\
1701	-40.2829999999999\\
1702	-68.3589999999999\\
1703	-91.5530000000001\\
1704	-65.9180000000001\\
1705	-64.6970000000001\\
1706	-75.684\\
1707	-59.8140000000001\\
1708	-50.049\\
1709	-56.152\\
1710	-51.27\\
1711	-57.373\\
1712	-70.8009999999999\\
1713	-56.152\\
1714	-63.4770000000001\\
1715	-58.5940000000001\\
1716	-59.8140000000001\\
1717	-48.828\\
1718	-63.4770000000001\\
1719	-45.1659999999999\\
1721	-52.49\\
1722	-50.049\\
1723	-30.518\\
1724	-19.5309999999999\\
1725	-15.8689999999999\\
1726	-28.076\\
1727	-53.711\\
1728	-67.1389999999999\\
1729	-65.9180000000001\\
1730	-46.3869999999999\\
1731	-54.932\\
1732	-50.049\\
1733	-43.9449999999999\\
1734	-30.518\\
1735	-42.7249999999999\\
1736	-37.8420000000001\\
1737	-36.6210000000001\\
1738	-41.5039999999999\\
1740	-34.1800000000001\\
1741	-25.635\\
1742	-35.4000000000001\\
1743	-58.5940000000001\\
1744	-42.7249999999999\\
1745	-52.49\\
1746	-45.1659999999999\\
1747	-59.8140000000001\\
1748	-57.373\\
1749	-76.904\\
1750	-58.5940000000001\\
1751	-64.6970000000001\\
1752	-65.9180000000001\\
1753	-37.8420000000001\\
1754	-61.0350000000001\\
1755	-46.3869999999999\\
1756	-51.27\\
1757	-57.373\\
1758	-69.5799999999999\\
1759	-53.711\\
1760	-70.8009999999999\\
1761	-81.787\\
1762	-69.5799999999999\\
1763	-58.5940000000001\\
1764	-50.049\\
1765	-58.5940000000001\\
1766	-52.49\\
1767	-43.9449999999999\\
1768	-37.8420000000001\\
1769	-45.1659999999999\\
1770	-39.0630000000001\\
1771	-72.021\\
1772	-90.3320000000001\\
1773	-79.346\\
1774	-75.684\\
1775	-74.463\\
1776	-46.3869999999999\\
1777	-42.7249999999999\\
1778	-35.4000000000001\\
1779	-30.518\\
1780	-32.9590000000001\\
1781	-52.49\\
1782	-59.8140000000001\\
1783	-70.8009999999999\\
1784	-59.8140000000001\\
1785	-42.7249999999999\\
1786	-72.021\\
1787	-84.229\\
1788	-90.3320000000001\\
1789	-73.242\\
1790	-115.967\\
1791	-85.4490000000001\\
1792	-51.27\\
1793	-39.0630000000001\\
1794	-34.1800000000001\\
1795	-56.152\\
1796	-69.5799999999999\\
1797	-89.1110000000001\\
1798	-68.3589999999999\\
1799	-53.711\\
1800	-31.7380000000001\\
1801	-36.6210000000001\\
1802	-34.1800000000001\\
1803	-39.0630000000001\\
1804	-25.635\\
1805	-35.4000000000001\\
};
\addlegendentry{True output}

\addplot [color=mycolor2, dashed, line width=2.0pt]
  table[row sep=crcr]{%
1006	-75.4433374680916\\
1007	-84.0131295580413\\
1008	-69.0521531457325\\
1009	-41.527862705022\\
1010	-50.5808092701227\\
1011	-53.3625838437169\\
1012	-52.5017454682952\\
1013	-51.5106805168393\\
1014	-18.3152294117463\\
1015	-18.9672076785489\\
1016	-17.730321148327\\
1017	-43.2030631259656\\
1018	-72.9292681188454\\
1019	-66.6886816977985\\
1020	-71.3581898214229\\
1021	-55.3233104702945\\
1022	-36.514111246973\\
1023	-71.427209012443\\
1024	-44.7514131601715\\
1025	-42.7899249928632\\
1026	-62.6255684833764\\
1027	-50.858585042017\\
1028	-53.4864617855922\\
1029	-67.9489849394838\\
1030	-67.1505177859704\\
1031	-54.161303006955\\
1032	-70.9053420612499\\
1033	-73.0246900685511\\
1034	-62.9101834577809\\
1035	-53.4351092488355\\
1036	-46.9141874363636\\
1037	-49.1769087784583\\
1038	-58.1147918722631\\
1039	-54.4745437030183\\
1040	-56.1134228944545\\
1041	-59.6427521359253\\
1042	-61.105974492328\\
1043	-85.4202746155054\\
1044	-76.3024919667121\\
1045	-57.1519754043952\\
1046	-55.6472744623397\\
1047	-33.372551878673\\
1048	-38.2535725437513\\
1049	-43.6092154873616\\
1050	-37.4448121423302\\
1051	-38.9574595795589\\
1052	-52.3246912950588\\
1053	-54.2996922510822\\
1054	-51.0278827598008\\
1055	-77.7176092383484\\
1056	-60.5711513896563\\
1057	-42.3306640600572\\
1058	-35.4781300453008\\
1059	-41.7247465378443\\
1060	-48.739186031517\\
1061	-28.5324165041038\\
1062	-30.7095459668142\\
1063	-40.6355143111768\\
1064	-32.6668799334741\\
1065	-31.3268566441484\\
1066	-36.0907810430976\\
1067	-39.1544663946975\\
1068	-62.3441326385387\\
1069	-55.7754810683552\\
1070	-65.0399772947783\\
1071	-64.7824089778883\\
1072	-64.2177056768451\\
1073	-61.5697431649955\\
1074	-57.0616948640775\\
1075	-54.6913619264408\\
1076	-51.4659168054789\\
1077	-54.0646638591479\\
1078	-77.2066854772547\\
1079	-106.56861212406\\
1080	-112.131431872272\\
1081	-104.859099469491\\
1082	-78.9165675167469\\
1083	-104.033774013299\\
1084	-119.268857150666\\
1085	-120.579689027033\\
1086	-98.5849720037509\\
1087	-123.563350501141\\
1088	-152.310641361103\\
1089	-123.394540827739\\
1090	-102.288223801237\\
1091	-72.9029665109572\\
1092	-55.506571677382\\
1093	-51.7417372410766\\
1094	-57.6000576631634\\
1095	-43.6042877455441\\
1096	-27.8562168576732\\
1097	-29.7102015534165\\
1098	-37.4197331277919\\
1099	-54.8524532996428\\
1100	-47.8278864837109\\
1101	-51.1172833279811\\
1102	-61.6155266685366\\
1103	-63.7502114220056\\
1104	-84.0688846564087\\
1105	-76.679519764763\\
1106	-80.9482121987025\\
1107	-66.7399763773915\\
1108	-60.4725378884127\\
1109	-65.5915331640574\\
1110	-52.670580533699\\
1111	-49.2952238617108\\
1112	-53.7676530433412\\
1113	-55.3581246539075\\
1114	-43.5585599995411\\
1115	-44.9141813770348\\
1116	-38.8383469899659\\
1117	-38.4682653346154\\
1118	-39.5955104422471\\
1119	-33.4419012643518\\
1120	-37.2609400240256\\
1121	-44.8544876382387\\
1122	-64.1183234911039\\
1123	-65.8666129119429\\
1124	-70.1062712424305\\
1125	-46.5877845426476\\
1126	-62.6801705044209\\
1127	-90.369721866329\\
1128	-70.220397182929\\
1129	-77.6611052879584\\
1130	-80.5197054217313\\
1131	-51.5926342304451\\
1132	-42.1694961479768\\
1133	-50.291593642318\\
1134	-55.6892851288001\\
1135	-84.5365030589865\\
1136	-95.9887991010849\\
1137	-98.6075858448826\\
1138	-79.4051204197388\\
1139	-82.4833122716714\\
1140	-73.2095802697593\\
1141	-74.2388413111735\\
1142	-71.4308853217333\\
1143	-50.0408752041317\\
1144	-46.777940616963\\
1145	-44.8429540267291\\
1146	-38.7152080832257\\
1147	-45.0437149546137\\
1148	-59.1706650520287\\
1149	-82.8915748964932\\
1150	-66.1056946502115\\
1151	-55.5468229287017\\
1152	-46.7112788272855\\
1153	-45.474535386187\\
1154	-28.0507902845648\\
1155	-22.2821012118\\
1156	-27.9874739970203\\
1157	-43.2103685238005\\
1158	-34.8633901665269\\
1159	-37.4562252494923\\
1160	-39.4920946566008\\
1161	-37.4922794809286\\
1163	-22.6954088928064\\
1164	-20.1728081544522\\
1165	-28.5457691796644\\
1166	-55.2962754802988\\
1167	-76.8765897917249\\
1168	-81.1023719764321\\
1169	-56.9779638526877\\
1170	-43.0632710509749\\
1172	-25.0640713236705\\
1173	-46.7526681671582\\
1174	-42.023413761824\\
1175	-57.0263010058052\\
1176	-70.7278001637728\\
1177	-112.693417144016\\
1178	-128.95280059711\\
1179	-105.846089465673\\
1180	-105.242093089805\\
1181	-69.9529425756227\\
1182	-80.5422226926403\\
1183	-77.9265933183672\\
1184	-89.9680907044724\\
1185	-66.2236255644011\\
1186	-67.1308387170059\\
1187	-64.8088849275589\\
1188	-58.5102076490546\\
1189	-66.1845325184574\\
1190	-52.7729125338203\\
1191	-83.7071120678834\\
1192	-87.7235641926513\\
1193	-71.5533582305859\\
1194	-48.511178250589\\
1195	-50.2939227371173\\
1196	-80.2826984743067\\
1197	-90.4798196839301\\
1198	-106.772558294258\\
1199	-114.446020118896\\
1200	-120.361457367064\\
1201	-92.6474100563123\\
1202	-92.776469121218\\
1203	-95.7084974572481\\
1204	-106.034162526593\\
1205	-75.278458764403\\
1206	-47.9081468508389\\
1207	-66.3152907402318\\
1208	-58.8771975574816\\
1209	-36.7479142865216\\
1210	-48.745510600213\\
1211	-52.6710028904188\\
1212	-47.2697790914604\\
1213	-37.9404638290007\\
1214	-46.5520094672161\\
1215	-42.584213374404\\
1216	-48.7674813641129\\
1217	-72.7688831171527\\
1218	-61.9532442922321\\
1219	-49.834658171309\\
1220	-50.485292144466\\
1221	-64.8340333662743\\
1222	-107.098292514133\\
1223	-78.8327282246012\\
1224	-54.6803526388403\\
1225	-41.5451921031965\\
1226	-50.9965180491247\\
1227	-46.8862384082456\\
1228	-34.1394649020847\\
1229	-37.159091696667\\
1230	-47.2652903147091\\
1231	-50.5169644050009\\
1232	-58.0373801626952\\
1233	-41.2350802908984\\
1234	-68.1685523267129\\
1235	-77.4952832484755\\
1236	-70.6153829844602\\
1237	-62.1721153652154\\
1238	-63.8926471243701\\
1239	-83.0520927517932\\
1240	-56.4857371652895\\
1241	-27.6725931577487\\
1242	-37.9121784728841\\
1243	-41.2069875273401\\
1244	-55.4261673039641\\
1245	-47.9185382005317\\
1246	-39.6190913742719\\
1247	-54.8598031452389\\
1248	-52.230626854873\\
1249	-42.3284313257464\\
1250	-51.2369287858558\\
1251	-44.8753139677715\\
1252	-42.0526300962222\\
1253	-44.5272226855668\\
1254	-31.5586318396417\\
1255	-37.337407073363\\
1256	-25.5531583397119\\
1257	-32.2201060140876\\
1258	-52.7458773402384\\
1259	-52.7608451281264\\
1260	-68.9930736648041\\
1261	-90.1050990198337\\
1262	-64.9377518366648\\
1263	-41.6237508763625\\
1264	-33.8018326567924\\
1265	-33.7574521616452\\
1266	-46.1313219616529\\
1267	-26.9377810872775\\
1268	-36.0229905528652\\
1269	-40.600870718913\\
1270	-61.0883452453581\\
1271	-60.2202248172907\\
1272	-76.3337975292591\\
1273	-56.7256838468779\\
1274	-53.9501315578782\\
1275	-29.9120069027033\\
1276	-32.9957024379662\\
1277	-23.701600340358\\
1278	-24.0991903174286\\
1279	-29.7463098970695\\
1280	-44.6522340795534\\
1281	-44.8952568540574\\
1282	-51.9829284463244\\
1283	-52.7182915717938\\
1284	-83.0740800344188\\
1285	-73.5594115095219\\
1286	-57.9927902159368\\
1287	-65.8968543774117\\
1288	-68.5551444430639\\
1289	-84.0863625888855\\
1290	-73.3458901080749\\
1291	-48.8244466757108\\
1292	-29.8525782006859\\
1293	-24.268192056014\\
1294	-25.3325082467297\\
1296	-29.0279409301929\\
1297	-26.7498145731722\\
1298	-34.3322345614349\\
1299	-50.6625497148075\\
1300	-47.1192778473569\\
1301	-44.6942254945532\\
1302	-52.7706227893652\\
1303	-35.8078712297126\\
1304	-26.8167426754635\\
1305	-39.3003444161343\\
1306	-58.8406934644488\\
1307	-51.8753586421512\\
1308	-61.8734581817921\\
1309	-52.040238603747\\
1310	-43.984331962773\\
1311	-54.0697870945853\\
1312	-71.4433621607272\\
1313	-71.5734468714074\\
1314	-55.4789138089257\\
1315	-79.471233811546\\
1316	-67.1455022189496\\
1317	-60.3465892276574\\
1318	-72.3324672718452\\
1319	-70.606696197613\\
1320	-59.08015417518\\
1321	-53.7129613757077\\
1322	-76.9480684363068\\
1323	-101.727517224951\\
1324	-86.9506924393061\\
1325	-52.3316494261662\\
1326	-53.2036640556962\\
1327	-55.9092701514141\\
1328	-63.2044886234801\\
1329	-67.536980911455\\
1330	-57.7286496232095\\
1331	-57.0098862836039\\
1332	-74.1109219395598\\
1333	-76.7427991541294\\
1334	-58.3135647149752\\
1335	-48.3096693361215\\
1336	-59.4088889576269\\
1337	-86.7275782727663\\
1338	-78.7139482754651\\
1339	-78.5872212004638\\
1340	-51.889006176046\\
1341	-54.0844531981822\\
1342	-50.0627547220936\\
1343	-32.0316680058199\\
1344	-23.2266930441876\\
1345	-18.697025098036\\
1346	-18.6714527429021\\
1347	-36.4549417478131\\
1348	-48.3291700717834\\
1349	-58.2671796111822\\
1350	-45.5885567141124\\
1351	-47.48249194291\\
1352	-53.3742432697011\\
1353	-35.0684614492422\\
1354	-39.173094624244\\
1355	-37.3783361861576\\
1356	-30.0021744948954\\
1357	-37.545797453263\\
1358	-53.2124704895143\\
1359	-62.9496496805491\\
1360	-40.6449726718226\\
1361	-37.6486518777569\\
1362	-32.8812125464869\\
1363	-37.9131682149443\\
1364	-24.9507257550554\\
1365	-52.0829712428417\\
1366	-82.7799525282769\\
1367	-62.398571132838\\
1368	-83.7595986601439\\
1369	-93.4437223789062\\
1370	-96.8618560148659\\
1372	-61.4407232532321\\
1373	-63.7157421605323\\
1374	-56.1622868308191\\
1375	-66.4362687295834\\
1376	-73.8674099811467\\
1377	-73.3177546034269\\
1378	-98.0590246589486\\
1379	-107.897758069125\\
1380	-119.04403245842\\
1381	-96.2761740859569\\
1382	-101.725417911519\\
1383	-109.414614083687\\
1384	-86.4544543415866\\
1385	-98.5515281254218\\
1386	-85.8924331895073\\
1387	-52.2858500853665\\
1388	-36.7809563478111\\
1389	-36.2836305740748\\
1390	-56.6036258654876\\
1391	-41.2275696484867\\
1392	-33.9513888458598\\
1393	-41.2660453139963\\
1394	-35.6147957025023\\
1395	-27.8634929228003\\
1396	-31.7589428333104\\
1397	-36.8927560783006\\
1398	-28.2075369093268\\
1399	-48.1290397949351\\
1400	-57.4006496881764\\
1401	-45.0559265457659\\
1402	-72.0244562187343\\
1403	-102.108261815335\\
1404	-87.0226531089279\\
1405	-110.762503578218\\
1406	-79.7878735336442\\
1407	-88.6705891556262\\
1408	-61.0124733474283\\
1409	-41.858297315146\\
1410	-53.2395384920501\\
1411	-37.9665055039329\\
1412	-35.1566642840482\\
1413	-42.3444644561544\\
1415	-17.2365168038818\\
1416	-23.5102519216084\\
1417	-46.1234836216042\\
1418	-58.7600807407507\\
1419	-66.7848798283778\\
1420	-67.4545970209751\\
1421	-62.3931805965644\\
1422	-72.265345271966\\
1423	-57.2691347864127\\
1425	-52.039832277934\\
1426	-43.3469195035029\\
1427	-64.3746750312084\\
1428	-45.1633795743944\\
1429	-39.7965765048355\\
1430	-53.8958621891163\\
1431	-68.9823591153852\\
1432	-55.0148231296046\\
1433	-48.0740999281302\\
1434	-40.1716020877107\\
1435	-46.5230762288388\\
1436	-33.2779512585421\\
1437	-31.198919737571\\
1438	-40.7257849067503\\
1439	-50.9422046763307\\
1440	-53.4434566769935\\
1441	-45.7160504256774\\
1442	-53.9360972242339\\
1443	-50.5899316200441\\
1444	-35.6781854944902\\
1445	-37.7359529595049\\
1446	-62.0412288139185\\
1447	-79.2034421969447\\
1448	-78.6717153813579\\
1449	-65.5852610907043\\
1450	-68.2375137602339\\
1451	-52.2819866723269\\
1452	-52.3653066256818\\
1453	-42.6147382547624\\
1454	-57.2871116877841\\
1455	-48.892278081368\\
1456	-37.3009651076509\\
1457	-49.4999944745805\\
1458	-51.6150658874153\\
1459	-61.8717974366511\\
1460	-65.4693564068812\\
1461	-66.3552102963147\\
1462	-88.5264615389249\\
1463	-105.400964925194\\
1464	-119.663426547141\\
1465	-81.5704585938049\\
1466	-50.0252500045078\\
1467	-34.6248038275805\\
1468	-27.4741983331639\\
1469	-34.0492654185834\\
1470	-34.5628274242313\\
1471	-52.9892194273025\\
1472	-46.1514965164954\\
1473	-45.1544115059814\\
1474	-55.7881537889748\\
1475	-58.1582931493497\\
1476	-72.0365882666083\\
1477	-73.830428635549\\
1478	-103.196549908081\\
1479	-94.176771445231\\
1480	-79.4778726887321\\
1481	-58.1429129448572\\
1482	-53.994411944655\\
1483	-57.232825896227\\
1484	-67.2936745854267\\
1485	-50.8500682383112\\
1486	-52.5564627090228\\
1487	-50.9581545529536\\
1488	-29.2982716295023\\
1490	-68.5560875731173\\
1491	-48.1054111623635\\
1492	-56.5810311689117\\
1493	-82.6530484998164\\
1494	-69.5517172340349\\
1495	-63.6310150398758\\
1496	-87.612742605327\\
1497	-76.5744400186775\\
1498	-59.3984297311847\\
1499	-40.2580288644035\\
1500	-39.7827280819038\\
1501	-62.0268854208352\\
1502	-88.6847155227222\\
1503	-96.688086924979\\
1504	-92.3038769742705\\
1505	-81.0817341402858\\
1506	-58.2494280859912\\
1507	-51.7018744187926\\
1508	-37.7103104294961\\
1509	-38.2142092725746\\
1510	-34.2747809802224\\
1511	-42.7982507078434\\
1512	-45.3110933705243\\
1513	-30.733690255802\\
1514	-47.0267358712044\\
1515	-55.9221223566703\\
1516	-62.6149707252371\\
1517	-74.8051799319387\\
1518	-98.166715312474\\
1519	-111.310765420224\\
1520	-87.4201682532209\\
1521	-79.4038791701573\\
1522	-76.9486435361032\\
1523	-70.9077734904672\\
1524	-52.0125743561211\\
1525	-55.7017171048994\\
1526	-89.2205537065438\\
1527	-94.595042777195\\
1528	-64.7008526769007\\
1529	-38.6683969432027\\
1530	-26.5907081674288\\
1531	-16.1993991247903\\
1532	-21.7869382814947\\
1533	-33.2962683813057\\
1534	-36.4673553358739\\
1535	-45.752775675023\\
1536	-38.8076934562689\\
1537	-32.6387394398703\\
1538	-32.8435004488158\\
1539	-32.7869015926099\\
1540	-31.3328167697659\\
1541	-32.7079034927622\\
1542	-44.9630797612858\\
1543	-65.9680160225305\\
1544	-67.0207880529556\\
1545	-66.4125518012213\\
1546	-76.1795696957902\\
1547	-89.6602208842983\\
1548	-95.303843612568\\
1549	-105.131057474371\\
1550	-103.171637045094\\
1551	-79.6153673146005\\
1552	-57.7361988253363\\
1553	-57.7893451699763\\
1554	-81.8549238861679\\
1555	-93.5013329909982\\
1556	-68.5950935935971\\
1557	-54.630739768236\\
1558	-19.2995777209346\\
1559	-24.3054576535885\\
1560	-23.6220686472268\\
1561	-14.6508147936677\\
1562	-30.0244444883956\\
1563	-37.0199058042322\\
1564	-40.3928832327197\\
1565	-38.2443120213227\\
1566	-26.2626063229047\\
1567	-24.8655067222626\\
1568	-26.2710903240438\\
1570	-32.5575025077683\\
1571	-27.0634544772386\\
1572	-24.3961456356078\\
1573	-35.1274956764564\\
1574	-77.3948270052865\\
1575	-88.6251512621211\\
1576	-85.7588985135546\\
1577	-68.6731016106703\\
1578	-80.9514511522409\\
1579	-116.31860736415\\
1580	-105.843329622528\\
1581	-105.539180586678\\
1582	-84.5858084662912\\
1583	-86.0956169918711\\
1584	-89.1541143849392\\
1585	-62.4084769181623\\
1586	-58.2401851696659\\
1587	-36.0553613304062\\
1588	-36.0372684156607\\
1589	-67.4775926293689\\
1590	-43.4726544020334\\
1591	-49.8291216018588\\
1592	-55.1123403023439\\
1593	-49.465384301496\\
1594	-55.0933282412793\\
1595	-53.5636725625161\\
1596	-60.3056998385682\\
1597	-60.9710046961113\\
1598	-59.4617738201505\\
1599	-45.2764177871768\\
1600	-56.6356916376683\\
1601	-27.7305434860809\\
1602	-16.9414998775981\\
1603	-24.9235052251965\\
1604	-35.6763709459492\\
1605	-21.094530173275\\
1606	-16.5563172420229\\
1607	-20.7585941601683\\
1608	-34.2722348243649\\
1609	-39.4017360091295\\
1610	-48.8687382596506\\
1611	-43.1757835034621\\
1612	-56.2901527781012\\
1613	-57.6404056320075\\
1614	-38.0333022360765\\
1615	-53.0535023487366\\
1616	-33.1098850541734\\
1617	-32.5934348130991\\
1618	-23.2927718418216\\
1619	-19.7056282473031\\
1620	-26.4427055955173\\
1621	-20.9302745398893\\
1622	-29.583728374324\\
1623	-50.3081716562522\\
1624	-43.9936212724315\\
1625	-38.2896075922863\\
1626	-38.4103716437639\\
1627	-34.6452773773374\\
1628	-39.4408272694327\\
1629	-34.4645353866911\\
1630	-33.2065539206503\\
1631	-30.7779782019572\\
1632	-27.3089026022399\\
1633	-31.1751503448279\\
1634	-26.8621407577903\\
1635	-38.6698272806993\\
1636	-40.8043376082915\\
1637	-31.2810987834293\\
1638	-33.9795718402399\\
1639	-27.4956396398434\\
1640	-32.1818638227435\\
1641	-59.6075791974565\\
1642	-79.1140767876755\\
1643	-54.6123177364982\\
1644	-52.3230012125516\\
1645	-39.1932798701334\\
1646	-63.3290109535453\\
1647	-56.3678099830001\\
1648	-40.3415823855075\\
1649	-48.3503068094074\\
1650	-36.2516995628873\\
1651	-50.6637021590172\\
1652	-33.856420996513\\
1653	-33.5453750125969\\
1654	-43.3087835056619\\
1655	-45.7842639226217\\
1656	-41.500987694677\\
1657	-38.8362104950049\\
1658	-42.5946978411737\\
1659	-59.074107098181\\
1660	-51.9824087299805\\
1661	-71.787706049123\\
1662	-98.807713093277\\
1663	-75.8373008086833\\
1664	-56.58953363399\\
1665	-40.8597144754278\\
1666	-58.3674994164719\\
1667	-70.4909706676137\\
1668	-51.9594166374716\\
1669	-66.6542001515136\\
1670	-43.3301446859707\\
1671	-26.7533611572742\\
1672	-27.0988375353138\\
1673	-50.9961291160196\\
1674	-52.3272423663602\\
1675	-40.966426866016\\
1676	-65.1264976136365\\
1677	-59.0321494241318\\
1678	-54.8907508813213\\
1679	-47.5779408720825\\
1680	-48.2734011139808\\
1681	-64.3798678411033\\
1682	-56.2386358514686\\
1684	-52.94686718905\\
1685	-40.2433284030128\\
1686	-46.9714761914011\\
1687	-33.3267985707134\\
1688	-40.0821088384293\\
1689	-59.4214420854889\\
1690	-62.1826815490356\\
1691	-65.7930753070375\\
1692	-65.244975417389\\
1693	-47.6520349881814\\
1694	-38.9073140337516\\
1695	-41.655612160336\\
1696	-47.4794739762035\\
1697	-56.0551225035665\\
1699	-77.7009418672294\\
1700	-64.7007104925899\\
1701	-45.3215080746052\\
1702	-68.1351729152427\\
1703	-87.7959919664445\\
1704	-68.2250708545912\\
1705	-65.4298230651714\\
1706	-75.5139341517608\\
1707	-62.3040322951342\\
1708	-53.384684248781\\
1709	-58.0324712385379\\
1710	-52.2144470791648\\
1711	-59.1121983237408\\
1712	-69.092751755712\\
1713	-55.9128832540334\\
1714	-63.7836371332273\\
1715	-60.8058809921927\\
1716	-59.4605174304265\\
1717	-51.9079377212834\\
1718	-61.0688771311102\\
1719	-50.1867662590089\\
1720	-48.1646870023765\\
1721	-55.9481272834405\\
1722	-50.6356346322127\\
1723	-30.1815811002925\\
1724	-20.499557064031\\
1725	-18.3844415032343\\
1726	-23.422680331415\\
1727	-59.1868124746154\\
1728	-73.842690167962\\
1729	-69.8719098000106\\
1730	-49.2841635809682\\
1731	-55.2393854220813\\
1732	-52.8321919345249\\
1733	-43.964086232028\\
1734	-32.2034821632658\\
1735	-40.1666529921235\\
1736	-44.0185061280424\\
1737	-37.673857469327\\
1738	-44.9936655549659\\
1739	-38.0825338776078\\
1740	-38.3180228999774\\
1741	-26.6255391141351\\
1742	-35.6678496459026\\
1743	-60.5147559233244\\
1744	-42.1331121209678\\
1745	-51.0619909662896\\
1746	-48.2447860562313\\
1747	-59.9155166366779\\
1748	-57.8672386988094\\
1749	-72.386242786022\\
1750	-61.8524216341616\\
1751	-63.9640423701424\\
1752	-64.4357160883176\\
1753	-41.7265309434777\\
1754	-59.2174319105993\\
1755	-46.6668044160776\\
1756	-54.732824495512\\
1757	-55.6411728732669\\
1758	-68.339513624667\\
1759	-56.1817221253984\\
1760	-65.6482924074146\\
1761	-79.6130341195428\\
1763	-62.7567935136785\\
1764	-53.4565455872234\\
1765	-58.9275099998083\\
1766	-52.6982031835414\\
1767	-47.8636519609381\\
1768	-41.8432559617358\\
1769	-46.2585613546651\\
1770	-43.4530187971636\\
1771	-70.0412890990553\\
1772	-91.152484824875\\
1773	-75.7784915661978\\
1774	-72.655837737748\\
1775	-75.3350188341381\\
1776	-51.8107710607646\\
1777	-44.6641122245837\\
1778	-38.0077668408367\\
1779	-33.3359012864082\\
1780	-36.3702572073646\\
1781	-51.1787613300612\\
1782	-61.9092656132407\\
1783	-66.6326691007969\\
1784	-60.1594774827895\\
1785	-47.035030379179\\
1786	-72.2206113769296\\
1787	-80.6752071394151\\
1788	-88.623103169381\\
1789	-77.3501924969605\\
1790	-106.018312202308\\
1791	-96.001684628581\\
1792	-51.2032971023432\\
1793	-42.6712569243562\\
1794	-37.5564062399801\\
1795	-56.6523852999096\\
1796	-68.3077985077348\\
1797	-84.4637529127199\\
1798	-72.1836104484087\\
1799	-58.2979986279331\\
1800	-31.4178737500022\\
1802	-39.7093289763129\\
1803	-42.5550671851713\\
1804	-29.072201530407\\
1805	-35.8005668366116\\
};
\addlegendentry{OSA predition}

\addplot [color=mycolor3, dotted, line width=2.0pt]
  table[row sep=crcr]{%
1006	-76.904\\
1007	-90.3320000000001\\
1008	-68.3589999999999\\
1009	-40.2829999999999\\
1010	-50.5808092701232\\
1011	-54.0821540872721\\
1012	-55.8788220715292\\
1013	-53.0134977745779\\
1014	-22.0048170934431\\
1016	-19.3264634857696\\
1017	-44.8141520938855\\
1018	-75.5139019738849\\
1019	-73.1467221127898\\
1020	-76.1058887789843\\
1021	-60.3861605584868\\
1022	-42.1375237198029\\
1023	-76.9856703017342\\
1024	-52.0585945756527\\
1025	-45.4474025069674\\
1026	-67.2162926458113\\
1027	-55.3505763999976\\
1028	-55.4766429741451\\
1029	-73.7294177081371\\
1030	-69.4140573039758\\
1031	-56.2491664374727\\
1032	-73.7980163944617\\
1033	-73.0174547392314\\
1034	-62.6668008474146\\
1035	-55.1492872998947\\
1036	-49.2570514183938\\
1037	-53.1679691272427\\
1038	-62.2703641701976\\
1039	-57.9364215835981\\
1040	-59.1100171113599\\
1041	-60.6246805813653\\
1042	-62.004069941788\\
1043	-84.1158778643769\\
1044	-75.2931172944709\\
1045	-56.0750941966319\\
1046	-55.4342047195066\\
1047	-35.5507624168474\\
1048	-38.889920220844\\
1049	-46.0620847030975\\
1050	-38.9171769112229\\
1051	-41.4054760467657\\
1052	-55.4107805408546\\
1053	-57.1806128763651\\
1054	-53.39789511706\\
1055	-79.8930055352464\\
1056	-62.1993322226112\\
1057	-42.8796288066774\\
1058	-38.3162614305943\\
1059	-45.8862864042255\\
1060	-52.5566290294607\\
1061	-32.8303648359056\\
1062	-33.9625002955954\\
1063	-43.9941220743576\\
1064	-37.1069909346354\\
1065	-35.0614078126564\\
1066	-41.131315072188\\
1067	-43.6564616631917\\
1068	-66.0071767967224\\
1069	-60.7110263267105\\
1070	-68.7699006877463\\
1071	-67.6515786309899\\
1072	-68.4925903980732\\
1073	-62.1540420595861\\
1074	-59.5692310521606\\
1075	-56.1607389651349\\
1076	-53.7399944106367\\
1077	-56.7410955776261\\
1078	-81.6473250073013\\
1079	-109.097088804536\\
1080	-113.195882036124\\
1081	-104.900814603931\\
1082	-75.6111601999962\\
1083	-103.130429082776\\
1084	-119.515162949437\\
1085	-117.875241672881\\
1086	-93.9687343336859\\
1088	-148.127425898107\\
1089	-117.184161111939\\
1090	-100.673512739027\\
1091	-73.3664949939794\\
1092	-56.9973506750234\\
1093	-54.0372866025716\\
1094	-61.3459198223561\\
1095	-47.2906906977698\\
1096	-29.9590631007725\\
1097	-30.5603052543077\\
1098	-38.0024661784116\\
1099	-55.5708142378287\\
1100	-50.2080226121416\\
1101	-52.4913962675755\\
1102	-64.5455232810907\\
1103	-65.9884516671586\\
1104	-85.7055279152639\\
1105	-77.7051540538648\\
1106	-83.6980211514608\\
1107	-66.9099219769228\\
1108	-62.1102388360193\\
1109	-69.3916465906409\\
1110	-55.6104602116238\\
1111	-53.5562580548108\\
1112	-59.6883716763462\\
1113	-60.1206852206133\\
1114	-47.8380177038798\\
1115	-50.0627052743007\\
1116	-43.216243190793\\
1117	-44.4839475014792\\
1118	-45.296105911362\\
1119	-37.7302265731842\\
1120	-42.4286172420091\\
1121	-50.3467859598411\\
1122	-69.1618032892716\\
1123	-70.7676633847\\
1124	-74.4206786349309\\
1125	-49.2205643569762\\
1126	-65.6472838068876\\
1127	-95.5881660528537\\
1128	-74.7861232030159\\
1129	-81.6799097191172\\
1130	-82.6174707712373\\
1131	-53.3048410586125\\
1132	-44.1597469407161\\
1133	-53.930700284302\\
1134	-59.7359122306184\\
1135	-89.7461625028832\\
1136	-101.932386450318\\
1137	-100.597675450981\\
1138	-79.9229711159983\\
1139	-84.0674671857996\\
1140	-74.7177770621197\\
1141	-75.8977760868672\\
1142	-75.0098108770258\\
1143	-54.1232634856824\\
1144	-49.4461254038649\\
1145	-47.9756873493641\\
1146	-43.7469606320906\\
1147	-49.2072118722842\\
1148	-64.4836448964613\\
1149	-89.0322587992493\\
1150	-70.6450718786477\\
1151	-57.5685411364057\\
1152	-50.8743787638286\\
1153	-50.2566590837903\\
1154	-33.4547563577844\\
1155	-26.6647009211572\\
1156	-31.301529272461\\
1157	-47.9404566600426\\
1158	-39.1448290321205\\
1159	-41.7077942721787\\
1160	-44.5384873097485\\
1161	-42.2583516892919\\
1162	-34.6951198364\\
1163	-27.7320252802253\\
1164	-24.3180744400979\\
1165	-32.8488507368206\\
1166	-60.4692788400428\\
1167	-83.0678292476271\\
1168	-88.016366181982\\
1169	-63.9237771090088\\
1170	-49.4440490630407\\
1171	-40.9162629158434\\
1172	-31.2652236594993\\
1173	-52.4462626696552\\
1174	-49.2159917048355\\
1176	-76.8824784134201\\
1177	-117.414489011073\\
1178	-135.19589286819\\
1179	-112.171514749521\\
1180	-111.41385205079\\
1181	-76.725213301544\\
1182	-87.4762519772016\\
1183	-85.7987042012578\\
1184	-96.3150584861512\\
1185	-73.0609514159198\\
1186	-71.5441247649885\\
1187	-70.5072773096865\\
1188	-64.8528799920325\\
1189	-72.2563599147845\\
1190	-57.6706378301642\\
1191	-88.5409117573113\\
1192	-94.7238017468915\\
1193	-72.9138222241761\\
1194	-49.6608110338689\\
1195	-53.2450713266189\\
1196	-83.0028658711612\\
1197	-93.9416956187151\\
1198	-108.758362526627\\
1199	-114.302144001072\\
1200	-118.170989773957\\
1201	-90.9572477202869\\
1202	-91.4883102696003\\
1203	-97.9369373872903\\
1204	-107.428204571842\\
1205	-74.5607783683301\\
1206	-48.2190479301819\\
1207	-66.9836667236859\\
1208	-60.7140063752211\\
1209	-39.8431221208973\\
1210	-52.3553192773747\\
1211	-56.3123462917738\\
1212	-49.9983374898993\\
1213	-42.1422955763587\\
1214	-51.4161437819341\\
1215	-47.4845360967686\\
1216	-54.1276636351561\\
1217	-76.0866857516287\\
1218	-67.9235623066618\\
1219	-54.271508765215\\
1220	-55.890825211347\\
1221	-71.9868117502306\\
1222	-112.539461911205\\
1223	-84.9219158866222\\
1224	-59.3833618054912\\
1225	-46.3675449730727\\
1226	-55.630272909973\\
1227	-52.0901956379073\\
1228	-41.1742737290795\\
1229	-43.6480473802553\\
1230	-53.6701739086373\\
1231	-58.1772309624616\\
1232	-64.0914784297472\\
1233	-46.5278649055572\\
1234	-72.4383441938014\\
1235	-83.1999357734594\\
1236	-75.9432572286257\\
1237	-66.4897874174665\\
1238	-69.8641344602217\\
1239	-88.7138450151849\\
1241	-33.4170928277376\\
1242	-42.3062367942584\\
1243	-45.4434190660929\\
1244	-58.7030877225213\\
1245	-52.4037170534145\\
1246	-43.7573132592165\\
1247	-59.5293059048042\\
1248	-57.3100829970617\\
1249	-46.3313842720902\\
1250	-56.6633802035342\\
1251	-50.7265888576694\\
1252	-48.1013940869457\\
1253	-51.4702092184395\\
1254	-36.8189765603543\\
1255	-42.8113239813131\\
1256	-32.3347303843468\\
1257	-35.8735747328287\\
1258	-58.8085172969288\\
1259	-59.8655572069565\\
1260	-74.4285207748005\\
1261	-95.492178266021\\
1262	-69.9467356773944\\
1263	-46.45444982852\\
1264	-39.4817195571741\\
1265	-39.0939450309725\\
1266	-52.6996963859835\\
1267	-34.6748247111325\\
1268	-40.1807655054174\\
1269	-46.9296924637674\\
1270	-67.0464331641977\\
1271	-65.1362973971611\\
1272	-83.8472203345054\\
1273	-61.3459794058672\\
1274	-58.6768042273759\\
1275	-36.039528024296\\
1276	-36.1498806121217\\
1277	-27.7721121395859\\
1278	-27.8798495885626\\
1279	-32.9540705316192\\
1280	-49.3677062365477\\
1281	-50.4446042969648\\
1282	-56.0177070921345\\
1283	-57.8525491928081\\
1284	-88.1319596161445\\
1285	-79.3207646669307\\
1286	-62.9197004323521\\
1287	-71.9302963947223\\
1288	-74.6664884413442\\
1289	-89.0285608045947\\
1290	-77.3124497696226\\
1291	-52.6171860138172\\
1292	-33.1515572338108\\
1293	-27.3533705537175\\
1294	-28.8131234854782\\
1295	-31.343976424926\\
1296	-32.5484244534828\\
1297	-30.6120615743691\\
1298	-38.0757430029962\\
1299	-54.4437916411439\\
1300	-51.6357081233864\\
1301	-49.644329774351\\
1302	-56.7106808644357\\
1303	-39.2369784286993\\
1304	-29.8016388427163\\
1305	-44.2124615461519\\
1306	-64.4021349518227\\
1307	-57.7690943636489\\
1308	-65.9931376189818\\
1310	-46.903252936914\\
1311	-57.9481306658449\\
1312	-75.9480380070888\\
1313	-75.1284987349813\\
1314	-59.031658362856\\
1315	-84.3345933224039\\
1316	-69.9921262170096\\
1317	-65.0014217413107\\
1318	-74.9722531710288\\
1319	-72.5406631209141\\
1320	-60.9042573890113\\
1321	-56.6661111393219\\
1322	-81.8047621618284\\
1323	-105.486786368137\\
1324	-87.9566497039184\\
1325	-54.0951443423919\\
1326	-55.4010653801415\\
1327	-57.991440895124\\
1328	-67.1354740031775\\
1329	-71.7597768551477\\
1330	-59.4332374130206\\
1331	-60.0203486766752\\
1332	-76.8463397628027\\
1333	-79.2312597069649\\
1334	-59.5190170167643\\
1335	-50.2840001336951\\
1336	-62.2931667094238\\
1337	-89.8393998873792\\
1338	-81.1410809904319\\
1339	-81.1913150322457\\
1340	-53.1492033117831\\
1341	-53.80047374828\\
1342	-53.7078238558486\\
1343	-35.6203226742809\\
1344	-25.5979354368746\\
1345	-22.0901873833454\\
1346	-20.4789622040587\\
1347	-38.8104562354761\\
1348	-52.2547127741934\\
1349	-61.0721471989041\\
1350	-49.5917223439417\\
1351	-53.0439019129601\\
1352	-58.6616164989637\\
1353	-39.7408845099503\\
1354	-43.1308997031458\\
1355	-42.2531675307534\\
1356	-34.5870466803724\\
1357	-41.541151488339\\
1358	-59.3976913491424\\
1359	-68.8121125689549\\
1360	-44.6835314895827\\
1361	-40.8881497376615\\
1362	-37.5100104433345\\
1363	-43.3660680112414\\
1364	-31.0784828109072\\
1365	-56.8658756810191\\
1366	-90.7191511853239\\
1367	-69.5249058701843\\
1368	-89.6220254269565\\
1369	-99.9715342453096\\
1370	-100.135198858202\\
1371	-81.5033740697611\\
1372	-64.4969161986012\\
1373	-67.0303083673061\\
1374	-62.4807463132315\\
1375	-71.1226818094019\\
1376	-78.6389445352888\\
1377	-76.4642579151143\\
1378	-99.3433377136139\\
1379	-108.297585513428\\
1380	-118.90414061125\\
1381	-91.2951509321429\\
1383	-108.417697111482\\
1384	-85.3314954329637\\
1385	-98.5642231919385\\
1386	-86.8297929083842\\
1387	-53.1128655065297\\
1388	-37.7554857546897\\
1389	-38.2451337715929\\
1390	-58.0260467260928\\
1391	-44.6843846844808\\
1392	-35.5356607328122\\
1393	-43.5599554685389\\
1394	-37.6374039265752\\
1395	-30.3108716318156\\
1396	-35.7499623886722\\
1397	-39.9596295990468\\
1398	-31.9986781498449\\
1399	-52.5500112035686\\
1400	-63.3008647982031\\
1401	-49.3591209323179\\
1402	-76.253889880667\\
1403	-107.610368674624\\
1404	-92.1635985580622\\
1405	-115.159370340434\\
1406	-84.0574060388021\\
1407	-91.489441336897\\
1408	-66.0614442254425\\
1409	-45.3080085364481\\
1410	-57.6241589456986\\
1411	-44.7058495362123\\
1412	-38.1567442233413\\
1413	-47.8192596645945\\
1414	-35.0005792366344\\
1415	-23.9740421765803\\
1416	-28.0737173141736\\
1417	-50.2003598338276\\
1418	-63.1374064510885\\
1419	-70.7091757413052\\
1420	-71.1203777872224\\
1421	-66.2179212315521\\
1422	-75.6386672754522\\
1423	-61.1742913900284\\
1424	-57.8728807286259\\
1425	-57.0867263978867\\
1426	-48.219254035442\\
1427	-68.348428976851\\
1428	-50.6653304503695\\
1429	-43.5772055346138\\
1431	-74.1530626428744\\
1432	-58.9314121183522\\
1433	-51.5519428418561\\
1434	-45.7784890211196\\
1435	-51.7887565859401\\
1436	-39.4851514815064\\
1437	-36.5627152251448\\
1439	-55.0219947090964\\
1440	-58.3090708056982\\
1441	-49.9630093026328\\
1442	-58.7492236922499\\
1443	-54.7925013344448\\
1444	-38.7368811042891\\
1445	-41.5492348524663\\
1446	-67.1639837110333\\
1447	-84.9674596773805\\
1448	-83.3668894022801\\
1449	-70.0726306643598\\
1450	-71.3648782966347\\
1451	-57.1865621489992\\
1452	-57.1880078648733\\
1453	-47.7977079890773\\
1454	-62.486459631062\\
1455	-54.2105189842073\\
1456	-41.1662041512066\\
1457	-54.6469907972978\\
1458	-57.6585622621194\\
1459	-66.526398053144\\
1460	-70.3035010767774\\
1461	-69.342284208062\\
1462	-90.4293021928834\\
1463	-107.405727445411\\
1464	-120.336402098544\\
1465	-82.2265117201543\\
1466	-51.050191803778\\
1467	-36.7679435826615\\
1468	-29.7322396933703\\
1469	-37.0832063947817\\
1470	-37.1159298681241\\
1471	-57.551510935217\\
1472	-50.481071941045\\
1473	-47.5992935936986\\
1474	-60.4133261707989\\
1475	-63.0981050989631\\
1476	-73.9387961770356\\
1477	-76.7777276802174\\
1478	-104.753360503624\\
1479	-95.1186148191566\\
1480	-80.9955245423096\\
1481	-62.311295995891\\
1482	-59.9633277191863\\
1483	-62.6501952330693\\
1484	-73.4642280098915\\
1485	-56.659585919356\\
1486	-57.105833734867\\
1487	-55.8903903564703\\
1488	-34.0541758642639\\
1489	-52.0208804329848\\
1490	-73.3522736212208\\
1491	-52.0332275503797\\
1492	-59.5621655499865\\
1493	-87.2892385309797\\
1494	-71.5486360293494\\
1495	-68.08800553177\\
1496	-91.3223684320722\\
1497	-80.5450005428552\\
1498	-60.816458946492\\
1499	-43.7012494546018\\
1500	-44.6490218822696\\
1501	-66.9607516235067\\
1502	-94.0934476713396\\
1503	-101.137525734276\\
1504	-95.8628410897334\\
1505	-80.7668721906709\\
1506	-58.830927496871\\
1507	-54.1326536118579\\
1508	-41.2050549786436\\
1509	-40.9767930291246\\
1510	-38.4207048746503\\
1511	-48.4254528451877\\
1512	-50.7500385383448\\
1513	-36.3137427164302\\
1514	-52.031645643819\\
1515	-62.1295692062538\\
1516	-67.7019620841297\\
1517	-79.6180852798946\\
1518	-101.303324399302\\
1519	-114.991375353477\\
1520	-88.8636173617963\\
1521	-81.9427982437985\\
1522	-81.6493579192688\\
1523	-74.4304914041543\\
1524	-58.1640003062682\\
1525	-63.1959550238696\\
1526	-95.0417225767471\\
1527	-100.10082171543\\
1528	-66.550671335453\\
1529	-41.5089107672634\\
1531	-17.8019813927256\\
1532	-22.5341789836177\\
1533	-34.1503124789147\\
1534	-38.2587691282333\\
1535	-48.8470643289093\\
1537	-36.0623936110305\\
1538	-36.6894191769438\\
1539	-37.3844129561146\\
1540	-36.3592409005912\\
1541	-38.6666405453941\\
1542	-50.4660392715223\\
1543	-71.5833225248302\\
1544	-73.0194608278116\\
1545	-70.7842193355416\\
1546	-80.3802993643219\\
1547	-94.1225569831176\\
1548	-98.242171373332\\
1549	-109.053907270136\\
1550	-105.980110595631\\
1551	-83.1818224738415\\
1552	-62.8059143392838\\
1553	-62.8155037966685\\
1554	-89.1780357412526\\
1555	-98.9873725800476\\
1556	-71.8025935503154\\
1557	-57.1814038829496\\
1558	-24.5058652578803\\
1559	-24.8688801771484\\
1560	-24.8745316247077\\
1561	-17.7044099236625\\
1562	-31.7379961974896\\
1563	-40.5746720993172\\
1564	-41.9221823472205\\
1565	-39.8268861916308\\
1566	-28.8819472230118\\
1567	-27.3079182876252\\
1568	-30.3788790837039\\
1569	-31.9482631878911\\
1570	-35.5080307540964\\
1571	-30.465025734117\\
1572	-27.8814369101983\\
1573	-39.0344843937892\\
1574	-82.048823478598\\
1575	-96.1378534975977\\
1576	-95.7194686246307\\
1577	-76.4443001794771\\
1578	-90.5307875052426\\
1579	-125.136465414746\\
1580	-114.22538340402\\
1581	-113.8817865212\\
1582	-91.1123364643865\\
1583	-92.9502815605383\\
1584	-96.8134936059851\\
1585	-69.5139463107623\\
1586	-65.262093537347\\
1587	-42.5225142440172\\
1588	-39.4707131248556\\
1589	-71.344610307825\\
1590	-48.7982464256054\\
1591	-52.0282712586129\\
1592	-57.7980405519045\\
1593	-52.5338589920239\\
1594	-57.041936685215\\
1595	-57.4395807111271\\
1596	-63.867255259938\\
1597	-65.2430864743678\\
1598	-61.6693032433786\\
1599	-48.5557398809572\\
1600	-60.914196942349\\
1601	-32.3594751326013\\
1602	-18.7246221075097\\
1603	-26.0076657873844\\
1604	-36.7467558079031\\
1605	-22.2758850743851\\
1606	-17.0117379541023\\
1607	-23.9117278814751\\
1608	-36.1484659801265\\
1609	-41.8117501671488\\
1610	-51.7999942936374\\
1611	-47.3673173606126\\
1612	-61.3828236802253\\
1613	-61.0218874341933\\
1614	-44.7793042667938\\
1615	-58.9448777074185\\
1616	-36.6482811894625\\
1617	-33.6222963208968\\
1618	-26.6784002758482\\
1619	-22.2797276531205\\
1620	-29.1126979280889\\
1621	-23.7844876626596\\
1622	-32.4863517300141\\
1623	-53.030537016657\\
1624	-48.9258880220164\\
1625	-42.5525156855974\\
1626	-44.0947760956749\\
1627	-38.6181799923613\\
1628	-45.111877361433\\
1629	-37.224062911029\\
1630	-36.5963326806384\\
1631	-34.3909172206556\\
1632	-29.856371715309\\
1633	-34.7218355177843\\
1634	-31.4361400229504\\
1635	-42.3238939517316\\
1636	-42.949761209866\\
1637	-35.0949183756043\\
1638	-37.0850117539519\\
1639	-30.1222680247165\\
1640	-34.921274211147\\
1641	-62.2668322975601\\
1642	-81.9229559661196\\
1643	-58.3050478664227\\
1644	-56.5983476224824\\
1645	-43.933836266777\\
1646	-67.6094949275789\\
1647	-62.1522652317387\\
1648	-43.9979214638377\\
1649	-52.4963407663606\\
1650	-40.6013681263601\\
1651	-52.9322109920797\\
1652	-35.6921921622611\\
1653	-35.5501783528725\\
1654	-43.9757936136186\\
1655	-48.7441914838766\\
1656	-42.9533781585012\\
1657	-42.0713368295731\\
1658	-44.2691136739847\\
1659	-61.1083728879253\\
1660	-54.5201812205273\\
1661	-73.9894841425278\\
1662	-100.884808804407\\
1664	-58.573893142772\\
1665	-45.0043120331952\\
1666	-62.204645588728\\
1667	-75.1916174068547\\
1668	-56.4493475898032\\
1669	-68.7252116801897\\
1670	-46.5883330180636\\
1671	-29.0512877235731\\
1672	-29.9782931806392\\
1673	-54.759629654475\\
1674	-54.8453707708184\\
1675	-46.3519129014937\\
1676	-68.5947596036451\\
1677	-62.3278315905504\\
1678	-57.1973821268161\\
1679	-48.8012253601948\\
1680	-52.8815299513151\\
1681	-65.4791527567081\\
1682	-57.0167879257215\\
1683	-57.2947548645222\\
1684	-54.8631403760571\\
1685	-43.4058779727545\\
1686	-50.7236968606865\\
1687	-37.993480705996\\
1688	-43.1069838387102\\
1689	-63.852568358373\\
1690	-66.7232075454408\\
1691	-67.4619375590571\\
1692	-65.1423642064228\\
1693	-49.0251115605438\\
1694	-40.6585943608209\\
1695	-44.1372427465935\\
1696	-50.8118781268608\\
1697	-59.4383652873123\\
1699	-79.5138866003153\\
1700	-65.5622613147143\\
1701	-45.4646440788536\\
1702	-70.3420116302527\\
1703	-89.2384607649508\\
1704	-68.0851766374474\\
1705	-66.7326460458894\\
1706	-76.5946105756625\\
1707	-62.8975834816124\\
1708	-55.1308202057548\\
1709	-60.7896728539931\\
1710	-55.0972251506423\\
1711	-62.061553261208\\
1712	-72.5271848837399\\
1713	-58.0211624391245\\
1714	-65.6153866036752\\
1715	-62.5376802282683\\
1716	-61.6917645546514\\
1717	-53.5293051189212\\
1718	-63.8213722628584\\
1719	-51.4546568228654\\
1720	-51.3465092565909\\
1721	-58.2652815706999\\
1722	-53.906470765311\\
1723	-33.2164972838984\\
1724	-22.3806065391898\\
1725	-20.4557313088173\\
1726	-26.1290605253841\\
1727	-59.554809416559\\
1728	-77.1082963686595\\
1729	-74.7307830776813\\
1730	-54.4483954582493\\
1731	-61.0480653286954\\
1732	-58.0174529405504\\
1733	-49.428858176535\\
1734	-36.5015921406323\\
1735	-44.4007159241385\\
1736	-46.6435734684703\\
1737	-42.5121204681925\\
1738	-49.189070642911\\
1739	-42.7902891301944\\
1740	-42.572448736724\\
1741	-31.6189657279824\\
1742	-40.2166530594425\\
1743	-64.7177243976816\\
1744	-46.6350497914818\\
1745	-54.4754502472481\\
1746	-50.5143858217291\\
1747	-63.3715634224304\\
1748	-60.5634917654875\\
1749	-74.9968737514\\
1750	-62.3433085577165\\
1751	-65.9367605146485\\
1752	-65.5520475563096\\
1753	-41.7442163833234\\
1754	-61.2502212036652\\
1755	-47.3426580651537\\
1756	-55.4836545054147\\
1757	-57.8915963058239\\
1758	-69.2000416522731\\
1759	-56.5893363423711\\
1760	-67.251121093394\\
1761	-78.5237992711625\\
1762	-69.7200551598155\\
1763	-62.3591736549677\\
1764	-54.4460086606832\\
1765	-60.9072772918748\\
1766	-54.5177989632746\\
1768	-45.0387967007935\\
1769	-50.3139072383572\\
1770	-47.1061697990542\\
1771	-75.4833174528385\\
1772	-95.0797764252136\\
1773	-79.7665060197271\\
1774	-74.5924668652547\\
1775	-75.7236061539927\\
1776	-52.6750373824916\\
1777	-47.3949805189234\\
1778	-40.6870580133486\\
1779	-36.5040751958218\\
1780	-40.3999831423928\\
1781	-55.9663367494279\\
1782	-65.5023513810208\\
1783	-70.9697069310739\\
1784	-62.0222730332941\\
1785	-48.7016802085566\\
1786	-75.5295608576939\\
1787	-83.1645427002306\\
1788	-89.7042307787092\\
1789	-77.9471850817752\\
1790	-108.059419121774\\
1791	-93.1953882875075\\
1792	-53.5692493368699\\
1793	-44.8948744070772\\
1794	-39.6675979081945\\
1795	-60.6113118556041\\
1796	-71.7663926579653\\
1797	-87.3200038958507\\
1798	-72.8003981998243\\
1799	-60.5161603759761\\
1800	-34.960507543472\\
1801	-37.8469353566363\\
1802	-41.4270907798268\\
1803	-46.6206580141218\\
1804	-33.4328285039894\\
1805	-40.571416270813\\
};
\addlegendentry{MPO prediction}

\end{axis}
\end{tikzpicture}%
%	\caption{Validation results for model estimated using LASSO.Compression direction d1.}\label{fig:ols_vs}
%\end{figure}
\begin{figure}[!h]
	\definecolor{mycolor1}{rgb}{0.00000,0.44700,0.74100}%
	\definecolor{mycolor2}{rgb}{0.85000,0.32500,0.09800}%
	\centering
	% This file was created by matlab2tikz.
%
\definecolor{mycolor1}{rgb}{0.00000,0.44700,0.74100}%
\definecolor{mycolor2}{rgb}{0.85000,0.32500,0.09800}%
\definecolor{mycolor3}{rgb}{0.92900,0.69400,0.12500}%
%
\begin{tikzpicture}

\begin{axis}[%
width=6.159cm,
height=1.831cm,
at={(0cm,10.169cm)},
scale only axis,
xmin=1000,
xmax=1405,
xlabel style={font=\color{white!15!black}},
xlabel={Sample index},
ymin=-58.594,
ymax=0,
ylabel style={font=\color{white!15!black}},
ylabel={$y(t)$},
axis background/.style={fill=white},
title style={font=\bfseries},
title={C1: RMSE(OSA) = 1.5371, RMSE(MPO) = 1.5952},
legend style={legend cell align=left, align=left, draw=white!15!black}
]
\addplot [color=mycolor1, line width=2.0pt]
  table[row sep=crcr]{%
1006	-28.076\\
1007	-32.9590000000001\\
1008	-25.635\\
1009	-14.6479999999999\\
1010	-20.752\\
1011	-17.0899999999999\\
1012	-18.3109999999999\\
1013	-20.752\\
1014	-7.32400000000007\\
1015	-4.88300000000004\\
1016	-4.88300000000004\\
1017	-13.4280000000001\\
1018	-23.193\\
1020	-25.635\\
1021	-19.5309999999999\\
1022	-12.2070000000001\\
1023	-24.414\\
1025	-14.6479999999999\\
1026	-20.752\\
1027	-18.3109999999999\\
1028	-17.0899999999999\\
1029	-29.297\\
1030	-25.635\\
1031	-18.3109999999999\\
1032	-28.076\\
1033	-28.076\\
1034	-21.973\\
1035	-18.3109999999999\\
1036	-15.8689999999999\\
1037	-15.8689999999999\\
1038	-21.973\\
1041	-21.973\\
1042	-23.193\\
1043	-30.518\\
1044	-29.297\\
1045	-19.5309999999999\\
1046	-18.3109999999999\\
1047	-10.9860000000001\\
1048	-14.6479999999999\\
1049	-15.8689999999999\\
1050	-12.2070000000001\\
1051	-13.4280000000001\\
1052	-18.3109999999999\\
1053	-19.5309999999999\\
1054	-18.3109999999999\\
1055	-29.297\\
1056	-21.973\\
1057	-12.2070000000001\\
1058	-12.2070000000001\\
1059	-13.4280000000001\\
1060	-17.0899999999999\\
1061	-9.76600000000008\\
1062	-10.9860000000001\\
1063	-14.6479999999999\\
1064	-9.76600000000008\\
1065	-7.32400000000007\\
1066	-13.4280000000001\\
1067	-13.4280000000001\\
1068	-21.973\\
1069	-20.752\\
1070	-25.635\\
1071	-21.973\\
1072	-25.635\\
1073	-20.752\\
1074	-20.752\\
1075	-18.3109999999999\\
1076	-18.3109999999999\\
1077	-17.0899999999999\\
1078	-30.518\\
1079	-41.5039999999999\\
1080	-41.5039999999999\\
1081	-42.7249999999999\\
1082	-28.076\\
1083	-37.8420000000001\\
1084	-46.3869999999999\\
1085	-46.3869999999999\\
1086	-35.4000000000001\\
1087	-47.607\\
1088	-58.5940000000001\\
1089	-43.9449999999999\\
1091	-24.414\\
1093	-17.0899999999999\\
1094	-21.973\\
1096	-12.2070000000001\\
1097	-9.76600000000008\\
1098	-12.2070000000001\\
1099	-18.3109999999999\\
1101	-18.3109999999999\\
1102	-21.973\\
1103	-23.193\\
1104	-30.518\\
1105	-28.076\\
1106	-30.518\\
1108	-20.752\\
1109	-24.414\\
1110	-18.3109999999999\\
1111	-13.4280000000001\\
1112	-19.5309999999999\\
1113	-20.752\\
1114	-12.2070000000001\\
1115	-15.8689999999999\\
1116	-12.2070000000001\\
1117	-13.4280000000001\\
1118	-13.4280000000001\\
1119	-9.76600000000008\\
1120	-10.9860000000001\\
1122	-23.193\\
1123	-25.635\\
1124	-26.855\\
1125	-15.8689999999999\\
1126	-20.752\\
1127	-32.9590000000001\\
1128	-24.414\\
1129	-28.076\\
1130	-29.297\\
1131	-18.3109999999999\\
1132	-12.2070000000001\\
1133	-17.0899999999999\\
1134	-18.3109999999999\\
1135	-30.518\\
1136	-36.6210000000001\\
1137	-36.6210000000001\\
1138	-28.076\\
1139	-28.076\\
1140	-25.635\\
1142	-25.635\\
1144	-15.8689999999999\\
1146	-13.4280000000001\\
1147	-13.4280000000001\\
1148	-20.752\\
1149	-31.7380000000001\\
1150	-26.855\\
1151	-17.0899999999999\\
1152	-15.8689999999999\\
1153	-15.8689999999999\\
1154	-9.76600000000008\\
1155	-7.32400000000007\\
1156	-7.32400000000007\\
1157	-15.8689999999999\\
1158	-10.9860000000001\\
1159	-14.6479999999999\\
1160	-14.6479999999999\\
1161	-13.4280000000001\\
1163	-6.10400000000004\\
1164	-4.88300000000004\\
1165	-7.32400000000007\\
1166	-18.3109999999999\\
1167	-26.855\\
1168	-28.076\\
1170	-13.4280000000001\\
1171	-10.9860000000001\\
1172	-6.10400000000004\\
1173	-12.2070000000001\\
1174	-14.6479999999999\\
1175	-18.3109999999999\\
1176	-26.855\\
1177	-39.0630000000001\\
1178	-47.607\\
1180	-37.8420000000001\\
1181	-28.076\\
1182	-30.518\\
1183	-31.7380000000001\\
1184	-34.1800000000001\\
1185	-24.414\\
1186	-23.193\\
1187	-23.193\\
1188	-21.973\\
1189	-23.193\\
1190	-18.3109999999999\\
1191	-30.518\\
1192	-37.8420000000001\\
1194	-18.3109999999999\\
1195	-18.3109999999999\\
1196	-30.518\\
1199	-45.1659999999999\\
1200	-45.1659999999999\\
1201	-34.1800000000001\\
1202	-32.9590000000001\\
1203	-36.6210000000001\\
1204	-41.5039999999999\\
1206	-17.0899999999999\\
1207	-23.193\\
1208	-20.752\\
1209	-13.4280000000001\\
1210	-18.3109999999999\\
1211	-19.5309999999999\\
1212	-14.6479999999999\\
1213	-13.4280000000001\\
1214	-15.8689999999999\\
1215	-13.4280000000001\\
1216	-17.0899999999999\\
1217	-25.635\\
1218	-25.635\\
1219	-15.8689999999999\\
1220	-14.6479999999999\\
1221	-24.414\\
1222	-40.2829999999999\\
1223	-29.297\\
1224	-20.752\\
1225	-14.6479999999999\\
1226	-18.3109999999999\\
1227	-15.8689999999999\\
1228	-12.2070000000001\\
1229	-13.4280000000001\\
1232	-20.752\\
1233	-15.8689999999999\\
1234	-23.193\\
1235	-29.297\\
1236	-25.635\\
1237	-20.752\\
1238	-23.193\\
1239	-30.518\\
1240	-21.973\\
1241	-8.54500000000007\\
1242	-13.4280000000001\\
1243	-13.4280000000001\\
1244	-17.0899999999999\\
1245	-17.0899999999999\\
1246	-13.4280000000001\\
1247	-17.0899999999999\\
1248	-19.5309999999999\\
1249	-13.4280000000001\\
1250	-15.8689999999999\\
1251	-15.8689999999999\\
1252	-12.2070000000001\\
1253	-15.8689999999999\\
1254	-9.76600000000008\\
1255	-10.9860000000001\\
1256	-8.54500000000007\\
1257	-8.54500000000007\\
1259	-20.752\\
1260	-24.414\\
1261	-35.4000000000001\\
1263	-13.4280000000001\\
1265	-10.9860000000001\\
1266	-13.4280000000001\\
1267	-10.9860000000001\\
1268	-12.2070000000001\\
1269	-14.6479999999999\\
1270	-24.414\\
1271	-21.973\\
1272	-26.855\\
1273	-23.193\\
1274	-17.0899999999999\\
1275	-13.4280000000001\\
1278	-6.10400000000004\\
1279	-9.76600000000008\\
1280	-14.6479999999999\\
1281	-18.3109999999999\\
1282	-18.3109999999999\\
1283	-19.5309999999999\\
1284	-29.297\\
1285	-25.635\\
1286	-18.3109999999999\\
1287	-24.414\\
1288	-24.414\\
1289	-31.7380000000001\\
1290	-26.855\\
1291	-17.0899999999999\\
1292	-9.76600000000008\\
1293	-6.10400000000004\\
1294	-7.32400000000007\\
1295	-9.76600000000008\\
1296	-8.54500000000007\\
1297	-8.54500000000007\\
1298	-12.2070000000001\\
1299	-17.0899999999999\\
1300	-15.8689999999999\\
1301	-15.8689999999999\\
1302	-19.5309999999999\\
1303	-13.4280000000001\\
1304	-4.88300000000004\\
1305	-9.76600000000008\\
1306	-21.973\\
1307	-19.5309999999999\\
1308	-21.973\\
1310	-14.6479999999999\\
1311	-19.5309999999999\\
1312	-25.635\\
1313	-25.635\\
1314	-19.5309999999999\\
1315	-26.855\\
1316	-26.855\\
1317	-23.193\\
1318	-28.076\\
1319	-26.855\\
1320	-20.752\\
1321	-19.5309999999999\\
1322	-29.297\\
1323	-40.2829999999999\\
1324	-30.518\\
1325	-18.3109999999999\\
1326	-17.0899999999999\\
1327	-18.3109999999999\\
1328	-21.973\\
1329	-26.855\\
1330	-19.5309999999999\\
1331	-19.5309999999999\\
1332	-26.855\\
1333	-28.076\\
1334	-20.752\\
1335	-14.6479999999999\\
1336	-20.752\\
1337	-34.1800000000001\\
1338	-30.518\\
1339	-31.7380000000001\\
1340	-21.973\\
1341	-17.0899999999999\\
1342	-17.0899999999999\\
1343	-10.9860000000001\\
1344	-6.10400000000004\\
1345	-6.10400000000004\\
1346	-4.88300000000004\\
1347	-10.9860000000001\\
1348	-19.5309999999999\\
1349	-18.3109999999999\\
1350	-15.8689999999999\\
1351	-14.6479999999999\\
1352	-20.752\\
1353	-14.6479999999999\\
1354	-9.76600000000008\\
1355	-13.4280000000001\\
1356	-8.54500000000007\\
1357	-10.9860000000001\\
1358	-18.3109999999999\\
1359	-23.193\\
1360	-17.0899999999999\\
1361	-9.76600000000008\\
1363	-12.2070000000001\\
1364	-9.76600000000008\\
1365	-13.4280000000001\\
1366	-31.7380000000001\\
1367	-25.635\\
1369	-37.8420000000001\\
1370	-36.6210000000001\\
1371	-26.855\\
1372	-20.752\\
1373	-18.3109999999999\\
1374	-21.973\\
1375	-23.193\\
1376	-29.297\\
1377	-29.297\\
1378	-36.6210000000001\\
1379	-42.7249999999999\\
1380	-46.3869999999999\\
1381	-37.8420000000001\\
1382	-35.4000000000001\\
1383	-40.2829999999999\\
1384	-31.7380000000001\\
1385	-34.1800000000001\\
1386	-31.7380000000001\\
1387	-18.3109999999999\\
1388	-12.2070000000001\\
1389	-13.4280000000001\\
1390	-18.3109999999999\\
1391	-17.0899999999999\\
1392	-8.54500000000007\\
1393	-14.6479999999999\\
1394	-13.4280000000001\\
1395	-6.10400000000004\\
1396	-10.9860000000001\\
1397	-12.2070000000001\\
1398	-7.32400000000007\\
1399	-18.3109999999999\\
1400	-20.752\\
1401	-14.6479999999999\\
1402	-23.193\\
1403	-37.8420000000001\\
1404	-32.9590000000001\\
1405	-37.8420000000001\\
};
\addlegendentry{True output}

\addplot [color=mycolor2, dashed, line width=2.0pt]
  table[row sep=crcr]{%
1006	-28.3734904745222\\
1007	-30.8287513378341\\
1008	-26.7121079728224\\
1009	-13.5860530618645\\
1010	-16.4336285164409\\
1011	-18.0813431859847\\
1012	-21.1232930613435\\
1013	-19.1577704514355\\
1014	-7.12390642486844\\
1015	-5.89416150748207\\
1016	-4.81634156735004\\
1017	-14.2923236455918\\
1018	-24.4340855194887\\
1019	-24.2896482113324\\
1020	-25.4344269145213\\
1021	-18.7848558530757\\
1022	-11.553106580503\\
1023	-24.6619604734353\\
1024	-19.5118651615578\\
1025	-12.1583932123615\\
1026	-20.6764482067044\\
1027	-21.1487620215053\\
1028	-16.4293698150568\\
1029	-25.4821520580792\\
1030	-24.6880878948268\\
1031	-20.050811105916\\
1032	-26.6866423022043\\
1033	-26.9231510055297\\
1034	-22.0993977449975\\
1035	-19.1342961391426\\
1036	-14.9651186829628\\
1038	-20.9312408058815\\
1039	-19.5796477019251\\
1040	-19.9790542854985\\
1041	-22.3410282906068\\
1042	-22.6502575932\\
1043	-31.4637410557232\\
1044	-27.5799406802948\\
1045	-19.7887899257832\\
1046	-19.1079831019179\\
1047	-12.2600344778321\\
1048	-10.211058775888\\
1049	-15.8359702108485\\
1050	-13.5916344114144\\
1051	-13.0736881828668\\
1052	-17.551926007214\\
1053	-20.049261301757\\
1054	-18.614860547247\\
1055	-27.0069650246455\\
1056	-21.9622351700104\\
1057	-13.4646403733109\\
1058	-11.299285619468\\
1059	-15.0874893888686\\
1060	-16.7714543805246\\
1061	-8.87734730713419\\
1062	-9.25073574906196\\
1063	-13.6073877810256\\
1064	-12.0643678263325\\
1065	-9.21313840836592\\
1066	-11.9793485331149\\
1067	-13.2637781097189\\
1068	-21.5744478783779\\
1069	-20.8368324837097\\
1070	-22.9316748597273\\
1071	-22.5789799784959\\
1072	-23.887554364278\\
1073	-21.1332431936823\\
1074	-19.8628635443345\\
1075	-19.3603299077945\\
1076	-17.1822750039773\\
1077	-18.4555310410824\\
1078	-27.6438822783884\\
1079	-41.4872978384326\\
1080	-41.663147689078\\
1081	-39.1598254006401\\
1082	-25.4664889403091\\
1083	-37.1776638025449\\
1084	-46.5674243686324\\
1085	-44.674657970316\\
1086	-36.3591097907001\\
1087	-44.2098118473284\\
1088	-57.8726719565182\\
1089	-44.3530782641892\\
1090	-36.8430882184132\\
1092	-19.8885733781146\\
1093	-16.7602652398009\\
1094	-20.7097619250392\\
1095	-16.6886813190852\\
1096	-10.0350916249524\\
1097	-9.57115150001209\\
1098	-13.2011305273475\\
1099	-19.1265693304372\\
1100	-17.9330466353567\\
1101	-17.7008252391431\\
1102	-23.0494295782341\\
1103	-23.9397172510664\\
1104	-32.2659562045319\\
1105	-28.1555887451022\\
1106	-29.8460155965849\\
1107	-24.6932447366862\\
1108	-20.6395574975229\\
1109	-25.1871362541788\\
1110	-19.019441549427\\
1111	-17.2232048591584\\
1112	-19.7558776348164\\
1113	-19.8163044803612\\
1114	-15.9581525250644\\
1115	-15.9323526483511\\
1116	-12.144993909862\\
1117	-12.5804710579657\\
1118	-15.1249914453597\\
1119	-11.2907622353896\\
1120	-12.1671383944458\\
1121	-15.1767684880781\\
1122	-23.5477613978094\\
1123	-24.9653360929092\\
1124	-26.1741977523131\\
1125	-14.9403299118285\\
1126	-20.6440285859205\\
1127	-34.7936671394118\\
1128	-25.6346317169111\\
1129	-27.295552669377\\
1130	-28.5048015010721\\
1131	-19.356061016809\\
1132	-12.5604735106426\\
1133	-17.14727820704\\
1134	-20.371942345497\\
1135	-28.8447478384307\\
1136	-37.6787992539059\\
1137	-36.5143822319253\\
1138	-26.998880117362\\
1139	-28.6151753453066\\
1140	-26.5384323983149\\
1141	-25.8640521851253\\
1142	-25.9912332235572\\
1143	-17.8807469078511\\
1144	-15.4500546384784\\
1145	-16.5252719924235\\
1146	-14.4612411848991\\
1147	-15.8304800398539\\
1148	-21.3071987779908\\
1149	-33.16482562691\\
1150	-25.9664576571654\\
1151	-17.5378208259649\\
1152	-16.1286679049006\\
1153	-16.1272101312848\\
1154	-10.4719565172957\\
1155	-6.65352125594086\\
1156	-7.86547244433814\\
1157	-15.197554415673\\
1158	-12.9068082398442\\
1159	-12.0243019784473\\
1160	-13.9598342633092\\
1161	-13.4777854094646\\
1162	-10.6434629914393\\
1163	-7.69269312234223\\
1164	-5.40534434077404\\
1165	-7.76840184679759\\
1166	-19.2589231680308\\
1167	-26.3807712287783\\
1168	-29.4670154225987\\
1169	-18.3253840258817\\
1170	-13.6681767217053\\
1171	-11.0534471452206\\
1172	-7.1983055817866\\
1173	-14.0106405580775\\
1174	-15.6093074637661\\
1175	-19.3388389635174\\
1176	-25.5547622978822\\
1177	-40.2919235994832\\
1178	-50.6462494099057\\
1179	-40.4591568557278\\
1180	-38.2521870585906\\
1181	-27.9561720414308\\
1182	-28.9788439744154\\
1183	-31.3852680178886\\
1184	-32.0426824993347\\
1185	-27.3473067966593\\
1186	-24.0386732865095\\
1187	-25.0171273931139\\
1188	-21.2162376776892\\
1189	-24.4373462552892\\
1190	-19.8388769655089\\
1191	-32.2126922913014\\
1192	-36.9670129953731\\
1194	-16.7846303547315\\
1195	-18.7889897290086\\
1196	-30.0922300541915\\
1197	-36.0356873748003\\
1198	-43.3754448506274\\
1199	-44.2601502931061\\
1200	-43.8433025751171\\
1201	-35.9771496010308\\
1202	-33.5413144851714\\
1203	-36.5395254692487\\
1204	-40.2503735975092\\
1205	-27.7346793725981\\
1206	-17.0542918594106\\
1207	-25.368751223338\\
1208	-22.1001879739708\\
1209	-12.8425009865582\\
1210	-18.6257712327501\\
1211	-21.7168705251172\\
1212	-17.4986848472299\\
1213	-13.1055601490943\\
1214	-16.4154825193959\\
1215	-15.4533115045867\\
1216	-18.2130948104254\\
1217	-27.094855377047\\
1218	-22.7320152221307\\
1219	-17.1733291769071\\
1220	-18.1795136160761\\
1221	-27.0753658943606\\
1222	-39.3273444268716\\
1223	-30.5914201436449\\
1224	-20.8874705321916\\
1225	-15.9146779876273\\
1226	-16.7634362087595\\
1227	-17.0579855596332\\
1228	-12.9274013270326\\
1229	-13.8517506250075\\
1230	-17.548319302515\\
1231	-20.6230713039779\\
1232	-21.5626197591603\\
1233	-14.8738728806456\\
1234	-24.442222278595\\
1235	-29.7025087062848\\
1236	-26.4792775064634\\
1237	-21.6090179836615\\
1238	-23.8942292084344\\
1239	-30.146056957492\\
1240	-20.0118417906897\\
1241	-9.27515620902682\\
1242	-12.1763320351895\\
1243	-15.4682784500972\\
1244	-19.0656580796535\\
1245	-18.8831876408105\\
1246	-12.8808376890095\\
1247	-18.9913190432901\\
1248	-19.6012046675585\\
1249	-14.0541239748757\\
1250	-17.1085515875066\\
1251	-15.5361153537049\\
1252	-14.3480370757043\\
1253	-17.1412068272161\\
1254	-9.35760985095067\\
1255	-9.78098234425352\\
1256	-9.37645000570137\\
1257	-9.69561230735121\\
1258	-17.9513028719266\\
1259	-19.8022965428906\\
1260	-25.8195502957617\\
1261	-33.4090004870184\\
1262	-23.0633979145543\\
1263	-14.3319388962777\\
1264	-12.1139521881039\\
1265	-11.3210518378071\\
1266	-15.259749857732\\
1267	-10.8789682968479\\
1268	-10.4752345999107\\
1269	-14.6980910838872\\
1270	-23.4875869202933\\
1271	-23.2418204727751\\
1272	-29.2117190514712\\
1273	-20.2162192762821\\
1274	-18.4100949022863\\
1275	-10.7850011460291\\
1276	-9.59907987039787\\
1277	-7.93921036100596\\
1278	-7.82661413250548\\
1279	-9.01481597093243\\
1280	-15.8063288661783\\
1282	-18.1859445268478\\
1283	-18.7873950526659\\
1284	-31.0204362667009\\
1285	-29.2699285105048\\
1286	-19.3799561617495\\
1287	-23.7524939890691\\
1288	-24.4164777265175\\
1289	-29.4630692980734\\
1290	-24.4988167996505\\
1291	-18.0714004227893\\
1292	-9.59876371476093\\
1293	-7.05150429832247\\
1294	-6.86166591393658\\
1295	-8.85487954294445\\
1296	-9.43684186697033\\
1297	-8.77660647550147\\
1298	-12.0581378854527\\
1299	-16.4928282942301\\
1300	-15.6974918155383\\
1301	-16.4648528523458\\
1302	-18.0254294156448\\
1303	-11.2135219762936\\
1304	-7.32467002557655\\
1305	-12.8492013298633\\
1306	-20.4750076242005\\
1307	-17.6383285684201\\
1308	-19.9257246624034\\
1309	-19.6334946246332\\
1310	-14.299368560942\\
1311	-18.2184778709916\\
1312	-25.4591182499644\\
1313	-25.884626520753\\
1314	-18.18309449507\\
1315	-30.2152132645995\\
1316	-25.2977119323\\
1317	-20.9476824429464\\
1318	-26.2029926715102\\
1319	-25.3353197838683\\
1320	-21.6203716882465\\
1321	-18.5647320517821\\
1322	-26.6895281615232\\
1323	-40.7096719378258\\
1324	-32.1691391459335\\
1325	-17.8863524552244\\
1327	-18.6180652527789\\
1328	-23.0049684708645\\
1329	-26.3586242479685\\
1330	-19.2846232323834\\
1331	-21.4022997257671\\
1332	-27.0038235884988\\
1333	-28.5187531682591\\
1334	-19.5208877363305\\
1335	-16.5513383690313\\
1336	-20.9125555894641\\
1337	-32.2384733965296\\
1338	-31.8520733260107\\
1339	-28.7846047373655\\
1340	-19.4935182464042\\
1341	-17.7474201862035\\
1342	-18.8642780857547\\
1343	-12.6582552036523\\
1344	-6.84733769662648\\
1345	-5.07042597523309\\
1346	-4.51194303372677\\
1347	-11.1525463318685\\
1348	-18.0826056199996\\
1349	-19.6729734851026\\
1350	-15.7392099090976\\
1351	-15.5028365739188\\
1352	-18.4453723178985\\
1353	-11.545902890196\\
1354	-12.6190828977556\\
1355	-13.5696886400683\\
1356	-10.0262609571171\\
1357	-11.1951953795606\\
1358	-19.0305195629476\\
1359	-23.5409999207359\\
1360	-14.3074673522888\\
1361	-11.508646008492\\
1362	-10.3399140963884\\
1363	-12.172118100038\\
1364	-9.22913847192717\\
1365	-17.916195830911\\
1366	-31.2288974184771\\
1367	-22.31663814546\\
1368	-29.3896318940351\\
1369	-37.1853745077678\\
1370	-35.2015703683298\\
1371	-29.4353931269513\\
1372	-20.798586313435\\
1373	-21.136418871562\\
1374	-21.4378288486121\\
1375	-23.8346032765978\\
1376	-26.9499708127048\\
1377	-26.8944753435931\\
1378	-35.5778076175479\\
1379	-42.095362875312\\
1380	-47.0903161750061\\
1381	-34.9726237493214\\
1382	-35.3624369487741\\
1383	-42.0097922391353\\
1384	-31.4048940285238\\
1385	-36.1853532138796\\
1386	-32.3519656528199\\
1387	-18.5443235122784\\
1388	-11.6593703968015\\
1389	-11.8209731979011\\
1390	-19.198806513959\\
1391	-16.3935305453624\\
1392	-10.905345011651\\
1393	-14.5428818695389\\
1394	-12.707902249203\\
1395	-8.40051222832403\\
1396	-10.8032991266375\\
1397	-12.3857056300174\\
1398	-9.82798759240973\\
1399	-14.7556507508116\\
1400	-20.8437381223227\\
1401	-17.1254003684314\\
1402	-25.760176769101\\
1403	-38.2576352694002\\
1404	-33.9709693295435\\
1405	-41.5069219498037\\
};
\addlegendentry{OSA predition}

\addplot [color=mycolor3, dotted, line width=2.0pt]
  table[row sep=crcr]{%
1006	-28.076\\
1007	-32.9590000000001\\
1008	-25.635\\
1009	-14.6479999999999\\
1010	-16.4336285164411\\
1011	-17.7144588465353\\
1012	-20.2025659964947\\
1013	-18.85949493576\\
1014	-7.29025560289961\\
1015	-5.6372418921444\\
1016	-4.76597514576815\\
1017	-14.3725898254361\\
1018	-24.6801550278421\\
1019	-24.5887177708835\\
1020	-25.9104105278475\\
1021	-19.0524843516321\\
1022	-11.6413684831164\\
1023	-24.5544940179636\\
1024	-19.3417782075783\\
1025	-12.1035748919921\\
1026	-20.4036385625691\\
1027	-20.5261733037967\\
1028	-16.2231839538847\\
1029	-25.7518034659938\\
1030	-24.5857167647321\\
1031	-19.1549948366717\\
1032	-25.890280973053\\
1033	-26.6774396370388\\
1034	-21.5850500272804\\
1035	-18.533994709901\\
1036	-14.6588471849263\\
1038	-20.8450631717931\\
1039	-19.7448277610233\\
1040	-19.7761157890868\\
1041	-21.5524058391782\\
1042	-21.7526774952476\\
1043	-30.7538946358261\\
1044	-27.12890032156\\
1045	-19.5178744036634\\
1046	-18.5805681529498\\
1047	-12.0610411741116\\
1048	-10.2999750131019\\
1049	-15.7150468010727\\
1050	-12.7602215650281\\
1051	-12.5850397400477\\
1052	-17.4265528166513\\
1053	-19.8736230341683\\
1054	-18.355039382159\\
1055	-26.8913231470342\\
1056	-21.8287998214644\\
1057	-12.9337802330333\\
1058	-11.0772729670432\\
1059	-15.0472760567952\\
1060	-16.7695068937683\\
1061	-9.12195192292688\\
1062	-9.25880569868332\\
1063	-13.3126163255295\\
1064	-11.4919311390458\\
1065	-8.8983389720579\\
1066	-12.2608725769464\\
1067	-13.6757045500985\\
1068	-21.7134912400659\\
1069	-20.7413352377007\\
1070	-22.876999501417\\
1071	-22.3395582685648\\
1072	-23.2912366845335\\
1073	-20.6661377930202\\
1074	-19.2822705269921\\
1075	-18.9114699474505\\
1076	-16.8114406779296\\
1077	-18.250531259134\\
1078	-27.4722300328167\\
1079	-41.1956915317974\\
1080	-41.1644074203641\\
1081	-38.5940536777309\\
1082	-25.1168472048248\\
1083	-35.8790023328636\\
1084	-44.8651359524126\\
1085	-43.4739436188563\\
1086	-35.5468151024706\\
1087	-43.1316989800996\\
1088	-57.0964131712067\\
1089	-43.2245950641898\\
1090	-35.8154610926683\\
1091	-27.9854713715567\\
1092	-20.4204965761562\\
1093	-17.8229101224906\\
1094	-21.2905963756875\\
1095	-16.8850867635163\\
1096	-9.91859826133555\\
1097	-9.25562305568428\\
1098	-12.5777851223022\\
1099	-18.6659377689004\\
1100	-17.8937056232514\\
1101	-17.7766342350812\\
1102	-22.9821705734839\\
1103	-23.8452496271252\\
1104	-32.4196876068588\\
1105	-28.5554898695475\\
1106	-30.4562830754653\\
1107	-25.0879823207763\\
1108	-20.7096310517366\\
1109	-25.0598877964183\\
1110	-18.9867640200723\\
1111	-17.4058828284192\\
1112	-20.3633106319396\\
1113	-20.9156181689318\\
1114	-16.669694798429\\
1115	-16.6156161290314\\
1116	-13.2743204904839\\
1117	-13.4103434305835\\
1118	-15.6133386239069\\
1119	-11.625756183955\\
1120	-12.8514088189891\\
1121	-16.0715160847969\\
1122	-24.2864108598064\\
1123	-25.2113864991841\\
1124	-26.2754435839247\\
1125	-14.8738666867737\\
1126	-20.3952875420168\\
1127	-34.3977200104205\\
1128	-25.4718745090884\\
1129	-27.607410055492\\
1130	-28.9446899051809\\
1131	-19.3934761728349\\
1132	-12.5581187045946\\
1133	-17.3695586928613\\
1134	-20.6308518452906\\
1135	-29.2068581700771\\
1136	-38.1881363546709\\
1137	-36.7194682449183\\
1138	-27.2766046997942\\
1139	-28.7825260393831\\
1140	-26.4570933792381\\
1141	-25.9737999207471\\
1142	-26.2942357507002\\
1143	-18.1277737276741\\
1144	-15.4542105069545\\
1145	-15.9671205137267\\
1146	-14.1819698153479\\
1147	-16.0511693704323\\
1148	-21.9196519982108\\
1149	-34.0518760059235\\
1150	-26.7629961941132\\
1151	-18.2659690502692\\
1152	-16.5315367184674\\
1153	-16.5276800715556\\
1154	-10.7986494649515\\
1155	-6.99903008398269\\
1156	-8.16044003316006\\
1157	-15.3922041679307\\
1158	-13.0322482769768\\
1159	-12.2085609049998\\
1160	-14.1378713770171\\
1161	-13.1477511192938\\
1162	-10.3179572152389\\
1163	-7.55585483693812\\
1164	-5.62247989770299\\
1165	-8.20227614828946\\
1166	-19.8096426361926\\
1167	-26.9056097368868\\
1168	-29.9744405245092\\
1169	-18.641910661883\\
1170	-13.9787979455709\\
1171	-10.8165596542844\\
1172	-7.17407802508455\\
1173	-14.0899090834209\\
1174	-16.0051136801883\\
1175	-20.0263502516202\\
1176	-26.3825211462042\\
1177	-41.0400077461338\\
1178	-51.1106315522495\\
1179	-40.8862738415989\\
1180	-39.3504990973922\\
1181	-28.0805937403109\\
1182	-29.1899409265911\\
1183	-31.4947401014713\\
1184	-31.7585590033466\\
1185	-26.9647102207728\\
1186	-23.5596637888739\\
1187	-25.282133849011\\
1188	-21.732171869691\\
1189	-25.0550949338544\\
1190	-20.2097800373685\\
1191	-32.8280162211893\\
1192	-37.9153566500236\\
1193	-27.7371709648642\\
1194	-17.0743979029858\\
1195	-18.6862858706827\\
1196	-29.7663499291225\\
1197	-35.8110560410721\\
1198	-43.2650735722355\\
1199	-44.303017107723\\
1200	-44.6038411615593\\
1201	-36.2233877767824\\
1202	-33.4649667400026\\
1203	-36.9552587430608\\
1204	-40.7611574691125\\
1205	-27.9294460790832\\
1206	-16.8316318939576\\
1207	-24.8942696686404\\
1208	-21.9459914034355\\
1209	-13.2637934239649\\
1210	-19.1412647431564\\
1211	-21.9913243143451\\
1212	-17.9052461779768\\
1213	-14.0433607890295\\
1214	-17.6021474275267\\
1215	-16.253073637017\\
1216	-19.0543995985001\\
1217	-28.2509944929147\\
1218	-23.8207614344515\\
1219	-17.947907302517\\
1220	-18.3563946069762\\
1221	-27.6387924391879\\
1222	-40.8357334041411\\
1223	-31.9784715636019\\
1224	-21.6490549039472\\
1225	-16.8106878257174\\
1226	-17.5530869139286\\
1227	-17.7111764573854\\
1228	-13.1905485499767\\
1229	-14.332054553603\\
1230	-18.1020458651117\\
1231	-21.2371543096594\\
1232	-22.4716427414317\\
1233	-15.9437213509386\\
1234	-25.4326000477483\\
1235	-30.3037310634738\\
1236	-27.1108101678346\\
1237	-22.2377437169091\\
1238	-24.5880953502547\\
1239	-30.9015454908363\\
1240	-20.583159202067\\
1241	-9.40447431029133\\
1242	-12.0234238073131\\
1243	-15.3596852309372\\
1244	-19.0264301788613\\
1245	-19.2872211077647\\
1246	-13.6531978993037\\
1247	-19.8707463246617\\
1248	-20.2761537440322\\
1249	-14.8231206800451\\
1250	-17.8060184079332\\
1251	-16.2069955312581\\
1252	-15.0057754521345\\
1253	-17.8003823575139\\
1254	-10.2387812882248\\
1255	-10.5816892129392\\
1256	-9.74825556827113\\
1257	-9.84404533984298\\
1258	-18.2749196135446\\
1259	-20.5813553115338\\
1260	-26.8554589722364\\
1261	-34.3595876881659\\
1262	-23.6527703597951\\
1263	-14.2501882666343\\
1264	-11.976557976993\\
1265	-11.3669059187639\\
1266	-15.3367169610401\\
1267	-11.1372750879475\\
1268	-10.9338136701037\\
1269	-14.8962509642581\\
1270	-23.3318519925706\\
1271	-23.0089217249949\\
1272	-29.0654600177659\\
1273	-20.4737757683972\\
1274	-18.7896975693918\\
1275	-10.5413711441693\\
1276	-9.50055712042877\\
1277	-7.22480844991946\\
1278	-7.13052589548624\\
1279	-8.56854828849623\\
1280	-15.6268367372443\\
1281	-16.9677336047998\\
1282	-18.1186203986024\\
1283	-18.5875969093311\\
1284	-30.6453799748397\\
1285	-29.0696413425519\\
1286	-19.8015750032832\\
1287	-24.8505051318571\\
1288	-25.3747833910111\\
1289	-30.0156526471785\\
1290	-24.7270443799662\\
1291	-17.6421796934951\\
1292	-9.02879378444732\\
1293	-6.84885007970774\\
1294	-6.75435358185518\\
1295	-8.89087771574918\\
1296	-9.300622738627\\
1297	-8.63302380875302\\
1298	-12.07845607374\\
1299	-16.5926609787439\\
1300	-15.6652030929351\\
1301	-16.3283369068745\\
1302	-17.9461502737497\\
1303	-11.1420542738363\\
1304	-6.79932022493563\\
1305	-12.3340925970488\\
1306	-20.6567096594572\\
1307	-18.2605409077369\\
1308	-19.9989616341752\\
1309	-19.0991268348691\\
1310	-13.7504734229149\\
1311	-17.9799796360344\\
1312	-25.2038758347132\\
1313	-25.392672569649\\
1314	-17.8650427562338\\
1315	-29.8846297420589\\
1316	-25.0653501235981\\
1317	-21.2751589854395\\
1318	-26.0735263564529\\
1319	-24.5847923877686\\
1320	-20.6854187289359\\
1321	-17.7177706351326\\
1322	-26.0573340374935\\
1323	-39.9832838136638\\
1324	-31.1463516872068\\
1325	-17.5085327535839\\
1326	-18.1863502159374\\
1327	-18.5963155524446\\
1328	-23.1720095274891\\
1329	-26.679396222738\\
1330	-19.6155352520861\\
1331	-21.5536855725595\\
1332	-27.1989604002035\\
1333	-29.0149828725321\\
1334	-19.9007005146191\\
1335	-16.8089031981456\\
1336	-21.057324069777\\
1337	-32.6183484268086\\
1338	-32.1994541388226\\
1339	-28.6457984536903\\
1340	-19.5327923147236\\
1341	-16.9616430568924\\
1342	-17.9029396444753\\
1343	-12.3318027743353\\
1344	-7.0648362763327\\
1345	-5.54010715136087\\
1346	-4.8207670929462\\
1347	-11.2147389111674\\
1348	-18.0187861150005\\
1349	-19.5602080394935\\
1350	-15.5593033279843\\
1351	-15.557698746961\\
1352	-18.5931637690842\\
1353	-11.5758229439548\\
1354	-11.964730514572\\
1355	-12.8616682385941\\
1356	-10.0459680011602\\
1357	-11.3908767450459\\
1358	-19.3624800470332\\
1359	-23.9773506533229\\
1360	-14.6785059361066\\
1361	-11.5856670980163\\
1362	-10.0884642142191\\
1363	-12.2042445644026\\
1364	-9.1413995562782\\
1365	-17.7598200201642\\
1366	-31.4526084613242\\
1367	-23.1100683208431\\
1369	-36.5808775788919\\
1370	-34.2834848543127\\
1371	-28.6859715006531\\
1372	-20.1553296158811\\
1373	-21.1594647485774\\
1374	-21.7217555695242\\
1375	-24.4388042698802\\
1376	-27.449657684524\\
1377	-27.1134592910012\\
1378	-35.2399022135858\\
1379	-41.177867793122\\
1380	-46.1965446407098\\
1381	-34.325772384632\\
1382	-34.8601352750179\\
1383	-41.0118329894212\\
1384	-30.7807821607951\\
1385	-36.095311549632\\
1386	-32.3047058923742\\
1387	-18.967973678157\\
1388	-12.0199924847236\\
1389	-12.098981311821\\
1390	-19.1538753254242\\
1391	-16.1645886480528\\
1392	-10.8694727856223\\
1393	-14.6271327890802\\
1394	-13.1418521555865\\
1395	-8.56167088713505\\
1396	-11.053769667499\\
1397	-12.8867107906938\\
1398	-10.1645186329088\\
1399	-15.2641393385366\\
1400	-21.3024864199438\\
1401	-16.9171812987233\\
1402	-25.6047122779742\\
1403	-38.9704777206937\\
1404	-34.9748398418565\\
1405	-42.4420780085477\\
};
\addlegendentry{MPO prediction}

\end{axis}

\begin{axis}[%
width=6.159cm,
height=1.831cm,
at={(8.104cm,10.169cm)},
scale only axis,
xmin=1000,
xmax=1405,
xlabel style={font=\color{white!15!black}},
xlabel={Sample index},
ymin=-50,
ymax=0,
ylabel style={font=\color{white!15!black}},
ylabel={$y(t)$},
axis background/.style={fill=white},
title style={font=\bfseries},
title={C2: RMSE(OSA) = 1.5452, RMSE(MPO) = 1.5631},
legend style={legend cell align=left, align=left, draw=white!15!black}
]
\addplot [color=mycolor1, line width=2.0pt]
  table[row sep=crcr]{%
1006	-24.414\\
1007	-28.076\\
1008	-23.193\\
1009	-13.4280000000001\\
1010	-17.0899999999999\\
1011	-17.0899999999999\\
1012	-18.3109999999999\\
1013	-15.8689999999999\\
1014	-8.54500000000007\\
1015	-6.10400000000004\\
1016	-7.32400000000007\\
1017	-9.76600000000008\\
1018	-21.973\\
1019	-23.193\\
1020	-23.193\\
1021	-19.5309999999999\\
1022	-10.9860000000001\\
1023	-17.0899999999999\\
1024	-19.5309999999999\\
1025	-13.4280000000001\\
1026	-18.3109999999999\\
1027	-19.5309999999999\\
1028	-14.6479999999999\\
1029	-24.414\\
1030	-21.973\\
1031	-15.8689999999999\\
1032	-23.193\\
1033	-25.635\\
1034	-19.5309999999999\\
1036	-14.6479999999999\\
1037	-17.0899999999999\\
1038	-20.752\\
1039	-18.3109999999999\\
1040	-18.3109999999999\\
1042	-20.752\\
1043	-28.076\\
1044	-25.635\\
1045	-17.0899999999999\\
1046	-15.8689999999999\\
1047	-12.2070000000001\\
1048	-10.9860000000001\\
1049	-15.8689999999999\\
1050	-12.2070000000001\\
1051	-12.2070000000001\\
1052	-15.8689999999999\\
1053	-18.3109999999999\\
1054	-15.8689999999999\\
1055	-25.635\\
1056	-21.973\\
1057	-12.2070000000001\\
1058	-9.76600000000008\\
1059	-13.4280000000001\\
1060	-15.8689999999999\\
1061	-10.9860000000001\\
1062	-10.9860000000001\\
1063	-12.2070000000001\\
1064	-9.76600000000008\\
1065	-8.54500000000007\\
1066	-12.2070000000001\\
1068	-17.0899999999999\\
1069	-18.3109999999999\\
1070	-23.193\\
1071	-21.973\\
1072	-21.973\\
1073	-17.0899999999999\\
1075	-17.0899999999999\\
1076	-15.8689999999999\\
1077	-17.0899999999999\\
1078	-24.414\\
1079	-35.4000000000001\\
1080	-36.6210000000001\\
1081	-34.1800000000001\\
1082	-24.414\\
1083	-29.297\\
1084	-36.6210000000001\\
1085	-40.2829999999999\\
1086	-31.7380000000001\\
1087	-37.8420000000001\\
1088	-48.828\\
1089	-39.0630000000001\\
1090	-30.518\\
1091	-25.635\\
1092	-19.5309999999999\\
1093	-14.6479999999999\\
1094	-19.5309999999999\\
1095	-15.8689999999999\\
1096	-9.76600000000008\\
1097	-9.76600000000008\\
1098	-12.2070000000001\\
1099	-17.0899999999999\\
1100	-15.8689999999999\\
1101	-15.8689999999999\\
1102	-19.5309999999999\\
1103	-19.5309999999999\\
1104	-28.076\\
1105	-23.193\\
1106	-25.635\\
1107	-23.193\\
1108	-18.3109999999999\\
1109	-20.752\\
1110	-17.0899999999999\\
1111	-14.6479999999999\\
1112	-18.3109999999999\\
1114	-15.8689999999999\\
1116	-10.9860000000001\\
1118	-13.4280000000001\\
1119	-10.9860000000001\\
1120	-9.76600000000008\\
1122	-19.5309999999999\\
1123	-23.193\\
1124	-23.193\\
1125	-17.0899999999999\\
1126	-18.3109999999999\\
1127	-30.518\\
1128	-23.193\\
1129	-24.414\\
1130	-26.855\\
1131	-17.0899999999999\\
1132	-10.9860000000001\\
1133	-17.0899999999999\\
1134	-18.3109999999999\\
1135	-26.855\\
1136	-34.1800000000001\\
1137	-31.7380000000001\\
1138	-25.635\\
1140	-23.193\\
1141	-23.193\\
1142	-21.973\\
1144	-14.6479999999999\\
1146	-12.2070000000001\\
1147	-13.4280000000001\\
1149	-25.635\\
1150	-21.973\\
1151	-14.6479999999999\\
1152	-13.4280000000001\\
1153	-14.6479999999999\\
1154	-12.2070000000001\\
1155	-8.54500000000007\\
1156	-8.54500000000007\\
1157	-13.4280000000001\\
1158	-12.2070000000001\\
1161	-12.2070000000001\\
1162	-8.54500000000007\\
1163	-7.32400000000007\\
1164	-4.88300000000004\\
1165	-7.32400000000007\\
1166	-17.0899999999999\\
1167	-24.414\\
1168	-24.414\\
1169	-19.5309999999999\\
1170	-12.2070000000001\\
1171	-12.2070000000001\\
1172	-8.54500000000007\\
1173	-10.9860000000001\\
1174	-14.6479999999999\\
1175	-14.6479999999999\\
1177	-34.1800000000001\\
1178	-40.2829999999999\\
1179	-32.9590000000001\\
1180	-31.7380000000001\\
1181	-23.193\\
1182	-25.635\\
1184	-28.076\\
1185	-21.973\\
1187	-19.5309999999999\\
1188	-19.5309999999999\\
1189	-21.973\\
1190	-19.5309999999999\\
1191	-23.193\\
1192	-29.297\\
1193	-24.414\\
1194	-14.6479999999999\\
1195	-15.8689999999999\\
1196	-25.635\\
1197	-32.9590000000001\\
1198	-35.4000000000001\\
1199	-39.0630000000001\\
1200	-37.8420000000001\\
1201	-30.518\\
1202	-26.855\\
1203	-30.518\\
1204	-35.4000000000001\\
1205	-25.635\\
1206	-14.6479999999999\\
1207	-19.5309999999999\\
1208	-20.752\\
1209	-12.2070000000001\\
1210	-13.4280000000001\\
1211	-18.3109999999999\\
1212	-14.6479999999999\\
1213	-12.2070000000001\\
1214	-18.3109999999999\\
1215	-10.9860000000001\\
1216	-14.6479999999999\\
1217	-24.414\\
1218	-21.973\\
1219	-14.6479999999999\\
1220	-14.6479999999999\\
1221	-20.752\\
1222	-34.1800000000001\\
1223	-28.076\\
1224	-15.8689999999999\\
1225	-14.6479999999999\\
1226	-14.6479999999999\\
1227	-12.2070000000001\\
1228	-10.9860000000001\\
1229	-10.9860000000001\\
1230	-13.4280000000001\\
1231	-17.0899999999999\\
1232	-19.5309999999999\\
1233	-14.6479999999999\\
1234	-19.5309999999999\\
1235	-26.855\\
1236	-23.193\\
1237	-17.0899999999999\\
1238	-21.973\\
1239	-25.635\\
1240	-21.973\\
1241	-8.54500000000007\\
1242	-12.2070000000001\\
1243	-13.4280000000001\\
1244	-15.8689999999999\\
1245	-15.8689999999999\\
1246	-10.9860000000001\\
1247	-17.0899999999999\\
1248	-15.8689999999999\\
1249	-10.9860000000001\\
1250	-14.6479999999999\\
1251	-14.6479999999999\\
1252	-12.2070000000001\\
1253	-14.6479999999999\\
1254	-10.9860000000001\\
1255	-9.76600000000008\\
1256	-12.2070000000001\\
1257	-8.54500000000007\\
1258	-17.0899999999999\\
1259	-18.3109999999999\\
1260	-20.752\\
1261	-29.297\\
1262	-20.752\\
1263	-10.9860000000001\\
1264	-10.9860000000001\\
1265	-8.54500000000007\\
1266	-13.4280000000001\\
1267	-12.2070000000001\\
1268	-9.76600000000008\\
1269	-15.8689999999999\\
1270	-19.5309999999999\\
1271	-20.752\\
1272	-23.193\\
1273	-23.193\\
1274	-17.0899999999999\\
1275	-15.8689999999999\\
1276	-10.9860000000001\\
1277	-10.9860000000001\\
1278	-7.32400000000007\\
1279	-8.54500000000007\\
1280	-10.9860000000001\\
1281	-17.0899999999999\\
1282	-18.3109999999999\\
1283	-17.0899999999999\\
1284	-24.414\\
1285	-25.635\\
1286	-15.8689999999999\\
1289	-26.855\\
1290	-25.635\\
1291	-17.0899999999999\\
1292	-10.9860000000001\\
1293	-7.32400000000007\\
1294	-6.10400000000004\\
1295	-8.54500000000007\\
1296	-9.76600000000008\\
1297	-8.54500000000007\\
1298	-10.9860000000001\\
1299	-17.0899999999999\\
1300	-15.8689999999999\\
1301	-13.4280000000001\\
1302	-15.8689999999999\\
1303	-12.2070000000001\\
1304	-6.10400000000004\\
1305	-9.76600000000008\\
1306	-20.752\\
1307	-17.0899999999999\\
1308	-18.3109999999999\\
1309	-15.8689999999999\\
1310	-10.9860000000001\\
1312	-23.193\\
1313	-23.193\\
1314	-17.0899999999999\\
1315	-26.855\\
1316	-25.635\\
1317	-18.3109999999999\\
1318	-23.193\\
1319	-25.635\\
1320	-19.5309999999999\\
1321	-15.8689999999999\\
1322	-24.414\\
1323	-35.4000000000001\\
1324	-29.297\\
1325	-17.0899999999999\\
1326	-17.0899999999999\\
1327	-18.3109999999999\\
1329	-23.193\\
1330	-19.5309999999999\\
1331	-17.0899999999999\\
1332	-24.414\\
1333	-25.635\\
1334	-18.3109999999999\\
1335	-15.8689999999999\\
1336	-18.3109999999999\\
1337	-28.076\\
1338	-26.855\\
1339	-24.414\\
1340	-19.5309999999999\\
1341	-17.0899999999999\\
1342	-18.3109999999999\\
1343	-13.4280000000001\\
1344	-6.10400000000004\\
1346	-6.10400000000004\\
1347	-9.76600000000008\\
1348	-19.5309999999999\\
1349	-15.8689999999999\\
1351	-13.4280000000001\\
1352	-17.0899999999999\\
1354	-9.76600000000008\\
1355	-12.2070000000001\\
1356	-9.76600000000008\\
1357	-8.54500000000007\\
1358	-17.0899999999999\\
1359	-20.752\\
1360	-17.0899999999999\\
1361	-10.9860000000001\\
1362	-10.9860000000001\\
1363	-13.4280000000001\\
1364	-9.76600000000008\\
1365	-10.9860000000001\\
1366	-28.076\\
1367	-20.752\\
1368	-25.635\\
1369	-35.4000000000001\\
1370	-30.518\\
1372	-18.3109999999999\\
1373	-18.3109999999999\\
1374	-19.5309999999999\\
1375	-21.973\\
1376	-25.635\\
1377	-25.635\\
1378	-31.7380000000001\\
1379	-34.1800000000001\\
1380	-41.5039999999999\\
1381	-32.9590000000001\\
1382	-28.076\\
1383	-35.4000000000001\\
1384	-26.855\\
1385	-29.297\\
1386	-28.076\\
1387	-17.0899999999999\\
1388	-10.9860000000001\\
1389	-12.2070000000001\\
1390	-17.0899999999999\\
1391	-14.6479999999999\\
1392	-9.76600000000008\\
1393	-13.4280000000001\\
1394	-13.4280000000001\\
1395	-7.32400000000007\\
1396	-8.54500000000007\\
1397	-12.2070000000001\\
1398	-7.32400000000007\\
1399	-14.6479999999999\\
1400	-20.752\\
1401	-13.4280000000001\\
1403	-30.518\\
1404	-29.297\\
1405	-35.4000000000001\\
};
\addlegendentry{True output}

\addplot [color=mycolor2, dashed, line width=2.0pt]
  table[row sep=crcr]{%
1006	-24.5192122402532\\
1007	-26.6277885528164\\
1008	-23.4342653885014\\
1009	-13.1738057908742\\
1010	-15.5068595562459\\
1011	-17.4156327787528\\
1012	-17.7106683869799\\
1013	-17.1123404759057\\
1014	-7.81679636714625\\
1015	-6.42585167004813\\
1016	-6.11855903763626\\
1017	-13.6452442546772\\
1018	-21.4168089231403\\
1019	-20.9816378275495\\
1020	-21.2851054458656\\
1021	-17.2790022293648\\
1022	-12.2225520590887\\
1023	-22.4708634750045\\
1024	-17.8900663675774\\
1025	-11.1026688142447\\
1026	-19.2993575338644\\
1027	-19.4689242771492\\
1028	-15.4586186024897\\
1029	-22.0728064675438\\
1030	-23.65865587335\\
1031	-16.9129596673963\\
1032	-23.3933541205961\\
1033	-23.1376029713401\\
1034	-18.6278406558285\\
1035	-16.8958427208463\\
1036	-14.6409275126377\\
1037	-16.214324800749\\
1038	-19.4843831125293\\
1039	-18.5864377382049\\
1040	-19.1186598061443\\
1041	-18.8507486485432\\
1042	-19.9244193690035\\
1043	-26.1371652268281\\
1044	-24.8278511302324\\
1045	-17.6849977729164\\
1046	-17.1815123501633\\
1047	-11.2809806299078\\
1048	-10.2925969418077\\
1049	-14.6992425506628\\
1050	-13.214592086985\\
1051	-11.9338085652014\\
1052	-16.46592831099\\
1053	-18.2119822597269\\
1054	-16.7325685803414\\
1055	-23.275577271982\\
1056	-20.6265059200296\\
1057	-12.6226719177819\\
1058	-11.3830372871796\\
1059	-13.9839018705643\\
1060	-15.1094078763717\\
1061	-9.54306447718977\\
1062	-9.2445084849976\\
1063	-13.5767050531292\\
1064	-12.5016103802027\\
1065	-8.76556507584633\\
1066	-11.254593080474\\
1067	-13.3253939564095\\
1068	-18.4760816893354\\
1069	-18.9181508684926\\
1070	-19.4048260487282\\
1071	-19.6446837795834\\
1072	-23.2559311336465\\
1073	-20.5467673610744\\
1074	-17.0640322928934\\
1075	-16.1624130908667\\
1076	-16.09959010175\\
1077	-15.8754944511995\\
1078	-24.3089408705644\\
1079	-34.1005770016866\\
1080	-34.64963481208\\
1081	-34.7979104975652\\
1082	-23.2843385569761\\
1083	-31.3904580142166\\
1084	-37.7933433513217\\
1085	-36.336168249413\\
1086	-29.9897663551619\\
1087	-37.9989861880949\\
1088	-49.5959185116551\\
1089	-36.5118021500371\\
1090	-32.8937043318988\\
1091	-22.8700357547871\\
1092	-18.0118206597006\\
1093	-16.2713431497948\\
1094	-18.9854872641911\\
1095	-16.6100137214657\\
1096	-9.6545044312661\\
1097	-9.23156260468636\\
1098	-12.1102216647057\\
1099	-17.3728609435329\\
1100	-16.5793316284082\\
1101	-16.0797653955894\\
1102	-20.2356216995991\\
1103	-21.6270298485254\\
1104	-26.1285002720711\\
1105	-24.0386938213517\\
1106	-26.2068655866997\\
1107	-22.8676880187584\\
1108	-18.7561945020634\\
1109	-21.6118983843\\
1110	-17.771391715665\\
1111	-15.226841311292\\
1112	-18.4002330615497\\
1113	-18.2223209272781\\
1114	-14.1643535819023\\
1115	-14.6295884309782\\
1116	-12.2402686534451\\
1117	-12.5273949222797\\
1118	-13.0710997002668\\
1119	-10.40963798503\\
1120	-11.9852465073543\\
1121	-14.869012930352\\
1122	-20.3004283872435\\
1123	-21.0366305072348\\
1124	-22.2367791700885\\
1125	-15.6808296967267\\
1126	-20.4780548832289\\
1127	-29.3257948799144\\
1128	-24.3520365858365\\
1129	-24.8535093225985\\
1130	-27.2245918983899\\
1131	-15.9046031415762\\
1132	-12.7003083270681\\
1133	-15.7573973326896\\
1134	-18.6346144055326\\
1135	-27.2900682958891\\
1136	-32.4501806428668\\
1137	-31.8970689369835\\
1138	-26.18594235572\\
1139	-25.8566976360164\\
1140	-23.2684226232061\\
1141	-22.8340445330155\\
1142	-23.200385786178\\
1143	-17.2535711774269\\
1144	-14.7522752677737\\
1145	-14.1895278002012\\
1146	-13.0868275459648\\
1147	-14.3916202595249\\
1148	-18.8441835578606\\
1149	-26.8999415187989\\
1150	-22.7661505101605\\
1151	-16.1846204929998\\
1152	-14.9882579486211\\
1153	-14.198686286408\\
1154	-9.47577881611437\\
1155	-7.3368964206752\\
1156	-8.75375048887327\\
1157	-14.6428495561877\\
1158	-12.6137585703884\\
1159	-11.7532739928126\\
1160	-12.8832843054511\\
1161	-12.6865058343733\\
1162	-9.80224374206273\\
1163	-7.4484509065403\\
1164	-6.08885416283374\\
1165	-8.14759679067288\\
1166	-17.0875932378353\\
1167	-22.6327893563825\\
1168	-24.4194240440911\\
1169	-18.479508805483\\
1170	-13.2502954166448\\
1171	-10.4897393817043\\
1172	-8.22444499523021\\
1173	-14.2858684377115\\
1174	-15.237350316969\\
1175	-17.1007297149065\\
1176	-21.8325638982808\\
1177	-34.6341934184852\\
1178	-40.3639344177002\\
1179	-35.0212772906325\\
1180	-33.8690677854422\\
1181	-21.5947716827084\\
1182	-25.1926470369967\\
1183	-26.295949115301\\
1184	-27.7804428749894\\
1185	-23.7518397853637\\
1186	-20.7768424435528\\
1187	-21.2888509640638\\
1188	-19.1983426545571\\
1189	-21.544792768085\\
1190	-17.9271340705143\\
1191	-26.4964495625372\\
1192	-32.0099084863459\\
1193	-22.9694648796428\\
1194	-15.045435017963\\
1195	-15.8390262707931\\
1196	-26.0229055791685\\
1197	-30.354273701535\\
1198	-35.315092500686\\
1199	-38.1980803532022\\
1200	-38.2573233296916\\
1201	-30.4040840409443\\
1202	-28.576526137075\\
1203	-31.2775663885377\\
1204	-34.9642339351471\\
1205	-24.1880613685071\\
1206	-15.3641937051987\\
1207	-21.6039340663976\\
1208	-19.9554082365694\\
1209	-12.8826125724995\\
1210	-15.4726798329052\\
1211	-18.9877530637584\\
1212	-15.359279648633\\
1213	-12.5810212152155\\
1214	-14.8518576202816\\
1215	-15.27526384211\\
1216	-16.8248426207301\\
1217	-23.1895722222421\\
1218	-21.1421729954554\\
1219	-16.2310039596532\\
1220	-16.1120957366202\\
1221	-22.0575835956133\\
1222	-34.7598850518114\\
1223	-26.4047159708155\\
1224	-17.26623656425\\
1225	-14.24917846528\\
1226	-16.9677786622287\\
1227	-15.651626061356\\
1228	-11.8956748512371\\
1229	-11.9002772052843\\
1230	-14.3969144820255\\
1231	-17.2724268318673\\
1232	-19.6849534774447\\
1233	-14.314198111646\\
1234	-21.2525743065376\\
1235	-25.780752595615\\
1236	-22.3806594706427\\
1237	-20.4576387458999\\
1238	-20.3341868824687\\
1239	-26.6193089771696\\
1240	-19.0146504120135\\
1241	-9.218244642738\\
1242	-11.5799904332698\\
1243	-15.2360564396092\\
1244	-17.863926271122\\
1245	-16.2389868657442\\
1246	-12.484093248929\\
1247	-17.6055781323562\\
1248	-18.1300931707656\\
1249	-13.1794932583043\\
1250	-14.6398473151398\\
1251	-14.9672303058098\\
1252	-12.6634760894401\\
1253	-14.6743321957354\\
1254	-10.8269138291569\\
1255	-10.9392616080111\\
1256	-9.72945115506263\\
1257	-10.244024964343\\
1258	-15.3363293150703\\
1259	-18.3496156446204\\
1260	-21.9940188238509\\
1261	-29.8954129662202\\
1262	-21.6164922015928\\
1263	-12.6575017907335\\
1264	-10.9346463859895\\
1265	-10.6833766885065\\
1266	-13.9341078548673\\
1267	-10.1797309282804\\
1268	-10.1795298518409\\
1269	-13.7836663450682\\
1270	-22.8754834059721\\
1271	-20.8711932252761\\
1272	-24.9465967392573\\
1273	-18.9802665348213\\
1274	-16.9814825125397\\
1275	-11.0743016503279\\
1276	-10.5777455007683\\
1277	-9.4823533423787\\
1278	-9.04191919921504\\
1279	-10.3936480958964\\
1280	-14.1304931356112\\
1281	-15.6908806874301\\
1282	-15.1119289781921\\
1283	-16.9276099395975\\
1284	-26.9196896741705\\
1285	-24.9678089671525\\
1286	-17.937900816436\\
1287	-20.7397425574841\\
1288	-21.6091299453535\\
1289	-26.5455893152687\\
1290	-22.3327432382014\\
1291	-16.3185934698492\\
1292	-9.93652871725044\\
1293	-8.26170974592378\\
1294	-8.36641503768988\\
1295	-9.23006809605295\\
1296	-9.14111401651121\\
1297	-8.68421331302625\\
1298	-10.5715550128348\\
1299	-16.0644182038227\\
1300	-15.1957504141819\\
1301	-14.9399160188868\\
1302	-16.5109431550845\\
1303	-11.4109287569622\\
1304	-6.98905151594363\\
1305	-11.2974404714862\\
1306	-18.2770625654609\\
1307	-16.8069076381259\\
1308	-17.8881504207366\\
1309	-17.1662138466452\\
1310	-13.3501128825656\\
1311	-16.5866465606268\\
1312	-21.5980660741131\\
1313	-22.0780471498558\\
1314	-15.9258480760893\\
1315	-25.112074695671\\
1316	-23.9100348229542\\
1317	-20.0029575172991\\
1318	-23.8787277578583\\
1319	-23.2502918415717\\
1320	-18.8251419143626\\
1321	-16.9464145663019\\
1322	-26.1038050536627\\
1323	-33.4114306387582\\
1324	-27.0169766201197\\
1325	-17.133200162978\\
1326	-17.7092122924371\\
1327	-18.0634835233163\\
1328	-20.6384694095661\\
1329	-23.5844951473521\\
1330	-18.6170762667582\\
1331	-19.1336956398509\\
1332	-23.7837423340379\\
1333	-25.2689761739603\\
1334	-18.6044837331831\\
1335	-15.1514209545769\\
1336	-19.0622885060995\\
1337	-29.7557556104794\\
1338	-26.2290348294025\\
1339	-24.7774738609908\\
1340	-18.0892880607819\\
1341	-15.3295231762297\\
1342	-16.1792380991508\\
1343	-12.499217295715\\
1344	-8.40016076485813\\
1345	-6.91219895232348\\
1346	-6.38565761484938\\
1347	-10.8300766925233\\
1348	-15.4817131683583\\
1349	-18.5154451218473\\
1350	-14.2468529493351\\
1351	-14.9864270686314\\
1352	-17.0090090979975\\
1353	-11.3291475391043\\
1354	-11.36574631297\\
1355	-12.4143097192161\\
1356	-9.71144431078892\\
1357	-11.4058869465234\\
1358	-17.4215566770188\\
1359	-20.337050761935\\
1360	-13.7295378152678\\
1361	-11.1147312994499\\
1362	-11.3392065997284\\
1363	-12.0932177987702\\
1364	-9.70898219629248\\
1365	-16.3646840966094\\
1366	-26.8211772029902\\
1367	-20.5796131034472\\
1368	-25.1395529834558\\
1369	-31.0275663546636\\
1370	-30.6635834703079\\
1371	-25.3351167106978\\
1372	-19.3946058423312\\
1373	-18.8600986809513\\
1374	-19.3268319663375\\
1375	-21.4245463366844\\
1376	-23.9395180004628\\
1377	-24.0044300864693\\
1378	-30.758062125889\\
1379	-35.9693665377804\\
1380	-39.6348501325299\\
1381	-29.976522076289\\
1382	-30.7233955612037\\
1383	-35.1452430089869\\
1384	-28.6205591288126\\
1385	-30.4010562115902\\
1386	-27.9378556425568\\
1387	-16.0200984962655\\
1388	-11.0575504896799\\
1389	-11.4574894913935\\
1390	-16.6425939099111\\
1391	-15.1860039969936\\
1392	-10.5198627597024\\
1393	-13.8617436837078\\
1394	-12.2219089556538\\
1395	-8.99397112953898\\
1396	-10.7224682683573\\
1397	-12.0126678893948\\
1398	-9.61971302946768\\
1399	-13.6206904818223\\
1400	-19.5639509496264\\
1401	-14.7962548697726\\
1402	-24.0839263972762\\
1403	-34.4296612620979\\
1404	-28.5303599116482\\
1405	-33.8299220730519\\
};
\addlegendentry{OSA predition}

\addplot [color=mycolor3, dotted, line width=2.0pt]
  table[row sep=crcr]{%
1006	-24.414\\
1007	-28.076\\
1008	-23.193\\
1009	-13.4280000000001\\
1010	-15.5068595562461\\
1011	-17.1847031180312\\
1012	-17.433281705943\\
1013	-16.7525491016265\\
1014	-7.72764797090122\\
1015	-6.40305417813624\\
1016	-6.06948565647485\\
1017	-13.3644657667048\\
1018	-21.8103711525121\\
1019	-21.2890404020602\\
1020	-21.605401481715\\
1021	-16.8443742353577\\
1022	-11.341902406567\\
1023	-21.5996407628204\\
1024	-18.1198509086569\\
1025	-11.6306940295665\\
1026	-19.6868540559749\\
1027	-19.2795978726663\\
1028	-15.3547687583598\\
1029	-22.342939582144\\
1030	-23.5262447937885\\
1031	-16.827397898596\\
1032	-23.4316266523983\\
1033	-23.5498701563452\\
1034	-18.6112254906068\\
1035	-16.4466911995567\\
1036	-14.1037862277783\\
1037	-15.7795625074818\\
1038	-19.0377763708316\\
1039	-17.9969935533754\\
1040	-18.5189274939933\\
1041	-18.4865241717762\\
1042	-19.6916246290923\\
1043	-25.8308576147522\\
1044	-24.1803241999569\\
1045	-16.9070623990478\\
1046	-16.504134004068\\
1047	-11.0930507786927\\
1048	-10.1589095286004\\
1049	-14.4694234261954\\
1050	-12.7339444852516\\
1051	-11.6106716120682\\
1052	-16.1798958990028\\
1053	-18.1919534574088\\
1054	-16.6957298300906\\
1055	-23.4175982470131\\
1056	-20.4875258723957\\
1057	-12.0978913674585\\
1058	-10.862356995746\\
1059	-13.8065873094965\\
1060	-15.2844533316159\\
1061	-9.63189285422027\\
1062	-9.02156284877105\\
1063	-12.9497557796142\\
1064	-12.0637053938915\\
1065	-8.96599076234861\\
1066	-11.7540551288773\\
1067	-13.7290547691834\\
1068	-18.4413115227019\\
1069	-18.8902133490858\\
1070	-19.5389142605854\\
1071	-19.4460067064977\\
1072	-22.4124597838561\\
1073	-19.5860669208514\\
1074	-16.9823593377803\\
1075	-16.5779515028548\\
1076	-16.4117523589598\\
1077	-15.9449179989297\\
1078	-24.1225808435647\\
1079	-33.9797451187944\\
1080	-34.2246160334601\\
1081	-34.2075934920852\\
1082	-22.6131274128099\\
1083	-30.5978201953963\\
1084	-37.4845224696894\\
1085	-36.2677023518702\\
1086	-29.9990362116921\\
1087	-37.2455717779915\\
1088	-48.1162272097852\\
1089	-35.8840518612753\\
1090	-32.312471120739\\
1091	-22.2639130406308\\
1092	-17.4962830793709\\
1093	-15.421905485347\\
1094	-18.175465298749\\
1095	-16.122455866363\\
1096	-9.43073629347305\\
1097	-9.09297046189818\\
1098	-11.9277402568907\\
1099	-17.1544597070565\\
1100	-16.4151259778241\\
1101	-16.0989171461811\\
1102	-20.3692986725177\\
1103	-21.8903854213277\\
1104	-26.6751637845668\\
1105	-24.4684036517174\\
1106	-26.6059039038444\\
1107	-23.0948909694182\\
1108	-19.0595806863016\\
1109	-21.9266253139965\\
1110	-18.0984053169439\\
1111	-15.6957853986273\\
1112	-18.9862961321442\\
1113	-18.7290351210818\\
1114	-14.7053316666947\\
1115	-14.9059019990539\\
1116	-12.48443077964\\
1117	-12.919894770333\\
1118	-13.6281182583807\\
1119	-10.8096485170111\\
1120	-12.1614614601415\\
1121	-15.2698737340984\\
1122	-20.7793698325715\\
1123	-21.7714247299241\\
1124	-22.4821386117083\\
1125	-15.5059823997794\\
1126	-19.9058607589923\\
1127	-29.0267019517003\\
1128	-24.1721229153457\\
1129	-24.9354868603064\\
1130	-27.250560275502\\
1131	-16.1024125994452\\
1132	-12.7320433769182\\
1133	-15.9193630312545\\
1134	-18.6287359948815\\
1135	-27.5104040230772\\
1136	-32.3745879545179\\
1137	-31.858879161658\\
1138	-25.9723904429372\\
1139	-25.6315177538563\\
1140	-23.4361750707976\\
1141	-23.1688307912787\\
1142	-23.4777767045668\\
1143	-17.5298460586478\\
1144	-14.9374887498013\\
1145	-14.2983403925414\\
1146	-13.2225132585897\\
1147	-14.7214548271986\\
1148	-19.403912918373\\
1149	-27.3937678646623\\
1150	-23.2222522298816\\
1151	-16.7099291449408\\
1152	-15.7765435561043\\
1153	-15.1975435291697\\
1154	-10.2651209738988\\
1155	-7.4452168799105\\
1156	-8.39327312276077\\
1157	-14.0851734804749\\
1158	-12.4527617083825\\
1159	-11.8342612501392\\
1160	-12.9618991567484\\
1161	-12.7991421512465\\
1162	-9.98915752707671\\
1163	-7.83979214443298\\
1164	-6.50743523713186\\
1165	-8.72810180803754\\
1166	-17.7569157982803\\
1167	-23.3850355345141\\
1168	-24.7603656920901\\
1169	-18.4927603518272\\
1170	-13.0811751766257\\
1171	-10.4228024778431\\
1172	-8.02470881486988\\
1173	-14.0079917642925\\
1174	-15.3511771951712\\
1175	-17.5403516963743\\
1176	-23.002334728259\\
1177	-35.2821176697294\\
1178	-41.0340693427936\\
1179	-35.1152448843407\\
1180	-34.3806590860843\\
1181	-22.4408011691191\\
1182	-26.0593801657942\\
1183	-26.7858500713103\\
1184	-27.812369315159\\
1185	-23.7478277451751\\
1186	-20.9646200130046\\
1187	-21.619609969955\\
1188	-19.8032292414316\\
1189	-22.0953464190504\\
1190	-18.3190887690364\\
1191	-26.4937237148004\\
1192	-32.2340882435922\\
1193	-23.7225690578086\\
1194	-15.8551394239755\\
1195	-16.4552947063457\\
1196	-26.3049556601622\\
1197	-30.8001335831864\\
1198	-35.4052742835477\\
1199	-37.9615199943084\\
1200	-37.7322758948271\\
1201	-30.0867241237006\\
1202	-28.3130272719295\\
1203	-31.2940883475749\\
1204	-35.2409222205747\\
1205	-24.5311492119647\\
1206	-15.3358194911673\\
1207	-21.4818171701122\\
1208	-20.1689828719545\\
1209	-13.2104795659534\\
1210	-15.8612537662159\\
1211	-19.5118584025922\\
1212	-16.1020988599303\\
1213	-13.3837426901598\\
1214	-15.5959041572237\\
1215	-15.3730760080111\\
1216	-17.2598642961443\\
1217	-23.8271951066833\\
1218	-22.2145571053356\\
1219	-16.686392353422\\
1220	-16.4742556904646\\
1221	-22.7370208539935\\
1222	-35.9508728108633\\
1223	-27.3318239541456\\
1224	-17.8118565042528\\
1225	-14.6837029956744\\
1226	-17.3022285636512\\
1227	-16.3169307441426\\
1228	-13.0802572801215\\
1229	-13.3685674938617\\
1230	-15.9479281395929\\
1231	-18.6281315002486\\
1232	-20.8949297707409\\
1233	-15.2055827480167\\
1234	-22.0001090028436\\
1235	-26.5746282886046\\
1236	-22.9478726924885\\
1237	-20.7911167276388\\
1238	-20.8601680583156\\
1239	-27.1104114708512\\
1240	-19.5007444449127\\
1241	-9.2317740434778\\
1242	-11.3683621807347\\
1243	-14.8867771273544\\
1244	-17.9643225375401\\
1245	-16.6909114274399\\
1246	-13.130682772083\\
1247	-18.5378077817579\\
1248	-18.9522044735554\\
1249	-14.1960176983737\\
1250	-16.0096111802591\\
1251	-16.301271750006\\
1252	-13.7185112783441\\
1253	-15.5542674978699\\
1254	-11.4811656005327\\
1255	-11.4528343954871\\
1256	-10.2507687532566\\
1257	-10.34247717933\\
1258	-15.6165205609793\\
1259	-18.1411868519365\\
1260	-22.0171736418938\\
1261	-29.7490958904077\\
1262	-21.8096011342452\\
1263	-12.9980078922406\\
1264	-11.54911051562\\
1265	-11.3081595474716\\
1266	-14.8585059965169\\
1267	-11.0270247968231\\
1268	-10.6451968607626\\
1269	-14.0611638071352\\
1270	-22.5403565474001\\
1271	-21.0824968970549\\
1272	-25.1742762114184\\
1273	-19.6650329264137\\
1274	-17.0413959631019\\
1275	-10.6042011856871\\
1276	-9.45410887350181\\
1277	-8.12970711334788\\
1278	-7.70962623356036\\
1279	-9.5312790616515\\
1280	-13.7625665776354\\
1281	-16.2329008287334\\
1282	-15.6262694842399\\
1283	-16.9037364799021\\
1284	-26.3255276301336\\
1285	-24.6284239635302\\
1286	-18.0210761471621\\
1287	-21.1937295088846\\
1288	-22.1422498044112\\
1289	-27.141061739741\\
1290	-22.5289249032103\\
1291	-15.9118850873238\\
1292	-9.14878181003746\\
1293	-7.48531466164218\\
1294	-7.78375305398231\\
1295	-9.22677937405251\\
1296	-9.47635742359444\\
1297	-8.97587831210922\\
1298	-10.7629306931344\\
1299	-16.0717038179159\\
1300	-15.0681517903572\\
1301	-14.6332204332009\\
1302	-16.3723499886275\\
1303	-11.56405110879\\
1304	-7.04096490922257\\
1305	-11.4632930732027\\
1306	-18.5877441080129\\
1307	-16.9212939171036\\
1308	-17.8295616995629\\
1309	-16.8028048102974\\
1310	-13.2911114030828\\
1311	-17.0153528394605\\
1312	-22.1751361328311\\
1313	-22.4056309695825\\
1314	-15.7225219455258\\
1315	-24.5452130177812\\
1316	-23.1872691465369\\
1317	-19.0299756686766\\
1318	-23.0245147932553\\
1319	-22.8338308211009\\
1320	-18.443118473074\\
1321	-16.2691780090611\\
1322	-25.3295602786254\\
1323	-33.2148809541493\\
1324	-27.0057907329956\\
1325	-16.5677395697853\\
1326	-16.8981808522014\\
1327	-17.4761489272582\\
1328	-20.2933110160236\\
1329	-23.307624353803\\
1330	-18.4158305756225\\
1331	-18.8890874741396\\
1332	-23.8328776220783\\
1333	-25.3149822915839\\
1334	-18.6650294569686\\
1335	-15.1373650996775\\
1336	-18.9422207745179\\
1337	-29.7824741106597\\
1338	-26.4146936503087\\
1339	-25.1243452653459\\
1340	-18.3481830246142\\
1341	-15.3117769581022\\
1342	-15.7818448677383\\
1343	-11.6859889609898\\
1344	-7.46206646434962\\
1345	-6.48813680310786\\
1346	-6.4020047072936\\
1348	-15.8709561806124\\
1349	-18.2848441018091\\
1350	-14.1765099149304\\
1351	-14.8309781704809\\
1352	-17.4048305047497\\
1353	-11.6684669401604\\
1354	-11.3226841474429\\
1355	-12.4404399216996\\
1356	-9.82654863539869\\
1357	-11.6242574711857\\
1358	-18.0196548904223\\
1359	-21.0149921523118\\
1360	-14.3130577889201\\
1361	-10.9811016676749\\
1362	-10.9006849694169\\
1363	-11.6931818191019\\
1364	-9.31807025700982\\
1365	-15.97911005275\\
1366	-27.0261658224765\\
1367	-21.1525827471953\\
1368	-26.026085308109\\
1369	-31.1187853604997\\
1370	-30.353725749218\\
1372	-18.8234270132516\\
1373	-18.7668248647662\\
1374	-19.5122166424555\\
1375	-21.6100578638645\\
1376	-23.9971377273175\\
1377	-23.7446526531937\\
1378	-30.1270049147211\\
1379	-35.034794990795\\
1380	-38.8850727274287\\
1381	-29.6138189300216\\
1382	-29.9369212580241\\
1383	-34.1395108360866\\
1384	-28.183354249034\\
1385	-30.544156145497\\
1386	-28.1876528806108\\
1387	-16.3812310506796\\
1388	-11.1555309889698\\
1389	-11.4076678507417\\
1390	-16.4272642918679\\
1391	-14.9414255808717\\
1392	-10.3600854369267\\
1393	-13.8658538338755\\
1394	-12.4083103648722\\
1395	-8.99478119569176\\
1396	-10.8913084907379\\
1397	-12.5334138154844\\
1398	-10.2409590401664\\
1399	-14.6269414712569\\
1400	-20.2283416297996\\
1401	-15.1397947288756\\
1402	-24.3604422715473\\
1403	-34.9102978326171\\
1404	-29.7139524141623\\
1405	-35.3060352508978\\
};
\addlegendentry{MPO prediction}

\end{axis}

\begin{axis}[%
width=6.159cm,
height=1.831cm,
at={(0cm,7.627cm)},
scale only axis,
xmin=1000,
xmax=1405,
xlabel style={font=\color{white!15!black}},
xlabel={Sample index},
ymin=-40.283,
ymax=0,
ylabel style={font=\color{white!15!black}},
ylabel={$y(t)$},
axis background/.style={fill=white},
title style={font=\bfseries},
title={C3: RMSE(OSA) = 1.6731, RMSE(MPO) = 1.7983},
legend style={legend cell align=left, align=left, draw=white!15!black}
]
\addplot [color=mycolor1, line width=2.0pt]
  table[row sep=crcr]{%
1006	-18.3109999999999\\
1007	-23.193\\
1008	-18.3109999999999\\
1009	-9.76600000000008\\
1010	-15.8689999999999\\
1011	-12.2070000000001\\
1012	-14.6479999999999\\
1013	-10.9860000000001\\
1014	-3.66200000000003\\
1015	-2.44100000000003\\
1016	-4.88300000000004\\
1017	-6.10400000000004\\
1018	-14.6479999999999\\
1019	-17.0899999999999\\
1020	-17.0899999999999\\
1022	-9.76600000000008\\
1023	-18.3109999999999\\
1025	-10.9860000000001\\
1026	-13.4280000000001\\
1028	-10.9860000000001\\
1029	-20.752\\
1030	-19.5309999999999\\
1031	-12.2070000000001\\
1032	-20.752\\
1033	-20.752\\
1034	-15.8689999999999\\
1035	-13.4280000000001\\
1036	-9.76600000000008\\
1037	-14.6479999999999\\
1041	-14.6479999999999\\
1042	-15.8689999999999\\
1043	-21.973\\
1044	-20.752\\
1045	-13.4280000000001\\
1046	-13.4280000000001\\
1047	-8.54500000000007\\
1048	-12.2070000000001\\
1049	-12.2070000000001\\
1050	-13.4280000000001\\
1051	-10.9860000000001\\
1052	-13.4280000000001\\
1053	-14.6479999999999\\
1054	-13.4280000000001\\
1055	-20.752\\
1056	-13.4280000000001\\
1057	-9.76600000000008\\
1058	-7.32400000000007\\
1059	-8.54500000000007\\
1060	-10.9860000000001\\
1061	-6.10400000000004\\
1062	-9.76600000000008\\
1063	-10.9860000000001\\
1064	-6.10400000000004\\
1065	-6.10400000000004\\
1066	-9.76600000000008\\
1067	-9.76600000000008\\
1068	-15.8689999999999\\
1069	-14.6479999999999\\
1070	-17.0899999999999\\
1071	-15.8689999999999\\
1072	-18.3109999999999\\
1073	-13.4280000000001\\
1074	-14.6479999999999\\
1075	-12.2070000000001\\
1076	-10.9860000000001\\
1077	-12.2070000000001\\
1078	-23.193\\
1079	-28.076\\
1080	-30.518\\
1081	-29.297\\
1082	-18.3109999999999\\
1083	-28.076\\
1084	-32.9590000000001\\
1085	-31.7380000000001\\
1086	-20.752\\
1087	-34.1800000000001\\
1088	-40.2829999999999\\
1089	-29.297\\
1091	-19.5309999999999\\
1093	-12.2070000000001\\
1094	-14.6479999999999\\
1095	-9.76600000000008\\
1096	-7.32400000000007\\
1097	-7.32400000000007\\
1098	-10.9860000000001\\
1099	-13.4280000000001\\
1100	-12.2070000000001\\
1101	-12.2070000000001\\
1102	-15.8689999999999\\
1103	-18.3109999999999\\
1104	-19.5309999999999\\
1105	-15.8689999999999\\
1106	-21.973\\
1107	-15.8689999999999\\
1108	-15.8689999999999\\
1109	-17.0899999999999\\
1110	-12.2070000000001\\
1111	-12.2070000000001\\
1112	-15.8689999999999\\
1113	-14.6479999999999\\
1114	-8.54500000000007\\
1115	-12.2070000000001\\
1116	-8.54500000000007\\
1117	-9.76600000000008\\
1118	-9.76600000000008\\
1119	-7.32400000000007\\
1121	-12.2070000000001\\
1122	-17.0899999999999\\
1124	-17.0899999999999\\
1125	-10.9860000000001\\
1126	-12.2070000000001\\
1127	-23.193\\
1128	-15.8689999999999\\
1129	-20.752\\
1130	-21.973\\
1131	-12.2070000000001\\
1132	-9.76600000000008\\
1133	-12.2070000000001\\
1134	-13.4280000000001\\
1135	-21.973\\
1136	-25.635\\
1137	-26.855\\
1138	-19.5309999999999\\
1139	-20.752\\
1140	-18.3109999999999\\
1141	-17.0899999999999\\
1142	-18.3109999999999\\
1143	-13.4280000000001\\
1144	-12.2070000000001\\
1145	-9.76600000000008\\
1146	-9.76600000000008\\
1147	-10.9860000000001\\
1148	-15.8689999999999\\
1149	-23.193\\
1150	-17.0899999999999\\
1151	-13.4280000000001\\
1152	-10.9860000000001\\
1153	-10.9860000000001\\
1154	-7.32400000000007\\
1155	-6.10400000000004\\
1156	-6.10400000000004\\
1157	-12.2070000000001\\
1158	-7.32400000000007\\
1159	-8.54500000000007\\
1161	-8.54500000000007\\
1162	-7.32400000000007\\
1163	-4.88300000000004\\
1164	-3.66200000000003\\
1165	-8.54500000000007\\
1166	-15.8689999999999\\
1168	-20.752\\
1170	-10.9860000000001\\
1171	-8.54500000000007\\
1172	-4.88300000000004\\
1173	-9.76600000000008\\
1174	-8.54500000000007\\
1175	-15.8689999999999\\
1176	-17.0899999999999\\
1177	-29.297\\
1178	-32.9590000000001\\
1179	-25.635\\
1180	-25.635\\
1181	-18.3109999999999\\
1182	-23.193\\
1183	-21.973\\
1184	-24.414\\
1185	-15.8689999999999\\
1186	-17.0899999999999\\
1187	-17.0899999999999\\
1188	-15.8689999999999\\
1189	-15.8689999999999\\
1190	-13.4280000000001\\
1191	-23.193\\
1192	-25.635\\
1194	-13.4280000000001\\
1195	-14.6479999999999\\
1196	-23.193\\
1197	-24.414\\
1198	-28.076\\
1199	-30.518\\
1200	-29.297\\
1201	-23.193\\
1202	-21.973\\
1203	-25.635\\
1204	-28.076\\
1205	-18.3109999999999\\
1206	-12.2070000000001\\
1207	-19.5309999999999\\
1209	-9.76600000000008\\
1210	-13.4280000000001\\
1211	-14.6479999999999\\
1212	-10.9860000000001\\
1214	-10.9860000000001\\
1215	-8.54500000000007\\
1216	-10.9860000000001\\
1217	-18.3109999999999\\
1218	-17.0899999999999\\
1219	-12.2070000000001\\
1220	-12.2070000000001\\
1221	-18.3109999999999\\
1222	-26.855\\
1223	-17.0899999999999\\
1224	-14.6479999999999\\
1225	-10.9860000000001\\
1226	-13.4280000000001\\
1228	-8.54500000000007\\
1229	-8.54500000000007\\
1231	-13.4280000000001\\
1232	-14.6479999999999\\
1233	-12.2070000000001\\
1234	-15.8689999999999\\
1235	-20.752\\
1236	-17.0899999999999\\
1237	-12.2070000000001\\
1238	-15.8689999999999\\
1239	-20.752\\
1240	-13.4280000000001\\
1241	-7.32400000000007\\
1242	-13.4280000000001\\
1243	-10.9860000000001\\
1244	-12.2070000000001\\
1245	-9.76600000000008\\
1246	-8.54500000000007\\
1247	-13.4280000000001\\
1248	-13.4280000000001\\
1249	-9.76600000000008\\
1250	-12.2070000000001\\
1252	-7.32400000000007\\
1253	-12.2070000000001\\
1254	-8.54500000000007\\
1255	-9.76600000000008\\
1257	-4.88300000000004\\
1258	-12.2070000000001\\
1259	-15.8689999999999\\
1260	-17.0899999999999\\
1261	-23.193\\
1262	-15.8689999999999\\
1263	-9.76600000000008\\
1264	-8.54500000000007\\
1265	-8.54500000000007\\
1266	-10.9860000000001\\
1267	-8.54500000000007\\
1268	-4.88300000000004\\
1269	-10.9860000000001\\
1270	-15.8689999999999\\
1271	-13.4280000000001\\
1272	-20.752\\
1273	-15.8689999999999\\
1274	-14.6479999999999\\
1275	-8.54500000000007\\
1276	-10.9860000000001\\
1277	-4.88300000000004\\
1278	-3.66200000000003\\
1279	-4.88300000000004\\
1280	-9.76600000000008\\
1283	-13.4280000000001\\
1284	-20.752\\
1285	-17.0899999999999\\
1286	-14.6479999999999\\
1288	-19.5309999999999\\
1289	-20.752\\
1291	-13.4280000000001\\
1292	-7.32400000000007\\
1294	-4.88300000000004\\
1295	-7.32400000000007\\
1296	-7.32400000000007\\
1297	-4.88300000000004\\
1299	-12.2070000000001\\
1300	-10.9860000000001\\
1301	-12.2070000000001\\
1302	-12.2070000000001\\
1304	-4.88300000000004\\
1305	-9.76600000000008\\
1306	-15.8689999999999\\
1307	-13.4280000000001\\
1308	-15.8689999999999\\
1309	-13.4280000000001\\
1310	-9.76600000000008\\
1311	-14.6479999999999\\
1312	-18.3109999999999\\
1313	-18.3109999999999\\
1314	-13.4280000000001\\
1315	-20.752\\
1316	-15.8689999999999\\
1319	-19.5309999999999\\
1320	-13.4280000000001\\
1321	-13.4280000000001\\
1322	-18.3109999999999\\
1323	-26.855\\
1324	-21.973\\
1325	-13.4280000000001\\
1327	-13.4280000000001\\
1328	-15.8689999999999\\
1329	-17.0899999999999\\
1330	-12.2070000000001\\
1331	-14.6479999999999\\
1332	-18.3109999999999\\
1333	-20.752\\
1334	-13.4280000000001\\
1335	-12.2070000000001\\
1336	-15.8689999999999\\
1337	-23.193\\
1338	-20.752\\
1339	-21.973\\
1340	-14.6479999999999\\
1341	-14.6479999999999\\
1343	-9.76600000000008\\
1344	-4.88300000000004\\
1345	-3.66200000000003\\
1346	-3.66200000000003\\
1347	-7.32400000000007\\
1348	-13.4280000000001\\
1349	-14.6479999999999\\
1350	-9.76600000000008\\
1351	-12.2070000000001\\
1352	-13.4280000000001\\
1353	-9.76600000000008\\
1354	-8.54500000000007\\
1355	-8.54500000000007\\
1356	-6.10400000000004\\
1357	-9.76600000000008\\
1358	-14.6479999999999\\
1359	-15.8689999999999\\
1360	-10.9860000000001\\
1361	-8.54500000000007\\
1363	-8.54500000000007\\
1364	-7.32400000000007\\
1365	-10.9860000000001\\
1366	-20.752\\
1367	-18.3109999999999\\
1368	-18.3109999999999\\
1369	-24.414\\
1370	-23.193\\
1371	-18.3109999999999\\
1372	-14.6479999999999\\
1374	-14.6479999999999\\
1376	-19.5309999999999\\
1377	-19.5309999999999\\
1378	-25.635\\
1379	-26.855\\
1380	-31.7380000000001\\
1381	-24.414\\
1383	-26.855\\
1384	-21.973\\
1385	-24.414\\
1386	-20.752\\
1387	-13.4280000000001\\
1388	-9.76600000000008\\
1389	-9.76600000000008\\
1390	-12.2070000000001\\
1392	-7.32400000000007\\
1393	-10.9860000000001\\
1395	-6.10400000000004\\
1396	-7.32400000000007\\
1397	-9.76600000000008\\
1398	-6.10400000000004\\
1399	-12.2070000000001\\
1400	-14.6479999999999\\
1401	-10.9860000000001\\
1402	-17.0899999999999\\
1403	-24.414\\
1404	-24.414\\
1405	-28.076\\
};
\addlegendentry{True output}

\addplot [color=mycolor2, dashed, line width=2.0pt]
  table[row sep=crcr]{%
1006	-19.1871419588074\\
1007	-20.2170256872116\\
1008	-17.7656157699494\\
1009	-11.1116533934828\\
1010	-12.1603887820268\\
1011	-13.6587985830529\\
1012	-14.5359354288282\\
1013	-13.6960986312022\\
1014	-6.39674896463907\\
1015	-5.30978301921891\\
1016	-4.41165996277505\\
1017	-8.85448981018226\\
1018	-15.6474567597757\\
1019	-15.1890751492519\\
1020	-15.8478001617268\\
1021	-12.9597209762608\\
1022	-9.35724642547029\\
1023	-15.9884431788246\\
1024	-14.870828738721\\
1025	-10.1917655877414\\
1026	-14.6469178733025\\
1027	-15.6165205088018\\
1028	-11.3021244306881\\
1029	-16.3846198209856\\
1030	-17.6101041219652\\
1031	-13.6813226430804\\
1032	-18.3517851747495\\
1033	-19.4024094229137\\
1034	-15.2048159201363\\
1035	-13.9470283760816\\
1036	-12.5328293329912\\
1037	-12.4325061356897\\
1038	-15.1061312075722\\
1039	-14.7459706488548\\
1040	-14.0976343104767\\
1041	-14.7359286310905\\
1042	-15.5691664450028\\
1043	-19.9876024195914\\
1044	-19.7134984782649\\
1045	-14.1461320159465\\
1046	-13.3741841940898\\
1047	-9.62302373440184\\
1048	-8.37144719965067\\
1049	-11.8384458875828\\
1050	-11.2810886873463\\
1051	-11.0015801341203\\
1052	-13.6125403828021\\
1053	-15.2346400792774\\
1054	-13.7190317744546\\
1055	-18.7448521765066\\
1056	-16.9596752019934\\
1057	-9.44556636993957\\
1058	-8.9509614392914\\
1059	-10.4784603324104\\
1060	-12.0585063604306\\
1061	-7.78420006620127\\
1062	-6.29344667862165\\
1063	-10.0400108618151\\
1064	-10.2429935263383\\
1065	-7.44130979115243\\
1066	-8.84925775870829\\
1067	-10.4477172692546\\
1068	-14.0535394606468\\
1069	-15.0216371283334\\
1070	-15.6587699626061\\
1071	-15.3978865405622\\
1072	-17.0029573252998\\
1073	-15.6222121036674\\
1074	-13.6545796996631\\
1075	-13.3428095976433\\
1076	-12.7809267628199\\
1077	-12.3619807205084\\
1078	-18.0880836835793\\
1079	-26.5818709337407\\
1080	-27.0413182320337\\
1081	-26.8295800779952\\
1082	-18.4692563520625\\
1083	-24.7595630474277\\
1084	-29.76354088012\\
1085	-28.7113279020266\\
1086	-24.7308893021577\\
1087	-29.2461420345176\\
1088	-35.8510028950855\\
1089	-28.9559526520502\\
1090	-25.9993383491171\\
1091	-18.8939503937315\\
1092	-14.8654169979318\\
1093	-13.1651126185943\\
1094	-14.6033023615125\\
1095	-13.1969225000482\\
1096	-7.7927588503062\\
1097	-7.03966033352413\\
1098	-9.03149144674512\\
1099	-13.7695109520519\\
1100	-13.7961738648171\\
1101	-13.0381579860218\\
1102	-15.3949668476289\\
1103	-16.2522336780112\\
1104	-21.8551992447619\\
1105	-19.6664173609151\\
1106	-19.8557553010762\\
1108	-13.9970647514028\\
1109	-17.386515086625\\
1110	-14.4420470618086\\
1111	-11.8403852439535\\
1112	-14.3575036948728\\
1113	-15.028451189168\\
1114	-12.4013816350168\\
1115	-11.3910957192049\\
1116	-10.1630575764775\\
1117	-9.2215521732121\\
1118	-11.5295459358549\\
1119	-8.41136653499484\\
1120	-8.82649699046215\\
1121	-12.0844777053885\\
1122	-16.3319668092438\\
1123	-16.8342096878853\\
1124	-16.8724318940233\\
1125	-12.7200552613563\\
1126	-15.215621236664\\
1127	-21.9705716767301\\
1128	-17.3973029026974\\
1129	-18.3786606102021\\
1130	-19.5005026749957\\
1131	-13.7664163438101\\
1132	-10.7033903251117\\
1133	-12.5337637099551\\
1134	-14.4147370660435\\
1135	-20.2487332788594\\
1136	-24.8991645912172\\
1137	-23.5429165709904\\
1138	-20.2095309212309\\
1139	-19.8805989308853\\
1140	-19.7073348892309\\
1141	-17.8876973726747\\
1142	-17.5407497776844\\
1143	-13.4687241311372\\
1144	-11.5530069326619\\
1145	-11.8104150276563\\
1146	-10.9616300459645\\
1147	-11.3083367679678\\
1148	-15.1177287845992\\
1149	-21.2685539742379\\
1150	-19.0785743362892\\
1151	-13.1988854608785\\
1152	-12.0004519129875\\
1153	-11.8126283002434\\
1154	-8.27959414594807\\
1155	-6.1315761103258\\
1156	-6.87758871347705\\
1157	-11.1362333870645\\
1158	-10.5594421863736\\
1159	-8.79053941239727\\
1160	-10.3152636662169\\
1161	-9.48955762203673\\
1162	-7.7032063565016\\
1163	-5.77465401635754\\
1164	-5.08817003390504\\
1165	-6.66504221256537\\
1166	-13.4546930147901\\
1167	-18.3599300190735\\
1168	-19.7842062358252\\
1169	-14.1172749052453\\
1170	-11.2390782346008\\
1171	-9.27843382852188\\
1172	-7.15262494937724\\
1173	-10.4794667357355\\
1174	-12.5480349036673\\
1175	-13.1850211312033\\
1176	-17.484712432898\\
1177	-26.124626939275\\
1178	-31.0884369442597\\
1179	-26.0717558892593\\
1180	-26.1453996404691\\
1181	-20.1211532832765\\
1182	-18.9279321806027\\
1183	-21.2045704154641\\
1184	-22.0735756444856\\
1185	-19.0817990755227\\
1186	-17.087106974577\\
1187	-17.3187782124573\\
1188	-15.4272757227343\\
1189	-17.9249784418987\\
1190	-14.7570864412553\\
1191	-19.2103218440543\\
1192	-24.2240511973289\\
1193	-18.4979187099789\\
1194	-13.3786200443299\\
1195	-14.0793719395567\\
1196	-20.5886970632719\\
1197	-24.4705229838714\\
1198	-27.6300212257611\\
1199	-29.3232568211226\\
1200	-28.8480493498178\\
1201	-23.4579925317764\\
1202	-22.0340749036579\\
1203	-23.8797520480987\\
1204	-26.0523998030485\\
1205	-18.9915092707301\\
1206	-13.0648079657076\\
1207	-16.4500329413406\\
1208	-16.0464596755999\\
1209	-10.6078374610827\\
1211	-15.5337465762914\\
1212	-12.7076680015541\\
1213	-10.0335477650378\\
1214	-12.2634749347155\\
1215	-12.035667290522\\
1216	-13.3079465590658\\
1217	-17.6383091743601\\
1218	-15.9973200128068\\
1219	-12.3811012690517\\
1220	-12.6357811026005\\
1221	-17.7705454104878\\
1222	-26.1707957861584\\
1223	-21.8427753067137\\
1224	-13.6246150203469\\
1225	-12.0144730097929\\
1226	-11.8426292923646\\
1227	-13.1381025237715\\
1228	-10.0088557214663\\
1229	-10.4202584838783\\
1230	-11.5862550145189\\
1231	-14.1144202498979\\
1232	-15.2987815254107\\
1233	-10.8569718416352\\
1235	-19.1853539934787\\
1236	-18.2103876308449\\
1237	-15.4118042200139\\
1238	-15.6990763807078\\
1239	-19.874900626231\\
1240	-14.5558949934493\\
1241	-7.59548378620843\\
1242	-9.08738676541611\\
1243	-12.0947702534559\\
1244	-14.7935114451923\\
1245	-13.6826700228062\\
1246	-9.78439756576427\\
1247	-12.8016445339695\\
1248	-13.7589804641566\\
1249	-10.4052113671696\\
1250	-12.1766695952999\\
1251	-12.2424330556153\\
1252	-10.7242345588486\\
1253	-11.0944620518821\\
1254	-8.42501181828447\\
1255	-8.40153828032953\\
1256	-8.46679364826582\\
1257	-8.27840541825162\\
1258	-12.5379456065812\\
1259	-14.4627150826159\\
1260	-17.1844157382684\\
1261	-22.5671761971155\\
1262	-16.7993553844356\\
1263	-10.7586342256482\\
1264	-9.18223025572979\\
1265	-8.8289229035488\\
1266	-11.4766253543758\\
1267	-8.93516855437383\\
1268	-8.32820076509142\\
1269	-10.4751718104978\\
1270	-15.6297852068371\\
1271	-15.5215297076782\\
1272	-18.0322133362645\\
1273	-14.7465759674878\\
1274	-12.8625909343798\\
1275	-9.55056282464807\\
1277	-7.36475498932077\\
1278	-6.32731382428051\\
1279	-7.59396684685953\\
1280	-11.4361763576183\\
1281	-11.2921909872039\\
1282	-12.0619246569565\\
1283	-13.1628509185407\\
1284	-19.500077413825\\
1285	-19.6181069249717\\
1286	-13.5035553322548\\
1287	-16.1033191169345\\
1288	-17.0349532747402\\
1289	-21.1547348492518\\
1290	-18.5153189991593\\
1291	-13.0973065041492\\
1292	-8.6078188111569\\
1293	-6.68845975705108\\
1294	-6.62681660263161\\
1295	-7.53186000127539\\
1296	-7.64505445099985\\
1297	-7.20296019003581\\
1298	-8.34502401179634\\
1299	-12.1819961613357\\
1300	-12.039550679686\\
1301	-11.0341671020867\\
1302	-13.0925794322427\\
1304	-5.96226263850622\\
1305	-9.11415114869715\\
1306	-14.4966249730574\\
1307	-13.905075615899\\
1308	-13.8750598705806\\
1309	-13.7018129109208\\
1310	-11.0236543737478\\
1311	-13.0693288982905\\
1312	-17.6485264264913\\
1313	-17.7704232821425\\
1314	-13.3312306358166\\
1315	-20.0057140699221\\
1316	-18.2782567011914\\
1317	-15.2030288631515\\
1318	-18.2158233918117\\
1319	-17.4591479568057\\
1320	-14.8753764019204\\
1321	-13.0525843823887\\
1322	-17.9831664463597\\
1323	-25.2132269197211\\
1324	-20.1905459947166\\
1325	-13.2114817443785\\
1326	-13.2207951904402\\
1328	-16.0993801265101\\
1329	-17.9857559411334\\
1330	-14.2163613117\\
1331	-14.5800593157185\\
1332	-18.017283456858\\
1333	-19.1268841769077\\
1334	-14.6356612887394\\
1335	-12.4468035524621\\
1336	-15.1043632565897\\
1337	-22.4230630610266\\
1338	-20.5653465266682\\
1339	-20.0671788949337\\
1340	-14.2735359348492\\
1341	-12.8289238724628\\
1342	-13.9153665006304\\
1343	-9.26837705568528\\
1344	-7.2498113266995\\
1345	-5.76481775215575\\
1346	-4.90080319641697\\
1347	-8.05691282372641\\
1348	-12.5561379453843\\
1349	-13.7461697350102\\
1350	-10.9227717801427\\
1351	-11.4171030795533\\
1352	-13.1438184007889\\
1353	-9.86018409690314\\
1354	-9.31945660436895\\
1355	-10.1058983030359\\
1356	-8.26765629044075\\
1357	-8.34436021204556\\
1358	-12.6212005823656\\
1359	-16.1094838406445\\
1360	-11.2873165511239\\
1361	-9.12509792703668\\
1362	-8.62938482960135\\
1363	-9.73509092646964\\
1364	-7.66012346533489\\
1365	-12.8564694808319\\
1366	-19.8979665766578\\
1367	-15.9672230675935\\
1368	-20.0749125696009\\
1369	-24.2606181184306\\
1370	-22.9559854202935\\
1371	-18.5967460484794\\
1372	-14.672648537673\\
1373	-14.55015625434\\
1374	-15.1040316645303\\
1375	-16.6294203030441\\
1376	-18.4831573222189\\
1377	-18.967462976553\\
1378	-22.2579849355054\\
1379	-25.9176692936812\\
1380	-29.7103674737239\\
1381	-23.2986423491152\\
1382	-23.2997483812678\\
1383	-27.2513112446782\\
1384	-22.6171498977515\\
1385	-23.2122927840994\\
1386	-21.6628008564071\\
1387	-12.5119737069303\\
1388	-9.99909399694775\\
1389	-10.004427696377\\
1390	-14.1165329737285\\
1391	-12.6300392434557\\
1392	-8.26778990140997\\
1393	-10.2320143441889\\
1394	-9.38470482461025\\
1395	-6.8958969503758\\
1396	-8.38317148766873\\
1397	-9.69936351612523\\
1398	-7.73411441576695\\
1399	-10.9974344805216\\
1400	-15.8072748688385\\
1401	-12.0994416220224\\
1402	-16.4636782683135\\
1403	-24.0906885235286\\
1404	-22.3005738832203\\
1405	-26.3466451626552\\
};
\addlegendentry{OSA predition}

\addplot [color=mycolor3, dotted, line width=2.0pt]
  table[row sep=crcr]{%
1006	-18.3109999999999\\
1007	-23.193\\
1008	-18.3109999999999\\
1009	-9.76600000000008\\
1010	-12.160388782027\\
1011	-13.093176870166\\
1012	-14.1665198579817\\
1013	-13.1280842127801\\
1014	-6.64630453244968\\
1015	-6.08570920220427\\
1016	-5.68281081067425\\
1017	-10.0893831355354\\
1018	-17.0465737442628\\
1019	-16.2497043649435\\
1020	-16.6584322576093\\
1021	-13.3188214169186\\
1022	-9.24834446102227\\
1023	-15.7732496511994\\
1024	-14.4026430971803\\
1025	-9.74185140648865\\
1026	-13.9408541967307\\
1027	-15.3902376637623\\
1028	-11.6550140008144\\
1029	-16.9634412531404\\
1030	-17.7525589901629\\
1031	-13.1573404699327\\
1032	-17.5880721491633\\
1033	-18.6404445442631\\
1034	-14.5821305493932\\
1035	-12.9657402007681\\
1036	-11.8094913591333\\
1037	-12.3955836955915\\
1038	-14.9083441391861\\
1039	-14.8075203698631\\
1040	-13.861250495863\\
1041	-14.5979567993452\\
1042	-15.4914327909407\\
1043	-19.7880539623723\\
1044	-19.3411677723147\\
1045	-13.5939434572563\\
1046	-12.8139897432409\\
1047	-9.25717899149981\\
1048	-8.37575797555451\\
1049	-11.2162452161915\\
1050	-10.70710759244\\
1051	-9.89268957917966\\
1052	-12.7724408675772\\
1053	-14.5129368157268\\
1054	-13.3395052837293\\
1055	-18.5591449475467\\
1056	-16.6066726648621\\
1057	-9.60842131187565\\
1058	-9.03056439293005\\
1059	-11.2127300301336\\
1060	-12.8114526943939\\
1061	-8.65523007310617\\
1062	-7.39752008692858\\
1063	-10.3928597399417\\
1064	-10.2918445670982\\
1065	-7.82563291919109\\
1066	-9.49364798609463\\
1067	-11.2452451667443\\
1068	-14.682722309517\\
1069	-15.055506379191\\
1070	-15.7907033423037\\
1071	-15.1242415860952\\
1072	-16.7427108000948\\
1073	-15.0947511852869\\
1074	-13.5211615548949\\
1075	-13.1085969651226\\
1076	-12.9889351367099\\
1077	-12.6760786500747\\
1078	-18.5542824937827\\
1079	-26.4415102300288\\
1080	-26.4128390490632\\
1081	-25.3818095281079\\
1082	-17.0144004625265\\
1083	-23.0653318631039\\
1084	-28.0246553335887\\
1085	-27.1854344602677\\
1086	-22.5856230933844\\
1087	-27.5416132643102\\
1088	-34.3918908431676\\
1089	-27.9624694852025\\
1090	-23.7034072004465\\
1091	-17.2325176711299\\
1092	-13.9676559652564\\
1093	-12.348676977269\\
1094	-13.8772384938804\\
1095	-12.6777564308327\\
1096	-8.0988078703283\\
1097	-7.46374217551033\\
1098	-9.63682947928896\\
1099	-13.8146966067109\\
1100	-13.7318525915041\\
1101	-13.11642506154\\
1102	-15.7576234550281\\
1103	-16.6357312725579\\
1104	-21.8229018373079\\
1105	-19.7488059263785\\
1106	-20.4122579120572\\
1108	-14.7920482255897\\
1109	-17.3243472062545\\
1110	-14.5679394522465\\
1111	-12.1232983246134\\
1112	-14.6800285201093\\
1113	-15.2660684160564\\
1114	-12.4069752336841\\
1115	-11.8970146883159\\
1116	-10.7027068711766\\
1117	-10.1737123075227\\
1118	-11.9797744189352\\
1119	-9.12645198350401\\
1120	-9.56025530389593\\
1121	-12.6783541316017\\
1122	-16.7596452319458\\
1123	-16.8826346569126\\
1124	-16.8689669134817\\
1125	-12.6248009877509\\
1126	-15.394759470322\\
1127	-22.6356333905392\\
1128	-18.0345731501623\\
1129	-19.2409010615795\\
1130	-19.6252134086155\\
1131	-13.5905708943137\\
1132	-10.425931250549\\
1133	-12.3541946357161\\
1134	-14.6909495700538\\
1135	-20.6315543150779\\
1136	-24.9493681334247\\
1137	-23.5153726097235\\
1138	-19.5488696558266\\
1139	-19.2119798444535\\
1140	-18.832918411227\\
1141	-17.6135833822905\\
1142	-17.4025916352668\\
1143	-13.4485158458704\\
1144	-11.5341165607745\\
1145	-11.5689083101954\\
1146	-11.1267456685855\\
1147	-11.6614807204894\\
1148	-15.704307464867\\
1149	-21.6440143969626\\
1150	-19.03264632174\\
1151	-13.2563016979889\\
1152	-11.9862770372335\\
1153	-12.2290594199562\\
1154	-8.66319261354693\\
1155	-6.65327138384237\\
1156	-7.37884496224819\\
1157	-11.6960261415677\\
1158	-10.7692536091945\\
1159	-9.49499335508131\\
1160	-10.8854893387756\\
1161	-10.4921105867827\\
1162	-8.58731501189754\\
1163	-6.5791576141271\\
1164	-5.87586808935134\\
1165	-7.52066583667079\\
1166	-13.8718447269607\\
1167	-18.2983517536195\\
1168	-19.479359066416\\
1169	-13.5958220933783\\
1170	-10.6453938258455\\
1172	-6.7861317342456\\
1173	-10.6760956547178\\
1174	-12.9537728139553\\
1175	-14.2919777854831\\
1176	-18.0338097200029\\
1177	-26.8128491782552\\
1178	-30.8765543507177\\
1179	-25.6921683069502\\
1180	-25.3963858169429\\
1181	-19.6121369926072\\
1182	-19.002660516951\\
1183	-20.8158404845396\\
1184	-21.562170158714\\
1185	-17.8408445240802\\
1186	-16.5810327660276\\
1187	-16.9447566402021\\
1188	-15.5879423075421\\
1189	-17.8398406347296\\
1190	-14.9558711611021\\
1192	-24.3747976642362\\
1193	-18.2103520188907\\
1194	-12.4877654293391\\
1195	-13.3743023496811\\
1196	-19.9348159468302\\
1197	-23.6541784867138\\
1198	-26.8221942067421\\
1199	-28.407971790622\\
1200	-28.2046258108107\\
1201	-22.7651454635381\\
1202	-21.3445265189512\\
1203	-23.4224219557138\\
1204	-25.5762048442903\\
1205	-18.2742672806894\\
1206	-12.2893979482392\\
1207	-15.8620985011542\\
1208	-15.4398610022474\\
1209	-10.2205549405173\\
1210	-12.5963328911994\\
1211	-15.4778200186315\\
1212	-12.8235656132676\\
1213	-10.3222706710594\\
1214	-12.5258264988813\\
1215	-12.527765962898\\
1216	-14.111621507272\\
1217	-18.9135620069012\\
1218	-17.2318710272091\\
1219	-13.1439416869869\\
1220	-13.0228452744561\\
1221	-18.0794594110973\\
1222	-26.4331664749645\\
1223	-21.9472743960659\\
1224	-14.210007816635\\
1225	-12.5816709971864\\
1226	-12.9019687940156\\
1227	-13.3868383353044\\
1228	-10.5511807753758\\
1229	-11.0476455712908\\
1230	-12.6570853447747\\
1231	-15.1463491303009\\
1232	-16.2790910923788\\
1233	-11.6857566301364\\
1235	-19.4294814865932\\
1236	-17.9616114690639\\
1237	-15.2760646275708\\
1238	-16.0419323242577\\
1239	-20.4517903889168\\
1240	-15.0964183408723\\
1241	-7.94660920546721\\
1242	-9.38519136703303\\
1243	-11.8248911241926\\
1244	-14.5628602128304\\
1245	-13.5402687692635\\
1246	-10.6678757182681\\
1247	-14.0969798650808\\
1248	-14.8733464180361\\
1249	-11.1839129129214\\
1250	-12.7921223796045\\
1251	-12.7898841975082\\
1252	-11.5637779281517\\
1253	-12.3401871430879\\
1254	-9.48349053331867\\
1255	-9.3432147826345\\
1256	-8.70230092589736\\
1257	-8.59722052419534\\
1258	-13.3017028662903\\
1259	-15.3029469759551\\
1260	-17.8653071854208\\
1261	-22.9271119096265\\
1262	-16.8185038358122\\
1263	-10.9580890765008\\
1264	-9.49907332490807\\
1265	-9.29611482996393\\
1266	-11.9707629794643\\
1267	-9.37348685557117\\
1268	-8.73842770233387\\
1269	-11.4180917761983\\
1270	-16.3679653788063\\
1271	-16.2829993679691\\
1272	-18.7411636923273\\
1273	-14.9622640453977\\
1274	-12.9391135123774\\
1275	-8.97771441809709\\
1276	-8.08063871838499\\
1277	-6.69041451398243\\
1278	-6.24961264037142\\
1279	-7.84514176886046\\
1280	-12.4183275606185\\
1281	-12.5056488636374\\
1282	-13.1498336575844\\
1283	-14.0314188430302\\
1284	-20.0709602413929\\
1285	-19.823690421335\\
1286	-13.923789482918\\
1287	-16.275033668526\\
1288	-17.2898781141455\\
1289	-20.7015454680309\\
1290	-18.0274162638609\\
1291	-12.8248207397924\\
1292	-8.55661724243168\\
1293	-6.9287645556999\\
1294	-6.87917114515426\\
1295	-8.14215705405877\\
1296	-8.19436216546478\\
1297	-7.75519053578046\\
1298	-9.12901565691368\\
1299	-12.7970691656035\\
1300	-12.6542766269574\\
1301	-11.5658739618902\\
1302	-13.3238955755255\\
1303	-9.90861840186631\\
1304	-6.31388994000713\\
1305	-9.70676983213502\\
1306	-14.9436435436598\\
1307	-14.0197128696595\\
1308	-13.8932824013218\\
1309	-13.3336619763804\\
1310	-10.8012019427101\\
1311	-12.9113834927909\\
1312	-17.430013231176\\
1313	-17.5853967636863\\
1314	-12.8754155410002\\
1315	-19.5905163725822\\
1316	-17.8871235806641\\
1317	-15.2149862641518\\
1318	-17.9888289439207\\
1319	-17.4701645696261\\
1320	-14.3026395889408\\
1321	-12.7713786787911\\
1322	-17.5967442201556\\
1323	-25.112742113158\\
1324	-19.8267024466061\\
1325	-12.5494097013811\\
1326	-12.4415026742286\\
1327	-13.9198437604416\\
1328	-15.7976195506744\\
1329	-17.8342243533757\\
1330	-14.3575486456584\\
1331	-15.0215889603153\\
1332	-18.5138233454504\\
1333	-19.6260054243703\\
1334	-14.6595180714439\\
1335	-12.5238013009161\\
1336	-15.1264595013249\\
1337	-22.5576967181862\\
1338	-20.4894929237441\\
1339	-19.8243656803911\\
1340	-13.8227006621612\\
1341	-12.3243659190066\\
1342	-13.038538894152\\
1343	-8.86905169919964\\
1344	-6.85923297509999\\
1345	-5.99709791699547\\
1346	-5.47173290508317\\
1347	-8.9559253102168\\
1348	-13.4695611921159\\
1349	-14.2803767910452\\
1350	-11.1765737798212\\
1351	-11.6645329716764\\
1352	-13.1868331008986\\
1353	-9.94661362126772\\
1354	-9.28815852845037\\
1355	-10.1933923203562\\
1356	-8.65016394823874\\
1357	-9.09894606854664\\
1358	-13.1370423874935\\
1359	-16.2536394471497\\
1360	-11.1494169070247\\
1361	-8.95673100895056\\
1362	-8.72008120680061\\
1363	-9.85894122507034\\
1364	-7.9608291005718\\
1365	-13.173347533741\\
1366	-20.5454474464157\\
1367	-16.3591651478821\\
1368	-20.1560611360508\\
1369	-24.2830608101913\\
1370	-22.8341613052078\\
1372	-14.7339943945019\\
1373	-14.5631895418674\\
1374	-15.1794995414834\\
1375	-16.7312550188881\\
1376	-18.4931898270647\\
1377	-18.8689259868067\\
1378	-21.9749386116566\\
1379	-25.1833268583498\\
1380	-28.841804656083\\
1381	-22.0944648517773\\
1382	-22.1262208384726\\
1383	-25.8036187988507\\
1384	-21.4183196396896\\
1385	-22.1951171107748\\
1386	-20.9728195828741\\
1387	-12.078481652271\\
1388	-9.45447544254171\\
1389	-9.7095166476538\\
1390	-13.813949802216\\
1391	-12.7696970859547\\
1392	-8.9408113801594\\
1393	-11.1807102620453\\
1394	-10.1639843071769\\
1395	-7.53074585575996\\
1396	-8.92849375706919\\
1397	-10.4123862523791\\
1398	-8.30602530279498\\
1399	-11.7578389211528\\
1400	-16.1852841908103\\
1401	-12.6250735821445\\
1402	-16.9318716696046\\
1403	-24.5517461029542\\
1404	-22.6334384626007\\
1405	-26.1890427368044\\
};
\addlegendentry{MPO prediction}

\end{axis}

\begin{axis}[%
width=6.159cm,
height=1.831cm,
at={(8.104cm,7.627cm)},
scale only axis,
xmin=1000,
xmax=1405,
xlabel style={font=\color{white!15!black}},
xlabel={Sample index},
ymin=-26.855,
ymax=0,
ylabel style={font=\color{white!15!black}},
ylabel={$y(t)$},
axis background/.style={fill=white},
title style={font=\bfseries},
title={C4: RMSE(OSA) = 1.501, RMSE(MPO) = 1.4703},
legend style={legend cell align=left, align=left, draw=white!15!black}
]
\addplot [color=mycolor1, line width=2.0pt]
  table[row sep=crcr]{%
1006	-14.6479999999999\\
1007	-17.0899999999999\\
1008	-10.9860000000001\\
1009	-7.32400000000007\\
1010	-14.6479999999999\\
1011	-8.54500000000007\\
1012	-9.76600000000008\\
1013	-6.10400000000004\\
1014	-3.66200000000003\\
1016	-3.66200000000003\\
1017	-2.44100000000003\\
1018	-13.4280000000001\\
1020	-13.4280000000001\\
1021	-9.76600000000008\\
1022	-7.32400000000007\\
1023	-10.9860000000001\\
1024	-12.2070000000001\\
1025	-6.10400000000004\\
1027	-13.4280000000001\\
1028	-7.32400000000007\\
1029	-15.8689999999999\\
1031	-10.9860000000001\\
1032	-17.0899999999999\\
1033	-13.4280000000001\\
1035	-8.54500000000007\\
1036	-8.54500000000007\\
1037	-10.9860000000001\\
1038	-12.2070000000001\\
1040	-9.76600000000008\\
1041	-10.9860000000001\\
1042	-10.9860000000001\\
1043	-15.8689999999999\\
1044	-14.6479999999999\\
1045	-10.9860000000001\\
1046	-10.9860000000001\\
1047	-6.10400000000004\\
1048	-4.88300000000004\\
1049	-8.54500000000007\\
1050	-7.32400000000007\\
1051	-7.32400000000007\\
1052	-10.9860000000001\\
1053	-9.76600000000008\\
1054	-9.76600000000008\\
1055	-15.8689999999999\\
1056	-8.54500000000007\\
1057	-6.10400000000004\\
1060	-9.76600000000008\\
1061	-4.88300000000004\\
1062	-7.32400000000007\\
1064	-7.32400000000007\\
1065	-6.10400000000004\\
1069	-10.9860000000001\\
1070	-10.9860000000001\\
1072	-13.4280000000001\\
1073	-10.9860000000001\\
1074	-12.2070000000001\\
1075	-9.76600000000008\\
1077	-9.76600000000008\\
1078	-17.0899999999999\\
1079	-20.752\\
1080	-19.5309999999999\\
1081	-19.5309999999999\\
1082	-15.8689999999999\\
1083	-20.752\\
1084	-23.193\\
1085	-23.193\\
1086	-14.6479999999999\\
1087	-23.193\\
1088	-26.855\\
1089	-21.973\\
1090	-15.8689999999999\\
1091	-14.6479999999999\\
1092	-10.9860000000001\\
1093	-8.54500000000007\\
1094	-10.9860000000001\\
1096	-6.10400000000004\\
1097	-6.10400000000004\\
1098	-8.54500000000007\\
1100	-10.9860000000001\\
1101	-9.76600000000008\\
1102	-12.2070000000001\\
1103	-10.9860000000001\\
1104	-15.8689999999999\\
1105	-13.4280000000001\\
1106	-18.3109999999999\\
1107	-12.2070000000001\\
1108	-10.9860000000001\\
1109	-12.2070000000001\\
1110	-9.76600000000008\\
1111	-8.54500000000007\\
1112	-10.9860000000001\\
1113	-10.9860000000001\\
1114	-8.54500000000007\\
1115	-8.54500000000007\\
1116	-7.32400000000007\\
1118	-7.32400000000007\\
1119	-4.88300000000004\\
1120	-6.10400000000004\\
1121	-8.54500000000007\\
1122	-13.4280000000001\\
1123	-12.2070000000001\\
1124	-13.4280000000001\\
1125	-8.54500000000007\\
1126	-15.8689999999999\\
1127	-17.0899999999999\\
1128	-13.4280000000001\\
1129	-14.6479999999999\\
1130	-14.6479999999999\\
1132	-7.32400000000007\\
1133	-8.54500000000007\\
1134	-8.54500000000007\\
1135	-17.0899999999999\\
1136	-20.752\\
1137	-19.5309999999999\\
1138	-14.6479999999999\\
1139	-15.8689999999999\\
1140	-13.4280000000001\\
1141	-12.2070000000001\\
1142	-12.2070000000001\\
1143	-9.76600000000008\\
1144	-8.54500000000007\\
1145	-8.54500000000007\\
1146	-9.76600000000008\\
1147	-8.54500000000007\\
1148	-13.4280000000001\\
1149	-15.8689999999999\\
1150	-10.9860000000001\\
1151	-8.54500000000007\\
1152	-9.76600000000008\\
1153	-8.54500000000007\\
1155	-3.66200000000003\\
1156	-4.88300000000004\\
1157	-8.54500000000007\\
1158	-9.76600000000008\\
1159	-4.88300000000004\\
1160	-7.32400000000007\\
1161	-7.32400000000007\\
1162	-3.66200000000003\\
1163	-3.66200000000003\\
1164	-2.44100000000003\\
1165	-4.88300000000004\\
1166	-10.9860000000001\\
1167	-12.2070000000001\\
1168	-14.6479999999999\\
1169	-13.4280000000001\\
1170	-8.54500000000007\\
1171	-7.32400000000007\\
1172	-4.88300000000004\\
1173	-7.32400000000007\\
1174	-6.10400000000004\\
1175	-10.9860000000001\\
1176	-12.2070000000001\\
1177	-21.973\\
1178	-23.193\\
1179	-21.973\\
1180	-18.3109999999999\\
1181	-13.4280000000001\\
1182	-17.0899999999999\\
1183	-15.8689999999999\\
1184	-19.5309999999999\\
1185	-12.2070000000001\\
1186	-13.4280000000001\\
1187	-12.2070000000001\\
1188	-13.4280000000001\\
1189	-13.4280000000001\\
1190	-10.9860000000001\\
1191	-18.3109999999999\\
1192	-17.0899999999999\\
1193	-13.4280000000001\\
1194	-7.32400000000007\\
1195	-12.2070000000001\\
1196	-15.8689999999999\\
1197	-17.0899999999999\\
1199	-21.973\\
1200	-20.752\\
1201	-15.8689999999999\\
1202	-14.6479999999999\\
1203	-18.3109999999999\\
1204	-19.5309999999999\\
1205	-13.4280000000001\\
1206	-9.76600000000008\\
1207	-13.4280000000001\\
1208	-9.76600000000008\\
1209	-7.32400000000007\\
1210	-9.76600000000008\\
1211	-10.9860000000001\\
1212	-7.32400000000007\\
1213	-7.32400000000007\\
1214	-9.76600000000008\\
1215	-6.10400000000004\\
1216	-10.9860000000001\\
1217	-13.4280000000001\\
1218	-13.4280000000001\\
1219	-8.54500000000007\\
1220	-9.76600000000008\\
1221	-12.2070000000001\\
1222	-18.3109999999999\\
1223	-15.8689999999999\\
1224	-9.76600000000008\\
1225	-8.54500000000007\\
1226	-9.76600000000008\\
1227	-7.32400000000007\\
1228	-8.54500000000007\\
1229	-7.32400000000007\\
1230	-8.54500000000007\\
1231	-10.9860000000001\\
1232	-10.9860000000001\\
1233	-6.10400000000004\\
1234	-10.9860000000001\\
1235	-13.4280000000001\\
1236	-13.4280000000001\\
1237	-9.76600000000008\\
1238	-12.2070000000001\\
1239	-13.4280000000001\\
1240	-10.9860000000001\\
1241	-4.88300000000004\\
1242	-10.9860000000001\\
1243	-8.54500000000007\\
1244	-8.54500000000007\\
1245	-6.10400000000004\\
1246	-6.10400000000004\\
1247	-10.9860000000001\\
1248	-10.9860000000001\\
1249	-6.10400000000004\\
1250	-8.54500000000007\\
1251	-8.54500000000007\\
1252	-4.88300000000004\\
1253	-8.54500000000007\\
1254	-4.88300000000004\\
1255	-7.32400000000007\\
1256	-4.88300000000004\\
1257	-4.88300000000004\\
1258	-10.9860000000001\\
1259	-12.2070000000001\\
1260	-12.2070000000001\\
1261	-15.8689999999999\\
1263	-6.10400000000004\\
1264	-7.32400000000007\\
1265	-4.88300000000004\\
1266	-7.32400000000007\\
1267	-6.10400000000004\\
1268	-3.66200000000003\\
1269	-8.54500000000007\\
1270	-9.76600000000008\\
1271	-9.76600000000008\\
1272	-17.0899999999999\\
1273	-10.9860000000001\\
1274	-13.4280000000001\\
1275	-7.32400000000007\\
1276	-8.54500000000007\\
1277	-3.66200000000003\\
1278	-1.221\\
1279	-6.10400000000004\\
1280	-9.76600000000008\\
1281	-8.54500000000007\\
1282	-9.76600000000008\\
1283	-9.76600000000008\\
1284	-17.0899999999999\\
1285	-10.9860000000001\\
1287	-10.9860000000001\\
1289	-15.8689999999999\\
1290	-15.8689999999999\\
1292	-6.10400000000004\\
1293	-3.66200000000003\\
1295	-6.10400000000004\\
1296	-4.88300000000004\\
1297	-6.10400000000004\\
1298	-6.10400000000004\\
1299	-9.76600000000008\\
1300	-9.76600000000008\\
1301	-8.54500000000007\\
1302	-9.76600000000008\\
1303	-6.10400000000004\\
1304	-3.66200000000003\\
1306	-10.9860000000001\\
1307	-8.54500000000007\\
1308	-12.2070000000001\\
1309	-8.54500000000007\\
1310	-7.32400000000007\\
1311	-8.54500000000007\\
1312	-12.2070000000001\\
1313	-14.6479999999999\\
1314	-10.9860000000001\\
1315	-17.0899999999999\\
1316	-9.76600000000008\\
1317	-12.2070000000001\\
1318	-13.4280000000001\\
1319	-13.4280000000001\\
1320	-9.76600000000008\\
1321	-9.76600000000008\\
1322	-12.2070000000001\\
1323	-18.3109999999999\\
1324	-15.8689999999999\\
1325	-8.54500000000007\\
1326	-12.2070000000001\\
1327	-9.76600000000008\\
1328	-12.2070000000001\\
1329	-13.4280000000001\\
1330	-9.76600000000008\\
1331	-9.76600000000008\\
1332	-14.6479999999999\\
1334	-12.2070000000001\\
1335	-8.54500000000007\\
1336	-9.76600000000008\\
1337	-15.8689999999999\\
1338	-14.6479999999999\\
1339	-15.8689999999999\\
1340	-12.2070000000001\\
1343	-8.54500000000007\\
1344	-4.88300000000004\\
1345	-3.66200000000003\\
1346	-3.66200000000003\\
1348	-10.9860000000001\\
1350	-8.54500000000007\\
1351	-10.9860000000001\\
1353	-8.54500000000007\\
1354	-3.66200000000003\\
1355	-6.10400000000004\\
1357	-6.10400000000004\\
1359	-10.9860000000001\\
1360	-7.32400000000007\\
1361	-6.10400000000004\\
1362	-7.32400000000007\\
1363	-7.32400000000007\\
1364	-6.10400000000004\\
1365	-8.54500000000007\\
1366	-15.8689999999999\\
1367	-14.6479999999999\\
1368	-17.0899999999999\\
1370	-17.0899999999999\\
1371	-13.4280000000001\\
1372	-10.9860000000001\\
1373	-9.76600000000008\\
1375	-12.2070000000001\\
1376	-14.6479999999999\\
1377	-14.6479999999999\\
1380	-21.973\\
1381	-18.3109999999999\\
1382	-21.973\\
1383	-19.5309999999999\\
1384	-14.6479999999999\\
1385	-17.0899999999999\\
1386	-14.6479999999999\\
1387	-10.9860000000001\\
1388	-8.54500000000007\\
1389	-8.54500000000007\\
1390	-10.9860000000001\\
1391	-7.32400000000007\\
1394	-7.32400000000007\\
1395	-3.66200000000003\\
1396	-7.32400000000007\\
1397	-7.32400000000007\\
1398	-6.10400000000004\\
1399	-8.54500000000007\\
1400	-12.2070000000001\\
1401	-8.54500000000007\\
1402	-12.2070000000001\\
1403	-18.3109999999999\\
1404	-17.0899999999999\\
1405	-19.5309999999999\\
};
\addlegendentry{True output}

\addplot [color=mycolor2, dashed, line width=2.0pt]
  table[row sep=crcr]{%
1006	-14.2292537067281\\
1007	-14.7888697203041\\
1008	-13.4363392139035\\
1009	-7.9286579076861\\
1010	-9.15567496596918\\
1011	-9.52642590232381\\
1012	-11.206107571217\\
1013	-10.3550266929815\\
1014	-4.47936290165558\\
1015	-3.87672029792566\\
1016	-2.90150343416394\\
1017	-6.78115289550624\\
1018	-12.8346419557627\\
1019	-12.4665361521752\\
1020	-12.3651208709316\\
1021	-10.3541037452317\\
1022	-7.1139828221983\\
1023	-11.5531085748041\\
1024	-10.4232026288066\\
1025	-6.53528646201812\\
1026	-11.1392116820984\\
1027	-11.9720757392533\\
1028	-8.82571217021791\\
1029	-12.6928475190805\\
1030	-13.1740918589678\\
1031	-9.65521564787809\\
1032	-13.4126879129419\\
1033	-13.817098679564\\
1034	-11.4407088486048\\
1035	-10.1449486247575\\
1036	-8.52513322273808\\
1037	-8.92609227446951\\
1038	-10.9633879419782\\
1039	-10.7786080110643\\
1040	-11.0195090968637\\
1041	-11.2288007390443\\
1042	-11.2036630543057\\
1043	-14.512470582903\\
1044	-14.0124116337317\\
1045	-9.89513533495619\\
1046	-9.66019581288856\\
1047	-6.87799952607065\\
1048	-6.24927090576352\\
1049	-8.43643598075937\\
1050	-7.34704204722448\\
1051	-6.42047509771214\\
1052	-9.16763234284986\\
1053	-10.5855469198123\\
1054	-9.62692243314837\\
1055	-13.2978129765042\\
1056	-12.0662501056272\\
1057	-7.0238109694933\\
1058	-6.68909287692804\\
1059	-6.80237218483217\\
1060	-8.70836447597162\\
1061	-6.09578041928171\\
1062	-5.11719399485401\\
1063	-7.53138908963592\\
1064	-7.11667504723755\\
1065	-5.30007081368649\\
1066	-6.44069960320485\\
1067	-7.85587656120629\\
1068	-11.4869286118555\\
1069	-11.2489357502618\\
1070	-11.9213224713646\\
1071	-11.563091964101\\
1072	-12.3131095466058\\
1073	-10.8292027062421\\
1074	-10.376788417042\\
1075	-10.1278121368732\\
1076	-9.18540076166641\\
1077	-9.64873088330592\\
1078	-13.5786640011893\\
1079	-19.2151042552443\\
1080	-20.4309925051282\\
1081	-19.4011934568148\\
1082	-13.9846018462665\\
1083	-17.5820935659285\\
1084	-21.3914735569547\\
1085	-20.9894371365499\\
1086	-19.2251129801361\\
1087	-20.6567985350146\\
1088	-25.871540397085\\
1089	-20.1040084686842\\
1090	-19.7788734176554\\
1091	-14.0547919030103\\
1092	-10.5020775769694\\
1093	-8.93948635895595\\
1094	-11.9095211130746\\
1095	-9.74022782075849\\
1096	-5.90471713389843\\
1097	-6.00028633532452\\
1098	-7.18321881403131\\
1099	-10.5253796365896\\
1100	-10.6239949520661\\
1101	-9.58476993754812\\
1102	-12.1227085870494\\
1103	-12.396746947778\\
1104	-15.7277964974055\\
1105	-13.998104857161\\
1106	-15.3174934437861\\
1107	-12.9288283530607\\
1108	-11.1973067148197\\
1109	-12.9829019920442\\
1110	-10.1244754782817\\
1111	-9.17638357805026\\
1112	-10.9813025964627\\
1113	-11.2641596204735\\
1114	-8.07178111958888\\
1115	-8.78209716521906\\
1116	-7.3400757490474\\
1117	-6.98962072961626\\
1118	-8.25921154359844\\
1119	-6.53933528702919\\
1120	-6.87248281729808\\
1121	-8.42679818985221\\
1122	-12.1345377799273\\
1123	-13.2318263067621\\
1124	-13.1849452153433\\
1125	-8.86895014546189\\
1126	-10.4700855614419\\
1127	-16.5199350593346\\
1128	-14.1341405450662\\
1129	-14.0662491672206\\
1130	-15.2944942466017\\
1131	-9.76272482503055\\
1132	-7.87360739568294\\
1133	-9.67621835086311\\
1134	-11.4232770819779\\
1135	-15.3539914967366\\
1136	-19.1266233856859\\
1137	-18.6738119604634\\
1138	-14.5209016685303\\
1139	-15.011625102487\\
1140	-14.5340154403361\\
1141	-13.3793567461551\\
1142	-13.3025421126588\\
1143	-9.62030860126197\\
1144	-8.24464968831148\\
1145	-8.39287377115738\\
1146	-8.11105520804358\\
1147	-8.92847612562673\\
1148	-11.6697682023273\\
1149	-16.093087477873\\
1150	-13.9200428072577\\
1151	-10.1933919276889\\
1152	-8.91658967399758\\
1153	-8.48155572090559\\
1154	-5.92440504004321\\
1155	-4.57700238576831\\
1156	-4.78266726750849\\
1157	-8.15687491983908\\
1158	-7.0805856913928\\
1159	-6.1612259347869\\
1160	-7.67345839587915\\
1161	-7.55029988198112\\
1162	-5.50681834519605\\
1163	-4.19905370328888\\
1164	-3.38041113617692\\
1165	-4.08491600793036\\
1166	-9.62703669004009\\
1167	-14.0755446195146\\
1168	-14.696865264699\\
1169	-10.7684982725\\
1170	-7.99602773501988\\
1171	-6.67348339275327\\
1172	-5.32367119440414\\
1173	-7.29279942747394\\
1174	-9.81198104281407\\
1175	-10.4706540381392\\
1176	-12.7525031722078\\
1177	-19.8384004505224\\
1178	-22.918227150124\\
1179	-20.3474579150579\\
1180	-19.9298469468868\\
1181	-14.2655558042409\\
1182	-15.1290099706657\\
1183	-16.3873488583124\\
1184	-15.9699316071353\\
1185	-14.1455177599073\\
1186	-12.8807447747108\\
1187	-13.8902505682761\\
1188	-11.5917954227659\\
1189	-13.2774310707789\\
1190	-11.5954324617232\\
1191	-15.5730271829786\\
1192	-18.8394038952938\\
1193	-13.8816498581666\\
1194	-10.3782253513564\\
1195	-9.92022381071251\\
1196	-15.2767894078243\\
1197	-18.0769476030248\\
1198	-21.0224988363236\\
1199	-21.1126299285631\\
1200	-21.0906852738883\\
1201	-17.3808153708005\\
1202	-16.4248726900355\\
1203	-17.5628424189294\\
1204	-19.119659533992\\
1205	-14.5398965673073\\
1206	-9.80192311144719\\
1207	-12.9378266943684\\
1208	-12.1715982277294\\
1209	-7.98938096083589\\
1210	-9.63340065777425\\
1211	-11.4165954293437\\
1212	-9.21527699490935\\
1213	-8.1150508430635\\
1214	-9.42037161621442\\
1215	-8.77903618911932\\
1216	-10.2836654308715\\
1217	-13.9377230713051\\
1218	-13.055110628118\\
1219	-9.64342812764312\\
1220	-9.49931449758219\\
1221	-13.2486645393858\\
1222	-20.6153290634618\\
1223	-17.2142646588563\\
1224	-10.6267243179461\\
1225	-9.3828858767622\\
1226	-9.77188456379417\\
1227	-9.74099518557841\\
1228	-7.54729372400971\\
1229	-7.79037529350285\\
1230	-9.25226196432845\\
1231	-11.1424200178358\\
1232	-11.5840050447398\\
1233	-9.19877733528665\\
1234	-11.7371298515982\\
1235	-14.420486911348\\
1236	-12.9877044991565\\
1237	-11.2822929772171\\
1238	-12.149052166099\\
1239	-14.5951788410357\\
1240	-10.7169915813411\\
1241	-5.71698502656091\\
1242	-6.61013963880578\\
1243	-8.68365034626413\\
1244	-10.6946467232635\\
1245	-10.1332513797397\\
1246	-6.99716677617562\\
1247	-9.54108861490454\\
1248	-10.3221358313149\\
1249	-7.62969049227968\\
1250	-9.12404324153567\\
1251	-9.03208371465757\\
1252	-7.76341122771987\\
1253	-8.79051266287206\\
1254	-6.44608015066683\\
1255	-5.8890643359066\\
1256	-5.90286882016949\\
1257	-5.1006239783635\\
1258	-9.21018485215723\\
1259	-10.8714062431882\\
1260	-13.3899415019494\\
1261	-16.4322221205612\\
1262	-12.3446746719892\\
1263	-7.69357000215405\\
1264	-6.71202854156968\\
1265	-6.16600513081835\\
1266	-8.558653012994\\
1267	-7.05891728443567\\
1268	-6.25772581592992\\
1269	-7.76575210029955\\
1270	-12.9077781734527\\
1271	-12.5327935549237\\
1272	-14.2362364074163\\
1273	-10.6595119531592\\
1274	-9.37342403819116\\
1275	-7.092914610055\\
1276	-5.90309986920442\\
1277	-5.64558445390639\\
1278	-4.53471942083138\\
1279	-5.44809308147455\\
1280	-8.09365565425128\\
1281	-9.09720717688697\\
1282	-9.20426488603948\\
1283	-9.78168668812918\\
1284	-14.5509421993681\\
1285	-14.2610222821343\\
1286	-10.2742550611081\\
1287	-12.2734030995721\\
1288	-12.4442003833285\\
1289	-14.5890623638497\\
1290	-12.9816121247413\\
1291	-9.91051054613331\\
1292	-6.05807881231885\\
1293	-5.39933640659956\\
1294	-4.72279353237423\\
1295	-5.18232006141648\\
1296	-5.59931864753185\\
1297	-5.23208326753115\\
1298	-6.27409721652907\\
1299	-9.19192379160995\\
1300	-8.80574870329815\\
1301	-7.96543625564595\\
1302	-9.36663975124634\\
1303	-6.7225437143361\\
1304	-4.12729823677682\\
1305	-6.41779486963264\\
1306	-10.8213369269854\\
1307	-10.0090302163976\\
1308	-10.1304524003872\\
1309	-9.93344781374367\\
1310	-7.04180533879048\\
1311	-8.99565311145125\\
1312	-13.3939066286905\\
1313	-12.7301403513688\\
1314	-9.51925453195167\\
1315	-13.7374500880801\\
1316	-13.6390720304655\\
1317	-11.4204654252405\\
1318	-12.9933390653828\\
1319	-12.6064105995024\\
1320	-10.759869990475\\
1321	-9.87222304271972\\
1322	-13.2599974609559\\
1323	-18.0630605728004\\
1324	-15.1474151343816\\
1325	-9.7676796687133\\
1326	-10.0451806857047\\
1327	-10.2955321988341\\
1328	-12.0493146693568\\
1329	-13.5525256465517\\
1330	-10.6650526497306\\
1331	-11.1847731816526\\
1332	-13.2315198034332\\
1333	-14.2281136471527\\
1334	-9.82620868964159\\
1335	-9.33502901839506\\
1336	-11.9984112766574\\
1337	-16.4249940595621\\
1338	-15.5066874717907\\
1339	-15.1995833611672\\
1340	-10.309398523224\\
1341	-9.23332837100838\\
1342	-10.4832640563839\\
1343	-7.339590531574\\
1344	-5.14345936194491\\
1345	-4.18722799976445\\
1346	-3.83265555283629\\
1347	-6.23048202941641\\
1348	-10.1661969674683\\
1349	-10.9263030407637\\
1350	-8.26059971905534\\
1351	-8.18875359654294\\
1352	-9.63553070361559\\
1353	-6.63409783118982\\
1354	-6.69754881880885\\
1355	-7.24577794144716\\
1356	-5.53383929910615\\
1357	-5.70248435984672\\
1358	-9.80969975129278\\
1359	-12.1607027646826\\
1360	-7.8666167797835\\
1361	-6.12232931898666\\
1362	-5.72277092240961\\
1363	-6.28931942162421\\
1364	-5.49940331402831\\
1365	-9.16136567902026\\
1366	-14.9693895647581\\
1367	-13.0491596827928\\
1368	-15.1487898387186\\
1369	-17.4672884669312\\
1370	-17.513670025385\\
1371	-14.1529269776986\\
1372	-10.8941903058567\\
1373	-10.7348043964066\\
1374	-11.1370228521639\\
1375	-12.4174660881088\\
1376	-13.7760612981936\\
1377	-14.1654465188799\\
1378	-17.8890853860576\\
1379	-19.808225135728\\
1380	-22.4927120075943\\
1381	-17.7199476547464\\
1382	-17.6397154162739\\
1383	-19.8607312879099\\
1384	-15.829443356228\\
1385	-18.1952448190057\\
1386	-16.3447546913017\\
1387	-9.98359117275868\\
1388	-7.50059558070802\\
1389	-7.03334756354138\\
1390	-10.6328836840751\\
1391	-10.14218572864\\
1392	-6.79383622047681\\
1393	-8.29325901665379\\
1394	-7.43252082583922\\
1395	-5.38225046798743\\
1396	-5.9976082279145\\
1397	-7.25563368779672\\
1398	-6.05965616598428\\
1399	-8.60973646062257\\
1400	-11.6061632097164\\
1401	-9.11305841491185\\
1403	-17.9023425827522\\
1404	-17.2543256664778\\
1405	-19.8322895980052\\
};
\addlegendentry{OSA predition}

\addplot [color=mycolor3, dotted, line width=2.0pt]
  table[row sep=crcr]{%
1006	-14.6479999999999\\
1007	-17.0899999999999\\
1008	-10.9860000000001\\
1009	-7.32400000000007\\
1010	-9.15567496596873\\
1011	-9.31783769544859\\
1012	-10.8344800628556\\
1013	-9.65800110641703\\
1014	-4.8140064257027\\
1015	-4.36120419579242\\
1016	-3.61668848126578\\
1017	-6.99250901817891\\
1018	-12.887658298481\\
1019	-12.7427397806966\\
1020	-12.3980767585117\\
1021	-10.2145935196754\\
1022	-6.916922056472\\
1023	-11.4329647793566\\
1024	-10.4742475358926\\
1025	-6.4339986879038\\
1026	-11.0714746166375\\
1027	-11.8544075123627\\
1028	-8.88847615145573\\
1029	-12.7332012443435\\
1030	-13.0104065509565\\
1031	-9.64720621907099\\
1032	-13.0246384443301\\
1033	-13.4872153861907\\
1034	-10.9433196669913\\
1035	-9.41146598968749\\
1036	-8.53642742338934\\
1037	-8.98377989584742\\
1038	-10.9870317007692\\
1039	-10.5931853828861\\
1040	-10.6951893405835\\
1041	-11.0683551414597\\
1042	-11.2179622581291\\
1043	-14.5756437353252\\
1044	-13.994964166639\\
1045	-9.81851353451475\\
1046	-9.33092674888758\\
1047	-6.58903085037286\\
1048	-5.92721614681022\\
1049	-8.24463657968226\\
1050	-7.46612451939131\\
1051	-6.54276092367809\\
1052	-9.12793182795963\\
1053	-10.4686394358782\\
1054	-9.45335247663616\\
1055	-13.3065715064338\\
1056	-12.0121086433167\\
1057	-6.91228058996057\\
1058	-6.49992201971872\\
1059	-7.44119523464951\\
1060	-8.73095119883305\\
1061	-5.84766657989553\\
1062	-4.90161857384078\\
1063	-7.36376937720752\\
1064	-7.06848465517487\\
1065	-4.97041136718644\\
1066	-6.36811790379966\\
1067	-7.71833778644168\\
1068	-11.4342020603265\\
1069	-11.2090543644704\\
1070	-12.0191384706045\\
1071	-11.7830530415622\\
1072	-12.4043903842787\\
1073	-10.914380894744\\
1074	-10.2424345600832\\
1075	-9.87997615046856\\
1076	-9.02571801181102\\
1077	-9.34082814423323\\
1078	-13.501455366988\\
1079	-19.106853131052\\
1080	-20.1184864464817\\
1081	-19.049595549327\\
1082	-13.6245918864315\\
1083	-17.5351754089943\\
1084	-21.0378814283602\\
1085	-20.4483681645529\\
1086	-18.4233517679581\\
1087	-20.2452371704621\\
1088	-25.8856770620366\\
1089	-20.5128815176461\\
1090	-19.1004753845546\\
1091	-13.7997635327245\\
1092	-10.3929608850708\\
1093	-9.56441998324249\\
1094	-11.7605909827444\\
1095	-9.73093712572381\\
1096	-6.20992075485128\\
1097	-6.24868843747913\\
1098	-7.41150807864847\\
1099	-10.5100340949812\\
1100	-10.5855140684544\\
1101	-9.50135592782431\\
1102	-12.1307872386026\\
1103	-12.3395465586525\\
1104	-15.7688537012325\\
1105	-14.0833789485578\\
1106	-15.4406609179614\\
1107	-12.8602382910044\\
1108	-11.1193359914889\\
1109	-12.5730925179932\\
1110	-10.2485375964286\\
1111	-9.24955410651842\\
1112	-11.1006775791325\\
1113	-11.3876935390222\\
1114	-8.22442028025034\\
1115	-8.82275075210077\\
1116	-7.40063369908626\\
1117	-6.96510073878653\\
1118	-8.28810413538599\\
1119	-6.56440805704779\\
1120	-6.95824386207005\\
1121	-8.68390454573819\\
1122	-12.2032222095127\\
1123	-13.2203669391827\\
1124	-13.1500620944253\\
1125	-8.6934156623679\\
1126	-10.5896534394776\\
1127	-16.3206404720836\\
1128	-13.7361105883931\\
1129	-13.4773830467825\\
1130	-15.140560779452\\
1131	-9.7525152391222\\
1132	-7.63167339430015\\
1133	-9.68088149697064\\
1134	-11.3251029886651\\
1135	-15.5299197983807\\
1136	-19.2070684558046\\
1137	-18.6049395323701\\
1138	-14.0956776065996\\
1139	-14.6383090670295\\
1140	-14.2607811162309\\
1141	-13.2157526035855\\
1142	-13.2000493688561\\
1143	-9.89564640330673\\
1144	-8.5283136159369\\
1145	-8.59817162233503\\
1146	-8.14132378017166\\
1147	-8.86968049637676\\
1148	-11.5690262002092\\
1149	-16.0858005072666\\
1150	-13.8135759113982\\
1151	-9.9758868572917\\
1152	-9.21436701292987\\
1153	-9.09260540181731\\
1154	-6.19688559206134\\
1155	-4.53124873684237\\
1156	-4.91599056059977\\
1157	-8.23255719376948\\
1158	-7.17412474136836\\
1159	-6.05238664539979\\
1160	-7.50338080155348\\
1161	-7.34097096201754\\
1162	-5.66717323380021\\
1163	-4.31294632134473\\
1164	-3.54694631585016\\
1165	-4.45193547563804\\
1166	-9.6752618135788\\
1167	-13.7601015841906\\
1168	-14.7301250650521\\
1169	-10.7014789725306\\
1170	-8.18338464632825\\
1172	-4.77683426889462\\
1173	-7.15387883029553\\
1174	-9.72355723976648\\
1175	-10.589143179571\\
1176	-12.9876192043757\\
1177	-19.2485611525699\\
1178	-22.9569543615446\\
1179	-20.1862789287948\\
1180	-19.491691457559\\
1181	-14.092528274796\\
1182	-14.8947322289209\\
1183	-16.555435960405\\
1184	-15.9318157630435\\
1185	-13.6646298500757\\
1186	-12.7667851758647\\
1187	-13.3468880355183\\
1188	-11.8245381584229\\
1189	-13.2186184522097\\
1190	-11.6608466681798\\
1191	-15.4473175647436\\
1192	-18.7576529305586\\
1193	-13.8461194442887\\
1194	-9.89594018998901\\
1195	-10.3772296412783\\
1196	-15.4754662961361\\
1197	-18.0211628952591\\
1198	-21.1034650903509\\
1199	-21.2105676379751\\
1200	-21.2747938274472\\
1201	-17.65536206301\\
1202	-16.3951861494261\\
1203	-17.8665083572853\\
1204	-19.4944748809644\\
1205	-14.8760634774303\\
1206	-9.7614891603987\\
1207	-13.0439462433228\\
1208	-12.4392675468757\\
1209	-8.07100810217366\\
1210	-9.79803586017169\\
1211	-11.8763977598182\\
1212	-9.39109113513246\\
1213	-8.26888555629534\\
1214	-9.74531414728017\\
1215	-9.16829503094436\\
1216	-10.5085270191416\\
1217	-14.1477034078084\\
1218	-13.3147774779077\\
1219	-9.62978044590454\\
1220	-9.65185988261533\\
1221	-13.3176001736367\\
1222	-20.557872440125\\
1223	-17.3818707357548\\
1224	-11.1108832471214\\
1225	-10.0362787962463\\
1226	-10.2679685752339\\
1227	-10.1143344344885\\
1228	-7.96523646794003\\
1229	-8.06620300900454\\
1231	-11.204545310685\\
1232	-11.7272227921421\\
1233	-9.41321267371472\\
1234	-11.9449057413763\\
1235	-14.7547090800535\\
1236	-13.4424530798512\\
1237	-11.5503244520676\\
1238	-12.4206871369647\\
1239	-14.7497531807539\\
1240	-11.1021259606009\\
1241	-5.88115993960696\\
1242	-6.92719658670035\\
1243	-8.63664902687651\\
1244	-10.4859997027406\\
1245	-9.81525712504981\\
1246	-7.25936335765323\\
1247	-10.0587289205014\\
1248	-10.8179681376691\\
1249	-7.76042797770469\\
1250	-9.06403289978016\\
1251	-9.14058893653942\\
1252	-8.04989265571248\\
1253	-9.02773428190471\\
1254	-6.79622473332438\\
1255	-6.4689139496013\\
1256	-6.1097512017484\\
1257	-5.37470033175259\\
1258	-9.27560949690974\\
1259	-10.878523370571\\
1260	-13.2066270360583\\
1261	-16.4832648576692\\
1263	-7.98984988484995\\
1264	-7.0059073828711\\
1265	-6.54060194268959\\
1266	-8.85159230597469\\
1267	-7.19672276994834\\
1268	-6.6078854112186\\
1269	-8.15688786046303\\
1270	-13.0787612708714\\
1271	-12.7614717620195\\
1272	-14.605433822461\\
1273	-11.3203212561145\\
1274	-9.72643654674107\\
1275	-6.50658855588267\\
1276	-5.64155034167516\\
1277	-4.82071824684681\\
1278	-4.21558042269817\\
1279	-5.23286861000406\\
1280	-8.27359980365145\\
1281	-9.06910474126221\\
1282	-9.12266178942696\\
1283	-9.68083259441505\\
1284	-14.4681997273822\\
1285	-14.1267991931049\\
1286	-10.1885601813588\\
1287	-12.1342670486183\\
1288	-12.8476485276035\\
1289	-14.6168013840663\\
1290	-13.0426479542461\\
1291	-9.60526446065774\\
1292	-5.53665251586222\\
1293	-4.7731348781233\\
1294	-4.43481121385025\\
1295	-5.14588800768797\\
1296	-5.64711960346858\\
1297	-5.15719080482563\\
1298	-6.21690506812524\\
1299	-9.13271471469761\\
1300	-8.74458144860523\\
1301	-7.88718493543342\\
1302	-9.21613736752283\\
1303	-6.47709049225432\\
1304	-3.97804426629386\\
1305	-6.36416426638721\\
1306	-10.8191479326335\\
1307	-9.94941711761453\\
1308	-10.1293137318573\\
1309	-9.92852624436159\\
1310	-7.18479239289127\\
1311	-8.83416855473729\\
1312	-13.4429390645068\\
1313	-12.7987864139209\\
1314	-9.57894784532118\\
1315	-13.6540689242331\\
1316	-13.1664129203441\\
1317	-11.0890496774823\\
1318	-12.7517040065914\\
1319	-12.9312384552006\\
1320	-10.5295836485102\\
1321	-9.73936123037333\\
1322	-13.2527161878359\\
1323	-18.1012973836666\\
1324	-15.2186237910723\\
1325	-9.95629999236735\\
1326	-10.0130748623922\\
1327	-10.1941442732304\\
1328	-12.072567751879\\
1329	-13.3834027744856\\
1330	-10.7045312521866\\
1331	-11.2009380516863\\
1332	-13.3501285899738\\
1333	-14.371185601645\\
1334	-10.0432211558714\\
1335	-9.09719794050943\\
1336	-11.9932631669917\\
1337	-16.4745000218197\\
1338	-15.7140442292366\\
1339	-15.4990277180441\\
1340	-10.5238634245445\\
1341	-9.3568032248827\\
1342	-10.2266775001121\\
1343	-6.90263607465477\\
1344	-4.80044451512663\\
1345	-4.13167176946968\\
1346	-3.57031450126374\\
1347	-6.23817185043617\\
1348	-10.1069430248608\\
1349	-10.8328223217843\\
1350	-8.12287678992243\\
1351	-8.13355347917582\\
1352	-9.61587881993546\\
1353	-6.35924040180794\\
1354	-6.18306155482924\\
1355	-7.14073360429961\\
1356	-5.40916883806221\\
1357	-6.0371221292271\\
1358	-9.82434243928492\\
1359	-12.2319222514639\\
1360	-7.98612719581047\\
1361	-6.44836632595911\\
1362	-6.0123535341304\\
1363	-6.37517970746012\\
1364	-5.43136276167252\\
1365	-9.03160502147648\\
1366	-15.0465732657142\\
1367	-13.0319689513497\\
1368	-15.0233158589615\\
1369	-17.2338496644415\\
1370	-17.240811014432\\
1371	-13.8017209906718\\
1372	-10.9350229708293\\
1373	-10.793156439182\\
1374	-11.239278940433\\
1375	-12.5003773068845\\
1376	-13.908430424585\\
1377	-14.1928741621634\\
1378	-17.8298084255853\\
1379	-19.7726517388774\\
1380	-22.5494324150336\\
1381	-17.8904550110913\\
1382	-17.7162080458199\\
1383	-19.756515475462\\
1384	-15.41251213484\\
1385	-17.5802354866073\\
1386	-16.4379127429631\\
1387	-10.2716167291121\\
1388	-7.72692793554029\\
1389	-7.33581987533717\\
1390	-10.4116788287322\\
1391	-9.90902712666275\\
1392	-6.62995729113368\\
1393	-8.3613222202639\\
1394	-7.84393612517692\\
1395	-5.37661352118334\\
1396	-6.25190027106987\\
1397	-7.41183553160113\\
1398	-6.22041832597642\\
1399	-8.56428956179843\\
1400	-11.6089331937219\\
1401	-9.10481872672221\\
1402	-13.4791651588353\\
1403	-18.0234732322385\\
1404	-17.3413075491781\\
1405	-19.8329242062991\\
};
\addlegendentry{MPO prediction}

\end{axis}

\begin{axis}[%
width=6.159cm,
height=1.831cm,
at={(0cm,5.085cm)},
scale only axis,
xmin=1000,
xmax=1405,
xlabel style={font=\color{white!15!black}},
xlabel={Sample index},
ymin=-28.1498026372379,
ymax=0,
ylabel style={font=\color{white!15!black}},
ylabel={$y(t)$},
axis background/.style={fill=white},
title style={font=\bfseries},
title={C5: RMSE(OSA) = 1.5754, RMSE(MPO) = 1.5597},
legend style={legend cell align=left, align=left, draw=white!15!black}
]
\addplot [color=mycolor1, line width=2.0pt]
  table[row sep=crcr]{%
1006	-15.8689999999999\\
1007	-14.6479999999999\\
1008	-14.6479999999999\\
1009	-7.32400000000007\\
1010	-8.54500000000007\\
1011	-10.9860000000001\\
1012	-9.76600000000008\\
1013	-9.76600000000008\\
1015	-2.44100000000003\\
1016	-3.66200000000003\\
1017	-2.44100000000003\\
1018	-12.2070000000001\\
1019	-13.4280000000001\\
1021	-10.9860000000001\\
1022	-6.10400000000004\\
1023	-6.10400000000004\\
1024	-1.221\\
1026	-10.9860000000001\\
1027	-12.2070000000001\\
1028	-8.54500000000007\\
1029	-14.6479999999999\\
1030	-14.6479999999999\\
1031	-9.76600000000008\\
1032	-13.4280000000001\\
1033	-12.2070000000001\\
1034	-9.76600000000008\\
1035	-9.76600000000008\\
1036	-8.54500000000007\\
1037	-8.54500000000007\\
1038	-12.2070000000001\\
1039	-10.9860000000001\\
1042	-10.9860000000001\\
1043	-14.6479999999999\\
1044	-15.8689999999999\\
1045	-9.76600000000008\\
1046	-8.54500000000007\\
1047	-8.54500000000007\\
1048	-6.10400000000004\\
1049	-7.32400000000007\\
1050	-7.32400000000007\\
1051	-4.88300000000004\\
1052	-9.76600000000008\\
1053	-10.9860000000001\\
1054	-8.54500000000007\\
1055	-13.4280000000001\\
1056	-15.8689999999999\\
1057	-8.54500000000007\\
1058	-6.10400000000004\\
1059	-6.10400000000004\\
1060	-9.76600000000008\\
1061	-6.10400000000004\\
1062	-4.88300000000004\\
1063	-7.32400000000007\\
1064	-7.32400000000007\\
1065	-3.66200000000003\\
1068	-10.9860000000001\\
1069	-10.9860000000001\\
1070	-9.76600000000008\\
1071	-13.4280000000001\\
1072	-10.9860000000001\\
1073	-12.2070000000001\\
1074	-10.9860000000001\\
1075	-10.9860000000001\\
1076	-7.32400000000007\\
1077	-8.54500000000007\\
1079	-20.752\\
1080	-20.752\\
1081	-19.5309999999999\\
1082	-13.4280000000001\\
1083	-15.8689999999999\\
1084	-24.414\\
1085	-23.193\\
1086	-15.8689999999999\\
1088	-28.076\\
1089	-23.193\\
1090	-17.0899999999999\\
1091	-14.6479999999999\\
1092	-13.4280000000001\\
1093	-9.76600000000008\\
1095	-12.2070000000001\\
1096	-4.88300000000004\\
1097	-4.88300000000004\\
1098	-7.32400000000007\\
1099	-10.9860000000001\\
1100	-9.76600000000008\\
1101	-9.76600000000008\\
1102	-10.9860000000001\\
1103	-13.4280000000001\\
1105	-15.8689999999999\\
1106	-14.6479999999999\\
1107	-17.0899999999999\\
1108	-12.2070000000001\\
1109	-12.2070000000001\\
1110	-10.9860000000001\\
1111	-8.54500000000007\\
1112	-9.76600000000008\\
1113	-12.2070000000001\\
1114	-8.54500000000007\\
1115	-9.76600000000008\\
1116	-7.32400000000007\\
1117	-6.10400000000004\\
1118	-8.54500000000007\\
1119	-6.10400000000004\\
1120	-4.88300000000004\\
1121	-8.54500000000007\\
1122	-13.4280000000001\\
1124	-10.9860000000001\\
1125	-7.32400000000007\\
1126	-8.54500000000007\\
1127	-19.5309999999999\\
1128	-14.6479999999999\\
1129	-10.9860000000001\\
1130	-17.0899999999999\\
1131	-12.2070000000001\\
1132	-6.10400000000004\\
1133	-9.76600000000008\\
1134	-8.54500000000007\\
1135	-14.6479999999999\\
1136	-15.8689999999999\\
1137	-19.5309999999999\\
1138	-14.6479999999999\\
1139	-14.6479999999999\\
1140	-13.4280000000001\\
1141	-13.4280000000001\\
1142	-14.6479999999999\\
1144	-7.32400000000007\\
1145	-7.32400000000007\\
1146	-6.10400000000004\\
1148	-10.9860000000001\\
1149	-15.8689999999999\\
1150	-14.6479999999999\\
1151	-8.54500000000007\\
1152	-8.54500000000007\\
1153	-9.76600000000008\\
1155	-2.44100000000003\\
1157	-7.32400000000007\\
1158	-8.54500000000007\\
1159	-6.10400000000004\\
1160	-7.32400000000007\\
1162	-4.88300000000004\\
1163	-4.88300000000004\\
1164	-1.221\\
1165	-3.66200000000003\\
1166	-12.2070000000001\\
1167	-14.6479999999999\\
1168	-15.8689999999999\\
1169	-12.2070000000001\\
1170	-7.32400000000007\\
1171	-6.10400000000004\\
1172	-3.66200000000003\\
1173	-7.32400000000007\\
1175	-9.76600000000008\\
1176	-12.2070000000001\\
1177	-18.3109999999999\\
1178	-18.3109999999999\\
1179	-19.5309999999999\\
1180	-18.3109999999999\\
1181	-15.8689999999999\\
1182	-15.8689999999999\\
1184	-18.3109999999999\\
1185	-12.2070000000001\\
1186	-10.9860000000001\\
1187	-12.2070000000001\\
1188	-10.9860000000001\\
1189	-12.2070000000001\\
1190	-9.76600000000008\\
1191	-15.8689999999999\\
1192	-18.3109999999999\\
1193	-12.2070000000001\\
1194	-7.32400000000007\\
1195	-9.76600000000008\\
1196	-17.0899999999999\\
1197	-18.3109999999999\\
1198	-18.3109999999999\\
1199	-21.973\\
1200	-20.752\\
1201	-17.0899999999999\\
1202	-15.8689999999999\\
1204	-20.752\\
1205	-17.0899999999999\\
1206	-8.54500000000007\\
1207	-14.6479999999999\\
1208	-12.2070000000001\\
1209	-7.32400000000007\\
1210	-10.9860000000001\\
1213	-7.32400000000007\\
1214	-8.54500000000007\\
1215	-8.54500000000007\\
1216	-9.76600000000008\\
1217	-13.4280000000001\\
1218	-14.6479999999999\\
1219	-7.32400000000007\\
1220	-8.54500000000007\\
1221	-12.2070000000001\\
1222	-17.0899999999999\\
1223	-15.8689999999999\\
1224	-8.54500000000007\\
1225	-8.54500000000007\\
1226	-9.76600000000008\\
1229	-6.10400000000004\\
1230	-8.54500000000007\\
1231	-9.76600000000008\\
1233	-9.76600000000008\\
1235	-14.6479999999999\\
1236	-15.8689999999999\\
1237	-10.9860000000001\\
1238	-13.4280000000001\\
1239	-17.0899999999999\\
1240	-12.2070000000001\\
1241	-3.66200000000003\\
1242	-9.76600000000008\\
1243	-8.54500000000007\\
1244	-9.76600000000008\\
1245	-8.54500000000007\\
1247	-8.54500000000007\\
1248	-10.9860000000001\\
1249	-7.32400000000007\\
1250	-7.32400000000007\\
1251	-9.76600000000008\\
1252	-6.10400000000004\\
1253	-8.54500000000007\\
1255	-3.66200000000003\\
1256	-6.10400000000004\\
1257	-4.88300000000004\\
1258	-10.9860000000001\\
1259	-12.2070000000001\\
1260	-12.2070000000001\\
1261	-18.3109999999999\\
1262	-10.9860000000001\\
1263	-4.88300000000004\\
1264	-6.10400000000004\\
1265	-6.10400000000004\\
1266	-8.54500000000007\\
1267	-6.10400000000004\\
1268	-8.54500000000007\\
1269	-7.32400000000007\\
1270	-13.4280000000001\\
1271	-9.76600000000008\\
1272	-14.6479999999999\\
1273	-10.9860000000001\\
1274	-9.76600000000008\\
1275	-6.10400000000004\\
1276	-3.66200000000003\\
1277	-3.66200000000003\\
1278	-1.221\\
1279	-2.44100000000003\\
1280	-8.54500000000007\\
1281	-8.54500000000007\\
1282	-7.32400000000007\\
1283	-9.76600000000008\\
1284	-15.8689999999999\\
1285	-10.9860000000001\\
1286	-9.76600000000008\\
1287	-9.76600000000008\\
1288	-13.4280000000001\\
1289	-14.6479999999999\\
1290	-12.2070000000001\\
1291	-12.2070000000001\\
1292	-6.10400000000004\\
1293	-2.44100000000003\\
1294	-4.88300000000004\\
1295	-4.88300000000004\\
1296	-6.10400000000004\\
1297	-3.66200000000003\\
1298	-4.88300000000004\\
1299	-9.76600000000008\\
1300	-7.32400000000007\\
1301	-7.32400000000007\\
1302	-10.9860000000001\\
1303	-7.32400000000007\\
1304	-4.88300000000004\\
1305	-9.76600000000008\\
1306	-12.2070000000001\\
1307	-9.76600000000008\\
1308	-10.9860000000001\\
1309	-7.32400000000007\\
1310	-7.32400000000007\\
1312	-14.6479999999999\\
1313	-13.4280000000001\\
1314	-8.54500000000007\\
1315	-14.6479999999999\\
1316	-13.4280000000001\\
1319	-13.4280000000001\\
1320	-10.9860000000001\\
1321	-10.9860000000001\\
1322	-12.2070000000001\\
1323	-18.3109999999999\\
1324	-17.0899999999999\\
1325	-10.9860000000001\\
1327	-10.9860000000001\\
1328	-13.4280000000001\\
1329	-13.4280000000001\\
1330	-9.76600000000008\\
1332	-14.6479999999999\\
1333	-14.6479999999999\\
1334	-9.76600000000008\\
1335	-8.54500000000007\\
1336	-10.9860000000001\\
1337	-18.3109999999999\\
1338	-14.6479999999999\\
1339	-13.4280000000001\\
1340	-9.76600000000008\\
1341	-10.9860000000001\\
1342	-9.76600000000008\\
1343	-6.10400000000004\\
1345	-3.66200000000003\\
1346	-3.66200000000003\\
1348	-10.9860000000001\\
1349	-9.76600000000008\\
1350	-6.10400000000004\\
1351	-7.32400000000007\\
1352	-9.76600000000008\\
1353	-6.10400000000004\\
1354	-6.10400000000004\\
1355	-9.76600000000008\\
1356	-6.10400000000004\\
1357	-7.32400000000007\\
1358	-12.2070000000001\\
1359	-12.2070000000001\\
1361	-4.88300000000004\\
1363	-7.32400000000007\\
1364	-6.10400000000004\\
1365	-7.32400000000007\\
1366	-14.6479999999999\\
1367	-9.76600000000008\\
1368	-17.0899999999999\\
1369	-20.752\\
1370	-18.3109999999999\\
1371	-13.4280000000001\\
1372	-10.9860000000001\\
1373	-10.9860000000001\\
1377	-15.8689999999999\\
1378	-20.752\\
1379	-20.752\\
1380	-24.414\\
1381	-15.8689999999999\\
1382	-20.752\\
1383	-20.752\\
1384	-14.6479999999999\\
1385	-17.0899999999999\\
1386	-18.3109999999999\\
1387	-9.76600000000008\\
1388	-7.32400000000007\\
1390	-9.76600000000008\\
1391	-7.32400000000007\\
1392	-6.10400000000004\\
1393	-7.32400000000007\\
1394	-6.10400000000004\\
1395	-3.66200000000003\\
1396	-4.88300000000004\\
1397	-7.32400000000007\\
1398	-2.44100000000003\\
1399	-10.9860000000001\\
1400	-8.54500000000007\\
1401	-7.32400000000007\\
1402	-15.8689999999999\\
1403	-19.5309999999999\\
1405	-19.5309999999999\\
};
\addlegendentry{True output}

\addplot [color=mycolor2, dashed, line width=2.0pt]
  table[row sep=crcr]{%
1006	-14.141662001158\\
1007	-16.1340647066668\\
1008	-13.3988444853255\\
1009	-7.68188441804045\\
1010	-9.28272331540165\\
1011	-10.5180980613625\\
1012	-10.4673830598222\\
1013	-9.40922167579311\\
1014	-3.9387865614608\\
1015	-3.42261315070732\\
1016	-3.16052161173116\\
1017	-6.98885576132352\\
1018	-12.5269754750441\\
1019	-11.9027905663277\\
1020	-12.3137341190943\\
1021	-9.931095639369\\
1022	-5.93881194264577\\
1023	-12.1883163357256\\
1024	-11.3683960416033\\
1025	-6.44161294997116\\
1026	-11.0368072201609\\
1027	-10.1389652448552\\
1028	-8.63662178178447\\
1029	-13.259397877441\\
1030	-12.8084671409363\\
1031	-9.49016818113751\\
1032	-13.9783574909138\\
1033	-13.6038921857487\\
1034	-10.874052190841\\
1035	-9.71527850215875\\
1036	-7.95241780820902\\
1037	-8.86872910948591\\
1038	-10.8253273106063\\
1039	-10.1170155039781\\
1040	-10.5131231073383\\
1041	-11.0648782752162\\
1042	-11.8744035021662\\
1043	-14.8444983632228\\
1044	-13.8482677981485\\
1045	-9.52356463363708\\
1046	-9.85208467605207\\
1047	-5.82376654951963\\
1048	-5.92008940336359\\
1049	-8.35191233215937\\
1050	-7.51433870435881\\
1051	-6.50590809503592\\
1052	-9.02786428172976\\
1053	-9.9740362419584\\
1054	-9.06799348950221\\
1055	-13.6234942570302\\
1056	-11.6089250741159\\
1057	-5.98688829891648\\
1058	-6.00517031349909\\
1059	-9.00926292478903\\
1060	-8.83318636240392\\
1061	-5.02636486825213\\
1062	-4.80129448071443\\
1063	-7.61922027492369\\
1064	-6.10642739706554\\
1065	-4.91818316384479\\
1066	-6.64363266305213\\
1067	-7.13561652760313\\
1068	-9.61082532980186\\
1069	-10.481906760217\\
1070	-11.0664600846044\\
1071	-11.314493035334\\
1072	-11.8394710904524\\
1073	-10.2306996731797\\
1074	-9.95144290782537\\
1075	-9.3294467209646\\
1076	-9.50775122838763\\
1077	-9.38260705623588\\
1078	-13.2115413075408\\
1079	-19.1373895105799\\
1080	-19.7318948973814\\
1081	-19.1940809603263\\
1082	-13.6222265756633\\
1083	-18.7729192270453\\
1084	-21.9462860185379\\
1085	-21.6623558249466\\
1086	-18.2786014318483\\
1087	-22.7241228174587\\
1088	-28.0683195398872\\
1089	-22.9671975004412\\
1090	-19.3739514254507\\
1091	-14.5271372517855\\
1092	-10.4121203351533\\
1093	-10.159739386226\\
1094	-12.1565478605801\\
1095	-10.2127303698326\\
1096	-5.88456675821635\\
1097	-6.27770380208949\\
1098	-8.19879071992\\
1099	-10.1835258753347\\
1100	-10.1318530122162\\
1101	-9.79754785397154\\
1102	-11.8297534173832\\
1103	-12.5975717456076\\
1104	-15.7971678553927\\
1105	-14.3229464165845\\
1106	-15.1449517026626\\
1107	-13.2420162540227\\
1108	-11.0845244887087\\
1109	-13.150358429669\\
1110	-11.0438629458995\\
1111	-9.17922100714759\\
1112	-11.0417043886598\\
1113	-11.1966600849582\\
1114	-7.97108711868373\\
1115	-8.94535111662731\\
1116	-7.41240629292861\\
1117	-7.53969560236601\\
1118	-8.5101759842687\\
1119	-6.09005666154098\\
1120	-6.70286595708671\\
1121	-9.2655841624487\\
1122	-11.3750602363684\\
1123	-12.4782606495696\\
1124	-13.3440466768354\\
1125	-8.71089499519803\\
1126	-11.0636489378194\\
1127	-16.6096525320056\\
1128	-13.2975065308451\\
1129	-14.7362743125489\\
1130	-15.8036032300099\\
1131	-8.93097802303464\\
1132	-6.87525270080232\\
1133	-11.0121804725577\\
1134	-11.1435847560642\\
1135	-15.9138953789229\\
1136	-18.7918417698875\\
1137	-17.8661981921744\\
1138	-14.3079724630099\\
1139	-14.2523036732543\\
1140	-15.0463291539768\\
1141	-13.0490932951902\\
1142	-13.7272548932986\\
1143	-9.95566977329736\\
1144	-9.23392373659658\\
1145	-9.04672194533146\\
1146	-7.91297783973346\\
1147	-8.71448241020744\\
1148	-11.9086811691566\\
1149	-16.5441658859042\\
1150	-13.8963824147052\\
1151	-9.47145341790178\\
1152	-9.09763566287097\\
1153	-9.12894577818088\\
1154	-5.3751333650016\\
1155	-4.09344786617021\\
1156	-5.08994340046002\\
1157	-8.01764261947915\\
1158	-6.96408670736946\\
1159	-6.11856173374144\\
1160	-7.62501707769593\\
1161	-6.78127971976119\\
1162	-5.17581871865445\\
1163	-3.83352361792686\\
1164	-3.07909044948792\\
1165	-4.47956740024938\\
1166	-9.32305291461125\\
1167	-13.3318141425771\\
1168	-15.0784710862561\\
1169	-10.1905758363016\\
1170	-7.97136025435179\\
1171	-6.00276433892031\\
1172	-3.96002154813209\\
1173	-7.34756327213563\\
1174	-9.01967486707804\\
1175	-10.029209728265\\
1176	-12.938965963802\\
1177	-19.4277409070837\\
1178	-22.703095405152\\
1179	-19.4985471263321\\
1180	-19.1503516445439\\
1181	-13.4066900801395\\
1182	-15.0001712542203\\
1183	-16.5160086098472\\
1184	-16.6441366826809\\
1185	-14.4404716746178\\
1186	-12.9223914165927\\
1187	-13.4511926932032\\
1188	-12.4754268429447\\
1189	-13.4242856799804\\
1190	-10.1545000447004\\
1191	-15.1779881621892\\
1192	-18.8549731614628\\
1193	-13.8977213443827\\
1194	-9.93461697203634\\
1195	-9.82575149801255\\
1196	-15.4987593109911\\
1197	-18.5484685098338\\
1198	-21.8764413128522\\
1199	-21.9821614536183\\
1200	-21.5288965158213\\
1201	-17.8947851699131\\
1202	-16.3855608438234\\
1203	-19.142824889168\\
1204	-21.0735993351029\\
1205	-14.3627377687744\\
1206	-9.97190921529022\\
1207	-13.9858880922022\\
1208	-13.3530529458058\\
1209	-7.22793572838941\\
1210	-10.9180450991566\\
1211	-12.1840232075861\\
1212	-9.67556453267593\\
1213	-7.8297482777125\\
1214	-9.765814511288\\
1215	-8.9766600684909\\
1216	-10.5264720949401\\
1217	-13.8815164822463\\
1218	-12.8096483414945\\
1219	-9.48840589841552\\
1220	-10.2632961065276\\
1221	-13.9174857844303\\
1222	-19.5296074203557\\
1223	-15.8422694068934\\
1224	-10.018172452837\\
1225	-8.27044082272391\\
1226	-10.3427629815812\\
1227	-9.52680433294449\\
1228	-7.01436417342552\\
1229	-7.86346054508113\\
1230	-9.70004222893499\\
1231	-11.328655563633\\
1232	-11.5907085596134\\
1233	-8.30067074579392\\
1234	-13.6919021655413\\
1235	-15.8693603367292\\
1236	-13.7366066981176\\
1237	-11.9038441668299\\
1238	-12.817743253079\\
1239	-15.6253561574254\\
1240	-10.9756608534872\\
1241	-5.23789599137604\\
1242	-8.26436370397187\\
1243	-8.67274870811957\\
1244	-10.389559043399\\
1245	-9.99904528922639\\
1246	-7.36307738937398\\
1247	-10.274522419634\\
1248	-11.0004152531058\\
1249	-7.52751216055162\\
1250	-9.20471027980921\\
1251	-9.13987172600878\\
1252	-7.47519913731958\\
1253	-8.89920773799395\\
1254	-6.28658597240656\\
1255	-6.42869303306588\\
1256	-6.00142187130746\\
1257	-5.20968702063965\\
1258	-9.23485095653314\\
1259	-10.7514233969816\\
1260	-13.1525262121877\\
1261	-16.7153073705451\\
1262	-12.1881178129609\\
1263	-7.50127597463461\\
1264	-6.54242130958824\\
1265	-6.09938032946548\\
1266	-8.03876270237924\\
1267	-5.95893535935261\\
1268	-5.56297222230137\\
1269	-8.01089239808471\\
1270	-11.951081598412\\
1271	-11.9477100273484\\
1272	-14.1709805356661\\
1273	-10.8138516477684\\
1274	-9.09261683638715\\
1275	-4.57063434262477\\
1276	-5.12652776092477\\
1277	-4.25375321200204\\
1278	-3.68094753195282\\
1279	-4.86011752429795\\
1280	-7.46205172374994\\
1281	-8.77888006613193\\
1282	-8.76874709109688\\
1283	-9.77122659054953\\
1284	-13.93425181253\\
1285	-13.8200558129461\\
1286	-10.535503429039\\
1287	-12.0013267634731\\
1288	-12.5704982773182\\
1289	-14.7178347451688\\
1290	-14.2353808904879\\
1291	-8.89438111176059\\
1292	-4.95218122160804\\
1293	-3.83484221476579\\
1294	-4.58825519192442\\
1295	-4.90446870609162\\
1296	-5.11401409198766\\
1297	-4.72764098221501\\
1298	-5.88145093831827\\
1299	-9.00850120610653\\
1300	-8.25899911032161\\
1301	-8.14279568457141\\
1302	-9.19952261117442\\
1303	-5.87903915094694\\
1304	-3.41340560889535\\
1305	-6.96706330359166\\
1306	-10.3521141717824\\
1307	-9.87719065374131\\
1308	-10.3411293832069\\
1309	-9.19282061255672\\
1310	-7.17032931363224\\
1311	-9.2761869024539\\
1312	-12.7919587735546\\
1313	-12.6764017141529\\
1314	-9.70344304486912\\
1315	-14.9824418180783\\
1316	-13.20689304951\\
1317	-10.7914999266723\\
1318	-13.673264422433\\
1319	-12.9873039382007\\
1320	-11.0835119015705\\
1321	-9.64384143019151\\
1322	-15.7560272043822\\
1323	-19.8044556129266\\
1324	-15.8661753244944\\
1325	-9.5878955359351\\
1326	-9.99308011079302\\
1327	-11.1589443104197\\
1328	-12.5223869577624\\
1329	-13.7971075600008\\
1330	-11.2957114610067\\
1331	-11.2467887597065\\
1332	-14.4592375689226\\
1333	-14.1392685795374\\
1334	-12.0388329666894\\
1335	-8.62802502604018\\
1336	-11.8804571504822\\
1337	-16.8345802216243\\
1338	-16.2519339069331\\
1339	-15.2158824377259\\
1340	-10.6971535242335\\
1341	-8.6410849311419\\
1342	-10.0397418654156\\
1343	-6.01907525696447\\
1344	-4.09211626526167\\
1345	-3.54550731947347\\
1346	-3.16425183270121\\
1347	-5.23026367044076\\
1348	-8.98043688254029\\
1349	-9.87364583515318\\
1350	-8.10633493566093\\
1351	-8.16021883590406\\
1352	-9.33400344812048\\
1353	-5.71210827300979\\
1354	-5.97501781925325\\
1355	-6.85514964310096\\
1356	-5.01522394244034\\
1357	-6.21165999665209\\
1358	-9.7733464002149\\
1359	-11.6751926789386\\
1360	-7.79733403012324\\
1361	-6.40579743334888\\
1362	-5.94715834182466\\
1363	-6.10588103322129\\
1364	-4.81726686044772\\
1365	-8.72394444950464\\
1366	-15.0782145097369\\
1367	-12.3744155573386\\
1368	-15.3238950224088\\
1369	-16.994645882533\\
1370	-17.7614464898181\\
1371	-14.8273159128435\\
1372	-11.4154203383155\\
1373	-10.8285790463417\\
1374	-11.285533758961\\
1375	-12.3186144120625\\
1376	-13.9539162674071\\
1377	-14.302781507143\\
1378	-17.762052129108\\
1379	-19.9966134321228\\
1380	-23.0240808163303\\
1381	-18.4622472728288\\
1382	-18.0707412849483\\
1383	-20.8763986071424\\
1384	-16.7472254569655\\
1385	-18.9539065273134\\
1386	-17.5830788804381\\
1387	-9.86846437279769\\
1388	-6.10174050713317\\
1389	-7.68168029572894\\
1390	-10.6908496332378\\
1391	-9.5626004430319\\
1392	-6.534307578783\\
1393	-8.6471275246945\\
1394	-7.26167330903422\\
1395	-4.79364855624658\\
1396	-6.21541334615517\\
1397	-7.05872459375155\\
1398	-5.29231408478927\\
1399	-8.40698006649541\\
1400	-10.8679527554159\\
1401	-8.42089606089689\\
1402	-13.0291227117873\\
1403	-17.9630491745859\\
1404	-17.3405056628303\\
1405	-20.2937810332573\\
};
\addlegendentry{OSA predition}

\addplot [color=mycolor3, dotted, line width=2.0pt]
  table[row sep=crcr]{%
1006	-15.8689999999999\\
1007	-14.6479999999999\\
1008	-14.6479999999999\\
1009	-7.32400000000007\\
1010	-9.28272331540325\\
1011	-10.4891043680996\\
1012	-10.5255303255769\\
1013	-9.43876563203344\\
1014	-3.94118987418415\\
1015	-3.55519974573622\\
1016	-2.91789213836205\\
1017	-6.61059074557238\\
1018	-12.0825594844218\\
1019	-12.215466052982\\
1020	-12.2707222872989\\
1021	-9.7903416616457\\
1022	-5.84488281007179\\
1023	-12.1577683722001\\
1024	-10.9860494370273\\
1025	-6.28674067020302\\
1026	-12.431875242044\\
1027	-11.1280640800046\\
1028	-8.63350026890043\\
1029	-13.3410295446527\\
1030	-12.7007526084794\\
1031	-9.44912891672107\\
1032	-13.7192041775852\\
1033	-13.3832798828848\\
1034	-10.8671573001445\\
1035	-9.79500297977233\\
1036	-8.1767144448047\\
1037	-9.03700004510802\\
1038	-10.773043930709\\
1039	-10.1418989505867\\
1041	-10.8661179421567\\
1042	-11.7848972314703\\
1043	-14.7990164257085\\
1044	-13.8935748165593\\
1045	-9.62514248953062\\
1046	-9.74614028721885\\
1047	-5.56146550679659\\
1048	-6.14122722802335\\
1049	-8.38325337786182\\
1050	-7.12648653706469\\
1051	-6.61790167031518\\
1052	-9.03609392898397\\
1053	-10.071317646999\\
1054	-9.18058530679446\\
1055	-13.5017162168517\\
1056	-11.5783756788669\\
1057	-6.10655909281036\\
1058	-5.86104895799826\\
1059	-8.00956046121405\\
1060	-8.49238571249089\\
1061	-5.1871622503827\\
1062	-5.00439141035417\\
1063	-7.33772048612332\\
1064	-6.03410760086558\\
1065	-5.00711388695413\\
1066	-6.49367514440587\\
1067	-7.07160542017141\\
1068	-9.74506265602463\\
1069	-10.4304819658971\\
1070	-10.969946849737\\
1071	-11.1776234197819\\
1072	-11.9205686736645\\
1073	-10.214409128537\\
1074	-9.77237624851773\\
1075	-9.39993601505716\\
1076	-9.21796186638608\\
1077	-9.08317322897074\\
1078	-13.1809996343713\\
1079	-19.0912210068648\\
1080	-19.4978281126409\\
1081	-18.9603140182792\\
1082	-13.3019759103524\\
1083	-18.5559025757034\\
1084	-21.9562903193464\\
1085	-21.5830785516659\\
1086	-18.3064884412897\\
1087	-22.3179518122224\\
1088	-28.1498026372378\\
1089	-23.2955064549851\\
1090	-19.4080498999481\\
1091	-14.6307773874939\\
1092	-10.5829600900779\\
1093	-10.6101864718951\\
1094	-11.9127045831135\\
1095	-9.84087955497193\\
1096	-6.13770493334914\\
1097	-6.18044476981368\\
1098	-7.75933681859283\\
1099	-10.4066406784073\\
1100	-10.2397931441287\\
1101	-9.7652808361961\\
1102	-11.8014438606465\\
1103	-12.6233389005015\\
1104	-15.8484396868291\\
1105	-14.3461331695141\\
1106	-15.1331131813338\\
1107	-13.3123631908734\\
1108	-10.8657077794123\\
1109	-13.0394937977037\\
1110	-10.422029323533\\
1111	-9.11903712687467\\
1112	-11.129354836653\\
1113	-11.1082317411383\\
1114	-8.15290398269531\\
1115	-9.07113334639689\\
1116	-7.25274815178682\\
1117	-7.4649066201182\\
1118	-8.35239597607938\\
1119	-6.1773785255366\\
1120	-6.85327134192244\\
1121	-9.12491298005011\\
1122	-11.468285607715\\
1123	-12.5600923040759\\
1124	-13.2170989761607\\
1125	-8.44203880887017\\
1126	-11.2209123452878\\
1127	-16.828221621558\\
1128	-13.41019741379\\
1129	-14.7392874445645\\
1130	-15.3186090393806\\
1131	-9.0061837917624\\
1132	-7.33037929323837\\
1133	-10.3313608908393\\
1134	-10.7677123550236\\
1135	-15.9212668826669\\
1136	-18.8928296785414\\
1137	-18.060768313647\\
1138	-14.5502193081745\\
1139	-14.5844628781781\\
1140	-14.766901701496\\
1141	-13.0470779538387\\
1142	-13.807738009416\\
1143	-10.1166771822816\\
1144	-9.13605413318896\\
1145	-8.79065584261912\\
1146	-7.84707168351429\\
1147	-8.98092325065181\\
1148	-12.1187981014814\\
1149	-16.4987347620836\\
1150	-13.9742151827043\\
1151	-9.64708613790481\\
1152	-9.08900114964536\\
1153	-9.05672026117986\\
1154	-5.56847526041133\\
1155	-4.11834745785109\\
1156	-4.83573682769088\\
1157	-8.04371425431896\\
1158	-7.06454837418528\\
1159	-6.19880769072779\\
1160	-7.6112010805673\\
1161	-6.61383750132723\\
1162	-5.19514776789515\\
1163	-3.87385280334979\\
1164	-3.20769502429948\\
1165	-4.3410445437944\\
1166	-9.37286993734801\\
1167	-13.0521607260644\\
1168	-14.7967734876529\\
1169	-9.79249312422348\\
1170	-7.81366614084732\\
1171	-5.72712270767488\\
1172	-3.80273685815223\\
1173	-7.44972632803933\\
1174	-8.97854857231982\\
1175	-10.0226811763025\\
1176	-12.9512062310782\\
1177	-19.3597703602381\\
1178	-22.7443956539034\\
1179	-19.879437198523\\
1180	-19.548057912338\\
1181	-14.135312814564\\
1182	-15.0726188147066\\
1183	-16.6069037938562\\
1184	-16.3418921938551\\
1185	-14.2714286597334\\
1186	-12.7176439917373\\
1187	-13.2401005741942\\
1188	-12.9015483999628\\
1189	-13.649198452852\\
1190	-10.3702551521324\\
1191	-15.4822559124898\\
1192	-18.9758840293548\\
1193	-13.9488817174799\\
1194	-9.89735767790421\\
1195	-9.94637011437612\\
1196	-15.9886315951437\\
1197	-18.601618233859\\
1198	-21.7785794895885\\
1199	-22.1784200892571\\
1200	-21.8067364834412\\
1201	-18.4679234319965\\
1202	-16.4883405073015\\
1203	-19.428490546441\\
1204	-21.3344135502728\\
1205	-14.5481725547684\\
1206	-10.1127692321188\\
1207	-13.8734608693676\\
1208	-13.0056165364447\\
1209	-7.39824299213456\\
1210	-10.8160359536839\\
1211	-12.3345247056814\\
1212	-9.59645945964212\\
1213	-7.9209704406685\\
1214	-10.198746366186\\
1215	-9.05323990860688\\
1216	-10.6458268818642\\
1217	-13.9862642815331\\
1218	-12.8695306531959\\
1219	-9.626167845385\\
1220	-10.1248447562514\\
1221	-13.7373399388216\\
1222	-19.5939356567671\\
1223	-16.0960675636386\\
1224	-10.3678351622482\\
1225	-8.52958345034313\\
1226	-10.4727875742965\\
1227	-9.76793357670817\\
1228	-6.98427847199059\\
1229	-8.05153293069429\\
1230	-9.74907091685509\\
1231	-11.3543862753484\\
1232	-11.735015793776\\
1233	-8.44826036209565\\
1234	-13.9972372033089\\
1235	-15.8776097544051\\
1236	-13.7443620497211\\
1237	-12.1931231402643\\
1238	-12.8084253065467\\
1239	-15.4239196315038\\
1240	-11.0880092138859\\
1241	-5.08600279211237\\
1242	-7.80768385388797\\
1243	-8.64343391897205\\
1244	-10.4070256921225\\
1245	-9.80905381718048\\
1246	-7.38839430039002\\
1247	-10.4772381622104\\
1248	-10.9863610264604\\
1249	-7.49534545818005\\
1250	-9.44647762313525\\
1251	-9.03570769639123\\
1252	-7.64451092707827\\
1253	-9.03311494252375\\
1254	-6.24410786621797\\
1255	-6.65134520937022\\
1256	-5.88527391237017\\
1257	-5.40092361675261\\
1258	-9.49470678537728\\
1259	-10.7828136334454\\
1260	-13.0810024122648\\
1261	-16.7011535239324\\
1262	-12.0718514983753\\
1263	-7.50823180379712\\
1264	-6.26876557440278\\
1265	-6.45692489887574\\
1266	-8.42481142299516\\
1267	-5.96804929340033\\
1268	-5.60056085170481\\
1269	-8.1238108518387\\
1270	-11.7480195162279\\
1271	-12.0302144326424\\
1272	-14.0394711365632\\
1273	-10.7625423450404\\
1274	-9.35771643212365\\
1275	-4.49317437612808\\
1276	-5.16724242382952\\
1277	-4.02925009788987\\
1278	-3.53785080319562\\
1279	-4.96125171764766\\
1280	-7.43533044145056\\
1281	-8.92500176869589\\
1282	-8.69301421534919\\
1283	-9.65236689906396\\
1284	-14.0294689222092\\
1285	-13.828680603455\\
1286	-10.3527662759539\\
1287	-11.9287929346094\\
1288	-12.9515282992706\\
1289	-14.8614935453406\\
1290	-14.3842871935851\\
1291	-8.7966143205465\\
1292	-5.21191465944185\\
1293	-3.80708281477905\\
1294	-3.90424581543903\\
1295	-4.93737385196982\\
1296	-5.21005594106373\\
1297	-4.67499403716124\\
1298	-5.79149701788606\\
1299	-8.99038010095387\\
1300	-8.38198813446684\\
1301	-8.0850349421969\\
1302	-9.14695198673735\\
1303	-6.11053535607516\\
1304	-3.42096523828036\\
1305	-6.59517801242464\\
1306	-10.311497108934\\
1307	-9.77923153251936\\
1308	-10.1789790099135\\
1309	-9.07747420994792\\
1310	-7.07421646877242\\
1311	-9.28265540255006\\
1312	-12.9814642818703\\
1313	-12.5546386535252\\
1314	-9.44436281857861\\
1315	-14.7559797156709\\
1316	-13.1898436707438\\
1317	-10.905581076855\\
1318	-13.7064091164307\\
1319	-12.7999397161332\\
1320	-10.8102744799817\\
1321	-9.68040615831774\\
1322	-15.6979734253234\\
1323	-19.7193408031135\\
1324	-16.0562137322659\\
1325	-10.1581844770883\\
1326	-10.1275888304031\\
1327	-10.9158145838189\\
1328	-12.3728035186759\\
1329	-13.7417177555519\\
1330	-11.2335441935211\\
1331	-11.0923816121963\\
1332	-14.6290296937614\\
1333	-14.2080164354122\\
1334	-11.8960888793001\\
1335	-8.5542079818556\\
1336	-11.936186310686\\
1337	-17.0486369330956\\
1338	-16.2403609901635\\
1339	-15.2262215769472\\
1340	-10.6142189050177\\
1341	-8.96702897461068\\
1342	-10.4656826289038\\
1343	-5.94603420837166\\
1344	-3.96727578154355\\
1345	-3.69177363766084\\
1346	-3.08200335487072\\
1347	-5.12591663236526\\
1348	-9.07512563228693\\
1349	-9.86756469634975\\
1350	-7.90636186735878\\
1351	-7.9347901247927\\
1352	-9.44650282447151\\
1353	-5.96156415946552\\
1354	-6.01414363338699\\
1355	-6.79806578972421\\
1356	-5.12419700637543\\
1357	-6.09028914162718\\
1358	-9.55262165186787\\
1359	-11.7401548127198\\
1360	-7.55546689364564\\
1361	-6.13999437489178\\
1362	-5.79156731665989\\
1363	-6.0584998182344\\
1364	-5.01975598334502\\
1365	-8.66964210075662\\
1366	-15.0928612669472\\
1367	-12.473464491605\\
1368	-15.2582227286614\\
1369	-17.13919619238\\
1370	-17.6079771152345\\
1371	-14.3311953131015\\
1372	-10.8971706806906\\
1373	-10.821825065399\\
1374	-11.4450104059231\\
1375	-12.3008802727172\\
1376	-13.9071878673656\\
1377	-14.1677095650093\\
1378	-17.6536321953356\\
1379	-19.720292672396\\
1380	-22.7681382955175\\
1381	-17.8679222095238\\
1382	-17.9120204495698\\
1383	-20.6024870720653\\
1384	-16.8557255918388\\
1385	-18.5659549006589\\
1386	-17.7924596438093\\
1387	-10.2478862889952\\
1388	-6.13572040192412\\
1389	-7.65890887386172\\
1390	-10.7694300097482\\
1391	-9.3393221372371\\
1392	-6.42024598486455\\
1393	-8.91114709139197\\
1394	-7.45579774655789\\
1395	-4.8266966066542\\
1396	-6.45447962569233\\
1397	-7.1939019253748\\
1398	-5.48614671905148\\
1399	-8.33808382247707\\
1400	-11.105566563012\\
1401	-8.40305435493383\\
1402	-13.0501704571464\\
1403	-17.9669562861075\\
1404	-17.0616465905389\\
1405	-20.1264173348761\\
};
\addlegendentry{MPO prediction}

\end{axis}

\begin{axis}[%
width=6.159cm,
height=1.831cm,
at={(8.104cm,5.085cm)},
scale only axis,
xmin=1000,
xmax=1405,
xlabel style={font=\color{white!15!black}},
xlabel={Sample index},
ymin=-404.053,
ymax=0,
ylabel style={font=\color{white!15!black}},
ylabel={$y(t)$},
axis background/.style={fill=white},
title style={font=\bfseries},
title={C6: RMSE(OSA) = 5.8768, RMSE(MPO) = 7.5894},
legend style={legend cell align=left, align=left, draw=white!15!black}
]
\addplot [color=mycolor1, line width=2.0pt]
  table[row sep=crcr]{%
1006	-185.547\\
1007	-219.727\\
1008	-169.678\\
1009	-86.6700000000001\\
1010	-107.422\\
1011	-102.539\\
1012	-126.953\\
1013	-103.76\\
1014	-41.5039999999999\\
1015	-31.7380000000001\\
1016	-28.076\\
1017	-89.1110000000001\\
1018	-157.471\\
1019	-167.236\\
1020	-174.561\\
1022	-74.463\\
1023	-158.691\\
1024	-111.084\\
1025	-83.008\\
1026	-142.822\\
1028	-103.76\\
1029	-173.34\\
1030	-164.795\\
1031	-124.512\\
1032	-180.664\\
1033	-179.443\\
1034	-142.822\\
1036	-89.1110000000001\\
1037	-108.643\\
1038	-137.939\\
1039	-130.615\\
1040	-136.719\\
1041	-140.381\\
1042	-151.367\\
1043	-213.623\\
1044	-191.65\\
1045	-126.953\\
1046	-115.967\\
1047	-64.6970000000001\\
1048	-69.5799999999999\\
1049	-96.4359999999999\\
1050	-74.463\\
1051	-78.125\\
1052	-111.084\\
1053	-128.174\\
1054	-119.629\\
1055	-190.43\\
1056	-139.16\\
1057	-81.787\\
1058	-63.4770000000001\\
1059	-85.4490000000001\\
1060	-101.318\\
1061	-51.27\\
1062	-58.5940000000001\\
1063	-81.787\\
1064	-65.9180000000001\\
1065	-56.152\\
1066	-74.463\\
1067	-86.6700000000001\\
1068	-139.16\\
1069	-130.615\\
1070	-163.574\\
1071	-146.484\\
1072	-170.898\\
1073	-139.16\\
1074	-133.057\\
1075	-115.967\\
1076	-111.084\\
1077	-111.084\\
1079	-280.762\\
1080	-291.748\\
1081	-280.762\\
1082	-175.781\\
1084	-303.955\\
1085	-317.383\\
1086	-239.258\\
1087	-313.721\\
1088	-404.053\\
1089	-289.307\\
1090	-238.037\\
1091	-158.691\\
1092	-122.07\\
1093	-102.539\\
1094	-129.395\\
1095	-92.7729999999999\\
1096	-61.0350000000001\\
1097	-59.8140000000001\\
1098	-75.684\\
1099	-118.408\\
1100	-107.422\\
1101	-111.084\\
1102	-146.484\\
1103	-152.588\\
1104	-207.52\\
1105	-167.236\\
1106	-208.74\\
1107	-148.926\\
1108	-130.615\\
1109	-151.367\\
1110	-108.643\\
1111	-98.877\\
1112	-122.07\\
1113	-123.291\\
1114	-85.4490000000001\\
1115	-95.2149999999999\\
1116	-67.1389999999999\\
1117	-76.904\\
1118	-81.787\\
1119	-57.373\\
1121	-93.9939999999999\\
1122	-151.367\\
1123	-152.588\\
1124	-170.898\\
1125	-97.6559999999999\\
1126	-140.381\\
1127	-211.182\\
1128	-161.133\\
1129	-186.768\\
1130	-191.65\\
1131	-112.305\\
1132	-75.684\\
1133	-107.422\\
1134	-123.291\\
1135	-205.078\\
1136	-252.686\\
1137	-252.686\\
1138	-184.326\\
1139	-189.209\\
1140	-167.236\\
1141	-163.574\\
1142	-162.354\\
1143	-108.643\\
1144	-96.4359999999999\\
1145	-86.6700000000001\\
1146	-78.125\\
1147	-87.8910000000001\\
1148	-133.057\\
1149	-203.857\\
1150	-161.133\\
1151	-109.863\\
1152	-92.7729999999999\\
1153	-89.1110000000001\\
1154	-52.49\\
1155	-41.5039999999999\\
1156	-47.607\\
1157	-90.3320000000001\\
1158	-67.1389999999999\\
1159	-78.125\\
1160	-85.4490000000001\\
1161	-80.566\\
1162	-54.932\\
1163	-37.8420000000001\\
1164	-30.518\\
1165	-54.932\\
1166	-119.629\\
1167	-172.119\\
1168	-194.092\\
1169	-136.719\\
1170	-91.5530000000001\\
1171	-64.6970000000001\\
1172	-43.9449999999999\\
1173	-98.877\\
1174	-91.5530000000001\\
1175	-137.939\\
1176	-168.457\\
1177	-275.879\\
1178	-316.162\\
1179	-260.01\\
1180	-249.023\\
1181	-159.912\\
1182	-185.547\\
1183	-195.313\\
1184	-212.402\\
1185	-158.691\\
1186	-144.043\\
1187	-142.822\\
1188	-129.395\\
1189	-156.25\\
1190	-109.863\\
1191	-195.313\\
1192	-234.375\\
1193	-175.781\\
1194	-104.98\\
1195	-111.084\\
1196	-192.871\\
1197	-234.375\\
1198	-289.307\\
1199	-303.955\\
1200	-302.734\\
1201	-228.271\\
1202	-205.078\\
1203	-235.596\\
1204	-275.879\\
1205	-170.898\\
1206	-104.98\\
1207	-152.588\\
1208	-122.07\\
1209	-79.346\\
1210	-109.863\\
1211	-122.07\\
1212	-95.2149999999999\\
1213	-75.684\\
1214	-101.318\\
1215	-80.566\\
1216	-115.967\\
1217	-162.354\\
1218	-147.705\\
1219	-100.098\\
1220	-101.318\\
1221	-157.471\\
1222	-261.23\\
1224	-118.408\\
1225	-85.4490000000001\\
1226	-101.318\\
1227	-90.3320000000001\\
1228	-67.1389999999999\\
1229	-76.904\\
1230	-92.7729999999999\\
1231	-120.85\\
1232	-134.277\\
1233	-89.1110000000001\\
1234	-156.25\\
1235	-179.443\\
1236	-170.898\\
1237	-140.381\\
1238	-150.146\\
1239	-202.637\\
1241	-53.711\\
1242	-78.125\\
1243	-85.4490000000001\\
1244	-119.629\\
1245	-101.318\\
1246	-74.463\\
1247	-118.408\\
1248	-112.305\\
1249	-79.346\\
1250	-102.539\\
1251	-90.3320000000001\\
1252	-80.566\\
1253	-95.2149999999999\\
1254	-54.932\\
1255	-68.3589999999999\\
1256	-47.607\\
1257	-51.27\\
1258	-106.201\\
1259	-122.07\\
1260	-167.236\\
1261	-219.727\\
1262	-142.822\\
1263	-83.008\\
1264	-63.4770000000001\\
1265	-62.2560000000001\\
1266	-91.5530000000001\\
1267	-54.932\\
1269	-85.4490000000001\\
1270	-150.146\\
1271	-133.057\\
1272	-190.43\\
1273	-124.512\\
1274	-117.188\\
1275	-56.152\\
1276	-62.2560000000001\\
1277	-34.1800000000001\\
1278	-46.3869999999999\\
1279	-51.27\\
1280	-95.2149999999999\\
1281	-100.098\\
1282	-117.188\\
1283	-123.291\\
1284	-197.754\\
1285	-170.898\\
1286	-128.174\\
1287	-153.809\\
1288	-163.574\\
1289	-213.623\\
1292	-53.711\\
1293	-40.2829999999999\\
1294	-41.5039999999999\\
1295	-52.49\\
1296	-54.932\\
1297	-51.27\\
1298	-70.8009999999999\\
1299	-108.643\\
1300	-98.877\\
1301	-103.76\\
1302	-118.408\\
1303	-69.5799999999999\\
1304	-36.6210000000001\\
1306	-125.732\\
1307	-125.732\\
1308	-142.822\\
1309	-118.408\\
1310	-87.8910000000001\\
1311	-123.291\\
1312	-172.119\\
1313	-172.119\\
1314	-120.85\\
1315	-207.52\\
1316	-155.029\\
1317	-151.367\\
1318	-173.34\\
1319	-172.119\\
1320	-130.615\\
1321	-112.305\\
1323	-266.113\\
1324	-202.637\\
1325	-124.512\\
1326	-118.408\\
1327	-115.967\\
1328	-150.146\\
1329	-173.34\\
1330	-125.732\\
1331	-134.277\\
1332	-177.002\\
1333	-192.871\\
1334	-120.85\\
1335	-97.6559999999999\\
1336	-130.615\\
1337	-218.506\\
1338	-195.313\\
1339	-203.857\\
1340	-115.967\\
1342	-101.318\\
1343	-63.4770000000001\\
1344	-41.5039999999999\\
1345	-34.1800000000001\\
1346	-29.297\\
1347	-78.125\\
1348	-108.643\\
1349	-131.836\\
1350	-93.9939999999999\\
1352	-119.629\\
1353	-72.021\\
1354	-79.346\\
1355	-73.242\\
1356	-54.932\\
1357	-75.684\\
1358	-120.85\\
1359	-156.25\\
1360	-95.2149999999999\\
1361	-73.242\\
1362	-58.5940000000001\\
1363	-74.463\\
1364	-45.1659999999999\\
1366	-195.313\\
1367	-163.574\\
1368	-216.064\\
1369	-241.699\\
1370	-249.023\\
1371	-181.885\\
1372	-131.836\\
1374	-129.395\\
1375	-155.029\\
1376	-184.326\\
1377	-181.885\\
1378	-247.803\\
1379	-270.996\\
1380	-328.369\\
1381	-217.285\\
1382	-238.037\\
1383	-264.893\\
1384	-205.078\\
1385	-239.258\\
1386	-200.195\\
1387	-113.525\\
1388	-74.463\\
1389	-79.346\\
1390	-117.188\\
1391	-97.6559999999999\\
1392	-64.6970000000001\\
1393	-90.3320000000001\\
1394	-63.4770000000001\\
1395	-48.828\\
1396	-64.6970000000001\\
1397	-72.021\\
1398	-48.828\\
1399	-100.098\\
1400	-135.498\\
1401	-97.6559999999999\\
1403	-241.699\\
1404	-220.947\\
1405	-279.541\\
};
\addlegendentry{True output}

\addplot [color=mycolor2, dashed, line width=2.0pt]
  table[row sep=crcr]{%
1006	-172.010378631054\\
1007	-211.918675751263\\
1008	-172.693139661322\\
1009	-81.6122750116901\\
1010	-107.028756733575\\
1011	-109.839332893192\\
1012	-130.587043111721\\
1013	-102.443671178399\\
1014	-33.9792912834178\\
1015	-36.9054311491623\\
1016	-29.7570413425472\\
1018	-149.56199066639\\
1019	-164.504927196799\\
1020	-162.170388873548\\
1021	-125.718149822819\\
1022	-75.7444945801651\\
1023	-163.270660450885\\
1024	-116.164816992277\\
1025	-79.1423274264052\\
1026	-143.851868783818\\
1027	-121.283419821125\\
1028	-106.213157021848\\
1029	-169.287129032768\\
1030	-161.662189447781\\
1031	-118.569696772933\\
1032	-177.649833953728\\
1033	-174.159890507108\\
1034	-140.393893458525\\
1036	-94.8897078948096\\
1037	-104.342522203841\\
1038	-130.304929452446\\
1039	-124.589086906544\\
1040	-134.265801266709\\
1041	-142.613247917592\\
1042	-149.888362021168\\
1043	-198.460758504744\\
1044	-183.533826223289\\
1045	-127.99456176254\\
1046	-122.499169432911\\
1047	-65.8237074759011\\
1048	-69.3480179338617\\
1049	-93.096347025833\\
1050	-76.4915595180537\\
1051	-75.5685957712178\\
1052	-105.796000313186\\
1053	-128.834817587964\\
1054	-118.845657820564\\
1055	-181.204439213791\\
1056	-145.769413847177\\
1057	-83.7132903959027\\
1058	-68.2974685153993\\
1060	-100.265033540657\\
1061	-53.0621990895941\\
1062	-51.4795148063581\\
1063	-83.3903843328039\\
1065	-54.0474262881153\\
1066	-76.0866729297575\\
1067	-86.0893746861386\\
1068	-129.876500852092\\
1069	-138.016489551653\\
1070	-153.900727759592\\
1071	-139.720623356452\\
1072	-170.196244906942\\
1073	-134.098062706719\\
1074	-139.14422829565\\
1075	-115.264683972467\\
1076	-111.537654197326\\
1077	-112.512177705091\\
1079	-273.321316596471\\
1080	-281.222626089532\\
1081	-262.913328317447\\
1082	-172.461933192623\\
1083	-243.851281934752\\
1084	-299.386959583766\\
1085	-294.008568452356\\
1086	-233.844659280978\\
1087	-297.220767051334\\
1088	-392.557078217418\\
1089	-299.982401995104\\
1090	-250.466353819492\\
1091	-148.223380972505\\
1092	-119.779477563513\\
1093	-103.147013769748\\
1094	-119.783543826208\\
1095	-94.4888557995434\\
1096	-54.6655126050284\\
1097	-45.1640040706407\\
1098	-72.4084193883175\\
1099	-119.129800469032\\
1100	-115.430770214369\\
1101	-114.36962047701\\
1102	-146.958885949437\\
1103	-152.000245135991\\
1104	-211.163735523403\\
1105	-179.830808105325\\
1106	-198.140728437546\\
1107	-151.612439570118\\
1108	-133.074509664312\\
1109	-156.622302162165\\
1110	-111.239656105171\\
1111	-105.009641448059\\
1112	-119.066678776077\\
1113	-129.992506770892\\
1114	-87.7850590115884\\
1115	-98.0644136686747\\
1116	-67.9091740018837\\
1117	-80.2668883825384\\
1118	-83.8454464532915\\
1119	-59.9417783030813\\
1120	-67.7123413289169\\
1121	-93.7825293425835\\
1122	-153.234876046248\\
1123	-157.861023472323\\
1124	-169.759627789853\\
1125	-94.4394219069172\\
1126	-135.863816962161\\
1127	-215.234699913015\\
1128	-166.851812754179\\
1129	-190.162908167856\\
1130	-185.882611975015\\
1131	-112.387576397249\\
1132	-83.2939708112485\\
1134	-123.788025028863\\
1135	-203.641055845625\\
1136	-251.362783792388\\
1137	-248.877123536768\\
1138	-184.968151191233\\
1139	-189.06695055443\\
1140	-176.740103281937\\
1141	-173.240959225824\\
1142	-160.174294262926\\
1143	-104.865667107103\\
1144	-91.8570875021273\\
1145	-88.3526186256474\\
1146	-83.6785625639386\\
1147	-89.2216546392303\\
1148	-129.922160364613\\
1149	-208.979695151951\\
1150	-167.413541719031\\
1151	-107.583645241869\\
1153	-91.5904163447765\\
1154	-49.2025350250012\\
1155	-35.7194726250057\\
1156	-41.0456011529675\\
1157	-91.2408897702458\\
1158	-75.8468425263545\\
1159	-79.5467480430782\\
1160	-79.4853692063525\\
1161	-78.8914257953447\\
1162	-61.491016326906\\
1163	-39.3467185165136\\
1164	-28.6256253265665\\
1165	-50.7456546837589\\
1166	-119.48985516071\\
1167	-168.783838036087\\
1168	-195.773265863671\\
1169	-139.564597129007\\
1170	-90.4367127066805\\
1171	-69.6436942820812\\
1172	-43.62506863607\\
1173	-92.7913881581139\\
1174	-95.0110971610843\\
1175	-130.632065719438\\
1176	-178.84309200932\\
1177	-281.873297124665\\
1178	-342.085934907652\\
1179	-267.242570102423\\
1180	-248.918298099846\\
1181	-158.332607250558\\
1182	-173.573999825525\\
1183	-202.126015919515\\
1184	-205.177345042683\\
1185	-159.668810867505\\
1186	-144.039988202243\\
1187	-151.201783694902\\
1188	-131.599768139614\\
1189	-153.082832181148\\
1190	-115.50656480661\\
1191	-197.025107491673\\
1192	-225.817602875582\\
1193	-182.403580560032\\
1194	-102.855888676345\\
1195	-109.739175578894\\
1196	-194.885573958672\\
1198	-274.200096422486\\
1199	-296.718053411409\\
1200	-299.392423226058\\
1201	-225.750295311026\\
1202	-215.325673885541\\
1203	-233.087599846553\\
1204	-269.331834052804\\
1205	-172.043694980231\\
1206	-99.9370807162647\\
1207	-149.208081145306\\
1208	-124.290761692881\\
1209	-82.782629539766\\
1210	-99.9726997004284\\
1211	-127.169285333673\\
1213	-78.0667619346\\
1214	-100.076061493403\\
1215	-88.1831985983063\\
1216	-118.303154699553\\
1217	-175.762841554248\\
1218	-143.685294691163\\
1219	-105.970743931327\\
1220	-108.90819680847\\
1221	-158.335536945043\\
1222	-263.433923184239\\
1223	-197.192763494919\\
1224	-115.89050122403\\
1225	-85.1897926017293\\
1226	-100.246545223459\\
1227	-102.862144729026\\
1228	-69.2616608023952\\
1229	-72.6122569884262\\
1230	-98.7387698537748\\
1231	-121.81418085399\\
1232	-136.783193417222\\
1233	-89.9813154243311\\
1234	-162.22059292837\\
1235	-185.523858088795\\
1236	-164.743244681433\\
1237	-151.595722172219\\
1238	-147.492857424924\\
1239	-204.717802917181\\
1241	-51.6895887184155\\
1242	-74.3455737424729\\
1243	-88.4639399545993\\
1244	-125.10618008064\\
1245	-108.354267372771\\
1246	-80.3040267652418\\
1247	-120.397573688828\\
1248	-116.780280388955\\
1249	-86.1078036463741\\
1250	-99.7179213708625\\
1251	-91.687388584214\\
1252	-89.0554762613067\\
1253	-99.0848077435978\\
1254	-54.0374197206784\\
1255	-70.4400692639917\\
1256	-48.8189109443024\\
1257	-55.101554594416\\
1258	-111.978273052232\\
1259	-117.401486024748\\
1260	-173.005070833644\\
1261	-221.448676271775\\
1263	-85.8267572045936\\
1264	-61.506897545325\\
1265	-68.8409560843779\\
1266	-93.0630405610052\\
1267	-54.7102816461893\\
1268	-64.4638049734949\\
1269	-86.6605679873637\\
1270	-154.175072070732\\
1271	-140.193728981227\\
1272	-191.364168437148\\
1273	-120.071867336293\\
1274	-123.074747602554\\
1275	-53.4314959199783\\
1276	-60.3651167450437\\
1277	-36.3211937878655\\
1278	-43.7240999869207\\
1279	-47.3806659826453\\
1280	-95.712028402\\
1281	-101.643985528772\\
1282	-115.77472328983\\
1283	-126.749510365143\\
1284	-197.675097655567\\
1285	-193.290216761539\\
1286	-131.979098168984\\
1287	-147.521271152958\\
1288	-161.007418798995\\
1289	-206.508242268968\\
1291	-112.342036836474\\
1292	-52.206406512053\\
1293	-43.9530233364783\\
1294	-39.2525694040178\\
1295	-51.3661459842533\\
1296	-57.4100670811156\\
1297	-49.2559931891587\\
1298	-67.9515643762334\\
1299	-112.210285273333\\
1300	-105.225376328502\\
1301	-95.9437407561702\\
1302	-116.726134894121\\
1303	-72.683705501189\\
1304	-40.4263530337475\\
1306	-126.353187045652\\
1307	-116.551416708723\\
1308	-133.949480056039\\
1309	-126.326263715904\\
1310	-89.8760868879085\\
1311	-119.102832266091\\
1312	-170.970092014129\\
1313	-166.654384106255\\
1314	-123.24113863983\\
1315	-203.361515633951\\
1316	-158.622494651585\\
1317	-152.390704382425\\
1318	-173.446588960485\\
1319	-163.621339991637\\
1320	-132.456588966823\\
1321	-119.977074393513\\
1322	-180.806041839734\\
1323	-261.328126382444\\
1324	-207.793981996785\\
1325	-120.719047161332\\
1327	-119.904682295797\\
1328	-148.232718923076\\
1329	-164.658501015255\\
1330	-133.348122169703\\
1331	-138.119246240745\\
1332	-173.940734382032\\
1333	-181.857833733938\\
1334	-123.740458077907\\
1335	-105.611909827082\\
1336	-134.604700240745\\
1337	-206.042488020533\\
1338	-199.585895882446\\
1339	-198.874069978816\\
1340	-111.497866110996\\
1341	-110.236413865282\\
1342	-112.86201291432\\
1343	-61.4297838144601\\
1344	-33.6201218680321\\
1345	-30.3204543016211\\
1346	-28.1817975764441\\
1347	-74.5706591798992\\
1348	-107.535228279811\\
1349	-128.217291438948\\
1350	-99.3993407375488\\
1351	-106.543756652836\\
1352	-115.841520293287\\
1353	-75.065618703321\\
1354	-77.110852367397\\
1355	-76.874051624028\\
1356	-56.8905067015123\\
1357	-75.8326140611437\\
1359	-152.359772473188\\
1360	-95.4883627701033\\
1361	-80.7805618667157\\
1362	-60.2692027170924\\
1363	-73.9992690561317\\
1364	-49.2462627345969\\
1366	-189.496982430347\\
1367	-163.232589208815\\
1368	-210.49221620448\\
1369	-245.720953676278\\
1370	-246.368543167541\\
1371	-187.567819342392\\
1372	-138.141058868737\\
1373	-126.580996660381\\
1374	-133.655672907787\\
1376	-173.074961717554\\
1377	-174.677882343506\\
1378	-238.20426309487\\
1379	-269.115615846569\\
1380	-324.93790972504\\
1381	-212.689893706159\\
1382	-243.464885499779\\
1383	-269.608691292822\\
1384	-199.165043050324\\
1385	-239.656902256143\\
1386	-194.393844136425\\
1387	-112.662857611324\\
1388	-66.4012761489539\\
1389	-70.6710392205491\\
1390	-123.825483990049\\
1391	-105.869605576148\\
1392	-64.6105979731101\\
1393	-85.8613081178476\\
1394	-67.6252548251589\\
1395	-45.5502039227611\\
1396	-65.6125500557976\\
1397	-69.2296217322751\\
1398	-56.4579524802289\\
1400	-142.136036742025\\
1401	-100.792686457733\\
1402	-171.099398547213\\
1403	-257.620918185745\\
1404	-233.226263319512\\
1405	-286.035208922001\\
};
\addlegendentry{OSA predition}

\addplot [color=mycolor3, dotted, line width=2.0pt]
  table[row sep=crcr]{%
1006	-185.547\\
1007	-219.727\\
1008	-169.678\\
1009	-86.6700000000001\\
1010	-107.02875673358\\
1011	-109.742353078777\\
1012	-132.228258882878\\
1013	-106.199948836171\\
1014	-36.0545924083908\\
1015	-36.677625508552\\
1016	-29.4583817312864\\
1018	-151.589373245242\\
1019	-164.550348159093\\
1020	-159.925376608622\\
1021	-121.633385648453\\
1022	-69.6243599834315\\
1023	-159.449108828175\\
1024	-113.828579960995\\
1025	-80.0282099586857\\
1026	-143.885054355085\\
1027	-120.646685826593\\
1028	-106.019623204646\\
1029	-168.885921600512\\
1030	-161.351628813077\\
1031	-116.132522488355\\
1032	-174.137851204946\\
1033	-169.313597131689\\
1034	-135.627248317323\\
1036	-90.776516585111\\
1037	-102.440299245025\\
1038	-128.705440168341\\
1039	-120.126983961716\\
1042	-144.921531819087\\
1043	-194.133097922997\\
1044	-176.710037565898\\
1045	-117.781099767332\\
1046	-114.099064429139\\
1047	-61.1526253436978\\
1048	-66.8659957628715\\
1049	-90.5976928346638\\
1050	-73.9378913019675\\
1051	-73.2461199888071\\
1052	-103.82501663298\\
1053	-125.084529906819\\
1054	-115.457828725234\\
1055	-178.079143243077\\
1056	-141.233449147452\\
1057	-79.9329522035518\\
1058	-68.2653379854025\\
1060	-101.784241471462\\
1061	-52.8680847999894\\
1062	-52.283632424331\\
1063	-82.3028815714511\\
1064	-66.5886391989322\\
1065	-54.4072699942546\\
1066	-75.5712389230805\\
1067	-85.868143157667\\
1068	-129.871352591398\\
1069	-135.562008141486\\
1070	-152.0042483338\\
1071	-138.340610772503\\
1072	-164.181664798409\\
1073	-129.704554121198\\
1074	-133.683487134068\\
1075	-111.484121723139\\
1076	-110.039760367618\\
1077	-110.24435565383\\
1079	-271.513890067712\\
1080	-277.8672213762\\
1081	-257.318676600674\\
1082	-163.513137611743\\
1083	-230.768954695468\\
1084	-289.635147484739\\
1085	-286.419122788572\\
1086	-223.842307931251\\
1087	-280.846838149509\\
1088	-377.277588720642\\
1089	-284.477020712003\\
1090	-238.433628925385\\
1091	-144.92644788725\\
1092	-115.442153602362\\
1093	-95.9255542313545\\
1094	-115.064111044564\\
1095	-87.8086818296613\\
1096	-48.0250641169127\\
1097	-39.5372426752642\\
1098	-61.5930564121884\\
1099	-107.173605819466\\
1100	-106.185433377793\\
1101	-108.68211246467\\
1102	-144.332990251969\\
1103	-150.033093578503\\
1104	-209.496665867009\\
1105	-179.094780070125\\
1106	-201.166800011964\\
1107	-154.362715392183\\
1108	-131.784337399363\\
1109	-159.044391093993\\
1110	-113.102088752473\\
1111	-109.144235785035\\
1112	-123.671919003575\\
1113	-134.641865341041\\
1114	-91.2984592353489\\
1115	-104.105109376371\\
1116	-72.4144907542202\\
1117	-85.4672368968857\\
1118	-88.5260684141645\\
1119	-65.0338082857338\\
1120	-72.5159374567861\\
1121	-96.4257642096447\\
1122	-154.063364764192\\
1123	-159.390544423795\\
1124	-172.229149684278\\
1125	-97.1872082722641\\
1126	-136.704816226961\\
1127	-214.706455064641\\
1128	-166.308131173746\\
1129	-192.296771414817\\
1130	-188.956469081097\\
1131	-113.912328366762\\
1132	-82.9222732072255\\
1133	-106.684907267462\\
1134	-126.146145089833\\
1135	-205.166229233614\\
1136	-252.618968644483\\
1137	-249.090758625005\\
1138	-184.482266198198\\
1139	-187.749815009811\\
1140	-176.341722140381\\
1141	-174.519842991211\\
1142	-165.910702282484\\
1143	-109.605855540261\\
1144	-93.970546444235\\
1146	-83.5169798053435\\
1147	-91.2617327231617\\
1148	-132.691851976394\\
1149	-210.963960940298\\
1150	-169.104561723793\\
1151	-112.028404061018\\
1152	-103.079692104972\\
1153	-95.5235694956341\\
1154	-54.8440718936133\\
1155	-38.6508019663345\\
1156	-41.7756878503437\\
1157	-89.2808201054315\\
1158	-73.7062111206135\\
1159	-80.5238533351192\\
1160	-81.935938309638\\
1161	-78.9593353448874\\
1163	-40.3271740799087\\
1164	-30.8661070512267\\
1165	-51.8257875662014\\
1166	-119.443406408962\\
1167	-167.980236914523\\
1168	-194.41473491222\\
1169	-138.098050601255\\
1170	-90.9407737666345\\
1171	-69.7837616569605\\
1172	-44.7346454186911\\
1173	-94.7654538172419\\
1174	-94.5247557055661\\
1175	-130.386801044206\\
1176	-177.557161715826\\
1177	-281.747538185038\\
1178	-344.715220919636\\
1179	-272.065183347905\\
1180	-262.147953103643\\
1181	-165.335754919433\\
1182	-181.658098463664\\
1183	-204.854793717264\\
1184	-206.610009181511\\
1185	-162.2679319317\\
1186	-142.664606094905\\
1187	-152.43709030476\\
1188	-133.126247682629\\
1189	-157.581274560409\\
1190	-117.268745534593\\
1191	-200.172075230252\\
1192	-229.832372965031\\
1193	-183.541012827121\\
1194	-103.073758878921\\
1195	-112.317543335281\\
1196	-195.040097174163\\
1198	-275.000505513828\\
1199	-294.919111600359\\
1200	-292.951098696117\\
1201	-220.837765464645\\
1202	-209.165181301827\\
1203	-229.947222444166\\
1204	-268.777814830419\\
1205	-168.423880972122\\
1206	-97.0500667602764\\
1207	-145.618952061781\\
1208	-119.253260162056\\
1209	-79.534581608151\\
1210	-98.3143032245662\\
1211	-123.858702769629\\
1212	-98.5043746453862\\
1213	-79.4204355374959\\
1214	-101.71792558725\\
1215	-89.7850521738803\\
1216	-121.183229448058\\
1217	-180.45992610545\\
1218	-149.843526299785\\
1219	-113.238751679018\\
1220	-113.756609573883\\
1221	-167.209008574574\\
1222	-272.445798985773\\
1223	-203.435393908812\\
1224	-123.812531604106\\
1225	-93.7592960708873\\
1226	-105.559746751226\\
1227	-108.017099503378\\
1228	-75.4562530385001\\
1229	-81.4569889480756\\
1231	-127.669263590464\\
1232	-142.995846595019\\
1233	-94.5882932267787\\
1234	-167.858242065041\\
1235	-191.254617051218\\
1236	-171.914290160172\\
1237	-156.677023469334\\
1238	-152.653858581045\\
1239	-211.891080194716\\
1240	-127.654079854894\\
1241	-56.1430971656657\\
1242	-76.0875770949374\\
1243	-89.3654012821805\\
1244	-125.707321623022\\
1245	-111.165126254659\\
1246	-84.7125170294464\\
1247	-127.047882483386\\
1248	-123.240992607964\\
1249	-92.2351016286184\\
1250	-107.660826631168\\
1251	-98.2509849941171\\
1252	-93.7056898867525\\
1253	-106.015783131413\\
1254	-61.4117080594322\\
1255	-76.6052368825115\\
1256	-53.9571793403964\\
1257	-60.1963394246857\\
1258	-117.609995936145\\
1259	-124.047744707065\\
1260	-178.633611512216\\
1261	-226.630149966572\\
1263	-91.7616078310518\\
1264	-70.3875537795457\\
1265	-74.6058711137985\\
1266	-99.9715022973453\\
1267	-61.3783372350881\\
1268	-69.6060081051714\\
1269	-89.6504724197509\\
1270	-156.130159715046\\
1271	-143.171914041991\\
1272	-196.144108165421\\
1273	-125.015300160264\\
1274	-125.70075061788\\
1275	-56.0778190841729\\
1276	-63.6434927869102\\
1277	-36.7525125022066\\
1278	-45.2811136070923\\
1279	-47.9796431563104\\
1280	-94.8434948369584\\
1281	-100.55332862739\\
1282	-115.476947263984\\
1283	-126.293878572685\\
1284	-197.870162369563\\
1285	-194.202182770783\\
1286	-136.884421365508\\
1287	-158.055967967111\\
1288	-166.531374700729\\
1289	-210.643757201658\\
1291	-111.774458139284\\
1292	-54.6285556715266\\
1293	-46.5214826834258\\
1294	-41.4263398033499\\
1295	-53.8015213502715\\
1296	-58.0514348218128\\
1297	-50.7825636236673\\
1298	-68.881837706748\\
1299	-111.97884574229\\
1301	-98.5808858467869\\
1302	-117.784555300411\\
1303	-70.9268649063488\\
1304	-41.0341009214924\\
1306	-128.374981270849\\
1307	-118.057078582158\\
1308	-133.285301124217\\
1309	-121.675201581545\\
1310	-87.5269151317459\\
1311	-119.189920556843\\
1312	-169.736320639704\\
1313	-164.883678289676\\
1314	-120.798157029035\\
1315	-200.338354231919\\
1316	-156.759881480952\\
1317	-149.909477930313\\
1318	-173.250921424805\\
1319	-162.801694322398\\
1320	-130.427292705376\\
1321	-116.49462528095\\
1322	-180.900002942325\\
1323	-259.808044565773\\
1324	-202.729428518556\\
1325	-119.125780124246\\
1327	-117.563577564425\\
1328	-148.246273436566\\
1329	-164.515937962945\\
1330	-130.758433379143\\
1331	-136.168724131236\\
1332	-175.414946457691\\
1333	-182.094865579545\\
1334	-121.145228936967\\
1335	-101.850421570094\\
1336	-135.042742364613\\
1337	-208.097435229465\\
1338	-199.438059447692\\
1339	-196.456938948919\\
1340	-111.910357057636\\
1341	-106.034787167633\\
1342	-110.620739811959\\
1343	-62.461964234742\\
1344	-36.4648008679387\\
1345	-28.6071486824508\\
1346	-25.4514755648011\\
1347	-71.2681468443639\\
1348	-103.623530710263\\
1349	-124.200327647454\\
1350	-95.6359918876165\\
1351	-103.861260719729\\
1352	-114.843820117595\\
1353	-72.3128640290854\\
1354	-75.5651454852359\\
1355	-75.6599902086728\\
1356	-56.0155975698058\\
1357	-76.7335666474098\\
1359	-151.592018005383\\
1360	-92.6486353026935\\
1361	-78.6112936344787\\
1362	-60.4090427998856\\
1363	-75.9659429225107\\
1364	-50.1534405955929\\
1366	-192.952130449929\\
1367	-165.03626155489\\
1368	-210.863850428635\\
1369	-245.274364369432\\
1370	-245.12054208262\\
1372	-137.757040420858\\
1373	-130.433229650303\\
1374	-135.882014181161\\
1376	-175.895658832525\\
1377	-173.572260956235\\
1378	-233.801089397334\\
1379	-262.802181888207\\
1380	-317.121786129802\\
1381	-208.025109898807\\
1383	-264.392319703629\\
1384	-197.753844065014\\
1385	-237.420031334386\\
1386	-191.133662698239\\
1387	-110.277735004903\\
1388	-61.7995035588119\\
1389	-65.8662389236831\\
1390	-115.083675257164\\
1391	-100.048715700574\\
1392	-64.0215016650327\\
1393	-85.7686272410194\\
1395	-44.819652751647\\
1396	-65.2538057090349\\
1397	-67.9121895650139\\
1398	-55.6558620825349\\
1400	-143.35996422068\\
1401	-101.802343362286\\
1402	-174.839479154904\\
1403	-260.654949664821\\
1404	-237.636901183788\\
1405	-296.177879506107\\
};
\addlegendentry{MPO prediction}

\end{axis}

\begin{axis}[%
width=6.159cm,
height=1.831cm,
at={(0cm,2.542cm)},
scale only axis,
xmin=1000,
xmax=1405,
xlabel style={font=\color{white!15!black}},
xlabel={Sample index},
ymin=-400,
ymax=0,
ylabel style={font=\color{white!15!black}},
ylabel={$y(t)$},
axis background/.style={fill=white},
title style={font=\bfseries},
title={C7: RMSE(OSA) = 5.0395, RMSE(MPO) = 6.4094},
legend style={legend cell align=left, align=left, draw=white!15!black}
]
\addplot [color=mycolor1, line width=2.0pt]
  table[row sep=crcr]{%
1006	-147.705\\
1007	-175.781\\
1008	-134.277\\
1009	-68.3589999999999\\
1010	-80.566\\
1011	-80.566\\
1012	-100.098\\
1013	-81.787\\
1014	-34.1800000000001\\
1015	-24.414\\
1016	-23.193\\
1018	-128.174\\
1020	-137.939\\
1021	-92.7729999999999\\
1022	-56.152\\
1023	-125.732\\
1024	-86.6700000000001\\
1025	-68.3589999999999\\
1026	-114.746\\
1028	-85.4490000000001\\
1029	-142.822\\
1030	-133.057\\
1031	-102.539\\
1032	-150.146\\
1033	-145.264\\
1034	-113.525\\
1036	-72.021\\
1037	-86.6700000000001\\
1038	-112.305\\
1039	-103.76\\
1040	-109.863\\
1041	-112.305\\
1042	-122.07\\
1043	-175.781\\
1044	-155.029\\
1045	-102.539\\
1046	-92.7729999999999\\
1047	-53.711\\
1048	-58.5940000000001\\
1049	-79.346\\
1050	-59.8140000000001\\
1051	-62.2560000000001\\
1052	-89.1110000000001\\
1053	-102.539\\
1054	-95.2149999999999\\
1055	-151.367\\
1056	-111.084\\
1057	-64.6970000000001\\
1058	-51.27\\
1059	-69.5799999999999\\
1060	-81.787\\
1061	-43.9449999999999\\
1062	-45.1659999999999\\
1063	-67.1389999999999\\
1064	-51.27\\
1065	-42.7249999999999\\
1066	-61.0350000000001\\
1067	-70.8009999999999\\
1068	-109.863\\
1069	-104.98\\
1070	-129.395\\
1071	-120.85\\
1072	-137.939\\
1073	-109.863\\
1074	-107.422\\
1075	-92.7729999999999\\
1076	-91.5530000000001\\
1077	-87.8910000000001\\
1079	-224.609\\
1080	-231.934\\
1081	-225.83\\
1082	-140.381\\
1083	-205.078\\
1084	-252.686\\
1085	-261.23\\
1086	-200.195\\
1088	-335.693\\
1089	-235.596\\
1090	-191.65\\
1091	-128.174\\
1092	-98.877\\
1093	-84.229\\
1094	-104.98\\
1095	-74.463\\
1096	-50.049\\
1097	-48.828\\
1098	-63.4770000000001\\
1099	-95.2149999999999\\
1100	-86.6700000000001\\
1101	-89.1110000000001\\
1102	-119.629\\
1103	-123.291\\
1104	-167.236\\
1105	-134.277\\
1106	-169.678\\
1107	-122.07\\
1108	-108.643\\
1109	-125.732\\
1110	-89.1110000000001\\
1111	-80.566\\
1112	-97.6559999999999\\
1113	-98.877\\
1114	-68.3589999999999\\
1115	-73.242\\
1116	-54.932\\
1117	-61.0350000000001\\
1118	-64.6970000000001\\
1119	-45.1659999999999\\
1121	-72.021\\
1122	-117.188\\
1123	-122.07\\
1124	-136.719\\
1125	-76.904\\
1126	-109.863\\
1127	-173.34\\
1128	-131.836\\
1129	-157.471\\
1130	-157.471\\
1131	-85.4490000000001\\
1132	-59.8140000000001\\
1133	-85.4490000000001\\
1134	-98.877\\
1135	-161.133\\
1136	-202.637\\
1137	-198.975\\
1138	-145.264\\
1139	-150.146\\
1140	-135.498\\
1141	-131.836\\
1142	-126.953\\
1143	-85.4490000000001\\
1144	-76.904\\
1146	-62.2560000000001\\
1147	-70.8009999999999\\
1148	-107.422\\
1149	-164.795\\
1151	-86.6700000000001\\
1152	-73.242\\
1153	-70.8009999999999\\
1154	-42.7249999999999\\
1155	-31.7380000000001\\
1156	-39.0630000000001\\
1157	-72.021\\
1158	-57.373\\
1159	-59.8140000000001\\
1160	-67.1389999999999\\
1161	-63.4770000000001\\
1162	-43.9449999999999\\
1163	-29.297\\
1164	-25.635\\
1165	-43.9449999999999\\
1166	-96.4359999999999\\
1167	-140.381\\
1168	-155.029\\
1169	-101.318\\
1170	-69.5799999999999\\
1171	-50.049\\
1172	-34.1800000000001\\
1173	-83.008\\
1174	-81.787\\
1175	-119.629\\
1176	-145.264\\
1177	-230.713\\
1178	-262.451\\
1179	-211.182\\
1180	-202.637\\
1181	-136.719\\
1182	-151.367\\
1183	-159.912\\
1184	-172.119\\
1185	-129.395\\
1186	-118.408\\
1187	-115.967\\
1188	-103.76\\
1189	-123.291\\
1190	-87.8910000000001\\
1191	-153.809\\
1192	-189.209\\
1194	-78.125\\
1195	-85.4490000000001\\
1196	-146.484\\
1197	-181.885\\
1198	-229.492\\
1199	-241.699\\
1200	-240.479\\
1201	-180.664\\
1202	-163.574\\
1203	-187.988\\
1204	-222.168\\
1205	-139.16\\
1206	-81.787\\
1207	-122.07\\
1208	-102.539\\
1209	-65.9180000000001\\
1210	-87.8910000000001\\
1211	-98.877\\
1212	-76.904\\
1213	-58.5940000000001\\
1214	-80.566\\
1215	-65.9180000000001\\
1216	-91.5530000000001\\
1217	-133.057\\
1218	-122.07\\
1219	-79.346\\
1220	-83.008\\
1221	-126.953\\
1222	-209.961\\
1224	-92.7729999999999\\
1225	-69.5799999999999\\
1226	-83.008\\
1227	-74.463\\
1228	-56.152\\
1229	-62.2560000000001\\
1230	-74.463\\
1231	-96.4359999999999\\
1232	-107.422\\
1233	-72.021\\
1234	-122.07\\
1235	-144.043\\
1236	-137.939\\
1237	-106.201\\
1238	-118.408\\
1239	-158.691\\
1241	-41.5039999999999\\
1242	-58.5940000000001\\
1243	-65.9180000000001\\
1244	-91.5530000000001\\
1245	-81.787\\
1246	-59.8140000000001\\
1247	-96.4359999999999\\
1248	-91.5530000000001\\
1249	-65.9180000000001\\
1250	-83.008\\
1251	-76.904\\
1252	-67.1389999999999\\
1253	-79.346\\
1254	-46.3869999999999\\
1255	-53.711\\
1256	-40.2829999999999\\
1257	-43.9449999999999\\
1258	-86.6700000000001\\
1259	-100.098\\
1261	-177.002\\
1262	-115.967\\
1263	-68.3589999999999\\
1264	-48.828\\
1265	-48.828\\
1266	-73.242\\
1267	-46.3869999999999\\
1268	-52.49\\
1269	-70.8009999999999\\
1270	-119.629\\
1271	-107.422\\
1272	-150.146\\
1273	-102.539\\
1274	-91.5530000000001\\
1275	-47.607\\
1276	-47.607\\
1277	-31.7380000000001\\
1278	-34.1800000000001\\
1279	-41.5039999999999\\
1280	-75.684\\
1281	-83.008\\
1282	-92.7729999999999\\
1283	-97.6559999999999\\
1284	-152.588\\
1285	-130.615\\
1286	-97.6559999999999\\
1287	-119.629\\
1288	-129.395\\
1289	-167.236\\
1290	-133.057\\
1291	-84.229\\
1292	-43.9449999999999\\
1293	-31.7380000000001\\
1294	-34.1800000000001\\
1295	-41.5039999999999\\
1296	-43.9449999999999\\
1297	-40.2829999999999\\
1298	-56.152\\
1299	-87.8910000000001\\
1300	-79.346\\
1301	-81.787\\
1302	-96.4359999999999\\
1303	-58.5940000000001\\
1304	-29.297\\
1306	-97.6559999999999\\
1307	-96.4359999999999\\
1308	-109.863\\
1309	-93.9939999999999\\
1310	-69.5799999999999\\
1311	-97.6559999999999\\
1312	-137.939\\
1313	-139.16\\
1314	-97.6559999999999\\
1315	-162.354\\
1316	-128.174\\
1317	-118.408\\
1318	-137.939\\
1319	-137.939\\
1320	-103.76\\
1321	-90.3320000000001\\
1323	-214.844\\
1324	-164.795\\
1325	-92.7729999999999\\
1326	-92.7729999999999\\
1327	-91.5530000000001\\
1329	-136.719\\
1330	-100.098\\
1331	-104.98\\
1332	-140.381\\
1333	-153.809\\
1334	-103.76\\
1335	-79.346\\
1336	-104.98\\
1337	-175.781\\
1338	-159.912\\
1339	-162.354\\
1340	-95.2149999999999\\
1341	-84.229\\
1342	-83.008\\
1343	-52.49\\
1344	-34.1800000000001\\
1345	-28.076\\
1346	-23.193\\
1347	-59.8140000000001\\
1348	-86.6700000000001\\
1349	-104.98\\
1350	-76.904\\
1352	-97.6559999999999\\
1353	-59.8140000000001\\
1354	-63.4770000000001\\
1355	-61.0350000000001\\
1356	-45.1659999999999\\
1357	-57.373\\
1358	-97.6559999999999\\
1359	-124.512\\
1360	-78.125\\
1361	-56.152\\
1362	-46.3869999999999\\
1363	-59.8140000000001\\
1364	-41.5039999999999\\
1365	-91.5530000000001\\
1366	-158.691\\
1367	-122.07\\
1368	-167.236\\
1369	-194.092\\
1370	-197.754\\
1371	-140.381\\
1372	-106.201\\
1373	-106.201\\
1374	-104.98\\
1375	-120.85\\
1376	-150.146\\
1377	-147.705\\
1378	-200.195\\
1379	-220.947\\
1380	-263.672\\
1381	-178.223\\
1382	-190.43\\
1383	-212.402\\
1384	-163.574\\
1385	-189.209\\
1386	-163.574\\
1387	-91.5530000000001\\
1388	-58.5940000000001\\
1389	-62.2560000000001\\
1390	-95.2149999999999\\
1391	-75.684\\
1392	-51.27\\
1393	-72.021\\
1394	-52.49\\
1395	-40.2829999999999\\
1396	-48.828\\
1397	-56.152\\
1398	-40.2829999999999\\
1400	-109.863\\
1401	-78.125\\
1403	-195.313\\
1404	-178.223\\
1405	-222.168\\
};
\addlegendentry{True output}

\addplot [color=mycolor2, dashed, line width=2.0pt]
  table[row sep=crcr]{%
1006	-146.955168029269\\
1007	-166.895354589941\\
1008	-129.710872769074\\
1009	-68.6168586921294\\
1010	-81.8609510509511\\
1011	-84.9407264605738\\
1012	-101.996457766837\\
1013	-84.7751445265026\\
1014	-25.9893787035569\\
1015	-29.2646169576824\\
1016	-25.3057688251185\\
1017	-77.2750626110778\\
1018	-119.754432843242\\
1019	-124.840842669245\\
1020	-144.124888860455\\
1021	-87.4927763298506\\
1022	-60.2668112531605\\
1023	-131.381560388364\\
1024	-86.180920575265\\
1025	-65.540133976001\\
1026	-110.700652709088\\
1028	-90.8049661333985\\
1029	-134.849030185633\\
1030	-135.614641563021\\
1031	-97.5274104269338\\
1032	-145.776891973042\\
1033	-142.063955049464\\
1034	-114.66262583584\\
1035	-95.8045765129066\\
1036	-71.8191200501615\\
1037	-87.2338545585637\\
1038	-105.923777231205\\
1039	-96.177942762225\\
1040	-107.271818737302\\
1041	-108.735129646577\\
1042	-118.47081167193\\
1043	-169.673956239786\\
1044	-150.589707090588\\
1045	-108.957568781545\\
1046	-95.8608162247524\\
1047	-49.3363469719707\\
1048	-56.1316180332456\\
1049	-75.922041374321\\
1050	-66.215852076064\\
1051	-59.7138043332793\\
1052	-89.3499820860086\\
1053	-100.531669085553\\
1054	-94.8287071878003\\
1055	-151.108065464178\\
1056	-111.663975420561\\
1057	-65.6855583463107\\
1058	-54.6133481364145\\
1059	-68.3042478589045\\
1060	-84.9550771111483\\
1061	-43.2600821431486\\
1062	-42.9351301050372\\
1063	-65.2301405839482\\
1065	-44.1789382193458\\
1066	-57.4716509399843\\
1067	-67.7768911198268\\
1068	-112.806080167753\\
1069	-103.307606584719\\
1070	-126.765271497552\\
1071	-116.653251419995\\
1072	-133.391855369351\\
1073	-112.040695476922\\
1074	-103.375968899755\\
1075	-98.8216901068636\\
1076	-91.3714879119586\\
1077	-87.2415073405357\\
1079	-219.560665950438\\
1080	-226.316613827266\\
1081	-210.551227802517\\
1082	-136.223261777004\\
1083	-202.582346938719\\
1084	-244.540489992504\\
1085	-244.121024330946\\
1086	-200.504945902605\\
1087	-256.830632014538\\
1088	-331.653081908424\\
1089	-237.932131099399\\
1090	-204.051603331821\\
1091	-122.133581746307\\
1092	-97.6478567980171\\
1093	-79.5516758344759\\
1094	-106.20202414745\\
1096	-43.8756049215365\\
1097	-39.7669588486492\\
1098	-57.3626056081621\\
1099	-93.8554588769762\\
1100	-93.0080943456881\\
1101	-88.1099854527622\\
1102	-121.974980370702\\
1103	-117.780351078145\\
1104	-181.56102374565\\
1105	-134.855716957794\\
1106	-169.954794732337\\
1107	-123.637555220838\\
1108	-113.917522295949\\
1109	-124.776187985356\\
1110	-95.4788549561342\\
1111	-84.0027055368048\\
1112	-96.8874951712432\\
1113	-100.730673789136\\
1114	-69.6140443401559\\
1115	-75.0254403326921\\
1116	-56.3140025620166\\
1118	-66.7024808631086\\
1119	-50.7873307169298\\
1120	-55.4247781733829\\
1121	-75.1486759961927\\
1122	-122.530295637238\\
1123	-119.218157825645\\
1124	-132.817318152912\\
1125	-74.0536436614987\\
1126	-118.40571947343\\
1127	-175.320574755302\\
1128	-134.176610500488\\
1129	-162.105426460861\\
1130	-148.069886465967\\
1131	-87.3578434595352\\
1132	-67.274564103664\\
1133	-78.1709971392252\\
1134	-101.565229459047\\
1135	-164.437463617798\\
1136	-202.092037406038\\
1137	-191.505625170563\\
1138	-144.906601550175\\
1139	-148.746252300442\\
1140	-138.931641923268\\
1141	-131.620899543403\\
1142	-137.89089719457\\
1143	-85.0280269266241\\
1144	-72.6453686510215\\
1145	-70.1776942743581\\
1146	-63.4890737988978\\
1147	-74.772742770504\\
1148	-105.603704324476\\
1149	-169.58881524923\\
1151	-89.7222082628728\\
1152	-74.9541329092481\\
1153	-71.9690538481432\\
1154	-41.9196728352222\\
1155	-27.3735672292746\\
1156	-34.9399954949292\\
1157	-77.6639809196333\\
1158	-55.3364711945114\\
1159	-60.5795126841033\\
1160	-69.8194778281622\\
1161	-63.3306334119425\\
1162	-45.9848151395011\\
1163	-33.3990861654595\\
1164	-24.6232784642898\\
1165	-40.1324153938008\\
1166	-101.760195055519\\
1167	-133.377135584582\\
1168	-158.868500497121\\
1169	-106.490253462923\\
1170	-69.6543531403813\\
1172	-33.9251712172843\\
1173	-83.8542307419336\\
1174	-82.6484184177993\\
1175	-118.43478808139\\
1176	-145.057000961991\\
1177	-239.281952847047\\
1178	-279.219276747907\\
1179	-220.250446075673\\
1180	-196.926360229417\\
1181	-132.994548338439\\
1182	-156.57703191114\\
1183	-159.008825109268\\
1184	-172.026568957833\\
1185	-131.727051556375\\
1186	-123.566623285041\\
1187	-114.425675151595\\
1188	-110.692501773604\\
1189	-120.897295257538\\
1190	-91.2321718443077\\
1191	-147.048931291545\\
1192	-189.797288674102\\
1193	-129.482850050119\\
1194	-83.186543035627\\
1195	-76.3273010017217\\
1196	-158.709907435124\\
1197	-176.096135812153\\
1198	-227.758263778255\\
1199	-228.974895788914\\
1200	-236.848375505436\\
1201	-185.787208966803\\
1202	-165.564395689653\\
1203	-197.588426127992\\
1204	-204.184918178192\\
1205	-127.080871144749\\
1206	-90.1981679732974\\
1207	-116.933954791607\\
1208	-112.530966272017\\
1209	-62.2639012261745\\
1210	-86.2407458560067\\
1211	-99.4473559249727\\
1213	-63.5747194005164\\
1214	-80.8280542079679\\
1215	-69.9833367305646\\
1216	-96.1185917246867\\
1217	-136.362932789559\\
1218	-129.726718358242\\
1219	-75.6455805949122\\
1220	-90.6816245267851\\
1221	-122.216751444261\\
1222	-224.764454794659\\
1224	-93.0577778929028\\
1225	-68.4443537785405\\
1226	-80.4063698323355\\
1227	-79.6283745229816\\
1228	-60.0123013494033\\
1229	-57.6270736027216\\
1231	-103.30009856597\\
1232	-102.30282997286\\
1233	-74.0869449464928\\
1234	-136.997239239132\\
1235	-145.181694849334\\
1236	-130.485562985132\\
1237	-110.318086227489\\
1238	-124.227653179938\\
1239	-156.897826596184\\
1241	-38.4994914685201\\
1242	-59.1735138813842\\
1243	-71.5130105662554\\
1244	-99.1005054749594\\
1245	-84.3247225097903\\
1246	-62.7938533746828\\
1247	-101.089757452922\\
1248	-89.7351825546054\\
1249	-69.1137890197651\\
1250	-85.3619746997131\\
1251	-79.4261498160988\\
1252	-70.8214760703809\\
1253	-81.8555726927893\\
1254	-47.9590463315374\\
1255	-53.7669110521831\\
1256	-44.6012039916868\\
1257	-46.7360824901066\\
1258	-94.502259314751\\
1259	-97.2660346729733\\
1260	-142.985936466051\\
1261	-181.703388601958\\
1263	-61.2573179307678\\
1264	-56.2146909860512\\
1265	-54.2517865313889\\
1266	-77.5952070950218\\
1267	-43.9262226219946\\
1268	-53.1545316297545\\
1269	-69.639267953052\\
1270	-129.032519578669\\
1271	-109.549817740588\\
1272	-148.633376546009\\
1273	-103.312702161353\\
1274	-91.6980241610011\\
1275	-45.0648638995619\\
1276	-49.0947194442717\\
1277	-29.6562056116361\\
1278	-36.0567198093738\\
1279	-44.3638217692501\\
1280	-80.1826023413382\\
1281	-76.8893594321012\\
1282	-97.1710369080088\\
1283	-99.0289520025667\\
1284	-153.77063441483\\
1285	-139.404691511036\\
1286	-104.153999423034\\
1287	-118.733429542844\\
1288	-120.570938100865\\
1289	-160.313803236661\\
1290	-136.531247781147\\
1292	-40.746263463732\\
1293	-38.132785745664\\
1294	-34.0015383040359\\
1295	-40.4306831530519\\
1296	-43.9458645656919\\
1297	-41.7251193484894\\
1298	-55.3826526502617\\
1299	-90.8803736887655\\
1300	-80.6143549807273\\
1301	-80.4275744905588\\
1302	-98.283984639477\\
1303	-54.8268756399095\\
1304	-32.27816558105\\
1305	-59.9644529055086\\
1306	-97.3084731101899\\
1307	-93.3661528486762\\
1308	-109.142993228648\\
1309	-93.6948525591613\\
1310	-68.7363289221598\\
1311	-97.4881057513587\\
1312	-136.181950674456\\
1313	-137.245850358488\\
1314	-94.2019060261505\\
1315	-162.636709965971\\
1316	-129.552155711737\\
1317	-122.865925231816\\
1318	-134.023312875237\\
1319	-127.4939134022\\
1321	-91.8284325846935\\
1322	-155.427194237411\\
1323	-206.073916537526\\
1324	-165.594015136128\\
1325	-92.0667417058537\\
1326	-90.7517909438659\\
1327	-95.164611451982\\
1328	-117.881472461255\\
1329	-134.56843374381\\
1330	-99.2299385923432\\
1331	-108.44092854911\\
1332	-136.783646607342\\
1333	-143.175034648781\\
1334	-102.131902080793\\
1335	-82.3344585858426\\
1336	-114.315714731704\\
1337	-174.0569427446\\
1338	-162.692861200529\\
1339	-155.603725671675\\
1340	-88.2367102293222\\
1341	-84.7485924522462\\
1342	-93.6625416906465\\
1343	-46.2342800091988\\
1344	-30.5773973356363\\
1345	-24.3052360709182\\
1346	-23.6300308564187\\
1347	-60.1930791566119\\
1348	-91.9540679077522\\
1349	-99.9847349793511\\
1350	-74.5863583764394\\
1351	-88.5120493345219\\
1352	-98.5175986240881\\
1353	-61.5000713335467\\
1354	-60.5878216385995\\
1355	-58.5880773291785\\
1356	-48.5651973157335\\
1357	-59.9542379648112\\
1358	-95.3151900727539\\
1359	-122.021779487206\\
1360	-80.3411957558274\\
1361	-55.5018414347696\\
1362	-55.4948105965677\\
1363	-57.5793954253766\\
1364	-38.9686214021278\\
1365	-95.4097300413659\\
1366	-162.307935721323\\
1367	-122.224036196152\\
1368	-166.323717077296\\
1369	-189.549892775779\\
1370	-200.660810265345\\
1371	-137.194857230308\\
1372	-111.069431878924\\
1373	-109.322927036247\\
1374	-110.395926188573\\
1375	-120.545427312857\\
1376	-137.616626436727\\
1377	-144.560493406965\\
1378	-188.494934252637\\
1379	-216.02084100008\\
1380	-262.638678049623\\
1381	-186.381997255498\\
1382	-186.716058473452\\
1383	-214.671870982853\\
1384	-158.806856227022\\
1385	-192.441356197629\\
1386	-160.862754182041\\
1387	-85.4331991660247\\
1388	-55.0910517447267\\
1389	-56.9011786986785\\
1390	-94.8841550380587\\
1391	-78.7664785818324\\
1392	-51.3191876133953\\
1393	-67.4704294831952\\
1394	-60.0556808370814\\
1395	-35.4388329570102\\
1396	-50.2937679067848\\
1397	-57.2269087341849\\
1398	-41.416059149517\\
1399	-85.0411328963671\\
1400	-105.554125517566\\
1401	-81.6893213083206\\
1403	-201.656382293216\\
1404	-184.175914727061\\
1405	-213.354918575294\\
};
\addlegendentry{OSA predition}

\addplot [color=mycolor3, dotted, line width=2.0pt]
  table[row sep=crcr]{%
1006	-147.705\\
1007	-175.781\\
1008	-134.277\\
1009	-68.3589999999999\\
1010	-81.86095105096\\
1011	-85.2137402545661\\
1012	-103.437469082222\\
1013	-87.1400496922327\\
1014	-28.1294081441858\\
1015	-30.1088796697099\\
1016	-24.7345671943613\\
1017	-79.2134489602472\\
1018	-122.513208037514\\
1019	-126.028971594574\\
1020	-141.467111731313\\
1021	-86.0489316839448\\
1022	-59.9582120913501\\
1023	-129.718781344367\\
1024	-88.349842316265\\
1025	-66.8232892661727\\
1026	-111.491823203439\\
1027	-99.3589768529746\\
1028	-89.5915636452621\\
1029	-135.0613967266\\
1030	-135.400733601153\\
1031	-95.1631044362075\\
1032	-144.683228983265\\
1033	-138.087540812665\\
1034	-111.429886404727\\
1035	-92.295459707846\\
1036	-70.5952976280396\\
1037	-86.2919920776997\\
1038	-105.191799340058\\
1039	-94.5376747416522\\
1040	-102.780777617254\\
1041	-103.653774852349\\
1042	-113.33756341606\\
1043	-163.561954339057\\
1044	-144.529293484735\\
1045	-102.288078718241\\
1046	-91.2500417688352\\
1047	-48.3484299315\\
1048	-53.8538104986526\\
1049	-72.8753874826609\\
1050	-62.9943638837524\\
1051	-57.8370182688448\\
1052	-88.7612437834148\\
1053	-98.8339001716545\\
1054	-93.9449759293286\\
1055	-149.286574725565\\
1056	-110.718305432252\\
1057	-64.7375860350892\\
1058	-54.4026932212273\\
1059	-68.8763431875811\\
1060	-85.8778552468518\\
1061	-43.8108523038654\\
1062	-44.4617845879893\\
1063	-65.2218585491412\\
1064	-53.982859027094\\
1065	-44.1265896365358\\
1066	-58.5620263433134\\
1067	-67.7627462018982\\
1068	-111.662855437388\\
1069	-102.578205518644\\
1070	-126.505483337226\\
1071	-115.181181604692\\
1072	-131.19802709754\\
1073	-108.519289950922\\
1074	-100.409527409403\\
1075	-96.3842494531241\\
1076	-89.2027031898074\\
1077	-87.7400294403321\\
1079	-218.658230967668\\
1080	-224.188021989118\\
1081	-207.17607439697\\
1082	-130.561647048115\\
1083	-192.558246111025\\
1084	-236.289145702737\\
1085	-235.694472823726\\
1086	-190.514696785921\\
1087	-243.18998475006\\
1088	-320.266535061585\\
1089	-224.959867364836\\
1090	-195.969905488275\\
1091	-117.00884964438\\
1093	-74.0890649877117\\
1094	-102.754930552992\\
1095	-69.9233607010099\\
1096	-42.1260439900236\\
1097	-35.791250155065\\
1098	-51.3374975028596\\
1099	-85.7793971226952\\
1100	-85.7761935989795\\
1101	-83.6045863618172\\
1102	-119.067359625594\\
1103	-115.173624274374\\
1104	-178.988895070494\\
1105	-133.620291313636\\
1106	-173.461938690879\\
1107	-123.832759677363\\
1108	-116.282723520334\\
1109	-127.13269672654\\
1110	-98.8260118052428\\
1111	-86.725189959403\\
1112	-102.244099229474\\
1113	-104.432825451707\\
1114	-72.8965521045984\\
1115	-78.4534811329936\\
1116	-59.3889665238303\\
1118	-69.5082303938927\\
1119	-53.3974922194386\\
1120	-59.2218618720653\\
1121	-78.6193766842471\\
1122	-125.419492609556\\
1123	-123.554066402327\\
1124	-136.422078005585\\
1125	-74.6925028437222\\
1126	-118.349200179197\\
1127	-176.246328265539\\
1128	-137.600923308487\\
1129	-164.710397269828\\
1130	-152.101824881288\\
1131	-89.069650242127\\
1132	-66.293375215088\\
1133	-81.0976505831056\\
1134	-102.950068259901\\
1135	-164.85207485291\\
1136	-203.873113484681\\
1137	-193.06705533045\\
1138	-144.66144183832\\
1139	-146.599842382714\\
1140	-137.885920660092\\
1141	-130.398889386009\\
1142	-138.311049884531\\
1143	-86.4622403555088\\
1144	-77.2225364000965\\
1146	-64.4939016318583\\
1147	-75.7776082031901\\
1148	-107.576707418931\\
1149	-171.776601293002\\
1150	-129.927837628146\\
1151	-93.2994195403587\\
1152	-78.0118008953507\\
1153	-75.9244537373133\\
1154	-44.9373343589132\\
1155	-30.0104833345715\\
1156	-35.7238767015479\\
1157	-76.8012728406429\\
1158	-55.3346427458332\\
1159	-61.5089021057754\\
1160	-69.5639234630323\\
1161	-64.6108972203281\\
1162	-46.939785123614\\
1163	-34.5280817677706\\
1164	-26.9240247645616\\
1165	-42.3409238568552\\
1166	-102.8549927294\\
1167	-134.724665432042\\
1168	-159.691577190463\\
1169	-105.246409475319\\
1170	-72.5829471675738\\
1172	-36.2120724911283\\
1173	-86.2422034690755\\
1174	-84.4621858204123\\
1175	-120.496932553448\\
1176	-146.618705628769\\
1177	-240.587040544677\\
1178	-281.248641532224\\
1179	-225.700577447631\\
1180	-206.495221856331\\
1181	-139.854476131303\\
1182	-160.256254104003\\
1183	-162.814796060702\\
1184	-176.332937155466\\
1185	-133.830007719581\\
1186	-126.565998554574\\
1187	-118.029866893095\\
1188	-114.570965276635\\
1189	-124.48098122709\\
1190	-95.9669664343864\\
1191	-150.196521548411\\
1192	-192.859741387361\\
1193	-128.713504398891\\
1194	-84.2409123068246\\
1195	-75.8226540136575\\
1196	-158.728433400262\\
1197	-175.235807737082\\
1198	-230.037129112411\\
1199	-227.103735248996\\
1200	-235.400114236572\\
1201	-179.081222376103\\
1202	-163.047841788494\\
1203	-195.753151891316\\
1204	-204.833645982712\\
1205	-127.545286087086\\
1206	-80.9882246582602\\
1207	-111.76050000074\\
1208	-108.783797406584\\
1209	-59.1286328590343\\
1210	-86.6129624321773\\
1211	-96.4499561866032\\
1212	-81.208242449329\\
1213	-63.2469139128741\\
1214	-83.215154351135\\
1215	-72.2045319368269\\
1216	-98.9178128757251\\
1217	-140.648431479368\\
1218	-134.275231444066\\
1219	-80.841909151437\\
1220	-96.1289781160087\\
1221	-126.81333064439\\
1222	-230.388567014241\\
1224	-103.759931770636\\
1225	-76.2734752428614\\
1226	-87.6164179413856\\
1227	-84.187823093579\\
1228	-64.3438247497099\\
1229	-63.3318133848452\\
1231	-106.905351411747\\
1232	-108.351706637626\\
1233	-78.0103905949024\\
1234	-140.172356737064\\
1235	-151.454799813873\\
1236	-138.513325901464\\
1237	-113.728219981712\\
1238	-127.250479283837\\
1239	-161.862143156791\\
1241	-39.3614808663444\\
1242	-58.9034655004421\\
1243	-70.916759991052\\
1244	-100.04832413857\\
1245	-87.6783399763067\\
1246	-66.9745739569546\\
1247	-105.612257341519\\
1248	-94.8439356982126\\
1249	-73.274326161514\\
1250	-88.9129775597505\\
1251	-83.8751904718699\\
1252	-74.6665521848267\\
1253	-86.5747207160041\\
1254	-52.3944951025298\\
1255	-58.10669213029\\
1256	-48.2047880919044\\
1257	-50.4919621630156\\
1258	-99.4042692591233\\
1259	-102.9887640789\\
1260	-149.077948196231\\
1261	-186.930468541756\\
1263	-66.284023730577\\
1264	-59.9605144603618\\
1265	-56.6737374199147\\
1266	-83.565458852129\\
1267	-48.641415780744\\
1268	-57.9105806814221\\
1269	-72.6043518376882\\
1270	-132.113149762788\\
1271	-113.021249383139\\
1272	-154.477597042052\\
1273	-106.360820141871\\
1274	-94.7086490718293\\
1275	-47.3666966802962\\
1276	-50.468666020801\\
1277	-30.4314554015641\\
1278	-36.9056270956876\\
1279	-44.6387172231639\\
1280	-81.7595532681221\\
1281	-79.2973572397134\\
1282	-98.7444075515812\\
1283	-99.5381084189958\\
1284	-156.019347797325\\
1285	-140.884549733413\\
1286	-107.615053939452\\
1287	-124.770814886906\\
1288	-125.943850671874\\
1289	-162.949283074177\\
1290	-135.238922201668\\
1292	-42.5466062983305\\
1293	-39.5560382157348\\
1294	-35.4658539334389\\
1295	-43.5967435912119\\
1296	-45.0417142130752\\
1297	-43.0820315242322\\
1298	-56.5338596594668\\
1299	-92.1055734988877\\
1300	-81.8213945223254\\
1301	-82.4895069867405\\
1302	-99.5826800266666\\
1303	-55.8978346057211\\
1304	-32.8808002278697\\
1305	-60.0074012033488\\
1306	-97.3810855576355\\
1307	-91.8962603616224\\
1308	-107.963676516113\\
1309	-91.4976241767542\\
1310	-67.5577448203919\\
1311	-95.8242241737305\\
1312	-134.725648555742\\
1313	-135.724983191509\\
1314	-92.3476515811703\\
1315	-159.875573264505\\
1316	-126.86601764121\\
1317	-121.439266926487\\
1318	-133.71648850568\\
1319	-127.620818477297\\
1320	-106.095290371249\\
1321	-88.6091274724454\\
1322	-154.653048817551\\
1323	-205.404184715016\\
1324	-164.59504011704\\
1325	-88.4067923957234\\
1326	-89.6413635362564\\
1327	-92.3436041612092\\
1328	-116.616391099253\\
1329	-134.99941027929\\
1330	-99.8575030026918\\
1331	-107.914414708265\\
1332	-137.321657309394\\
1333	-143.561220201303\\
1334	-99.1068134440322\\
1335	-77.7452515801765\\
1336	-111.462224357496\\
1337	-173.64707676112\\
1339	-155.762622885996\\
1340	-89.0905113164347\\
1341	-80.8966632906868\\
1342	-90.5958084828555\\
1343	-46.3303826113304\\
1344	-31.6589199464856\\
1345	-21.4781817624657\\
1346	-21.5891637055599\\
1348	-89.7703986298695\\
1349	-99.2914635674385\\
1350	-74.3361591141986\\
1352	-97.050757035172\\
1353	-60.4543046806336\\
1354	-60.3531553215344\\
1355	-58.0219663598591\\
1356	-46.8227868501733\\
1357	-59.1994704507026\\
1358	-95.798640549938\\
1359	-122.284794581697\\
1360	-79.3567283952821\\
1361	-55.1260543133662\\
1362	-55.5601727909266\\
1363	-59.1434667853728\\
1364	-41.9843739573573\\
1365	-96.2254510917362\\
1366	-163.77502489012\\
1367	-124.639171815788\\
1368	-169.052379733309\\
1369	-191.573041806387\\
1370	-201.398408221955\\
1371	-136.941201731048\\
1372	-111.898937154666\\
1373	-109.135226588547\\
1374	-113.186601451983\\
1375	-123.402786465649\\
1376	-141.618637183191\\
1377	-144.542031732932\\
1378	-185.452129046839\\
1379	-211.773539725567\\
1380	-254.997963036379\\
1381	-181.675374863018\\
1382	-182.886533306616\\
1383	-213.686858973466\\
1384	-155.603731116195\\
1385	-191.252035276851\\
1386	-157.688176197777\\
1387	-85.535510450454\\
1388	-51.1070966205675\\
1389	-52.8780180646872\\
1390	-89.3869016592039\\
1391	-73.7156525811224\\
1392	-48.6118052606732\\
1393	-65.4630654720615\\
1394	-57.3513064699671\\
1395	-34.1655643118734\\
1396	-50.2457014673821\\
1397	-55.2999691778462\\
1398	-41.919467412572\\
1399	-84.9278249094557\\
1400	-107.777946296915\\
1401	-83.8988735740447\\
1403	-205.694894553032\\
1404	-189.70327526838\\
1405	-220.251963668666\\
};
\addlegendentry{MPO prediction}

\end{axis}

\begin{axis}[%
width=6.159cm,
height=1.831cm,
at={(8.104cm,2.542cm)},
scale only axis,
xmin=1000,
xmax=1405,
xlabel style={font=\color{white!15!black}},
xlabel={Sample index},
ymin=-279.541,
ymax=0,
ylabel style={font=\color{white!15!black}},
ylabel={$y(t)$},
axis background/.style={fill=white},
title style={font=\bfseries},
title={C8: RMSE(OSA) = 4.1858, RMSE(MPO) = 5.2038},
legend style={legend cell align=left, align=left, draw=white!15!black}
]
\addplot [color=mycolor1, line width=2.0pt]
  table[row sep=crcr]{%
1006	-125.732\\
1007	-153.809\\
1008	-115.967\\
1009	-61.0350000000001\\
1010	-74.463\\
1011	-73.242\\
1012	-86.6700000000001\\
1013	-73.242\\
1014	-34.1800000000001\\
1015	-20.752\\
1016	-23.193\\
1017	-58.5940000000001\\
1018	-100.098\\
1019	-109.863\\
1020	-114.746\\
1022	-52.49\\
1023	-104.98\\
1025	-59.8140000000001\\
1026	-97.6559999999999\\
1027	-83.008\\
1028	-72.021\\
1029	-118.408\\
1030	-107.422\\
1031	-83.008\\
1032	-119.629\\
1033	-123.291\\
1034	-95.2149999999999\\
1035	-80.566\\
1036	-62.2560000000001\\
1037	-75.684\\
1038	-95.2149999999999\\
1039	-87.8910000000001\\
1040	-91.5530000000001\\
1041	-96.4359999999999\\
1042	-102.539\\
1043	-141.602\\
1044	-128.174\\
1045	-85.4490000000001\\
1046	-79.346\\
1047	-47.607\\
1048	-48.828\\
1049	-68.3589999999999\\
1050	-52.49\\
1051	-57.373\\
1052	-75.684\\
1053	-89.1110000000001\\
1054	-81.787\\
1055	-128.174\\
1056	-98.877\\
1057	-57.373\\
1058	-45.1659999999999\\
1059	-62.2560000000001\\
1060	-72.021\\
1061	-40.2829999999999\\
1062	-42.7249999999999\\
1063	-56.152\\
1064	-46.3869999999999\\
1065	-40.2829999999999\\
1066	-52.49\\
1067	-62.2560000000001\\
1068	-92.7729999999999\\
1069	-90.3320000000001\\
1070	-107.422\\
1071	-98.877\\
1072	-115.967\\
1073	-91.5530000000001\\
1074	-91.5530000000001\\
1075	-79.346\\
1076	-75.684\\
1077	-76.904\\
1079	-189.209\\
1080	-194.092\\
1081	-189.209\\
1082	-119.629\\
1083	-170.898\\
1084	-213.623\\
1085	-219.727\\
1086	-169.678\\
1087	-219.727\\
1088	-279.541\\
1089	-197.754\\
1090	-161.133\\
1091	-109.863\\
1092	-85.4490000000001\\
1093	-73.242\\
1094	-91.5530000000001\\
1095	-62.2560000000001\\
1096	-46.3869999999999\\
1097	-42.7249999999999\\
1098	-54.932\\
1099	-79.346\\
1100	-74.463\\
1101	-75.684\\
1102	-100.098\\
1103	-103.76\\
1104	-141.602\\
1105	-114.746\\
1106	-142.822\\
1107	-104.98\\
1108	-90.3320000000001\\
1109	-103.76\\
1110	-76.904\\
1111	-69.5799999999999\\
1112	-84.229\\
1113	-86.6700000000001\\
1114	-59.8140000000001\\
1115	-64.6970000000001\\
1116	-48.828\\
1117	-53.711\\
1118	-57.373\\
1119	-41.5039999999999\\
1120	-52.49\\
1121	-65.9180000000001\\
1122	-101.318\\
1123	-104.98\\
1124	-114.746\\
1125	-68.3589999999999\\
1126	-93.9939999999999\\
1127	-150.146\\
1128	-114.746\\
1129	-125.732\\
1130	-128.174\\
1131	-80.566\\
1132	-53.711\\
1133	-75.684\\
1134	-85.4490000000001\\
1135	-137.939\\
1136	-172.119\\
1137	-170.898\\
1138	-123.291\\
1139	-130.615\\
1140	-115.967\\
1142	-111.084\\
1143	-76.904\\
1144	-68.3589999999999\\
1145	-61.0350000000001\\
1146	-57.373\\
1147	-62.2560000000001\\
1148	-91.5530000000001\\
1149	-140.381\\
1151	-79.346\\
1152	-64.6970000000001\\
1153	-62.2560000000001\\
1154	-40.2829999999999\\
1155	-29.297\\
1156	-36.6210000000001\\
1157	-63.4770000000001\\
1158	-51.27\\
1159	-52.49\\
1160	-58.5940000000001\\
1161	-54.932\\
1162	-37.8420000000001\\
1163	-28.076\\
1164	-23.193\\
1165	-39.0630000000001\\
1166	-83.008\\
1167	-117.188\\
1168	-130.615\\
1169	-86.6700000000001\\
1170	-58.5940000000001\\
1172	-32.9590000000001\\
1173	-67.1389999999999\\
1174	-65.9180000000001\\
1176	-117.188\\
1177	-186.768\\
1178	-216.064\\
1179	-177.002\\
1180	-168.457\\
1181	-109.863\\
1182	-122.07\\
1183	-131.836\\
1184	-146.484\\
1185	-108.643\\
1186	-100.098\\
1187	-98.877\\
1188	-89.1110000000001\\
1189	-106.201\\
1190	-78.125\\
1191	-130.615\\
1192	-159.912\\
1194	-70.8009999999999\\
1195	-73.242\\
1196	-123.291\\
1197	-153.809\\
1198	-189.209\\
1199	-202.637\\
1200	-202.637\\
1201	-153.809\\
1202	-137.939\\
1203	-158.691\\
1204	-187.988\\
1205	-109.863\\
1206	-72.021\\
1207	-101.318\\
1208	-86.6700000000001\\
1209	-53.711\\
1210	-74.463\\
1211	-84.229\\
1213	-51.27\\
1214	-70.8009999999999\\
1215	-58.5940000000001\\
1216	-79.346\\
1217	-107.422\\
1218	-100.098\\
1219	-69.5799999999999\\
1220	-70.8009999999999\\
1221	-107.422\\
1222	-177.002\\
1224	-79.346\\
1225	-61.0350000000001\\
1226	-70.8009999999999\\
1227	-64.6970000000001\\
1228	-48.828\\
1229	-53.711\\
1230	-65.9180000000001\\
1231	-83.008\\
1232	-91.5530000000001\\
1233	-63.4770000000001\\
1234	-104.98\\
1235	-124.512\\
1236	-118.408\\
1237	-91.5530000000001\\
1238	-100.098\\
1239	-135.498\\
1241	-42.7249999999999\\
1242	-57.373\\
1243	-62.2560000000001\\
1244	-81.787\\
1245	-72.021\\
1246	-52.49\\
1247	-79.346\\
1248	-80.566\\
1249	-57.373\\
1250	-70.8009999999999\\
1252	-56.152\\
1253	-67.1389999999999\\
1254	-41.5039999999999\\
1255	-48.828\\
1256	-36.6210000000001\\
1257	-40.2829999999999\\
1258	-75.684\\
1259	-84.229\\
1260	-113.525\\
1261	-146.484\\
1262	-98.877\\
1263	-58.5940000000001\\
1264	-46.3869999999999\\
1265	-43.9449999999999\\
1266	-63.4770000000001\\
1267	-42.7249999999999\\
1268	-47.607\\
1269	-61.0350000000001\\
1270	-102.539\\
1271	-93.9939999999999\\
1272	-134.277\\
1273	-89.1110000000001\\
1274	-81.787\\
1275	-46.3869999999999\\
1276	-45.1659999999999\\
1277	-28.076\\
1278	-35.4000000000001\\
1279	-37.8420000000001\\
1280	-67.1389999999999\\
1281	-70.8009999999999\\
1282	-79.346\\
1283	-85.4490000000001\\
1284	-125.732\\
1285	-112.305\\
1286	-84.229\\
1287	-102.539\\
1288	-109.863\\
1289	-144.043\\
1290	-115.967\\
1291	-75.684\\
1292	-40.2829999999999\\
1293	-29.297\\
1294	-29.297\\
1295	-36.6210000000001\\
1296	-39.0630000000001\\
1297	-37.8420000000001\\
1298	-50.049\\
1299	-74.463\\
1300	-68.3589999999999\\
1301	-70.8009999999999\\
1302	-83.008\\
1303	-51.27\\
1304	-30.518\\
1305	-58.5940000000001\\
1306	-90.3320000000001\\
1307	-87.8910000000001\\
1308	-97.6559999999999\\
1309	-83.008\\
1310	-61.0350000000001\\
1311	-85.4490000000001\\
1312	-118.408\\
1313	-118.408\\
1314	-84.229\\
1315	-136.719\\
1316	-100.098\\
1317	-101.318\\
1318	-117.188\\
1319	-115.967\\
1320	-86.6700000000001\\
1321	-76.904\\
1323	-178.223\\
1324	-139.16\\
1325	-79.346\\
1326	-76.904\\
1327	-79.346\\
1328	-98.877\\
1329	-114.746\\
1330	-85.4490000000001\\
1331	-90.3320000000001\\
1332	-120.85\\
1333	-131.836\\
1334	-87.8910000000001\\
1335	-68.3589999999999\\
1336	-91.5530000000001\\
1337	-156.25\\
1338	-137.939\\
1339	-140.381\\
1340	-84.229\\
1341	-76.904\\
1342	-72.021\\
1343	-47.607\\
1344	-30.518\\
1346	-23.193\\
1347	-54.932\\
1349	-87.8910000000001\\
1350	-65.9180000000001\\
1351	-73.242\\
1352	-83.008\\
1353	-53.711\\
1354	-56.152\\
1355	-53.711\\
1356	-40.2829999999999\\
1357	-51.27\\
1358	-84.229\\
1359	-106.201\\
1360	-63.4770000000001\\
1361	-48.828\\
1362	-40.2829999999999\\
1363	-50.049\\
1364	-35.4000000000001\\
1365	-76.904\\
1366	-130.615\\
1367	-104.98\\
1368	-139.16\\
1369	-162.354\\
1370	-164.795\\
1371	-124.512\\
1372	-90.3320000000001\\
1373	-89.1110000000001\\
1374	-89.1110000000001\\
1375	-106.201\\
1376	-125.732\\
1377	-125.732\\
1378	-168.457\\
1379	-183.105\\
1380	-219.727\\
1381	-147.705\\
1383	-184.326\\
1384	-140.381\\
1385	-162.354\\
1386	-139.16\\
1387	-80.566\\
1388	-52.49\\
1389	-56.152\\
1390	-81.787\\
1391	-65.9180000000001\\
1392	-43.9449999999999\\
1393	-62.2560000000001\\
1394	-47.607\\
1395	-36.6210000000001\\
1396	-46.3869999999999\\
1397	-52.49\\
1398	-36.6210000000001\\
1399	-70.8009999999999\\
1400	-92.7729999999999\\
1401	-68.3589999999999\\
1403	-164.795\\
1404	-150.146\\
1405	-196.533\\
};
\addlegendentry{True output}

\addplot [color=mycolor2, dashed, line width=2.0pt]
  table[row sep=crcr]{%
1006	-115.93992834947\\
1007	-148.823399870046\\
1008	-118.538982872481\\
1009	-58.037148910912\\
1010	-72.3788963339402\\
1011	-72.6251268487251\\
1012	-92.8187592071399\\
1013	-72.1916890384689\\
1014	-25.7625632319164\\
1015	-27.1600003348044\\
1016	-23.2944172131229\\
1017	-62.032906346079\\
1018	-95.2345530002549\\
1019	-105.353854578872\\
1020	-109.226666818455\\
1021	-83.6360375832242\\
1022	-50.4507996596958\\
1023	-111.511290702861\\
1024	-77.1479846338204\\
1025	-59.7404954418926\\
1026	-102.741337172466\\
1028	-71.7749609790185\\
1029	-115.517957629535\\
1030	-104.772912279705\\
1031	-83.7607699649436\\
1032	-116.019044406542\\
1033	-114.883222450035\\
1036	-65.259547806892\\
1037	-77.0431445441541\\
1038	-87.3039622257377\\
1039	-88.8960474134115\\
1040	-89.290433429223\\
1041	-92.2248139123938\\
1042	-96.1328160217083\\
1043	-135.916565351667\\
1044	-127.143260340508\\
1045	-90.9911961460568\\
1046	-80.6628842605992\\
1047	-45.0346614361306\\
1048	-48.0750649443883\\
1049	-67.6793365348919\\
1050	-54.4738843557482\\
1051	-54.7180010195525\\
1052	-77.649509631186\\
1053	-86.3914268856222\\
1054	-81.3236476865195\\
1055	-127.806380012888\\
1056	-95.1331742882039\\
1057	-58.9489718789928\\
1058	-49.508099057032\\
1059	-60.9302321264538\\
1060	-73.7199242039349\\
1061	-38.1585372230052\\
1062	-41.3249751153337\\
1063	-58.6307770780888\\
1065	-39.5601877417462\\
1066	-54.0546543825249\\
1067	-57.4104558234783\\
1068	-98.2183421987779\\
1069	-85.0603887827331\\
1070	-106.792127258186\\
1071	-99.0126332117056\\
1072	-107.61592353183\\
1073	-93.0981276806685\\
1074	-91.2163886867911\\
1075	-81.0809574299631\\
1076	-81.6271792271307\\
1077	-75.3880274491592\\
1078	-121.190639258704\\
1079	-177.914370151446\\
1080	-189.571232660362\\
1081	-183.720902154167\\
1082	-114.851993412575\\
1083	-173.955269282282\\
1084	-201.372611593973\\
1085	-207.695436133982\\
1086	-170.014427629482\\
1087	-212.061716063864\\
1088	-278.269321072586\\
1089	-201.235871372318\\
1090	-172.169325166941\\
1091	-100.665321194169\\
1092	-83.811723391429\\
1093	-69.964626357093\\
1094	-87.859828314392\\
1095	-61.3246428444943\\
1096	-37.7955311560893\\
1097	-35.6015969958978\\
1098	-55.644123501082\\
1099	-82.9745696617842\\
1100	-75.8854550991061\\
1101	-77.5056436485183\\
1102	-99.4400461914177\\
1103	-106.309295958048\\
1104	-143.329566256406\\
1105	-126.526922029497\\
1106	-137.560367390696\\
1107	-104.27545731434\\
1108	-93.6220684922512\\
1109	-105.46123533924\\
1110	-80.6466764283282\\
1111	-71.7174319631047\\
1112	-87.1691903004164\\
1113	-87.1404539865923\\
1114	-65.5135820274177\\
1115	-67.263680129962\\
1116	-47.8125573274726\\
1117	-51.4556536720556\\
1118	-60.4254716849584\\
1119	-46.2373289106592\\
1120	-52.671910607573\\
1121	-66.8199073781739\\
1122	-105.070574921709\\
1123	-98.6271071350186\\
1124	-117.143971607898\\
1125	-65.3825077550537\\
1126	-98.6911618384686\\
1127	-148.380795052137\\
1128	-118.600311509229\\
1129	-130.528277938279\\
1130	-123.288154082686\\
1131	-80.0346708281379\\
1132	-53.2340751357026\\
1133	-76.3172084962305\\
1134	-86.1055207605418\\
1135	-138.12638559118\\
1136	-168.271858733656\\
1137	-166.98904736523\\
1138	-119.668148686621\\
1139	-138.31656599238\\
1140	-117.855694452064\\
1141	-114.427837958283\\
1142	-114.08428964669\\
1143	-74.3425727761817\\
1144	-69.2636862546844\\
1145	-61.050652026343\\
1146	-56.6637035598017\\
1147	-65.9233139303672\\
1148	-94.6764672371148\\
1149	-133.996563269191\\
1150	-120.538224913424\\
1151	-77.4638966346486\\
1152	-67.4347732672443\\
1153	-65.5768273726976\\
1154	-36.7913453139829\\
1155	-26.8774075838357\\
1156	-34.7839739648859\\
1157	-62.3287273218853\\
1158	-50.9353623875545\\
1159	-56.6181746896825\\
1160	-59.4175574633932\\
1161	-58.2103030449193\\
1162	-39.6644169535773\\
1163	-30.2696130702036\\
1164	-22.0421405295015\\
1165	-35.6204439229748\\
1166	-82.7322879356566\\
1167	-111.729129475326\\
1168	-127.318802250074\\
1169	-89.378242154404\\
1170	-60.2407312639239\\
1171	-45.8856378049218\\
1172	-33.8729121075305\\
1173	-66.0967130311903\\
1174	-65.1270181773216\\
1175	-89.2702446770334\\
1176	-116.679378665307\\
1177	-191.888170117714\\
1178	-223.696106030967\\
1179	-185.230471511957\\
1180	-164.941908355678\\
1181	-105.740892540964\\
1182	-124.512128293444\\
1183	-133.685208097376\\
1184	-141.393651023643\\
1185	-114.562888171162\\
1186	-96.8366202718855\\
1187	-104.077595387804\\
1188	-96.0071375236487\\
1189	-103.144641059149\\
1190	-79.0168618153907\\
1191	-129.40017197917\\
1192	-161.450177238967\\
1194	-72.9169536642323\\
1195	-70.9116968767048\\
1196	-120.371825266038\\
1197	-151.744578587033\\
1198	-176.860866313396\\
1199	-193.786989951709\\
1200	-203.764821891694\\
1201	-151.217569597925\\
1202	-145.318330137958\\
1203	-158.103392535546\\
1204	-181.107306325907\\
1205	-104.946529589974\\
1206	-71.239853821892\\
1207	-105.403526166714\\
1208	-89.7289327532224\\
1209	-52.755774724621\\
1210	-72.3704041417725\\
1211	-85.9953980478001\\
1212	-70.7196930987502\\
1213	-59.1689070041377\\
1214	-71.206282968679\\
1215	-61.5733083151324\\
1216	-75.3362937932257\\
1217	-116.133987647634\\
1218	-103.227459439589\\
1219	-73.9090411049335\\
1220	-72.5102172760373\\
1221	-107.016266008999\\
1222	-174.421085802227\\
1223	-141.60985433954\\
1224	-77.5262746863145\\
1225	-62.1914290248321\\
1226	-71.4515741877326\\
1227	-69.7272298753153\\
1228	-47.018216921995\\
1229	-55.3746610801979\\
1230	-66.5790742179963\\
1231	-82.9027653131868\\
1232	-96.5143410890792\\
1233	-66.7524585729218\\
1234	-105.298999056955\\
1235	-128.270174057368\\
1236	-116.651156826358\\
1237	-92.9061169507875\\
1238	-102.258472021907\\
1239	-131.431477033748\\
1240	-99.9249873002038\\
1241	-35.4356868984146\\
1243	-65.6522710976728\\
1244	-88.4268972782763\\
1245	-76.5169071285545\\
1246	-57.847404550336\\
1247	-85.4660931645274\\
1248	-76.8829634143187\\
1249	-57.2580210382037\\
1250	-77.0643695827152\\
1251	-66.1478892891892\\
1252	-57.7814614517431\\
1253	-71.1709698138488\\
1254	-42.4906590661551\\
1255	-50.1487825073104\\
1256	-36.2562085216271\\
1257	-41.4302539869655\\
1258	-78.9917133747435\\
1259	-81.483867439701\\
1260	-109.758308655455\\
1261	-150.763515199632\\
1263	-58.8969155870307\\
1264	-46.3155741648034\\
1265	-46.1275663414001\\
1266	-69.9001816278676\\
1267	-42.988513657554\\
1268	-46.0272422782912\\
1269	-60.9708156980187\\
1270	-103.902369209356\\
1271	-95.1256358853643\\
1272	-134.976159908913\\
1273	-93.2144057205824\\
1274	-79.3062541075278\\
1275	-40.2060291259893\\
1276	-46.5271522508242\\
1277	-29.5266671926749\\
1279	-38.8440773295331\\
1280	-66.2564616670497\\
1281	-71.246712759749\\
1282	-77.6848685273078\\
1283	-83.1802141995829\\
1284	-127.315410744539\\
1285	-125.7218004148\\
1286	-86.2378900841011\\
1288	-110.341266928658\\
1289	-138.074154095419\\
1290	-112.461696607889\\
1291	-78.8892618512452\\
1292	-36.5511489105149\\
1293	-34.0189597570518\\
1294	-31.8407235130153\\
1295	-37.8903474554513\\
1296	-38.7504475044748\\
1297	-37.291747504222\\
1298	-46.1476929776904\\
1299	-79.4386010045062\\
1300	-72.0848101046856\\
1301	-67.1532079117101\\
1302	-80.4866237524045\\
1303	-53.7385587165688\\
1304	-32.1128545466099\\
1305	-57.1795187294399\\
1306	-88.1050561998802\\
1307	-83.9938017941342\\
1308	-98.8631923683033\\
1310	-62.5826142384424\\
1311	-83.3127152906429\\
1312	-112.650253940999\\
1313	-115.706325917255\\
1314	-85.5890237623339\\
1315	-136.398197827352\\
1316	-105.447327742581\\
1317	-97.5728073651824\\
1318	-118.772320153952\\
1319	-114.401097596011\\
1320	-90.8220279851166\\
1321	-82.0505902106506\\
1322	-123.25571891508\\
1323	-174.139200048974\\
1324	-137.96960704228\\
1325	-77.7187784609496\\
1326	-74.3833158200839\\
1327	-84.6096817451928\\
1329	-108.083601290122\\
1330	-87.8940449039878\\
1331	-97.0530379004845\\
1332	-118.291811359817\\
1333	-121.852293652248\\
1334	-88.6939288690401\\
1335	-79.1954110767701\\
1336	-93.5107121793048\\
1337	-155.885965106936\\
1338	-135.956449517832\\
1339	-134.962580321526\\
1340	-81.2889687177349\\
1341	-76.5461776246\\
1342	-73.3693717158549\\
1343	-47.8158050306515\\
1344	-27.5595456311758\\
1345	-24.2097220996354\\
1346	-22.5672084417888\\
1347	-50.9268808774914\\
1348	-72.4713056803869\\
1349	-88.2236108986276\\
1350	-67.3346566396324\\
1351	-70.4766044844746\\
1352	-82.1631233451221\\
1353	-53.0205242295101\\
1354	-56.1035426874919\\
1355	-55.5587323599543\\
1356	-42.3048341320712\\
1357	-52.7224312629269\\
1358	-81.0318881666444\\
1359	-102.423045179955\\
1360	-63.0701632979512\\
1361	-50.3850965158904\\
1362	-45.5618874925972\\
1363	-54.7485440927996\\
1364	-33.1727791959106\\
1366	-127.312579809651\\
1367	-107.082474624176\\
1368	-135.045814831101\\
1369	-158.986316553142\\
1370	-165.707847303161\\
1371	-120.51983980323\\
1372	-91.6003992463106\\
1373	-95.4030962282002\\
1374	-90.1229835441814\\
1375	-104.609080767116\\
1376	-121.194169411157\\
1377	-119.432575420174\\
1378	-162.371278684482\\
1379	-182.348830102669\\
1380	-212.896510393907\\
1381	-147.388784125922\\
1382	-173.41726206602\\
1383	-174.138540426379\\
1384	-146.471114989037\\
1385	-162.924244201796\\
1386	-137.825709673518\\
1387	-79.6259383271611\\
1388	-47.2710559562927\\
1389	-51.6809773636371\\
1390	-79.7154621249872\\
1391	-68.1383183066716\\
1392	-46.4408036348241\\
1393	-65.5181047627093\\
1394	-48.4939614961245\\
1395	-37.3296604982213\\
1396	-46.5951047044343\\
1397	-49.5649767895009\\
1398	-41.8288305553344\\
1400	-93.853474375097\\
1401	-72.584873560156\\
1403	-176.675996975104\\
1404	-156.954141528294\\
1405	-195.175782443863\\
};
\addlegendentry{OSA predition}

\addplot [color=mycolor3, dotted, line width=2.0pt]
  table[row sep=crcr]{%
1006	-125.732\\
1007	-153.809\\
1008	-115.967\\
1009	-61.0350000000001\\
1010	-72.3788963339414\\
1011	-72.1910258293001\\
1012	-91.8732467146499\\
1013	-72.8176863041749\\
1014	-27.4399196704139\\
1015	-25.3329781314335\\
1016	-22.287117408325\\
1017	-62.3119600621367\\
1018	-96.7479300521532\\
1019	-105.924917282651\\
1020	-107.697238097631\\
1021	-81.1522817356833\\
1022	-47.5838698568871\\
1023	-108.566334454978\\
1024	-75.637189661508\\
1025	-59.7109426912969\\
1026	-100.754613859637\\
1028	-73.3801505394597\\
1029	-117.859233828287\\
1030	-105.70471808449\\
1031	-83.5253888433147\\
1032	-115.543927948748\\
1033	-114.124234212662\\
1034	-95.2288574509666\\
1035	-78.9299152688136\\
1036	-64.6114973568358\\
1038	-88.1294119863869\\
1039	-87.7801834122331\\
1040	-87.0300449702154\\
1041	-90.8529977470243\\
1042	-93.0065782654351\\
1043	-131.401731537255\\
1044	-121.301134779581\\
1045	-86.0401302480757\\
1046	-77.587819899614\\
1047	-44.4305604535791\\
1048	-46.3489772432613\\
1049	-65.9292264534799\\
1050	-52.9446792320753\\
1051	-53.7415340906136\\
1052	-76.604903732964\\
1053	-85.27749996532\\
1054	-80.6669390440136\\
1055	-126.225577796557\\
1056	-94.3055311550613\\
1057	-57.2166015767421\\
1058	-47.7920387529771\\
1060	-74.0617103859549\\
1061	-38.0655669844828\\
1062	-41.6671741098312\\
1063	-57.7479380125906\\
1065	-40.4256900248683\\
1066	-54.8899141788108\\
1067	-58.2409422322692\\
1068	-98.2967341203312\\
1069	-85.0676205281957\\
1070	-107.197607820813\\
1071	-97.3153969341543\\
1072	-107.155569722283\\
1073	-90.5546680950454\\
1074	-87.8735889588513\\
1075	-79.5751295350983\\
1076	-79.8063684163101\\
1077	-76.0941471357712\\
1078	-122.253016874409\\
1079	-176.798854714503\\
1080	-184.332404537266\\
1081	-177.070065547869\\
1082	-109.411304716342\\
1083	-165.666472525645\\
1084	-194.259586129728\\
1085	-201.585215406673\\
1086	-160.119750734127\\
1087	-201.206866220495\\
1088	-267.660479805598\\
1089	-192.46832860712\\
1090	-166.407633755924\\
1091	-98.5352284143726\\
1092	-83.0419046444333\\
1093	-64.8067587702433\\
1094	-84.8475537524853\\
1095	-56.2013786739712\\
1096	-34.3138304896438\\
1097	-30.2334494169097\\
1098	-47.8940242595158\\
1099	-75.2811049150505\\
1100	-71.1187980216509\\
1101	-74.281284902743\\
1102	-97.1068623147703\\
1103	-104.650832676948\\
1104	-142.033061530103\\
1105	-126.592463900745\\
1106	-139.784596976579\\
1107	-108.130900143281\\
1108	-93.2160695117007\\
1109	-107.707239279719\\
1110	-82.3119207290263\\
1111	-74.4653276963138\\
1112	-90.6029758612949\\
1113	-90.7460312891558\\
1114	-68.8886820716245\\
1115	-71.1422263954134\\
1116	-52.7753526981098\\
1117	-55.0218854698578\\
1118	-63.0288952682754\\
1119	-48.3144695993838\\
1120	-56.345164321194\\
1121	-70.5330715527512\\
1122	-108.924271494029\\
1123	-102.420174907934\\
1124	-119.954686336354\\
1125	-65.4616530025878\\
1126	-100.178141948084\\
1127	-149.391630330301\\
1128	-120.772220343759\\
1129	-131.684164015474\\
1130	-126.866673014214\\
1131	-81.726448514486\\
1132	-53.1471947795587\\
1133	-76.9032284044831\\
1134	-86.1741728922539\\
1135	-138.641051372045\\
1136	-168.868303501582\\
1137	-166.920931180302\\
1138	-117.948125567958\\
1139	-135.696539497114\\
1140	-116.367821186727\\
1141	-115.907899417406\\
1142	-114.652685239407\\
1143	-76.1941237785331\\
1144	-70.4625587030901\\
1145	-61.4607020986455\\
1146	-57.7049550639399\\
1147	-66.2210325476294\\
1148	-95.8139634040383\\
1149	-136.28700139978\\
1150	-121.631557520897\\
1151	-78.0593226710489\\
1152	-71.2207163094108\\
1153	-66.5361095207752\\
1154	-40.678454510904\\
1155	-28.2147415604688\\
1156	-35.2254935519293\\
1157	-61.973152064528\\
1158	-50.1501239733423\\
1159	-55.7591256923588\\
1160	-59.5662790416177\\
1161	-59.338184261431\\
1162	-40.8257611925621\\
1164	-24.0374888368626\\
1165	-37.7025480322161\\
1166	-83.9634060740341\\
1167	-112.043131205706\\
1168	-126.452301110381\\
1169	-86.8433855084816\\
1170	-59.0810748965407\\
1171	-45.756076532032\\
1172	-33.9144403007804\\
1173	-66.5679929576831\\
1174	-65.4012026643736\\
1175	-89.1483005615451\\
1176	-115.985297904745\\
1177	-190.660782737652\\
1178	-223.080108576178\\
1179	-187.024910271582\\
1180	-168.957477680418\\
1181	-109.749450612355\\
1182	-125.531363034432\\
1183	-134.927935117463\\
1184	-143.236284511048\\
1185	-115.233450146393\\
1186	-97.1144321685008\\
1187	-105.956060341002\\
1188	-96.3514772955943\\
1189	-107.357081126885\\
1190	-81.7788195749208\\
1191	-131.975732708376\\
1192	-163.657100030572\\
1193	-116.362526431902\\
1194	-74.9706867588932\\
1195	-72.9998432949153\\
1196	-122.450960611142\\
1197	-152.415486389718\\
1198	-176.558599082716\\
1199	-191.466075986021\\
1200	-197.52665477312\\
1201	-146.233227518739\\
1202	-141.033962246575\\
1203	-154.602462674163\\
1204	-180.444594753555\\
1205	-101.807223565511\\
1206	-67.2426752249357\\
1207	-100.50647710364\\
1208	-87.0022538625735\\
1209	-52.2639949546767\\
1210	-71.8393025969458\\
1211	-84.9067709131218\\
1212	-69.9414956291655\\
1213	-59.8754824819339\\
1214	-73.9207674502743\\
1215	-65.3571719211777\\
1216	-78.6350884011219\\
1217	-119.369257757821\\
1218	-105.932354520085\\
1219	-79.5794014229921\\
1220	-76.9679137920057\\
1221	-113.300778398441\\
1222	-180.683665015234\\
1223	-145.224100257132\\
1224	-81.6297349850392\\
1225	-69.3564816268217\\
1226	-74.9846944459639\\
1227	-74.7268096657158\\
1228	-50.6157927181155\\
1230	-69.516392597186\\
1231	-86.5629083787933\\
1232	-99.2605894934281\\
1233	-69.8627558255466\\
1234	-109.963275040844\\
1235	-132.853089438921\\
1236	-120.752403926128\\
1237	-96.7158532955982\\
1238	-104.908278241836\\
1239	-135.032762726694\\
1240	-101.811434105769\\
1241	-37.8076771937601\\
1242	-54.5978021350288\\
1243	-64.1463584541762\\
1244	-88.1567617463288\\
1245	-77.9760139260356\\
1246	-61.0547879663675\\
1247	-90.1204656243001\\
1248	-82.7971125024033\\
1249	-61.6228051975684\\
1250	-80.0801733881619\\
1251	-70.299547815988\\
1252	-62.5647915738107\\
1253	-75.8291164601956\\
1254	-46.9610657364117\\
1255	-54.8810410080614\\
1256	-39.9020627604816\\
1257	-44.8469061786859\\
1258	-82.3635310839431\\
1259	-85.1085668360624\\
1260	-113.030714246842\\
1261	-152.677160912486\\
1262	-104.497018029193\\
1263	-62.2469580914189\\
1264	-49.0525157667294\\
1265	-48.5119690811641\\
1266	-72.4979956561756\\
1267	-46.6206298468514\\
1268	-50.0966164107676\\
1269	-63.8546891541314\\
1270	-106.646170862544\\
1271	-97.3246064529098\\
1272	-137.647496355532\\
1273	-95.2005616120928\\
1274	-81.923387005879\\
1275	-42.5060381734472\\
1276	-45.9781718058093\\
1277	-28.7325928580467\\
1278	-34.6521548018329\\
1279	-38.6991050795427\\
1280	-66.3550595182405\\
1281	-71.298200412976\\
1282	-77.5760705301486\\
1283	-82.9613865561687\\
1284	-126.26585409441\\
1285	-124.793219736862\\
1286	-88.6612357650736\\
1287	-103.423441167956\\
1288	-113.248929410278\\
1289	-140.401082009914\\
1290	-113.344139996222\\
1291	-77.0430667724613\\
1292	-36.2939058464476\\
1293	-33.4301718304798\\
1294	-31.2240775005118\\
1295	-39.5652983414179\\
1296	-39.9765667775644\\
1297	-38.7334166814042\\
1298	-46.9575886954019\\
1299	-79.3928309353432\\
1300	-72.2735635816362\\
1301	-69.4463189002809\\
1302	-81.7503744643243\\
1303	-53.2962769394596\\
1304	-32.4413927349981\\
1305	-58.1020268494906\\
1306	-88.8999121643371\\
1307	-83.8638171754085\\
1308	-97.6497858532359\\
1310	-61.9324385252435\\
1311	-82.2617625545142\\
1312	-111.880152364373\\
1313	-113.338239394076\\
1314	-82.4978752172537\\
1315	-133.136680748439\\
1316	-103.557335583843\\
1317	-96.6584428736435\\
1318	-118.623186023527\\
1319	-113.026437978775\\
1320	-90.9368905332028\\
1321	-81.6056496961921\\
1322	-125.452203801892\\
1323	-176.090983315827\\
1324	-137.205404046497\\
1325	-77.1267952271553\\
1326	-73.0158158353715\\
1327	-82.7429964410976\\
1329	-108.024932902812\\
1330	-85.1849445838016\\
1331	-94.6302568684816\\
1332	-118.077517473884\\
1333	-122.419025327338\\
1334	-85.965393618259\\
1335	-75.6705848919355\\
1336	-93.4776870608521\\
1337	-157.877780682203\\
1338	-137.46308677095\\
1339	-136.011841653553\\
1340	-80.4348627203058\\
1341	-74.2558617399031\\
1342	-71.234678195108\\
1343	-46.5025558198138\\
1344	-26.9075935350897\\
1345	-22.868825800517\\
1346	-20.4135636890785\\
1347	-48.5391907220176\\
1348	-69.271002554322\\
1349	-85.249470445983\\
1350	-65.9888387653289\\
1351	-69.1587525202144\\
1352	-81.0143405359513\\
1353	-51.078127794008\\
1354	-54.7735706188585\\
1355	-54.0299259102856\\
1356	-41.7929236905818\\
1357	-52.9101636106768\\
1358	-81.8415034567504\\
1359	-102.746955507418\\
1360	-61.6708046297053\\
1361	-49.0052236571803\\
1362	-44.7622107010047\\
1363	-55.4685753783851\\
1364	-35.6925227402617\\
1366	-130.127369671749\\
1367	-108.72396615863\\
1368	-136.193314124427\\
1369	-159.987419883211\\
1370	-164.702180143391\\
1371	-119.883209634501\\
1372	-90.5814712836136\\
1373	-93.5828586630475\\
1374	-90.893865354034\\
1376	-122.353721751802\\
1377	-119.260680848866\\
1378	-160.111489116406\\
1379	-178.249713038798\\
1380	-207.944425628685\\
1381	-143.985235702909\\
1382	-167.5069813562\\
1383	-171.542314516932\\
1384	-144.44881225497\\
1385	-158.643960260941\\
1386	-138.321540460171\\
1387	-77.0587304368225\\
1388	-46.6289755189737\\
1389	-48.8448894712944\\
1390	-75.6713771748196\\
1391	-63.9242831719964\\
1392	-43.7672026088715\\
1393	-63.9126071647686\\
1394	-48.3707187979192\\
1395	-37.9294586122228\\
1396	-47.2134298006463\\
1397	-50.3236593902047\\
1398	-41.6966840048667\\
1399	-69.8543086024576\\
1400	-95.0017476061732\\
1401	-72.7197057453377\\
1403	-180.457757551776\\
1404	-163.056569457911\\
1405	-204.575085460168\\
};
\addlegendentry{MPO prediction}

\end{axis}

\begin{axis}[%
width=6.159cm,
height=1.831cm,
at={(0cm,0cm)},
scale only axis,
xmin=1000,
xmax=1405,
xlabel style={font=\color{white!15!black}},
xlabel={Sample index},
ymin=-200,
ymax=0,
ylabel style={font=\color{white!15!black}},
ylabel={$y(t)$},
axis background/.style={fill=white},
title style={font=\bfseries},
title={C9: RMSE(OSA) = 3.0702, RMSE(MPO) = 3.593},
legend style={legend cell align=left, align=left, draw=white!15!black}
]
\addplot [color=mycolor1, line width=2.0pt]
  table[row sep=crcr]{%
1006	-86.6700000000001\\
1007	-103.76\\
1008	-79.346\\
1009	-45.1659999999999\\
1010	-57.373\\
1011	-51.27\\
1012	-63.4770000000001\\
1013	-48.828\\
1014	-24.414\\
1015	-17.0899999999999\\
1016	-17.0899999999999\\
1017	-43.9449999999999\\
1018	-73.242\\
1019	-76.904\\
1020	-79.346\\
1022	-36.6210000000001\\
1023	-76.904\\
1024	-53.711\\
1025	-43.9449999999999\\
1026	-69.5799999999999\\
1027	-61.0350000000001\\
1028	-48.828\\
1029	-84.229\\
1030	-79.346\\
1031	-59.8140000000001\\
1032	-87.8910000000001\\
1033	-85.4490000000001\\
1034	-67.1389999999999\\
1035	-54.932\\
1036	-46.3869999999999\\
1037	-56.152\\
1038	-64.6970000000001\\
1039	-64.6970000000001\\
1040	-67.1389999999999\\
1041	-68.3589999999999\\
1042	-73.242\\
1043	-100.098\\
1044	-87.8910000000001\\
1045	-61.0350000000001\\
1046	-56.152\\
1047	-34.1800000000001\\
1048	-34.1800000000001\\
1049	-48.828\\
1050	-37.8420000000001\\
1051	-39.0630000000001\\
1052	-56.152\\
1053	-62.2560000000001\\
1054	-58.5940000000001\\
1055	-90.3320000000001\\
1056	-64.6970000000001\\
1057	-41.5039999999999\\
1058	-34.1800000000001\\
1059	-45.1659999999999\\
1060	-51.27\\
1061	-28.076\\
1062	-25.635\\
1063	-41.5039999999999\\
1064	-34.1800000000001\\
1065	-29.297\\
1066	-40.2829999999999\\
1067	-43.9449999999999\\
1068	-68.3589999999999\\
1069	-63.4770000000001\\
1070	-79.346\\
1071	-69.5799999999999\\
1072	-79.346\\
1073	-63.4770000000001\\
1074	-62.2560000000001\\
1075	-54.932\\
1076	-53.711\\
1077	-53.711\\
1079	-126.953\\
1080	-131.836\\
1081	-129.395\\
1082	-83.008\\
1083	-114.746\\
1084	-141.602\\
1085	-147.705\\
1086	-111.084\\
1087	-146.484\\
1088	-185.547\\
1089	-131.836\\
1090	-112.305\\
1091	-75.684\\
1092	-61.0350000000001\\
1093	-52.49\\
1094	-64.6970000000001\\
1095	-46.3869999999999\\
1096	-31.7380000000001\\
1097	-31.7380000000001\\
1098	-40.2829999999999\\
1099	-57.373\\
1100	-52.49\\
1101	-53.711\\
1102	-72.021\\
1103	-72.021\\
1104	-97.6559999999999\\
1105	-79.346\\
1106	-101.318\\
1107	-72.021\\
1108	-64.6970000000001\\
1109	-73.242\\
1110	-54.932\\
1111	-51.27\\
1112	-59.8140000000001\\
1113	-62.2560000000001\\
1114	-43.9449999999999\\
1115	-47.607\\
1116	-34.1800000000001\\
1117	-40.2829999999999\\
1118	-42.7249999999999\\
1119	-30.518\\
1121	-47.607\\
1122	-72.021\\
1123	-74.463\\
1124	-81.787\\
1125	-48.828\\
1126	-70.8009999999999\\
1127	-101.318\\
1128	-84.229\\
1129	-93.9939999999999\\
1130	-91.5530000000001\\
1131	-54.932\\
1132	-40.2829999999999\\
1133	-56.152\\
1134	-61.0350000000001\\
1135	-97.6559999999999\\
1136	-119.629\\
1137	-117.188\\
1138	-89.1110000000001\\
1139	-92.7729999999999\\
1140	-81.787\\
1142	-79.346\\
1143	-54.932\\
1144	-48.828\\
1145	-45.1659999999999\\
1146	-40.2829999999999\\
1147	-45.1659999999999\\
1148	-64.6970000000001\\
1149	-96.4359999999999\\
1151	-52.49\\
1152	-45.1659999999999\\
1153	-46.3869999999999\\
1154	-29.297\\
1155	-23.193\\
1156	-26.855\\
1157	-46.3869999999999\\
1158	-35.4000000000001\\
1159	-41.5039999999999\\
1161	-39.0630000000001\\
1162	-28.076\\
1163	-21.973\\
1164	-17.0899999999999\\
1165	-29.297\\
1166	-58.5940000000001\\
1167	-80.566\\
1168	-91.5530000000001\\
1169	-59.8140000000001\\
1170	-43.9449999999999\\
1171	-32.9590000000001\\
1172	-23.193\\
1173	-47.607\\
1174	-46.3869999999999\\
1175	-67.1389999999999\\
1176	-81.787\\
1177	-131.836\\
1178	-147.705\\
1179	-120.85\\
1180	-117.188\\
1181	-78.125\\
1182	-81.787\\
1183	-91.5530000000001\\
1184	-98.877\\
1185	-73.242\\
1186	-68.3589999999999\\
1187	-69.5799999999999\\
1188	-62.2560000000001\\
1189	-75.684\\
1190	-53.711\\
1191	-93.9939999999999\\
1192	-108.643\\
1194	-51.27\\
1195	-54.932\\
1196	-92.7729999999999\\
1197	-109.863\\
1198	-134.277\\
1199	-140.381\\
1200	-140.381\\
1201	-106.201\\
1202	-96.4359999999999\\
1203	-111.084\\
1204	-129.395\\
1205	-78.125\\
1206	-50.049\\
1207	-73.242\\
1208	-57.373\\
1209	-39.0630000000001\\
1210	-56.152\\
1211	-61.0350000000001\\
1212	-46.3869999999999\\
1213	-37.8420000000001\\
1214	-50.049\\
1215	-39.0630000000001\\
1216	-53.711\\
1217	-76.904\\
1218	-72.021\\
1219	-51.27\\
1220	-51.27\\
1221	-78.125\\
1222	-123.291\\
1223	-86.6700000000001\\
1224	-56.152\\
1225	-43.9449999999999\\
1226	-48.828\\
1227	-43.9449999999999\\
1228	-34.1800000000001\\
1229	-37.8420000000001\\
1230	-47.607\\
1231	-59.8140000000001\\
1232	-64.6970000000001\\
1233	-46.3869999999999\\
1234	-74.463\\
1235	-85.4490000000001\\
1236	-83.008\\
1237	-65.9180000000001\\
1238	-73.242\\
1239	-96.4359999999999\\
1240	-61.0350000000001\\
1241	-29.297\\
1242	-42.7249999999999\\
1243	-43.9449999999999\\
1244	-58.5940000000001\\
1245	-50.049\\
1246	-40.2829999999999\\
1247	-58.5940000000001\\
1248	-54.932\\
1249	-39.0630000000001\\
1250	-48.828\\
1252	-39.0630000000001\\
1253	-50.049\\
1254	-29.297\\
1255	-36.6210000000001\\
1256	-26.855\\
1257	-31.7380000000001\\
1258	-53.711\\
1259	-61.0350000000001\\
1260	-76.904\\
1261	-101.318\\
1262	-67.1389999999999\\
1263	-42.7249999999999\\
1264	-32.9590000000001\\
1265	-31.7380000000001\\
1266	-47.607\\
1267	-30.518\\
1268	-36.6210000000001\\
1269	-43.9449999999999\\
1270	-65.9180000000001\\
1271	-62.2560000000001\\
1272	-86.6700000000001\\
1273	-59.8140000000001\\
1274	-56.152\\
1275	-31.7380000000001\\
1276	-32.9590000000001\\
1277	-20.752\\
1278	-24.414\\
1279	-26.855\\
1280	-47.607\\
1281	-47.607\\
1282	-56.152\\
1283	-59.8140000000001\\
1284	-93.9939999999999\\
1285	-79.346\\
1286	-62.2560000000001\\
1287	-73.242\\
1288	-79.346\\
1289	-101.318\\
1290	-80.566\\
1291	-52.49\\
1292	-30.518\\
1293	-21.973\\
1294	-23.193\\
1295	-29.297\\
1296	-29.297\\
1297	-26.855\\
1298	-36.6210000000001\\
1299	-53.711\\
1300	-47.607\\
1301	-51.27\\
1302	-57.373\\
1303	-37.8420000000001\\
1304	-20.752\\
1306	-63.4770000000001\\
1307	-61.0350000000001\\
1308	-69.5799999999999\\
1309	-58.5940000000001\\
1310	-45.1659999999999\\
1311	-59.8140000000001\\
1312	-84.229\\
1313	-85.4490000000001\\
1314	-59.8140000000001\\
1315	-102.539\\
1316	-76.904\\
1317	-74.463\\
1318	-84.229\\
1319	-83.008\\
1320	-63.4770000000001\\
1321	-54.932\\
1323	-124.512\\
1324	-95.2149999999999\\
1325	-57.373\\
1326	-53.711\\
1327	-54.932\\
1328	-72.021\\
1329	-83.008\\
1330	-59.8140000000001\\
1331	-64.6970000000001\\
1332	-83.008\\
1333	-90.3320000000001\\
1334	-61.0350000000001\\
1335	-50.049\\
1336	-63.4770000000001\\
1337	-104.98\\
1338	-92.7729999999999\\
1339	-95.2149999999999\\
1340	-58.5940000000001\\
1341	-53.711\\
1342	-51.27\\
1343	-34.1800000000001\\
1344	-23.193\\
1345	-20.752\\
1346	-17.0899999999999\\
1347	-40.2829999999999\\
1348	-54.932\\
1349	-62.2560000000001\\
1350	-47.607\\
1352	-57.373\\
1353	-37.8420000000001\\
1354	-39.0630000000001\\
1355	-39.0630000000001\\
1356	-28.076\\
1357	-39.0630000000001\\
1358	-57.373\\
1359	-73.242\\
1360	-48.828\\
1361	-35.4000000000001\\
1362	-31.7380000000001\\
1363	-36.6210000000001\\
1364	-26.855\\
1365	-54.932\\
1366	-93.9939999999999\\
1367	-72.021\\
1368	-97.6559999999999\\
1369	-113.525\\
1370	-114.746\\
1371	-85.4490000000001\\
1372	-63.4770000000001\\
1374	-63.4770000000001\\
1375	-75.684\\
1376	-89.1110000000001\\
1377	-86.6700000000001\\
1378	-115.967\\
1379	-125.732\\
1380	-150.146\\
1381	-96.4359999999999\\
1382	-109.863\\
1383	-122.07\\
1384	-92.7729999999999\\
1385	-107.422\\
1386	-92.7729999999999\\
1387	-54.932\\
1388	-39.0630000000001\\
1389	-40.2829999999999\\
1390	-57.373\\
1391	-42.7249999999999\\
1392	-32.9590000000001\\
1393	-45.1659999999999\\
1394	-34.1800000000001\\
1395	-26.855\\
1396	-34.1800000000001\\
1397	-37.8420000000001\\
1398	-25.635\\
1399	-48.828\\
1400	-64.6970000000001\\
1401	-46.3869999999999\\
1403	-115.967\\
1404	-106.201\\
1405	-133.057\\
};
\addlegendentry{True output}

\addplot [color=mycolor2, dashed, line width=2.0pt]
  table[row sep=crcr]{%
1006	-85.587535083137\\
1007	-103.446388886553\\
1008	-79.3093233003451\\
1009	-43.000243339997\\
1010	-50.664046363177\\
1011	-53.0319049845652\\
1012	-63.911156782998\\
1013	-56.3272253347911\\
1014	-18.7213987728378\\
1015	-19.8775879843879\\
1016	-18.0184345312612\\
1017	-47.0708929546763\\
1018	-67.6671673556446\\
1019	-77.2986178889901\\
1020	-77.8748904992831\\
1021	-59.2117241578303\\
1022	-36.8052378539992\\
1023	-76.0019760954226\\
1024	-55.454063696264\\
1025	-41.7280021248084\\
1026	-67.4012934034795\\
1027	-61.5858346135415\\
1028	-52.9180190895445\\
1029	-84.7067673582408\\
1030	-78.0615267545163\\
1031	-58.3502341818553\\
1032	-81.7946758410026\\
1033	-85.7246604387735\\
1034	-65.7432175296374\\
1035	-59.8366707497496\\
1036	-45.0802864229756\\
1038	-62.293325143936\\
1039	-63.0044117359516\\
1040	-62.7554087253993\\
1041	-66.1817131783398\\
1042	-74.0233007649124\\
1043	-94.9842336240708\\
1044	-90.5095424797055\\
1045	-58.5398426486115\\
1046	-60.8393528497247\\
1047	-33.4957912985437\\
1048	-33.8553613410904\\
1049	-45.0282370270731\\
1050	-38.0167325646369\\
1051	-42.3224250346764\\
1052	-51.2543873456193\\
1053	-62.843211560536\\
1054	-58.0364492034007\\
1055	-86.7953487454902\\
1056	-69.7984319591526\\
1057	-43.3714300172981\\
1058	-35.0140960636202\\
1059	-41.8782314091955\\
1060	-50.1164146500789\\
1061	-28.9109500868499\\
1062	-30.830309279595\\
1063	-39.2601622542561\\
1064	-33.3755181006497\\
1065	-28.4484653605712\\
1066	-38.4157084590349\\
1067	-43.6793693814075\\
1068	-66.6422262101648\\
1069	-65.2397751960989\\
1070	-74.171252375105\\
1071	-70.4639108167221\\
1072	-76.2935795348556\\
1073	-62.7236992542134\\
1074	-62.3849030275885\\
1075	-56.160688606142\\
1076	-55.9068538767137\\
1077	-54.8784510291225\\
1078	-87.9750818496689\\
1079	-126.093337127038\\
1080	-128.210152392728\\
1081	-124.071999712553\\
1082	-80.582916474754\\
1083	-113.111662117789\\
1084	-139.474104336113\\
1085	-136.675186449817\\
1086	-104.48436685978\\
1087	-145.303316998764\\
1088	-178.72852795287\\
1089	-137.512583955425\\
1090	-111.739034658081\\
1091	-76.0666237904695\\
1092	-60.830724267822\\
1093	-52.214105260658\\
1094	-62.4646559757207\\
1096	-27.4593570613144\\
1097	-27.6620207222259\\
1098	-36.1315609972773\\
1099	-60.627501362625\\
1100	-54.5827548039933\\
1101	-57.9648769739638\\
1102	-68.3566560716745\\
1103	-72.1575604788122\\
1104	-98.8168989094177\\
1105	-91.0729350503098\\
1106	-95.8254993122914\\
1107	-77.3595305146305\\
1108	-63.5526743760756\\
1109	-76.1617983609606\\
1110	-55.855441636834\\
1111	-50.2045379588837\\
1112	-61.5173490345651\\
1113	-62.3623753939878\\
1114	-42.1122865931741\\
1115	-48.8933841459339\\
1116	-37.7284958437674\\
1117	-39.480518823801\\
1118	-42.1014585972482\\
1119	-31.9417324946594\\
1120	-37.253366223061\\
1121	-47.685856277315\\
1122	-74.6257891928935\\
1123	-75.8554380829505\\
1124	-77.9955115816872\\
1125	-51.7884884830828\\
1126	-67.3243728959544\\
1127	-108.03363098301\\
1128	-82.0850529614274\\
1129	-93.6585863176779\\
1130	-89.8606336822045\\
1131	-56.7050810861751\\
1132	-40.979866099528\\
1133	-54.0502752746706\\
1134	-62.8100346140238\\
1135	-96.1919794995895\\
1136	-118.786212712364\\
1137	-112.493755342286\\
1138	-93.9661560613151\\
1139	-91.5847020058654\\
1140	-86.8215482546418\\
1141	-81.4595537220425\\
1142	-77.838226093691\\
1143	-53.3604362822823\\
1144	-51.7589317743018\\
1145	-44.9650981125137\\
1146	-41.1013924550384\\
1147	-47.4598998114855\\
1148	-62.1916099375062\\
1149	-96.578747574463\\
1150	-78.1581779451958\\
1151	-54.8479419643113\\
1152	-46.8404926568744\\
1153	-47.2151482033273\\
1154	-27.2958323976018\\
1155	-20.0136463181377\\
1156	-24.0238259052294\\
1157	-46.2957049464262\\
1158	-35.1902217509573\\
1159	-39.7589765713471\\
1160	-40.2404442342965\\
1161	-41.2506925812029\\
1162	-32.7546404529571\\
1163	-21.5846300623587\\
1164	-16.8670540677426\\
1165	-27.0392687371241\\
1166	-58.8152029927271\\
1167	-81.3744583174928\\
1168	-90.122692012475\\
1169	-59.9823479759207\\
1170	-44.9216921434424\\
1171	-34.4103213224782\\
1172	-25.9484904961412\\
1173	-46.4606009315555\\
1174	-46.4687363515081\\
1175	-63.1522132900852\\
1176	-82.5268011784326\\
1177	-135.313682369927\\
1178	-156.482404771228\\
1179	-124.21988404521\\
1180	-121.212038872265\\
1181	-74.5047142981668\\
1182	-85.837525559527\\
1183	-88.9260765469792\\
1184	-95.5881933117828\\
1185	-78.990807084062\\
1186	-68.882497146662\\
1187	-68.9965333470984\\
1188	-65.7448346043982\\
1189	-74.2249376268919\\
1190	-56.8096875995261\\
1191	-90.1725674110614\\
1192	-109.221290634248\\
1193	-83.1125465524508\\
1194	-49.149610068014\\
1195	-55.5046681493527\\
1196	-89.8466542503129\\
1197	-108.821347099474\\
1198	-135.310409894752\\
1199	-135.906535665664\\
1200	-138.136114740253\\
1201	-109.705284371177\\
1202	-94.9714669198358\\
1203	-111.339839374529\\
1204	-121.78570167618\\
1205	-73.5359666491106\\
1206	-53.5886408395245\\
1207	-68.3425931789211\\
1208	-65.1586872551275\\
1209	-39.7412050162645\\
1210	-50.8516682193217\\
1211	-59.5126374745178\\
1212	-53.1811317278762\\
1213	-38.2442988342589\\
1214	-51.6123245303429\\
1215	-44.6162526443097\\
1216	-52.6979738522691\\
1217	-77.4167631660086\\
1218	-71.2298063544395\\
1219	-52.4746075963719\\
1220	-54.2435493082974\\
1221	-76.6670814500383\\
1222	-128.244674344346\\
1223	-97.9689439484848\\
1224	-54.1145722210872\\
1225	-41.1444718380696\\
1226	-50.9656510163625\\
1227	-46.6843110333775\\
1228	-36.135220916655\\
1229	-39.091555118395\\
1230	-47.0946313506963\\
1231	-58.6476091844065\\
1232	-65.93445391259\\
1233	-47.1100415725493\\
1234	-79.0796062550505\\
1235	-89.3415173360454\\
1236	-81.7963500953686\\
1237	-67.8822573533719\\
1238	-69.7235588986134\\
1239	-95.3972142612638\\
1240	-66.4419405992865\\
1241	-27.2472129582982\\
1242	-36.7103678082892\\
1243	-44.919524435529\\
1244	-64.5094414829944\\
1245	-52.3239063373808\\
1246	-38.5962046784525\\
1247	-58.3609008731962\\
1248	-57.3476431357267\\
1249	-41.0536869387474\\
1250	-53.144849886052\\
1251	-46.1433503629155\\
1252	-44.0362136029519\\
1253	-46.0727339505954\\
1254	-30.321283448596\\
1255	-35.2074866853509\\
1256	-26.7098854361864\\
1257	-31.4269127614241\\
1258	-54.9454970615398\\
1259	-60.9719026305431\\
1260	-78.690787676946\\
1261	-106.312424371496\\
1262	-71.8314509402564\\
1263	-42.0441909528759\\
1264	-34.0239193858254\\
1265	-33.2964097553968\\
1266	-46.9462045494749\\
1267	-31.0838140588744\\
1268	-34.6525161799173\\
1269	-44.3947275064932\\
1270	-69.8871412641754\\
1271	-64.9332181560922\\
1272	-86.9586444877664\\
1273	-59.041078453175\\
1274	-55.4719909376747\\
1275	-30.2343775499339\\
1276	-30.5539330214465\\
1277	-22.6139203745479\\
1278	-23.3088295776254\\
1279	-28.1863116352883\\
1280	-45.7667878770985\\
1281	-51.8206425526412\\
1282	-54.0857020860626\\
1283	-60.1920155261969\\
1284	-95.2752906018295\\
1285	-88.2460633174039\\
1286	-62.8464934646895\\
1287	-73.1475151524185\\
1288	-74.5077161646125\\
1289	-96.4684689239132\\
1290	-82.8705585201221\\
1291	-54.048265663585\\
1292	-29.1628521904147\\
1293	-26.6294047369711\\
1294	-23.7089306958139\\
1295	-26.67941209992\\
1296	-28.9056333210458\\
1297	-27.7977516586086\\
1298	-35.2625457253812\\
1299	-55.5807672632513\\
1300	-49.6143261314073\\
1301	-49.090403161948\\
1302	-57.0125677281035\\
1303	-38.609037956056\\
1304	-23.3750720670407\\
1305	-39.8752710157603\\
1306	-62.3206192401426\\
1307	-59.796585038712\\
1308	-66.5593079072999\\
1309	-59.3793318141306\\
1310	-45.0279029124331\\
1311	-61.1572568280603\\
1312	-83.5097510710655\\
1313	-80.6260088386348\\
1314	-58.4282517971133\\
1315	-103.971532791087\\
1316	-79.1708525411852\\
1317	-76.1141512426595\\
1318	-82.4086313884786\\
1319	-82.5080257838754\\
1320	-62.9381883807155\\
1321	-58.7996672057964\\
1322	-87.0831096263432\\
1323	-123.56210822175\\
1324	-93.6544351715984\\
1325	-55.0942267997611\\
1326	-55.2928558253971\\
1327	-55.9846727326876\\
1328	-71.1343529379922\\
1329	-80.6187823483979\\
1330	-58.908128407433\\
1331	-70.2660291557752\\
1332	-83.7759012107736\\
1333	-83.4119811363842\\
1334	-60.8877329234926\\
1335	-49.953119426541\\
1336	-65.7473877342018\\
1337	-104.447971672795\\
1338	-96.3098345997653\\
1339	-91.9065542199842\\
1340	-57.2278637606519\\
1341	-52.579549369596\\
1342	-56.110009843798\\
1343	-33.087481354921\\
1344	-20.369138031524\\
1345	-18.2053083891064\\
1346	-17.0712662202723\\
1347	-38.47865944968\\
1348	-53.2242762308094\\
1349	-62.9255437965048\\
1350	-49.1835380250452\\
1351	-50.0558463874747\\
1352	-58.5436401029697\\
1353	-37.8767771400228\\
1354	-39.0692333931545\\
1355	-39.2108648086612\\
1356	-31.6751838707123\\
1357	-34.2511440106005\\
1358	-58.8749691771302\\
1359	-72.6041391378869\\
1360	-46.1090278732345\\
1361	-36.6673928496666\\
1362	-32.7544403614686\\
1363	-39.3022692777281\\
1364	-25.375568035686\\
1365	-58.8309113046168\\
1366	-94.9644384189803\\
1367	-72.691929345252\\
1368	-97.422071777396\\
1369	-110.901293877675\\
1370	-112.30109536893\\
1371	-87.4745567713394\\
1372	-65.8023844763188\\
1373	-64.3576568120336\\
1374	-65.6309351071236\\
1376	-83.0889946863908\\
1377	-86.823329932357\\
1378	-113.998780149798\\
1379	-123.09859564089\\
1380	-145.003648101382\\
1381	-97.5932110832557\\
1383	-117.145578097926\\
1384	-98.0213939857842\\
1385	-111.301452671045\\
1386	-90.1936841592064\\
1387	-53.6046372808078\\
1388	-34.5091627854492\\
1389	-37.9359722378065\\
1390	-56.2513145754911\\
1391	-50.5326582602058\\
1392	-34.2762187205624\\
1393	-41.7505191167083\\
1394	-36.448709276568\\
1395	-25.0123060883436\\
1396	-32.9486602931861\\
1397	-37.1696709910777\\
1398	-30.4423807756723\\
1399	-49.2411509572823\\
1400	-66.3671681835144\\
1401	-53.0809559052832\\
1402	-86.1773535650907\\
1403	-124.312466988419\\
1404	-107.539339631597\\
1405	-126.041385468434\\
};
\addlegendentry{OSA predition}

\addplot [color=mycolor3, dotted, line width=2.0pt]
  table[row sep=crcr]{%
1006	-86.6700000000001\\
1007	-103.76\\
1008	-79.346\\
1009	-45.1659999999999\\
1010	-50.6640463631813\\
1011	-51.6874789838469\\
1012	-61.9198899965184\\
1013	-55.6449847156277\\
1014	-19.1802681181916\\
1015	-20.7901394370617\\
1016	-17.5537380540059\\
1017	-48.0787687895449\\
1018	-69.2237495144382\\
1019	-77.9745106034952\\
1020	-77.3939199117885\\
1021	-59.2542612892169\\
1022	-36.0938023716094\\
1023	-75.9263219131674\\
1024	-55.036426357487\\
1025	-41.7353507531234\\
1026	-67.2008974440148\\
1027	-60.5154895370686\\
1028	-52.0754149864467\\
1029	-84.7190123054604\\
1030	-78.9294280893171\\
1031	-58.5117605387395\\
1032	-81.6765662047228\\
1033	-84.2550420501527\\
1034	-63.5308228539363\\
1035	-58.3016635628035\\
1036	-44.1300371927903\\
1037	-53.6806188954279\\
1038	-61.1969309240255\\
1039	-61.6026123388424\\
1040	-60.7470013310813\\
1041	-63.538250640464\\
1042	-70.6002196697364\\
1043	-91.9124858422647\\
1044	-87.518412999774\\
1045	-55.5463308869564\\
1046	-58.9082668369438\\
1047	-31.7260183024052\\
1048	-33.8616433512875\\
1049	-44.0759792994681\\
1050	-36.9765235414764\\
1051	-40.5709345621915\\
1053	-61.8032461516968\\
1054	-56.450535242652\\
1055	-85.5174398449001\\
1056	-67.975207564147\\
1057	-42.4751320719622\\
1058	-35.8654691149925\\
1059	-42.4643125524935\\
1060	-50.3641468261008\\
1061	-28.0351519891619\\
1062	-30.5215956210866\\
1063	-40.0530107644602\\
1064	-34.4196665379066\\
1065	-28.3006508271346\\
1066	-38.4638640785661\\
1067	-42.9370246088633\\
1068	-65.7780469692404\\
1069	-64.19464946178\\
1070	-73.3249521773532\\
1071	-69.3074508228208\\
1072	-74.3093961643895\\
1073	-61.3972773061437\\
1074	-60.0141083676531\\
1075	-54.7105916474916\\
1077	-54.6961676605658\\
1078	-88.290283524997\\
1079	-126.076581344702\\
1080	-127.355042047166\\
1081	-123.069558196723\\
1082	-78.2172801863373\\
1083	-109.605547144749\\
1084	-135.943113750605\\
1085	-133.532205926831\\
1086	-100.974524454012\\
1087	-137.988747112204\\
1088	-171.735438114002\\
1089	-133.129025028836\\
1090	-106.021345982353\\
1091	-74.3873796055591\\
1092	-57.5726059333738\\
1093	-50.7370528834508\\
1094	-60.4369012942095\\
1096	-25.048344332095\\
1097	-24.6525311434\\
1098	-31.9648163264883\\
1099	-55.6339292833559\\
1100	-50.7373911296916\\
1101	-55.8990226743774\\
1102	-67.5247761014837\\
1103	-71.6935309581004\\
1104	-97.6080833539313\\
1105	-90.6275681325999\\
1106	-97.3753275071733\\
1108	-64.3467948948414\\
1109	-79.1666581958841\\
1110	-57.0159642270489\\
1111	-53.0149921580389\\
1112	-63.0577151097793\\
1113	-64.1065769228908\\
1114	-43.670774685269\\
1115	-49.7636302365061\\
1116	-38.4004338513009\\
1118	-43.7140460085022\\
1119	-32.7373835232652\\
1120	-38.3603915525705\\
1121	-48.3504100671921\\
1122	-74.9986986896527\\
1123	-76.6680775097834\\
1124	-79.3071254305482\\
1125	-52.1955353095473\\
1126	-67.62355370878\\
1127	-108.219112945793\\
1128	-82.1673795026511\\
1129	-95.3222492355333\\
1130	-89.8531300096176\\
1131	-57.2498549495174\\
1132	-40.8098405375315\\
1133	-54.8614036148833\\
1134	-62.8327409948038\\
1135	-96.3926278893994\\
1136	-118.930351231181\\
1137	-112.150368026047\\
1138	-93.1552055489449\\
1139	-90.4235549229475\\
1140	-87.3207118234893\\
1141	-81.611356434908\\
1142	-79.8971655227976\\
1143	-53.8921029493065\\
1144	-52.1552162008193\\
1145	-45.4088253032307\\
1146	-42.1715317595979\\
1147	-48.1342335900661\\
1148	-63.6259988770291\\
1149	-97.6376766539079\\
1150	-78.3856578008447\\
1151	-56.0828076859284\\
1152	-48.8214009038925\\
1153	-49.4732777202032\\
1154	-29.4594893878391\\
1155	-21.3720995973099\\
1156	-24.0987501485638\\
1157	-45.3789599529503\\
1158	-34.1016799380454\\
1159	-38.8780074662207\\
1160	-39.0536035324478\\
1161	-40.072139918718\\
1162	-32.3920269089879\\
1163	-22.5968911530147\\
1164	-18.2175851688551\\
1165	-27.8473215670051\\
1166	-59.2126423506486\\
1167	-81.4341502968823\\
1168	-90.2681113635149\\
1169	-59.9550692575281\\
1170	-44.6195586174194\\
1171	-34.5854293095836\\
1172	-26.4284001122273\\
1173	-47.7283287822299\\
1174	-47.6357643016643\\
1175	-63.9817819176801\\
1176	-82.520950083237\\
1177	-134.855129947629\\
1178	-156.272435221408\\
1179	-125.425084986175\\
1180	-124.855443148185\\
1181	-77.4814917280871\\
1182	-89.2550809311435\\
1183	-90.9976333530237\\
1184	-98.3833905660842\\
1185	-79.2050367622207\\
1186	-70.235671619191\\
1187	-71.2383322214848\\
1188	-67.0896895920666\\
1189	-76.2517002658838\\
1190	-58.683209139464\\
1191	-92.093745304179\\
1192	-110.920226519225\\
1193	-83.1298136109538\\
1194	-50.614312119671\\
1195	-56.262599739018\\
1196	-90.3751520258029\\
1197	-108.919921916841\\
1198	-134.641796988022\\
1199	-135.384417248828\\
1200	-137.594433077074\\
1201	-107.759333761461\\
1202	-93.9219860040964\\
1203	-110.8644853865\\
1204	-120.908911607564\\
1205	-72.6066123461219\\
1206	-49.5603663680006\\
1207	-65.6144235149054\\
1208	-62.5810406798612\\
1209	-37.9599188072436\\
1210	-51.6192703139977\\
1211	-58.240817449861\\
1212	-51.5029227183709\\
1213	-38.1493849959838\\
1214	-52.7290690442321\\
1215	-45.4258875030966\\
1216	-55.1998797889648\\
1217	-80.0671635406293\\
1218	-73.0191570978852\\
1219	-54.0962277382523\\
1220	-55.448543851845\\
1221	-78.7250420741959\\
1222	-130.36814765603\\
1223	-99.4281861689142\\
1224	-58.6130719117214\\
1225	-46.2683345907394\\
1226	-53.6857257391441\\
1227	-49.4105661689796\\
1228	-39.0303939288281\\
1229	-42.2976020446101\\
1230	-50.3188352817092\\
1231	-61.427372960263\\
1232	-67.8679196018547\\
1233	-48.6077075555866\\
1234	-80.9006651635209\\
1235	-91.7165803076457\\
1236	-85.1895632560106\\
1237	-70.760214103504\\
1238	-72.2482663058745\\
1239	-97.462041093348\\
1240	-66.6738104743808\\
1241	-28.9755203062027\\
1242	-38.4684148639258\\
1243	-44.4219840348812\\
1244	-63.6666208010729\\
1245	-53.1098778985879\\
1246	-40.4684557909916\\
1247	-59.8924772625874\\
1248	-58.2634936695886\\
1249	-42.3073414427024\\
1250	-54.9676839945612\\
1251	-48.7001685977457\\
1252	-47.2838597380648\\
1253	-49.9283892892997\\
1254	-33.3515439760083\\
1255	-36.8700930596933\\
1256	-28.5074983781944\\
1257	-32.1092653689846\\
1258	-55.8022957083042\\
1259	-61.7280085428599\\
1260	-79.585693814359\\
1261	-107.416796142568\\
1262	-73.6684104684564\\
1263	-45.2064472647689\\
1264	-37.1400210874388\\
1265	-35.7766393378754\\
1266	-49.7348130435425\\
1267	-33.0499230949531\\
1268	-36.3357585519946\\
1269	-45.5439345779951\\
1270	-70.6709504455287\\
1271	-66.418388588527\\
1272	-89.3726984900359\\
1273	-61.1035076982416\\
1274	-57.0868574719848\\
1275	-31.1931193033051\\
1276	-30.9696385142763\\
1277	-22.1527482191925\\
1278	-23.0642924873175\\
1279	-28.111318182755\\
1280	-45.7169399281363\\
1281	-51.6925918912154\\
1282	-54.4355270529747\\
1283	-60.905202975667\\
1284	-95.6239479219555\\
1285	-88.8768295789077\\
1286	-64.8728614993847\\
1287	-76.8665750902501\\
1288	-76.985056564354\\
1289	-98.1543531511097\\
1290	-82.1360875195887\\
1291	-53.5024354580303\\
1292	-29.6159128396762\\
1293	-26.6540156488966\\
1294	-24.6119560152399\\
1295	-28.3689439703701\\
1296	-29.3751292452378\\
1297	-27.9490107727352\\
1298	-35.5318127096871\\
1299	-55.6056831679425\\
1300	-49.7965874377173\\
1301	-50.0596114350676\\
1302	-57.5260709995166\\
1303	-38.4885861061309\\
1304	-23.6780377251373\\
1305	-40.6424440672265\\
1306	-62.8436135718043\\
1307	-59.4489935862837\\
1308	-66.0381854558445\\
1309	-58.0578511328154\\
1310	-43.7465111429503\\
1311	-60.2501477198487\\
1312	-82.8550839030336\\
1313	-80.3289620012454\\
1314	-57.0641923037804\\
1315	-101.648607284942\\
1316	-77.6677678159908\\
1317	-75.4565857553291\\
1318	-82.562643460898\\
1319	-82.5546630267377\\
1320	-62.4353941733445\\
1321	-58.4127396603176\\
1322	-87.274275129407\\
1323	-123.922901823315\\
1324	-92.8246433121378\\
1325	-54.7524145281632\\
1326	-53.7417798384658\\
1327	-54.9099210255961\\
1328	-70.7062483386994\\
1329	-80.2031712563571\\
1330	-58.0195615623618\\
1331	-68.9141172794384\\
1332	-83.5857924338416\\
1333	-84.4630216595715\\
1334	-60.1701913509924\\
1335	-48.2098651700467\\
1336	-64.5840995517106\\
1337	-103.55038657646\\
1338	-96.1725622985098\\
1339	-91.9649494444602\\
1340	-57.8444868539673\\
1341	-51.4580948751727\\
1342	-55.3791245653056\\
1343	-32.9587357793846\\
1344	-21.2543702661919\\
1345	-17.4838528601751\\
1346	-15.9769342001816\\
1347	-37.0135793074921\\
1348	-51.5697982659779\\
1349	-60.9910441514739\\
1350	-47.7326874763733\\
1351	-49.2588733278658\\
1352	-57.5859220786963\\
1353	-36.8778389194672\\
1354	-38.7468223413252\\
1355	-38.6586058937728\\
1356	-31.4640998309114\\
1357	-34.780303678923\\
1358	-58.8747809683439\\
1359	-72.1868174999336\\
1360	-46.4533718836899\\
1361	-35.6324694360947\\
1362	-32.1151374430717\\
1363	-39.2072652609099\\
1364	-25.8874635226277\\
1365	-59.4324586855055\\
1366	-96.2405494781053\\
1367	-74.3963167820361\\
1368	-99.1019000501756\\
1369	-112.488387636476\\
1370	-113.055387675975\\
1371	-87.1729069002513\\
1372	-65.5835770472618\\
1373	-65.1404760967196\\
1374	-66.6654171385253\\
1376	-84.440260003953\\
1377	-86.5327676739321\\
1378	-112.72382633045\\
1379	-121.990709443312\\
1380	-143.23303726787\\
1381	-95.7763534587807\\
1382	-104.576825721329\\
1383	-115.169907039313\\
1384	-94.9475115893147\\
1385	-108.72266086733\\
1386	-90.2161070507159\\
1387	-53.2379528256706\\
1388	-33.5142562343246\\
1389	-36.1015870203532\\
1390	-53.2692396781424\\
1391	-47.9169869675634\\
1392	-33.609657761408\\
1393	-42.8879840402894\\
1394	-36.2281529192999\\
1395	-25.1297737355271\\
1396	-33.0639217318069\\
1397	-36.4319396846549\\
1398	-29.8684456609517\\
1399	-49.5101030702924\\
1400	-67.4502781725114\\
1401	-53.988071044367\\
1402	-88.9227104220186\\
1403	-128.500249735253\\
1404	-112.159327370518\\
1405	-132.278304356364\\
};
\addlegendentry{MPO prediction}

\end{axis}

\begin{axis}[%
width=6.159cm,
height=1.831cm,
at={(8.104cm,0cm)},
scale only axis,
xmin=1000,
xmax=1405,
xlabel style={font=\color{white!15!black}},
xlabel={Sample index},
ymin=-200,
ymax=0,
ylabel style={font=\color{white!15!black}},
ylabel={$y(t)$},
axis background/.style={fill=white},
title style={font=\bfseries},
title={C10: RMSE(OSA) = 7.4133, RMSE(MPO) = 11.5662},
legend style={legend cell align=left, align=left, draw=white!15!black}
]
\addplot [color=mycolor1, line width=2.0pt]
  table[row sep=crcr]{%
1006	-76.904\\
1007	-90.3320000000001\\
1008	-68.3589999999999\\
1009	-40.2829999999999\\
1010	-48.828\\
1011	-46.3869999999999\\
1012	-54.932\\
1013	-45.1659999999999\\
1014	-21.973\\
1015	-17.0899999999999\\
1016	-18.3109999999999\\
1018	-62.2560000000001\\
1019	-67.1389999999999\\
1020	-69.5799999999999\\
1022	-34.1800000000001\\
1023	-64.6970000000001\\
1024	-52.49\\
1025	-39.0630000000001\\
1026	-61.0350000000001\\
1027	-53.711\\
1028	-45.1659999999999\\
1029	-73.242\\
1030	-67.1389999999999\\
1031	-52.49\\
1032	-75.684\\
1033	-74.463\\
1034	-58.5940000000001\\
1036	-41.5039999999999\\
1037	-47.607\\
1038	-58.5940000000001\\
1039	-54.932\\
1040	-59.8140000000001\\
1041	-59.8140000000001\\
1042	-65.9180000000001\\
1043	-85.4490000000001\\
1044	-76.904\\
1045	-54.932\\
1046	-50.049\\
1047	-35.4000000000001\\
1048	-34.1800000000001\\
1049	-45.1659999999999\\
1050	-34.1800000000001\\
1051	-36.6210000000001\\
1052	-51.27\\
1053	-54.932\\
1054	-51.27\\
1055	-78.125\\
1056	-62.2560000000001\\
1057	-36.6210000000001\\
1058	-30.518\\
1059	-40.2829999999999\\
1060	-46.3869999999999\\
1061	-29.297\\
1062	-29.297\\
1063	-36.6210000000001\\
1065	-26.855\\
1066	-35.4000000000001\\
1067	-40.2829999999999\\
1068	-58.5940000000001\\
1069	-56.152\\
1070	-65.9180000000001\\
1071	-61.0350000000001\\
1072	-70.8009999999999\\
1073	-57.373\\
1074	-58.5940000000001\\
1075	-51.27\\
1077	-48.828\\
1079	-111.084\\
1080	-114.746\\
1081	-112.305\\
1082	-74.463\\
1084	-125.732\\
1085	-128.174\\
1086	-96.4359999999999\\
1087	-124.512\\
1088	-158.691\\
1089	-115.967\\
1090	-97.6559999999999\\
1091	-68.3589999999999\\
1092	-53.711\\
1093	-47.607\\
1094	-56.152\\
1095	-45.1659999999999\\
1096	-30.518\\
1097	-30.518\\
1098	-36.6210000000001\\
1099	-50.049\\
1101	-47.607\\
1102	-62.2560000000001\\
1103	-64.6970000000001\\
1104	-85.4490000000001\\
1105	-72.021\\
1106	-85.4490000000001\\
1107	-63.4770000000001\\
1108	-54.932\\
1109	-64.6970000000001\\
1110	-48.828\\
1111	-43.9449999999999\\
1112	-53.711\\
1113	-54.932\\
1114	-40.2829999999999\\
1115	-43.9449999999999\\
1116	-32.9590000000001\\
1118	-40.2829999999999\\
1119	-30.518\\
1120	-34.1800000000001\\
1121	-43.9449999999999\\
1122	-63.4770000000001\\
1123	-65.9180000000001\\
1124	-72.021\\
1125	-45.1659999999999\\
1126	-56.152\\
1127	-89.1110000000001\\
1128	-70.8009999999999\\
1129	-81.787\\
1130	-80.566\\
1131	-50.049\\
1132	-36.6210000000001\\
1133	-47.607\\
1134	-52.49\\
1135	-81.787\\
1136	-103.76\\
1137	-102.539\\
1138	-76.904\\
1139	-80.566\\
1140	-72.021\\
1141	-69.5799999999999\\
1142	-68.3589999999999\\
1143	-51.27\\
1145	-39.0630000000001\\
1146	-37.8420000000001\\
1147	-41.5039999999999\\
1148	-56.152\\
1149	-84.229\\
1150	-70.8009999999999\\
1151	-50.049\\
1152	-42.7249999999999\\
1153	-40.2829999999999\\
1154	-28.076\\
1155	-23.193\\
1156	-24.414\\
1157	-41.5039999999999\\
1158	-32.9590000000001\\
1159	-34.1800000000001\\
1160	-37.8420000000001\\
1161	-35.4000000000001\\
1162	-26.855\\
1163	-21.973\\
1164	-18.3109999999999\\
1165	-25.635\\
1166	-51.27\\
1167	-73.242\\
1168	-78.125\\
1169	-54.932\\
1170	-39.0630000000001\\
1172	-24.414\\
1173	-41.5039999999999\\
1174	-42.7249999999999\\
1175	-54.932\\
1176	-73.242\\
1177	-108.643\\
1178	-125.732\\
1179	-103.76\\
1180	-101.318\\
1181	-67.1389999999999\\
1182	-75.684\\
1183	-79.346\\
1184	-86.6700000000001\\
1185	-69.5799999999999\\
1186	-62.2560000000001\\
1187	-61.0350000000001\\
1188	-56.152\\
1189	-67.1389999999999\\
1190	-52.49\\
1191	-76.904\\
1192	-97.6559999999999\\
1194	-45.1659999999999\\
1195	-47.607\\
1196	-78.125\\
1197	-93.9939999999999\\
1198	-112.305\\
1199	-119.629\\
1200	-119.629\\
1201	-91.5530000000001\\
1202	-84.229\\
1203	-96.4359999999999\\
1204	-111.084\\
1205	-74.463\\
1206	-46.3869999999999\\
1207	-62.2560000000001\\
1208	-54.932\\
1209	-34.1800000000001\\
1210	-47.607\\
1211	-53.711\\
1212	-42.7249999999999\\
1213	-34.1800000000001\\
1214	-43.9449999999999\\
1215	-40.2829999999999\\
1216	-52.49\\
1217	-65.9180000000001\\
1218	-62.2560000000001\\
1219	-45.1659999999999\\
1220	-45.1659999999999\\
1221	-67.1389999999999\\
1222	-104.98\\
1223	-79.346\\
1224	-51.27\\
1225	-40.2829999999999\\
1226	-47.607\\
1227	-40.2829999999999\\
1228	-31.7380000000001\\
1229	-35.4000000000001\\
1230	-42.7249999999999\\
1231	-51.27\\
1232	-57.373\\
1233	-42.7249999999999\\
1234	-63.4770000000001\\
1235	-75.684\\
1236	-70.8009999999999\\
1237	-57.373\\
1238	-62.2560000000001\\
1239	-81.787\\
1240	-52.49\\
1241	-28.076\\
1242	-36.6210000000001\\
1243	-42.7249999999999\\
1244	-51.27\\
1245	-46.3869999999999\\
1246	-36.6210000000001\\
1247	-52.49\\
1248	-52.49\\
1249	-37.8420000000001\\
1250	-47.607\\
1251	-41.5039999999999\\
1252	-37.8420000000001\\
1253	-45.1659999999999\\
1254	-29.297\\
1255	-31.7380000000001\\
1257	-26.855\\
1258	-47.607\\
1259	-53.711\\
1260	-69.5799999999999\\
1261	-89.1110000000001\\
1263	-36.6210000000001\\
1264	-31.7380000000001\\
1265	-29.297\\
1266	-40.2829999999999\\
1267	-31.7380000000001\\
1268	-30.518\\
1269	-39.0630000000001\\
1270	-61.0350000000001\\
1271	-53.711\\
1272	-79.346\\
1273	-54.932\\
1274	-48.828\\
1275	-32.9590000000001\\
1276	-29.297\\
1277	-23.193\\
1278	-23.193\\
1279	-25.635\\
1280	-40.2829999999999\\
1281	-46.3869999999999\\
1283	-51.27\\
1284	-79.346\\
1285	-73.242\\
1286	-53.711\\
1287	-63.4770000000001\\
1288	-69.5799999999999\\
1289	-85.4490000000001\\
1290	-72.021\\
1291	-47.607\\
1292	-28.076\\
1293	-21.973\\
1294	-21.973\\
1295	-26.855\\
1296	-26.855\\
1297	-25.635\\
1298	-32.9590000000001\\
1299	-47.607\\
1300	-43.9449999999999\\
1301	-45.1659999999999\\
1302	-52.49\\
1303	-35.4000000000001\\
1304	-20.752\\
1305	-35.4000000000001\\
1306	-56.152\\
1307	-54.932\\
1308	-59.8140000000001\\
1309	-53.711\\
1310	-40.2829999999999\\
1311	-51.27\\
1312	-72.021\\
1313	-70.8009999999999\\
1314	-51.27\\
1315	-81.787\\
1316	-62.2560000000001\\
1317	-63.4770000000001\\
1318	-73.242\\
1319	-70.8009999999999\\
1320	-54.932\\
1321	-47.607\\
1323	-108.643\\
1324	-85.4490000000001\\
1325	-50.049\\
1326	-51.27\\
1327	-51.27\\
1328	-61.0350000000001\\
1329	-72.021\\
1330	-54.932\\
1331	-56.152\\
1332	-73.242\\
1333	-79.346\\
1334	-56.152\\
1335	-45.1659999999999\\
1336	-57.373\\
1337	-87.8910000000001\\
1338	-78.125\\
1339	-80.566\\
1340	-54.932\\
1341	-45.1659999999999\\
1342	-47.607\\
1343	-31.7380000000001\\
1344	-20.752\\
1345	-20.752\\
1346	-17.0899999999999\\
1347	-31.7380000000001\\
1348	-48.828\\
1349	-54.932\\
1350	-40.2829999999999\\
1351	-45.1659999999999\\
1352	-52.49\\
1353	-35.4000000000001\\
1355	-35.4000000000001\\
1356	-29.297\\
1357	-31.7380000000001\\
1358	-51.27\\
1359	-64.6970000000001\\
1360	-41.5039999999999\\
1361	-32.9590000000001\\
1362	-28.076\\
1363	-32.9590000000001\\
1364	-26.855\\
1365	-45.1659999999999\\
1366	-80.566\\
1367	-63.4770000000001\\
1369	-98.877\\
1370	-97.6559999999999\\
1371	-76.904\\
1372	-58.5940000000001\\
1373	-54.932\\
1374	-57.373\\
1375	-65.9180000000001\\
1376	-76.904\\
1377	-76.904\\
1378	-100.098\\
1379	-108.643\\
1380	-130.615\\
1381	-91.5530000000001\\
1382	-98.877\\
1383	-108.643\\
1384	-85.4490000000001\\
1385	-96.4359999999999\\
1386	-85.4490000000001\\
1387	-51.27\\
1388	-34.1800000000001\\
1389	-36.6210000000001\\
1390	-51.27\\
1391	-43.9449999999999\\
1392	-31.7380000000001\\
1393	-41.5039999999999\\
1394	-32.9590000000001\\
1395	-23.193\\
1396	-31.7380000000001\\
1397	-34.1800000000001\\
1398	-25.635\\
1400	-58.5940000000001\\
1401	-45.1659999999999\\
1402	-68.3589999999999\\
1403	-100.098\\
1404	-87.8910000000001\\
1405	-109.863\\
};
\addlegendentry{True output}

\addplot [color=mycolor2, dashed, line width=2.0pt]
  table[row sep=crcr]{%
1006	-79.926407842808\\
1007	-91.0901261788222\\
1008	-64.5059333488762\\
1009	-26.1175019066841\\
1010	-47.3096810284665\\
1011	-56.407105750196\\
1012	-56.1351685513298\\
1013	-52.2048238964771\\
1014	-10.4558418262607\\
1015	-9.6452484467186\\
1016	-15.513470775704\\
1017	-46.0537637125149\\
1018	-66.894160232235\\
1019	-67.1109087040682\\
1020	-75.5557970195978\\
1021	-53.4413343723786\\
1022	-26.2445523201925\\
1023	-86.473077841747\\
1024	-47.0084608310062\\
1025	-38.2999480960282\\
1026	-73.8416111307797\\
1027	-53.9640037064985\\
1028	-53.5930930911263\\
1029	-80.1801004309402\\
1030	-70.790984783557\\
1031	-52.9966596993227\\
1032	-81.0830058162794\\
1033	-78.4602612430317\\
1034	-62.7431275593678\\
1035	-50.7051138481684\\
1036	-43.2696618713987\\
1037	-50.5548199903417\\
1038	-63.1991819297848\\
1039	-56.9749114544461\\
1040	-58.5933394504204\\
1041	-62.5140670146413\\
1042	-64.8306049247778\\
1043	-97.7045225949646\\
1044	-81.2695683701925\\
1045	-50.5202292390604\\
1046	-52.1922467408799\\
1047	-23.3863730882413\\
1048	-33.6968720706468\\
1049	-46.9503996029223\\
1050	-37.877183109191\\
1051	-39.6444014886702\\
1052	-58.5436038421676\\
1053	-57.9787920755261\\
1054	-52.4076688580062\\
1055	-90.2570294195314\\
1057	-34.2142827872572\\
1058	-28.0350575760399\\
1059	-43.6825765015376\\
1060	-51.868188179393\\
1061	-27.2375981523026\\
1062	-27.2682388774974\\
1063	-44.4096959517224\\
1064	-34.3245411449147\\
1065	-29.8966727395657\\
1066	-38.0407836695861\\
1067	-41.0408579212174\\
1068	-66.693223140102\\
1069	-59.8401899096766\\
1070	-69.9239945923052\\
1071	-69.3917065216724\\
1072	-68.9638062504762\\
1073	-60.809140480935\\
1074	-58.6115934918425\\
1075	-54.2748422443731\\
1076	-52.2679321853673\\
1077	-56.747556140652\\
1078	-89.7066401514621\\
1079	-120.012311284395\\
1080	-119.622944639664\\
1081	-108.993907704005\\
1082	-58.8759480689946\\
1083	-120.929869292731\\
1084	-141.233701203731\\
1085	-124.285171493629\\
1086	-86.3962092109218\\
1087	-144.46703019715\\
1088	-181.430248484543\\
1089	-106.784343443303\\
1090	-90.7702854093372\\
1091	-44.5816271709839\\
1092	-35.0350687444541\\
1093	-41.6557754599132\\
1094	-59.1705896035949\\
1095	-41.4106672041496\\
1096	-20.1562987914808\\
1097	-23.5222504845203\\
1098	-38.2406945849154\\
1099	-59.3863108222379\\
1100	-51.253070532779\\
1101	-52.6223358972406\\
1102	-68.8480231990961\\
1103	-68.7885109657584\\
1104	-94.3841560405874\\
1105	-81.5784247125487\\
1106	-88.9303799117743\\
1107	-61.4425148671189\\
1108	-57.9646510928731\\
1109	-70.963039262665\\
1110	-50.424125003864\\
1111	-48.0625326239231\\
1112	-57.6803402910496\\
1113	-58.0547540627906\\
1114	-42.131799947168\\
1115	-44.9486063120046\\
1116	-37.1162873754381\\
1117	-38.8167846394867\\
1118	-40.3057185474199\\
1119	-32.4851916858711\\
1120	-37.9930517385781\\
1121	-48.75212430084\\
1122	-69.3623423494716\\
1123	-70.6981588492529\\
1124	-75.8059399123501\\
1125	-39.2225402014792\\
1126	-67.5632448838148\\
1127	-111.081285849685\\
1128	-72.3080961219289\\
1129	-82.7750316304309\\
1130	-86.4951143233779\\
1131	-41.6722396379764\\
1132	-30.2553605375476\\
1133	-52.4654270249825\\
1134	-60.1949794557543\\
1135	-94.4163101881381\\
1136	-107.50349354592\\
1137	-103.932999162574\\
1138	-73.7290054469743\\
1139	-85.1857340134059\\
1140	-70.8612312036864\\
1141	-75.5730277343025\\
1142	-75.127564255567\\
1143	-43.1767633338025\\
1144	-40.0073574842565\\
1145	-42.579537432722\\
1146	-37.9742434055356\\
1147	-46.0415870314121\\
1148	-65.2211651247758\\
1149	-92.6641491095711\\
1150	-68.3281488861878\\
1151	-50.6529097350681\\
1152	-41.5662032363305\\
1153	-43.5486325953657\\
1154	-22.7024429948349\\
1155	-16.1396391094559\\
1156	-26.4868883271456\\
1157	-47.0623622851094\\
1158	-36.2449694414456\\
1159	-37.3896382079661\\
1160	-41.6338817426981\\
1161	-38.3361896348292\\
1162	-28.3683948794985\\
1163	-19.3622449121744\\
1164	-17.3860749302344\\
1165	-29.7742061234633\\
1166	-57.5375893404132\\
1167	-73.7345906504618\\
1168	-84.2003762505076\\
1169	-54.2961306746022\\
1170	-32.6237775442519\\
1172	-19.0524072918342\\
1173	-51.1109887760895\\
1174	-45.9562854545577\\
1175	-59.7359098005297\\
1176	-79.1019299067166\\
1177	-127.210665749716\\
1178	-143.932787353846\\
1179	-105.997831974624\\
1180	-108.604036403086\\
1181	-48.2855194822937\\
1182	-80.7923648522831\\
1183	-82.0972425855998\\
1184	-97.5937311823193\\
1185	-63.0137865059639\\
1186	-64.0895034335217\\
1187	-66.3664555037801\\
1188	-58.2226630566124\\
1189	-70.4219150003671\\
1190	-50.7495887504842\\
1191	-95.8465043353842\\
1192	-101.244696459916\\
1194	-35.582236967326\\
1195	-45.2825513667276\\
1196	-94.3169342951774\\
1197	-99.4762279203514\\
1198	-117.654855743167\\
1199	-125.828976057689\\
1200	-129.115213767591\\
1201	-80.0931039188031\\
1202	-89.7020981305329\\
1203	-104.049968213744\\
1204	-115.31737395686\\
1205	-62.6842556571542\\
1206	-26.6749411792402\\
1207	-68.4892751961477\\
1208	-60.2889168814552\\
1209	-31.7234386622831\\
1210	-49.2926730975951\\
1211	-56.6021895711187\\
1212	-48.1705493342131\\
1213	-36.1136852823711\\
1214	-48.9586856251226\\
1215	-43.8419159474111\\
1216	-51.1284684050941\\
1217	-81.3001506395938\\
1218	-67.7703419783693\\
1219	-46.704837598762\\
1220	-50.2689750823088\\
1221	-73.6114270414637\\
1222	-121.032969767742\\
1224	-43.533534518353\\
1225	-25.6692167578351\\
1226	-49.9401459428743\\
1227	-47.9203805314376\\
1228	-32.8417111789265\\
1229	-35.9075448780802\\
1230	-50.4870605455951\\
1231	-54.5091434733897\\
1232	-61.9731097224299\\
1233	-39.7927774097573\\
1234	-76.8840942906988\\
1235	-89.3791201766137\\
1237	-61.1771403221858\\
1238	-67.3691663588611\\
1239	-95.4555230071092\\
1241	-13.0572576750928\\
1242	-32.2703009613699\\
1243	-43.0236570561731\\
1244	-58.5414035172782\\
1245	-51.1758400855872\\
1246	-37.8143976938652\\
1247	-60.4765548880937\\
1248	-56.2007996859829\\
1249	-41.251379906801\\
1250	-55.2868404087781\\
1251	-46.7135785762232\\
1252	-42.1175891014411\\
1253	-46.5401449649089\\
1254	-28.8252481609541\\
1255	-37.1008890766905\\
1256	-24.1137256130212\\
1257	-30.6254225589696\\
1258	-58.8655998937008\\
1259	-56.6297730689776\\
1260	-73.1627487994058\\
1261	-102.374962419485\\
1263	-28.06706473074\\
1264	-22.325529681633\\
1265	-30.2432981477407\\
1266	-49.638512637209\\
1267	-27.4357056057122\\
1268	-34.1220168069458\\
1269	-44.5730835242937\\
1270	-66.7423413346999\\
1271	-63.4330776334536\\
1272	-86.1851767008293\\
1273	-54.3940134866934\\
1274	-52.4331092286989\\
1275	-17.8669065922277\\
1276	-26.7520750875542\\
1277	-20.3727728480435\\
1278	-22.6695335115605\\
1279	-30.726232849591\\
1280	-47.5728826513125\\
1281	-46.5339757241952\\
1282	-53.4007512244634\\
1283	-56.7427895514666\\
1284	-96.5427034368115\\
1285	-79.4240109609752\\
1286	-55.1643375775004\\
1287	-71.5035016333843\\
1288	-75.0742826753228\\
1289	-93.1439980341133\\
1290	-75.9249077841955\\
1291	-38.9256494620506\\
1292	-11.7665773186868\\
1293	-14.246755086461\\
1294	-22.9205370169018\\
1295	-28.1215806069399\\
1296	-29.4717543040174\\
1297	-26.6735083188439\\
1298	-35.3988484045556\\
1299	-54.1785241622422\\
1300	-49.1097874226539\\
1301	-45.608918881084\\
1302	-56.6631576115797\\
1303	-33.4367400907968\\
1304	-20.3050495318116\\
1305	-44.0076872493364\\
1306	-65.1423439994799\\
1307	-53.2295960310892\\
1308	-65.3537525094555\\
1309	-53.6133233477776\\
1310	-39.9014069531961\\
1311	-59.0185543239388\\
1312	-82.4474691143839\\
1313	-76.4274262619851\\
1314	-53.8709211268897\\
1315	-94.2816324565483\\
1316	-68.798401468325\\
1317	-61.1027173268092\\
1318	-79.4512831781492\\
1319	-74.8177742518469\\
1320	-57.2101998231979\\
1321	-51.9453738275306\\
1322	-90.9812321191534\\
1323	-115.275669018575\\
1324	-87.5548562866456\\
1325	-36.6904967395601\\
1326	-44.2852309378998\\
1328	-68.4471956743637\\
1329	-73.9563099978222\\
1330	-57.0203781249108\\
1331	-58.254089539008\\
1332	-84.9025742368119\\
1333	-83.7902475345172\\
1334	-54.9429782049303\\
1335	-41.724569262753\\
1336	-64.0671514571011\\
1337	-100.172292190244\\
1338	-83.3306998324181\\
1339	-83.7487584452174\\
1340	-38.5686859206094\\
1341	-47.7319143033665\\
1342	-50.9705301528627\\
1343	-25.9306834383344\\
1344	-15.8940424412467\\
1345	-13.9541945466083\\
1346	-16.9810396011883\\
1347	-38.5977881047431\\
1348	-49.2762788263801\\
1349	-57.962146072487\\
1350	-46.0953237191495\\
1351	-49.6014041889882\\
1352	-58.2352701744028\\
1353	-32.5125081930764\\
1354	-37.4475376455298\\
1355	-38.9022698807112\\
1356	-28.8427614162197\\
1357	-38.2635701122124\\
1358	-59.8052783127969\\
1359	-66.9249123870086\\
1360	-38.3828122050488\\
1361	-32.4880148777627\\
1362	-30.7636002041477\\
1363	-39.8411892408017\\
1364	-23.8849852829269\\
1366	-89.7300979451138\\
1367	-63.6938461347993\\
1368	-93.3406750595664\\
1369	-109.593567405679\\
1370	-102.912090871803\\
1371	-75.1863666117124\\
1372	-50.0100879047457\\
1373	-62.0742271387869\\
1374	-56.8618915938146\\
1375	-70.4805304865674\\
1376	-81.3058735024163\\
1377	-77.8316592558958\\
1378	-112.658922810053\\
1379	-121.7867499215\\
1380	-132.772765420428\\
1381	-82.3349028004841\\
1382	-107.942057299831\\
1383	-122.573483175184\\
1384	-78.5785277870177\\
1385	-106.166227847068\\
1386	-85.3974443109103\\
1387	-34.1752115592597\\
1388	-17.1958845559548\\
1389	-28.1950685775107\\
1390	-61.1159344227244\\
1391	-43.8151205167094\\
1392	-30.222012787267\\
1393	-42.1297403563681\\
1394	-35.5368407509889\\
1395	-25.9923958532659\\
1396	-31.7911713785606\\
1397	-38.3941593599191\\
1398	-28.4451598576966\\
1399	-52.5894890718432\\
1400	-63.0991783409436\\
1401	-45.3460820430855\\
1402	-81.7600526592501\\
1403	-121.32311693172\\
1404	-90.9358563839435\\
1405	-127.648059995245\\
};
\addlegendentry{OSA predition}

\addplot [color=mycolor3, dotted, line width=2.0pt]
  table[row sep=crcr]{%
1006	-76.904\\
1007	-90.3320000000001\\
1008	-68.3589999999999\\
1009	-40.2829999999999\\
1010	-47.309681028468\\
1011	-56.0096802179141\\
1012	-58.1716468524949\\
1013	-55.4479135578713\\
1014	-11.0088049074693\\
1015	-9.23423216849369\\
1016	-9.7185291057483\\
1017	-37.6178520948404\\
1018	-57.6214539897251\\
1019	-61.4209224575143\\
1020	-71.6702317342551\\
1021	-51.9551088363276\\
1022	-27.2057123691923\\
1023	-84.6522282668002\\
1024	-49.0678403245313\\
1025	-43.1210755527682\\
1026	-79.8271486060687\\
1027	-60.1929905061336\\
1028	-60.2787661657846\\
1029	-92.4326121707745\\
1030	-80.9327483180166\\
1031	-61.5554981935788\\
1032	-93.5188574499462\\
1033	-88.3344578715669\\
1034	-71.6799601845623\\
1035	-58.5531415813277\\
1036	-50.6555928784842\\
1037	-57.9507512928235\\
1038	-71.3976305812162\\
1039	-64.6180686589189\\
1040	-66.5730177460687\\
1041	-69.5846748329291\\
1042	-71.4604678073338\\
1043	-105.005465620792\\
1044	-87.5368023313736\\
1045	-58.1896909442448\\
1046	-57.1002130815586\\
1047	-26.2293165460137\\
1048	-33.8560718804622\\
1049	-45.7977824798406\\
1050	-38.0434815823162\\
1051	-40.1186570728275\\
1052	-60.7036710440059\\
1053	-62.1708939080133\\
1054	-57.354611699127\\
1055	-98.0943266631057\\
1057	-40.64480596206\\
1058	-30.1952896841319\\
1059	-47.113857481138\\
1060	-55.0320022531043\\
1061	-31.5976793350337\\
1062	-30.0706965916604\\
1063	-47.5914929203204\\
1064	-38.3949088667869\\
1065	-34.7911876283299\\
1066	-43.9559862922868\\
1067	-47.5522205505958\\
1068	-74.9193506263298\\
1069	-67.8903617815117\\
1070	-79.4635440686561\\
1071	-79.0611475864725\\
1072	-79.7181164847298\\
1073	-69.7188629729294\\
1074	-66.504602952326\\
1075	-61.6944881958907\\
1076	-58.6910572173315\\
1077	-63.9112399715943\\
1078	-100.798277845529\\
1079	-132.743920337869\\
1080	-132.080646769783\\
1081	-120.29086232358\\
1082	-63.9898807121779\\
1083	-127.708011226842\\
1084	-143.520818374054\\
1085	-128.791584933237\\
1086	-91.5275808199574\\
1087	-151.861391856126\\
1088	-179.102811322071\\
1089	-108.686085264161\\
1090	-96.2562842662217\\
1091	-41.9614029155773\\
1092	-30.8353871067502\\
1093	-26.7410035293358\\
1094	-43.0124765711694\\
1095	-29.9869828559226\\
1096	-14.0009061463654\\
1097	-13.1607068979558\\
1098	-25.4969325988368\\
1099	-44.6253722075157\\
1100	-42.502073730645\\
1101	-46.4124297295723\\
1102	-64.3983372106163\\
1103	-67.1009926324227\\
1104	-94.5071522242172\\
1105	-83.8290084479136\\
1106	-94.3210637958985\\
1107	-66.5383907165512\\
1108	-62.1658276016942\\
1109	-76.0143378156477\\
1110	-55.6035086674431\\
1111	-53.4948498784893\\
1112	-64.5112807704422\\
1113	-65.2507142383276\\
1114	-48.0729105542716\\
1115	-51.5245784939409\\
1116	-42.4309347084604\\
1117	-44.9485216861517\\
1118	-46.6696347465365\\
1119	-37.1614841794863\\
1120	-43.3015933176482\\
1121	-55.177259778435\\
1122	-77.8804611520065\\
1123	-79.8477006740384\\
1124	-85.9426067217862\\
1125	-45.202298642289\\
1126	-75.2608857360433\\
1127	-123.177959144447\\
1128	-81.1048024331449\\
1129	-97.9119140462537\\
1130	-99.9069728935601\\
1131	-50.3757400039215\\
1132	-34.8581218139709\\
1133	-55.8226690709478\\
1134	-63.4395075762905\\
1135	-98.388943730133\\
1136	-116.463857339535\\
1137	-113.056641373945\\
1138	-79.3193891937224\\
1139	-91.6161344112149\\
1140	-75.0145749290873\\
1141	-80.5452327109458\\
1142	-79.8328075044778\\
1143	-49.2893473649344\\
1144	-42.6985803149182\\
1145	-42.6932857461545\\
1146	-39.1675984309331\\
1147	-46.9143558828307\\
1148	-67.9743558126397\\
1149	-97.7891868829299\\
1150	-74.1867721976514\\
1151	-55.1025717329462\\
1152	-44.5974976980801\\
1153	-46.7586754830329\\
1154	-24.6667124197786\\
1155	-17.0724424039968\\
1156	-24.2135423460034\\
1157	-44.8340081814297\\
1158	-36.1189946703857\\
1159	-38.6070243173938\\
1160	-44.2256589953397\\
1161	-41.8320508595662\\
1163	-22.4223725112788\\
1164	-18.9844959248246\\
1165	-31.4861104546674\\
1166	-60.0950501609339\\
1167	-77.7775867582927\\
1168	-89.8800887684731\\
1169	-57.8537929490055\\
1170	-36.7913810170758\\
1171	-25.3951611497114\\
1172	-17.1780984592301\\
1173	-46.7969737879769\\
1174	-43.9775961936496\\
1175	-58.509937002\\
1176	-81.85466153794\\
1177	-131.057717079348\\
1178	-148.698078188954\\
1179	-111.969487383654\\
1180	-120.131825219782\\
1181	-50.9848659132499\\
1182	-85.4135457684995\\
1183	-82.3894549347979\\
1184	-98.428402060732\\
1185	-64.2087320562914\\
1186	-67.3263969192521\\
1187	-67.356748583705\\
1188	-61.7369735075181\\
1189	-74.0534390402522\\
1190	-54.336273318278\\
1191	-100.93997675666\\
1192	-108.037010090719\\
1193	-75.9919659534607\\
1194	-35.7193880080486\\
1195	-46.0124658559341\\
1196	-93.4781518899354\\
1197	-97.9383075590406\\
1198	-120.915146474704\\
1199	-132.238051797591\\
1200	-133.702918704699\\
1201	-83.7883638104474\\
1202	-94.5373394929186\\
1204	-118.514321099548\\
1205	-64.7794439369179\\
1206	-25.4210807558306\\
1207	-61.4584505451976\\
1208	-53.5619158970808\\
1209	-33.2199989202018\\
1210	-46.8035912505948\\
1211	-55.8051089631474\\
1212	-48.2871708845323\\
1213	-38.1009125310161\\
1214	-51.8785095594355\\
1215	-47.7305499323411\\
1216	-56.4373757825056\\
1217	-87.8550496891307\\
1218	-74.9714504928299\\
1219	-55.9422760430855\\
1220	-58.6470298298229\\
1221	-85.4161613148231\\
1222	-136.063434573951\\
1223	-89.2303004630801\\
1224	-52.1226609722773\\
1225	-26.6159394052988\\
1226	-49.6843496115243\\
1227	-45.4620441732029\\
1228	-35.2994885807946\\
1229	-37.3754154375972\\
1230	-53.3209231925243\\
1231	-59.0605395951252\\
1232	-67.5010056842232\\
1233	-44.1682041412455\\
1234	-83.550291898868\\
1235	-98.0738994154613\\
1237	-71.2865123796805\\
1238	-79.0705203954833\\
1239	-109.314255856183\\
1241	-19.3785503921561\\
1242	-34.5980498723229\\
1243	-43.4870368937964\\
1244	-56.5629324606025\\
1245	-51.686365497196\\
1246	-40.4363256292804\\
1247	-64.3762459057086\\
1248	-61.44771215233\\
1249	-47.0395581912494\\
1250	-62.7815828797102\\
1251	-54.8354052819489\\
1252	-50.9910605374594\\
1253	-56.601138637486\\
1254	-35.8430976240343\\
1255	-44.2880829983842\\
1256	-30.2380988057253\\
1257	-35.5276872917213\\
1258	-67.3307228888441\\
1259	-64.9001845773853\\
1260	-84.2252921108054\\
1261	-118.445246508335\\
1262	-73.1937936560178\\
1263	-36.5390350488879\\
1264	-23.6749033904098\\
1265	-30.2641607304463\\
1266	-48.6067415380467\\
1267	-30.1585568445041\\
1268	-35.2374614122396\\
1269	-47.3136625009454\\
1270	-70.2277152121292\\
1271	-68.709398987447\\
1272	-95.8060315149141\\
1273	-61.9358023358452\\
1274	-59.7671217713421\\
1275	-21.2182057327218\\
1276	-27.2411702617796\\
1277	-16.3860377482079\\
1278	-21.1021231835323\\
1279	-27.0906237864199\\
1280	-44.7657920732945\\
1281	-46.9916801120155\\
1282	-54.8741256168159\\
1283	-59.5144930126257\\
1284	-100.844813617382\\
1285	-86.1908020423202\\
1286	-63.66851572222\\
1287	-81.9710841148903\\
1288	-85.911316077412\\
1289	-105.619778379659\\
1290	-86.2667049511431\\
1291	-46.4725860460624\\
1292	-12.2406399379518\\
1293	-8.62923124016061\\
1294	-15.628739494533\\
1295	-21.0268936584796\\
1296	-23.1705965220092\\
1297	-22.5923540344997\\
1298	-31.9200263374066\\
1299	-51.1905203558642\\
1300	-48.7931113561854\\
1301	-47.4182668314372\\
1302	-59.232333452881\\
1303	-35.737802213431\\
1304	-22.2330539804368\\
1306	-69.1832946496099\\
1307	-59.2236974031971\\
1308	-72.6646374411175\\
1310	-45.7698577369677\\
1311	-64.9930170102366\\
1312	-90.5556335271824\\
1313	-85.2048839888439\\
1314	-62.1291018168868\\
1315	-108.528157382368\\
1316	-78.3099964961557\\
1317	-73.5994807785673\\
1318	-92.9968802512335\\
1319	-85.7646316341486\\
1320	-67.0583961052275\\
1321	-60.5724171033596\\
1322	-104.547439142113\\
1323	-129.747824519237\\
1324	-98.2833304132225\\
1325	-40.0652695651172\\
1326	-47.1739739239581\\
1327	-55.0585925323937\\
1328	-67.1279814071634\\
1329	-74.5452972011285\\
1330	-59.0887708624646\\
1331	-60.8263952829197\\
1332	-88.4383932314474\\
1333	-88.9856420520894\\
1334	-61.0243579707571\\
1335	-45.24862610234\\
1336	-68.3838229134792\\
1337	-105.65686958407\\
1338	-88.7835827162453\\
1339	-92.4070698382031\\
1340	-42.3413899244588\\
1341	-49.2175820393586\\
1342	-50.1895129497082\\
1343	-30.1777453316399\\
1344	-14.4210199448921\\
1345	-11.668301997895\\
1346	-12.8182587696947\\
1347	-32.7734266068087\\
1349	-54.563584392972\\
1350	-44.5529791608315\\
1351	-50.1200997572946\\
1352	-60.2192843521777\\
1353	-35.3831988373786\\
1354	-40.0855286002509\\
1355	-41.364756123447\\
1356	-32.3724622486664\\
1357	-41.3889942458475\\
1358	-65.9873976170884\\
1359	-74.5492603625473\\
1360	-43.8818965569358\\
1361	-35.7788229311616\\
1362	-33.5283055328457\\
1363	-43.3061963973933\\
1364	-28.2923291807454\\
1365	-61.4001853417883\\
1366	-101.567712828127\\
1367	-71.7190461623327\\
1368	-106.750463344503\\
1369	-123.945263705645\\
1370	-114.987520925376\\
1372	-56.8337146951619\\
1373	-67.3943068230333\\
1374	-61.2743003400062\\
1375	-75.2748898603124\\
1376	-87.4956645347636\\
1377	-83.9795691457816\\
1378	-120.897157360328\\
1379	-129.557939241828\\
1380	-142.418805439797\\
1381	-88.6840010177425\\
1382	-116.126725045385\\
1383	-127.477631767074\\
1384	-85.0019631439498\\
1385	-115.068309792755\\
1386	-89.9332637501675\\
1387	-42.0588263117252\\
1388	-12.3985312127802\\
1389	-20.3028660153579\\
1390	-48.1034233223422\\
1391	-36.3335844260125\\
1392	-26.5913109379419\\
1393	-36.7979828379378\\
1394	-32.2160737223353\\
1395	-24.1841107839439\\
1397	-37.9373647795107\\
1398	-29.2743706927745\\
1399	-54.4646389543339\\
1400	-68.4557792595403\\
1401	-51.1093827925692\\
1403	-133.849304624378\\
1404	-99.9350923832749\\
1405	-148.485873471472\\
};
\addlegendentry{MPO prediction}

\end{axis}
\end{tikzpicture}%
	\caption{Validation results for the model estimated using OLS. Compression direction d3.}\label{fig:spgl_hs}
\end{figure}
\begin{figure}[!h]
	\definecolor{mycolor1}{rgb}{0.00000,0.44700,0.74100}%
	\definecolor{mycolor2}{rgb}{0.85000,0.32500,0.09800}%
	\centering
	% This file was created by matlab2tikz.
%
\definecolor{mycolor1}{rgb}{0.00000,0.44700,0.74100}%
\definecolor{mycolor2}{rgb}{0.85000,0.32500,0.09800}%
\definecolor{mycolor3}{rgb}{0.92900,0.69400,0.12500}%
%
\begin{tikzpicture}

\begin{axis}[%
width=6.159cm,
height=1.831cm,
at={(0cm,10.169cm)},
scale only axis,
xmin=1000,
xmax=2000,
xlabel style={font=\color{white!15!black}},
xlabel={Sample index},
ymin=-68.7387310771141,
ymax=0,
ylabel style={font=\color{white!15!black}},
ylabel={$y(t)$},
axis background/.style={fill=white},
title style={font=\bfseries},
title={C1: RMSE(OSA) = 2.6487, RMSE(MPO) = 5.431},
legend style={legend cell align=left, align=left, draw=white!15!black}
]
\addplot [color=mycolor1, line width=2.0pt]
  table[row sep=crcr]{%
1006	-28.076\\
1007	-32.9590000000001\\
1008	-25.635\\
1009	-14.6479999999999\\
1010	-20.752\\
1011	-17.0899999999999\\
1012	-18.3109999999999\\
1013	-20.752\\
1014	-7.32400000000007\\
1015	-4.88300000000004\\
1016	-4.88300000000004\\
1017	-13.4280000000001\\
1018	-23.193\\
1020	-25.635\\
1021	-19.5309999999999\\
1022	-12.2070000000001\\
1023	-24.414\\
1025	-14.6479999999999\\
1026	-20.752\\
1027	-18.3109999999999\\
1028	-17.0899999999999\\
1029	-29.297\\
1030	-25.635\\
1031	-18.3109999999999\\
1032	-28.076\\
1033	-28.076\\
1034	-21.973\\
1035	-18.3109999999999\\
1036	-15.8689999999999\\
1037	-15.8689999999999\\
1038	-21.973\\
1041	-21.973\\
1042	-23.193\\
1043	-30.518\\
1044	-29.297\\
1045	-19.5309999999999\\
1046	-18.3109999999999\\
1047	-10.9860000000001\\
1048	-14.6479999999999\\
1049	-15.8689999999999\\
1050	-12.2070000000001\\
1051	-13.4280000000001\\
1052	-18.3109999999999\\
1053	-19.5309999999999\\
1054	-18.3109999999999\\
1055	-29.297\\
1056	-21.973\\
1057	-12.2070000000001\\
1058	-12.2070000000001\\
1059	-13.4280000000001\\
1060	-17.0899999999999\\
1061	-9.76600000000008\\
1062	-10.9860000000001\\
1063	-14.6479999999999\\
1064	-9.76600000000008\\
1065	-7.32400000000007\\
1066	-13.4280000000001\\
1067	-13.4280000000001\\
1068	-21.973\\
1069	-20.752\\
1070	-25.635\\
1071	-21.973\\
1072	-25.635\\
1073	-20.752\\
1074	-20.752\\
1075	-18.3109999999999\\
1076	-18.3109999999999\\
1077	-17.0899999999999\\
1078	-30.518\\
1079	-41.5039999999999\\
1080	-41.5039999999999\\
1081	-42.7249999999999\\
1082	-28.076\\
1083	-37.8420000000001\\
1084	-46.3869999999999\\
1085	-46.3869999999999\\
1086	-35.4000000000001\\
1087	-47.607\\
1088	-58.5940000000001\\
1089	-43.9449999999999\\
1091	-24.414\\
1093	-17.0899999999999\\
1094	-21.973\\
1096	-12.2070000000001\\
1097	-9.76600000000008\\
1098	-12.2070000000001\\
1099	-18.3109999999999\\
1101	-18.3109999999999\\
1102	-21.973\\
1103	-23.193\\
1104	-30.518\\
1105	-28.076\\
1106	-30.518\\
1108	-20.752\\
1109	-24.414\\
1110	-18.3109999999999\\
1111	-13.4280000000001\\
1112	-19.5309999999999\\
1113	-20.752\\
1114	-12.2070000000001\\
1115	-15.8689999999999\\
1116	-12.2070000000001\\
1117	-13.4280000000001\\
1118	-13.4280000000001\\
1119	-9.76600000000008\\
1120	-10.9860000000001\\
1122	-23.193\\
1123	-25.635\\
1124	-26.855\\
1125	-15.8689999999999\\
1126	-20.752\\
1127	-32.9590000000001\\
1128	-24.414\\
1129	-28.076\\
1130	-29.297\\
1131	-18.3109999999999\\
1132	-12.2070000000001\\
1133	-17.0899999999999\\
1134	-18.3109999999999\\
1135	-30.518\\
1136	-36.6210000000001\\
1137	-36.6210000000001\\
1138	-28.076\\
1139	-28.076\\
1140	-25.635\\
1142	-25.635\\
1144	-15.8689999999999\\
1146	-13.4280000000001\\
1147	-13.4280000000001\\
1148	-20.752\\
1149	-31.7380000000001\\
1150	-26.855\\
1151	-17.0899999999999\\
1152	-15.8689999999999\\
1153	-15.8689999999999\\
1154	-9.76600000000008\\
1155	-7.32400000000007\\
1156	-7.32400000000007\\
1157	-15.8689999999999\\
1158	-10.9860000000001\\
1159	-14.6479999999999\\
1160	-14.6479999999999\\
1161	-13.4280000000001\\
1163	-6.10400000000004\\
1164	-4.88300000000004\\
1165	-7.32400000000007\\
1166	-18.3109999999999\\
1167	-26.855\\
1168	-28.076\\
1170	-13.4280000000001\\
1171	-10.9860000000001\\
1172	-6.10400000000004\\
1173	-12.2070000000001\\
1174	-14.6479999999999\\
1175	-18.3109999999999\\
1176	-26.855\\
1177	-39.0630000000001\\
1178	-47.607\\
1180	-37.8420000000001\\
1181	-28.076\\
1182	-30.518\\
1183	-31.7380000000001\\
1184	-34.1800000000001\\
1185	-24.414\\
1186	-23.193\\
1187	-23.193\\
1188	-21.973\\
1189	-23.193\\
1190	-18.3109999999999\\
1191	-30.518\\
1192	-37.8420000000001\\
1194	-18.3109999999999\\
1195	-18.3109999999999\\
1196	-30.518\\
1199	-45.1659999999999\\
1200	-45.1659999999999\\
1201	-34.1800000000001\\
1202	-32.9590000000001\\
1203	-36.6210000000001\\
1204	-41.5039999999999\\
1206	-17.0899999999999\\
1207	-23.193\\
1208	-20.752\\
1209	-13.4280000000001\\
1210	-18.3109999999999\\
1211	-19.5309999999999\\
1212	-14.6479999999999\\
1213	-13.4280000000001\\
1214	-15.8689999999999\\
1215	-13.4280000000001\\
1216	-17.0899999999999\\
1217	-25.635\\
1218	-25.635\\
1219	-15.8689999999999\\
1220	-14.6479999999999\\
1221	-24.414\\
1222	-40.2829999999999\\
1223	-29.297\\
1224	-20.752\\
1225	-14.6479999999999\\
1226	-18.3109999999999\\
1227	-15.8689999999999\\
1228	-12.2070000000001\\
1229	-13.4280000000001\\
1232	-20.752\\
1233	-15.8689999999999\\
1234	-23.193\\
1235	-29.297\\
1236	-25.635\\
1237	-20.752\\
1238	-23.193\\
1239	-30.518\\
1240	-21.973\\
1241	-8.54500000000007\\
1242	-13.4280000000001\\
1243	-13.4280000000001\\
1244	-17.0899999999999\\
1245	-17.0899999999999\\
1246	-13.4280000000001\\
1247	-17.0899999999999\\
1248	-19.5309999999999\\
1249	-13.4280000000001\\
1250	-15.8689999999999\\
1251	-15.8689999999999\\
1252	-12.2070000000001\\
1253	-15.8689999999999\\
1254	-9.76600000000008\\
1255	-10.9860000000001\\
1256	-8.54500000000007\\
1257	-8.54500000000007\\
1259	-20.752\\
1260	-24.414\\
1261	-35.4000000000001\\
1263	-13.4280000000001\\
1265	-10.9860000000001\\
1266	-13.4280000000001\\
1267	-10.9860000000001\\
1268	-12.2070000000001\\
1269	-14.6479999999999\\
1270	-24.414\\
1271	-21.973\\
1272	-26.855\\
1273	-23.193\\
1274	-17.0899999999999\\
1275	-13.4280000000001\\
1278	-6.10400000000004\\
1279	-9.76600000000008\\
1280	-14.6479999999999\\
1281	-18.3109999999999\\
1282	-18.3109999999999\\
1283	-19.5309999999999\\
1284	-29.297\\
1285	-25.635\\
1286	-18.3109999999999\\
1287	-24.414\\
1288	-24.414\\
1289	-31.7380000000001\\
1290	-26.855\\
1291	-17.0899999999999\\
1292	-9.76600000000008\\
1293	-6.10400000000004\\
1294	-7.32400000000007\\
1295	-9.76600000000008\\
1296	-8.54500000000007\\
1297	-8.54500000000007\\
1298	-12.2070000000001\\
1299	-17.0899999999999\\
1300	-15.8689999999999\\
1301	-15.8689999999999\\
1302	-19.5309999999999\\
1303	-13.4280000000001\\
1304	-4.88300000000004\\
1305	-9.76600000000008\\
1306	-21.973\\
1307	-19.5309999999999\\
1308	-21.973\\
1310	-14.6479999999999\\
1311	-19.5309999999999\\
1312	-25.635\\
1313	-25.635\\
1314	-19.5309999999999\\
1315	-26.855\\
1316	-26.855\\
1317	-23.193\\
1318	-28.076\\
1319	-26.855\\
1320	-20.752\\
1321	-19.5309999999999\\
1322	-29.297\\
1323	-40.2829999999999\\
1324	-30.518\\
1325	-18.3109999999999\\
1326	-17.0899999999999\\
1327	-18.3109999999999\\
1328	-21.973\\
1329	-26.855\\
1330	-19.5309999999999\\
1331	-19.5309999999999\\
1332	-26.855\\
1333	-28.076\\
1334	-20.752\\
1335	-14.6479999999999\\
1336	-20.752\\
1337	-34.1800000000001\\
1338	-30.518\\
1339	-31.7380000000001\\
1340	-21.973\\
1341	-17.0899999999999\\
1342	-17.0899999999999\\
1343	-10.9860000000001\\
1344	-6.10400000000004\\
1345	-6.10400000000004\\
1346	-4.88300000000004\\
1347	-10.9860000000001\\
1348	-19.5309999999999\\
1349	-18.3109999999999\\
1350	-15.8689999999999\\
1351	-14.6479999999999\\
1352	-20.752\\
1353	-14.6479999999999\\
1354	-9.76600000000008\\
1355	-13.4280000000001\\
1356	-8.54500000000007\\
1357	-10.9860000000001\\
1358	-18.3109999999999\\
1359	-23.193\\
1360	-17.0899999999999\\
1361	-9.76600000000008\\
1363	-12.2070000000001\\
1364	-9.76600000000008\\
1365	-13.4280000000001\\
1366	-31.7380000000001\\
1367	-25.635\\
1369	-37.8420000000001\\
1370	-36.6210000000001\\
1371	-26.855\\
1372	-20.752\\
1373	-18.3109999999999\\
1374	-21.973\\
1375	-23.193\\
1376	-29.297\\
1377	-29.297\\
1378	-36.6210000000001\\
1379	-42.7249999999999\\
1380	-46.3869999999999\\
1381	-37.8420000000001\\
1382	-35.4000000000001\\
1383	-40.2829999999999\\
1384	-31.7380000000001\\
1385	-34.1800000000001\\
1386	-31.7380000000001\\
1387	-18.3109999999999\\
1388	-12.2070000000001\\
1389	-13.4280000000001\\
1390	-18.3109999999999\\
1391	-17.0899999999999\\
1392	-8.54500000000007\\
1393	-14.6479999999999\\
1394	-13.4280000000001\\
1395	-6.10400000000004\\
1396	-10.9860000000001\\
1397	-12.2070000000001\\
1398	-7.32400000000007\\
1399	-18.3109999999999\\
1400	-20.752\\
1401	-14.6479999999999\\
1402	-23.193\\
1403	-37.8420000000001\\
1404	-32.9590000000001\\
1405	-37.8420000000001\\
1406	-32.9590000000001\\
1407	-29.297\\
1408	-20.752\\
1409	-14.6479999999999\\
1410	-17.0899999999999\\
1411	-13.4280000000001\\
1412	-10.9860000000001\\
1413	-14.6479999999999\\
1414	-13.4280000000001\\
1415	-3.66200000000003\\
1416	-7.32400000000007\\
1417	-15.8689999999999\\
1418	-20.752\\
1420	-23.193\\
1421	-21.973\\
1422	-26.855\\
1424	-17.0899999999999\\
1425	-17.0899999999999\\
1426	-14.6479999999999\\
1427	-20.752\\
1428	-18.3109999999999\\
1429	-13.4280000000001\\
1430	-18.3109999999999\\
1431	-26.855\\
1432	-20.752\\
1433	-15.8689999999999\\
1434	-13.4280000000001\\
1435	-14.6479999999999\\
1436	-12.2070000000001\\
1437	-8.54500000000007\\
1438	-14.6479999999999\\
1439	-19.5309999999999\\
1440	-18.3109999999999\\
1441	-15.8689999999999\\
1442	-19.5309999999999\\
1443	-18.3109999999999\\
1444	-10.9860000000001\\
1445	-10.9860000000001\\
1446	-21.973\\
1447	-30.518\\
1448	-29.297\\
1449	-23.193\\
1450	-21.973\\
1451	-19.5309999999999\\
1452	-18.3109999999999\\
1453	-15.8689999999999\\
1454	-19.5309999999999\\
1455	-17.0899999999999\\
1456	-13.4280000000001\\
1457	-17.0899999999999\\
1459	-21.973\\
1460	-25.635\\
1461	-25.635\\
1462	-32.9590000000001\\
1463	-41.5039999999999\\
1464	-45.1659999999999\\
1465	-31.7380000000001\\
1466	-17.0899999999999\\
1467	-13.4280000000001\\
1468	-8.54500000000007\\
1470	-13.4280000000001\\
1471	-17.0899999999999\\
1472	-18.3109999999999\\
1473	-13.4280000000001\\
1474	-19.5309999999999\\
1475	-24.414\\
1477	-26.855\\
1478	-39.0630000000001\\
1479	-35.4000000000001\\
1480	-25.635\\
1481	-20.752\\
1482	-20.752\\
1483	-19.5309999999999\\
1484	-23.193\\
1486	-15.8689999999999\\
1487	-17.0899999999999\\
1488	-10.9860000000001\\
1489	-14.6479999999999\\
1490	-28.076\\
1491	-19.5309999999999\\
1492	-15.8689999999999\\
1493	-34.1800000000001\\
1495	-21.973\\
1496	-34.1800000000001\\
1497	-31.7380000000001\\
1498	-20.752\\
1499	-15.8689999999999\\
1500	-13.4280000000001\\
1501	-23.193\\
1502	-35.4000000000001\\
1503	-37.8420000000001\\
1504	-36.6210000000001\\
1505	-29.297\\
1506	-20.752\\
1507	-15.8689999999999\\
1508	-14.6479999999999\\
1509	-10.9860000000001\\
1510	-10.9860000000001\\
1512	-15.8689999999999\\
1513	-10.9860000000001\\
1514	-13.4280000000001\\
1515	-21.973\\
1516	-21.973\\
1517	-26.855\\
1519	-43.9449999999999\\
1520	-28.076\\
1522	-28.076\\
1523	-24.414\\
1524	-18.3109999999999\\
1525	-20.752\\
1526	-34.1800000000001\\
1527	-37.8420000000001\\
1528	-24.414\\
1529	-12.2070000000001\\
1530	-10.9860000000001\\
1531	-3.66200000000003\\
1532	-8.54500000000007\\
1533	-9.76600000000008\\
1534	-12.2070000000001\\
1535	-13.4280000000001\\
1536	-13.4280000000001\\
1537	-7.32400000000007\\
1538	-8.54500000000007\\
1539	-10.9860000000001\\
1540	-8.54500000000007\\
1541	-12.2070000000001\\
1542	-14.6479999999999\\
1543	-24.414\\
1544	-28.076\\
1545	-24.414\\
1546	-28.076\\
1547	-35.4000000000001\\
1548	-35.4000000000001\\
1549	-40.2829999999999\\
1550	-40.2829999999999\\
1551	-29.297\\
1552	-19.5309999999999\\
1553	-18.3109999999999\\
1554	-30.518\\
1555	-37.8420000000001\\
1557	-15.8689999999999\\
1558	-9.76600000000008\\
1559	-4.88300000000004\\
1560	-6.10400000000004\\
1562	-6.10400000000004\\
1563	-12.2070000000001\\
1564	-14.6479999999999\\
1566	-9.76600000000008\\
1567	-4.88300000000004\\
1568	-8.54500000000007\\
1570	-8.54500000000007\\
1571	-7.32400000000007\\
1572	-4.88300000000004\\
1573	-12.2070000000001\\
1574	-25.635\\
1575	-24.414\\
1576	-31.7380000000001\\
1577	-25.635\\
1578	-32.9590000000001\\
1579	-46.3869999999999\\
1581	-39.0630000000001\\
1582	-34.1800000000001\\
1583	-31.7380000000001\\
1584	-32.9590000000001\\
1585	-20.752\\
1586	-20.752\\
1587	-13.4280000000001\\
1588	-14.6479999999999\\
1589	-23.193\\
1590	-20.752\\
1591	-15.8689999999999\\
1592	-19.5309999999999\\
1593	-17.0899999999999\\
1594	-17.0899999999999\\
1595	-20.752\\
1596	-19.5309999999999\\
1597	-24.414\\
1598	-20.752\\
1599	-15.8689999999999\\
1600	-18.3109999999999\\
1601	-14.6479999999999\\
1602	-2.44100000000003\\
1603	-9.76600000000008\\
1604	-13.4280000000001\\
1605	-8.54500000000007\\
1606	-4.88300000000004\\
1608	-12.2070000000001\\
1609	-12.2070000000001\\
1610	-14.6479999999999\\
1611	-13.4280000000001\\
1612	-19.5309999999999\\
1613	-19.5309999999999\\
1614	-12.2070000000001\\
1615	-19.5309999999999\\
1616	-17.0899999999999\\
1617	-7.32400000000007\\
1618	-7.32400000000007\\
1619	-3.66200000000003\\
1620	-10.9860000000001\\
1621	-8.54500000000007\\
1622	-7.32400000000007\\
1623	-19.5309999999999\\
1624	-20.752\\
1625	-10.9860000000001\\
1626	-14.6479999999999\\
1627	-12.2070000000001\\
1629	-14.6479999999999\\
1630	-8.54500000000007\\
1631	-10.9860000000001\\
1632	-7.32400000000007\\
1633	-7.32400000000007\\
1634	-9.76600000000008\\
1635	-14.6479999999999\\
1636	-14.6479999999999\\
1637	-9.76600000000008\\
1640	-9.76600000000008\\
1642	-26.855\\
1643	-18.3109999999999\\
1645	-15.8689999999999\\
1646	-21.973\\
1647	-21.973\\
1648	-13.4280000000001\\
1649	-14.6479999999999\\
1650	-13.4280000000001\\
1651	-17.0899999999999\\
1652	-12.2070000000001\\
1653	-8.54500000000007\\
1654	-10.9860000000001\\
1655	-15.8689999999999\\
1656	-12.2070000000001\\
1657	-14.6479999999999\\
1658	-14.6479999999999\\
1659	-21.973\\
1660	-18.3109999999999\\
1661	-25.635\\
1662	-35.4000000000001\\
1663	-28.076\\
1664	-18.3109999999999\\
1665	-15.8689999999999\\
1666	-23.193\\
1667	-28.076\\
1668	-20.752\\
1669	-23.193\\
1670	-18.3109999999999\\
1671	-7.32400000000007\\
1672	-8.54500000000007\\
1673	-20.752\\
1674	-18.3109999999999\\
1675	-18.3109999999999\\
1676	-23.193\\
1677	-23.193\\
1679	-15.8689999999999\\
1680	-18.3109999999999\\
1681	-23.193\\
1682	-19.5309999999999\\
1683	-18.3109999999999\\
1684	-19.5309999999999\\
1685	-13.4280000000001\\
1686	-14.6479999999999\\
1687	-13.4280000000001\\
1688	-13.4280000000001\\
1689	-20.752\\
1690	-25.635\\
1691	-23.193\\
1692	-23.193\\
1693	-17.0899999999999\\
1694	-12.2070000000001\\
1695	-15.8689999999999\\
1696	-15.8689999999999\\
1697	-21.973\\
1698	-24.414\\
1699	-28.076\\
1700	-23.193\\
1701	-13.4280000000001\\
1702	-24.414\\
1703	-36.6210000000001\\
1704	-24.414\\
1705	-23.193\\
1706	-29.297\\
1707	-21.973\\
1708	-20.752\\
1709	-20.752\\
1710	-18.3109999999999\\
1711	-20.752\\
1712	-25.635\\
1713	-19.5309999999999\\
1714	-21.973\\
1715	-21.973\\
1717	-19.5309999999999\\
1718	-21.973\\
1719	-17.0899999999999\\
1720	-18.3109999999999\\
1721	-17.0899999999999\\
1722	-17.0899999999999\\
1723	-8.54500000000007\\
1724	-2.44100000000003\\
1725	-2.44100000000003\\
1726	-8.54500000000007\\
1727	-19.5309999999999\\
1728	-25.635\\
1729	-23.193\\
1730	-17.0899999999999\\
1731	-20.752\\
1732	-19.5309999999999\\
1734	-12.2070000000001\\
1735	-14.6479999999999\\
1736	-14.6479999999999\\
1737	-12.2070000000001\\
1738	-15.8689999999999\\
1741	-8.54500000000007\\
1742	-14.6479999999999\\
1743	-21.973\\
1744	-17.0899999999999\\
1746	-17.0899999999999\\
1747	-23.193\\
1748	-20.752\\
1749	-28.076\\
1750	-21.973\\
1751	-23.193\\
1752	-23.193\\
1753	-15.8689999999999\\
1754	-20.752\\
1755	-19.5309999999999\\
1757	-21.973\\
1758	-25.635\\
1759	-19.5309999999999\\
1760	-24.414\\
1761	-30.518\\
1762	-25.635\\
1763	-23.193\\
1764	-18.3109999999999\\
1765	-21.973\\
1766	-19.5309999999999\\
1767	-14.6479999999999\\
1768	-13.4280000000001\\
1769	-15.8689999999999\\
1770	-13.4280000000001\\
1771	-26.855\\
1772	-35.4000000000001\\
1773	-29.297\\
1775	-29.297\\
1776	-18.3109999999999\\
1777	-14.6479999999999\\
1778	-12.2070000000001\\
1779	-10.9860000000001\\
1780	-10.9860000000001\\
1781	-18.3109999999999\\
1782	-23.193\\
1783	-25.635\\
1784	-21.973\\
1785	-15.8689999999999\\
1786	-25.635\\
1787	-30.518\\
1788	-31.7380000000001\\
1789	-25.635\\
1790	-42.7249999999999\\
1791	-32.9590000000001\\
1792	-18.3109999999999\\
1793	-13.4280000000001\\
1794	-13.4280000000001\\
1795	-19.5309999999999\\
1796	-26.855\\
1797	-31.7380000000001\\
1798	-26.855\\
1799	-19.5309999999999\\
1800	-13.4280000000001\\
1802	-13.4280000000001\\
1803	-15.8689999999999\\
1804	-9.76600000000008\\
1805	-9.76600000000008\\
};
\addlegendentry{True output}

\addplot [color=mycolor2, dashed, line width=2.0pt]
  table[row sep=crcr]{%
1006	-29.2441057622657\\
1007	-32.3902621771167\\
1008	-27.3837868832763\\
1009	-14.6035404206339\\
1010	-18.9702236356895\\
1011	-19.8026453614564\\
1012	-21.9878645537181\\
1013	-18.9974074558529\\
1014	-11.6245016558196\\
1015	-13.0114128118389\\
1016	-8.87855040430918\\
1017	-11.8211575016446\\
1018	-30.421687712791\\
1019	-26.7945075155137\\
1020	-27.0326040654804\\
1021	-19.8633501147619\\
1022	-15.180756800467\\
1023	-25.4041835296027\\
1024	-18.3198379289081\\
1025	-16.1698381379583\\
1026	-21.4757525440721\\
1027	-21.3135122293925\\
1028	-17.8126458077127\\
1029	-26.803174520121\\
1030	-26.3862533635854\\
1031	-21.6445632067059\\
1032	-27.1897321016911\\
1033	-28.9326427729061\\
1034	-23.7544127232402\\
1035	-20.3288765507627\\
1036	-16.7491385163812\\
1037	-19.9934364336852\\
1038	-20.9761557890552\\
1039	-21.0808789952125\\
1040	-21.7075069302161\\
1041	-23.5392674778184\\
1042	-23.6246188996804\\
1043	-33.7845326475715\\
1044	-27.9789449562516\\
1045	-21.9377762251288\\
1046	-20.43317292446\\
1047	-14.057231888793\\
1048	-13.5915117943616\\
1049	-17.5778479462767\\
1050	-13.6061209678821\\
1051	-14.8098973216788\\
1052	-18.206103539751\\
1053	-21.1695696784259\\
1054	-19.540230434314\\
1055	-28.2918940362799\\
1056	-23.1002545841443\\
1057	-16.3767058401261\\
1058	-13.8377335140979\\
1059	-16.7031002665767\\
1060	-16.8524021080566\\
1061	-11.6794337753126\\
1062	-12.4897546658931\\
1063	-13.8399493788068\\
1064	-13.3751856903202\\
1065	-11.4120156890317\\
1066	-12.2478690085213\\
1067	-14.3164847342191\\
1068	-22.6018558059161\\
1069	-21.5229271310757\\
1070	-24.699475850374\\
1071	-24.2997105797372\\
1072	-25.1538987961308\\
1073	-22.9511429764293\\
1074	-21.51697208835\\
1075	-20.9669957258202\\
1076	-18.4892297899514\\
1077	-20.2708941425278\\
1078	-28.745884991303\\
1079	-47.9541166175095\\
1080	-44.4857156662245\\
1081	-40.0110822427678\\
1082	-26.9277529630888\\
1083	-39.8179864641368\\
1084	-50.7997905609093\\
1085	-47.7907162255442\\
1086	-36.8076120396045\\
1087	-45.5986924099361\\
1088	-65.5490495633719\\
1089	-45.1098819580925\\
1090	-35.4938259370304\\
1091	-27.1453913842499\\
1092	-19.860396224175\\
1093	-18.943990753678\\
1094	-21.9397955671989\\
1095	-16.8194885126024\\
1096	-12.4648793738438\\
1097	-12.5954827637338\\
1098	-13.8753552997593\\
1099	-19.3240647786965\\
1100	-18.2914755171623\\
1101	-18.9643526408747\\
1102	-23.8946661236835\\
1103	-24.538408203955\\
1104	-34.3731260420118\\
1105	-28.1926572928164\\
1106	-31.4318205797679\\
1107	-25.2397057417897\\
1108	-22.4310263772677\\
1109	-26.5108375904322\\
1110	-19.277628556684\\
1111	-19.191327959817\\
1112	-19.4824214084231\\
1113	-20.9472914392018\\
1114	-17.237080160721\\
1115	-15.8934438031984\\
1116	-13.6929793982513\\
1117	-14.5473663537368\\
1118	-15.6520903317762\\
1119	-12.1384571710635\\
1120	-13.5235426506692\\
1121	-15.1872086984681\\
1122	-25.4881311847948\\
1123	-25.6490064173765\\
1124	-27.664321832933\\
1125	-16.3908606914918\\
1126	-22.5441951071061\\
1127	-37.5575251586649\\
1128	-25.4025827613966\\
1129	-28.4413622611935\\
1130	-29.5122062376054\\
1131	-20.6303860135924\\
1132	-15.021848494343\\
1133	-18.643554296963\\
1134	-20.9752994416904\\
1135	-30.4497758615598\\
1136	-41.5601469140336\\
1137	-37.3852745697473\\
1138	-27.1800246011858\\
1139	-30.1858492516922\\
1140	-27.1996678966614\\
1141	-27.2521057869837\\
1142	-26.9505617291684\\
1143	-18.390544429823\\
1144	-18.3359759552654\\
1145	-17.6205341424829\\
1146	-15.2070503900477\\
1147	-17.1192787388072\\
1148	-21.3412059784332\\
1149	-36.2798523546123\\
1150	-25.1428081326419\\
1151	-19.2237327343498\\
1152	-17.4082347722974\\
1153	-17.6501301763437\\
1154	-12.6659445505036\\
1155	-10.8638476014783\\
1156	-10.0745206194581\\
1157	-14.4423249410149\\
1158	-13.2862249227737\\
1159	-13.566279058778\\
1160	-15.3085561550656\\
1161	-14.8394741856132\\
1162	-12.8718216276047\\
1163	-10.9805828728367\\
1164	-9.28588527083366\\
1165	-8.31770260493909\\
1166	-19.181924198369\\
1167	-31.8793437999955\\
1168	-32.4558736310173\\
1169	-18.7671503733395\\
1170	-17.5877080479781\\
1171	-13.6564586296267\\
1172	-11.6375344983876\\
1173	-13.7133203441224\\
1174	-14.4198319094289\\
1175	-20.9495802746537\\
1176	-26.5364912138937\\
1177	-45.9825858493223\\
1178	-57.6815891014542\\
1179	-40.1063246561389\\
1180	-40.1173785056458\\
1181	-27.4331023718223\\
1182	-30.885010628107\\
1183	-33.5004148381336\\
1184	-33.4428280860102\\
1185	-28.2917697712332\\
1186	-23.7936967989733\\
1187	-26.2014558884123\\
1188	-21.5129426746396\\
1189	-25.9327754132566\\
1190	-19.3384926418237\\
1191	-33.8316054371371\\
1192	-38.7696044115808\\
1193	-26.8219134280498\\
1194	-17.6243917798424\\
1195	-20.6567831308309\\
1196	-32.1445289477003\\
1198	-46.1694855301928\\
1199	-44.9113914575933\\
1200	-46.1515264324528\\
1201	-36.2330987791222\\
1202	-33.277759560827\\
1203	-38.3585893508052\\
1204	-42.7499253019885\\
1205	-26.9066576148068\\
1206	-17.7137756251173\\
1207	-26.9506658806217\\
1208	-21.2588954078165\\
1209	-14.6706926404822\\
1210	-19.7873836811832\\
1211	-21.5959582378875\\
1212	-17.7863135819375\\
1213	-13.8710535891619\\
1214	-17.6813978208604\\
1215	-15.5941622723299\\
1216	-18.710634866359\\
1217	-28.1503836854126\\
1218	-22.4334858352961\\
1219	-19.4612598042597\\
1220	-18.6521639946013\\
1221	-27.6793511652938\\
1222	-44.153023932864\\
1223	-30.5505184445756\\
1224	-20.6490816948433\\
1225	-17.1743532713881\\
1226	-18.60521759262\\
1227	-18.5738386325186\\
1228	-13.9271892899233\\
1229	-15.1504463115941\\
1230	-17.9898901280235\\
1231	-20.984732470461\\
1232	-21.7753616230814\\
1233	-15.7190266484702\\
1234	-25.7200603673361\\
1235	-30.503518829863\\
1236	-27.5963120725082\\
1237	-22.1195811707312\\
1238	-24.8516143574295\\
1239	-31.838865984317\\
1240	-20.0884358289152\\
1241	-13.6349570053799\\
1242	-14.0638141173588\\
1243	-16.0874734007243\\
1244	-19.7530556296958\\
1245	-18.9285002140468\\
1246	-14.17108634636\\
1247	-20.0692769784437\\
1248	-19.116828737702\\
1249	-16.0065890355843\\
1250	-18.0322729431525\\
1251	-15.6873683689701\\
1253	-17.1815442265502\\
1254	-10.8337844369271\\
1255	-12.9508998914189\\
1256	-10.5241616746168\\
1257	-11.2991984658559\\
1258	-17.4606124231188\\
1259	-19.912511205868\\
1260	-28.3445506601395\\
1261	-35.3668848402069\\
1262	-23.947267788006\\
1263	-17.5824177940424\\
1264	-14.4625225440129\\
1265	-13.9906373863223\\
1266	-16.1638788380205\\
1267	-11.2063210417596\\
1268	-13.0997390609407\\
1269	-15.2020259222923\\
1270	-24.6005094150753\\
1271	-24.6901355296784\\
1272	-30.6675968165671\\
1273	-19.9533922152368\\
1274	-21.902214118503\\
1275	-12.3846377403972\\
1276	-14.4811565319715\\
1277	-11.0555289947006\\
1278	-10.6483893833345\\
1279	-9.23071236577016\\
1280	-16.0054674581932\\
1281	-17.4750560953707\\
1282	-20.1451341808925\\
1283	-19.8365485779975\\
1284	-33.3239173894799\\
1285	-30.4191322329084\\
1286	-20.1021081799713\\
1287	-24.9889826180379\\
1288	-25.66605700319\\
1289	-31.3469604106158\\
1290	-25.353300104191\\
1291	-20.3773797563456\\
1292	-12.561998856431\\
1293	-12.8971463012981\\
1294	-9.61548774065091\\
1295	-9.39102905458594\\
1296	-10.4129383589359\\
1297	-10.0260628257738\\
1298	-12.4073821091802\\
1299	-17.0276967251286\\
1300	-17.0134831371056\\
1302	-18.848970421413\\
1303	-14.1325677324844\\
1304	-13.3643081242567\\
1305	-11.8640863838996\\
1306	-20.4981972158355\\
1307	-19.8068306545385\\
1308	-21.4787991347368\\
1309	-20.7791497121495\\
1310	-16.3233953824406\\
1311	-19.8146236734078\\
1312	-27.1307893071546\\
1313	-27.3674323745477\\
1314	-19.4812693774397\\
1315	-31.9390361527917\\
1316	-24.3577069844987\\
1317	-23.8664399348811\\
1318	-27.6639836667694\\
1319	-26.5726148125038\\
1320	-23.6075149797912\\
1321	-19.701520370367\\
1322	-28.6844996161258\\
1323	-46.2666147136285\\
1324	-31.9955493440712\\
1325	-18.7760920752009\\
1326	-20.2312941256582\\
1327	-19.7006147218435\\
1328	-24.6480670268929\\
1329	-27.0194932162765\\
1330	-19.8135004230439\\
1331	-22.7168889951236\\
1332	-27.4154657004879\\
1333	-30.2175720776113\\
1334	-19.6851413121492\\
1335	-18.6587028552392\\
1336	-21.4256970459442\\
1337	-35.0622326744624\\
1338	-33.8532389505092\\
1339	-28.528391259782\\
1340	-21.1091485220568\\
1341	-20.4462199153479\\
1342	-19.8262525189568\\
1343	-14.0584789380353\\
1344	-11.1778894028\\
1345	-9.48640925629434\\
1346	-8.56664679466053\\
1347	-9.52401486514373\\
1348	-18.5850028120685\\
1349	-22.1914244724599\\
1350	-16.7183101117375\\
1351	-17.4221406151105\\
1352	-19.2612055169736\\
1353	-14.0822423732996\\
1354	-15.9851077278834\\
1355	-13.0704940894504\\
1357	-12.4783902149059\\
1359	-25.2975708654033\\
1360	-15.3070371492347\\
1361	-15.7173702067867\\
1362	-11.887918300994\\
1363	-13.9544882607622\\
1364	-10.9303682367649\\
1365	-17.9573550273556\\
1366	-34.6754994879961\\
1367	-23.3248212495071\\
1368	-31.667173888088\\
1369	-39.364443365117\\
1370	-37.6007074133288\\
1371	-30.9463446522072\\
1372	-21.1829985310292\\
1373	-23.4409341416751\\
1374	-21.9603836282081\\
1375	-25.5754261633617\\
1376	-27.7305049731231\\
1377	-28.2997364387631\\
1378	-38.1023772608503\\
1379	-44.9869593480403\\
1380	-50.735244032682\\
1381	-34.5920998839783\\
1382	-37.3515222966316\\
1383	-44.2204798307275\\
1384	-31.1674301861601\\
1385	-38.1094616086666\\
1386	-31.8518099808496\\
1387	-18.7648393136928\\
1388	-13.7339894304223\\
1389	-14.4168261552263\\
1390	-20.5679175116632\\
1391	-15.810715991737\\
1392	-13.6662354048633\\
1393	-14.7306711959793\\
1395	-11.90433052918\\
1396	-11.1363796096516\\
1397	-12.7600132839025\\
1398	-11.6634411152411\\
1399	-14.293743345034\\
1400	-22.6428248214013\\
1401	-17.8050145182449\\
1402	-26.0213027648713\\
1403	-42.470224520818\\
1404	-36.0486833309251\\
1405	-43.0289495321913\\
1406	-28.3762800395116\\
1407	-33.6951579613024\\
1408	-21.1357267055794\\
1409	-16.2960062875197\\
1410	-18.5461032833991\\
1411	-14.9665431158094\\
1412	-12.8513305384679\\
1413	-15.8854560606592\\
1414	-10.0435472823153\\
1415	-12.6970073396494\\
1416	-8.37117878782169\\
1417	-14.7984669986583\\
1418	-23.6745743401937\\
1419	-25.3122014476228\\
1420	-24.8401279683251\\
1421	-22.7057593405113\\
1422	-25.6951752005639\\
1423	-24.0258583132679\\
1424	-19.3438032013939\\
1425	-20.4220158122373\\
1426	-16.6064407322879\\
1427	-23.0596182339905\\
1428	-15.8463198289624\\
1429	-16.0257879457556\\
1430	-20.3468387410837\\
1431	-27.9300021404083\\
1432	-19.7625210172073\\
1433	-18.4958774600063\\
1434	-16.1492418657865\\
1435	-17.6094095310364\\
1436	-13.1588701511871\\
1437	-12.6396014325305\\
1438	-14.7266338545469\\
1439	-17.7643410219844\\
1440	-21.0751545713997\\
1441	-17.4130248110894\\
1442	-19.830175138329\\
1443	-20.9382146768519\\
1444	-13.1521400015786\\
1445	-14.2476905285696\\
1446	-21.8145186015113\\
1447	-33.8712591069952\\
1448	-28.8271614873142\\
1449	-26.5884008756141\\
1450	-23.2490055124824\\
1451	-21.0859795448407\\
1452	-19.0815343591357\\
1453	-17.9724272354185\\
1454	-21.3828293583399\\
1455	-18.3229366725207\\
1456	-14.1121777607896\\
1457	-18.3326577093735\\
1458	-21.0345995296316\\
1459	-23.422612580857\\
1460	-25.545718411237\\
1461	-24.4694404665372\\
1462	-34.7436660507915\\
1463	-43.2639446732856\\
1464	-51.1275674738079\\
1465	-29.2567010845073\\
1466	-20.4478877202655\\
1467	-15.1414974166344\\
1468	-12.8063968380507\\
1469	-13.2549557323064\\
1470	-12.0488434969\\
1471	-20.9769096373327\\
1472	-18.2126691658596\\
1473	-17.2741301438668\\
1474	-19.4947676102088\\
1475	-23.1954873851289\\
1476	-27.177776035996\\
1477	-28.6925175709241\\
1478	-41.642201400153\\
1479	-39.9294789974397\\
1480	-27.3180215524885\\
1481	-19.7120060477673\\
1482	-21.9967950966652\\
1483	-22.340210662861\\
1484	-26.5337792142036\\
1485	-18.0698298518753\\
1486	-20.5982105707844\\
1487	-18.2597784383192\\
1488	-12.0055889424409\\
1489	-17.2884296280399\\
1490	-24.6781519129938\\
1491	-19.7374331317546\\
1492	-20.9182814005151\\
1493	-35.6462751846427\\
1494	-27.7210567773395\\
1495	-26.0679247617463\\
1496	-36.4987072781469\\
1497	-33.005341732827\\
1498	-20.9181539962435\\
1499	-14.7518086902719\\
1500	-17.9940603760715\\
1501	-23.7943808785958\\
1502	-41.8210362638372\\
1503	-38.9908641441652\\
1504	-38.4898551991498\\
1505	-29.8441364401165\\
1506	-20.5222579262445\\
1507	-19.8361545960138\\
1508	-14.3890295112717\\
1509	-14.5683888538488\\
1510	-13.3374886033603\\
1511	-15.3403678187817\\
1512	-17.6549969253579\\
1513	-11.4247040518603\\
1514	-15.291953093219\\
1515	-21.4153399471556\\
1516	-24.3502297766295\\
1517	-28.9574172542261\\
1518	-37.6036257700734\\
1519	-47.9660896881264\\
1520	-34.8406660704504\\
1521	-27.2606566263801\\
1522	-30.6798428555551\\
1523	-25.427348548161\\
1524	-18.6417074057256\\
1525	-21.4550124682271\\
1526	-34.9836161519027\\
1527	-40.660546582995\\
1528	-21.5770441274494\\
1529	-16.0759204314249\\
1530	-12.6632878313478\\
1531	-11.8716825342772\\
1532	-9.23380731310203\\
1533	-9.91530916713418\\
1534	-12.5820556199783\\
1535	-17.4004702272966\\
1536	-14.6169625540831\\
1537	-12.0096785833455\\
1538	-11.6993734829334\\
1539	-9.75490629309206\\
1540	-11.6622484730169\\
1541	-11.1161380159106\\
1542	-16.1888335207454\\
1543	-26.2651330557371\\
1544	-28.110037798343\\
1545	-25.8107224404221\\
1546	-29.9533220080839\\
1547	-36.6867332612694\\
1548	-38.844271700488\\
1549	-41.3787171523429\\
1550	-40.8030150367754\\
1551	-29.0400290960351\\
1552	-23.0378392274424\\
1553	-20.9037618232958\\
1554	-32.3150390637791\\
1555	-36.412154562689\\
1556	-26.257062829505\\
1557	-19.567945727548\\
1558	-11.3576003607232\\
1559	-13.02919390229\\
1560	-8.99734714884903\\
1561	-7.44740127560772\\
1562	-8.79580089139085\\
1563	-12.3258882206112\\
1564	-12.8394654003316\\
1565	-13.5785204881145\\
1566	-10.882163122935\\
1567	-11.1427401939472\\
1568	-8.85698403261677\\
1569	-9.75298017725504\\
1570	-10.930273194482\\
1571	-8.86485612416845\\
1572	-9.37845289567463\\
1573	-9.59531159160792\\
1574	-29.4170267761938\\
1575	-37.5183111874319\\
1576	-33.1125270938976\\
1577	-24.743484714799\\
1578	-32.465282563586\\
1579	-52.3655304956499\\
1580	-43.5085213291059\\
1581	-43.8657568327328\\
1582	-30.5484700736076\\
1583	-34.3048511577908\\
1584	-36.0417159432996\\
1585	-24.2014444197471\\
1586	-21.6518896368302\\
1587	-12.7628997803206\\
1588	-14.0981992954673\\
1589	-24.325571993882\\
1590	-16.8932861196968\\
1591	-20.4477880427221\\
1592	-21.6704463207363\\
1593	-19.0821945117134\\
1594	-19.8836592115822\\
1595	-20.4299784333919\\
1596	-22.5139749326968\\
1597	-22.874321855073\\
1598	-22.4401099517684\\
1599	-17.3261024992862\\
1600	-21.3681311924167\\
1601	-11.2677942931603\\
1602	-14.5303159820248\\
1603	-8.36809112924266\\
1604	-11.5880823288492\\
1605	-9.61613809616279\\
1606	-13.056813292349\\
1607	-6.9832279450477\\
1608	-9.68588627456938\\
1609	-15.2120306184943\\
1610	-17.0331131524674\\
1611	-15.1652527616313\\
1612	-22.2185590346521\\
1613	-20.3554463294913\\
1614	-15.5837773291873\\
1615	-18.6255982215787\\
1616	-12.8928189108783\\
1617	-15.2970192522923\\
1618	-10.2525286416756\\
1619	-10.317637100711\\
1620	-7.11849422508067\\
1621	-8.35854550274439\\
1622	-10.6175820477029\\
1623	-16.8614735140156\\
1624	-18.5499630444854\\
1625	-16.3639372468256\\
1626	-14.5340457001705\\
1627	-13.914996701372\\
1628	-15.708449592131\\
1629	-12.1111711774813\\
1630	-13.7984726157204\\
1631	-10.9934428611973\\
1632	-10.640757842428\\
1633	-10.5492585639986\\
1634	-8.78876970783062\\
1635	-13.173985617379\\
1636	-13.9399420277489\\
1638	-12.2998178749101\\
1639	-10.5180642905534\\
1640	-11.2516327827518\\
1641	-18.1096963553914\\
1642	-29.9200663026852\\
1643	-19.6878502472398\\
1644	-19.3887789486828\\
1645	-14.4473399960902\\
1646	-21.6133971486161\\
1647	-22.2823402255508\\
1648	-15.682582399498\\
1649	-18.1504062779475\\
1650	-12.9261431925033\\
1651	-17.0594351836451\\
1652	-12.5677638579432\\
1653	-13.4889548207932\\
1654	-13.370894606338\\
1655	-15.4929969914513\\
1656	-13.4192068985496\\
1657	-13.685657311351\\
1658	-15.1108458784511\\
1659	-21.3433385095911\\
1660	-19.9494528296179\\
1661	-27.8926416688387\\
1662	-38.1627845270511\\
1664	-21.5884921116917\\
1665	-16.5232333497138\\
1666	-22.517625931099\\
1667	-29.7847007366142\\
1668	-21.8123621481443\\
1669	-24.6417886430033\\
1670	-17.0765174815983\\
1671	-13.9455587426912\\
1672	-11.6947228576075\\
1673	-17.9519402169065\\
1674	-20.7019788949258\\
1675	-16.4199560387945\\
1676	-24.919032362282\\
1677	-23.2596691869187\\
1678	-21.9494591441646\\
1679	-16.7778916447919\\
1680	-19.1848421679836\\
1681	-24.7641622993017\\
1682	-20.2592414606238\\
1683	-20.6568825015643\\
1684	-18.4813150671625\\
1685	-16.564970039753\\
1686	-16.7571888936279\\
1687	-13.4296261643954\\
1688	-15.115092091841\\
1689	-21.4376527198776\\
1690	-25.2233655924997\\
1691	-25.3713851719001\\
1692	-22.8932999033022\\
1693	-18.5591184007683\\
1694	-15.2436506933689\\
1695	-15.1486585648515\\
1696	-18.6775053159909\\
1697	-21.2144193756774\\
1698	-25.5750870638717\\
1699	-30.0680380028618\\
1700	-24.4911773050458\\
1701	-16.2276495678639\\
1702	-23.8707763879277\\
1703	-36.0562082829163\\
1704	-25.9950236498814\\
1705	-24.6582283409189\\
1706	-28.9865038684168\\
1707	-24.0949826672472\\
1708	-19.3935041412847\\
1709	-23.4030738219208\\
1710	-21.0894592752284\\
1711	-21.131128128256\\
1712	-26.670448163346\\
1713	-20.5069751354215\\
1714	-23.2502940248928\\
1715	-22.3365271824171\\
1716	-23.1184756249045\\
1717	-18.2142980391159\\
1718	-22.5646487190825\\
1719	-17.7069694784545\\
1720	-18.5636036616215\\
1721	-20.2689038359458\\
1722	-19.7937098744749\\
1723	-11.6982725776836\\
1724	-11.6293180747334\\
1725	-8.41974100382549\\
1726	-5.91147050027325\\
1727	-17.7700235187756\\
1728	-30.6098725562902\\
1729	-24.86542017494\\
1730	-18.8242881498752\\
1731	-20.1615314609503\\
1732	-20.2680230024505\\
1733	-17.826717506601\\
1734	-13.5466439943609\\
1735	-15.6824045034984\\
1736	-15.2666127097107\\
1738	-16.2687494070533\\
1739	-15.010268499939\\
1740	-14.5823292499176\\
1741	-10.5199190482867\\
1742	-12.0222180591647\\
1743	-22.1203980702105\\
1744	-16.7414923722215\\
1745	-20.2790849772691\\
1746	-18.0434536486891\\
1747	-22.2229426612789\\
1748	-23.3659547396753\\
1749	-27.9635670442269\\
1750	-22.7119924776093\\
1751	-23.414152479656\\
1752	-25.6239701710044\\
1753	-14.7507211889063\\
1754	-22.8232358085966\\
1755	-16.5892885422049\\
1756	-21.3502350631363\\
1757	-22.1040751173887\\
1758	-26.1399005279136\\
1759	-22.5097489643933\\
1760	-24.9396170249906\\
1761	-31.8911741418433\\
1762	-24.9630735510668\\
1763	-24.1420543814809\\
1764	-20.34688283271\\
1765	-21.7412110416928\\
1766	-20.24847180558\\
1767	-18.8968323912125\\
1768	-15.5801718156299\\
1769	-16.6221631220003\\
1770	-16.1138238634542\\
1771	-26.1134061849077\\
1772	-36.4377907091607\\
1773	-30.4222732194096\\
1774	-29.1174129888907\\
1775	-28.5485825676046\\
1776	-19.1577623790449\\
1777	-18.4012095293908\\
1778	-15.1383515487657\\
1779	-13.2913328501661\\
1780	-13.6637792643437\\
1781	-18.5467553599108\\
1782	-23.8315739676357\\
1783	-26.2249073530024\\
1784	-22.4593786357052\\
1785	-17.1715876896553\\
1786	-25.9469346818983\\
1787	-32.7357704723252\\
1788	-34.3550146272814\\
1789	-27.276809084047\\
1790	-43.4556458934808\\
1791	-34.9525132890869\\
1792	-19.1021380799411\\
1793	-16.3340029970041\\
1794	-14.2158690928243\\
1796	-26.2915196616489\\
1797	-35.2176456908223\\
1798	-26.7849972015915\\
1799	-22.1563982498169\\
1800	-14.8510504170952\\
1801	-14.5114865644503\\
1802	-15.979570113371\\
1803	-15.5312822566418\\
1804	-12.5562390588645\\
1805	-13.1062638324208\\
};
\addlegendentry{OSA predition}

\addplot [color=mycolor3, dotted, line width=2.0pt]
  table[row sep=crcr]{%
1006	-28.076\\
1007	-32.9590000000001\\
1008	-25.635\\
1009	-14.6479999999999\\
1010	-18.9702236356895\\
1011	-19.137769017907\\
1012	-22.5296921457948\\
1013	-20.6396618397639\\
1014	-12.0023264489146\\
1015	-14.9240775587991\\
1016	-13.6183003265924\\
1017	-16.7687312280152\\
1018	-34.977932301025\\
1020	-33.4407821553698\\
1021	-25.4666282944581\\
1022	-19.9911612221304\\
1023	-30.9402457828623\\
1024	-23.0372220338973\\
1025	-19.4678534805953\\
1026	-25.1545582492249\\
1027	-24.5541160329026\\
1028	-21.5040130853959\\
1029	-30.2801738997134\\
1030	-28.496666969228\\
1031	-23.9102861963349\\
1032	-30.1883605266164\\
1033	-30.8738196680672\\
1034	-25.9469257497451\\
1035	-22.8910802134285\\
1036	-19.3455066520978\\
1037	-22.4913934913204\\
1038	-24.7606496697954\\
1039	-23.7676116972443\\
1040	-23.7817670687491\\
1041	-25.3885015826813\\
1042	-25.5946498873043\\
1043	-35.5516548403539\\
1044	-30.777757262382\\
1045	-23.6860152883944\\
1046	-22.7958036070447\\
1047	-16.8185060283024\\
1048	-16.6726183948481\\
1049	-19.8553711151017\\
1050	-16.2364695166334\\
1051	-17.538139471199\\
1052	-20.8511282275206\\
1053	-23.491518386945\\
1054	-22.1930006561847\\
1055	-30.9909914190289\\
1056	-24.90955938107\\
1057	-18.3412365342976\\
1058	-16.9508379579245\\
1059	-19.6334343852436\\
1060	-20.6404205105162\\
1061	-14.6493943185287\\
1062	-15.5882071408441\\
1063	-17.0444663817545\\
1064	-15.5643344080891\\
1065	-14.6746715262002\\
1066	-16.33238487836\\
1067	-17.0576473317028\\
1068	-25.9541364175823\\
1069	-24.5358367816966\\
1070	-27.3737016321807\\
1071	-26.2333665344372\\
1072	-27.7958088363275\\
1073	-24.8328841116449\\
1074	-23.8857780136377\\
1075	-23.2614519285253\\
1076	-21.2851256927927\\
1077	-22.6960159694684\\
1078	-32.2043480247921\\
1079	-50.3383555617272\\
1080	-49.3624521929084\\
1081	-44.889034467006\\
1082	-29.7196618993203\\
1083	-42.5336976530002\\
1084	-53.8670901159094\\
1085	-51.6876581202775\\
1086	-40.5605701134862\\
1087	-49.7475858586806\\
1088	-68.7387310771142\\
1089	-50.4689336007277\\
1090	-40.0794556237684\\
1091	-31.1683524644966\\
1092	-24.4950230189388\\
1093	-22.2096133522232\\
1094	-25.4888843559763\\
1095	-19.7409666074038\\
1096	-14.5097962051182\\
1097	-14.4368889023362\\
1098	-16.4353212366791\\
1099	-21.9976372044434\\
1100	-20.9525135534809\\
1101	-21.3301905758835\\
1102	-26.2176586892451\\
1103	-27.184180008888\\
1104	-37.1286793950142\\
1105	-32.0253434507254\\
1106	-34.7049661039352\\
1107	-28.3704575179008\\
1108	-24.9684464925988\\
1109	-29.2550767693035\\
1110	-22.2559395929413\\
1111	-21.9022839618196\\
1112	-24.0171281885257\\
1113	-24.5737064308225\\
1114	-20.3695124314868\\
1115	-20.6314943460297\\
1116	-17.1846427288781\\
1117	-17.9891854337804\\
1118	-19.1881571327406\\
1119	-15.6694086259995\\
1120	-17.2962605277269\\
1121	-19.2664919456543\\
1122	-28.4793778303992\\
1123	-29.3915105412048\\
1124	-30.6313202759727\\
1125	-18.8713634102514\\
1126	-25.0209124699606\\
1127	-40.4701413483749\\
1128	-29.4964072589949\\
1129	-32.1464631605886\\
1130	-32.9791423245404\\
1131	-23.6716479915829\\
1132	-18.1719978178116\\
1133	-22.238292253697\\
1134	-24.5450995525509\\
1135	-34.8677390214625\\
1136	-45.4801652994604\\
1137	-42.7979559886196\\
1138	-31.741252626261\\
1139	-33.7227216387892\\
1140	-31.2293536120198\\
1141	-31.0647323704991\\
1142	-30.6745154393839\\
1143	-21.945070545727\\
1144	-20.3283656939516\\
1145	-20.3097129766361\\
1146	-18.4592419694216\\
1147	-20.2293440499677\\
1148	-25.6215532199133\\
1149	-40.3654572903185\\
1150	-30.4187136668622\\
1151	-22.748597676595\\
1152	-20.992847565054\\
1153	-21.3754083538602\\
1154	-15.9921938182774\\
1155	-14.5778770715133\\
1156	-14.3025075208623\\
1157	-19.0995865925152\\
1158	-16.6416526297785\\
1159	-17.4700677680962\\
1160	-18.0152420919362\\
1161	-17.2521773307292\\
1162	-15.4539935224786\\
1163	-14.000563188097\\
1164	-13.3572365579817\\
1165	-13.1555509193668\\
1166	-24.0497210660019\\
1167	-37.101741273159\\
1168	-38.8961737714903\\
1169	-25.2540310654456\\
1170	-21.8274346235285\\
1171	-18.7221738540097\\
1172	-16.7082833829743\\
1173	-19.8282816896854\\
1174	-20.13663798125\\
1176	-32.3362164667558\\
1177	-50.8419077508274\\
1178	-65.0796700336546\\
1179	-49.8489727209314\\
1180	-47.2793950907731\\
1181	-34.4552671246856\\
1182	-36.9014608353568\\
1183	-38.4426372556154\\
1184	-38.4911225739193\\
1185	-32.0916338109744\\
1186	-28.4568240290707\\
1187	-30.2620876587412\\
1188	-25.9390934935755\\
1189	-29.6640581319461\\
1190	-23.4003511679327\\
1191	-37.8359022562179\\
1192	-43.5753169861487\\
1193	-31.1809272718949\\
1194	-20.4921275376591\\
1195	-22.9529208472766\\
1196	-35.0990188250046\\
1197	-42.1498938003372\\
1198	-50.3693467812607\\
1199	-50.8060894399539\\
1200	-51.1302123189851\\
1201	-41.0806351027022\\
1202	-38.2385062704961\\
1203	-42.5249406281798\\
1204	-47.207720405512\\
1205	-31.0316551377766\\
1206	-19.9222865868876\\
1207	-29.2975687594094\\
1208	-24.610266447329\\
1209	-17.2480115161375\\
1210	-22.5399855645614\\
1211	-24.652069115517\\
1212	-20.9840903333713\\
1213	-17.5684398622848\\
1214	-20.8998615158814\\
1215	-19.0269138814001\\
1216	-22.4928888124016\\
1217	-31.9978512276471\\
1218	-26.6769486605444\\
1219	-21.705732786545\\
1220	-22.0116630306204\\
1221	-31.9221795118369\\
1222	-49.0743231565218\\
1223	-36.4841187930133\\
1224	-25.9008128967669\\
1225	-21.214369870193\\
1226	-23.037008038136\\
1227	-22.2911691088616\\
1228	-17.8773087049485\\
1229	-19.0871744337294\\
1230	-21.9062406179155\\
1231	-25.2509628659252\\
1232	-26.3792799507942\\
1233	-19.7668409309545\\
1234	-29.3581039458636\\
1235	-34.7632698378318\\
1236	-31.4874792993176\\
1237	-26.019141623822\\
1238	-28.7396049870565\\
1239	-35.8839595599209\\
1240	-23.8647389967091\\
1241	-15.6972680199299\\
1242	-17.7515581402686\\
1243	-19.2692013412291\\
1244	-23.4975463838293\\
1245	-23.2369564419007\\
1246	-18.1649025868903\\
1247	-23.8732810115052\\
1248	-23.509419911705\\
1249	-19.3406212415377\\
1250	-21.8593138161696\\
1251	-19.6513686805815\\
1252	-19.5082024858668\\
1253	-21.5513864350828\\
1254	-14.6452712884125\\
1255	-16.3244309188108\\
1256	-14.1251776538709\\
1257	-14.9195645848433\\
1258	-21.6431705406324\\
1259	-24.5571748545212\\
1260	-32.1937703372882\\
1261	-40.5886995322508\\
1262	-27.9867210492503\\
1263	-20.4217195805772\\
1264	-18.4641987097659\\
1265	-17.8793145927511\\
1266	-20.4815999550981\\
1267	-15.8019761165206\\
1268	-16.9022561918125\\
1269	-18.876814006042\\
1270	-28.1218348171674\\
1271	-27.7192165586703\\
1272	-34.39693985447\\
1273	-24.2766913272142\\
1274	-24.1402854140399\\
1275	-15.9557679012626\\
1276	-17.0856196506825\\
1277	-13.9271795589616\\
1278	-14.1919534958608\\
1279	-13.5368168384528\\
1280	-19.5994157650009\\
1282	-23.1436248114105\\
1283	-22.9566595799399\\
1284	-36.1639020632738\\
1285	-34.4252141486759\\
1286	-25.0632264548929\\
1287	-29.7163047942931\\
1288	-30.1154416171246\\
1289	-35.9438949870359\\
1290	-28.9026784613029\\
1291	-22.6433149770178\\
1292	-15.5170293584417\\
1293	-16.1667368125873\\
1294	-14.4183694460171\\
1295	-14.1324871551169\\
1296	-14.2514643797645\\
1297	-14.2214887943271\\
1298	-16.3137126878289\\
1299	-20.3519127877034\\
1300	-19.9454448001143\\
1301	-20.8604312140785\\
1302	-21.9582084045489\\
1303	-16.2371397256495\\
1304	-15.2728538901783\\
1305	-16.7580448872818\\
1306	-24.9356035301114\\
1307	-23.4188554396362\\
1308	-25.2226557053316\\
1309	-23.4588873863986\\
1310	-19.2297870982582\\
1311	-22.8567914314142\\
1312	-29.9281150785848\\
1313	-30.5230852668344\\
1314	-22.6124316182179\\
1315	-34.6294295082421\\
1316	-28.6137243792703\\
1317	-26.300786941692\\
1318	-30.2320301088357\\
1319	-28.7198436775661\\
1320	-25.0557134338144\\
1321	-22.0191628245871\\
1322	-30.5971320358594\\
1323	-48.0168238162398\\
1324	-35.8399215368208\\
1325	-22.0960935603375\\
1326	-22.9385970124936\\
1327	-23.4941218230697\\
1328	-28.2721233187422\\
1329	-31.1681706428728\\
1330	-23.3106643873498\\
1331	-25.8349415907996\\
1332	-31.3836192039876\\
1333	-33.6392268108414\\
1334	-23.3136218604509\\
1335	-21.2727735017029\\
1336	-25.1713397070116\\
1337	-38.4800962369513\\
1338	-37.2987484515515\\
1339	-32.8485880127932\\
1340	-23.1717055547551\\
1341	-21.8644910069379\\
1342	-22.502016556921\\
1343	-16.8616934507411\\
1344	-14.3029310991419\\
1345	-13.820054350087\\
1346	-13.0708297399165\\
1347	-14.8320383927412\\
1348	-22.8408907091712\\
1349	-25.9214633648235\\
1350	-21.1896842643578\\
1351	-21.0125696143134\\
1352	-23.3976954061022\\
1353	-16.9398923825916\\
1354	-18.1644290994541\\
1355	-17.2961098712665\\
1356	-15.6266543768636\\
1357	-16.505064971366\\
1358	-23.170297736134\\
1359	-29.221498359035\\
1360	-19.3983217112755\\
1361	-18.1822610347544\\
1362	-16.0600657548819\\
1363	-17.6258398890502\\
1364	-14.2987446167115\\
1365	-21.7430450529616\\
1366	-39.8091033397291\\
1367	-28.522151699699\\
1368	-35.4928347112223\\
1369	-43.1779388707016\\
1370	-41.2835116075912\\
1371	-34.1041605561381\\
1372	-25.2448641845808\\
1373	-26.955873654646\\
1374	-26.8568939221555\\
1375	-29.7233650104017\\
1376	-32.3451257202782\\
1377	-31.7264738828649\\
1378	-40.8747400488635\\
1379	-48.0309273348851\\
1380	-54.005057773004\\
1381	-38.9255477934807\\
1382	-39.8238538159344\\
1383	-47.5196295747462\\
1384	-35.3242677296062\\
1385	-41.1025881825037\\
1386	-36.1311034015512\\
1387	-22.269089036608\\
1388	-16.2999156498697\\
1389	-17.2971307989808\\
1390	-23.3330914519188\\
1391	-18.9341538773485\\
1392	-15.6471328328223\\
1393	-18.4072958175664\\
1395	-14.0073299141977\\
1396	-15.3945462190138\\
1397	-15.9547705729619\\
1398	-14.4888379322892\\
1399	-18.6620163823063\\
1400	-24.5504519885394\\
1401	-20.4781562130129\\
1402	-29.3941908787313\\
1403	-46.2387189789147\\
1404	-41.3223155295784\\
1405	-48.8955862292148\\
1406	-35.1666157993338\\
1407	-37.7701725945217\\
1408	-26.1564785364692\\
1409	-20.6086124542182\\
1410	-22.2363608215392\\
1411	-18.8046525430454\\
1412	-16.3757313925969\\
1413	-19.5999116409919\\
1414	-13.334646165004\\
1415	-13.96080735757\\
1416	-12.9332947004616\\
1417	-18.5669347387577\\
1418	-26.9429914710595\\
1419	-29.8773613869998\\
1420	-29.4560345688928\\
1421	-26.9960945084872\\
1422	-29.8939299599651\\
1423	-27.2174118880075\\
1424	-22.7172352957521\\
1425	-23.9469462726161\\
1426	-20.5700243068065\\
1427	-27.3027154791064\\
1428	-20.1794985359984\\
1429	-18.6447477254096\\
1430	-23.7803045161638\\
1431	-31.5092888683457\\
1432	-22.9415955108113\\
1433	-20.8752806576338\\
1434	-19.1348506543254\\
1435	-21.0028766921471\\
1436	-16.7954842302499\\
1437	-15.9734979547186\\
1438	-19.2096942346798\\
1440	-23.7514176572829\\
1441	-20.882396358207\\
1442	-23.0171269450211\\
1443	-23.6283871313333\\
1444	-16.4177416126981\\
1445	-17.6535384607387\\
1446	-25.9491245969205\\
1447	-37.5993510002061\\
1448	-33.5532670150785\\
1449	-30.2106692066545\\
1450	-27.4881535891714\\
1451	-25.091174460295\\
1452	-22.8215921077181\\
1453	-21.4713611909233\\
1454	-25.1566489624445\\
1455	-22.0941798475694\\
1456	-17.5036531283447\\
1457	-21.5481445330254\\
1458	-24.2980196034539\\
1459	-26.7613477294226\\
1460	-28.9102742457385\\
1461	-27.2892923625611\\
1462	-36.9509373827509\\
1463	-45.982903097093\\
1464	-53.9398895944455\\
1465	-33.7601108890472\\
1466	-23.006520445768\\
1467	-18.1608729456766\\
1468	-16.2139606005696\\
1469	-17.2251561863482\\
1470	-16.2314796545022\\
1471	-24.2479052403801\\
1472	-22.7299707712793\\
1473	-20.6775681178051\\
1476	-30.0596196507959\\
1477	-31.950233520862\\
1478	-44.946059728728\\
1479	-43.7310021217763\\
1480	-32.2181931505374\\
1481	-24.1552226056642\\
1482	-25.271008044813\\
1483	-25.8274310359311\\
1484	-30.5103299694026\\
1485	-22.3507178312104\\
1486	-23.6187678179845\\
1487	-22.7612424592453\\
1488	-15.9569883350066\\
1489	-20.8303156029315\\
1490	-29.0859476804594\\
1491	-21.9783911039253\\
1492	-23.1193005049395\\
1493	-39.4194812533431\\
1494	-31.0072684403228\\
1495	-28.9595453453589\\
1496	-40.9462046468589\\
1497	-37.3363928623007\\
1498	-24.8015039671636\\
1499	-17.9099340009279\\
1500	-20.2324816435812\\
1501	-27.571303837723\\
1502	-45.0636253193188\\
1503	-44.5984264457672\\
1504	-43.6279290233615\\
1505	-34.8826820149195\\
1506	-24.9143515720305\\
1507	-23.1709859726352\\
1508	-18.5650185577267\\
1509	-17.7248879149936\\
1510	-17.2220833192573\\
1511	-19.5689490751433\\
1512	-21.8420294498474\\
1513	-15.5184839825165\\
1514	-18.8820166208584\\
1515	-25.3288481614657\\
1516	-27.4651159747441\\
1517	-32.7028195570629\\
1518	-41.5624729909714\\
1519	-52.374488608443\\
1520	-40.1712629952965\\
1521	-34.1290381353765\\
1522	-36.0330225656874\\
1523	-31.1133906886907\\
1524	-23.7959550481726\\
1525	-25.6297538789988\\
1526	-39.2282414282151\\
1527	-44.7377837642628\\
1528	-25.8624168080198\\
1529	-18.2011998184423\\
1530	-15.6050367965095\\
1531	-15.0289608839819\\
1532	-14.2597860654632\\
1533	-14.3350183442681\\
1534	-16.6052566721378\\
1535	-21.4472665062119\\
1536	-19.1858862097379\\
1537	-15.8626631968536\\
1538	-16.6052487733525\\
1539	-14.9284888951781\\
1540	-15.2928156912631\\
1541	-15.6184225908282\\
1542	-19.4796918750346\\
1543	-29.9458318723832\\
1544	-31.9482481596772\\
1545	-28.9236418578403\\
1546	-33.3409223179985\\
1547	-40.3367020822823\\
1548	-42.4346089253515\\
1549	-45.9217764036216\\
1550	-45.0650447775811\\
1551	-32.7629286676995\\
1552	-26.0250912962615\\
1553	-24.6273855525608\\
1554	-36.4205887285502\\
1555	-40.7440566825526\\
1556	-29.4957856908391\\
1558	-14.5592265791922\\
1559	-15.883867651685\\
1560	-14.1149785698565\\
1561	-12.4397027035832\\
1562	-13.4811140255117\\
1563	-17.7996876702202\\
1564	-17.2736517632295\\
1565	-16.5390486479503\\
1566	-13.8230378317073\\
1567	-13.814245020993\\
1568	-13.2009611074891\\
1569	-13.2437538032868\\
1570	-14.4772808584503\\
1571	-12.8061964241247\\
1572	-12.8376309922289\\
1573	-14.21043248776\\
1574	-32.6434729334599\\
1575	-42.5892718000905\\
1576	-41.8590550719496\\
1577	-31.8386147397878\\
1578	-39.1433119428229\\
1579	-59.0613983125111\\
1580	-51.0658890128122\\
1581	-50.2330910554065\\
1582	-37.7344470552448\\
1583	-39.2400197501031\\
1584	-41.3878747957672\\
1585	-29.7515713070648\\
1586	-27.0872367299958\\
1587	-17.3349142829195\\
1588	-17.5222741417349\\
1589	-27.4760565744991\\
1590	-19.8366386000239\\
1591	-21.3162918074927\\
1592	-24.3243191444687\\
1593	-21.8729118541255\\
1594	-22.7469372131543\\
1595	-24.1792261508733\\
1596	-25.5087856482739\\
1597	-26.669725981988\\
1598	-24.950119140876\\
1599	-19.9843619174053\\
1600	-24.2394484035271\\
1601	-14.4130596366927\\
1602	-15.5789459732323\\
1603	-13.7594983890842\\
1604	-15.207873046885\\
1605	-11.5977282480849\\
1606	-15.7461227671024\\
1607	-11.8384629788818\\
1608	-12.5899973326791\\
1609	-17.4899772329202\\
1610	-20.5842663715946\\
1611	-18.3239957164217\\
1612	-25.5225154860725\\
1613	-24.4276732018639\\
1614	-19.1357009059809\\
1615	-22.9038505448862\\
1616	-15.8624179060114\\
1617	-16.0521391429131\\
1618	-14.1296124955888\\
1619	-14.1925101470056\\
1620	-12.223669783428\\
1621	-11.2734117674408\\
1622	-13.5416134756751\\
1623	-20.6776012130474\\
1624	-20.1468898892324\\
1625	-17.1783223194177\\
1626	-17.4768946829174\\
1627	-15.7691909093785\\
1628	-17.9076832916976\\
1629	-15.0121104116215\\
1630	-14.9820996135411\\
1631	-14.077378869378\\
1632	-12.9870124756962\\
1633	-13.5566431942852\\
1634	-12.6044109893051\\
1635	-15.8576286631214\\
1636	-15.9237270051465\\
1637	-14.5935202769888\\
1638	-14.6188673631862\\
1639	-13.078320463266\\
1640	-13.5157839129211\\
1641	-21.011155030199\\
1642	-32.4082221395943\\
1643	-22.9062060723916\\
1644	-22.4099842725866\\
1645	-17.5748827459861\\
1646	-23.9232318091304\\
1647	-24.2617616939801\\
1648	-17.4212661497354\\
1649	-20.26215578046\\
1650	-15.8374320041073\\
1651	-19.2321392401195\\
1652	-14.3937751299939\\
1653	-15.2652273207732\\
1654	-16.559885975325\\
1655	-18.7990106146769\\
1656	-15.9888487047333\\
1657	-16.5783151256671\\
1658	-17.0795533340711\\
1659	-23.2494128992751\\
1660	-21.3056799029382\\
1661	-29.7201072080779\\
1662	-40.5066525649816\\
1663	-32.8407144840155\\
1664	-24.685492446692\\
1665	-20.1712192368125\\
1666	-25.8635650368396\\
1667	-32.6130922448115\\
1668	-24.9467295021498\\
1669	-27.5026637872027\\
1670	-19.794861598853\\
1671	-15.5745599466507\\
1672	-15.4897257654366\\
1673	-22.1376694819351\\
1674	-23.2341507246633\\
1675	-20.0163877384541\\
1676	-27.0031070559764\\
1677	-25.6021093367101\\
1678	-23.9777928175279\\
1679	-19.1849099073124\\
1680	-21.5552079530737\\
1681	-27.1487932552407\\
1682	-22.9243263075041\\
1683	-23.0741330682345\\
1684	-21.3462278462916\\
1685	-18.4602762628826\\
1686	-19.5851812900214\\
1687	-16.4744816930711\\
1688	-17.4329754654127\\
1689	-24.4016165373223\\
1690	-27.9730008436657\\
1691	-27.534921590697\\
1692	-25.6103928947311\\
1693	-20.5417032158625\\
1694	-17.2747863913423\\
1695	-18.0462988784352\\
1696	-20.6733560982598\\
1698	-27.7872522557875\\
1699	-32.5360292298794\\
1700	-27.3108320364036\\
1701	-18.7580374464592\\
1702	-27.1832992689599\\
1703	-38.8057171996652\\
1704	-28.2480480327933\\
1705	-27.1898042747409\\
1706	-31.4892554515448\\
1707	-26.0043846788831\\
1708	-21.8668654220307\\
1709	-24.9294175751809\\
1710	-23.3736938337991\\
1711	-24.0493572331634\\
1712	-29.1684633685632\\
1713	-23.1563394905816\\
1714	-25.9047443248026\\
1715	-24.9397487501035\\
1716	-25.4507763996535\\
1717	-21.0309208406964\\
1718	-24.4012283295503\\
1719	-19.4612714579455\\
1720	-20.3372409449153\\
1721	-21.6989973514212\\
1722	-22.2454552450793\\
1723	-14.5305480206864\\
1724	-14.7530734317158\\
1725	-14.2223430056968\\
1726	-12.5428028055655\\
1727	-22.9977856971118\\
1728	-35.8287726558151\\
1729	-30.8908038909558\\
1730	-23.6581827323214\\
1731	-24.8330856799214\\
1732	-24.2757777026684\\
1733	-21.3935118087538\\
1734	-17.0416204200669\\
1735	-18.9213696736292\\
1736	-18.3645288208852\\
1737	-18.6617425445868\\
1738	-20.0330956735156\\
1739	-18.0810627919388\\
1740	-17.7340761599901\\
1741	-14.438726310744\\
1742	-15.7675435511401\\
1743	-24.5888850837623\\
1744	-19.1844309423968\\
1745	-22.0720292504006\\
1746	-20.5169919359512\\
1747	-24.6031131425648\\
1748	-25.1046835633297\\
1749	-30.728276266959\\
1750	-24.7247504919135\\
1751	-25.3714268702349\\
1752	-27.4950740218424\\
1753	-17.0464274774436\\
1754	-24.2947707170802\\
1755	-18.5901865082881\\
1756	-21.9699844031684\\
1757	-22.931215132636\\
1758	-26.8699049225104\\
1759	-23.1300171248597\\
1760	-26.6678314620328\\
1761	-33.4478326739779\\
1762	-26.8765287441297\\
1763	-25.5582517458006\\
1764	-21.8311508850904\\
1765	-23.7522907878856\\
1766	-21.6841814738468\\
1767	-20.4567677894822\\
1768	-18.4929988780734\\
1769	-19.5900260122116\\
1770	-18.8387134406992\\
1771	-29.8830302320591\\
1772	-39.358488542345\\
1773	-33.4673411541444\\
1774	-32.0872629539542\\
1775	-30.8468142694817\\
1776	-20.7335458342861\\
1777	-20.0635277499446\\
1778	-17.8205029769611\\
1779	-16.3458669832687\\
1780	-16.9937478454017\\
1781	-22.6160166114512\\
1782	-27.4594842385277\\
1783	-29.7497323479579\\
1784	-25.6113760418611\\
1785	-19.7779296452775\\
1786	-28.7588096840682\\
1787	-35.3046133711282\\
1788	-37.5085639718077\\
1789	-30.8044551292303\\
1790	-47.2236858487663\\
1791	-38.53304021489\\
1792	-22.7024202604366\\
1793	-19.3703874675052\\
1794	-17.5973104373832\\
1796	-29.5019188757378\\
1797	-37.9783819779564\\
1798	-30.3841657333501\\
1799	-24.9021125051754\\
1800	-17.8266750804851\\
1801	-17.5819848314293\\
1802	-18.7962660106839\\
1803	-18.9544547859141\\
1804	-15.0646951378587\\
1805	-16.2646809755402\\
};
\addlegendentry{MPO prediction}

\end{axis}

\begin{axis}[%
width=6.159cm,
height=1.831cm,
at={(8.104cm,10.169cm)},
scale only axis,
xmin=1000,
xmax=2000,
xlabel style={font=\color{white!15!black}},
xlabel={Sample index},
ymin=-60.4542102409581,
ymax=0,
ylabel style={font=\color{white!15!black}},
ylabel={$y(t)$},
axis background/.style={fill=white},
title style={font=\bfseries},
title={C2: RMSE(OSA) = 2.6945, RMSE(MPO) = 5.5293},
legend style={legend cell align=left, align=left, draw=white!15!black}
]
\addplot [color=mycolor1, line width=2.0pt]
  table[row sep=crcr]{%
1006	-24.414\\
1007	-28.076\\
1008	-23.193\\
1009	-13.4280000000001\\
1010	-17.0899999999999\\
1011	-17.0899999999999\\
1012	-18.3109999999999\\
1013	-15.8689999999999\\
1014	-8.54500000000007\\
1015	-6.10400000000004\\
1016	-7.32400000000007\\
1017	-9.76600000000008\\
1018	-21.973\\
1019	-23.193\\
1020	-23.193\\
1021	-19.5309999999999\\
1022	-10.9860000000001\\
1023	-17.0899999999999\\
1024	-19.5309999999999\\
1025	-13.4280000000001\\
1026	-18.3109999999999\\
1027	-19.5309999999999\\
1028	-14.6479999999999\\
1029	-24.414\\
1030	-21.973\\
1031	-15.8689999999999\\
1032	-23.193\\
1033	-25.635\\
1034	-19.5309999999999\\
1036	-14.6479999999999\\
1037	-17.0899999999999\\
1038	-20.752\\
1039	-18.3109999999999\\
1040	-18.3109999999999\\
1042	-20.752\\
1043	-28.076\\
1044	-25.635\\
1045	-17.0899999999999\\
1046	-15.8689999999999\\
1047	-12.2070000000001\\
1048	-10.9860000000001\\
1049	-15.8689999999999\\
1050	-12.2070000000001\\
1051	-12.2070000000001\\
1052	-15.8689999999999\\
1053	-18.3109999999999\\
1054	-15.8689999999999\\
1055	-25.635\\
1056	-21.973\\
1057	-12.2070000000001\\
1058	-9.76600000000008\\
1059	-13.4280000000001\\
1060	-15.8689999999999\\
1061	-10.9860000000001\\
1062	-10.9860000000001\\
1063	-12.2070000000001\\
1064	-9.76600000000008\\
1065	-8.54500000000007\\
1066	-12.2070000000001\\
1068	-17.0899999999999\\
1069	-18.3109999999999\\
1070	-23.193\\
1071	-21.973\\
1072	-21.973\\
1073	-17.0899999999999\\
1075	-17.0899999999999\\
1076	-15.8689999999999\\
1077	-17.0899999999999\\
1078	-24.414\\
1079	-35.4000000000001\\
1080	-36.6210000000001\\
1081	-34.1800000000001\\
1082	-24.414\\
1083	-29.297\\
1084	-36.6210000000001\\
1085	-40.2829999999999\\
1086	-31.7380000000001\\
1087	-37.8420000000001\\
1088	-48.828\\
1089	-39.0630000000001\\
1090	-30.518\\
1091	-25.635\\
1092	-19.5309999999999\\
1093	-14.6479999999999\\
1094	-19.5309999999999\\
1095	-15.8689999999999\\
1096	-9.76600000000008\\
1097	-9.76600000000008\\
1098	-12.2070000000001\\
1099	-17.0899999999999\\
1100	-15.8689999999999\\
1101	-15.8689999999999\\
1102	-19.5309999999999\\
1103	-19.5309999999999\\
1104	-28.076\\
1105	-23.193\\
1106	-25.635\\
1107	-23.193\\
1108	-18.3109999999999\\
1109	-20.752\\
1110	-17.0899999999999\\
1111	-14.6479999999999\\
1112	-18.3109999999999\\
1114	-15.8689999999999\\
1116	-10.9860000000001\\
1118	-13.4280000000001\\
1119	-10.9860000000001\\
1120	-9.76600000000008\\
1122	-19.5309999999999\\
1123	-23.193\\
1124	-23.193\\
1125	-17.0899999999999\\
1126	-18.3109999999999\\
1127	-30.518\\
1128	-23.193\\
1129	-24.414\\
1130	-26.855\\
1131	-17.0899999999999\\
1132	-10.9860000000001\\
1133	-17.0899999999999\\
1134	-18.3109999999999\\
1135	-26.855\\
1136	-34.1800000000001\\
1137	-31.7380000000001\\
1138	-25.635\\
1140	-23.193\\
1141	-23.193\\
1142	-21.973\\
1144	-14.6479999999999\\
1146	-12.2070000000001\\
1147	-13.4280000000001\\
1149	-25.635\\
1150	-21.973\\
1151	-14.6479999999999\\
1152	-13.4280000000001\\
1153	-14.6479999999999\\
1154	-12.2070000000001\\
1155	-8.54500000000007\\
1156	-8.54500000000007\\
1157	-13.4280000000001\\
1158	-12.2070000000001\\
1161	-12.2070000000001\\
1162	-8.54500000000007\\
1163	-7.32400000000007\\
1164	-4.88300000000004\\
1165	-7.32400000000007\\
1166	-17.0899999999999\\
1167	-24.414\\
1168	-24.414\\
1169	-19.5309999999999\\
1170	-12.2070000000001\\
1171	-12.2070000000001\\
1172	-8.54500000000007\\
1173	-10.9860000000001\\
1174	-14.6479999999999\\
1175	-14.6479999999999\\
1177	-34.1800000000001\\
1178	-40.2829999999999\\
1179	-32.9590000000001\\
1180	-31.7380000000001\\
1181	-23.193\\
1182	-25.635\\
1184	-28.076\\
1185	-21.973\\
1187	-19.5309999999999\\
1188	-19.5309999999999\\
1189	-21.973\\
1190	-19.5309999999999\\
1191	-23.193\\
1192	-29.297\\
1193	-24.414\\
1194	-14.6479999999999\\
1195	-15.8689999999999\\
1196	-25.635\\
1197	-32.9590000000001\\
1198	-35.4000000000001\\
1199	-39.0630000000001\\
1200	-37.8420000000001\\
1201	-30.518\\
1202	-26.855\\
1203	-30.518\\
1204	-35.4000000000001\\
1205	-25.635\\
1206	-14.6479999999999\\
1207	-19.5309999999999\\
1208	-20.752\\
1209	-12.2070000000001\\
1210	-13.4280000000001\\
1211	-18.3109999999999\\
1212	-14.6479999999999\\
1213	-12.2070000000001\\
1214	-18.3109999999999\\
1215	-10.9860000000001\\
1216	-14.6479999999999\\
1217	-24.414\\
1218	-21.973\\
1219	-14.6479999999999\\
1220	-14.6479999999999\\
1221	-20.752\\
1222	-34.1800000000001\\
1223	-28.076\\
1224	-15.8689999999999\\
1225	-14.6479999999999\\
1226	-14.6479999999999\\
1227	-12.2070000000001\\
1228	-10.9860000000001\\
1229	-10.9860000000001\\
1230	-13.4280000000001\\
1231	-17.0899999999999\\
1232	-19.5309999999999\\
1233	-14.6479999999999\\
1234	-19.5309999999999\\
1235	-26.855\\
1236	-23.193\\
1237	-17.0899999999999\\
1238	-21.973\\
1239	-25.635\\
1240	-21.973\\
1241	-8.54500000000007\\
1242	-12.2070000000001\\
1243	-13.4280000000001\\
1244	-15.8689999999999\\
1245	-15.8689999999999\\
1246	-10.9860000000001\\
1247	-17.0899999999999\\
1248	-15.8689999999999\\
1249	-10.9860000000001\\
1250	-14.6479999999999\\
1251	-14.6479999999999\\
1252	-12.2070000000001\\
1253	-14.6479999999999\\
1254	-10.9860000000001\\
1255	-9.76600000000008\\
1256	-12.2070000000001\\
1257	-8.54500000000007\\
1258	-17.0899999999999\\
1259	-18.3109999999999\\
1260	-20.752\\
1261	-29.297\\
1262	-20.752\\
1263	-10.9860000000001\\
1264	-10.9860000000001\\
1265	-8.54500000000007\\
1266	-13.4280000000001\\
1267	-12.2070000000001\\
1268	-9.76600000000008\\
1269	-15.8689999999999\\
1270	-19.5309999999999\\
1271	-20.752\\
1272	-23.193\\
1273	-23.193\\
1274	-17.0899999999999\\
1275	-15.8689999999999\\
1276	-10.9860000000001\\
1277	-10.9860000000001\\
1278	-7.32400000000007\\
1279	-8.54500000000007\\
1280	-10.9860000000001\\
1281	-17.0899999999999\\
1282	-18.3109999999999\\
1283	-17.0899999999999\\
1284	-24.414\\
1285	-25.635\\
1286	-15.8689999999999\\
1289	-26.855\\
1290	-25.635\\
1291	-17.0899999999999\\
1292	-10.9860000000001\\
1293	-7.32400000000007\\
1294	-6.10400000000004\\
1295	-8.54500000000007\\
1296	-9.76600000000008\\
1297	-8.54500000000007\\
1298	-10.9860000000001\\
1299	-17.0899999999999\\
1300	-15.8689999999999\\
1301	-13.4280000000001\\
1302	-15.8689999999999\\
1303	-12.2070000000001\\
1304	-6.10400000000004\\
1305	-9.76600000000008\\
1306	-20.752\\
1307	-17.0899999999999\\
1308	-18.3109999999999\\
1309	-15.8689999999999\\
1310	-10.9860000000001\\
1312	-23.193\\
1313	-23.193\\
1314	-17.0899999999999\\
1315	-26.855\\
1316	-25.635\\
1317	-18.3109999999999\\
1318	-23.193\\
1319	-25.635\\
1320	-19.5309999999999\\
1321	-15.8689999999999\\
1322	-24.414\\
1323	-35.4000000000001\\
1324	-29.297\\
1325	-17.0899999999999\\
1326	-17.0899999999999\\
1327	-18.3109999999999\\
1329	-23.193\\
1330	-19.5309999999999\\
1331	-17.0899999999999\\
1332	-24.414\\
1333	-25.635\\
1334	-18.3109999999999\\
1335	-15.8689999999999\\
1336	-18.3109999999999\\
1337	-28.076\\
1338	-26.855\\
1339	-24.414\\
1340	-19.5309999999999\\
1341	-17.0899999999999\\
1342	-18.3109999999999\\
1343	-13.4280000000001\\
1344	-6.10400000000004\\
1346	-6.10400000000004\\
1347	-9.76600000000008\\
1348	-19.5309999999999\\
1349	-15.8689999999999\\
1351	-13.4280000000001\\
1352	-17.0899999999999\\
1354	-9.76600000000008\\
1355	-12.2070000000001\\
1356	-9.76600000000008\\
1357	-8.54500000000007\\
1358	-17.0899999999999\\
1359	-20.752\\
1360	-17.0899999999999\\
1361	-10.9860000000001\\
1362	-10.9860000000001\\
1363	-13.4280000000001\\
1364	-9.76600000000008\\
1365	-10.9860000000001\\
1366	-28.076\\
1367	-20.752\\
1368	-25.635\\
1369	-35.4000000000001\\
1370	-30.518\\
1372	-18.3109999999999\\
1373	-18.3109999999999\\
1374	-19.5309999999999\\
1375	-21.973\\
1376	-25.635\\
1377	-25.635\\
1378	-31.7380000000001\\
1379	-34.1800000000001\\
1380	-41.5039999999999\\
1381	-32.9590000000001\\
1382	-28.076\\
1383	-35.4000000000001\\
1384	-26.855\\
1385	-29.297\\
1386	-28.076\\
1387	-17.0899999999999\\
1388	-10.9860000000001\\
1389	-12.2070000000001\\
1390	-17.0899999999999\\
1391	-14.6479999999999\\
1392	-9.76600000000008\\
1393	-13.4280000000001\\
1394	-13.4280000000001\\
1395	-7.32400000000007\\
1396	-8.54500000000007\\
1397	-12.2070000000001\\
1398	-7.32400000000007\\
1399	-14.6479999999999\\
1400	-20.752\\
1401	-13.4280000000001\\
1403	-30.518\\
1404	-29.297\\
1405	-35.4000000000001\\
1406	-29.297\\
1407	-28.076\\
1408	-23.193\\
1409	-12.2070000000001\\
1410	-14.6479999999999\\
1411	-15.8689999999999\\
1412	-10.9860000000001\\
1414	-13.4280000000001\\
1415	-3.66200000000003\\
1416	-8.54500000000007\\
1417	-17.0899999999999\\
1418	-19.5309999999999\\
1419	-20.752\\
1420	-23.193\\
1421	-20.752\\
1422	-23.193\\
1423	-18.3109999999999\\
1424	-15.8689999999999\\
1425	-15.8689999999999\\
1426	-12.2070000000001\\
1427	-18.3109999999999\\
1428	-17.0899999999999\\
1429	-12.2070000000001\\
1430	-17.0899999999999\\
1431	-23.193\\
1432	-17.0899999999999\\
1433	-13.4280000000001\\
1434	-12.2070000000001\\
1435	-14.6479999999999\\
1436	-7.32400000000007\\
1437	-8.54500000000007\\
1438	-12.2070000000001\\
1439	-17.0899999999999\\
1440	-17.0899999999999\\
1441	-14.6479999999999\\
1442	-15.8689999999999\\
1443	-18.3109999999999\\
1444	-12.2070000000001\\
1445	-9.76600000000008\\
1447	-26.855\\
1448	-26.855\\
1449	-21.973\\
1452	-14.6479999999999\\
1453	-13.4280000000001\\
1454	-17.0899999999999\\
1455	-17.0899999999999\\
1456	-10.9860000000001\\
1458	-18.3109999999999\\
1459	-19.5309999999999\\
1460	-21.973\\
1461	-21.973\\
1462	-29.297\\
1463	-35.4000000000001\\
1464	-36.6210000000001\\
1465	-26.855\\
1466	-14.6479999999999\\
1468	-7.32400000000007\\
1469	-10.9860000000001\\
1470	-10.9860000000001\\
1471	-14.6479999999999\\
1473	-12.2070000000001\\
1474	-15.8689999999999\\
1475	-20.752\\
1476	-20.752\\
1477	-24.414\\
1478	-31.7380000000001\\
1480	-26.855\\
1481	-18.3109999999999\\
1482	-17.0899999999999\\
1483	-20.752\\
1484	-20.752\\
1485	-17.0899999999999\\
1486	-14.6479999999999\\
1487	-18.3109999999999\\
1488	-12.2070000000001\\
1489	-13.4280000000001\\
1490	-20.752\\
1491	-17.0899999999999\\
1492	-18.3109999999999\\
1493	-32.9590000000001\\
1494	-25.635\\
1495	-20.752\\
1496	-28.076\\
1497	-28.076\\
1498	-19.5309999999999\\
1499	-13.4280000000001\\
1500	-13.4280000000001\\
1501	-19.5309999999999\\
1502	-30.518\\
1503	-34.1800000000001\\
1504	-31.7380000000001\\
1505	-25.635\\
1506	-17.0899999999999\\
1507	-14.6479999999999\\
1508	-13.4280000000001\\
1510	-8.54500000000007\\
1511	-12.2070000000001\\
1512	-14.6479999999999\\
1513	-10.9860000000001\\
1514	-13.4280000000001\\
1515	-19.5309999999999\\
1517	-21.973\\
1518	-29.297\\
1519	-34.1800000000001\\
1520	-25.635\\
1521	-23.193\\
1522	-24.414\\
1523	-20.752\\
1524	-14.6479999999999\\
1525	-18.3109999999999\\
1526	-29.297\\
1527	-31.7380000000001\\
1528	-19.5309999999999\\
1529	-12.2070000000001\\
1530	-8.54500000000007\\
1531	-7.32400000000007\\
1532	-9.76600000000008\\
1533	-8.54500000000007\\
1534	-12.2070000000001\\
1535	-14.6479999999999\\
1536	-13.4280000000001\\
1537	-9.76600000000008\\
1538	-10.9860000000001\\
1539	-10.9860000000001\\
1540	-9.76600000000008\\
1541	-10.9860000000001\\
1542	-14.6479999999999\\
1543	-21.973\\
1544	-18.3109999999999\\
1545	-20.752\\
1546	-21.973\\
1547	-30.518\\
1548	-31.7380000000001\\
1549	-34.1800000000001\\
1550	-34.1800000000001\\
1551	-24.414\\
1552	-17.0899999999999\\
1553	-15.8689999999999\\
1554	-26.855\\
1555	-31.7380000000001\\
1556	-23.193\\
1557	-18.3109999999999\\
1558	-9.76600000000008\\
1559	-4.88300000000004\\
1560	-6.10400000000004\\
1561	-6.10400000000004\\
1562	-7.32400000000007\\
1563	-14.6479999999999\\
1564	-10.9860000000001\\
1565	-12.2070000000001\\
1567	-4.88300000000004\\
1568	-8.54500000000007\\
1570	-8.54500000000007\\
1571	-9.76600000000008\\
1572	-7.32400000000007\\
1573	-10.9860000000001\\
1574	-24.414\\
1575	-26.855\\
1576	-25.635\\
1577	-21.973\\
1578	-30.518\\
1579	-40.2829999999999\\
1580	-34.1800000000001\\
1581	-31.7380000000001\\
1582	-26.855\\
1584	-29.297\\
1585	-20.752\\
1586	-17.0899999999999\\
1588	-12.2070000000001\\
1589	-20.752\\
1591	-15.8689999999999\\
1592	-18.3109999999999\\
1593	-18.3109999999999\\
1594	-14.6479999999999\\
1595	-17.0899999999999\\
1596	-17.0899999999999\\
1597	-20.752\\
1598	-19.5309999999999\\
1599	-13.4280000000001\\
1600	-17.0899999999999\\
1601	-9.76600000000008\\
1602	-4.88300000000004\\
1603	-10.9860000000001\\
1604	-13.4280000000001\\
1605	-8.54500000000007\\
1606	-2.44100000000003\\
1607	-8.54500000000007\\
1608	-12.2070000000001\\
1609	-12.2070000000001\\
1610	-15.8689999999999\\
1611	-13.4280000000001\\
1612	-18.3109999999999\\
1613	-17.0899999999999\\
1614	-10.9860000000001\\
1615	-17.0899999999999\\
1616	-14.6479999999999\\
1617	-9.76600000000008\\
1618	-7.32400000000007\\
1619	-6.10400000000004\\
1621	-8.54500000000007\\
1622	-6.10400000000004\\
1623	-13.4280000000001\\
1624	-17.0899999999999\\
1625	-10.9860000000001\\
1626	-14.6479999999999\\
1627	-10.9860000000001\\
1628	-14.6479999999999\\
1629	-10.9860000000001\\
1630	-9.76600000000008\\
1631	-9.76600000000008\\
1632	-7.32400000000007\\
1633	-8.54500000000007\\
1634	-10.9860000000001\\
1635	-14.6479999999999\\
1636	-12.2070000000001\\
1637	-8.54500000000007\\
1639	-8.54500000000007\\
1640	-10.9860000000001\\
1641	-21.973\\
1642	-24.414\\
1643	-18.3109999999999\\
1644	-17.0899999999999\\
1645	-13.4280000000001\\
1646	-19.5309999999999\\
1647	-17.0899999999999\\
1648	-9.76600000000008\\
1649	-14.6479999999999\\
1650	-13.4280000000001\\
1651	-15.8689999999999\\
1652	-12.2070000000001\\
1653	-14.6479999999999\\
1654	-12.2070000000001\\
1655	-15.8689999999999\\
1656	-12.2070000000001\\
1658	-14.6479999999999\\
1659	-19.5309999999999\\
1660	-18.3109999999999\\
1661	-21.973\\
1662	-30.518\\
1663	-24.414\\
1664	-14.6479999999999\\
1665	-14.6479999999999\\
1666	-15.8689999999999\\
1667	-23.193\\
1668	-18.3109999999999\\
1669	-23.193\\
1670	-17.0899999999999\\
1671	-8.54500000000007\\
1672	-9.76600000000008\\
1673	-19.5309999999999\\
1674	-14.6479999999999\\
1675	-13.4280000000001\\
1676	-20.752\\
1677	-21.973\\
1679	-14.6479999999999\\
1680	-18.3109999999999\\
1681	-20.752\\
1683	-15.8689999999999\\
1684	-15.8689999999999\\
1685	-12.2070000000001\\
1686	-14.6479999999999\\
1687	-12.2070000000001\\
1688	-13.4280000000001\\
1689	-19.5309999999999\\
1690	-20.752\\
1691	-19.5309999999999\\
1692	-19.5309999999999\\
1693	-14.6479999999999\\
1694	-10.9860000000001\\
1695	-13.4280000000001\\
1696	-14.6479999999999\\
1697	-18.3109999999999\\
1698	-20.752\\
1699	-24.414\\
1701	-12.2070000000001\\
1702	-21.973\\
1703	-30.518\\
1704	-20.752\\
1705	-20.752\\
1706	-25.635\\
1707	-18.3109999999999\\
1708	-17.0899999999999\\
1709	-18.3109999999999\\
1710	-15.8689999999999\\
1711	-18.3109999999999\\
1712	-21.973\\
1713	-17.0899999999999\\
1714	-20.752\\
1715	-20.752\\
1716	-18.3109999999999\\
1717	-14.6479999999999\\
1718	-21.973\\
1719	-14.6479999999999\\
1720	-17.0899999999999\\
1722	-17.0899999999999\\
1723	-10.9860000000001\\
1724	-6.10400000000004\\
1725	-4.88300000000004\\
1726	-10.9860000000001\\
1727	-20.752\\
1728	-21.973\\
1729	-19.5309999999999\\
1730	-13.4280000000001\\
1731	-17.0899999999999\\
1732	-14.6479999999999\\
1733	-14.6479999999999\\
1734	-9.76600000000008\\
1735	-15.8689999999999\\
1736	-14.6479999999999\\
1737	-12.2070000000001\\
1738	-13.4280000000001\\
1739	-12.2070000000001\\
1741	-7.32400000000007\\
1742	-8.54500000000007\\
1743	-18.3109999999999\\
1744	-13.4280000000001\\
1745	-18.3109999999999\\
1746	-15.8689999999999\\
1747	-21.973\\
1748	-19.5309999999999\\
1749	-25.635\\
1750	-15.8689999999999\\
1751	-23.193\\
1752	-21.973\\
1753	-13.4280000000001\\
1754	-17.0899999999999\\
1755	-14.6479999999999\\
1756	-19.5309999999999\\
1757	-20.752\\
1758	-24.414\\
1759	-17.0899999999999\\
1760	-24.414\\
1761	-25.635\\
1762	-23.193\\
1763	-18.3109999999999\\
1764	-14.6479999999999\\
1765	-19.5309999999999\\
1768	-12.2070000000001\\
1769	-13.4280000000001\\
1770	-12.2070000000001\\
1771	-25.635\\
1772	-30.518\\
1773	-25.635\\
1774	-23.193\\
1775	-25.635\\
1776	-14.6479999999999\\
1777	-13.4280000000001\\
1778	-9.76600000000008\\
1779	-9.76600000000008\\
1780	-12.2070000000001\\
1781	-18.3109999999999\\
1782	-19.5309999999999\\
1783	-23.193\\
1784	-18.3109999999999\\
1785	-15.8689999999999\\
1786	-19.5309999999999\\
1787	-28.076\\
1788	-28.076\\
1789	-23.193\\
1790	-37.8420000000001\\
1791	-25.635\\
1792	-15.8689999999999\\
1793	-13.4280000000001\\
1794	-12.2070000000001\\
1795	-19.5309999999999\\
1796	-23.193\\
1797	-28.076\\
1798	-24.414\\
1799	-18.3109999999999\\
1800	-10.9860000000001\\
1801	-14.6479999999999\\
1802	-10.9860000000001\\
1803	-13.4280000000001\\
1804	-8.54500000000007\\
1805	-10.9860000000001\\
};
\addlegendentry{True output}

\addplot [color=mycolor2, dashed, line width=2.0pt]
  table[row sep=crcr]{%
1006	-25.7061036614996\\
1007	-28.0556469959067\\
1008	-24.2828081013954\\
1009	-15.3073418931401\\
1010	-17.2179207345996\\
1011	-18.3035331647829\\
1012	-19.0108879578029\\
1013	-18.0730805047065\\
1014	-11.8362512594515\\
1015	-14.8373294630364\\
1016	-9.98567388227889\\
1017	-10.371271255666\\
1018	-26.2413412597543\\
1019	-24.4256428208673\\
1020	-23.4086378695351\\
1021	-18.5793969238809\\
1022	-16.3028971698375\\
1023	-21.9669446496684\\
1024	-15.8913263518259\\
1025	-15.7440314001115\\
1026	-19.4091670775554\\
1027	-18.7357255089578\\
1028	-17.9544294370398\\
1029	-22.5434985642535\\
1030	-25.1285084192473\\
1031	-18.1343966402014\\
1032	-23.8619501674345\\
1033	-24.0002183624567\\
1034	-20.8570750384292\\
1035	-18.3545462012376\\
1036	-16.3880205555167\\
1037	-17.7254150149608\\
1038	-20.4433857409981\\
1039	-19.855220725731\\
1040	-20.2123751339288\\
1041	-19.5917669657704\\
1042	-21.468989187432\\
1043	-28.0800862021715\\
1044	-26.2614197290636\\
1045	-19.4001553012949\\
1046	-18.6176840779319\\
1047	-13.6734426666055\\
1048	-14.0179370617179\\
1049	-14.6848476115506\\
1050	-13.9327051666583\\
1051	-13.8018062543213\\
1052	-16.4517744027901\\
1053	-18.9080799466924\\
1054	-18.0684220763701\\
1055	-24.255000303709\\
1056	-21.6679047666514\\
1057	-16.1858634133484\\
1058	-14.2350382493321\\
1059	-14.2468817834567\\
1060	-15.7485283839685\\
1061	-12.3853350728564\\
1062	-12.457409648541\\
1063	-13.2612577032905\\
1064	-12.8470258840116\\
1065	-11.353721862258\\
1066	-11.5974224991664\\
1067	-13.4844501188973\\
1068	-19.6425692637083\\
1069	-19.388101841548\\
1070	-20.848263870158\\
1071	-21.67564107312\\
1072	-24.9865421424222\\
1073	-21.05298256573\\
1074	-18.4206542740603\\
1075	-17.7861807614072\\
1076	-17.6371989179822\\
1077	-16.8597592590784\\
1078	-25.9180952363392\\
1079	-39.6538971267182\\
1080	-38.4496344320239\\
1081	-36.9778784406972\\
1082	-22.9595713962342\\
1083	-33.3242143887755\\
1084	-41.0899999458436\\
1085	-39.1640586361887\\
1086	-31.143899303499\\
1087	-40.1026984143746\\
1088	-55.858466347114\\
1089	-37.1621721873532\\
1090	-33.1486621025872\\
1091	-21.4836441636653\\
1092	-20.5325514912661\\
1093	-18.4722182481289\\
1094	-19.759351165897\\
1095	-17.517527746445\\
1096	-12.407061244598\\
1097	-12.0088352311552\\
1098	-12.6432524041088\\
1099	-17.812562048628\\
1100	-16.9827098550063\\
1101	-17.1806522065808\\
1102	-20.8868459521375\\
1103	-22.6044883730915\\
1104	-27.7370468585102\\
1105	-25.3367357579107\\
1106	-27.5228609241369\\
1107	-23.1327937215538\\
1108	-20.581298051068\\
1109	-22.5945948607584\\
1110	-18.2668442152224\\
1111	-17.1080964337209\\
1112	-19.0257040106396\\
1113	-18.909032860661\\
1114	-15.4551795431066\\
1115	-16.3117903118202\\
1116	-13.385211033682\\
1117	-13.8078559526641\\
1118	-13.7834638449137\\
1119	-12.3790939283467\\
1120	-13.1533380654803\\
1121	-14.3148207422341\\
1122	-21.6467865754148\\
1123	-22.2173729935018\\
1124	-23.9880336651202\\
1125	-17.0690845156048\\
1126	-21.7924107427334\\
1127	-31.1974843416299\\
1128	-25.879162179865\\
1129	-25.996471208744\\
1130	-27.9698442238428\\
1131	-17.1094693465393\\
1132	-16.2688547723062\\
1133	-16.070487035038\\
1134	-19.7941571183842\\
1135	-30.3049834347344\\
1136	-35.803863574494\\
1137	-34.4718543570834\\
1138	-26.3690727146397\\
1139	-26.6976996870785\\
1140	-24.1523578716267\\
1141	-24.3790363518801\\
1142	-24.3476058568683\\
1143	-17.7471632587949\\
1144	-16.9551170654104\\
1145	-15.5958323455154\\
1146	-14.3685462519068\\
1147	-15.3067530385442\\
1148	-19.3116160903171\\
1149	-29.9091690700575\\
1150	-22.7707531129822\\
1151	-17.4564256203059\\
1152	-16.2491924532408\\
1153	-15.2723939383893\\
1154	-12.4617189795056\\
1155	-12.6483191197167\\
1156	-10.4974948061595\\
1157	-13.378270023381\\
1158	-12.7641333705214\\
1159	-13.6436247190748\\
1160	-13.4559640578241\\
1161	-13.6595189311431\\
1162	-12.0699111424703\\
1163	-10.9241852468401\\
1164	-10.2331983291208\\
1165	-7.5489826954481\\
1166	-16.4812499094251\\
1167	-28.4186097215027\\
1168	-27.7563255695725\\
1169	-19.3949253589601\\
1170	-16.4753997742514\\
1171	-13.8774935818851\\
1172	-13.021738342376\\
1173	-13.3630591896242\\
1174	-13.6325513578061\\
1175	-18.8162532564077\\
1176	-22.6745212635949\\
1177	-41.3115591724738\\
1178	-47.7289341263506\\
1179	-35.9179477913183\\
1180	-34.3078748139403\\
1181	-21.1366273445144\\
1182	-27.4287990598036\\
1183	-27.3291670237584\\
1184	-29.7527118415708\\
1185	-24.1374957937603\\
1186	-21.2904086401297\\
1187	-22.3932089105238\\
1188	-19.8270513000309\\
1189	-22.8390277827591\\
1190	-18.4398846251167\\
1191	-28.4641153383689\\
1192	-33.8823964194603\\
1193	-22.4540079440701\\
1194	-16.937000876625\\
1195	-17.0282290539196\\
1196	-27.7940689973627\\
1197	-33.7341246445676\\
1198	-39.3011202831512\\
1199	-40.4357154579002\\
1200	-40.4008831397637\\
1201	-30.5340328390498\\
1202	-29.3110260466483\\
1203	-32.6187453170503\\
1204	-37.7724263248465\\
1205	-23.6387704386316\\
1206	-16.6789121876297\\
1207	-22.3360028585619\\
1208	-19.8504231236561\\
1209	-15.1977117439515\\
1210	-16.2590637838189\\
1211	-18.8152512183556\\
1212	-16.2757732783637\\
1213	-14.1100475343674\\
1214	-15.3895865959605\\
1215	-16.7631900563424\\
1216	-16.8004403119623\\
1217	-23.9078759900324\\
1218	-22.6278784330136\\
1219	-17.4777919406226\\
1220	-16.8374766778118\\
1221	-22.899431571044\\
1222	-41.0109907717067\\
1223	-25.7932539598282\\
1224	-18.6838133257561\\
1225	-15.6331960359769\\
1226	-18.5185478986173\\
1227	-16.1763985110285\\
1228	-12.9356997618606\\
1229	-13.1650685061065\\
1230	-14.6975512162153\\
1232	-20.6169461053241\\
1233	-15.0353753869479\\
1234	-22.051586525698\\
1235	-26.7400167276767\\
1236	-23.9531759269428\\
1237	-21.5930958416764\\
1238	-20.3413645678254\\
1239	-29.1129131959888\\
1240	-18.8199625705308\\
1241	-14.5477127470463\\
1242	-13.4681954526209\\
1243	-15.022956640245\\
1244	-18.8700259100356\\
1245	-16.4283817912415\\
1246	-14.4189064948248\\
1247	-17.7079081261063\\
1248	-18.2697883494216\\
1249	-14.6418321761546\\
1250	-15.0940128510056\\
1251	-15.6035134402396\\
1252	-14.2284370036298\\
1253	-15.2778319230447\\
1254	-12.2862527273614\\
1255	-13.1355302087827\\
1256	-10.7513425380546\\
1257	-12.5035239628567\\
1258	-14.4989440138138\\
1260	-24.0920672440491\\
1261	-32.6624354674282\\
1262	-22.0294512519831\\
1263	-15.8962588706081\\
1264	-13.5943723787516\\
1265	-12.8847045399505\\
1266	-13.9986205687651\\
1267	-10.9130386098775\\
1268	-12.952798445614\\
1269	-13.1528548866188\\
1270	-24.9163530574476\\
1271	-21.3114202702548\\
1272	-26.7984065942765\\
1273	-19.0939118221995\\
1274	-20.0103497857401\\
1275	-14.0926498697743\\
1276	-15.4253429535393\\
1277	-12.6104864202205\\
1278	-11.8466468617994\\
1279	-10.311658832811\\
1280	-13.2683318166526\\
1281	-16.1352927024529\\
1282	-16.7335661403852\\
1283	-18.6324932800933\\
1284	-28.5668002466687\\
1285	-25.9098368300274\\
1286	-20.1879683346297\\
1287	-21.2140898260318\\
1288	-22.2067957605284\\
1289	-29.5552740595897\\
1290	-22.5000543875792\\
1291	-19.0169548664494\\
1292	-14.6725929610732\\
1293	-14.4158958761091\\
1294	-10.5807889787066\\
1295	-8.99537804824763\\
1296	-9.83755749434317\\
1297	-10.1406179542014\\
1298	-10.6731000290931\\
1299	-15.9345476541689\\
1300	-16.2927841593698\\
1301	-16.6936205006757\\
1302	-16.7961557315382\\
1303	-13.6195946044222\\
1304	-13.6155583509387\\
1305	-9.9808979083939\\
1306	-17.671189916759\\
1307	-19.047008966729\\
1308	-19.1703585528101\\
1309	-17.944330286419\\
1310	-15.5822282786103\\
1311	-16.8300317696342\\
1312	-22.825692450589\\
1313	-24.0406403255492\\
1314	-17.3314139626495\\
1315	-26.0321628360475\\
1316	-24.790496744462\\
1317	-22.4399732489569\\
1318	-24.259549546932\\
1319	-24.326645336542\\
1320	-21.1951739320268\\
1321	-18.4963379060207\\
1322	-26.9918689334986\\
1323	-37.8172086109043\\
1324	-28.4548528196017\\
1325	-18.6420430471685\\
1326	-19.0223793701282\\
1327	-19.2216486665602\\
1328	-22.4427523423421\\
1329	-24.7994708809924\\
1330	-19.1637552765887\\
1331	-20.714553186632\\
1332	-24.3476941318299\\
1333	-27.0847091570836\\
1334	-19.4957484056526\\
1335	-16.8332877637465\\
1336	-20.1653612412515\\
1337	-32.945606140308\\
1338	-26.7622738433581\\
1339	-26.0922744918846\\
1340	-18.4303829306718\\
1341	-17.8823473126115\\
1342	-17.9940080553097\\
1343	-14.825483685712\\
1344	-13.581724562305\\
1345	-10.9486124825305\\
1346	-9.2905728844687\\
1347	-9.05916240173406\\
1349	-21.4447244750156\\
1350	-14.8184216663046\\
1351	-16.5416472403476\\
1352	-17.5160663897689\\
1353	-13.3231311289926\\
1354	-14.3649789496897\\
1355	-12.3819992268145\\
1356	-12.0512201059289\\
1357	-12.4099023192514\\
1358	-16.4791288748345\\
1359	-22.1763178030965\\
1360	-15.2134329274554\\
1361	-15.0837063940191\\
1362	-13.1286765887501\\
1363	-13.0653906683146\\
1364	-12.4209800295891\\
1365	-15.6156690102839\\
1366	-30.2186572150724\\
1367	-22.2236744052257\\
1368	-26.7308073982786\\
1369	-32.5277657210852\\
1370	-34.4094621602756\\
1371	-25.8081034219827\\
1372	-20.4752288790189\\
1373	-20.6234682352949\\
1374	-20.5344615400581\\
1375	-22.8689273834439\\
1376	-25.2728350769773\\
1377	-25.4214566016617\\
1378	-33.2755446367369\\
1379	-39.3680657675752\\
1380	-43.0135509582319\\
1381	-30.6019745137723\\
1382	-32.4144516621614\\
1383	-36.1494228399736\\
1384	-29.9522370877903\\
1385	-31.6498495202261\\
1386	-28.0262603224382\\
1387	-16.8976167048704\\
1388	-14.4448289520371\\
1389	-13.656061078153\\
1390	-17.4797932249221\\
1391	-15.612507033856\\
1392	-12.9222208509254\\
1393	-14.2184637570008\\
1394	-12.6732642086099\\
1395	-12.6906484874062\\
1396	-11.0427669199155\\
1397	-11.6029836705698\\
1398	-11.6856924421725\\
1399	-13.0304254305038\\
1400	-20.3267552271252\\
1401	-16.1536910658303\\
1402	-24.4584276760493\\
1403	-39.4425377213672\\
1404	-29.6628657295728\\
1405	-36.023265943199\\
1406	-28.177033338462\\
1407	-29.0498957388484\\
1408	-22.6513778973797\\
1409	-16.8007440565709\\
1410	-17.3544143202703\\
1411	-14.8137644839755\\
1412	-14.3883881839508\\
1413	-14.4029190554377\\
1414	-10.6214514228845\\
1415	-13.4336212042579\\
1416	-8.4843078957324\\
1417	-13.6046009690672\\
1418	-21.0318133691026\\
1419	-23.0389406112042\\
1420	-22.863273647675\\
1421	-22.5585193112347\\
1422	-23.9253696128001\\
1423	-21.4849185002254\\
1424	-17.7155731208559\\
1425	-18.3458484702508\\
1426	-15.6851132184365\\
1427	-19.6817759435926\\
1428	-14.91910316249\\
1429	-15.5348700399002\\
1430	-16.9079488723312\\
1431	-24.4437848960463\\
1432	-19.2661459721676\\
1433	-16.0132380767106\\
1434	-14.8172822006873\\
1435	-15.5708357853937\\
1436	-13.0907562820655\\
1437	-11.0305469611703\\
1438	-13.0748665438366\\
1439	-15.5879846637076\\
1440	-17.6155197987491\\
1441	-15.6142533739062\\
1442	-18.487792110586\\
1443	-17.822111230583\\
1444	-13.9441310703278\\
1445	-14.0478133760751\\
1446	-19.6056956909695\\
1447	-29.1452663915782\\
1448	-25.5676868200683\\
1449	-22.9078558388314\\
1450	-22.0921153922577\\
1451	-19.487228061929\\
1452	-17.7443668345143\\
1453	-15.5508583226283\\
1454	-17.5501515351837\\
1455	-17.3224111087952\\
1456	-14.1766698868371\\
1458	-18.157968758044\\
1459	-20.9076064980563\\
1460	-22.3444253174375\\
1461	-22.4166557933438\\
1462	-29.3000958600364\\
1463	-38.9912715385542\\
1464	-43.026560250627\\
1465	-23.0857003969725\\
1466	-19.4031276698745\\
1467	-14.4198538423357\\
1468	-13.2416690950395\\
1469	-11.3158015444619\\
1470	-11.3388839052957\\
1471	-17.6815775947507\\
1472	-16.0931120480941\\
1473	-14.5694167281479\\
1474	-17.0691029886689\\
1475	-18.9273540996994\\
1476	-23.7569771640631\\
1477	-24.243322523665\\
1478	-35.944262544255\\
1479	-32.7541607023375\\
1480	-23.3837118186334\\
1481	-20.8011128299866\\
1482	-19.7008111146185\\
1483	-20.01525547495\\
1484	-23.9362499350027\\
1485	-18.5239494996101\\
1486	-17.4885892900379\\
1487	-17.1193570417702\\
1488	-13.6208566958771\\
1489	-17.2524049605145\\
1490	-23.4188734119825\\
1491	-17.4691118136002\\
1492	-17.8045358768866\\
1493	-31.5647811287413\\
1494	-26.1713901061762\\
1495	-23.5563719195688\\
1496	-30.6950186482384\\
1497	-29.8386601682712\\
1498	-19.0836067628049\\
1499	-16.0718945084936\\
1500	-15.5877427612409\\
1501	-21.7521065711583\\
1502	-34.9058671295302\\
1503	-33.9469173275947\\
1504	-33.5735457640465\\
1505	-27.4346881618396\\
1506	-19.1183675516727\\
1507	-17.6054833486639\\
1508	-14.488661500747\\
1509	-13.8475688674605\\
1510	-12.5836652502967\\
1511	-13.2405666919994\\
1512	-14.9838989396533\\
1513	-12.2337369555221\\
1514	-13.8390434703745\\
1515	-19.5332040561932\\
1516	-21.1574632228003\\
1517	-27.4197067513849\\
1518	-32.9277982638991\\
1519	-38.9702463037072\\
1520	-26.5545742910217\\
1521	-23.352643223787\\
1522	-26.4416780450865\\
1523	-23.1831080715219\\
1524	-17.8028842446972\\
1525	-18.5242958775686\\
1526	-31.1720647682946\\
1527	-34.291487056606\\
1528	-19.9972462274425\\
1529	-14.1858483787673\\
1530	-13.5946890790692\\
1531	-12.8588497886096\\
1532	-10.729663406195\\
1533	-9.43034492526385\\
1534	-11.3784201493688\\
1535	-16.3756811065593\\
1536	-14.0808253550051\\
1537	-13.06640103526\\
1538	-12.1408955055076\\
1539	-11.7467562702261\\
1540	-11.7733053786014\\
1541	-11.3590101005411\\
1542	-14.7155316144012\\
1543	-23.5879559238151\\
1544	-24.4186769056869\\
1545	-21.2827726037167\\
1546	-24.4006319348425\\
1547	-30.7470052356218\\
1548	-31.6602059395414\\
1549	-36.7319843174541\\
1550	-35.1812576945188\\
1551	-25.7322626160503\\
1552	-21.2098388257766\\
1553	-18.2658951337382\\
1554	-29.469013062788\\
1555	-32.5973177575077\\
1556	-21.3507907845067\\
1557	-17.9058793011045\\
1558	-14.0843700062933\\
1559	-14.360864361925\\
1560	-9.33045436499583\\
1561	-9.29595540646278\\
1562	-7.27480389796938\\
1563	-10.76417256996\\
1564	-13.9200654198946\\
1565	-12.426172383615\\
1566	-11.6269574673972\\
1567	-11.5223404519452\\
1568	-7.88616306054541\\
1569	-9.35934792765124\\
1570	-9.96477914137654\\
1571	-9.36721806134528\\
1572	-10.2811781057846\\
1573	-9.46956622960238\\
1574	-24.6258074360728\\
1575	-35.1030988232826\\
1576	-30.2417692258821\\
1577	-21.4878923143795\\
1578	-28.739406977397\\
1579	-44.4581685749361\\
1580	-39.2585095221709\\
1581	-36.9109628662279\\
1582	-28.5601586121611\\
1583	-26.7549883505358\\
1584	-30.6867281743296\\
1585	-20.8020555172966\\
1586	-20.2473531112355\\
1587	-14.277226814439\\
1588	-14.7240397483297\\
1589	-20.4011418388866\\
1590	-16.7912697046563\\
1591	-18.2873414404833\\
1592	-19.2841088114221\\
1593	-18.619785837046\\
1594	-19.2379310284498\\
1595	-18.2450416997417\\
1596	-20.5551859193315\\
1597	-21.146275668126\\
1598	-18.5358690240976\\
1599	-16.9509240652974\\
1600	-19.2127996286595\\
1601	-11.7476662339764\\
1602	-14.6677622640734\\
1603	-8.56208133366317\\
1604	-11.1759985786657\\
1605	-10.1557884352571\\
1606	-14.9753662428618\\
1607	-5.2951103580283\\
1608	-9.06876669702069\\
1609	-13.3270654449343\\
1610	-15.2456830165199\\
1611	-15.0405577936428\\
1612	-19.5241440187845\\
1613	-19.8693014625271\\
1614	-14.6148563608708\\
1615	-16.7315359714214\\
1616	-12.3876703843716\\
1617	-14.636585615481\\
1618	-11.5239375476121\\
1619	-11.9235852490594\\
1620	-7.18132725717624\\
1621	-7.76940109791417\\
1622	-9.55421269253702\\
1623	-14.1024247561691\\
1624	-15.1728910075544\\
1625	-13.8704786862654\\
1626	-12.8950237048766\\
1627	-13.0753204668947\\
1628	-13.7687396199451\\
1629	-12.9484827576282\\
1630	-12.5390867749438\\
1631	-11.2284020970242\\
1632	-10.8086781454936\\
1633	-9.63268201837263\\
1634	-9.4079379719692\\
1635	-12.723441198749\\
1636	-13.5839009493286\\
1637	-12.7228563939343\\
1638	-11.4039092960522\\
1639	-10.3076357231005\\
1640	-10.1138953293785\\
1641	-16.3814857818372\\
1642	-29.0416171666063\\
1643	-17.9344422457759\\
1644	-18.5891748719553\\
1645	-15.6848841297401\\
1646	-18.8135698680521\\
1647	-20.3757598391958\\
1648	-14.2793337471458\\
1649	-14.4185235505427\\
1650	-13.3832765693005\\
1651	-16.1489792246064\\
1652	-12.6508961156749\\
1653	-13.3234294869821\\
1654	-14.1476582691973\\
1655	-15.5992926198385\\
1656	-14.5151578903317\\
1657	-14.1368772053306\\
1658	-14.0929065002601\\
1659	-18.5982216254681\\
1660	-18.6813248435894\\
1661	-24.1960422873456\\
1662	-34.1030522727215\\
1663	-26.5286825536452\\
1664	-18.6934214336534\\
1665	-15.5919391371724\\
1666	-20.4884938965342\\
1667	-24.8672264972183\\
1668	-18.0933424386892\\
1669	-21.4828746905564\\
1670	-17.0611793762198\\
1671	-14.6042297054014\\
1672	-12.4602064530111\\
1673	-16.0029726615974\\
1674	-18.7547250704695\\
1675	-15.4009706793308\\
1676	-21.132241418491\\
1677	-20.0881203402337\\
1678	-19.8062780403179\\
1679	-15.9654459920766\\
1680	-17.9908522671699\\
1681	-22.3417483726796\\
1682	-18.3511464438466\\
1683	-19.4243693603676\\
1684	-17.5108289182883\\
1685	-15.1137309441935\\
1686	-14.827154681049\\
1687	-13.0631656481182\\
1688	-13.6982343639859\\
1689	-19.4155764380873\\
1690	-22.264707256483\\
1691	-22.6993375750912\\
1692	-20.0670677207643\\
1693	-16.8481257966771\\
1694	-14.428308643118\\
1695	-14.0136672704098\\
1696	-15.6051485056878\\
1697	-18.5941553963619\\
1698	-21.7344113900699\\
1699	-25.7991435294487\\
1700	-20.9214764253957\\
1701	-15.1933671339002\\
1702	-20.105859288808\\
1703	-31.5126256901988\\
1704	-21.8144706563924\\
1705	-21.9800337733318\\
1706	-24.5037986498619\\
1707	-22.430966546367\\
1708	-17.2726082249519\\
1709	-20.2015008527851\\
1710	-18.1970657629581\\
1711	-18.9463894064891\\
1712	-23.1469633435697\\
1713	-18.5637214876811\\
1714	-20.2813218022447\\
1715	-20.4796314701305\\
1716	-21.1939203392876\\
1717	-18.2214220043095\\
1718	-19.579939322063\\
1719	-17.4872175500673\\
1720	-16.7343109104772\\
1721	-19.1639043071023\\
1722	-17.9185180187462\\
1723	-13.0854662999252\\
1724	-13.889984393403\\
1725	-11.4294863204291\\
1726	-7.20095511698332\\
1727	-17.0855224757477\\
1728	-28.6173015363865\\
1729	-23.7171052900601\\
1730	-17.4030640139488\\
1731	-17.6347296686508\\
1732	-17.5819380658886\\
1733	-15.0308914278557\\
1734	-13.0957978741988\\
1735	-12.9283956737752\\
1736	-14.64841157245\\
1738	-15.2715471357944\\
1739	-13.8047302396988\\
1740	-13.5332185871632\\
1741	-10.9207846585705\\
1742	-10.7360752776424\\
1743	-16.7382042077595\\
1744	-14.4865736523393\\
1745	-16.2172329995033\\
1746	-16.7375880575169\\
1747	-19.7302895408648\\
1748	-20.8760365823819\\
1749	-24.9810639668035\\
1750	-21.8566836154353\\
1751	-20.5189740652206\\
1752	-22.3580573423189\\
1753	-16.0401513666645\\
1754	-19.8785519171361\\
1755	-16.1975629599531\\
1756	-17.6140996878041\\
1757	-19.6719327593378\\
1758	-23.1518119624855\\
1759	-19.8092257539818\\
1760	-21.9372143752357\\
1761	-29.7386115242837\\
1762	-23.389171175354\\
1764	-18.2352873499219\\
1765	-19.3966523801885\\
1766	-18.8286311411991\\
1767	-16.112440871635\\
1768	-14.9666913483709\\
1769	-15.5516213284695\\
1770	-14.6986335315919\\
1771	-22.007502671692\\
1772	-32.355175166864\\
1773	-27.7878991261402\\
1774	-25.1078890279816\\
1775	-24.6263416902184\\
1776	-17.6762293763122\\
1777	-16.0281571383903\\
1778	-14.0382916951469\\
1779	-12.4300693290738\\
1780	-11.8941941059236\\
1781	-17.0727273749073\\
1782	-21.826031703509\\
1783	-22.440725055822\\
1784	-21.3288555025404\\
1785	-16.2639060250865\\
1786	-22.4864761252056\\
1787	-27.7594661747405\\
1788	-29.1599178888916\\
1789	-24.21029849823\\
1790	-37.1730146997984\\
1791	-32.6750452855531\\
1792	-16.2734777145249\\
1793	-15.5781793809363\\
1794	-14.5267943717404\\
1795	-18.737793378657\\
1796	-24.902145575843\\
1797	-31.6895658060441\\
1798	-21.6560133612752\\
1799	-21.1722401607499\\
1800	-14.5175720784684\\
1801	-14.1035606059006\\
1802	-15.0165008292045\\
1803	-14.4623719446611\\
1804	-12.0779099413166\\
1805	-12.5232740369374\\
};
\addlegendentry{OSA predition}

\addplot [color=mycolor3, dotted, line width=2.0pt]
  table[row sep=crcr]{%
1006	-24.414\\
1007	-28.076\\
1008	-23.193\\
1009	-13.4280000000001\\
1010	-17.2179207345996\\
1012	-19.4944403743407\\
1013	-18.6819305957893\\
1014	-13.087561983567\\
1015	-17.0009924965887\\
1016	-14.7381673482973\\
1017	-15.1814481159433\\
1018	-31.5486432766359\\
1019	-31.0264268340795\\
1020	-28.9984899026015\\
1021	-23.1874067824081\\
1022	-19.7631271859211\\
1023	-27.1594085768534\\
1024	-21.6438546389111\\
1025	-18.8459351390479\\
1026	-23.7058744265898\\
1027	-22.6137007115183\\
1028	-20.5385859615517\\
1029	-26.3491294614437\\
1030	-27.509753117163\\
1031	-21.3512946949729\\
1032	-27.414759438225\\
1033	-27.1182089524655\\
1034	-23.0692330171769\\
1035	-20.7546251339577\\
1036	-18.714198637218\\
1037	-20.1615603435478\\
1038	-22.8524829447522\\
1039	-21.8377763742046\\
1040	-22.5358527663473\\
1041	-22.1255116217885\\
1042	-23.5732182145255\\
1043	-30.4100911535406\\
1044	-28.242414346757\\
1045	-21.1870821086382\\
1046	-20.940148141565\\
1047	-16.426176654619\\
1048	-16.7003428903461\\
1049	-18.1885684429963\\
1050	-16.3450272543498\\
1051	-16.5597746803753\\
1052	-19.3801049105359\\
1053	-21.4747438861152\\
1054	-20.5878549163692\\
1055	-27.3296180143518\\
1056	-23.6018823463919\\
1057	-17.6624970870425\\
1058	-17.0096013453326\\
1059	-17.9108019250668\\
1060	-19.0017122680886\\
1061	-15.13943982605\\
1062	-15.3713604320678\\
1063	-16.1051323055858\\
1064	-15.4915127804863\\
1065	-14.6788979660012\\
1066	-15.3334055528653\\
1067	-16.3297583562191\\
1068	-22.0706306220316\\
1069	-22.4571815733193\\
1070	-23.5057014555673\\
1071	-23.0603965316805\\
1072	-26.4572693873506\\
1073	-23.2873121425559\\
1074	-21.4161297228518\\
1075	-20.6844264966498\\
1076	-20.5224448380718\\
1077	-20.0494632370269\\
1078	-28.5317117849177\\
1079	-42.7807953531105\\
1080	-42.7117407793226\\
1081	-41.17025478956\\
1082	-27.4865864263063\\
1083	-36.8951764691426\\
1084	-45.9704929438308\\
1085	-44.8252767861593\\
1086	-35.3899010478544\\
1087	-44.1869187423779\\
1088	-60.454210240958\\
1089	-43.331859934289\\
1090	-37.4943323945422\\
1091	-26.0675395088008\\
1092	-23.0288880466812\\
1093	-20.6193925169605\\
1094	-23.1579945416586\\
1095	-19.9941981446696\\
1096	-15.0633050270699\\
1097	-15.2607305720962\\
1098	-16.0067549255944\\
1099	-20.9404398433855\\
1100	-20.0553359981573\\
1101	-20.1787830321948\\
1102	-23.8605500490155\\
1103	-25.6688037204108\\
1104	-31.6292141271701\\
1105	-28.4291428530566\\
1106	-31.2001052703113\\
1107	-26.8691271003825\\
1108	-23.5029214609449\\
1109	-26.0988324392224\\
1110	-21.7898635425986\\
1111	-20.3426903722177\\
1112	-22.7956435148565\\
1113	-22.3093058389147\\
1114	-18.9263316451595\\
1115	-19.0527936174376\\
1116	-16.6570741264386\\
1117	-17.370577570995\\
1118	-17.1927310889453\\
1119	-15.3776621731631\\
1120	-16.2503601344233\\
1121	-18.1488739802055\\
1122	-24.7236814585542\\
1123	-25.9494602507261\\
1124	-26.7714512888017\\
1125	-19.5493862423159\\
1126	-23.9664443932511\\
1127	-34.4122758971121\\
1128	-28.7162930437069\\
1129	-29.5600804138505\\
1130	-31.6483068872424\\
1131	-20.3533045428278\\
1132	-18.9188401937015\\
1133	-20.2982066580009\\
1134	-22.7975074285941\\
1135	-34.0202956473631\\
1136	-40.3724192670368\\
1137	-38.7564327449702\\
1138	-31.1224209852826\\
1139	-31.0528690350986\\
1140	-28.641537435888\\
1141	-28.5482830967958\\
1142	-28.326826928497\\
1143	-21.9159461958855\\
1144	-20.0466074472349\\
1146	-18.0192494737073\\
1147	-19.021353276534\\
1148	-23.3251656404761\\
1149	-33.5250495451294\\
1151	-21.4279484183885\\
1152	-20.4527884996187\\
1153	-19.9335626146924\\
1154	-16.2075266177603\\
1155	-15.69646584412\\
1156	-14.5933075778755\\
1157	-17.4133112916193\\
1158	-16.0721757166298\\
1159	-16.8831584285867\\
1160	-16.6181053373484\\
1161	-16.6058632320801\\
1162	-14.9506377353805\\
1163	-14.5194083674883\\
1164	-14.2529092424372\\
1165	-12.7649964225798\\
1166	-21.2598676924345\\
1167	-33.0982391751111\\
1168	-33.3954064036307\\
1169	-24.8193455049332\\
1170	-20.5789434158844\\
1171	-18.8718259373809\\
1172	-17.5631101524943\\
1173	-18.8767177489813\\
1174	-19.1534584065814\\
1175	-23.3365432826283\\
1176	-28.518475531162\\
1177	-45.5722594988711\\
1178	-54.8141133200998\\
1179	-44.3253823579651\\
1180	-42.5260072762346\\
1181	-28.9537961052822\\
1182	-33.5990059115081\\
1183	-33.3170007976787\\
1184	-35.0092891584043\\
1185	-29.0202611907348\\
1186	-26.2316986959147\\
1187	-26.6820834218215\\
1188	-24.5493231444784\\
1189	-26.9497935447419\\
1190	-22.1501648980982\\
1191	-31.4839405685823\\
1192	-38.5805807035063\\
1193	-27.7801664969863\\
1194	-20.2831824949792\\
1195	-20.9990812202009\\
1196	-31.7741769347972\\
1197	-37.9218436372973\\
1198	-43.4336485697215\\
1199	-45.6523738106355\\
1200	-45.3235459652428\\
1201	-35.6682351461154\\
1202	-33.7431553387839\\
1203	-37.4182450042617\\
1204	-42.7509249885306\\
1205	-28.5419589509072\\
1206	-19.8179748757896\\
1207	-25.962240993023\\
1208	-23.9015148048329\\
1209	-17.8662563824284\\
1210	-19.834054338841\\
1211	-22.8789380759799\\
1212	-19.6522609991039\\
1213	-17.6346526859788\\
1214	-19.0789056491942\\
1215	-18.6369233547111\\
1216	-20.8398299023497\\
1217	-27.7842130620538\\
1218	-25.7569466799184\\
1219	-20.6834265362165\\
1220	-20.455625815704\\
1221	-26.6113097741641\\
1222	-45.4457502761738\\
1223	-32.1969695781131\\
1224	-22.7860838994056\\
1225	-19.8982992367266\\
1226	-22.7928854704805\\
1227	-20.8706212136867\\
1228	-18.2701794282364\\
1229	-18.1379833433612\\
1230	-19.912948864818\\
1231	-22.6481350170504\\
1232	-25.1503590256382\\
1233	-19.1358441712644\\
1234	-25.775493686707\\
1235	-30.960293554502\\
1236	-27.4077385803284\\
1237	-24.8270689260371\\
1238	-24.7852275972991\\
1239	-32.1307291047058\\
1240	-22.7884303921423\\
1241	-16.534708929363\\
1242	-17.1209766507984\\
1243	-18.5607498443896\\
1244	-22.4557418405063\\
1245	-20.8496863352216\\
1246	-18.0145044544856\\
1247	-22.1965103877326\\
1248	-22.2352272927956\\
1249	-18.7926284833554\\
1250	-19.9678684441992\\
1251	-19.6511051741484\\
1252	-18.0545767980202\\
1253	-19.3574021706631\\
1254	-15.616217017747\\
1255	-16.3818552237765\\
1256	-14.6351072813854\\
1257	-15.0119052759026\\
1258	-18.4140641485521\\
1259	-21.5488128811301\\
1260	-26.7214071709216\\
1261	-36.1296224248676\\
1262	-25.85060548761\\
1263	-19.3794929100386\\
1264	-18.2348303442323\\
1265	-17.564565122917\\
1266	-19.5515965048874\\
1267	-15.6355914854935\\
1268	-16.4298394648292\\
1269	-17.5109224909163\\
1270	-27.4455385718993\\
1271	-25.7705366029313\\
1272	-30.5134124869198\\
1273	-23.4697573579469\\
1274	-22.2768065244836\\
1275	-16.8405732665133\\
1276	-17.1627118351776\\
1277	-15.2300337025549\\
1278	-14.7113190676457\\
1279	-14.1312834752516\\
1280	-17.2854318535628\\
1281	-20.5422167472134\\
1282	-20.1782465747451\\
1283	-21.0591812198754\\
1284	-31.4402434856363\\
1285	-29.6099773649501\\
1286	-23.1568177866463\\
1287	-25.584391974865\\
1288	-26.5066691465681\\
1289	-33.014453316955\\
1290	-26.6451379508824\\
1291	-21.0451952157875\\
1292	-16.7363372628106\\
1293	-17.6643122927676\\
1294	-15.3701183284588\\
1295	-14.4242922683857\\
1296	-14.5324270930275\\
1297	-14.3749203777352\\
1298	-14.8599657690852\\
1299	-19.2415280240944\\
1300	-18.7289635622774\\
1301	-19.0361998103517\\
1302	-19.8349449027514\\
1303	-16.1922788096015\\
1304	-16.0549212272913\\
1305	-15.0208479819453\\
1306	-21.6966438763668\\
1307	-21.796850611609\\
1308	-22.7131582904055\\
1309	-20.7557131968224\\
1310	-18.4207250005988\\
1311	-21.1140864619847\\
1312	-26.3102243425253\\
1313	-27.2295617192565\\
1314	-20.337091469612\\
1315	-28.5598570205275\\
1316	-26.6188525472162\\
1317	-23.7810774434849\\
1318	-26.9340554895452\\
1319	-26.6952315294834\\
1320	-22.7200863113819\\
1321	-20.664512178048\\
1322	-29.6836598992402\\
1323	-41.0382828394891\\
1324	-32.2538420787255\\
1325	-21.3824040309278\\
1326	-21.9076367816517\\
1327	-22.3511450356648\\
1328	-25.3332894476339\\
1329	-28.0720720238132\\
1330	-22.4572876106954\\
1331	-23.3029112773188\\
1332	-28.0772471817822\\
1333	-30.1077623330971\\
1334	-22.5627605460688\\
1335	-19.87302339692\\
1336	-22.9867272258962\\
1337	-36.2990632094486\\
1338	-31.4152519272088\\
1339	-29.8697381765021\\
1340	-22.2026531934318\\
1341	-20.6975910901547\\
1342	-20.5313056377563\\
1343	-16.7959234569912\\
1344	-15.4950625648794\\
1345	-15.256678437117\\
1346	-14.1486698513152\\
1347	-14.3742026267851\\
1348	-20.0038283503429\\
1349	-24.3267117011299\\
1350	-19.3375330743568\\
1351	-19.7806856177117\\
1352	-21.4082956855545\\
1353	-16.8670628582081\\
1354	-17.117476513303\\
1355	-16.4788035060953\\
1356	-15.1661855831471\\
1357	-15.932512941708\\
1358	-21.1245943591036\\
1359	-25.7767930271007\\
1360	-18.9026580652417\\
1361	-17.3727205431278\\
1362	-16.411261847461\\
1363	-16.477279519105\\
1364	-14.8004980850387\\
1365	-19.1444848995436\\
1366	-34.9207594012664\\
1367	-26.8002237677047\\
1368	-31.6234572445953\\
1369	-37.3630031149253\\
1370	-37.4986254062835\\
1371	-29.9613675661237\\
1372	-24.1960582967006\\
1373	-24.2952017796115\\
1374	-24.7173467079613\\
1375	-26.7898371082238\\
1376	-29.0839615606383\\
1377	-28.57050690347\\
1378	-36.057902359873\\
1379	-42.3990108981357\\
1380	-47.4940434240173\\
1381	-34.7799260840404\\
1382	-35.3268896076365\\
1383	-40.6648279731351\\
1384	-33.6880878165841\\
1385	-36.0271451454121\\
1386	-32.7667760998336\\
1387	-20.5103572188989\\
1388	-17.2082837769972\\
1389	-17.2678384430064\\
1390	-20.9349439358612\\
1391	-18.617446284959\\
1392	-15.8595447242405\\
1393	-17.8191885096742\\
1394	-15.7329056963849\\
1395	-14.8742354181538\\
1396	-15.0610994243787\\
1397	-15.5548340695234\\
1398	-14.5542829568894\\
1399	-17.5997071232086\\
1400	-23.3793337314171\\
1401	-18.6519125376117\\
1402	-27.8806183640006\\
1403	-43.0611872870447\\
1404	-36.1997132119232\\
1405	-41.7173154310033\\
1406	-33.5498022113279\\
1407	-33.5094545121226\\
1409	-19.5501057593317\\
1410	-21.3077644166983\\
1411	-18.9926189664266\\
1412	-17.2229316324342\\
1413	-18.4220181739736\\
1414	-14.5220180411347\\
1415	-15.1575134867576\\
1416	-13.7791418207246\\
1417	-17.6287875224352\\
1418	-23.6914844006578\\
1419	-26.6349663305823\\
1420	-26.120108037376\\
1421	-24.9322859182653\\
1422	-26.9787714070333\\
1423	-24.3011000156803\\
1424	-21.069876748775\\
1425	-21.7953336235128\\
1426	-19.3979542281652\\
1427	-24.3029367747499\\
1428	-19.0928435878543\\
1429	-18.2023617956181\\
1430	-20.6923892189791\\
1431	-27.3773976064115\\
1432	-22.181454889364\\
1433	-19.3321692379836\\
1434	-18.3114583668753\\
1435	-19.4325706507238\\
1436	-16.5569075066485\\
1437	-16.0536395836186\\
1438	-18.0538747833741\\
1439	-20.2071268307523\\
1440	-21.3219957161662\\
1441	-18.8977448944756\\
1442	-21.5074891014779\\
1443	-21.2007305005816\\
1444	-16.4224627375411\\
1445	-16.790383179871\\
1446	-23.6419365525489\\
1447	-32.9583220445127\\
1448	-30.0199224266821\\
1449	-26.1948412864042\\
1450	-25.2152028949713\\
1451	-23.0136014676782\\
1453	-19.6481103181623\\
1454	-21.8588234984345\\
1455	-21.1358762799744\\
1456	-17.4156383079885\\
1457	-20.1566409346303\\
1458	-21.9125486236533\\
1459	-24.1543371231051\\
1460	-25.8695115216813\\
1461	-25.4260219200639\\
1462	-32.1126391264281\\
1463	-41.602242748826\\
1464	-46.7132863283655\\
1465	-28.304696907386\\
1466	-21.94501095177\\
1467	-18.0919253397035\\
1468	-17.8350055062276\\
1469	-16.6828110677429\\
1470	-15.9800664267875\\
1471	-22.3534160742909\\
1472	-21.2752546464237\\
1473	-19.498600520971\\
1474	-22.1205646046976\\
1475	-23.8433871942414\\
1476	-27.4656773732847\\
1477	-28.7053352993432\\
1478	-39.5223742324201\\
1479	-37.5592979977621\\
1480	-28.6185629617132\\
1481	-23.580199124247\\
1482	-23.2493102992159\\
1483	-23.9518971745888\\
1484	-26.7124182790544\\
1485	-22.2922019290752\\
1486	-21.0747367626207\\
1487	-21.0267545145953\\
1488	-16.329192364464\\
1489	-20.2933110026022\\
1490	-27.4731741036073\\
1491	-21.5570227211888\\
1492	-21.4525522462363\\
1493	-35.0278266512948\\
1494	-28.5926749310506\\
1495	-25.738734157617\\
1496	-33.5199004801202\\
1497	-33.1058272688833\\
1498	-22.4104239970691\\
1499	-18.56499513938\\
1500	-18.6761353866762\\
1501	-25.1739534040771\\
1502	-38.8478643520957\\
1503	-39.1954563904835\\
1504	-37.9398335030455\\
1505	-31.9657746319233\\
1506	-23.4713566527705\\
1507	-21.6791441570638\\
1508	-18.8594805802816\\
1509	-17.7534577892641\\
1510	-16.8788663316011\\
1511	-18.3627911088374\\
1512	-19.5471310334829\\
1513	-16.1568526119295\\
1514	-17.7613079637688\\
1515	-22.9896816115836\\
1516	-24.143964847121\\
1517	-30.2980604910711\\
1518	-37.4512285146109\\
1519	-44.005195616026\\
1520	-32.7292321221591\\
1521	-28.9584488573132\\
1522	-31.3700328766818\\
1523	-28.1968459206628\\
1524	-22.6512172705015\\
1525	-23.6695113622657\\
1526	-35.980526427451\\
1527	-39.5191303570741\\
1528	-25.1476378380223\\
1529	-18.1230324094081\\
1530	-17.327688748494\\
1531	-17.5772983692714\\
1532	-16.2857399000602\\
1533	-14.354717232666\\
1534	-16.2346622335954\\
1535	-20.4331158327248\\
1536	-17.9734925550465\\
1537	-16.3309518304175\\
1538	-15.963372041602\\
1539	-15.2706922155066\\
1540	-14.9037383766654\\
1541	-14.8294998356703\\
1542	-17.725585349106\\
1543	-26.4224521727103\\
1544	-27.5462968810245\\
1545	-26.0500117818183\\
1546	-28.4087831845454\\
1547	-35.6619516224703\\
1548	-36.0958856378611\\
1549	-40.5884969601022\\
1550	-39.5241870673153\\
1551	-29.5359175746473\\
1552	-24.7447492367296\\
1553	-22.7712816250398\\
1554	-34.2484324377344\\
1555	-37.9256279982599\\
1556	-26.1810784885938\\
1557	-21.0501958896643\\
1558	-16.2929043043657\\
1559	-17.7489245493821\\
1560	-15.2988668541434\\
1561	-14.6765824798554\\
1562	-13.2523009546107\\
1563	-16.2279388027566\\
1564	-17.1252076972858\\
1565	-16.3496490961186\\
1566	-14.5464946509574\\
1567	-14.6728574507956\\
1568	-13.0949483468287\\
1569	-13.0531400339487\\
1570	-13.682700498149\\
1571	-13.0770854582724\\
1572	-12.8451087673445\\
1573	-12.8182770032413\\
1574	-27.016334798971\\
1575	-37.7538772603821\\
1576	-35.471998358727\\
1577	-26.9740000651511\\
1578	-33.5668605021467\\
1579	-49.0210294812041\\
1580	-44.7100308178353\\
1581	-42.9343545883719\\
1582	-35.4561622435726\\
1583	-33.4181908426763\\
1584	-36.1758014812344\\
1585	-25.9027311743966\\
1586	-24.3616317787273\\
1587	-18.5256788263023\\
1588	-18.0510082130058\\
1589	-24.3964316019137\\
1590	-19.9732770330425\\
1591	-20.3773347004892\\
1592	-22.129570705973\\
1593	-21.1264119841335\\
1594	-21.4130242287615\\
1595	-21.9601493261339\\
1596	-23.913683034647\\
1597	-25.4174493460491\\
1598	-22.3091320597114\\
1599	-19.6403357319921\\
1600	-23.0303896634987\\
1601	-15.3479388734661\\
1602	-17.7913106807221\\
1603	-14.8118039728649\\
1604	-15.2051280660619\\
1605	-12.6666093899182\\
1606	-18.0591163326426\\
1607	-12.0160216198658\\
1608	-12.6945619089277\\
1609	-16.3910039320162\\
1610	-18.8644587636859\\
1611	-17.0076831956601\\
1612	-21.8362730681717\\
1613	-22.3923703557118\\
1614	-17.6172996540113\\
1615	-20.5714284051735\\
1616	-15.158251932804\\
1617	-16.1415346590277\\
1618	-14.7074568434059\\
1619	-15.7583763025227\\
1620	-12.0748302587722\\
1621	-11.6928258282362\\
1622	-13.0525171368843\\
1623	-18.6476786722023\\
1624	-18.7636863615762\\
1625	-16.0895437692238\\
1626	-16.1236011715632\\
1627	-14.849074154703\\
1628	-16.0356117200315\\
1629	-14.5262571694202\\
1630	-14.4573946957391\\
1631	-13.8570325802691\\
1632	-13.214828237001\\
1633	-13.0045225289728\\
1634	-12.4943752451093\\
1635	-14.87626516261\\
1636	-14.8981702753345\\
1637	-14.321389758911\\
1638	-14.0198548165433\\
1639	-13.2036600401632\\
1640	-13.1758968235406\\
1641	-19.0514663927572\\
1642	-29.5738278295403\\
1643	-20.271540862758\\
1644	-19.8369416410212\\
1645	-16.9814592289663\\
1646	-21.1801577496387\\
1647	-21.9292541631614\\
1648	-16.840618253952\\
1649	-18.1252084377797\\
1650	-16.0454404144259\\
1651	-18.7719693465133\\
1652	-14.9876610831118\\
1653	-15.2101087612782\\
1654	-15.1871625558144\\
1655	-17.3547253244728\\
1656	-15.7324977985827\\
1657	-15.954964047241\\
1658	-15.8712765890305\\
1659	-19.9363196070892\\
1660	-19.6232971660672\\
1661	-25.1585854266057\\
1662	-35.6325719212655\\
1663	-29.0466595787705\\
1664	-21.4788189369881\\
1665	-19.3281302952541\\
1666	-24.0612853570778\\
1667	-29.8826552713419\\
1668	-22.7930881350553\\
1669	-25.4444089992182\\
1670	-19.8017843801367\\
1671	-16.6804804698802\\
1672	-16.3229294790867\\
1673	-19.9560866626066\\
1674	-20.9636096365766\\
1675	-19.3708839238302\\
1676	-24.78796141671\\
1677	-23.1118521280855\\
1678	-22.0192387870406\\
1679	-18.3752537750497\\
1680	-20.3346668519778\\
1681	-24.1122679108428\\
1682	-20.6108149357074\\
1683	-21.2731129378385\\
1684	-20.3238134853575\\
1685	-17.945958974374\\
1686	-18.1981292144849\\
1687	-15.9152419429447\\
1688	-16.4088942505839\\
1689	-21.9785139291182\\
1690	-24.3935109639831\\
1691	-25.1310835182594\\
1692	-23.170092823533\\
1693	-19.44584568955\\
1694	-17.3522188846362\\
1695	-17.7911612502307\\
1696	-18.8168152921601\\
1697	-21.9283783900528\\
1698	-24.7981443613844\\
1699	-28.792988369808\\
1700	-23.8850250621244\\
1701	-18.4513256571192\\
1702	-24.0058423830887\\
1703	-34.3225294305905\\
1704	-24.8288886497166\\
1705	-24.8119761641294\\
1706	-27.1637741515042\\
1707	-24.3105323299314\\
1708	-20.4346483624333\\
1709	-22.7581420889755\\
1710	-20.9808823561414\\
1711	-22.3486083616756\\
1712	-26.1816089501708\\
1713	-21.6090192887457\\
1714	-23.4461696365104\\
1715	-22.8879964780094\\
1716	-23.23553239219\\
1717	-21.020792210081\\
1718	-23.062105429645\\
1719	-19.3421843833271\\
1720	-19.6865778535657\\
1721	-21.4488465652009\\
1722	-20.4523933157882\\
1723	-15.4850480011726\\
1724	-16.3100803482125\\
1725	-16.2278085514895\\
1726	-13.1745750608907\\
1727	-21.1476567140157\\
1728	-32.1580645546162\\
1729	-29.024640466512\\
1730	-22.3840215245823\\
1731	-23.1729592957122\\
1732	-22.8348686586965\\
1733	-20.5326398504205\\
1734	-17.5374250770583\\
1735	-17.7524554656256\\
1736	-17.5894197269433\\
1737	-17.5944466191356\\
1738	-18.5718988286815\\
1739	-16.8021235000524\\
1740	-16.6109559725105\\
1741	-14.8572428380585\\
1742	-15.211350861551\\
1743	-21.4671359846072\\
1744	-17.9678759214714\\
1745	-19.8302838597137\\
1746	-18.8128370612226\\
1747	-21.8626106612958\\
1748	-21.8105846295628\\
1749	-26.3769044086223\\
1750	-22.6838396112412\\
1751	-23.3919766546924\\
1752	-23.5940370681433\\
1753	-17.1925045438411\\
1754	-22.1593220081625\\
1755	-18.7123326261781\\
1756	-20.307120299263\\
1757	-21.3877953685128\\
1758	-24.4969735412196\\
1759	-20.3940600569454\\
1760	-23.323556281669\\
1761	-29.823493467358\\
1762	-25.2301814086652\\
1763	-22.2751223144828\\
1764	-20.1999090705763\\
1765	-22.5703154260682\\
1766	-21.2485022827098\\
1767	-18.9182757160747\\
1768	-17.8520003282276\\
1769	-18.8533552720555\\
1770	-18.2070003858969\\
1771	-26.1241627204706\\
1772	-34.7441216221318\\
1773	-30.8949274234878\\
1774	-28.2350954614492\\
1775	-27.6992572823799\\
1776	-19.8577430810165\\
1777	-19.033368830251\\
1778	-17.3943266033909\\
1779	-16.4867866099628\\
1780	-16.2616869445023\\
1781	-20.8611176870177\\
1782	-24.9601309804536\\
1783	-26.0101649676669\\
1784	-23.7441904502743\\
1785	-19.3350683463127\\
1787	-31.3937582897356\\
1788	-32.1726151434145\\
1789	-27.218336190921\\
1790	-40.3014021553374\\
1791	-35.0043539274027\\
1792	-20.7770997859045\\
1793	-19.1481832717898\\
1794	-17.8809953343427\\
1796	-28.1744243792418\\
1797	-35.5188935673086\\
1798	-26.0317319244673\\
1799	-23.5555097648798\\
1800	-17.4737393478611\\
1801	-17.984645557653\\
1802	-17.6052206791769\\
1803	-18.4573133491583\\
1804	-15.6222887479719\\
1805	-16.6766187784908\\
};
\addlegendentry{MPO prediction}

\end{axis}

\begin{axis}[%
width=6.159cm,
height=1.831cm,
at={(0cm,7.627cm)},
scale only axis,
xmin=1000,
xmax=2000,
xlabel style={font=\color{white!15!black}},
xlabel={Sample index},
ymin=-50.8795298561816,
ymax=1.221,
ylabel style={font=\color{white!15!black}},
ylabel={$y(t)$},
axis background/.style={fill=white},
title style={font=\bfseries},
title={C3: RMSE(OSA) = 2.9683, RMSE(MPO) = 6.8292},
legend style={legend cell align=left, align=left, draw=white!15!black}
]
\addplot [color=mycolor1, line width=2.0pt]
  table[row sep=crcr]{%
1006	-18.3109999999999\\
1007	-23.193\\
1008	-18.3109999999999\\
1009	-9.76600000000008\\
1010	-15.8689999999999\\
1011	-12.2070000000001\\
1012	-14.6479999999999\\
1013	-10.9860000000001\\
1014	-3.66200000000003\\
1015	-2.44100000000003\\
1016	-4.88300000000004\\
1017	-6.10400000000004\\
1018	-14.6479999999999\\
1019	-17.0899999999999\\
1020	-17.0899999999999\\
1022	-9.76600000000008\\
1023	-18.3109999999999\\
1025	-10.9860000000001\\
1026	-13.4280000000001\\
1028	-10.9860000000001\\
1029	-20.752\\
1030	-19.5309999999999\\
1031	-12.2070000000001\\
1032	-20.752\\
1033	-20.752\\
1034	-15.8689999999999\\
1035	-13.4280000000001\\
1036	-9.76600000000008\\
1037	-14.6479999999999\\
1041	-14.6479999999999\\
1042	-15.8689999999999\\
1043	-21.973\\
1044	-20.752\\
1045	-13.4280000000001\\
1046	-13.4280000000001\\
1047	-8.54500000000007\\
1048	-12.2070000000001\\
1049	-12.2070000000001\\
1050	-13.4280000000001\\
1051	-10.9860000000001\\
1052	-13.4280000000001\\
1053	-14.6479999999999\\
1054	-13.4280000000001\\
1055	-20.752\\
1056	-13.4280000000001\\
1057	-9.76600000000008\\
1058	-7.32400000000007\\
1059	-8.54500000000007\\
1060	-10.9860000000001\\
1061	-6.10400000000004\\
1062	-9.76600000000008\\
1063	-10.9860000000001\\
1064	-6.10400000000004\\
1065	-6.10400000000004\\
1066	-9.76600000000008\\
1067	-9.76600000000008\\
1068	-15.8689999999999\\
1069	-14.6479999999999\\
1070	-17.0899999999999\\
1071	-15.8689999999999\\
1072	-18.3109999999999\\
1073	-13.4280000000001\\
1074	-14.6479999999999\\
1075	-12.2070000000001\\
1076	-10.9860000000001\\
1077	-12.2070000000001\\
1078	-23.193\\
1079	-28.076\\
1080	-30.518\\
1081	-29.297\\
1082	-18.3109999999999\\
1083	-28.076\\
1084	-32.9590000000001\\
1085	-31.7380000000001\\
1086	-20.752\\
1087	-34.1800000000001\\
1088	-40.2829999999999\\
1089	-29.297\\
1091	-19.5309999999999\\
1093	-12.2070000000001\\
1094	-14.6479999999999\\
1095	-9.76600000000008\\
1096	-7.32400000000007\\
1097	-7.32400000000007\\
1098	-10.9860000000001\\
1099	-13.4280000000001\\
1100	-12.2070000000001\\
1101	-12.2070000000001\\
1102	-15.8689999999999\\
1103	-18.3109999999999\\
1104	-19.5309999999999\\
1105	-15.8689999999999\\
1106	-21.973\\
1107	-15.8689999999999\\
1108	-15.8689999999999\\
1109	-17.0899999999999\\
1110	-12.2070000000001\\
1111	-12.2070000000001\\
1112	-15.8689999999999\\
1113	-14.6479999999999\\
1114	-8.54500000000007\\
1115	-12.2070000000001\\
1116	-8.54500000000007\\
1117	-9.76600000000008\\
1118	-9.76600000000008\\
1119	-7.32400000000007\\
1121	-12.2070000000001\\
1122	-17.0899999999999\\
1124	-17.0899999999999\\
1125	-10.9860000000001\\
1126	-12.2070000000001\\
1127	-23.193\\
1128	-15.8689999999999\\
1129	-20.752\\
1130	-21.973\\
1131	-12.2070000000001\\
1132	-9.76600000000008\\
1133	-12.2070000000001\\
1134	-13.4280000000001\\
1135	-21.973\\
1136	-25.635\\
1137	-26.855\\
1138	-19.5309999999999\\
1139	-20.752\\
1140	-18.3109999999999\\
1141	-17.0899999999999\\
1142	-18.3109999999999\\
1143	-13.4280000000001\\
1144	-12.2070000000001\\
1145	-9.76600000000008\\
1146	-9.76600000000008\\
1147	-10.9860000000001\\
1148	-15.8689999999999\\
1149	-23.193\\
1150	-17.0899999999999\\
1151	-13.4280000000001\\
1152	-10.9860000000001\\
1153	-10.9860000000001\\
1154	-7.32400000000007\\
1155	-6.10400000000004\\
1156	-6.10400000000004\\
1157	-12.2070000000001\\
1158	-7.32400000000007\\
1159	-8.54500000000007\\
1161	-8.54500000000007\\
1162	-7.32400000000007\\
1163	-4.88300000000004\\
1164	-3.66200000000003\\
1165	-8.54500000000007\\
1166	-15.8689999999999\\
1168	-20.752\\
1170	-10.9860000000001\\
1171	-8.54500000000007\\
1172	-4.88300000000004\\
1173	-9.76600000000008\\
1174	-8.54500000000007\\
1175	-15.8689999999999\\
1176	-17.0899999999999\\
1177	-29.297\\
1178	-32.9590000000001\\
1179	-25.635\\
1180	-25.635\\
1181	-18.3109999999999\\
1182	-23.193\\
1183	-21.973\\
1184	-24.414\\
1185	-15.8689999999999\\
1186	-17.0899999999999\\
1187	-17.0899999999999\\
1188	-15.8689999999999\\
1189	-15.8689999999999\\
1190	-13.4280000000001\\
1191	-23.193\\
1192	-25.635\\
1194	-13.4280000000001\\
1195	-14.6479999999999\\
1196	-23.193\\
1197	-24.414\\
1198	-28.076\\
1199	-30.518\\
1200	-29.297\\
1201	-23.193\\
1202	-21.973\\
1203	-25.635\\
1204	-28.076\\
1205	-18.3109999999999\\
1206	-12.2070000000001\\
1207	-19.5309999999999\\
1209	-9.76600000000008\\
1210	-13.4280000000001\\
1211	-14.6479999999999\\
1212	-10.9860000000001\\
1214	-10.9860000000001\\
1215	-8.54500000000007\\
1216	-10.9860000000001\\
1217	-18.3109999999999\\
1218	-17.0899999999999\\
1219	-12.2070000000001\\
1220	-12.2070000000001\\
1221	-18.3109999999999\\
1222	-26.855\\
1223	-17.0899999999999\\
1224	-14.6479999999999\\
1225	-10.9860000000001\\
1226	-13.4280000000001\\
1228	-8.54500000000007\\
1229	-8.54500000000007\\
1231	-13.4280000000001\\
1232	-14.6479999999999\\
1233	-12.2070000000001\\
1234	-15.8689999999999\\
1235	-20.752\\
1236	-17.0899999999999\\
1237	-12.2070000000001\\
1238	-15.8689999999999\\
1239	-20.752\\
1240	-13.4280000000001\\
1241	-7.32400000000007\\
1242	-13.4280000000001\\
1243	-10.9860000000001\\
1244	-12.2070000000001\\
1245	-9.76600000000008\\
1246	-8.54500000000007\\
1247	-13.4280000000001\\
1248	-13.4280000000001\\
1249	-9.76600000000008\\
1250	-12.2070000000001\\
1252	-7.32400000000007\\
1253	-12.2070000000001\\
1254	-8.54500000000007\\
1255	-9.76600000000008\\
1257	-4.88300000000004\\
1258	-12.2070000000001\\
1259	-15.8689999999999\\
1260	-17.0899999999999\\
1261	-23.193\\
1262	-15.8689999999999\\
1263	-9.76600000000008\\
1264	-8.54500000000007\\
1265	-8.54500000000007\\
1266	-10.9860000000001\\
1267	-8.54500000000007\\
1268	-4.88300000000004\\
1269	-10.9860000000001\\
1270	-15.8689999999999\\
1271	-13.4280000000001\\
1272	-20.752\\
1273	-15.8689999999999\\
1274	-14.6479999999999\\
1275	-8.54500000000007\\
1276	-10.9860000000001\\
1277	-4.88300000000004\\
1278	-3.66200000000003\\
1279	-4.88300000000004\\
1280	-9.76600000000008\\
1283	-13.4280000000001\\
1284	-20.752\\
1285	-17.0899999999999\\
1286	-14.6479999999999\\
1288	-19.5309999999999\\
1289	-20.752\\
1291	-13.4280000000001\\
1292	-7.32400000000007\\
1294	-4.88300000000004\\
1295	-7.32400000000007\\
1296	-7.32400000000007\\
1297	-4.88300000000004\\
1299	-12.2070000000001\\
1300	-10.9860000000001\\
1301	-12.2070000000001\\
1302	-12.2070000000001\\
1304	-4.88300000000004\\
1305	-9.76600000000008\\
1306	-15.8689999999999\\
1307	-13.4280000000001\\
1308	-15.8689999999999\\
1309	-13.4280000000001\\
1310	-9.76600000000008\\
1311	-14.6479999999999\\
1312	-18.3109999999999\\
1313	-18.3109999999999\\
1314	-13.4280000000001\\
1315	-20.752\\
1316	-15.8689999999999\\
1319	-19.5309999999999\\
1320	-13.4280000000001\\
1321	-13.4280000000001\\
1322	-18.3109999999999\\
1323	-26.855\\
1324	-21.973\\
1325	-13.4280000000001\\
1327	-13.4280000000001\\
1328	-15.8689999999999\\
1329	-17.0899999999999\\
1330	-12.2070000000001\\
1331	-14.6479999999999\\
1332	-18.3109999999999\\
1333	-20.752\\
1334	-13.4280000000001\\
1335	-12.2070000000001\\
1336	-15.8689999999999\\
1337	-23.193\\
1338	-20.752\\
1339	-21.973\\
1340	-14.6479999999999\\
1341	-14.6479999999999\\
1343	-9.76600000000008\\
1344	-4.88300000000004\\
1345	-3.66200000000003\\
1346	-3.66200000000003\\
1347	-7.32400000000007\\
1348	-13.4280000000001\\
1349	-14.6479999999999\\
1350	-9.76600000000008\\
1351	-12.2070000000001\\
1352	-13.4280000000001\\
1353	-9.76600000000008\\
1354	-8.54500000000007\\
1355	-8.54500000000007\\
1356	-6.10400000000004\\
1357	-9.76600000000008\\
1358	-14.6479999999999\\
1359	-15.8689999999999\\
1360	-10.9860000000001\\
1361	-8.54500000000007\\
1363	-8.54500000000007\\
1364	-7.32400000000007\\
1365	-10.9860000000001\\
1366	-20.752\\
1367	-18.3109999999999\\
1368	-18.3109999999999\\
1369	-24.414\\
1370	-23.193\\
1371	-18.3109999999999\\
1372	-14.6479999999999\\
1374	-14.6479999999999\\
1376	-19.5309999999999\\
1377	-19.5309999999999\\
1378	-25.635\\
1379	-26.855\\
1380	-31.7380000000001\\
1381	-24.414\\
1383	-26.855\\
1384	-21.973\\
1385	-24.414\\
1386	-20.752\\
1387	-13.4280000000001\\
1388	-9.76600000000008\\
1389	-9.76600000000008\\
1390	-12.2070000000001\\
1392	-7.32400000000007\\
1393	-10.9860000000001\\
1395	-6.10400000000004\\
1396	-7.32400000000007\\
1397	-9.76600000000008\\
1398	-6.10400000000004\\
1399	-12.2070000000001\\
1400	-14.6479999999999\\
1401	-10.9860000000001\\
1402	-17.0899999999999\\
1403	-24.414\\
1404	-24.414\\
1405	-28.076\\
1406	-23.193\\
1407	-21.973\\
1408	-17.0899999999999\\
1409	-9.76600000000008\\
1410	-10.9860000000001\\
1411	-10.9860000000001\\
1412	-7.32400000000007\\
1413	-10.9860000000001\\
1414	-9.76600000000008\\
1415	-2.44100000000003\\
1417	-9.76600000000008\\
1420	-17.0899999999999\\
1421	-14.6479999999999\\
1422	-17.0899999999999\\
1423	-14.6479999999999\\
1426	-10.9860000000001\\
1427	-14.6479999999999\\
1428	-13.4280000000001\\
1429	-10.9860000000001\\
1430	-13.4280000000001\\
1431	-18.3109999999999\\
1432	-13.4280000000001\\
1433	-10.9860000000001\\
1435	-10.9860000000001\\
1436	-8.54500000000007\\
1437	-7.32400000000007\\
1438	-10.9860000000001\\
1439	-13.4280000000001\\
1440	-13.4280000000001\\
1441	-10.9860000000001\\
1442	-13.4280000000001\\
1443	-13.4280000000001\\
1444	-8.54500000000007\\
1445	-7.32400000000007\\
1446	-15.8689999999999\\
1447	-19.5309999999999\\
1448	-18.3109999999999\\
1449	-18.3109999999999\\
1452	-10.9860000000001\\
1453	-9.76600000000008\\
1454	-13.4280000000001\\
1455	-13.4280000000001\\
1456	-8.54500000000007\\
1457	-12.2070000000001\\
1458	-14.6479999999999\\
1459	-14.6479999999999\\
1460	-17.0899999999999\\
1461	-17.0899999999999\\
1462	-20.752\\
1463	-28.076\\
1464	-30.518\\
1465	-20.752\\
1466	-13.4280000000001\\
1467	-8.54500000000007\\
1468	-6.10400000000004\\
1469	-8.54500000000007\\
1470	-7.32400000000007\\
1471	-10.9860000000001\\
1472	-12.2070000000001\\
1473	-12.2070000000001\\
1474	-13.4280000000001\\
1475	-15.8689999999999\\
1476	-19.5309999999999\\
1477	-19.5309999999999\\
1478	-28.076\\
1480	-18.3109999999999\\
1481	-14.6479999999999\\
1483	-14.6479999999999\\
1484	-17.0899999999999\\
1486	-12.2070000000001\\
1487	-13.4280000000001\\
1488	-8.54500000000007\\
1489	-7.32400000000007\\
1490	-17.0899999999999\\
1492	-12.2070000000001\\
1493	-23.193\\
1494	-18.3109999999999\\
1495	-15.8689999999999\\
1496	-21.973\\
1497	-23.193\\
1498	-14.6479999999999\\
1499	-8.54500000000007\\
1500	-9.76600000000008\\
1501	-15.8689999999999\\
1502	-23.193\\
1503	-25.635\\
1504	-25.635\\
1505	-20.752\\
1506	-13.4280000000001\\
1507	-10.9860000000001\\
1508	-9.76600000000008\\
1509	-7.32400000000007\\
1510	-9.76600000000008\\
1511	-9.76600000000008\\
1512	-10.9860000000001\\
1513	-8.54500000000007\\
1514	-9.76600000000008\\
1515	-15.8689999999999\\
1516	-17.0899999999999\\
1517	-17.0899999999999\\
1518	-24.414\\
1519	-29.297\\
1520	-18.3109999999999\\
1521	-18.3109999999999\\
1522	-19.5309999999999\\
1523	-17.0899999999999\\
1524	-13.4280000000001\\
1525	-13.4280000000001\\
1526	-23.193\\
1527	-25.635\\
1529	-8.54500000000007\\
1531	-3.66200000000003\\
1532	-2.44100000000003\\
1533	-6.10400000000004\\
1534	-8.54500000000007\\
1535	-9.76600000000008\\
1536	-9.76600000000008\\
1537	-7.32400000000007\\
1538	-7.32400000000007\\
1539	-8.54500000000007\\
1540	-6.10400000000004\\
1541	-8.54500000000007\\
1543	-15.8689999999999\\
1544	-17.0899999999999\\
1545	-17.0899999999999\\
1546	-19.5309999999999\\
1547	-23.193\\
1548	-24.414\\
1549	-26.855\\
1550	-28.076\\
1551	-20.752\\
1552	-15.8689999999999\\
1553	-13.4280000000001\\
1554	-20.752\\
1555	-24.414\\
1556	-17.0899999999999\\
1557	-13.4280000000001\\
1558	-7.32400000000007\\
1559	-4.88300000000004\\
1560	-4.88300000000004\\
1561	-6.10400000000004\\
1562	-2.44100000000003\\
1563	-12.2070000000001\\
1564	-9.76600000000008\\
1566	-7.32400000000007\\
1567	-4.88300000000004\\
1568	-7.32400000000007\\
1569	-8.54500000000007\\
1570	-7.32400000000007\\
1571	-4.88300000000004\\
1572	-3.66200000000003\\
1573	-7.32400000000007\\
1574	-15.8689999999999\\
1575	-21.973\\
1576	-23.193\\
1577	-17.0899999999999\\
1578	-19.5309999999999\\
1579	-30.518\\
1580	-30.518\\
1581	-26.855\\
1583	-21.973\\
1584	-23.193\\
1585	-15.8689999999999\\
1587	-10.9860000000001\\
1588	-7.32400000000007\\
1589	-14.6479999999999\\
1590	-17.0899999999999\\
1591	-12.2070000000001\\
1593	-14.6479999999999\\
1594	-10.9860000000001\\
1595	-13.4280000000001\\
1596	-14.6479999999999\\
1597	-18.3109999999999\\
1599	-10.9860000000001\\
1600	-15.8689999999999\\
1601	-10.9860000000001\\
1602	-2.44100000000003\\
1603	-7.32400000000007\\
1604	-8.54500000000007\\
1605	-7.32400000000007\\
1606	1.221\\
1607	-2.44100000000003\\
1608	-8.54500000000007\\
1609	-8.54500000000007\\
1610	-12.2070000000001\\
1611	-10.9860000000001\\
1612	-14.6479999999999\\
1613	-17.0899999999999\\
1614	-8.54500000000007\\
1615	-12.2070000000001\\
1616	-14.6479999999999\\
1617	-4.88300000000004\\
1618	-6.10400000000004\\
1619	-3.66200000000003\\
1620	-9.76600000000008\\
1621	-6.10400000000004\\
1622	-3.66200000000003\\
1623	-13.4280000000001\\
1624	-13.4280000000001\\
1625	-7.32400000000007\\
1626	-8.54500000000007\\
1627	-7.32400000000007\\
1631	-7.32400000000007\\
1632	-4.88300000000004\\
1633	-4.88300000000004\\
1634	-6.10400000000004\\
1635	-8.54500000000007\\
1636	-9.76600000000008\\
1637	-6.10400000000004\\
1640	-6.10400000000004\\
1641	-14.6479999999999\\
1642	-21.973\\
1643	-14.6479999999999\\
1644	-13.4280000000001\\
1645	-10.9860000000001\\
1646	-10.9860000000001\\
1647	-15.8689999999999\\
1648	-8.54500000000007\\
1649	-10.9860000000001\\
1650	-12.2070000000001\\
1651	-10.9860000000001\\
1652	-12.2070000000001\\
1653	-6.10400000000004\\
1654	-10.9860000000001\\
1655	-12.2070000000001\\
1656	-9.76600000000008\\
1657	-8.54500000000007\\
1659	-13.4280000000001\\
1660	-13.4280000000001\\
1661	-15.8689999999999\\
1662	-25.635\\
1663	-19.5309999999999\\
1664	-12.2070000000001\\
1665	-10.9860000000001\\
1666	-13.4280000000001\\
1667	-19.5309999999999\\
1668	-13.4280000000001\\
1669	-15.8689999999999\\
1670	-14.6479999999999\\
1671	-6.10400000000004\\
1672	-4.88300000000004\\
1673	-13.4280000000001\\
1674	-12.2070000000001\\
1675	-12.2070000000001\\
1676	-18.3109999999999\\
1677	-18.3109999999999\\
1678	-14.6479999999999\\
1680	-12.2070000000001\\
1681	-18.3109999999999\\
1682	-13.4280000000001\\
1683	-12.2070000000001\\
1684	-13.4280000000001\\
1685	-10.9860000000001\\
1687	-10.9860000000001\\
1688	-9.76600000000008\\
1689	-12.2070000000001\\
1690	-18.3109999999999\\
1692	-15.8689999999999\\
1693	-12.2070000000001\\
1694	-7.32400000000007\\
1695	-8.54500000000007\\
1697	-13.4280000000001\\
1698	-18.3109999999999\\
1699	-20.752\\
1700	-17.0899999999999\\
1701	-10.9860000000001\\
1702	-15.8689999999999\\
1703	-25.635\\
1704	-15.8689999999999\\
1705	-14.6479999999999\\
1706	-20.752\\
1707	-15.8689999999999\\
1708	-13.4280000000001\\
1709	-15.8689999999999\\
1711	-13.4280000000001\\
1712	-17.0899999999999\\
1713	-14.6479999999999\\
1714	-14.6479999999999\\
1715	-15.8689999999999\\
1716	-13.4280000000001\\
1717	-13.4280000000001\\
1718	-17.0899999999999\\
1719	-13.4280000000001\\
1720	-12.2070000000001\\
1721	-14.6479999999999\\
1722	-13.4280000000001\\
1723	-8.54500000000007\\
1724	-2.44100000000003\\
1725	-3.66200000000003\\
1726	-6.10400000000004\\
1727	-14.6479999999999\\
1728	-19.5309999999999\\
1729	-15.8689999999999\\
1730	-9.76600000000008\\
1731	-13.4280000000001\\
1732	-12.2070000000001\\
1733	-12.2070000000001\\
1734	-8.54500000000007\\
1735	-8.54500000000007\\
1736	-10.9860000000001\\
1737	-8.54500000000007\\
1738	-10.9860000000001\\
1739	-10.9860000000001\\
1740	-9.76600000000008\\
1741	-7.32400000000007\\
1742	-9.76600000000008\\
1743	-13.4280000000001\\
1744	-14.6479999999999\\
1745	-10.9860000000001\\
1746	-15.8689999999999\\
1747	-14.6479999999999\\
1748	-14.6479999999999\\
1749	-18.3109999999999\\
1750	-17.0899999999999\\
1751	-14.6479999999999\\
1752	-18.3109999999999\\
1753	-10.9860000000001\\
1754	-19.5309999999999\\
1755	-13.4280000000001\\
1756	-10.9860000000001\\
1757	-15.8689999999999\\
1758	-17.0899999999999\\
1759	-14.6479999999999\\
1760	-14.6479999999999\\
1761	-21.973\\
1762	-18.3109999999999\\
1764	-13.4280000000001\\
1765	-14.6479999999999\\
1766	-12.2070000000001\\
1768	-9.76600000000008\\
1769	-10.9860000000001\\
1770	-10.9860000000001\\
1771	-17.0899999999999\\
1772	-26.855\\
1773	-20.752\\
1774	-21.973\\
1775	-20.752\\
1776	-14.6479999999999\\
1777	-9.76600000000008\\
1778	-9.76600000000008\\
1779	-6.10400000000004\\
1780	-8.54500000000007\\
1781	-13.4280000000001\\
1783	-18.3109999999999\\
1784	-17.0899999999999\\
1785	-12.2070000000001\\
1786	-19.5309999999999\\
1787	-23.193\\
1788	-23.193\\
1789	-21.973\\
1790	-28.076\\
1791	-28.076\\
1792	-13.4280000000001\\
1793	-9.76600000000008\\
1794	-9.76600000000008\\
1795	-13.4280000000001\\
1796	-20.752\\
1797	-20.752\\
1798	-14.6479999999999\\
1799	-14.6479999999999\\
1800	-8.54500000000007\\
1801	-8.54500000000007\\
1803	-10.9860000000001\\
1804	-7.32400000000007\\
1805	-7.32400000000007\\
};
\addlegendentry{True output}

\addplot [color=mycolor2, dashed, line width=2.0pt]
  table[row sep=crcr]{%
1006	-21.3779612002118\\
1007	-22.3046674829764\\
1008	-19.3186969381456\\
1009	-13.5959107476563\\
1010	-13.5371175914393\\
1011	-15.6844200354617\\
1012	-16.1267090963283\\
1013	-14.6420167838705\\
1014	-11.1704728381135\\
1015	-12.5812677930414\\
1016	-9.31675392815373\\
1017	-4.71169086319287\\
1018	-19.5865712801367\\
1019	-18.2410257080669\\
1020	-18.1741582485433\\
1021	-14.4073489614873\\
1022	-13.0884658004711\\
1023	-16.206044638772\\
1024	-15.0804444604701\\
1025	-13.9167616926445\\
1026	-15.0342654641877\\
1027	-16.458816536554\\
1028	-12.9573512902537\\
1029	-17.5067780124834\\
1030	-19.8570995014913\\
1031	-16.1534411819332\\
1032	-19.2885852518982\\
1033	-21.8280067331568\\
1034	-17.8143847948854\\
1035	-15.6980856397788\\
1036	-14.4622968590174\\
1037	-13.7419132164271\\
1038	-16.9148020940959\\
1039	-16.1509196220711\\
1040	-15.478468839734\\
1041	-16.4697226656567\\
1042	-17.0886484237774\\
1043	-22.7209603059939\\
1044	-22.1800366969067\\
1045	-16.004793950136\\
1046	-15.0693954529886\\
1047	-12.6394183712582\\
1048	-11.728506774271\\
1049	-12.7824096134423\\
1050	-12.4007270672091\\
1051	-13.2260385704712\\
1052	-14.4248080664249\\
1053	-16.4459341967597\\
1054	-15.5233608945055\\
1055	-20.8182454584605\\
1056	-18.4950802084622\\
1057	-12.3497021659271\\
1058	-12.0248564027197\\
1059	-11.1868300281678\\
1060	-12.2193547986301\\
1061	-10.134813124602\\
1062	-9.51800238071678\\
1063	-10.164637532074\\
1064	-11.0391064681237\\
1065	-9.33495574674203\\
1066	-9.13719377059783\\
1067	-11.0059512905552\\
1068	-14.7807570513005\\
1069	-16.611585726471\\
1070	-17.5868810605423\\
1071	-17.0476795508712\\
1072	-19.1625068314768\\
1073	-17.288223990962\\
1074	-15.3410030023329\\
1075	-15.1301928789421\\
1076	-14.4174714353437\\
1077	-13.3558851389621\\
1078	-20.1740458399211\\
1079	-35.0884242276604\\
1080	-32.1955418478078\\
1081	-30.2030845567713\\
1082	-19.4156480477343\\
1083	-27.2769757600518\\
1084	-36.3300260177855\\
1085	-34.3741978458997\\
1086	-26.1179076275048\\
1087	-31.9584942446002\\
1088	-46.1632077168604\\
1089	-32.9020378238306\\
1090	-25.2031021648988\\
1091	-18.1709030061568\\
1092	-17.2984270111258\\
1093	-15.4404912835648\\
1094	-16.1804466519063\\
1095	-14.4365473169346\\
1096	-10.1782318117985\\
1097	-9.98680551273583\\
1098	-9.96801571018455\\
1099	-14.559940068382\\
1100	-14.6871073079058\\
1101	-14.2109528065723\\
1102	-16.6019374946416\\
1103	-17.9583698942838\\
1104	-25.7090254983161\\
1105	-20.555549762028\\
1106	-21.2678761517241\\
1107	-18.6528048545231\\
1108	-15.7031953056701\\
1109	-19.0898838522473\\
1110	-15.6965800375572\\
1111	-13.4569742200083\\
1112	-15.7050395502913\\
1113	-16.7137331106769\\
1114	-13.9308431828988\\
1115	-12.2154472752311\\
1116	-12.1410324527394\\
1117	-11.0882227797297\\
1118	-12.1120893407485\\
1119	-10.0685105438554\\
1120	-10.1411664056702\\
1121	-12.3686285522192\\
1122	-18.0990121205612\\
1123	-18.7307052362539\\
1124	-18.8598426991346\\
1125	-14.3290397147896\\
1126	-16.4456112682108\\
1127	-25.1694468104542\\
1128	-18.729489920838\\
1129	-19.9054804186162\\
1130	-21.1277166622517\\
1131	-16.2257034367128\\
1132	-13.1410292277612\\
1133	-13.5807352628581\\
1134	-15.8418059875569\\
1135	-23.3789914093336\\
1136	-30.4823849982686\\
1137	-26.3968192263278\\
1138	-21.8608611150212\\
1139	-21.4047519665587\\
1140	-21.6750350583509\\
1141	-19.781440595742\\
1142	-18.9874909621572\\
1143	-15.1108151808187\\
1144	-13.8165797016663\\
1145	-13.4784803823663\\
1146	-12.2822649351767\\
1147	-12.4080016462747\\
1148	-16.3688786448304\\
1149	-25.2006889678016\\
1150	-21.0024386519651\\
1151	-14.25525045262\\
1152	-13.9548390865664\\
1153	-13.6319574540598\\
1154	-10.9329812939891\\
1155	-10.8055200853046\\
1156	-8.49928076861897\\
1157	-10.0967464345888\\
1158	-11.1537602739145\\
1159	-10.2676990421146\\
1160	-10.7047199226693\\
1161	-10.5953370742968\\
1162	-9.80137761397646\\
1163	-9.94947103917139\\
1164	-9.13417871673596\\
1165	-5.59421789724206\\
1166	-13.2442400023413\\
1167	-24.2634404386977\\
1168	-23.805579924848\\
1169	-15.8658622401854\\
1170	-14.9914930113628\\
1171	-12.7957326128567\\
1172	-12.2370866247973\\
1173	-8.80544928098288\\
1174	-11.8696357804424\\
1175	-14.1680906829536\\
1176	-19.9917917029982\\
1177	-33.4828367464604\\
1178	-41.1259625968573\\
1179	-28.6013329407774\\
1180	-27.5359330735284\\
1181	-20.6803759928866\\
1182	-20.8398069294101\\
1183	-24.4288006509391\\
1184	-25.0723051774251\\
1185	-20.3699890408684\\
1186	-18.1057012551059\\
1187	-18.8042455535326\\
1188	-17.3221772567219\\
1189	-19.8029063393888\\
1190	-15.1392522996773\\
1191	-21.2810863116956\\
1192	-29.1872685907774\\
1193	-19.3430382910137\\
1194	-14.7541129438391\\
1195	-15.8530158609919\\
1196	-23.6424103395163\\
1197	-29.7247190146893\\
1198	-32.6124537863209\\
1199	-33.6853053698187\\
1200	-32.7536625493444\\
1201	-24.1712410180112\\
1202	-23.218902312251\\
1203	-26.668721582842\\
1204	-30.4058144513647\\
1205	-19.3881862281605\\
1206	-13.961221974856\\
1207	-17.7220815423598\\
1208	-18.1185935210253\\
1209	-12.6766182530398\\
1210	-13.8373040380268\\
1211	-16.9790943444743\\
1212	-13.9739155805739\\
1213	-11.7266919109488\\
1214	-13.5521082244568\\
1215	-12.8226929439491\\
1216	-13.896846879562\\
1217	-19.341117223833\\
1218	-17.3923568876753\\
1219	-14.2139400698729\\
1220	-13.9285527530212\\
1221	-19.4606549967784\\
1222	-33.4019423422778\\
1223	-23.4780144277431\\
1224	-13.2852516492771\\
1225	-13.9908042634456\\
1226	-13.9999692378165\\
1227	-14.6319469907789\\
1228	-11.5069896238292\\
1229	-11.5804578046605\\
1230	-12.1237846410459\\
1231	-15.2161173877296\\
1232	-16.5070900951678\\
1233	-12.1525365165764\\
1234	-16.346757150922\\
1235	-21.3102857378628\\
1236	-20.573843865594\\
1237	-16.6984686376359\\
1238	-16.514982834165\\
1239	-22.7230187635298\\
1240	-15.4853911982077\\
1241	-11.9486794453685\\
1242	-10.8685104074164\\
1243	-13.4222084237431\\
1244	-15.9192236822114\\
1245	-14.0808270148996\\
1246	-11.5844305207124\\
1247	-13.3613174048712\\
1248	-14.6648636717016\\
1249	-12.3558645229027\\
1250	-13.0625464903762\\
1251	-13.1643648605866\\
1252	-12.1538211243862\\
1253	-11.4415386955711\\
1254	-10.8523583484973\\
1255	-10.5534993836875\\
1256	-10.0664022072219\\
1257	-9.87536167719827\\
1258	-11.5178543236484\\
1259	-15.6325159979897\\
1260	-19.9327435793812\\
1261	-26.4789321791388\\
1262	-17.5867095265028\\
1263	-13.8703487563871\\
1264	-12.2210503437132\\
1265	-11.2363223937939\\
1266	-11.9571793553314\\
1267	-10.1566335400632\\
1268	-10.5382530702295\\
1269	-9.71569986487884\\
1270	-17.1388103178128\\
1271	-17.2520879568435\\
1272	-19.6820874352227\\
1273	-16.271913686573\\
1274	-15.6056187400841\\
1275	-12.9080506770042\\
1276	-12.1424037452248\\
1277	-12.0167589163329\\
1278	-9.10088851509295\\
1279	-6.5556456970794\\
1280	-10.3830366987988\\
1281	-11.624912067643\\
1282	-13.4085385963863\\
1283	-14.0728767060684\\
1284	-22.2533241592096\\
1285	-22.1484703419674\\
1286	-14.7666397435507\\
1287	-17.6445589495081\\
1288	-18.9319145665243\\
1289	-24.764406883116\\
1290	-19.6507977573924\\
1291	-14.7740587243939\\
1292	-13.0904480047577\\
1293	-12.4270107729581\\
1294	-9.14918922736615\\
1295	-7.67092272932177\\
1296	-8.37772388442977\\
1297	-8.59559686256785\\
1298	-7.9683897821792\\
1299	-12.1549827942179\\
1300	-13.0851888556854\\
1301	-12.4525996167331\\
1302	-14.2809026356529\\
1303	-11.5323476408341\\
1304	-12.1744610120713\\
1305	-7.60482825998361\\
1306	-14.5177862473954\\
1307	-16.090997559395\\
1308	-15.3559289448835\\
1309	-15.4912715241292\\
1310	-13.5699205546548\\
1311	-13.8364668065635\\
1312	-19.7713775312941\\
1313	-20.1604006064397\\
1314	-15.1365947989214\\
1315	-21.8346065700982\\
1316	-19.279850918472\\
1317	-17.1300949124704\\
1318	-20.1244604080894\\
1319	-19.3174643620348\\
1321	-14.712101762075\\
1322	-19.7280881353568\\
1323	-31.3735532881635\\
1324	-21.5316480264719\\
1325	-15.2274529520669\\
1326	-15.0054201572734\\
1327	-16.3157456756869\\
1328	-17.8736181158054\\
1329	-19.9677519430227\\
1330	-15.2467722537285\\
1331	-15.7708077287784\\
1332	-19.8632869772196\\
1333	-21.5811687879529\\
1334	-15.9692314379995\\
1335	-14.1530168102984\\
1336	-16.3073405258101\\
1337	-26.9386361608451\\
1338	-22.4894730020917\\
1339	-21.9126762870376\\
1340	-15.4568705147162\\
1341	-15.484140156273\\
1342	-15.6992666256183\\
1343	-11.8807668071242\\
1344	-11.9983284519947\\
1345	-10.1447603910285\\
1346	-8.67084914823818\\
1347	-5.28577764962051\\
1348	-12.2966594423419\\
1349	-15.9494352569309\\
1350	-12.8358689526349\\
1351	-12.6712098548087\\
1352	-14.1975901618471\\
1353	-12.3349112895403\\
1354	-11.580712815851\\
1355	-10.8631759963496\\
1356	-10.2414143112421\\
1357	-9.21578683127359\\
1358	-12.7435400217169\\
1359	-18.2909324397383\\
1360	-12.7918283960846\\
1361	-12.5230901237071\\
1362	-10.9439708025648\\
1363	-10.6967964121195\\
1364	-9.70622126436206\\
1365	-11.87759910258\\
1366	-24.4676338160982\\
1367	-17.8093537038501\\
1368	-22.7311979394033\\
1369	-27.4516497897723\\
1370	-26.0381561725562\\
1371	-20.1517793504379\\
1372	-15.8861050588403\\
1373	-16.4865209432678\\
1374	-16.8731120849952\\
1375	-18.4116306252715\\
1376	-20.5762689062699\\
1377	-21.1715491471825\\
1378	-25.2482431753351\\
1379	-30.7860073503975\\
1380	-35.5318912926928\\
1381	-24.2272195788862\\
1382	-25.3615519333384\\
1383	-31.3236319058301\\
1384	-24.2305693823882\\
1385	-25.1891639370679\\
1386	-23.5996168846218\\
1387	-13.321657947042\\
1388	-13.3247876471455\\
1389	-12.3254332504218\\
1390	-15.1795860606699\\
1391	-13.0194603484576\\
1392	-10.463171083951\\
1393	-10.9418882172038\\
1394	-10.5890749246903\\
1395	-10.2108609883071\\
1396	-8.9081871738988\\
1397	-9.80140854484193\\
1398	-9.91680902239045\\
1399	-10.4550591579075\\
1400	-17.0557107313957\\
1401	-13.4563232437858\\
1402	-17.4688448447992\\
1403	-29.6251287360537\\
1404	-25.4891138110681\\
1405	-30.5400427166483\\
1406	-22.4406804931182\\
1407	-23.5656911100762\\
1408	-18.6608247227227\\
1409	-14.5119201075004\\
1410	-14.4237109945263\\
1411	-12.1403948272123\\
1412	-11.5368382222937\\
1413	-10.9682742139455\\
1414	-9.84567675886456\\
1415	-12.2939109561885\\
1416	-6.75969490009606\\
1417	-9.0916614003454\\
1418	-15.6088952228033\\
1419	-17.3331326474374\\
1420	-17.0976500399649\\
1421	-16.9660464435124\\
1422	-18.0069078432152\\
1423	-16.6766698976114\\
1424	-14.7519266050883\\
1425	-15.1741681784035\\
1426	-13.2060690987421\\
1427	-16.4336372258326\\
1428	-12.9944060312948\\
1429	-12.9537256882036\\
1430	-14.395351183784\\
1431	-20.9904881486084\\
1432	-16.8339292633914\\
1433	-13.2017313100564\\
1434	-12.4522146934228\\
1435	-13.2711470029112\\
1436	-11.076357013789\\
1437	-10.6585730956069\\
1438	-10.6798003958354\\
1439	-13.3530622900753\\
1440	-15.5608977929051\\
1441	-13.2202709401492\\
1442	-14.5662710850638\\
1443	-14.8688790915737\\
1444	-12.019700321564\\
1445	-11.1616901203054\\
1446	-13.9024158092745\\
1447	-24.0689022506965\\
1448	-20.8440775426184\\
1449	-17.5433039816169\\
1450	-18.5332492680798\\
1451	-15.6336287958566\\
1452	-14.9581923341059\\
1453	-12.650324923512\\
1454	-14.2489129462097\\
1455	-14.2142458523056\\
1456	-11.6791599699743\\
1457	-12.4038652855675\\
1458	-14.8320017114656\\
1459	-18.0915024521644\\
1460	-17.6659730850697\\
1461	-18.222214399533\\
1462	-24.1108035663312\\
1463	-30.2587452950133\\
1464	-34.7089713816572\\
1465	-21.3788620819632\\
1466	-15.1992179199217\\
1467	-13.9982526865449\\
1468	-12.412016900598\\
1469	-9.67161292932542\\
1470	-10.2148479723255\\
1471	-13.3270052436899\\
1472	-13.1947434491478\\
1473	-12.0991810317794\\
1474	-14.6246064668458\\
1475	-16.9433825912406\\
1476	-19.4184273028179\\
1477	-21.32583388906\\
1478	-31.8922241816188\\
1479	-27.9757639265342\\
1480	-20.5228946894545\\
1481	-15.7369128430803\\
1482	-15.9294485250095\\
1483	-17.0512365294953\\
1484	-19.0243941303881\\
1485	-15.1569139782328\\
1486	-15.2756283479207\\
1487	-14.5476959401674\\
1488	-11.6452802622257\\
1489	-12.5827258943391\\
1490	-17.0263439607952\\
1491	-14.0218852198\\
1492	-14.8639751215997\\
1493	-23.8466714343999\\
1494	-20.3560311902691\\
1495	-18.2387534760915\\
1496	-23.8423991218715\\
1497	-23.8127048967995\\
1498	-16.9995634699553\\
1499	-13.0966693565908\\
1500	-12.5024168779917\\
1501	-16.5120954716779\\
1502	-30.2305746983295\\
1503	-27.8555276748771\\
1504	-24.9153909632682\\
1505	-22.8833027325663\\
1506	-15.3010050956216\\
1507	-15.4550984581469\\
1508	-11.8729110971972\\
1509	-11.8122207267761\\
1510	-9.86056310738695\\
1511	-12.1481321861886\\
1512	-12.7873428555563\\
1513	-10.4364014420228\\
1514	-11.3213300555308\\
1515	-15.3995612009012\\
1516	-17.4564746143415\\
1517	-22.4460767608784\\
1518	-26.1877249393149\\
1519	-35.8068820139038\\
1520	-23.6301655758987\\
1521	-18.8976426674401\\
1522	-21.3078375658658\\
1523	-18.6055845652684\\
1524	-13.9287879821638\\
1525	-16.1778616619599\\
1526	-25.6288174710162\\
1527	-28.8871373137281\\
1528	-15.8001659980289\\
1529	-13.7194077480001\\
1530	-12.0674356643931\\
1531	-13.3551025906045\\
1532	-9.12291265712906\\
1533	-4.49458797620923\\
1534	-8.19319669026891\\
1535	-11.7898660254432\\
1536	-10.1381909323648\\
1537	-10.6003900108835\\
1538	-9.54349247686037\\
1539	-9.26390358203457\\
1540	-9.65380146600819\\
1541	-8.76046112401377\\
1542	-10.9442578163801\\
1543	-18.5147488947462\\
1544	-19.8933646535309\\
1545	-17.702501685488\\
1546	-20.0396824186714\\
1547	-26.9302591472072\\
1548	-26.4085042677082\\
1549	-31.1031168827865\\
1550	-27.8733404241254\\
1551	-21.857629673387\\
1552	-17.0164064582088\\
1553	-17.2094370279051\\
1554	-24.5881905871859\\
1555	-28.7021218162265\\
1556	-18.045882417556\\
1557	-14.335725802961\\
1558	-12.3403934529326\\
1559	-13.2213251691014\\
1560	-8.49011259538838\\
1561	-9.48954746220784\\
1562	-5.54847093253034\\
1563	-7.92833447626708\\
1564	-11.619003528235\\
1565	-10.634758119427\\
1566	-9.96927322463421\\
1567	-10.4425110114134\\
1568	-7.00851062069614\\
1569	-8.09702147765029\\
1570	-9.08360407064288\\
1571	-8.79930063036613\\
1572	-8.42688964840386\\
1573	-6.18924609822488\\
1574	-20.0079384878136\\
1575	-26.1321133496112\\
1576	-23.9955426647539\\
1577	-19.2038046226264\\
1578	-23.8193533113083\\
1579	-36.5390628968219\\
1580	-30.5161900714063\\
1581	-30.2394002691401\\
1582	-22.6887765262795\\
1583	-24.0658472870468\\
1584	-25.2348312308968\\
1585	-16.4338161939486\\
1586	-17.4947355196834\\
1587	-11.6153283278561\\
1588	-12.7303657528269\\
1589	-16.2374055882633\\
1590	-12.267004564203\\
1591	-15.345088233049\\
1592	-15.8223790529578\\
1593	-14.7054312070786\\
1594	-15.8465875241518\\
1595	-15.0247534540338\\
1596	-16.2142650231322\\
1597	-17.0173520390949\\
1598	-16.9600118245191\\
1599	-14.1613545328303\\
1600	-14.7606469687778\\
1601	-12.3895910680828\\
1602	-16.3144851327788\\
1603	-7.20124676649084\\
1604	-8.4315213633854\\
1605	-8.75512519478184\\
1606	-13.8261814386981\\
1607	-2.85854588129632\\
1608	-4.55600573470883\\
1609	-9.48619624406047\\
1610	-11.0172554246797\\
1611	-11.7257936213932\\
1612	-15.4982547814864\\
1613	-15.9475593763459\\
1614	-13.9069399802395\\
1615	-13.908729025238\\
1616	-10.5785451157103\\
1617	-14.202239497944\\
1618	-9.89856281767925\\
1619	-10.5183303960532\\
1620	-5.82316274364871\\
1621	-8.154365869618\\
1622	-6.90568490287137\\
1623	-11.1005700171913\\
1624	-13.2803726804866\\
1625	-12.0820246997309\\
1626	-10.2070623893408\\
1627	-10.3471638098608\\
1628	-10.0731073011334\\
1629	-8.84249344038244\\
1630	-9.18861767355429\\
1631	-8.46181019799133\\
1632	-8.98198158165951\\
1633	-7.23889574962186\\
1634	-7.18844490912647\\
1635	-8.27989216711148\\
1636	-9.73779214013075\\
1637	-10.0041809911636\\
1638	-8.78632618023357\\
1639	-8.51035942412273\\
1640	-8.01961310402908\\
1641	-11.48372284635\\
1642	-22.7181903316916\\
1643	-15.6569554365035\\
1644	-15.0832239276692\\
1645	-13.4476917907723\\
1646	-15.4635893978757\\
1647	-14.8380637230732\\
1648	-12.9549645157042\\
1649	-11.8036536593115\\
1650	-11.2118678048973\\
1651	-13.7222848835183\\
1652	-10.605585047161\\
1653	-12.2761216833958\\
1654	-9.89969446186092\\
1655	-13.2431071598462\\
1656	-11.7362055674889\\
1657	-11.2318919346915\\
1658	-10.8735076668879\\
1659	-14.7377333772458\\
1660	-14.0758840116944\\
1661	-19.0508219407375\\
1662	-27.3546065307473\\
1663	-21.9746776175982\\
1664	-15.8165539680433\\
1665	-13.148360274253\\
1666	-16.2226610390132\\
1667	-20.1006344464465\\
1668	-15.5563388184491\\
1669	-18.0357002410622\\
1670	-12.3619641144664\\
1671	-14.5204325362745\\
1672	-10.5467585604629\\
1673	-12.0545805014156\\
1674	-15.7198093219743\\
1675	-12.2040250010127\\
1676	-17.8299607952333\\
1677	-17.7903209612437\\
1678	-16.800178986339\\
1679	-13.7901919077037\\
1680	-15.5006655151244\\
1681	-17.8642794766663\\
1682	-16.4517594100439\\
1683	-15.4116241336353\\
1684	-13.9195954975996\\
1685	-12.979707827313\\
1686	-12.7952853969239\\
1687	-11.4183535953287\\
1688	-11.9930802892347\\
1689	-15.2601814461427\\
1690	-17.1775579127782\\
1691	-18.8275048619785\\
1692	-16.8688757257175\\
1693	-14.394888028582\\
1694	-13.1222215514169\\
1695	-11.2091758318445\\
1696	-12.1679291792309\\
1697	-14.6338519238186\\
1698	-17.5736916503106\\
1699	-21.6882650714838\\
1700	-17.3676899342584\\
1701	-14.339864480103\\
1702	-16.9843629065942\\
1703	-27.167878366118\\
1704	-18.094838155889\\
1705	-18.2514734485358\\
1706	-19.5282234885906\\
1707	-18.0790989943519\\
1708	-15.1895763816285\\
1709	-16.1225941487014\\
1710	-15.9116451370753\\
1711	-16.9924863246165\\
1712	-17.9945166907289\\
1713	-16.5997470317175\\
1714	-17.1196008378788\\
1715	-15.9415799506705\\
1716	-16.5122462989307\\
1717	-14.0643753967674\\
1718	-15.9780717716901\\
1719	-14.9310503913935\\
1720	-14.5548534813267\\
1721	-15.5417313052076\\
1722	-15.6994721440342\\
1723	-11.8068027961433\\
1724	-13.2540408050334\\
1725	-10.6746622273074\\
1726	-5.243197265954\\
1727	-11.7804733542444\\
1728	-21.8815077252468\\
1729	-19.2957825117981\\
1730	-15.0781837936968\\
1731	-14.0375005480121\\
1732	-14.3962939843043\\
1733	-12.8034083835701\\
1734	-11.7391898346193\\
1735	-11.0363673161662\\
1736	-11.3104732847589\\
1737	-11.5224669510715\\
1738	-11.4803471719365\\
1739	-11.3337608647068\\
1740	-11.7096893645421\\
1741	-10.6429855791482\\
1742	-9.85126032164226\\
1743	-15.3581530775903\\
1744	-12.7149717316913\\
1745	-15.0660361238677\\
1746	-12.9979940347782\\
1747	-17.4967451796299\\
1748	-17.1835541577982\\
1749	-19.6717737760018\\
1750	-17.2788547467151\\
1751	-17.8232809093984\\
1752	-17.3308737031975\\
1753	-13.5858116733712\\
1754	-14.8098472261061\\
1755	-15.5307096335612\\
1756	-15.5482972926711\\
1757	-15.4657225766707\\
1758	-18.8107418338711\\
1759	-16.1093214004713\\
1760	-17.5826713027859\\
1761	-22.4791885629666\\
1762	-18.5968773675766\\
1763	-17.2590307981923\\
1764	-15.3699065667504\\
1765	-16.9534985930504\\
1766	-14.8521167183887\\
1767	-13.7871163112484\\
1768	-12.0227993921308\\
1769	-12.8481143170927\\
1770	-11.6879284602987\\
1771	-17.9943496897906\\
1772	-26.6201191564812\\
1773	-22.9146175719952\\
1774	-20.5068747780194\\
1775	-22.028154993824\\
1776	-15.5613295431185\\
1777	-15.3540605722164\\
1778	-12.5042193201293\\
1779	-11.7432168931459\\
1780	-9.89697527744079\\
1781	-12.7846756033687\\
1782	-17.7739441057456\\
1783	-18.8258442922076\\
1784	-17.4407102259247\\
1785	-14.3842979064743\\
1786	-20.2341257068563\\
1787	-26.3055220489014\\
1788	-24.998650393045\\
1789	-20.8514714460632\\
1790	-31.9811228446968\\
1791	-27.0506536936937\\
1792	-16.2024481712538\\
1793	-15.0004726376978\\
1794	-12.4162524261073\\
1795	-15.3739782438474\\
1796	-19.6562724394928\\
1797	-26.5960532156384\\
1798	-19.5254339316107\\
1799	-14.5212660905488\\
1800	-12.7500628571545\\
1801	-11.6798899352998\\
1802	-11.0765522077247\\
1803	-11.9566847083834\\
1804	-10.7485182938929\\
1805	-10.1603826450321\\
};
\addlegendentry{OSA predition}

\addplot [color=mycolor3, dotted, line width=2.0pt]
  table[row sep=crcr]{%
1006	-18.3109999999999\\
1007	-23.193\\
1008	-18.3109999999999\\
1009	-9.76600000000008\\
1010	-13.5371175914393\\
1011	-14.8247892207007\\
1012	-16.7998718633642\\
1013	-15.5155427822235\\
1014	-12.9984157496485\\
1015	-16.872336396437\\
1016	-16.0509823372786\\
1017	-11.9557184709174\\
1018	-26.397422103053\\
1019	-26.6176711966964\\
1020	-25.0269465607194\\
1021	-20.2629159154794\\
1022	-18.3204018143599\\
1023	-22.0765786894121\\
1024	-18.961752971237\\
1025	-17.3979428452315\\
1026	-19.1671488675499\\
1027	-20.2137324339456\\
1028	-17.6718875704769\\
1029	-22.3416825911409\\
1030	-22.8374940171411\\
1031	-19.0648054718499\\
1032	-23.0807011514996\\
1033	-24.1196518043491\\
1034	-20.4227513780552\\
1035	-18.6782477514228\\
1036	-17.4646319724618\\
1037	-17.9802738377546\\
1038	-20.1409513441574\\
1039	-19.9957174998522\\
1040	-19.2428750639099\\
1041	-19.7593399826333\\
1042	-20.6901135697713\\
1043	-26.3172468762818\\
1044	-25.5335648555995\\
1045	-19.2924819846837\\
1046	-18.6761864734333\\
1047	-15.9825692496436\\
1048	-15.9305311204964\\
1049	-16.1120010799168\\
1050	-15.4693245021458\\
1051	-15.5218788092059\\
1052	-17.1403026057906\\
1053	-18.9846219725453\\
1054	-18.2980558537772\\
1055	-24.1263407357462\\
1056	-21.2290420369909\\
1057	-16.4313127064681\\
1058	-16.1278214606614\\
1059	-16.2027314443851\\
1060	-17.5603655561392\\
1061	-14.8148998295792\\
1062	-14.8695257593083\\
1063	-14.4259977091315\\
1064	-14.3236396262444\\
1065	-14.0307867369856\\
1066	-13.8327831860518\\
1067	-14.6054389387436\\
1068	-18.9367314156223\\
1069	-19.6202635831983\\
1070	-20.8430565857684\\
1071	-19.8657056841894\\
1072	-22.0219958611187\\
1073	-20.0417098238627\\
1074	-19.0071784846596\\
1075	-18.3017287291023\\
1076	-18.1961842809444\\
1077	-17.8330704023786\\
1078	-24.4264189189273\\
1079	-38.1824645319173\\
1080	-37.8307782477773\\
1081	-34.935207263263\\
1082	-23.5474374414978\\
1083	-31.8697377835808\\
1084	-39.9885000902391\\
1085	-38.8739146957425\\
1086	-30.6996155736592\\
1087	-37.9936265510587\\
1088	-50.8795298561815\\
1089	-39.5556577750351\\
1090	-31.8909870529394\\
1091	-23.5328856447127\\
1092	-21.4713991290346\\
1093	-19.5085101970419\\
1094	-20.7040197985843\\
1095	-18.5096752710624\\
1096	-15.1589944569689\\
1097	-15.0179205524717\\
1098	-15.0424302454057\\
1099	-18.7760860645342\\
1100	-18.8612879008394\\
1101	-18.5326720546063\\
1102	-20.8107926292078\\
1103	-21.9146373463007\\
1104	-29.3054379781779\\
1105	-25.9040665542741\\
1106	-27.1906141833326\\
1107	-23.2899283640645\\
1108	-20.9907491420527\\
1109	-23.4911860896073\\
1110	-19.9115186352462\\
1111	-18.2791220565796\\
1112	-20.0845214428634\\
1113	-20.4902315162876\\
1114	-17.9497783726638\\
1115	-17.38621315709\\
1116	-16.0921178542681\\
1117	-15.8355999368712\\
1118	-16.6974779567479\\
1119	-14.4250038381517\\
1120	-14.863874946227\\
1121	-16.425373047304\\
1122	-21.9551108636199\\
1123	-22.4440287492862\\
1124	-22.4927775805238\\
1125	-17.8430132459739\\
1126	-20.7203172395737\\
1127	-30.5169778770144\\
1128	-23.9190841951734\\
1129	-25.6350128182655\\
1130	-25.6758743374473\\
1131	-19.628764922242\\
1132	-17.4331295227175\\
1133	-18.1404209966472\\
1134	-20.1342192393074\\
1135	-28.5178301250912\\
1136	-35.5303374578541\\
1137	-32.5492071397\\
1138	-26.714940981517\\
1140	-26.3317054340566\\
1141	-24.8267917817657\\
1142	-24.2266666724092\\
1143	-19.5433251508659\\
1144	-18.1967372303966\\
1145	-17.6805846628222\\
1146	-17.0303098450154\\
1147	-17.2320074761849\\
1148	-21.1471551354227\\
1149	-29.8754446965586\\
1150	-25.7349327154209\\
1151	-19.4148272953835\\
1152	-18.2815225992431\\
1153	-18.3995183253635\\
1154	-15.7570537517513\\
1155	-15.7378630017793\\
1156	-14.2118647704265\\
1157	-15.8584863193457\\
1158	-15.2539128408355\\
1159	-15.3907652802636\\
1160	-15.3098235892669\\
1161	-14.995371449673\\
1162	-14.2921775860027\\
1163	-14.3028362544303\\
1164	-14.3616701174697\\
1165	-11.776400186488\\
1166	-17.6642811113136\\
1167	-28.1843600496627\\
1168	-29.2818745220977\\
1169	-20.710667705474\\
1170	-18.743381051062\\
1171	-17.6051277849942\\
1172	-17.5209619063214\\
1173	-15.7902806161903\\
1174	-17.302367314797\\
1175	-20.7826509591134\\
1176	-24.9728439920743\\
1177	-39.3214460777699\\
1178	-47.6772054355217\\
1179	-37.1240560614715\\
1180	-35.9230780274763\\
1181	-28.4216637997795\\
1182	-28.5383778290663\\
1183	-30.0478983892351\\
1184	-31.057029463821\\
1185	-25.4638251643953\\
1186	-23.8663886085774\\
1187	-24.0422533779467\\
1188	-22.3742090483586\\
1189	-24.8503500636998\\
1190	-20.6425542712423\\
1191	-26.6575346746195\\
1192	-33.4177331042376\\
1193	-24.4379164991651\\
1194	-18.5531269851017\\
1195	-19.3150141278106\\
1196	-27.4115258162713\\
1197	-33.1754802590503\\
1198	-37.7229527772304\\
1199	-39.5690711090197\\
1200	-39.1930067303242\\
1201	-31.0823563263893\\
1202	-29.4158618061263\\
1203	-32.56867851181\\
1204	-36.0469317086399\\
1205	-24.8235082673305\\
1206	-18.6061221797777\\
1207	-22.3430799068465\\
1208	-21.372475869061\\
1209	-16.6323228098413\\
1210	-18.119067730215\\
1211	-20.5883083340009\\
1212	-18.1091599858082\\
1213	-16.1515713462486\\
1214	-17.3750096793192\\
1215	-17.068468688072\\
1216	-19.1221795836552\\
1217	-24.8167152297215\\
1218	-22.5392308202643\\
1219	-18.6912812816317\\
1220	-18.3770491617781\\
1221	-23.8688339046957\\
1222	-37.9313325463838\\
1223	-29.8604993654296\\
1224	-20.561254014058\\
1225	-19.1175363544908\\
1226	-19.7608987231761\\
1227	-19.790695185698\\
1228	-16.7725780999392\\
1229	-17.064802626338\\
1230	-17.7615274083864\\
1231	-20.6307224012216\\
1232	-21.9036860576487\\
1233	-17.1842334322316\\
1234	-20.6010329026267\\
1235	-25.3462892521159\\
1236	-24.2343636644282\\
1237	-20.9240639729878\\
1238	-21.5970699453462\\
1239	-27.3618681952325\\
1240	-20.2469089096792\\
1241	-16.4079819754238\\
1242	-16.012779449441\\
1243	-16.6982859268276\\
1244	-20.1375420419706\\
1245	-18.8703201906021\\
1246	-16.6864339568845\\
1247	-18.9286468786572\\
1248	-19.4739788874974\\
1249	-16.9140662483301\\
1250	-17.7953578432539\\
1251	-17.2116822341302\\
1252	-16.8556470031535\\
1253	-17.1191412852879\\
1254	-14.916818831307\\
1255	-14.9900760371572\\
1256	-14.0137262161093\\
1257	-13.9468206243962\\
1259	-19.838520437137\\
1260	-24.0378041110037\\
1261	-31.2120018533737\\
1262	-22.353363814379\\
1263	-18.1693312915791\\
1264	-17.1884196803755\\
1265	-16.6222794244966\\
1266	-17.439513822221\\
1267	-14.9937657348441\\
1268	-15.2148550520078\\
1269	-15.7119692992401\\
1270	-21.6460484212096\\
1271	-22.0380291832398\\
1272	-25.274748723514\\
1273	-20.0976232114451\\
1274	-19.1026424023296\\
1275	-16.1416116161442\\
1276	-16.1552688245072\\
1277	-15.4180641345292\\
1278	-14.4229026541695\\
1279	-12.8687690560039\\
1280	-16.4201553921191\\
1281	-17.3925702073584\\
1282	-18.6620193340536\\
1283	-18.7921882263965\\
1284	-26.6739869167486\\
1285	-26.5661277346696\\
1286	-20.2070546425359\\
1287	-22.0728537240204\\
1288	-23.0870855301785\\
1289	-28.4276160263471\\
1290	-24.0868641650397\\
1291	-19.1299167567745\\
1292	-16.741093296909\\
1293	-17.5123771935816\\
1294	-15.4408679558633\\
1295	-14.1213028638808\\
1297	-13.8240708766787\\
1298	-13.7078919095779\\
1299	-16.5835876505184\\
1300	-17.0513502034075\\
1301	-16.6579018918674\\
1302	-17.6465448350893\\
1303	-14.9798914078674\\
1304	-15.976385803986\\
1305	-13.3752915068901\\
1306	-18.4813258416632\\
1307	-19.6486877411978\\
1308	-19.4960535920427\\
1309	-18.2100661612135\\
1310	-16.5534189384368\\
1311	-17.9010255487422\\
1312	-22.7882084537928\\
1313	-23.6245028392007\\
1314	-18.6422888491504\\
1315	-25.3659575671641\\
1316	-22.6908989834324\\
1317	-21.3399992185243\\
1318	-23.6222548510111\\
1319	-23.0501203467065\\
1320	-20.0719357337716\\
1321	-18.4803837798827\\
1322	-23.3834615528763\\
1323	-35.3543356755072\\
1324	-26.6753314389105\\
1325	-18.9826662058169\\
1326	-18.732942126077\\
1327	-20.2170418154949\\
1328	-22.1403627031489\\
1329	-24.3778444794955\\
1330	-19.9701059732038\\
1331	-20.9028460223378\\
1332	-24.6790803625413\\
1333	-26.4891246995562\\
1334	-20.3078772870906\\
1335	-18.5661556799212\\
1336	-20.7286941448426\\
1337	-31.1538515448356\\
1338	-27.6345608430297\\
1339	-26.7583752078313\\
1340	-19.2490355587881\\
1341	-19.0789520518751\\
1342	-18.9832273457207\\
1343	-15.6628867067757\\
1344	-15.6687718855414\\
1345	-15.4768797405488\\
1346	-15.178491900632\\
1347	-12.5265835387511\\
1348	-18.1559260689478\\
1349	-21.2120348090014\\
1350	-17.5137979635026\\
1351	-17.3031614148715\\
1352	-18.1562886865195\\
1353	-15.9597490230083\\
1354	-15.5964719888866\\
1355	-15.0928311018329\\
1356	-14.412781782009\\
1357	-14.2703278491463\\
1358	-16.8443314343751\\
1359	-21.4714373593299\\
1360	-16.40495305731\\
1361	-15.7989950724866\\
1362	-14.8344947499577\\
1363	-14.8785080142075\\
1364	-13.7956848037495\\
1365	-16.5231507605683\\
1366	-28.9713457493451\\
1367	-23.0883340029973\\
1368	-26.9889025528555\\
1369	-33.0747360869441\\
1370	-31.8500441727581\\
1371	-25.9721341048355\\
1372	-21.4125667139472\\
1373	-21.5536714180741\\
1374	-21.8651188589597\\
1375	-23.4901196393473\\
1376	-25.4126803349779\\
1377	-25.7790662026791\\
1378	-29.9462014941814\\
1379	-34.7377348444543\\
1380	-40.665922678043\\
1381	-29.7394958951224\\
1382	-29.9203847682318\\
1383	-35.5871193278601\\
1384	-29.530491879832\\
1385	-30.2222291742523\\
1386	-28.1833024254995\\
1387	-18.180057642054\\
1388	-17.0474167004074\\
1389	-16.5859597542808\\
1390	-19.8336670679439\\
1391	-17.8926940361939\\
1392	-15.6544438197946\\
1393	-16.3844479936758\\
1394	-14.9595094143649\\
1395	-14.577279403906\\
1396	-14.0437022671638\\
1397	-14.3945433306733\\
1398	-13.72600846571\\
1399	-15.4094322470237\\
1400	-20.4251015960319\\
1401	-17.3295406972211\\
1402	-21.6919297192767\\
1403	-33.4128620312028\\
1404	-30.9827375798207\\
1405	-35.5090334857082\\
1406	-27.5926611888328\\
1407	-27.8206148388194\\
1408	-22.683020631076\\
1409	-18.3112540631821\\
1410	-19.1962581891705\\
1411	-17.2384183566908\\
1412	-16.0054182610268\\
1413	-16.5226611257419\\
1414	-14.1278603928445\\
1415	-15.5523508353017\\
1416	-13.2620258725547\\
1417	-14.2626759208154\\
1418	-20.4514608656832\\
1419	-23.2996643048964\\
1420	-22.5924337161259\\
1421	-21.4973931305017\\
1422	-23.0752259216185\\
1423	-21.1697843447985\\
1424	-19.0772115543057\\
1425	-19.3189809167518\\
1426	-17.6721195765165\\
1427	-21.0964898689588\\
1428	-17.4192170685988\\
1429	-16.461018907235\\
1430	-18.22847084923\\
1431	-24.6165802102282\\
1432	-20.822441139903\\
1433	-17.6860833519536\\
1434	-16.7902905850488\\
1435	-17.5094528491411\\
1436	-15.4013088730385\\
1437	-15.0491739281181\\
1438	-15.5415374950267\\
1440	-19.2163795916338\\
1441	-17.0569893434845\\
1442	-18.3949769917308\\
1443	-18.444977636568\\
1444	-15.4968704057242\\
1445	-15.2897128868283\\
1446	-18.7604755041216\\
1447	-27.6974732261992\\
1448	-26.139331654241\\
1449	-22.5690928059994\\
1450	-22.2696329130617\\
1451	-20.0242795482368\\
1452	-19.3455459987001\\
1453	-17.4843692601387\\
1454	-19.4594724661029\\
1455	-18.9063854731812\\
1456	-15.8441563884696\\
1457	-17.1172893646299\\
1458	-18.7162208351183\\
1459	-21.7012882048223\\
1460	-22.1172867962159\\
1461	-21.9329339418412\\
1462	-27.9666834658356\\
1463	-35.0569156467179\\
1464	-39.5664009805123\\
1465	-27.011577756527\\
1466	-19.8813864628587\\
1467	-18.1875803908927\\
1468	-17.8102753041887\\
1469	-16.2253733270586\\
1470	-15.8945518175101\\
1471	-19.8618839340859\\
1472	-19.5606494548188\\
1473	-17.5143603041015\\
1474	-19.3294422178265\\
1475	-21.531249610681\\
1476	-23.6685235701827\\
1477	-24.8816984635503\\
1478	-35.9546074447442\\
1479	-32.7578433557242\\
1480	-26.1386048502698\\
1481	-21.040030873196\\
1482	-20.8019847313801\\
1483	-21.8148277923174\\
1484	-23.9760226286187\\
1485	-19.8559570354823\\
1486	-19.3702306760495\\
1487	-19.1956358987352\\
1488	-15.6642816890276\\
1489	-17.1197432295899\\
1490	-22.9640443582466\\
1491	-18.7566897495569\\
1492	-18.8924094889194\\
1493	-28.6685578721185\\
1494	-24.32203656296\\
1495	-22.3687670033751\\
1496	-28.41378512087\\
1497	-28.3617798237863\\
1498	-20.982899995219\\
1499	-17.0765039335379\\
1500	-17.425561773967\\
1501	-21.6231799185546\\
1502	-35.4455668598744\\
1503	-35.2939909236875\\
1504	-31.6776861936844\\
1505	-28.4469558870862\\
1506	-20.823307673656\\
1507	-20.571873644137\\
1508	-17.352845761422\\
1509	-17.0629095753316\\
1510	-15.7620688446323\\
1511	-17.249843630332\\
1512	-18.0905769638478\\
1513	-15.4072975497645\\
1514	-16.0167305872985\\
1515	-20.1452316543043\\
1516	-21.3907455419553\\
1517	-26.1961595030702\\
1518	-31.382988248648\\
1519	-40.7476752453574\\
1520	-30.4853395725868\\
1521	-26.5806415493114\\
1522	-27.9109872778502\\
1523	-25.1087340290642\\
1524	-19.8091340816077\\
1525	-21.1620264161593\\
1526	-31.2793522343329\\
1527	-34.6838133931392\\
1528	-21.6731875061168\\
1529	-17.8041830668683\\
1530	-17.0848837587887\\
1531	-19.4647441709235\\
1532	-16.9684614140117\\
1533	-13.5044572154686\\
1534	-15.2943143389195\\
1535	-18.6696917788825\\
1536	-16.3999044095929\\
1537	-15.3680840398706\\
1538	-14.6442687958079\\
1539	-14.2948983434023\\
1540	-13.9180889380668\\
1541	-13.7523194071337\\
1542	-15.1169630579827\\
1543	-21.9569520203913\\
1544	-24.0511570281965\\
1545	-21.8933822247623\\
1546	-23.756112215005\\
1547	-30.74489580513\\
1548	-31.1331941921778\\
1549	-35.7146009196322\\
1550	-33.5339183028943\\
1551	-26.5347101847303\\
1552	-21.1867832291418\\
1553	-21.2166134318495\\
1554	-29.4438661216529\\
1555	-34.3120438349258\\
1556	-24.3086263815699\\
1557	-19.7647906341269\\
1558	-16.6724929055154\\
1559	-18.5285535466896\\
1560	-15.6217746393477\\
1561	-15.9546301857304\\
1562	-12.5200530713303\\
1563	-15.2905631358772\\
1564	-16.1233579942746\\
1565	-15.3540955013314\\
1566	-14.3854794252684\\
1567	-14.4676199946234\\
1568	-12.4421054380998\\
1569	-12.3124337353524\\
1570	-12.6687820736695\\
1571	-12.5263201316789\\
1572	-12.5917015871826\\
1573	-11.2294816268468\\
1574	-24.3256708852184\\
1575	-32.2087744381784\\
1576	-30.3216787174474\\
1577	-24.5608213625419\\
1578	-29.6999860555386\\
1579	-43.501214956096\\
1580	-38.5500264618749\\
1581	-37.2428243544055\\
1582	-30.057270165539\\
1583	-29.6904736339791\\
1584	-30.8184742532167\\
1585	-21.7544533293767\\
1586	-21.914974525062\\
1587	-16.6454907916636\\
1588	-17.0050323095022\\
1589	-22.1083674598469\\
1590	-17.6078522710293\\
1591	-18.0569720845701\\
1592	-19.7937058440098\\
1593	-18.5540893628149\\
1594	-18.7612356031659\\
1596	-20.5468103806688\\
1597	-21.2871411668514\\
1598	-20.2444428164329\\
1599	-17.7333885795395\\
1600	-18.8224734482658\\
1601	-14.9485515659073\\
1602	-18.756062079415\\
1603	-14.4646114325399\\
1604	-13.6476186750733\\
1605	-13.0101200206518\\
1606	-18.3236486557389\\
1607	-11.6768613946092\\
1608	-11.2019174611532\\
1609	-14.5274509571873\\
1610	-16.7079544647418\\
1611	-15.2468244801789\\
1612	-18.6795607832871\\
1613	-19.0773992082516\\
1614	-15.9397496874358\\
1615	-17.7411696568952\\
1616	-14.0385815316586\\
1617	-15.2795090989227\\
1618	-14.5716355103177\\
1619	-15.3764754472911\\
1620	-11.720280264034\\
1621	-11.7307824145601\\
1622	-11.2229972219063\\
1623	-15.9080173307052\\
1624	-15.9571554070967\\
1625	-14.6291897565197\\
1626	-14.1483053779359\\
1627	-13.7066323974832\\
1628	-14.0228891182339\\
1629	-13.1690690013195\\
1630	-13.136216798327\\
1631	-12.4056924354513\\
1632	-12.5418016050148\\
1633	-11.6764088612924\\
1634	-11.5056513326117\\
1635	-12.4006383544297\\
1636	-13.2517254924273\\
1637	-12.8721480026802\\
1638	-12.5224649929476\\
1639	-12.2580466490372\\
1640	-11.8805846226487\\
1641	-15.8563442684549\\
1642	-25.5051771503006\\
1643	-18.4881627583727\\
1644	-17.6030558218572\\
1645	-15.815273824835\\
1646	-18.5151268473091\\
1647	-19.0431154256703\\
1648	-15.8500771480781\\
1649	-16.0266111685903\\
1650	-14.9265249943896\\
1651	-16.4107978569534\\
1652	-14.0084926783181\\
1653	-14.2755643167056\\
1654	-13.8532331343708\\
1655	-15.9841551148738\\
1656	-14.4292089033088\\
1657	-14.4446883199614\\
1658	-14.2163916441482\\
1659	-17.5722314958923\\
1660	-17.1300020353385\\
1661	-21.9549769823955\\
1662	-31.0707581170223\\
1663	-25.6661195595527\\
1664	-19.7227938293461\\
1665	-17.6151857849616\\
1666	-20.7680176842941\\
1667	-25.207465348004\\
1668	-20.0402591829659\\
1669	-22.7535798268266\\
1670	-16.8832912463511\\
1671	-17.0066147327859\\
1672	-15.7315052821716\\
1673	-18.3252877111597\\
1674	-20.4466408620585\\
1675	-18.1711168584495\\
1676	-22.7577090815737\\
1677	-21.6353080546871\\
1678	-20.0488952245487\\
1679	-17.1174400061097\\
1680	-18.2810905404124\\
1681	-21.5263373593186\\
1682	-19.3156336906159\\
1683	-18.9992991590307\\
1684	-18.0513804254897\\
1685	-16.2914779624577\\
1686	-16.4891348258318\\
1687	-15.0988660664038\\
1688	-15.0630842724465\\
1689	-18.9133559387344\\
1690	-21.3732503435531\\
1691	-21.8971464203394\\
1692	-20.3195168062109\\
1693	-17.5501294545281\\
1694	-16.2498747916377\\
1695	-15.9617279073238\\
1696	-16.9479820088457\\
1697	-19.3395334275874\\
1698	-22.3717097411463\\
1699	-25.5129085781452\\
1700	-20.9000466968052\\
1701	-17.2110924259296\\
1702	-20.6737123573305\\
1703	-30.7634497397535\\
1704	-21.733833905844\\
1705	-22.2370212991564\\
1706	-24.1343306430572\\
1707	-21.3978683611942\\
1708	-18.9050432582624\\
1709	-19.9162276996765\\
1710	-18.94988452907\\
1711	-20.234567396735\\
1712	-22.0668007870959\\
1713	-20.1577134649388\\
1714	-20.9797521839064\\
1715	-20.1569763585776\\
1716	-19.957990194496\\
1717	-18.1316574014731\\
1718	-19.6041923210018\\
1719	-17.4586942292665\\
1720	-17.4334428711302\\
1721	-18.6928955819208\\
1722	-18.5109092273897\\
1723	-14.9136117222699\\
1724	-16.841808318298\\
1725	-17.2308305519327\\
1726	-12.7955454695127\\
1727	-18.4878176380557\\
1728	-27.8727012053751\\
1729	-25.2354690438297\\
1730	-20.6074410719439\\
1731	-20.3856648428457\\
1732	-19.9851873532248\\
1733	-18.3751412469173\\
1734	-16.4624414403916\\
1735	-15.9878480673476\\
1736	-16.3353404978413\\
1737	-15.7107238188717\\
1738	-16.3083813120663\\
1739	-15.3575382958256\\
1740	-15.1159243718328\\
1741	-14.1854715228146\\
1742	-13.9009594194886\\
1743	-18.8281106400766\\
1744	-16.5028549973531\\
1745	-17.5485464223639\\
1746	-16.5697173456313\\
1747	-19.2940642971832\\
1748	-19.9569912297495\\
1749	-22.9224814322417\\
1750	-20.2391195654891\\
1751	-20.614147438909\\
1752	-20.9467400943722\\
1753	-15.9641557810021\\
1754	-17.8682376040642\\
1755	-16.337176908389\\
1756	-17.1869597757532\\
1757	-18.4053660477587\\
1758	-20.7089629954551\\
1759	-18.7348638875574\\
1760	-20.4404567038027\\
1761	-25.8816525802285\\
1762	-21.6018069983918\\
1763	-19.9752955569948\\
1764	-18.1902237598988\\
1765	-19.943007447024\\
1766	-18.1060226273637\\
1767	-17.4649647149545\\
1768	-16.0258835339669\\
1769	-17.0392807148132\\
1770	-15.8263172727097\\
1771	-22.0282519928498\\
1772	-30.6505574593148\\
1773	-26.2425001212282\\
1774	-24.1099187585814\\
1775	-24.4371610932267\\
1776	-17.9344349017483\\
1777	-17.7133556943377\\
1778	-16.2830513426941\\
1779	-15.712617504222\\
1780	-15.1857370753912\\
1781	-17.9177227766493\\
1782	-22.2108731104802\\
1783	-23.6116452040176\\
1784	-21.3729199167892\\
1785	-17.5937640679724\\
1786	-24.0702268029791\\
1787	-29.8040071952294\\
1788	-29.2635292932137\\
1789	-25.0263398118539\\
1790	-35.3839081745223\\
1791	-31.620491991953\\
1792	-19.2968828203173\\
1793	-18.2863323045653\\
1794	-17.193075264245\\
1795	-20.1402601193336\\
1796	-24.7819739555089\\
1797	-30.9886072665447\\
1798	-25.4479638273519\\
1799	-20.8591850199607\\
1800	-17.3522174701443\\
1801	-17.2934756158877\\
1802	-16.9353486224384\\
1803	-17.1114090686581\\
1804	-15.4406026673448\\
1805	-15.3521481183593\\
};
\addlegendentry{MPO prediction}

\end{axis}

\begin{axis}[%
width=6.159cm,
height=1.831cm,
at={(8.104cm,7.627cm)},
scale only axis,
xmin=1000,
xmax=2000,
xlabel style={font=\color{white!15!black}},
xlabel={Sample index},
ymin=-40.7725560622456,
ymax=1.221,
ylabel style={font=\color{white!15!black}},
ylabel={$y(t)$},
axis background/.style={fill=white},
title style={font=\bfseries},
title={C4: RMSE(OSA) = 2.9177, RMSE(MPO) = 5.9764},
legend style={legend cell align=left, align=left, draw=white!15!black}
]
\addplot [color=mycolor1, line width=2.0pt]
  table[row sep=crcr]{%
1006	-14.6479999999999\\
1007	-17.0899999999999\\
1008	-10.9860000000001\\
1009	-7.32400000000007\\
1010	-14.6479999999999\\
1011	-8.54500000000007\\
1012	-9.76600000000008\\
1013	-6.10400000000004\\
1014	-3.66200000000003\\
1016	-3.66200000000003\\
1017	-2.44100000000003\\
1018	-13.4280000000001\\
1020	-13.4280000000001\\
1021	-9.76600000000008\\
1022	-7.32400000000007\\
1023	-10.9860000000001\\
1024	-12.2070000000001\\
1025	-6.10400000000004\\
1027	-13.4280000000001\\
1028	-7.32400000000007\\
1029	-15.8689999999999\\
1031	-10.9860000000001\\
1032	-17.0899999999999\\
1033	-13.4280000000001\\
1035	-8.54500000000007\\
1036	-8.54500000000007\\
1037	-10.9860000000001\\
1038	-12.2070000000001\\
1040	-9.76600000000008\\
1041	-10.9860000000001\\
1042	-10.9860000000001\\
1043	-15.8689999999999\\
1044	-14.6479999999999\\
1045	-10.9860000000001\\
1046	-10.9860000000001\\
1047	-6.10400000000004\\
1048	-4.88300000000004\\
1049	-8.54500000000007\\
1050	-7.32400000000007\\
1051	-7.32400000000007\\
1052	-10.9860000000001\\
1053	-9.76600000000008\\
1054	-9.76600000000008\\
1055	-15.8689999999999\\
1056	-8.54500000000007\\
1057	-6.10400000000004\\
1060	-9.76600000000008\\
1061	-4.88300000000004\\
1062	-7.32400000000007\\
1064	-7.32400000000007\\
1065	-6.10400000000004\\
1069	-10.9860000000001\\
1070	-10.9860000000001\\
1072	-13.4280000000001\\
1073	-10.9860000000001\\
1074	-12.2070000000001\\
1075	-9.76600000000008\\
1077	-9.76600000000008\\
1078	-17.0899999999999\\
1079	-20.752\\
1080	-19.5309999999999\\
1081	-19.5309999999999\\
1082	-15.8689999999999\\
1083	-20.752\\
1084	-23.193\\
1085	-23.193\\
1086	-14.6479999999999\\
1087	-23.193\\
1088	-26.855\\
1089	-21.973\\
1090	-15.8689999999999\\
1091	-14.6479999999999\\
1092	-10.9860000000001\\
1093	-8.54500000000007\\
1094	-10.9860000000001\\
1096	-6.10400000000004\\
1097	-6.10400000000004\\
1098	-8.54500000000007\\
1100	-10.9860000000001\\
1101	-9.76600000000008\\
1102	-12.2070000000001\\
1103	-10.9860000000001\\
1104	-15.8689999999999\\
1105	-13.4280000000001\\
1106	-18.3109999999999\\
1107	-12.2070000000001\\
1108	-10.9860000000001\\
1109	-12.2070000000001\\
1110	-9.76600000000008\\
1111	-8.54500000000007\\
1112	-10.9860000000001\\
1113	-10.9860000000001\\
1114	-8.54500000000007\\
1115	-8.54500000000007\\
1116	-7.32400000000007\\
1118	-7.32400000000007\\
1119	-4.88300000000004\\
1120	-6.10400000000004\\
1121	-8.54500000000007\\
1122	-13.4280000000001\\
1123	-12.2070000000001\\
1124	-13.4280000000001\\
1125	-8.54500000000007\\
1126	-15.8689999999999\\
1127	-17.0899999999999\\
1128	-13.4280000000001\\
1129	-14.6479999999999\\
1130	-14.6479999999999\\
1132	-7.32400000000007\\
1133	-8.54500000000007\\
1134	-8.54500000000007\\
1135	-17.0899999999999\\
1136	-20.752\\
1137	-19.5309999999999\\
1138	-14.6479999999999\\
1139	-15.8689999999999\\
1140	-13.4280000000001\\
1141	-12.2070000000001\\
1142	-12.2070000000001\\
1143	-9.76600000000008\\
1144	-8.54500000000007\\
1145	-8.54500000000007\\
1146	-9.76600000000008\\
1147	-8.54500000000007\\
1148	-13.4280000000001\\
1149	-15.8689999999999\\
1150	-10.9860000000001\\
1151	-8.54500000000007\\
1152	-9.76600000000008\\
1153	-8.54500000000007\\
1155	-3.66200000000003\\
1156	-4.88300000000004\\
1157	-8.54500000000007\\
1158	-9.76600000000008\\
1159	-4.88300000000004\\
1160	-7.32400000000007\\
1161	-7.32400000000007\\
1162	-3.66200000000003\\
1163	-3.66200000000003\\
1164	-2.44100000000003\\
1165	-4.88300000000004\\
1166	-10.9860000000001\\
1167	-12.2070000000001\\
1168	-14.6479999999999\\
1169	-13.4280000000001\\
1170	-8.54500000000007\\
1171	-7.32400000000007\\
1172	-4.88300000000004\\
1173	-7.32400000000007\\
1174	-6.10400000000004\\
1175	-10.9860000000001\\
1176	-12.2070000000001\\
1177	-21.973\\
1178	-23.193\\
1179	-21.973\\
1180	-18.3109999999999\\
1181	-13.4280000000001\\
1182	-17.0899999999999\\
1183	-15.8689999999999\\
1184	-19.5309999999999\\
1185	-12.2070000000001\\
1186	-13.4280000000001\\
1187	-12.2070000000001\\
1188	-13.4280000000001\\
1189	-13.4280000000001\\
1190	-10.9860000000001\\
1191	-18.3109999999999\\
1192	-17.0899999999999\\
1193	-13.4280000000001\\
1194	-7.32400000000007\\
1195	-12.2070000000001\\
1196	-15.8689999999999\\
1197	-17.0899999999999\\
1199	-21.973\\
1200	-20.752\\
1201	-15.8689999999999\\
1202	-14.6479999999999\\
1203	-18.3109999999999\\
1204	-19.5309999999999\\
1205	-13.4280000000001\\
1206	-9.76600000000008\\
1207	-13.4280000000001\\
1208	-9.76600000000008\\
1209	-7.32400000000007\\
1210	-9.76600000000008\\
1211	-10.9860000000001\\
1212	-7.32400000000007\\
1213	-7.32400000000007\\
1214	-9.76600000000008\\
1215	-6.10400000000004\\
1216	-10.9860000000001\\
1217	-13.4280000000001\\
1218	-13.4280000000001\\
1219	-8.54500000000007\\
1220	-9.76600000000008\\
1221	-12.2070000000001\\
1222	-18.3109999999999\\
1223	-15.8689999999999\\
1224	-9.76600000000008\\
1225	-8.54500000000007\\
1226	-9.76600000000008\\
1227	-7.32400000000007\\
1228	-8.54500000000007\\
1229	-7.32400000000007\\
1230	-8.54500000000007\\
1231	-10.9860000000001\\
1232	-10.9860000000001\\
1233	-6.10400000000004\\
1234	-10.9860000000001\\
1235	-13.4280000000001\\
1236	-13.4280000000001\\
1237	-9.76600000000008\\
1238	-12.2070000000001\\
1239	-13.4280000000001\\
1240	-10.9860000000001\\
1241	-4.88300000000004\\
1242	-10.9860000000001\\
1243	-8.54500000000007\\
1244	-8.54500000000007\\
1245	-6.10400000000004\\
1246	-6.10400000000004\\
1247	-10.9860000000001\\
1248	-10.9860000000001\\
1249	-6.10400000000004\\
1250	-8.54500000000007\\
1251	-8.54500000000007\\
1252	-4.88300000000004\\
1253	-8.54500000000007\\
1254	-4.88300000000004\\
1255	-7.32400000000007\\
1256	-4.88300000000004\\
1257	-4.88300000000004\\
1258	-10.9860000000001\\
1259	-12.2070000000001\\
1260	-12.2070000000001\\
1261	-15.8689999999999\\
1263	-6.10400000000004\\
1264	-7.32400000000007\\
1265	-4.88300000000004\\
1266	-7.32400000000007\\
1267	-6.10400000000004\\
1268	-3.66200000000003\\
1269	-8.54500000000007\\
1270	-9.76600000000008\\
1271	-9.76600000000008\\
1272	-17.0899999999999\\
1273	-10.9860000000001\\
1274	-13.4280000000001\\
1275	-7.32400000000007\\
1276	-8.54500000000007\\
1277	-3.66200000000003\\
1278	-1.221\\
1279	-6.10400000000004\\
1280	-9.76600000000008\\
1281	-8.54500000000007\\
1282	-9.76600000000008\\
1283	-9.76600000000008\\
1284	-17.0899999999999\\
1285	-10.9860000000001\\
1287	-10.9860000000001\\
1289	-15.8689999999999\\
1290	-15.8689999999999\\
1292	-6.10400000000004\\
1293	-3.66200000000003\\
1295	-6.10400000000004\\
1296	-4.88300000000004\\
1297	-6.10400000000004\\
1298	-6.10400000000004\\
1299	-9.76600000000008\\
1300	-9.76600000000008\\
1301	-8.54500000000007\\
1302	-9.76600000000008\\
1303	-6.10400000000004\\
1304	-3.66200000000003\\
1306	-10.9860000000001\\
1307	-8.54500000000007\\
1308	-12.2070000000001\\
1309	-8.54500000000007\\
1310	-7.32400000000007\\
1311	-8.54500000000007\\
1312	-12.2070000000001\\
1313	-14.6479999999999\\
1314	-10.9860000000001\\
1315	-17.0899999999999\\
1316	-9.76600000000008\\
1317	-12.2070000000001\\
1318	-13.4280000000001\\
1319	-13.4280000000001\\
1320	-9.76600000000008\\
1321	-9.76600000000008\\
1322	-12.2070000000001\\
1323	-18.3109999999999\\
1324	-15.8689999999999\\
1325	-8.54500000000007\\
1326	-12.2070000000001\\
1327	-9.76600000000008\\
1328	-12.2070000000001\\
1329	-13.4280000000001\\
1330	-9.76600000000008\\
1331	-9.76600000000008\\
1332	-14.6479999999999\\
1334	-12.2070000000001\\
1335	-8.54500000000007\\
1336	-9.76600000000008\\
1337	-15.8689999999999\\
1338	-14.6479999999999\\
1339	-15.8689999999999\\
1340	-12.2070000000001\\
1343	-8.54500000000007\\
1344	-4.88300000000004\\
1345	-3.66200000000003\\
1346	-3.66200000000003\\
1348	-10.9860000000001\\
1350	-8.54500000000007\\
1351	-10.9860000000001\\
1353	-8.54500000000007\\
1354	-3.66200000000003\\
1355	-6.10400000000004\\
1357	-6.10400000000004\\
1359	-10.9860000000001\\
1360	-7.32400000000007\\
1361	-6.10400000000004\\
1362	-7.32400000000007\\
1363	-7.32400000000007\\
1364	-6.10400000000004\\
1365	-8.54500000000007\\
1366	-15.8689999999999\\
1367	-14.6479999999999\\
1368	-17.0899999999999\\
1370	-17.0899999999999\\
1371	-13.4280000000001\\
1372	-10.9860000000001\\
1373	-9.76600000000008\\
1375	-12.2070000000001\\
1376	-14.6479999999999\\
1377	-14.6479999999999\\
1380	-21.973\\
1381	-18.3109999999999\\
1382	-21.973\\
1383	-19.5309999999999\\
1384	-14.6479999999999\\
1385	-17.0899999999999\\
1386	-14.6479999999999\\
1387	-10.9860000000001\\
1388	-8.54500000000007\\
1389	-8.54500000000007\\
1390	-10.9860000000001\\
1391	-7.32400000000007\\
1394	-7.32400000000007\\
1395	-3.66200000000003\\
1396	-7.32400000000007\\
1397	-7.32400000000007\\
1398	-6.10400000000004\\
1399	-8.54500000000007\\
1400	-12.2070000000001\\
1401	-8.54500000000007\\
1402	-12.2070000000001\\
1403	-18.3109999999999\\
1404	-17.0899999999999\\
1405	-19.5309999999999\\
1407	-14.6479999999999\\
1408	-14.6479999999999\\
1409	-7.32400000000007\\
1410	-9.76600000000008\\
1411	-8.54500000000007\\
1412	-6.10400000000004\\
1414	-8.54500000000007\\
1415	-2.44100000000003\\
1416	-4.88300000000004\\
1417	-8.54500000000007\\
1418	-9.76600000000008\\
1419	-12.2070000000001\\
1420	-12.2070000000001\\
1421	-10.9860000000001\\
1422	-12.2070000000001\\
1423	-14.6479999999999\\
1424	-10.9860000000001\\
1425	-9.76600000000008\\
1426	-7.32400000000007\\
1427	-10.9860000000001\\
1428	-10.9860000000001\\
1429	-7.32400000000007\\
1430	-10.9860000000001\\
1431	-13.4280000000001\\
1433	-8.54500000000007\\
1435	-8.54500000000007\\
1436	-6.10400000000004\\
1437	-7.32400000000007\\
1438	-6.10400000000004\\
1439	-9.76600000000008\\
1440	-8.54500000000007\\
1441	-10.9860000000001\\
1442	-10.9860000000001\\
1444	-6.10400000000004\\
1445	-6.10400000000004\\
1446	-10.9860000000001\\
1447	-14.6479999999999\\
1448	-14.6479999999999\\
1449	-12.2070000000001\\
1452	-8.54500000000007\\
1453	-8.54500000000007\\
1454	-9.76600000000008\\
1455	-12.2070000000001\\
1456	-7.32400000000007\\
1457	-7.32400000000007\\
1458	-10.9860000000001\\
1459	-10.9860000000001\\
1460	-13.4280000000001\\
1461	-13.4280000000001\\
1462	-14.6479999999999\\
1463	-20.752\\
1464	-21.973\\
1465	-14.6479999999999\\
1466	-9.76600000000008\\
1467	-7.32400000000007\\
1468	-6.10400000000004\\
1469	-7.32400000000007\\
1470	-4.88300000000004\\
1471	-7.32400000000007\\
1473	-7.32400000000007\\
1474	-10.9860000000001\\
1476	-13.4280000000001\\
1477	-13.4280000000001\\
1478	-18.3109999999999\\
1479	-17.0899999999999\\
1480	-12.2070000000001\\
1481	-12.2070000000001\\
1482	-9.76600000000008\\
1483	-9.76600000000008\\
1484	-13.4280000000001\\
1485	-9.76600000000008\\
1486	-8.54500000000007\\
1487	-10.9860000000001\\
1488	-6.10400000000004\\
1489	-10.9860000000001\\
1490	-14.6479999999999\\
1491	-8.54500000000007\\
1492	-8.54500000000007\\
1493	-15.8689999999999\\
1495	-10.9860000000001\\
1496	-17.0899999999999\\
1497	-17.0899999999999\\
1498	-10.9860000000001\\
1499	-8.54500000000007\\
1500	-9.76600000000008\\
1501	-12.2070000000001\\
1502	-18.3109999999999\\
1503	-19.5309999999999\\
1504	-18.3109999999999\\
1505	-15.8689999999999\\
1506	-9.76600000000008\\
1507	-9.76600000000008\\
1508	-7.32400000000007\\
1509	-6.10400000000004\\
1510	-6.10400000000004\\
1511	-7.32400000000007\\
1512	-9.76600000000008\\
1513	-8.54500000000007\\
1514	-6.10400000000004\\
1515	-12.2070000000001\\
1516	-10.9860000000001\\
1517	-13.4280000000001\\
1518	-17.0899999999999\\
1519	-19.5309999999999\\
1521	-12.2070000000001\\
1522	-15.8689999999999\\
1523	-12.2070000000001\\
1524	-9.76600000000008\\
1525	-10.9860000000001\\
1526	-15.8689999999999\\
1527	-18.3109999999999\\
1528	-12.2070000000001\\
1530	-4.88300000000004\\
1532	0\\
1533	-6.10400000000004\\
1534	-8.54500000000007\\
1535	-8.54500000000007\\
1537	-6.10400000000004\\
1538	-6.10400000000004\\
1539	-7.32400000000007\\
1540	-4.88300000000004\\
1541	-6.10400000000004\\
1544	-13.4280000000001\\
1545	-10.9860000000001\\
1546	-12.2070000000001\\
1547	-17.0899999999999\\
1548	-17.0899999999999\\
1549	-19.5309999999999\\
1550	-20.752\\
1551	-13.4280000000001\\
1552	-10.9860000000001\\
1553	-10.9860000000001\\
1554	-14.6479999999999\\
1555	-17.0899999999999\\
1557	-9.76600000000008\\
1558	-7.32400000000007\\
1559	-3.66200000000003\\
1561	-6.10400000000004\\
1562	-1.221\\
1563	-9.76600000000008\\
1564	-7.32400000000007\\
1567	-3.66200000000003\\
1568	-6.10400000000004\\
1569	-6.10400000000004\\
1571	-3.66200000000003\\
1572	-3.66200000000003\\
1573	-4.88300000000004\\
1574	-12.2070000000001\\
1575	-13.4280000000001\\
1576	-15.8689999999999\\
1577	-14.6479999999999\\
1578	-14.6479999999999\\
1579	-23.193\\
1581	-18.3109999999999\\
1582	-18.3109999999999\\
1583	-15.8689999999999\\
1584	-15.8689999999999\\
1585	-14.6479999999999\\
1586	-10.9860000000001\\
1587	-8.54500000000007\\
1588	-8.54500000000007\\
1589	-12.2070000000001\\
1590	-14.6479999999999\\
1591	-8.54500000000007\\
1592	-9.76600000000008\\
1594	-9.76600000000008\\
1595	-10.9860000000001\\
1596	-10.9860000000001\\
1597	-12.2070000000001\\
1599	-9.76600000000008\\
1600	-9.76600000000008\\
1601	-10.9860000000001\\
1602	-2.44100000000003\\
1603	-4.88300000000004\\
1604	-8.54500000000007\\
1605	-7.32400000000007\\
1606	1.221\\
1607	-2.44100000000003\\
1608	-7.32400000000007\\
1609	-7.32400000000007\\
1610	-8.54500000000007\\
1612	-8.54500000000007\\
1613	-10.9860000000001\\
1614	-7.32400000000007\\
1615	-8.54500000000007\\
1616	-10.9860000000001\\
1617	-4.88300000000004\\
1618	-4.88300000000004\\
1619	-3.66200000000003\\
1620	-3.66200000000003\\
1621	-2.44100000000003\\
1622	-3.66200000000003\\
1623	-10.9860000000001\\
1624	-12.2070000000001\\
1625	-8.54500000000007\\
1626	-8.54500000000007\\
1628	-6.10400000000004\\
1629	-8.54500000000007\\
1630	-6.10400000000004\\
1632	-6.10400000000004\\
1633	-3.66200000000003\\
1634	-4.88300000000004\\
1635	-7.32400000000007\\
1636	-7.32400000000007\\
1637	-3.66200000000003\\
1639	-6.10400000000004\\
1640	-6.10400000000004\\
1642	-13.4280000000001\\
1643	-6.10400000000004\\
1644	-7.32400000000007\\
1645	-9.76600000000008\\
1646	-8.54500000000007\\
1647	-12.2070000000001\\
1648	-7.32400000000007\\
1649	-7.32400000000007\\
1650	-10.9860000000001\\
1651	-7.32400000000007\\
1652	-9.76600000000008\\
1653	-6.10400000000004\\
1655	-8.54500000000007\\
1656	-8.54500000000007\\
1657	-7.32400000000007\\
1658	-8.54500000000007\\
1659	-8.54500000000007\\
1660	-10.9860000000001\\
1661	-10.9860000000001\\
1662	-18.3109999999999\\
1663	-13.4280000000001\\
1664	-10.9860000000001\\
1665	-9.76600000000008\\
1666	-9.76600000000008\\
1667	-13.4280000000001\\
1668	-9.76600000000008\\
1669	-10.9860000000001\\
1670	-14.6479999999999\\
1671	-4.88300000000004\\
1672	-3.66200000000003\\
1673	-10.9860000000001\\
1674	-7.32400000000007\\
1675	-9.76600000000008\\
1677	-12.2070000000001\\
1680	-8.54500000000007\\
1681	-12.2070000000001\\
1682	-8.54500000000007\\
1683	-10.9860000000001\\
1684	-10.9860000000001\\
1685	-7.32400000000007\\
1686	-8.54500000000007\\
1687	-4.88300000000004\\
1688	-6.10400000000004\\
1689	-10.9860000000001\\
1690	-13.4280000000001\\
1691	-10.9860000000001\\
1692	-13.4280000000001\\
1693	-9.76600000000008\\
1694	-7.32400000000007\\
1696	-7.32400000000007\\
1698	-12.2070000000001\\
1699	-13.4280000000001\\
1700	-12.2070000000001\\
1701	-7.32400000000007\\
1702	-8.54500000000007\\
1703	-19.5309999999999\\
1704	-12.2070000000001\\
1705	-13.4280000000001\\
1706	-15.8689999999999\\
1707	-10.9860000000001\\
1708	-9.76600000000008\\
1710	-9.76600000000008\\
1711	-8.54500000000007\\
1712	-13.4280000000001\\
1713	-10.9860000000001\\
1714	-9.76600000000008\\
1715	-10.9860000000001\\
1716	-10.9860000000001\\
1717	-8.54500000000007\\
1718	-9.76600000000008\\
1719	-4.88300000000004\\
1720	-9.76600000000008\\
1721	-12.2070000000001\\
1723	-7.32400000000007\\
1724	-3.66200000000003\\
1725	-2.44100000000003\\
1726	-3.66200000000003\\
1727	-9.76600000000008\\
1728	-13.4280000000001\\
1729	-10.9860000000001\\
1730	-10.9860000000001\\
1731	-13.4280000000001\\
1732	-12.2070000000001\\
1733	-7.32400000000007\\
1734	-6.10400000000004\\
1735	-3.66200000000003\\
1736	-8.54500000000007\\
1737	-4.88300000000004\\
1738	-8.54500000000007\\
1739	-7.32400000000007\\
1740	-7.32400000000007\\
1742	-4.88300000000004\\
1743	-9.76600000000008\\
1744	-10.9860000000001\\
1745	-6.10400000000004\\
1746	-10.9860000000001\\
1747	-8.54500000000007\\
1748	-9.76600000000008\\
1749	-14.6479999999999\\
1750	-8.54500000000007\\
1751	-12.2070000000001\\
1752	-12.2070000000001\\
1753	-3.66200000000003\\
1754	-12.2070000000001\\
1755	-13.4280000000001\\
1756	-9.76600000000008\\
1757	-15.8689999999999\\
1758	-13.4280000000001\\
1759	-13.4280000000001\\
1760	-10.9860000000001\\
1761	-15.8689999999999\\
1762	-13.4280000000001\\
1764	-10.9860000000001\\
1765	-8.54500000000007\\
1766	-9.76600000000008\\
1767	-8.54500000000007\\
1768	-6.10400000000004\\
1769	-8.54500000000007\\
1770	-7.32400000000007\\
1771	-10.9860000000001\\
1772	-15.8689999999999\\
1773	-13.4280000000001\\
1774	-14.6479999999999\\
1775	-14.6479999999999\\
1777	-7.32400000000007\\
1778	-7.32400000000007\\
1779	-4.88300000000004\\
1780	-6.10400000000004\\
1781	-9.76600000000008\\
1782	-12.2070000000001\\
1784	-12.2070000000001\\
1785	-8.54500000000007\\
1786	-9.76600000000008\\
1787	-17.0899999999999\\
1788	-14.6479999999999\\
1789	-13.4280000000001\\
1790	-17.0899999999999\\
1791	-23.193\\
1792	-7.32400000000007\\
1793	-7.32400000000007\\
1794	-8.54500000000007\\
1795	-8.54500000000007\\
1796	-12.2070000000001\\
1797	-17.0899999999999\\
1798	-14.6479999999999\\
1799	-10.9860000000001\\
1800	-9.76600000000008\\
1801	-6.10400000000004\\
1802	-6.10400000000004\\
1803	-8.54500000000007\\
1805	-3.66200000000003\\
};
\addlegendentry{True output}

\addplot [color=mycolor2, dashed, line width=2.0pt]
  table[row sep=crcr]{%
1006	-16.3165927036027\\
1007	-16.5210027330381\\
1008	-14.9450419802497\\
1009	-9.48396082256772\\
1010	-10.0316139263655\\
1011	-12.8525164531904\\
1012	-12.4405732540645\\
1013	-10.5929132477559\\
1014	-9.19980593080527\\
1015	-11.1870082720559\\
1016	-8.78619373448146\\
1017	-2.78591565695069\\
1018	-15.9460718653306\\
1019	-15.1744387797601\\
1020	-13.5701809914849\\
1021	-11.5304567570495\\
1022	-10.5464995004595\\
1023	-11.7717596579823\\
1024	-10.4479539085953\\
1025	-10.8251619796467\\
1026	-10.8918524381406\\
1027	-12.2280971196485\\
1028	-11.7017118265503\\
1029	-13.1484473338849\\
1030	-15.3490010063927\\
1031	-12.1143434027515\\
1032	-14.8972590062635\\
1033	-16.930953634135\\
1034	-13.2566802722909\\
1035	-11.3276626579197\\
1036	-10.5097026352626\\
1037	-10.4043533609436\\
1038	-12.7369036170132\\
1039	-12.3711793166226\\
1040	-12.5376564158719\\
1041	-12.2870773814504\\
1042	-12.6841168096216\\
1043	-16.69642307711\\
1044	-15.7986997907799\\
1045	-11.5879169160839\\
1046	-11.5042484947794\\
1047	-10.8268609610543\\
1048	-9.2716017171249\\
1049	-8.18307474169001\\
1050	-8.86643390819722\\
1051	-8.41117836460739\\
1052	-9.42484855486737\\
1053	-12.2508420267936\\
1054	-10.9993280083754\\
1055	-15.2299958598865\\
1056	-14.0461405942744\\
1057	-9.27118360705231\\
1058	-9.00415803576379\\
1059	-8.79724949635215\\
1060	-9.73487842508598\\
1061	-9.08168216065451\\
1062	-8.2773521587128\\
1063	-7.77796314292527\\
1064	-8.39278600229159\\
1065	-8.38167902732926\\
1066	-7.41414289823842\\
1067	-8.55245937199811\\
1068	-12.9271691129563\\
1069	-12.1369061550645\\
1070	-13.6570019754008\\
1071	-12.678229962281\\
1072	-14.1148778193549\\
1073	-12.5633632582762\\
1074	-12.0126503529943\\
1075	-12.165694586625\\
1076	-11.06842619638\\
1077	-11.1586184764042\\
1078	-15.6609129679039\\
1079	-26.9696908117091\\
1080	-25.7579468552735\\
1081	-21.3361613128225\\
1082	-13.252929411276\\
1084	-28.1369150329085\\
1085	-24.8678686902783\\
1086	-20.4705665670097\\
1087	-22.2360078739748\\
1088	-35.7180036908064\\
1089	-19.6322510725183\\
1090	-18.7166045270542\\
1091	-11.7529468661075\\
1092	-11.8966200450957\\
1093	-11.4834163563487\\
1094	-12.7313813432224\\
1095	-10.3128639723182\\
1096	-8.97456493005552\\
1097	-8.33729872580761\\
1098	-8.12026606974655\\
1099	-11.3323109746491\\
1100	-11.2911886559486\\
1101	-11.1232356283072\\
1102	-13.5228324818729\\
1103	-13.8165060498927\\
1104	-18.2474971041554\\
1105	-15.0436857998682\\
1106	-17.1794594077198\\
1107	-14.2700453621785\\
1108	-12.8592455989219\\
1109	-14.009066281938\\
1110	-11.5794539464785\\
1111	-10.7941150461022\\
1112	-11.726289314561\\
1113	-12.4848403962762\\
1114	-9.85539074320832\\
1115	-10.1155796998387\\
1116	-9.20119684870383\\
1117	-9.04249944062303\\
1118	-9.08715381391812\\
1119	-8.39280890592522\\
1120	-7.55615618356137\\
1121	-8.47152715676339\\
1122	-13.1785498379429\\
1123	-15.30587787168\\
1124	-14.6372500567304\\
1125	-10.2870371752413\\
1126	-11.6860626901509\\
1127	-22.0245749735489\\
1128	-15.5760784081776\\
1129	-15.2678920925403\\
1130	-17.9824583840689\\
1131	-10.6764099121715\\
1132	-11.1609523287068\\
1133	-10.5023399787822\\
1134	-12.2676515238427\\
1135	-17.2496075963961\\
1136	-24.6484500529295\\
1137	-21.4936060909467\\
1138	-14.1939426969445\\
1139	-16.3797804011017\\
1140	-16.535668134115\\
1141	-14.8564598808587\\
1142	-14.0970132123105\\
1143	-10.574076330834\\
1144	-10.3142939211634\\
1145	-9.93761524435399\\
1146	-9.74028996705215\\
1147	-10.591101138361\\
1148	-12.6143273507432\\
1149	-20.3800836101607\\
1150	-15.0797762803072\\
1151	-10.365032218481\\
1152	-10.1512773987779\\
1153	-10.6886072776804\\
1154	-9.1232946765283\\
1155	-9.68376084740657\\
1156	-6.61968862391996\\
1157	-7.1254106972865\\
1159	-9.03713595576119\\
1160	-7.77532731712677\\
1161	-8.68427175583179\\
1162	-8.55074209394593\\
1163	-7.72728449269357\\
1164	-8.19156519040598\\
1165	-3.53320352325522\\
1166	-8.46542382891971\\
1167	-19.0280954417503\\
1168	-17.4405000455615\\
1169	-11.7550257895275\\
1170	-11.7278888287278\\
1171	-10.5885597758045\\
1172	-10.5481581348056\\
1173	-6.87365455588542\\
1174	-9.88581812193479\\
1175	-10.7409116405217\\
1176	-14.5927808404711\\
1177	-25.8463193619723\\
1178	-33.4594872403436\\
1179	-21.6086634511478\\
1180	-21.3419431174489\\
1181	-13.3301339672278\\
1182	-16.1781316601075\\
1183	-19.5440879965581\\
1184	-17.6535885371632\\
1185	-15.45029502487\\
1186	-13.6704302096496\\
1187	-15.0407429464012\\
1188	-12.8074800624224\\
1189	-14.9977430492916\\
1190	-12.671669913565\\
1191	-17.4644497236857\\
1192	-24.0964124217251\\
1193	-13.2914831171383\\
1194	-10.55830444971\\
1195	-10.7589490573175\\
1196	-18.5209031031989\\
1197	-21.8804849191567\\
1198	-25.402653218782\\
1199	-24.1798043619449\\
1200	-23.817859533327\\
1201	-16.7936103387526\\
1202	-16.1932211978128\\
1203	-19.123874245066\\
1204	-22.4546276284461\\
1205	-13.8493853283335\\
1206	-9.51215660148614\\
1207	-13.5365255175318\\
1208	-13.5103677897375\\
1209	-9.21161812332639\\
1210	-9.9543765015062\\
1211	-12.6227866054223\\
1212	-10.2088601664341\\
1213	-9.06097952823984\\
1214	-9.92502818236812\\
1215	-10.1477729379183\\
1216	-10.2603961274147\\
1217	-15.9404624714568\\
1218	-14.2027508279305\\
1219	-10.8180391077669\\
1220	-10.6618920188814\\
1221	-14.8914527196337\\
1222	-28.0270035700532\\
1223	-17.1474699600644\\
1224	-9.92369484181836\\
1225	-10.4534730239031\\
1226	-11.0303596873816\\
1227	-11.2389727628283\\
1228	-8.59263670631071\\
1229	-9.34758946414286\\
1230	-9.9723091008882\\
1231	-11.7361914731973\\
1232	-12.8145242538562\\
1233	-10.3512008058892\\
1234	-11.6777272476277\\
1235	-16.3492597105369\\
1236	-13.942800180437\\
1237	-12.1043340660028\\
1238	-12.8685413984429\\
1239	-16.6778274400706\\
1240	-11.1672103625845\\
1241	-10.1973978732626\\
1242	-8.22872626547746\\
1243	-10.4292454929341\\
1244	-11.9482979351494\\
1245	-10.4543041247332\\
1246	-8.42592865329493\\
1247	-9.44415150216946\\
1248	-11.3725793650613\\
1249	-9.88133973886033\\
1250	-9.38553578635219\\
1251	-9.8968080599434\\
1252	-9.64686167323748\\
1253	-8.5691768567965\\
1254	-8.41896274725741\\
1255	-7.58274295847923\\
1256	-7.85674905699625\\
1257	-7.36112379458268\\
1258	-8.07093135327932\\
1259	-12.2313324768686\\
1260	-15.6886565420216\\
1261	-20.6623235116267\\
1262	-12.6757097918053\\
1263	-10.3721864233385\\
1264	-9.31271405359485\\
1265	-8.98644981765983\\
1266	-8.48676395502321\\
1267	-7.66533743228956\\
1268	-7.88843267202742\\
1269	-7.06746809062361\\
1270	-14.3796629702153\\
1271	-12.5552832663295\\
1272	-14.9999101810424\\
1273	-12.2280761852976\\
1274	-11.4232495969227\\
1275	-11.2776214833609\\
1276	-10.5515958543208\\
1277	-10.0591569239452\\
1278	-7.7182474770907\\
1279	-4.73091915581722\\
1280	-7.5593497223424\\
1281	-9.93041553677722\\
1282	-10.1917683634285\\
1283	-11.2469351829927\\
1284	-17.4843171015768\\
1285	-17.066094436658\\
1286	-10.975404549808\\
1287	-13.408129635221\\
1288	-14.0312356613213\\
1289	-16.9942159668587\\
1290	-14.4986485078014\\
1291	-12.3056304754341\\
1292	-12.127381745579\\
1293	-11.8899617876871\\
1294	-7.54504569504843\\
1295	-6.28849349887355\\
1296	-6.59809478949319\\
1297	-6.49936769279839\\
1298	-6.77542621897737\\
1299	-9.17161574435954\\
1300	-10.1071953812097\\
1301	-10.0989457238745\\
1302	-10.5014049178053\\
1303	-9.49255708038845\\
1304	-10.8045124737966\\
1305	-4.80446412274136\\
1306	-11.3563375549686\\
1307	-11.5350142588295\\
1308	-11.0116742035423\\
1309	-11.8519310196491\\
1310	-9.60561692780448\\
1311	-9.64017655377279\\
1312	-15.3392666061698\\
1313	-13.7712751335634\\
1314	-11.7115084452205\\
1315	-15.5429238408633\\
1316	-15.824562736703\\
1317	-12.3828628286187\\
1318	-14.4675961145633\\
1319	-14.8332448449457\\
1320	-12.0744809918344\\
1321	-11.1563681003038\\
1322	-14.8291728319705\\
1323	-22.5737308165722\\
1324	-15.841855116515\\
1325	-10.8695074218861\\
1326	-10.8031290687459\\
1327	-12.4791614005965\\
1328	-13.8642170483679\\
1329	-15.0378809529288\\
1330	-12.0066882013102\\
1331	-12.0862633395282\\
1332	-14.3373823140878\\
1333	-16.6971520856162\\
1334	-10.5256708744341\\
1335	-11.486636963818\\
1336	-13.1849542789219\\
1337	-19.598595599417\\
1338	-17.1293928630155\\
1339	-16.1286699666155\\
1340	-10.5627318634138\\
1341	-12.0360292595469\\
1342	-12.2736932698663\\
1343	-9.84450926888371\\
1344	-10.9086990064989\\
1345	-9.85294430925364\\
1346	-7.88438832642669\\
1347	-3.98079313183302\\
1348	-10.6087045292645\\
1349	-12.9791906412386\\
1350	-9.70997410831774\\
1351	-9.93051452071336\\
1352	-11.6812260069485\\
1353	-9.55169565140068\\
1354	-9.45023563885843\\
1355	-7.54774291099943\\
1356	-7.89390739669261\\
1357	-7.1610705424248\\
1358	-9.34983431964747\\
1359	-13.6413116755466\\
1360	-9.1979561409089\\
1361	-9.07937230717494\\
1362	-8.35420142122848\\
1363	-8.14782367725525\\
1364	-8.31398312747797\\
1365	-8.40890380856513\\
1366	-20.0877905562122\\
1367	-15.0057421915456\\
1368	-17.4803301625336\\
1369	-21.9802480044439\\
1370	-20.5858504764817\\
1371	-14.5165840597742\\
1372	-12.084811963982\\
1373	-12.288025235305\\
1374	-12.4016721546575\\
1375	-13.846099292582\\
1376	-15.3538886383601\\
1377	-15.9047435598916\\
1378	-21.4717804963416\\
1379	-23.9158383487079\\
1380	-27.7691248920364\\
1381	-16.7084191671959\\
1382	-18.3998355669612\\
1383	-24.2330431738028\\
1384	-15.8508295210124\\
1385	-19.3369128180229\\
1386	-17.3220429964092\\
1387	-9.89381354016996\\
1388	-10.2593934980312\\
1389	-10.0893230172135\\
1390	-11.8162837053005\\
1391	-11.0695556472508\\
1392	-8.72072699054252\\
1393	-9.09868921169641\\
1394	-8.65751349782545\\
1395	-8.60280375584239\\
1396	-6.70824518337736\\
1397	-8.00381867279589\\
1398	-8.1428890772404\\
1399	-8.4122227718558\\
1400	-12.829905962612\\
1401	-10.7866091046312\\
1402	-14.5612882178816\\
1403	-23.8397115242963\\
1404	-19.9341416759162\\
1405	-23.0837915242619\\
1406	-15.2082522713299\\
1407	-17.5632428689755\\
1408	-12.1523138434877\\
1409	-11.8338798967484\\
1410	-10.8742686906637\\
1411	-10.5773501239714\\
1412	-9.51013280225266\\
1413	-8.78082988383653\\
1414	-7.92167304291843\\
1415	-11.1985619971133\\
1416	-5.69297766527734\\
1417	-7.32020235572622\\
1418	-12.9294228713188\\
1419	-13.9142509709009\\
1420	-13.2546190230573\\
1421	-12.8947382245951\\
1422	-14.2161195564888\\
1423	-12.6458648025186\\
1424	-12.234677060055\\
1425	-11.9725247928961\\
1426	-10.9770027542884\\
1427	-12.6842682719353\\
1428	-9.85420212444023\\
1429	-10.4600983566404\\
1430	-10.2987989197293\\
1431	-16.3934414789308\\
1432	-12.2819894797742\\
1433	-10.5797187957389\\
1434	-9.95676951609562\\
1435	-10.3295529966165\\
1436	-9.14054185048531\\
1437	-8.404154577084\\
1438	-8.57888824573365\\
1439	-9.90240808174667\\
1440	-11.4090576338353\\
1441	-9.2702505869745\\
1442	-11.206035766892\\
1443	-11.7628618430701\\
1444	-9.19298465708926\\
1445	-8.51820519110402\\
1446	-10.7488112721637\\
1447	-18.681380594662\\
1448	-15.9257920639802\\
1449	-13.3321947244199\\
1450	-13.0116363934555\\
1451	-11.7627611343705\\
1452	-10.9058826791359\\
1453	-9.89047171114817\\
1454	-11.1296997341717\\
1455	-11.3620105156303\\
1456	-10.0329057525307\\
1457	-9.89772535498355\\
1458	-11.919819257356\\
1459	-13.024203516346\\
1460	-13.1238746769602\\
1461	-14.1620148930665\\
1462	-19.0471885613435\\
1463	-23.6874450971379\\
1464	-27.0706611098112\\
1465	-15.8033331638424\\
1466	-9.82112797708191\\
1467	-11.4991935278333\\
1468	-11.370236597653\\
1469	-7.67776389080177\\
1470	-8.39350356786349\\
1471	-9.91817355852845\\
1472	-9.9007448670277\\
1473	-8.69509050687316\\
1474	-9.93537877271478\\
1475	-12.8254603969592\\
1476	-14.5731721071988\\
1477	-15.7223861292853\\
1478	-23.392057451113\\
1479	-19.8876575433389\\
1480	-13.7526353712085\\
1481	-11.0702571458908\\
1482	-12.3253592287954\\
1483	-12.5948010310096\\
1484	-13.855210793457\\
1485	-11.2349605286322\\
1486	-10.8693728833011\\
1487	-10.8129765242934\\
1488	-9.70052321085086\\
1489	-8.50586112693554\\
1490	-15.1727628079061\\
1491	-12.0293319257182\\
1492	-11.1924865240494\\
1493	-19.1524444736601\\
1494	-15.5773213012667\\
1495	-13.0411833502601\\
1496	-17.0147802678321\\
1497	-18.0841586499796\\
1498	-12.8802947534091\\
1499	-10.2592407393122\\
1500	-10.4942316300612\\
1501	-14.1541256198975\\
1502	-23.8211191287496\\
1503	-23.5125033869738\\
1504	-20.0092520871626\\
1505	-15.6874704294341\\
1506	-12.0622375600133\\
1507	-11.4258760171701\\
1508	-10.2406024230538\\
1509	-9.53667592000534\\
1510	-8.17186483265891\\
1511	-8.51524494924274\\
1512	-9.76769597308999\\
1513	-8.64722286848678\\
1514	-9.18850262434603\\
1515	-11.3541136146632\\
1516	-14.5583195524669\\
1517	-15.7158899222241\\
1519	-25.3975147238668\\
1520	-16.2931885327594\\
1521	-14.1502803475023\\
1522	-15.2614462815618\\
1523	-14.7900407967638\\
1524	-10.7307632644838\\
1525	-11.9241001111957\\
1526	-19.3331351419763\\
1527	-22.4051847831404\\
1528	-11.5065275633081\\
1529	-10.3551668190164\\
1530	-11.3657924486538\\
1531	-11.8106590596694\\
1532	-7.71605625311622\\
1533	-2.65314005092887\\
1534	-6.63217109126026\\
1535	-8.91564119274722\\
1536	-8.79980742452608\\
1537	-8.73960437909318\\
1538	-7.87169732130747\\
1539	-7.61582247827732\\
1540	-7.94406999526495\\
1541	-7.00201516816151\\
1542	-8.13348987031668\\
1543	-13.906264355235\\
1544	-14.2782904086462\\
1545	-13.117650701696\\
1546	-15.0992987098427\\
1547	-19.3629378975775\\
1548	-18.795405431883\\
1549	-22.3827033246394\\
1550	-20.286724032756\\
1551	-15.7424394243417\\
1552	-12.2013009158286\\
1553	-12.4141928275321\\
1554	-20.1868831026406\\
1555	-20.8137767532407\\
1556	-12.2045943278599\\
1557	-10.9061067158545\\
1558	-10.8764518730695\\
1559	-12.5091840253201\\
1560	-7.40875596682918\\
1561	-8.9992175899622\\
1562	-4.13398509869171\\
1563	-6.2424203934363\\
1564	-9.04939558006959\\
1565	-8.32511371752048\\
1566	-8.29750341934687\\
1567	-8.5787263841064\\
1568	-5.11050228280078\\
1569	-6.22486148141843\\
1570	-6.76467259856054\\
1571	-6.75679376020412\\
1572	-6.58607912386537\\
1573	-4.51748541911184\\
1574	-16.792491707605\\
1575	-21.5814987662525\\
1576	-18.1713275762552\\
1577	-12.8412177558052\\
1578	-17.6232066907719\\
1579	-29.4344523658256\\
1580	-24.9896841392979\\
1581	-20.6794582031168\\
1582	-15.0338555599105\\
1583	-17.9008535764403\\
1584	-18.7053668449651\\
1585	-12.3410006956667\\
1586	-13.2629552900244\\
1587	-10.3249093403865\\
1588	-10.07339613115\\
1589	-13.1006067210901\\
1590	-10.8348184370391\\
1591	-12.7264447461657\\
1593	-11.6380908518536\\
1594	-11.6711875106496\\
1595	-11.15268450641\\
1596	-12.8336295841148\\
1597	-13.5707821708111\\
1598	-11.896330611874\\
1599	-10.7821446249116\\
1600	-11.6510825254456\\
1601	-9.21869607136045\\
1602	-15.3041605298997\\
1603	-6.80922708625189\\
1604	-5.61623378760373\\
1605	-8.53528474923223\\
1606	-14.0124866424296\\
1607	-1.39857938075579\\
1608	-3.26262531069119\\
1609	-7.38649445712394\\
1610	-8.34539752687238\\
1611	-8.83861227582634\\
1612	-11.667716072194\\
1613	-10.9985535219987\\
1614	-9.98844718624559\\
1615	-9.80492673939739\\
1616	-8.48382672816865\\
1617	-11.4752779646399\\
1618	-9.66689319103011\\
1619	-9.76074839200714\\
1620	-3.43280974703725\\
1621	-5.30058043064969\\
1622	-3.98929663724084\\
1623	-8.03612248834111\\
1624	-10.0070024606352\\
1625	-10.0357318054462\\
1626	-9.29198685775464\\
1627	-9.54075888934858\\
1628	-8.89282901294177\\
1629	-8.05713698363797\\
1630	-8.78464963010015\\
1631	-7.67798678422082\\
1632	-8.06072839929152\\
1633	-6.48747766979272\\
1634	-6.2345728758437\\
1635	-5.96454857488948\\
1636	-7.95528194121425\\
1637	-7.95270728935816\\
1638	-6.54366774000732\\
1639	-7.47761745933099\\
1640	-6.55949488232318\\
1641	-8.60863283478716\\
1642	-17.3385463559912\\
1643	-11.9992693346358\\
1644	-9.22123820427987\\
1645	-8.71786743411167\\
1646	-10.9793318156633\\
1647	-10.8239331319621\\
1648	-10.0717988570502\\
1649	-9.36161764101712\\
1650	-8.69074817528372\\
1651	-10.955368925353\\
1652	-8.451460777085\\
1653	-9.79933893938596\\
1654	-7.99304452882143\\
1655	-9.46296787096298\\
1656	-9.03652210637961\\
1657	-8.94800378458808\\
1658	-8.58856764677694\\
1659	-11.320646058716\\
1660	-10.18117384121\\
1661	-14.1077649756935\\
1662	-22.2632652415812\\
1663	-16.2416506959216\\
1664	-10.5104054427188\\
1665	-11.1238424125982\\
1666	-12.7510998439195\\
1667	-15.7906578493585\\
1668	-12.4450401148711\\
1669	-12.8582871537612\\
1670	-9.3280497891385\\
1671	-13.2725422885753\\
1672	-9.28945530000851\\
1673	-8.05630550184128\\
1674	-12.5428872740329\\
1675	-8.82402382412329\\
1676	-12.044982289955\\
1677	-13.5264233233286\\
1678	-11.2614484333451\\
1679	-10.6648068367463\\
1680	-10.8835095248089\\
1681	-13.1953630576531\\
1682	-11.4598406277971\\
1683	-10.8790568268316\\
1684	-11.1528194345838\\
1685	-10.6005669028382\\
1686	-9.59292572221307\\
1687	-9.41484634385301\\
1688	-8.17702361658212\\
1689	-10.3824407395336\\
1690	-13.9069674649184\\
1691	-13.5565569459411\\
1692	-12.1394102495315\\
1693	-11.4847190099322\\
1694	-10.7133322275622\\
1695	-9.35980865458714\\
1696	-10.0284016650908\\
1697	-10.8478298046357\\
1698	-13.3182047342041\\
1699	-15.9213500822957\\
1700	-11.9934809751858\\
1701	-10.5850524668786\\
1702	-11.9205839429126\\
1703	-18.6052360427489\\
1704	-13.6370367005659\\
1705	-12.8561826390614\\
1706	-15.7874468047896\\
1707	-13.7991690151205\\
1708	-11.5219769060316\\
1709	-12.3894192323019\\
1710	-11.3745045535463\\
1711	-12.0139417992248\\
1712	-13.1982505759179\\
1713	-11.5938716918738\\
1714	-12.4822361220006\\
1715	-12.3002541690371\\
1716	-12.2178381061233\\
1717	-10.9632101070347\\
1718	-11.802398309864\\
1719	-10.2358810958146\\
1720	-8.80097589391244\\
1721	-10.7401121957014\\
1722	-11.740085957726\\
1723	-9.07679197193374\\
1724	-12.0532113137899\\
1725	-10.6746919352383\\
1726	-3.73010160207718\\
1727	-8.61238070492982\\
1728	-17.7571966263426\\
1729	-13.4770398914804\\
1730	-9.94603770788353\\
1731	-11.2670718190461\\
1732	-12.3603862496022\\
1733	-11.7705058764675\\
1734	-9.83894180070138\\
1735	-8.41099083163726\\
1736	-7.50845205120095\\
1737	-8.61730128679551\\
1738	-7.82632121516349\\
1739	-8.64829983675031\\
1740	-8.59950822256064\\
1741	-8.59645345436752\\
1742	-7.29282591993228\\
1743	-10.5650758029533\\
1744	-9.192325050916\\
1745	-10.735586609904\\
1746	-9.04349856283216\\
1747	-12.5108276932258\\
1748	-11.1389553707345\\
1749	-14.6849902978402\\
1750	-13.5106088307741\\
1751	-11.4451681798296\\
1752	-13.060138951003\\
1753	-10.1804488467676\\
1754	-9.96450263693669\\
1755	-10.5113741234884\\
1756	-12.3817561432979\\
1757	-11.7248763220634\\
1758	-16.1919358256721\\
1759	-13.1664516537803\\
1760	-14.3912713235165\\
1761	-17.3426655307508\\
1762	-15.1391048227379\\
1763	-12.7218203397172\\
1764	-11.8172593582244\\
1765	-13.1133921239616\\
1766	-10.663658048524\\
1767	-10.3683926957922\\
1769	-8.83161825833554\\
1770	-9.29719273648061\\
1771	-13.5607707942852\\
1772	-20.6143635060264\\
1773	-14.8546541994021\\
1774	-14.0850281338501\\
1775	-14.5746160775416\\
1776	-11.2248893512217\\
1777	-11.3969584289921\\
1778	-9.71808280646769\\
1779	-9.22340941789821\\
1780	-7.76141874107634\\
1781	-9.66800139416932\\
1782	-13.2763253434891\\
1783	-14.3277624945283\\
1784	-12.5090782427778\\
1785	-10.8353498608553\\
1786	-13.5489692683179\\
1787	-17.6872082518494\\
1788	-17.814358055299\\
1789	-14.3439396542103\\
1790	-22.3509865895144\\
1791	-17.5060584988173\\
1792	-12.379968477554\\
1793	-10.7282560605631\\
1794	-9.56428600328536\\
1795	-12.1227170600928\\
1796	-15.4668555074416\\
1797	-16.7533371424686\\
1799	-12.2423090397492\\
1800	-10.7469452999671\\
1801	-10.9687513463582\\
1802	-9.29487927119112\\
1803	-8.89187665135023\\
1804	-8.8692360600171\\
1805	-7.85595846504134\\
};
\addlegendentry{OSA predition}

\addplot [color=mycolor3, dotted, line width=2.0pt]
  table[row sep=crcr]{%
1006	-14.6479999999999\\
1007	-17.0899999999999\\
1008	-10.9860000000001\\
1009	-7.32400000000007\\
1010	-10.0316139263653\\
1011	-11.1567920087077\\
1012	-12.8257251027273\\
1013	-11.6262460595835\\
1014	-11.2726974986422\\
1015	-14.9770851856208\\
1016	-14.2681202290998\\
1017	-9.26725110859752\\
1018	-22.0825408336441\\
1019	-21.8422744034292\\
1020	-19.5912841920056\\
1021	-16.277594950576\\
1022	-15.0167584114299\\
1023	-16.8578704661311\\
1024	-14.6898801522968\\
1025	-13.643897033885\\
1026	-15.3143072719936\\
1027	-16.0096550122769\\
1028	-14.2821959160367\\
1029	-17.386831547901\\
1030	-17.5875328556413\\
1031	-14.7223723335035\\
1032	-17.6339874799735\\
1033	-18.1886025896249\\
1034	-15.8395067346023\\
1035	-14.1211605443991\\
1036	-13.5540450142603\\
1037	-13.7887294540651\\
1038	-15.409950350476\\
1039	-14.9243836199748\\
1040	-15.1461320730191\\
1041	-15.3376746350984\\
1042	-15.6539729788285\\
1043	-20.0005804414923\\
1044	-18.9212713672887\\
1045	-14.4983361845987\\
1046	-14.0925072750213\\
1047	-13.0028643263529\\
1048	-12.796840989557\\
1049	-12.5294878670531\\
1050	-12.1115927803496\\
1051	-11.9446706126446\\
1052	-12.8449355035946\\
1053	-14.3756168411749\\
1054	-13.8501982784333\\
1055	-17.9678670096373\\
1056	-15.9841590861354\\
1057	-12.990197231024\\
1058	-12.9749508401894\\
1059	-12.415001786055\\
1060	-13.16223616201\\
1061	-11.8306462981445\\
1062	-12.0093328789064\\
1063	-10.9823478350665\\
1064	-11.1496213847167\\
1065	-11.2174574220724\\
1066	-10.4806227397726\\
1067	-10.9979809635338\\
1068	-15.2143760685876\\
1069	-15.2892735095998\\
1070	-16.4874995517537\\
1071	-16.0419612568508\\
1072	-17.2046760111029\\
1073	-15.3550292819691\\
1074	-14.9649006443176\\
1075	-14.4132314444525\\
1076	-13.8329455719761\\
1077	-13.8898930878113\\
1078	-18.5010940663115\\
1079	-29.0883688850383\\
1080	-30.0543618060642\\
1081	-26.8638948029816\\
1082	-18.2159395777471\\
1083	-24.5570792949791\\
1084	-31.8276294485001\\
1085	-29.5788195387011\\
1086	-24.6619705664129\\
1087	-28.2146871814334\\
1088	-40.7725560622457\\
1089	-27.2871975760129\\
1090	-24.1241176543824\\
1091	-16.9169861962444\\
1092	-15.5063544409893\\
1093	-14.3946912374745\\
1094	-16.4746687572394\\
1095	-13.6534829094603\\
1096	-12.3051397002248\\
1097	-12.1102342309177\\
1098	-11.9781729759263\\
1099	-14.4653333816425\\
1100	-14.6678011937665\\
1101	-13.9631226678132\\
1102	-16.374249946376\\
1103	-16.7155701007362\\
1104	-21.7924399993185\\
1105	-18.8180106319414\\
1106	-21.0227552657873\\
1107	-17.0671339775438\\
1108	-16.0239300534404\\
1109	-17.2677629464285\\
1110	-14.7298710815273\\
1111	-14.1307925960696\\
1112	-15.3582487417718\\
1113	-15.7629503697185\\
1114	-13.0638709636262\\
1115	-13.265088448017\\
1116	-12.2152661497817\\
1117	-12.2059136211233\\
1118	-12.2984375871335\\
1119	-11.60302947276\\
1120	-11.4873765329341\\
1121	-12.1701983026192\\
1122	-16.4657767832955\\
1123	-18.1895798382054\\
1124	-18.1744061980319\\
1125	-13.3557984577533\\
1126	-14.8898760006218\\
1127	-23.4646436554328\\
1128	-18.8195894916942\\
1129	-18.3941067573833\\
1130	-20.6706195066524\\
1131	-14.3776678448996\\
1132	-13.9129723843969\\
1133	-14.1622277809922\\
1134	-16.0644174681311\\
1135	-21.929522692041\\
1136	-28.8704177550842\\
1137	-26.7290209296748\\
1138	-19.1180887245689\\
1139	-20.2308653544837\\
1140	-20.1748350419984\\
1141	-19.0311992610691\\
1142	-18.3430653854557\\
1143	-14.6276280190978\\
1144	-13.9926132904925\\
1145	-13.6320914084181\\
1146	-13.2368394821367\\
1147	-13.4306718440678\\
1148	-15.9087912258726\\
1149	-22.8689280637466\\
1150	-18.8868073540407\\
1151	-14.83351017582\\
1152	-14.1920479054272\\
1153	-14.3491533568329\\
1154	-12.8957687269069\\
1155	-13.6721656932493\\
1156	-11.8417081896096\\
1157	-12.0571146200825\\
1158	-11.6766886756536\\
1159	-11.7111977447148\\
1160	-11.4771261410447\\
1161	-11.5073474477113\\
1162	-11.2484863008804\\
1163	-11.8747528451552\\
1164	-12.6571629984144\\
1165	-9.13256247211166\\
1166	-12.9051006798904\\
1167	-22.6397569501239\\
1168	-23.1552338755689\\
1169	-16.7376284355646\\
1170	-14.9057427298235\\
1171	-14.7220501886056\\
1172	-14.9933220997691\\
1173	-12.3153876375677\\
1174	-14.2440767298031\\
1175	-16.2467021433902\\
1176	-19.1070021788955\\
1177	-30.8793656370606\\
1178	-39.3321619068913\\
1179	-30.2278119253776\\
1180	-28.325021173433\\
1181	-20.2409407705072\\
1182	-22.3524061102062\\
1183	-24.2170622422507\\
1184	-23.1402630659022\\
1185	-19.0503073960942\\
1186	-17.8975741882814\\
1187	-18.7368666904763\\
1188	-16.730713697593\\
1189	-18.1780971161841\\
1190	-15.8456862027392\\
1191	-20.8919462968843\\
1192	-26.6424432930576\\
1193	-18.1180620577818\\
1194	-14.1831648475493\\
1195	-14.7765018789498\\
1196	-21.8208752074547\\
1197	-25.7922070944978\\
1198	-30.4242831556721\\
1199	-30.4447953262559\\
1200	-30.1673070122897\\
1201	-23.3996779612205\\
1202	-22.1227188087544\\
1203	-24.7176684648457\\
1204	-27.659351480529\\
1205	-19.1530967548561\\
1206	-13.7997396335859\\
1207	-17.074433827378\\
1208	-16.6331552853901\\
1209	-13.0341282945717\\
1210	-13.6206285583003\\
1211	-15.7635896613945\\
1212	-13.5992077906008\\
1213	-12.813586050489\\
1214	-13.5112430418426\\
1215	-13.1849741478668\\
1216	-14.4843461848175\\
1217	-19.094571634301\\
1218	-17.9089095830175\\
1219	-14.1611854361965\\
1220	-14.0746269140268\\
1221	-18.2179624286457\\
1222	-32.1210429539374\\
1223	-24.1169154023535\\
1224	-15.6640673525524\\
1225	-15.0771533848067\\
1226	-16.0919956806511\\
1227	-15.6728173568147\\
1228	-13.4890467163118\\
1229	-13.3154759143356\\
1230	-14.1265759336302\\
1231	-15.847509197301\\
1232	-16.4210091270425\\
1233	-14.0230927705722\\
1234	-16.3224039043862\\
1235	-20.4285124689825\\
1236	-18.6246420475177\\
1237	-16.1826178518668\\
1238	-17.0829015181341\\
1239	-20.5592407696822\\
1240	-15.484170468387\\
1241	-13.5733418245268\\
1242	-12.8193012276299\\
1243	-13.2052053302903\\
1244	-15.2347527295335\\
1245	-14.4909184188214\\
1246	-12.853090522914\\
1247	-14.0595762700698\\
1248	-14.7790789483179\\
1249	-13.0233656356247\\
1250	-13.3179274905478\\
1251	-13.1117805452914\\
1252	-12.8501549895354\\
1253	-13.1486041509997\\
1254	-11.84208824149\\
1255	-11.7102307652392\\
1256	-11.3525694263244\\
1257	-11.1525800206678\\
1258	-12.2704477258469\\
1259	-14.5883598803082\\
1260	-18.2079388615066\\
1261	-23.948180719734\\
1262	-16.7628560234998\\
1263	-14.1292939736907\\
1264	-13.9331895390296\\
1265	-13.5038100056677\\
1266	-13.7292410061141\\
1267	-12.3078838256108\\
1268	-12.2660157171654\\
1269	-12.3427999778746\\
1270	-18.2294679244965\\
1271	-17.7790375482714\\
1272	-20.3059839485013\\
1273	-15.5702850460859\\
1274	-14.9902771819727\\
1275	-13.3465622618421\\
1276	-13.4780766189956\\
1277	-13.1189964289542\\
1278	-12.2293818981334\\
1279	-10.7861230425926\\
1280	-11.9990654702819\\
1281	-13.4063764705963\\
1282	-13.8268590276004\\
1283	-13.9947828954043\\
1284	-20.4146409967179\\
1285	-19.7824353879087\\
1286	-15.4241455740146\\
1287	-16.8958118445628\\
1288	-17.952452077398\\
1289	-20.788051812259\\
1290	-17.9276357892918\\
1291	-14.6140759727191\\
1292	-14.28874586394\\
1293	-15.8749643315443\\
1294	-13.3747788463891\\
1295	-11.6890438375806\\
1296	-11.3442781979481\\
1297	-11.3253624410622\\
1298	-10.738775162645\\
1299	-12.7263850975\\
1300	-12.8993863632538\\
1301	-12.5538370270292\\
1302	-13.0843458461113\\
1303	-11.6855261999967\\
1304	-13.712755258384\\
1305	-9.80664094255712\\
1306	-14.3830176303243\\
1307	-14.8042951933692\\
1308	-14.9587962583273\\
1309	-14.186335644582\\
1310	-12.7849206217434\\
1311	-13.1848878266896\\
1312	-18.6601744565226\\
1313	-17.9029223865168\\
1314	-14.6650995902644\\
1315	-18.4932180660358\\
1316	-17.7611527541146\\
1317	-16.1767011041902\\
1318	-17.4485515097251\\
1319	-17.7624666498043\\
1320	-15.3013601646521\\
1321	-14.4480429026635\\
1322	-18.1102961287561\\
1323	-26.5407080770142\\
1324	-20.7259990866919\\
1325	-14.6273768555711\\
1326	-14.7856772567341\\
1327	-15.3671246117019\\
1328	-17.3338787181954\\
1329	-18.5021990592131\\
1330	-15.2916660579713\\
1331	-15.7864428906041\\
1332	-18.2850640139434\\
1333	-19.9305607726169\\
1334	-14.4211422250341\\
1335	-13.9806905987857\\
1336	-16.4220015903479\\
1337	-23.6510969057688\\
1338	-21.7408940277089\\
1339	-21.0713479614687\\
1340	-14.6095421535131\\
1341	-14.81063875365\\
1342	-15.0529493792183\\
1343	-12.9070686161986\\
1344	-13.5547939172093\\
1345	-14.0879399649157\\
1346	-13.4117406209859\\
1347	-9.94134765549029\\
1348	-14.7382020962925\\
1349	-16.9708451426727\\
1350	-13.9909803103108\\
1351	-13.3732251316094\\
1352	-14.2467949246404\\
1353	-12.5297380996046\\
1354	-12.1471304708575\\
1355	-11.6882165528109\\
1356	-11.6310113689174\\
1357	-10.8888740473628\\
1358	-13.1779719209912\\
1359	-17.1506168468718\\
1360	-12.9960549134555\\
1361	-12.6854581013549\\
1362	-12.2347816669906\\
1363	-11.7298240919165\\
1364	-11.4519759215473\\
1365	-12.0603462387355\\
1366	-23.14057268496\\
1367	-19.2482086362695\\
1368	-21.0453781219883\\
1369	-25.3262152203322\\
1370	-25.4236018145793\\
1371	-19.4951451903999\\
1372	-16.4007514117229\\
1373	-16.4792377060799\\
1374	-16.8833184636735\\
1375	-18.0091970064314\\
1376	-19.5196954882763\\
1377	-19.7573777709058\\
1378	-25.3345063401109\\
1379	-28.8624146609616\\
1380	-33.5463849591126\\
1381	-23.6643264128427\\
1382	-23.7598241767632\\
1383	-27.8570584596623\\
1384	-20.7767147410339\\
1385	-23.4427098533874\\
1386	-21.4923547910537\\
1387	-14.4801638200861\\
1388	-13.3051662689884\\
1389	-13.209775204378\\
1390	-15.1069898218229\\
1391	-13.921510680075\\
1392	-12.442855980888\\
1393	-12.5818052663565\\
1394	-12.0751519299008\\
1395	-11.9337262940669\\
1396	-11.1765042448778\\
1397	-11.2878783189731\\
1398	-11.1392393884269\\
1399	-11.9139506292279\\
1400	-15.5014885102514\\
1401	-13.2886342643101\\
1402	-17.6260013162384\\
1403	-27.2383337734143\\
1404	-24.8906695892092\\
1405	-28.3549983962773\\
1406	-20.9406743814598\\
1407	-21.8383197250105\\
1408	-16.6922511085795\\
1409	-14.5826978064897\\
1410	-14.6617476415329\\
1411	-14.1530924874746\\
1412	-12.8318930539469\\
1413	-13.0636348465762\\
1414	-11.6430509264469\\
1415	-13.8295917541968\\
1416	-11.1774663708802\\
1417	-11.8366564211924\\
1418	-16.612830248893\\
1419	-18.7874494410419\\
1420	-17.4856197166603\\
1421	-16.6827999655338\\
1422	-18.3448707462114\\
1423	-16.7588515846226\\
1424	-14.7521251467931\\
1425	-14.6676196665169\\
1426	-14.005480729332\\
1427	-16.3498729574474\\
1428	-13.4156208637094\\
1429	-12.9843955245497\\
1430	-13.7996626202778\\
1431	-18.9354298325939\\
1432	-15.482428296679\\
1433	-13.6899804876173\\
1434	-13.0474967111886\\
1435	-13.5099383136983\\
1436	-12.2846120104141\\
1437	-12.0625899852917\\
1438	-11.915995344498\\
1439	-13.7055906788978\\
1440	-14.6622298974896\\
1441	-12.9670908044673\\
1442	-13.6161301696086\\
1443	-13.910992279564\\
1444	-12.194554940176\\
1445	-11.7771150159688\\
1446	-14.3884821370987\\
1447	-21.9888650744065\\
1448	-20.3959169743762\\
1449	-17.2669421205665\\
1450	-16.6058119741579\\
1451	-15.6409194922851\\
1453	-13.7926745673426\\
1454	-14.9147116611348\\
1455	-15.0551802207381\\
1456	-12.6718535945008\\
1457	-13.174114918743\\
1458	-15.5135360474512\\
1459	-16.2696030589768\\
1460	-16.7965899217625\\
1461	-17.1130253627123\\
1462	-21.9462246789631\\
1463	-27.8378634922697\\
1464	-31.5087711985134\\
1465	-21.4234827144812\\
1466	-14.6703095404671\\
1467	-15.1661618555452\\
1468	-15.9048011029852\\
1469	-13.2050023689562\\
1470	-12.7622223366275\\
1471	-15.3836176202526\\
1472	-15.3587036918068\\
1473	-13.9159933274022\\
1474	-14.9082478935318\\
1475	-16.7142467556964\\
1476	-18.268963864567\\
1477	-19.1626447859924\\
1478	-27.1537795558208\\
1479	-24.9472879846157\\
1480	-18.8448827480477\\
1481	-15.7072366918028\\
1482	-15.870676936644\\
1483	-16.5990407933762\\
1484	-18.1501701969037\\
1485	-14.6979733424212\\
1486	-14.4392940607215\\
1487	-14.6592180203315\\
1488	-12.4938613669708\\
1489	-12.2467007235055\\
1490	-17.4113415010204\\
1491	-14.1961704605542\\
1492	-14.3467846401443\\
1493	-22.4857591391628\\
1494	-19.6746164762314\\
1495	-17.3480680504695\\
1496	-21.4973522940218\\
1497	-21.8838727330983\\
1498	-16.3947255848796\\
1499	-13.718802344187\\
1500	-13.7997626924878\\
1501	-17.2966697843449\\
1502	-27.4894098356083\\
1503	-28.6456617863071\\
1504	-25.6232158791845\\
1506	-16.4239576610569\\
1507	-15.8749557254232\\
1508	-14.3222190273634\\
1509	-13.7952544716604\\
1510	-12.8859254963807\\
1511	-13.1249827889869\\
1512	-14.1330984759998\\
1513	-12.1731392632223\\
1514	-12.2054651827757\\
1515	-15.0535894199452\\
1516	-17.190789935443\\
1517	-19.4922968155031\\
1518	-24.515055919825\\
1519	-30.1282261545921\\
1520	-22.4821223208328\\
1521	-19.2819016959709\\
1522	-20.4481916998302\\
1523	-19.0240706188611\\
1524	-14.9641899875915\\
1525	-15.7873575735691\\
1526	-23.040920625006\\
1527	-27.0495332934661\\
1528	-16.6220188760265\\
1529	-13.9467133261346\\
1530	-14.8023393196652\\
1531	-17.0306860683963\\
1532	-14.8710179241627\\
1533	-11.0803706684605\\
1534	-12.4310270194633\\
1535	-14.020348897538\\
1536	-13.1725903279055\\
1537	-12.2840059584066\\
1538	-11.5888486277436\\
1539	-11.2874482129455\\
1540	-11.0551333096655\\
1541	-10.7393706053822\\
1542	-11.475788954428\\
1543	-16.7094992076782\\
1544	-17.9340369256593\\
1545	-16.2388462385477\\
1546	-18.5463243242011\\
1547	-23.4967193847808\\
1548	-23.0633378585012\\
1549	-26.7978028205673\\
1550	-25.1501559253722\\
1551	-19.5236630069408\\
1552	-16.1022710927086\\
1553	-16.115116821707\\
1554	-23.9028217693917\\
1555	-26.1499573318649\\
1556	-17.7572205161869\\
1557	-14.8202476023941\\
1558	-14.3655520090383\\
1559	-16.681386186413\\
1560	-13.6580292421011\\
1561	-14.4645289183668\\
1562	-9.93246153001405\\
1563	-12.5204833239832\\
1564	-12.748101439161\\
1565	-12.2552765960352\\
1566	-12.1760101024879\\
1567	-12.4913269401061\\
1568	-10.0901109997765\\
1569	-9.82784560629261\\
1571	-10.1853930848745\\
1572	-10.1671050335042\\
1573	-8.41929455935065\\
1574	-20.3704780522714\\
1575	-26.7262966761098\\
1576	-25.200954937832\\
1577	-19.1649359625048\\
1578	-22.819583563821\\
1579	-35.7351141839474\\
1580	-32.2274062364509\\
1581	-28.0869827371253\\
1582	-22.1740090437834\\
1583	-22.9541811925112\\
1584	-23.9294072657453\\
1585	-17.5136870664203\\
1586	-16.3097311510903\\
1587	-13.7503988992487\\
1588	-13.5933407618174\\
1589	-16.4447977830748\\
1590	-13.9903196045709\\
1591	-13.9944719138291\\
1592	-14.9765698536526\\
1593	-14.5805023549101\\
1595	-14.513075739161\\
1596	-15.7170866829458\\
1597	-16.7239479603822\\
1598	-14.9404675878761\\
1599	-13.5094903655659\\
1600	-14.376613578577\\
1601	-12.0214763434119\\
1602	-16.698852586165\\
1603	-12.6800825965186\\
1604	-10.707054208033\\
1605	-11.162992646174\\
1606	-17.2842216129907\\
1607	-9.4921767125079\\
1608	-8.43995383014612\\
1609	-10.9595037329705\\
1610	-12.4349836717502\\
1611	-11.4600803570438\\
1612	-13.7355605378148\\
1613	-14.0125997069651\\
1614	-12.3221982053178\\
1615	-12.7802049863556\\
1616	-11.3259703034855\\
1617	-12.6525322541886\\
1618	-13.1721091858501\\
1619	-14.1360847411268\\
1620	-8.75858773658774\\
1621	-9.61178580458295\\
1622	-9.02736837445286\\
1623	-12.416722742023\\
1624	-12.4428478986438\\
1625	-11.4532714602531\\
1626	-11.0149644246935\\
1627	-10.9467456041652\\
1628	-10.7758782725402\\
1629	-10.6485494205735\\
1630	-10.5863414514788\\
1631	-10.2205026806225\\
1632	-10.6546599746082\\
1633	-9.15526357989984\\
1634	-9.4713728827121\\
1635	-9.10104413049498\\
1636	-10.074134347883\\
1637	-10.0666756696949\\
1638	-9.78499317275964\\
1639	-10.3375622813592\\
1640	-9.35708228778367\\
1641	-11.4882894763639\\
1642	-19.4330341299108\\
1643	-15.2178872206732\\
1644	-13.7586082232829\\
1645	-12.7411562770865\\
1646	-14.3255860466281\\
1647	-14.8101291757021\\
1648	-12.5605471850456\\
1649	-12.3476910744218\\
1650	-11.9074427923479\\
1651	-12.6374701627158\\
1652	-11.3337466812866\\
1653	-11.5468594774136\\
1654	-10.6649912167029\\
1655	-11.9887666314453\\
1656	-11.2944788027989\\
1657	-11.1464056834864\\
1658	-10.9158965563813\\
1659	-13.2722967031209\\
1660	-12.8441033430713\\
1661	-16.0137401284924\\
1662	-25.281440296078\\
1663	-20.1094615546774\\
1664	-14.4227858066035\\
1665	-14.1274360647606\\
1666	-16.0274966857496\\
1667	-19.6748145720405\\
1668	-16.2976843896615\\
1669	-17.1309175747626\\
1670	-13.4409503167851\\
1671	-14.4368923353838\\
1672	-13.4828964504252\\
1673	-13.3248051914991\\
1674	-15.4783960355678\\
1675	-14.022334742355\\
1676	-15.7986217906725\\
1677	-16.9293101973556\\
1678	-14.7759236301783\\
1679	-13.3988142458497\\
1680	-13.5280349087921\\
1681	-16.3275280574708\\
1682	-14.2877630051396\\
1683	-14.323562883465\\
1684	-13.9261646546977\\
1685	-12.907434540173\\
1686	-12.8265973064333\\
1687	-12.2019818529848\\
1688	-12.1024503903591\\
1689	-14.4681482397732\\
1690	-17.1973787298116\\
1691	-16.7793300895971\\
1692	-15.6674492810523\\
1693	-13.6556981203687\\
1694	-13.1141855627131\\
1695	-12.6873728488658\\
1696	-13.2549288938894\\
1697	-14.6514038354369\\
1698	-17.0221115265442\\
1699	-19.5612407840977\\
1700	-15.9135284378876\\
1701	-13.4996746995475\\
1702	-15.7217428637462\\
1703	-23.0830881123707\\
1704	-16.8642078082444\\
1705	-16.3298075304147\\
1706	-18.5092182859341\\
1707	-15.8882653592941\\
1708	-14.3340234490524\\
1709	-15.206730449994\\
1710	-14.6179449685819\\
1711	-15.3992206762287\\
1712	-17.3568838940171\\
1713	-14.8710106127485\\
1714	-15.5469525543385\\
1715	-15.9523352814183\\
1716	-15.518757385812\\
1717	-14.1137249738629\\
1718	-15.4558126465079\\
1719	-13.8797983262014\\
1720	-13.72871737639\\
1721	-14.3786336501803\\
1722	-14.4106969848885\\
1723	-12.1385543579036\\
1724	-14.8386434758681\\
1725	-15.6531570514026\\
1726	-10.6118159201333\\
1727	-14.5146033608455\\
1728	-23.4311930661852\\
1729	-19.9483502252576\\
1730	-15.5325984138235\\
1731	-15.312963699246\\
1732	-15.2697384707121\\
1733	-14.3447425956165\\
1735	-12.3254269721531\\
1736	-12.3948363807085\\
1737	-12.3437908695555\\
1738	-12.5167120627364\\
1740	-11.7937602840348\\
1741	-11.7573881330138\\
1742	-10.6425558125693\\
1743	-14.3266174097703\\
1744	-12.522517351432\\
1745	-12.9781194005834\\
1746	-12.7038194844847\\
1747	-14.5716605445803\\
1748	-14.3985801451388\\
1749	-17.9941342268014\\
1750	-16.1109770833336\\
1751	-15.6930051663549\\
1752	-16.0698007357073\\
1753	-12.8270477201352\\
1754	-14.895758587762\\
1755	-13.2948705010288\\
1756	-13.9118537899747\\
1757	-14.3637164642812\\
1758	-16.3963730925107\\
1759	-14.4510747588552\\
1760	-15.3833693353388\\
1761	-19.2897756289622\\
1762	-17.3431215494713\\
1763	-15.0956300281232\\
1764	-14.0611518179539\\
1765	-15.2858679063195\\
1766	-14.143636572051\\
1767	-13.341616665009\\
1768	-12.6926067744387\\
1769	-12.843817093559\\
1770	-12.4813173039229\\
1771	-17.2594916174198\\
1772	-24.8134791124987\\
1773	-19.9926148219697\\
1774	-18.830938607681\\
1775	-18.4307880164483\\
1776	-14.4353417957946\\
1777	-14.0331506541652\\
1778	-13.237655682188\\
1779	-12.7959841993647\\
1780	-12.204391899543\\
1781	-14.1129144652846\\
1782	-17.125496999166\\
1783	-18.1881684697696\\
1784	-16.3631182989971\\
1785	-13.9057039464274\\
1786	-17.1556892942986\\
1787	-22.1703281615942\\
1788	-21.6996832198315\\
1789	-18.9184550739562\\
1790	-26.6922758467149\\
1791	-23.0656564488834\\
1792	-14.6597932273833\\
1793	-14.3203552651621\\
1794	-14.0413084835134\\
1795	-15.6859097252479\\
1796	-20.337469135666\\
1797	-22.0346809473658\\
1798	-18.6679124797299\\
1799	-15.7228073265778\\
1800	-14.0166010594194\\
1801	-13.8113292077874\\
1802	-13.3877409539618\\
1803	-13.3119824109153\\
1804	-12.4603385603655\\
1805	-12.0637430931281\\
};
\addlegendentry{MPO prediction}

\end{axis}

\begin{axis}[%
width=6.159cm,
height=1.831cm,
at={(0cm,5.085cm)},
scale only axis,
xmin=1000,
xmax=2000,
xlabel style={font=\color{white!15!black}},
xlabel={Sample index},
ymin=-38.4123338858692,
ymax=2.441,
ylabel style={font=\color{white!15!black}},
ylabel={$y(t)$},
axis background/.style={fill=white},
title style={font=\bfseries},
title={C5: RMSE(OSA) = 2.757, RMSE(MPO) = 5.0346},
legend style={legend cell align=left, align=left, draw=white!15!black}
]
\addplot [color=mycolor1, line width=2.0pt]
  table[row sep=crcr]{%
1006	-15.8689999999999\\
1007	-14.6479999999999\\
1008	-14.6479999999999\\
1009	-7.32400000000007\\
1010	-8.54500000000007\\
1011	-10.9860000000001\\
1012	-9.76600000000008\\
1013	-9.76600000000008\\
1015	-2.44100000000003\\
1016	-3.66200000000003\\
1017	-2.44100000000003\\
1018	-12.2070000000001\\
1019	-13.4280000000001\\
1021	-10.9860000000001\\
1022	-6.10400000000004\\
1023	-6.10400000000004\\
1024	-1.221\\
1026	-10.9860000000001\\
1027	-12.2070000000001\\
1028	-8.54500000000007\\
1029	-14.6479999999999\\
1030	-14.6479999999999\\
1031	-9.76600000000008\\
1032	-13.4280000000001\\
1033	-12.2070000000001\\
1034	-9.76600000000008\\
1035	-9.76600000000008\\
1036	-8.54500000000007\\
1037	-8.54500000000007\\
1038	-12.2070000000001\\
1039	-10.9860000000001\\
1042	-10.9860000000001\\
1043	-14.6479999999999\\
1044	-15.8689999999999\\
1045	-9.76600000000008\\
1046	-8.54500000000007\\
1047	-8.54500000000007\\
1048	-6.10400000000004\\
1049	-7.32400000000007\\
1050	-7.32400000000007\\
1051	-4.88300000000004\\
1052	-9.76600000000008\\
1053	-10.9860000000001\\
1054	-8.54500000000007\\
1055	-13.4280000000001\\
1056	-15.8689999999999\\
1057	-8.54500000000007\\
1058	-6.10400000000004\\
1059	-6.10400000000004\\
1060	-9.76600000000008\\
1061	-6.10400000000004\\
1062	-4.88300000000004\\
1063	-7.32400000000007\\
1064	-7.32400000000007\\
1065	-3.66200000000003\\
1068	-10.9860000000001\\
1069	-10.9860000000001\\
1070	-9.76600000000008\\
1071	-13.4280000000001\\
1072	-10.9860000000001\\
1073	-12.2070000000001\\
1074	-10.9860000000001\\
1075	-10.9860000000001\\
1076	-7.32400000000007\\
1077	-8.54500000000007\\
1079	-20.752\\
1080	-20.752\\
1081	-19.5309999999999\\
1082	-13.4280000000001\\
1083	-15.8689999999999\\
1084	-24.414\\
1085	-23.193\\
1086	-15.8689999999999\\
1088	-28.076\\
1089	-23.193\\
1090	-17.0899999999999\\
1091	-14.6479999999999\\
1092	-13.4280000000001\\
1093	-9.76600000000008\\
1095	-12.2070000000001\\
1096	-4.88300000000004\\
1097	-4.88300000000004\\
1098	-7.32400000000007\\
1099	-10.9860000000001\\
1100	-9.76600000000008\\
1101	-9.76600000000008\\
1102	-10.9860000000001\\
1103	-13.4280000000001\\
1105	-15.8689999999999\\
1106	-14.6479999999999\\
1107	-17.0899999999999\\
1108	-12.2070000000001\\
1109	-12.2070000000001\\
1110	-10.9860000000001\\
1111	-8.54500000000007\\
1112	-9.76600000000008\\
1113	-12.2070000000001\\
1114	-8.54500000000007\\
1115	-9.76600000000008\\
1116	-7.32400000000007\\
1117	-6.10400000000004\\
1118	-8.54500000000007\\
1119	-6.10400000000004\\
1120	-4.88300000000004\\
1121	-8.54500000000007\\
1122	-13.4280000000001\\
1124	-10.9860000000001\\
1125	-7.32400000000007\\
1126	-8.54500000000007\\
1127	-19.5309999999999\\
1128	-14.6479999999999\\
1129	-10.9860000000001\\
1130	-17.0899999999999\\
1131	-12.2070000000001\\
1132	-6.10400000000004\\
1133	-9.76600000000008\\
1134	-8.54500000000007\\
1135	-14.6479999999999\\
1136	-15.8689999999999\\
1137	-19.5309999999999\\
1138	-14.6479999999999\\
1139	-14.6479999999999\\
1140	-13.4280000000001\\
1141	-13.4280000000001\\
1142	-14.6479999999999\\
1144	-7.32400000000007\\
1145	-7.32400000000007\\
1146	-6.10400000000004\\
1148	-10.9860000000001\\
1149	-15.8689999999999\\
1150	-14.6479999999999\\
1151	-8.54500000000007\\
1152	-8.54500000000007\\
1153	-9.76600000000008\\
1155	-2.44100000000003\\
1157	-7.32400000000007\\
1158	-8.54500000000007\\
1159	-6.10400000000004\\
1160	-7.32400000000007\\
1162	-4.88300000000004\\
1163	-4.88300000000004\\
1164	-1.221\\
1165	-3.66200000000003\\
1166	-12.2070000000001\\
1167	-14.6479999999999\\
1168	-15.8689999999999\\
1169	-12.2070000000001\\
1170	-7.32400000000007\\
1171	-6.10400000000004\\
1172	-3.66200000000003\\
1173	-7.32400000000007\\
1175	-9.76600000000008\\
1176	-12.2070000000001\\
1177	-18.3109999999999\\
1178	-18.3109999999999\\
1179	-19.5309999999999\\
1180	-18.3109999999999\\
1181	-15.8689999999999\\
1182	-15.8689999999999\\
1184	-18.3109999999999\\
1185	-12.2070000000001\\
1186	-10.9860000000001\\
1187	-12.2070000000001\\
1188	-10.9860000000001\\
1189	-12.2070000000001\\
1190	-9.76600000000008\\
1191	-15.8689999999999\\
1192	-18.3109999999999\\
1193	-12.2070000000001\\
1194	-7.32400000000007\\
1195	-9.76600000000008\\
1196	-17.0899999999999\\
1197	-18.3109999999999\\
1198	-18.3109999999999\\
1199	-21.973\\
1200	-20.752\\
1201	-17.0899999999999\\
1202	-15.8689999999999\\
1204	-20.752\\
1205	-17.0899999999999\\
1206	-8.54500000000007\\
1207	-14.6479999999999\\
1208	-12.2070000000001\\
1209	-7.32400000000007\\
1210	-10.9860000000001\\
1213	-7.32400000000007\\
1214	-8.54500000000007\\
1215	-8.54500000000007\\
1216	-9.76600000000008\\
1217	-13.4280000000001\\
1218	-14.6479999999999\\
1219	-7.32400000000007\\
1220	-8.54500000000007\\
1221	-12.2070000000001\\
1222	-17.0899999999999\\
1223	-15.8689999999999\\
1224	-8.54500000000007\\
1225	-8.54500000000007\\
1226	-9.76600000000008\\
1229	-6.10400000000004\\
1230	-8.54500000000007\\
1231	-9.76600000000008\\
1233	-9.76600000000008\\
1235	-14.6479999999999\\
1236	-15.8689999999999\\
1237	-10.9860000000001\\
1238	-13.4280000000001\\
1239	-17.0899999999999\\
1240	-12.2070000000001\\
1241	-3.66200000000003\\
1242	-9.76600000000008\\
1243	-8.54500000000007\\
1244	-9.76600000000008\\
1245	-8.54500000000007\\
1247	-8.54500000000007\\
1248	-10.9860000000001\\
1249	-7.32400000000007\\
1250	-7.32400000000007\\
1251	-9.76600000000008\\
1252	-6.10400000000004\\
1253	-8.54500000000007\\
1255	-3.66200000000003\\
1256	-6.10400000000004\\
1257	-4.88300000000004\\
1258	-10.9860000000001\\
1259	-12.2070000000001\\
1260	-12.2070000000001\\
1261	-18.3109999999999\\
1262	-10.9860000000001\\
1263	-4.88300000000004\\
1264	-6.10400000000004\\
1265	-6.10400000000004\\
1266	-8.54500000000007\\
1267	-6.10400000000004\\
1268	-8.54500000000007\\
1269	-7.32400000000007\\
1270	-13.4280000000001\\
1271	-9.76600000000008\\
1272	-14.6479999999999\\
1273	-10.9860000000001\\
1274	-9.76600000000008\\
1275	-6.10400000000004\\
1276	-3.66200000000003\\
1277	-3.66200000000003\\
1278	-1.221\\
1279	-2.44100000000003\\
1280	-8.54500000000007\\
1281	-8.54500000000007\\
1282	-7.32400000000007\\
1283	-9.76600000000008\\
1284	-15.8689999999999\\
1285	-10.9860000000001\\
1286	-9.76600000000008\\
1287	-9.76600000000008\\
1288	-13.4280000000001\\
1289	-14.6479999999999\\
1290	-12.2070000000001\\
1291	-12.2070000000001\\
1292	-6.10400000000004\\
1293	-2.44100000000003\\
1294	-4.88300000000004\\
1295	-4.88300000000004\\
1296	-6.10400000000004\\
1297	-3.66200000000003\\
1298	-4.88300000000004\\
1299	-9.76600000000008\\
1300	-7.32400000000007\\
1301	-7.32400000000007\\
1302	-10.9860000000001\\
1303	-7.32400000000007\\
1304	-4.88300000000004\\
1305	-9.76600000000008\\
1306	-12.2070000000001\\
1307	-9.76600000000008\\
1308	-10.9860000000001\\
1309	-7.32400000000007\\
1310	-7.32400000000007\\
1312	-14.6479999999999\\
1313	-13.4280000000001\\
1314	-8.54500000000007\\
1315	-14.6479999999999\\
1316	-13.4280000000001\\
1319	-13.4280000000001\\
1320	-10.9860000000001\\
1321	-10.9860000000001\\
1322	-12.2070000000001\\
1323	-18.3109999999999\\
1324	-17.0899999999999\\
1325	-10.9860000000001\\
1327	-10.9860000000001\\
1328	-13.4280000000001\\
1329	-13.4280000000001\\
1330	-9.76600000000008\\
1332	-14.6479999999999\\
1333	-14.6479999999999\\
1334	-9.76600000000008\\
1335	-8.54500000000007\\
1336	-10.9860000000001\\
1337	-18.3109999999999\\
1338	-14.6479999999999\\
1339	-13.4280000000001\\
1340	-9.76600000000008\\
1341	-10.9860000000001\\
1342	-9.76600000000008\\
1343	-6.10400000000004\\
1345	-3.66200000000003\\
1346	-3.66200000000003\\
1348	-10.9860000000001\\
1349	-9.76600000000008\\
1350	-6.10400000000004\\
1351	-7.32400000000007\\
1352	-9.76600000000008\\
1353	-6.10400000000004\\
1354	-6.10400000000004\\
1355	-9.76600000000008\\
1356	-6.10400000000004\\
1357	-7.32400000000007\\
1358	-12.2070000000001\\
1359	-12.2070000000001\\
1361	-4.88300000000004\\
1363	-7.32400000000007\\
1364	-6.10400000000004\\
1365	-7.32400000000007\\
1366	-14.6479999999999\\
1367	-9.76600000000008\\
1368	-17.0899999999999\\
1369	-20.752\\
1370	-18.3109999999999\\
1371	-13.4280000000001\\
1372	-10.9860000000001\\
1373	-10.9860000000001\\
1377	-15.8689999999999\\
1378	-20.752\\
1379	-20.752\\
1380	-24.414\\
1381	-15.8689999999999\\
1382	-20.752\\
1383	-20.752\\
1384	-14.6479999999999\\
1385	-17.0899999999999\\
1386	-18.3109999999999\\
1387	-9.76600000000008\\
1388	-7.32400000000007\\
1390	-9.76600000000008\\
1391	-7.32400000000007\\
1392	-6.10400000000004\\
1393	-7.32400000000007\\
1394	-6.10400000000004\\
1395	-3.66200000000003\\
1396	-4.88300000000004\\
1397	-7.32400000000007\\
1398	-2.44100000000003\\
1399	-10.9860000000001\\
1400	-8.54500000000007\\
1401	-7.32400000000007\\
1402	-15.8689999999999\\
1403	-19.5309999999999\\
1405	-19.5309999999999\\
1406	-14.6479999999999\\
1407	-17.0899999999999\\
1409	-7.32400000000007\\
1410	-10.9860000000001\\
1412	-3.66200000000003\\
1413	-6.10400000000004\\
1414	-4.88300000000004\\
1415	-2.44100000000003\\
1416	-4.88300000000004\\
1417	-10.9860000000001\\
1418	-9.76600000000008\\
1419	-10.9860000000001\\
1420	-10.9860000000001\\
1421	-12.2070000000001\\
1422	-14.6479999999999\\
1423	-10.9860000000001\\
1424	-10.9860000000001\\
1425	-8.54500000000007\\
1426	-7.32400000000007\\
1427	-13.4280000000001\\
1429	-6.10400000000004\\
1430	-10.9860000000001\\
1431	-13.4280000000001\\
1432	-10.9860000000001\\
1433	-7.32400000000007\\
1435	-7.32400000000007\\
1436	-4.88300000000004\\
1438	-7.32400000000007\\
1439	-10.9860000000001\\
1440	-8.54500000000007\\
1441	-7.32400000000007\\
1442	-10.9860000000001\\
1444	-6.10400000000004\\
1445	-7.32400000000007\\
1446	-13.4280000000001\\
1447	-14.6479999999999\\
1449	-12.2070000000001\\
1450	-12.2070000000001\\
1451	-8.54500000000007\\
1452	-9.76600000000008\\
1453	-8.54500000000007\\
1454	-12.2070000000001\\
1455	-7.32400000000007\\
1456	-4.88300000000004\\
1457	-9.76600000000008\\
1458	-9.76600000000008\\
1460	-12.2070000000001\\
1461	-10.9860000000001\\
1462	-17.0899999999999\\
1463	-20.752\\
1464	-23.193\\
1465	-15.8689999999999\\
1466	-9.76600000000008\\
1467	-6.10400000000004\\
1468	-6.10400000000004\\
1469	-9.76600000000008\\
1470	-6.10400000000004\\
1471	-9.76600000000008\\
1472	-6.10400000000004\\
1474	-8.54500000000007\\
1476	-13.4280000000001\\
1477	-14.6479999999999\\
1478	-19.5309999999999\\
1479	-14.6479999999999\\
1480	-13.4280000000001\\
1481	-10.9860000000001\\
1483	-10.9860000000001\\
1484	-12.2070000000001\\
1485	-9.76600000000008\\
1486	-10.9860000000001\\
1487	-9.76600000000008\\
1488	-7.32400000000007\\
1489	-12.2070000000001\\
1490	-13.4280000000001\\
1491	-8.54500000000007\\
1492	-9.76600000000008\\
1493	-18.3109999999999\\
1494	-12.2070000000001\\
1495	-12.2070000000001\\
1496	-17.0899999999999\\
1497	-14.6479999999999\\
1499	-7.32400000000007\\
1500	-7.32400000000007\\
1502	-17.0899999999999\\
1503	-20.752\\
1504	-18.3109999999999\\
1505	-13.4280000000001\\
1506	-9.76600000000008\\
1507	-8.54500000000007\\
1508	-6.10400000000004\\
1509	-7.32400000000007\\
1510	-4.88300000000004\\
1511	-8.54500000000007\\
1512	-8.54500000000007\\
1513	-6.10400000000004\\
1514	-9.76600000000008\\
1516	-12.2070000000001\\
1519	-19.5309999999999\\
1520	-18.3109999999999\\
1521	-13.4280000000001\\
1522	-14.6479999999999\\
1523	-12.2070000000001\\
1524	-7.32400000000007\\
1526	-19.5309999999999\\
1527	-18.3109999999999\\
1529	-6.10400000000004\\
1530	-4.88300000000004\\
1532	2.44100000000003\\
1533	-4.88300000000004\\
1534	-6.10400000000004\\
1535	-8.54500000000007\\
1536	-6.10400000000004\\
1537	-4.88300000000004\\
1538	-4.88300000000004\\
1539	-7.32400000000007\\
1540	-4.88300000000004\\
1541	-4.88300000000004\\
1543	-12.2070000000001\\
1544	-13.4280000000001\\
1545	-10.9860000000001\\
1546	-14.6479999999999\\
1547	-17.0899999999999\\
1548	-17.0899999999999\\
1549	-20.752\\
1550	-20.752\\
1551	-14.6479999999999\\
1552	-10.9860000000001\\
1553	-10.9860000000001\\
1554	-18.3109999999999\\
1555	-18.3109999999999\\
1556	-12.2070000000001\\
1558	-4.88300000000004\\
1560	-4.88300000000004\\
1561	-2.44100000000003\\
1562	-6.10400000000004\\
1563	-8.54500000000007\\
1564	-7.32400000000007\\
1565	-7.32400000000007\\
1566	-6.10400000000004\\
1567	-3.66200000000003\\
1568	-4.88300000000004\\
1569	-4.88300000000004\\
1570	-6.10400000000004\\
1571	-2.44100000000003\\
1572	-2.44100000000003\\
1573	-7.32400000000007\\
1574	-9.76600000000008\\
1575	-13.4280000000001\\
1576	-15.8689999999999\\
1577	-10.9860000000001\\
1578	-18.3109999999999\\
1579	-23.193\\
1580	-21.973\\
1581	-19.5309999999999\\
1582	-14.6479999999999\\
1583	-14.6479999999999\\
1584	-17.0899999999999\\
1586	-9.76600000000008\\
1587	-7.32400000000007\\
1588	-7.32400000000007\\
1589	-13.4280000000001\\
1590	-8.54500000000007\\
1591	-12.2070000000001\\
1592	-10.9860000000001\\
1593	-8.54500000000007\\
1594	-8.54500000000007\\
1595	-10.9860000000001\\
1596	-9.76600000000008\\
1597	-12.2070000000001\\
1598	-10.9860000000001\\
1599	-8.54500000000007\\
1600	-12.2070000000001\\
1601	-6.10400000000004\\
1602	-2.44100000000003\\
1603	-7.32400000000007\\
1604	-7.32400000000007\\
1605	-2.44100000000003\\
1606	-1.221\\
1607	-2.44100000000003\\
1608	-6.10400000000004\\
1609	-6.10400000000004\\
1610	-8.54500000000007\\
1611	-7.32400000000007\\
1612	-13.4280000000001\\
1613	-8.54500000000007\\
1614	-6.10400000000004\\
1615	-10.9860000000001\\
1616	-7.32400000000007\\
1617	-4.88300000000004\\
1618	-4.88300000000004\\
1619	-2.44100000000003\\
1620	-3.66200000000003\\
1621	-3.66200000000003\\
1622	-6.10400000000004\\
1623	-9.76600000000008\\
1624	-8.54500000000007\\
1625	-4.88300000000004\\
1626	-7.32400000000007\\
1628	-7.32400000000007\\
1629	-6.10400000000004\\
1630	-2.44100000000003\\
1631	-7.32400000000007\\
1632	-3.66200000000003\\
1633	-4.88300000000004\\
1634	-4.88300000000004\\
1635	-7.32400000000007\\
1638	-3.66200000000003\\
1639	-4.88300000000004\\
1640	-4.88300000000004\\
1641	-13.4280000000001\\
1642	-13.4280000000001\\
1643	-7.32400000000007\\
1644	-12.2070000000001\\
1645	-7.32400000000007\\
1646	-12.2070000000001\\
1647	-8.54500000000007\\
1648	-7.32400000000007\\
1649	-8.54500000000007\\
1650	-7.32400000000007\\
1651	-4.88300000000004\\
1652	-6.10400000000004\\
1653	-9.76600000000008\\
1654	-7.32400000000007\\
1655	-12.2070000000001\\
1656	-6.10400000000004\\
1657	-8.54500000000007\\
1658	-6.10400000000004\\
1659	-12.2070000000001\\
1660	-7.32400000000007\\
1661	-14.6479999999999\\
1662	-17.0899999999999\\
1664	-9.76600000000008\\
1665	-9.76600000000008\\
1666	-12.2070000000001\\
1667	-13.4280000000001\\
1668	-8.54500000000007\\
1669	-13.4280000000001\\
1670	-10.9860000000001\\
1671	-4.88300000000004\\
1672	-3.66200000000003\\
1673	-12.2070000000001\\
1674	-10.9860000000001\\
1675	-8.54500000000007\\
1676	-10.9860000000001\\
1677	-12.2070000000001\\
1679	-7.32400000000007\\
1681	-12.2070000000001\\
1682	-8.54500000000007\\
1683	-10.9860000000001\\
1684	-9.76600000000008\\
1685	-6.10400000000004\\
1686	-8.54500000000007\\
1687	-4.88300000000004\\
1688	-8.54500000000007\\
1689	-10.9860000000001\\
1690	-14.6479999999999\\
1691	-13.4280000000001\\
1692	-13.4280000000001\\
1693	-8.54500000000007\\
1694	-6.10400000000004\\
1696	-8.54500000000007\\
1697	-12.2070000000001\\
1698	-12.2070000000001\\
1699	-14.6479999999999\\
1700	-10.9860000000001\\
1701	-8.54500000000007\\
1702	-12.2070000000001\\
1703	-18.3109999999999\\
1704	-12.2070000000001\\
1705	-14.6479999999999\\
1706	-14.6479999999999\\
1707	-9.76600000000008\\
1708	-9.76600000000008\\
1709	-12.2070000000001\\
1710	-9.76600000000008\\
1711	-10.9860000000001\\
1712	-14.6479999999999\\
1713	-9.76600000000008\\
1714	-12.2070000000001\\
1715	-10.9860000000001\\
1716	-10.9860000000001\\
1717	-6.10400000000004\\
1718	-9.76600000000008\\
1719	-8.54500000000007\\
1720	-9.76600000000008\\
1722	-9.76600000000008\\
1723	-6.10400000000004\\
1724	-3.66200000000003\\
1725	-2.44100000000003\\
1726	-6.10400000000004\\
1727	-10.9860000000001\\
1728	-12.2070000000001\\
1729	-10.9860000000001\\
1730	-8.54500000000007\\
1731	-12.2070000000001\\
1732	-8.54500000000007\\
1733	-9.76600000000008\\
1734	-6.10400000000004\\
1735	-8.54500000000007\\
1736	-7.32400000000007\\
1739	-7.32400000000007\\
1740	-6.10400000000004\\
1741	-6.10400000000004\\
1742	-4.88300000000004\\
1743	-10.9860000000001\\
1744	-7.32400000000007\\
1745	-9.76600000000008\\
1746	-7.32400000000007\\
1747	-12.2070000000001\\
1748	-9.76600000000008\\
1749	-14.6479999999999\\
1750	-8.54500000000007\\
1751	-13.4280000000001\\
1752	-12.2070000000001\\
1753	-7.32400000000007\\
1754	-10.9860000000001\\
1755	-9.76600000000008\\
1758	-13.4280000000001\\
1759	-7.32400000000007\\
1760	-13.4280000000001\\
1761	-14.6479999999999\\
1762	-14.6479999999999\\
1763	-9.76600000000008\\
1764	-9.76600000000008\\
1765	-10.9860000000001\\
1766	-7.32400000000007\\
1767	-9.76600000000008\\
1768	-7.32400000000007\\
1769	-8.54500000000007\\
1770	-7.32400000000007\\
1771	-13.4280000000001\\
1772	-15.8689999999999\\
1773	-15.8689999999999\\
1774	-14.6479999999999\\
1775	-15.8689999999999\\
1776	-8.54500000000007\\
1779	-4.88300000000004\\
1780	-4.88300000000004\\
1781	-8.54500000000007\\
1783	-13.4280000000001\\
1784	-12.2070000000001\\
1785	-7.32400000000007\\
1786	-12.2070000000001\\
1787	-15.8689999999999\\
1788	-15.8689999999999\\
1789	-13.4280000000001\\
1790	-20.752\\
1791	-17.0899999999999\\
1792	-10.9860000000001\\
1793	-6.10400000000004\\
1794	-7.32400000000007\\
1795	-10.9860000000001\\
1796	-13.4280000000001\\
1797	-17.0899999999999\\
1798	-12.2070000000001\\
1799	-10.9860000000001\\
1800	-6.10400000000004\\
1801	-9.76600000000008\\
1802	-7.32400000000007\\
1803	-7.32400000000007\\
1804	-4.88300000000004\\
1805	-4.88300000000004\\
};
\addlegendentry{True output}

\addplot [color=mycolor2, dashed, line width=2.0pt]
  table[row sep=crcr]{%
1006	-14.7114842952997\\
1007	-17.5445951906447\\
1008	-13.3844042393337\\
1009	-10.4430158982946\\
1010	-10.30741003042\\
1011	-10.7206674570182\\
1012	-12.1487209751747\\
1013	-10.0621907030395\\
1014	-10.0755820678819\\
1015	-13.4811124636426\\
1016	-8.33318843671714\\
1017	-3.09461132545948\\
1018	-15.8059015517833\\
1019	-14.0376657699996\\
1020	-12.9764877487653\\
1021	-10.6526596596684\\
1022	-10.5915758885014\\
1023	-10.6638584165623\\
1024	-9.68090985702725\\
1025	-5.57455780120085\\
1026	-8.90497187049232\\
1027	-10.164210093373\\
1028	-10.0427043193208\\
1029	-13.1006444039972\\
1030	-14.6039607987823\\
1031	-12.1145918726531\\
1032	-14.1471403146795\\
1033	-14.9982847267709\\
1034	-12.0187661174539\\
1035	-10.3211197795545\\
1036	-10.0437555265546\\
1037	-10.1087328681681\\
1038	-11.1455139128275\\
1039	-11.7559829854074\\
1040	-11.8406895501628\\
1041	-12.0814058801309\\
1042	-13.1671195146191\\
1043	-16.2018001807319\\
1044	-14.9835520232664\\
1045	-11.6022217010875\\
1046	-11.087977770592\\
1047	-9.4678466362775\\
1048	-9.73428703208879\\
1049	-8.47618454831013\\
1051	-8.40982396231084\\
1052	-8.17677472207538\\
1053	-10.952664209233\\
1054	-10.7179428526558\\
1055	-14.023555499236\\
1056	-12.6040082463714\\
1057	-11.4630936184083\\
1058	-10.7011844817271\\
1059	-8.65453954934105\\
1060	-9.41959489336318\\
1061	-8.8638621004161\\
1062	-8.14066362072117\\
1063	-6.74590536025039\\
1064	-7.95025648381738\\
1065	-7.96603223401598\\
1066	-6.27493894309328\\
1067	-7.57123872516627\\
1068	-10.4414610852816\\
1069	-11.9105776080692\\
1070	-12.6178279359488\\
1071	-12.1020715772579\\
1072	-13.9429798817778\\
1073	-11.3847188223447\\
1074	-11.755523897288\\
1075	-11.477125150958\\
1076	-11.3549410691037\\
1077	-10.0295803857573\\
1078	-13.6725236346856\\
1079	-24.913684981562\\
1080	-23.0987536333337\\
1081	-20.3350814880378\\
1082	-13.5076660765642\\
1083	-19.4704396377856\\
1084	-25.1955927428253\\
1085	-24.4676995741195\\
1086	-17.765099054217\\
1087	-22.9152670533424\\
1088	-34.8615850084409\\
1089	-21.630802932269\\
1090	-16.6614549145377\\
1091	-13.4114743460564\\
1092	-12.2423215138701\\
1093	-12.7420586669816\\
1094	-12.5429746393431\\
1095	-10.7877831896581\\
1096	-10.2736644451554\\
1097	-8.4652714418055\\
1098	-7.62176940192603\\
1099	-10.571651882331\\
1100	-10.9677346943558\\
1101	-10.5075801227897\\
1102	-12.4074750256666\\
1103	-13.6629943814235\\
1104	-18.0714923536461\\
1105	-14.9042852435352\\
1106	-16.5814307047769\\
1107	-13.9870573239907\\
1108	-13.6135152923325\\
1109	-14.6332367772213\\
1110	-11.8257798403956\\
1111	-11.375235968181\\
1112	-11.3248161196345\\
1113	-11.6589637418977\\
1114	-10.2598221471735\\
1115	-10.0011949772456\\
1116	-9.54884054654235\\
1117	-9.25450171077159\\
1118	-8.56039625690551\\
1119	-8.65464119456396\\
1120	-7.85441344708397\\
1121	-8.24962849883627\\
1122	-12.1321964416286\\
1123	-14.3732010910405\\
1124	-14.3726885788233\\
1125	-9.29692485432906\\
1126	-10.8661474386356\\
1127	-17.8568619809405\\
1128	-14.7358625982474\\
1129	-15.6067763558517\\
1130	-15.0437787712015\\
1131	-11.3837978777592\\
1132	-11.7159510108718\\
1133	-9.83857848469597\\
1134	-12.1780013380437\\
1135	-17.0402769359264\\
1136	-22.2541819031894\\
1137	-17.7838147693712\\
1138	-14.0659646122213\\
1139	-14.9746320955589\\
1140	-15.7219904021258\\
1141	-13.8810278552844\\
1142	-14.4563418739124\\
1143	-11.5795838930148\\
1144	-11.1557710599343\\
1145	-9.65449601221076\\
1146	-9.14364793213053\\
1147	-8.80611691317768\\
1148	-11.9384702840116\\
1149	-18.6475413925073\\
1150	-14.0588999574288\\
1151	-10.8827745591877\\
1152	-10.2000791523651\\
1153	-10.2773003765822\\
1154	-9.66372419726213\\
1155	-9.69634330963072\\
1156	-6.04117439856213\\
1157	-6.89140996021888\\
1158	-7.61165428476284\\
1159	-8.20154110498333\\
1160	-7.95755638462288\\
1161	-8.24039553043372\\
1162	-7.92499894172056\\
1163	-7.98483615513942\\
1164	-7.92809033175627\\
1165	-3.58377830988138\\
1166	-7.87424035878507\\
1167	-18.1523147506136\\
1168	-18.0693494784105\\
1169	-11.2621918142186\\
1170	-11.7065902613165\\
1171	-10.0875178203592\\
1172	-10.2374631150515\\
1173	-5.68453064862024\\
1174	-8.99422144854134\\
1175	-10.7505332182209\\
1176	-14.1107384770664\\
1177	-25.44994408868\\
1178	-29.8940324006737\\
1179	-17.3854810344135\\
1180	-18.4669817666006\\
1181	-13.2458247732091\\
1182	-15.7478663087022\\
1183	-17.9211656023986\\
1184	-18.1686856863951\\
1185	-15.4705152206602\\
1186	-12.8227780991508\\
1187	-13.2253621358807\\
1188	-13.2939752064287\\
1189	-13.3258323151558\\
1190	-10.3000431735222\\
1191	-15.0746717331629\\
1192	-22.0725897571137\\
1193	-13.4446753865179\\
1194	-10.0831735749796\\
1195	-9.97969311271254\\
1196	-16.3725555722299\\
1197	-22.5121596799729\\
1198	-25.0848220365358\\
1199	-23.0718948825383\\
1200	-22.2706281925366\\
1201	-17.0753862074373\\
1202	-15.5530226250469\\
1203	-19.8061404453208\\
1204	-23.155308024928\\
1205	-12.8593910538541\\
1206	-11.8146285189846\\
1207	-12.9471196081624\\
1208	-14.8134375768543\\
1209	-9.78116285830697\\
1210	-10.0691834509901\\
1211	-13.1431640242777\\
1212	-10.2318515485838\\
1213	-8.9223612905937\\
1214	-9.86633101742723\\
1215	-9.43073799367403\\
1216	-10.8249143923131\\
1217	-14.5862019579986\\
1218	-13.6291081856129\\
1219	-11.4012190254498\\
1220	-10.258565967644\\
1221	-13.8525769561941\\
1222	-23.9168681122151\\
1223	-14.689104435316\\
1224	-10.1898596801534\\
1225	-9.92496909217698\\
1226	-10.7889066576038\\
1227	-10.8995779848246\\
1228	-8.72328909777798\\
1229	-8.82561427324367\\
1230	-9.34878072595961\\
1231	-11.6212407727348\\
1232	-11.8230208951959\\
1233	-8.88532370715075\\
1234	-13.8095864776155\\
1235	-17.6213381475452\\
1236	-14.1303001490401\\
1237	-13.2658830780365\\
1238	-13.2209441037367\\
1239	-17.1865608323346\\
1240	-12.6916547839969\\
1241	-10.9957468702096\\
1242	-8.07793311414275\\
1243	-10.1712437132092\\
1244	-11.5326791544626\\
1245	-10.3395633895811\\
1246	-9.19301204425847\\
1247	-10.5805923150674\\
1248	-11.250912846539\\
1249	-9.5237169659099\\
1250	-9.61486549411961\\
1251	-9.13168829316874\\
1252	-9.69441090444138\\
1253	-8.85903671638084\\
1254	-8.32627298856187\\
1255	-8.05940159952024\\
1256	-6.61120265567888\\
1257	-7.25446834924583\\
1258	-7.98995077955351\\
1259	-12.1276758527817\\
1260	-15.2004408227392\\
1261	-20.0021734265608\\
1262	-13.1178980621337\\
1263	-10.3783408785009\\
1264	-8.90967582950202\\
1265	-8.47710378897591\\
1266	-8.32693554035677\\
1267	-7.89999060963578\\
1268	-7.77332651008214\\
1269	-8.68877715524468\\
1270	-12.9024017788693\\
1271	-14.3631410268747\\
1272	-15.004923577173\\
1273	-11.6461527315375\\
1274	-11.2540067929958\\
1275	-10.1836575052057\\
1276	-9.30600076783458\\
1277	-7.9917437088618\\
1278	-6.40943528785579\\
1279	-4.00154262538149\\
1280	-5.28154786235473\\
1281	-9.36656982824047\\
1282	-9.27319900569682\\
1283	-9.78025510074485\\
1284	-15.7176161068057\\
1285	-16.1753925838702\\
1286	-10.524316109382\\
1287	-11.9682874512164\\
1288	-13.1852216577599\\
1289	-16.2645939641102\\
1290	-15.1004495658076\\
1291	-9.35547871197787\\
1292	-11.8807039801586\\
1293	-11.6072497176251\\
1294	-7.08411910581844\\
1295	-6.0471778392955\\
1296	-6.07601783222003\\
1297	-6.54098886889096\\
1298	-5.75272586793267\\
1299	-8.27339373209747\\
1300	-9.58908448495981\\
1301	-8.79502028519687\\
1302	-9.34135203461915\\
1303	-9.50688381200689\\
1304	-10.9550629213923\\
1305	-5.30056425092152\\
1306	-12.1107477419614\\
1307	-12.47323769834\\
1308	-11.724849461124\\
1309	-11.0683242594375\\
1310	-9.23795858581275\\
1311	-9.49009380694565\\
1312	-14.7377083316594\\
1313	-14.688778740476\\
1314	-11.4055649655843\\
1315	-15.0385502380987\\
1316	-14.4401510604337\\
1317	-12.4189519301731\\
1318	-14.8426719721142\\
1319	-14.3577351520576\\
1320	-12.578293390415\\
1321	-11.2419469608187\\
1322	-16.9631434427965\\
1323	-23.9260804073929\\
1324	-15.1429580064755\\
1325	-10.689047357198\\
1326	-11.4496323610192\\
1327	-12.4638634035794\\
1328	-14.0391706749963\\
1329	-15.3633613661355\\
1330	-12.1846276373724\\
1331	-11.4851738806733\\
1332	-15.7600145461718\\
1333	-15.6870608129834\\
1334	-12.8531632422244\\
1335	-9.79217892221232\\
1336	-11.7352164336753\\
1337	-19.2636463584374\\
1338	-18.1430682079044\\
1339	-15.0239529622936\\
1340	-10.5555679643651\\
1341	-10.6854122823083\\
1342	-11.5745881385851\\
1343	-9.30484444068065\\
1344	-10.4021051179084\\
1345	-9.11782561546784\\
1346	-7.68717636119982\\
1347	-3.53405988251029\\
1348	-9.69974660729099\\
1349	-12.3111208230721\\
1350	-9.79790094131158\\
1351	-8.80716177669592\\
1352	-9.79627779179305\\
1353	-8.81464014152289\\
1354	-7.98713679690582\\
1355	-7.60882538748683\\
1356	-9.13723566038993\\
1357	-7.48447130904151\\
1358	-9.97286439925983\\
1359	-14.8546185122866\\
1360	-9.85324475603738\\
1361	-9.81709147446259\\
1362	-8.2455476557036\\
1363	-7.71250704169393\\
1364	-8.04188176315324\\
1365	-7.78804442615933\\
1366	-19.5470175812011\\
1367	-13.3290558087042\\
1368	-14.7187559266595\\
1369	-19.5976567879011\\
1370	-21.1498097796266\\
1371	-15.0959188143624\\
1372	-12.2286643409091\\
1373	-12.2872057002226\\
1374	-12.6472775739692\\
1375	-13.6193319214501\\
1376	-15.424889800594\\
1377	-15.7183559815721\\
1378	-20.5601348975683\\
1379	-24.2497671057956\\
1380	-27.5707017823149\\
1381	-17.7646758552455\\
1382	-17.7130970848671\\
1383	-22.9351530624092\\
1384	-17.91942171347\\
1385	-18.339133506581\\
1386	-17.6398478046558\\
1387	-11.0225482987191\\
1388	-10.2396634369215\\
1389	-9.55617283944503\\
1390	-11.6090860229499\\
1391	-10.0933926931839\\
1392	-8.3075448260729\\
1393	-8.40012993024402\\
1395	-8.03120015860532\\
1396	-6.16350060701166\\
1398	-7.55821053235491\\
1399	-6.30021634077093\\
1400	-12.4281634502515\\
1401	-9.02259809920633\\
1402	-12.4137403837931\\
1403	-24.0736280270257\\
1404	-19.565802735151\\
1405	-23.5524457186796\\
1406	-15.5576823011133\\
1407	-16.5583163577016\\
1408	-13.198124338991\\
1409	-10.9391573203534\\
1410	-10.2178114615297\\
1411	-10.6722690164804\\
1412	-8.98996233721755\\
1413	-7.43169222035817\\
1414	-7.29793435661782\\
1415	-9.36315910556755\\
1416	-4.90463360460808\\
1417	-6.21313557584426\\
1418	-13.0097155358642\\
1419	-13.5294062366102\\
1420	-13.485476632623\\
1421	-12.3020459642885\\
1422	-13.2618005192717\\
1423	-13.39952807945\\
1424	-10.8806202570083\\
1425	-11.583026276892\\
1426	-10.2745095704099\\
1427	-11.0759278658354\\
1428	-10.9767884379323\\
1429	-9.88161669204442\\
1430	-9.69406770089563\\
1431	-14.8643031910717\\
1432	-11.2661035173367\\
1434	-9.49704566027162\\
1435	-9.47608396122905\\
1436	-8.44207728010679\\
1437	-7.49811346076581\\
1438	-7.50547764725366\\
1439	-9.73629486662253\\
1440	-11.385263288432\\
1441	-9.34025551877926\\
1442	-9.94962996055006\\
1443	-11.2227271911979\\
1444	-8.86039102879295\\
1445	-8.23480317389271\\
1446	-11.2046248710531\\
1447	-18.4801050929768\\
1448	-15.7967832218671\\
1449	-12.8079965402503\\
1450	-12.7952207967951\\
1451	-11.6063219104153\\
1452	-10.4592444695102\\
1453	-9.91105356664343\\
1454	-10.9486174725334\\
1455	-11.263328149779\\
1456	-8.75790722089164\\
1457	-8.30814671009261\\
1458	-11.8412999807638\\
1459	-11.6485088545255\\
1460	-13.3765273572349\\
1461	-12.7176122371854\\
1462	-16.8691156386521\\
1463	-23.9213540197536\\
1464	-24.2061216943391\\
1465	-15.2183683672997\\
1466	-11.501336999856\\
1467	-11.7105714978331\\
1468	-10.9971031996388\\
1469	-7.73117433345828\\
1470	-9.07944471154269\\
1471	-10.2912391236721\\
1472	-10.8252940962923\\
1473	-8.58675688036806\\
1474	-9.54829007420244\\
1475	-11.4457830211131\\
1476	-14.119774053537\\
1477	-14.5501506306707\\
1478	-22.5412799957862\\
1479	-19.6967891069246\\
1480	-13.2621999886082\\
1481	-10.8413419705098\\
1482	-12.0627120343181\\
1483	-12.2524582671633\\
1484	-14.4627697578326\\
1485	-10.6085411020754\\
1486	-10.7314721016178\\
1487	-11.5351009890805\\
1488	-9.2900681500787\\
1489	-9.53668525070429\\
1490	-16.2583441663012\\
1491	-11.5057186779854\\
1492	-10.3118024225598\\
1493	-18.4899548915214\\
1494	-16.0480524091729\\
1495	-12.2246459414414\\
1496	-17.6103204399658\\
1497	-18.0495472712018\\
1498	-10.8399972673094\\
1499	-10.201232965698\\
1500	-9.77595964003285\\
1501	-12.3742112682462\\
1502	-23.1664790681371\\
1503	-20.0366458450067\\
1504	-19.8702468716931\\
1505	-15.8768100335669\\
1506	-10.9299370376787\\
1507	-11.1083317749453\\
1508	-9.60792074945971\\
1509	-8.54020600029207\\
1510	-8.30625789249143\\
1511	-7.85216057444836\\
1512	-10.0899156473884\\
1513	-8.0138739686131\\
1514	-7.92993623508232\\
1515	-12.3026010598276\\
1516	-12.8831453559187\\
1517	-17.0454422388859\\
1518	-20.6035000905308\\
1519	-24.9722317896606\\
1520	-13.8420429974369\\
1521	-15.2215731484157\\
1522	-15.6132943270625\\
1523	-14.3874951752646\\
1524	-11.0340025555599\\
1525	-10.5931427626301\\
1526	-19.5441525228366\\
1527	-23.1530378985378\\
1528	-11.1356779767398\\
1529	-10.9260783395575\\
1530	-10.5282015820546\\
1531	-12.4910239730607\\
1532	-7.39080382588986\\
1533	-1.00236078963758\\
1534	-5.67485906800289\\
1535	-7.17152480577784\\
1536	-7.83084421641911\\
1537	-7.70372309670006\\
1538	-7.00684734997185\\
1539	-6.80222386728406\\
1540	-7.40947253743957\\
1541	-6.59229179197791\\
1542	-7.54093794860887\\
1543	-13.0368406107452\\
1544	-14.5005704771706\\
1545	-13.2197833834489\\
1546	-14.1785653158431\\
1547	-19.4698930058719\\
1548	-19.2304223872327\\
1549	-22.2923103431808\\
1550	-20.4798029501635\\
1551	-15.5131863857907\\
1552	-12.7856540131727\\
1553	-11.9937133988183\\
1554	-18.150398722034\\
1555	-21.8424986312727\\
1556	-13.8289281869868\\
1557	-10.5542008444577\\
1558	-10.462756447766\\
1559	-11.844122658079\\
1560	-7.10557144533323\\
1561	-8.73091391778075\\
1562	-2.90233958139538\\
1563	-7.10370433281219\\
1564	-8.50056187818222\\
1565	-8.03177162031693\\
1566	-8.80013191483431\\
1567	-9.19061515305589\\
1568	-5.02814475944911\\
1569	-6.00563102518595\\
1570	-5.90925192392206\\
1571	-6.75166559601098\\
1572	-6.17721776254507\\
1573	-3.65813552647273\\
1574	-12.5958746757019\\
1575	-18.699697019533\\
1576	-17.3352955926571\\
1577	-12.4289207380953\\
1578	-15.5673074048561\\
1579	-27.7511635677463\\
1581	-21.2297407852809\\
1582	-16.3745621775402\\
1583	-17.1294911087859\\
1584	-16.9684605509474\\
1585	-12.3488218971111\\
1586	-12.6479396155241\\
1587	-9.37096328597477\\
1588	-9.54757588454913\\
1589	-12.1284639448716\\
1590	-10.6675953024412\\
1591	-10.3197304185078\\
1592	-12.5913492185186\\
1593	-11.8957509272386\\
1594	-10.5399431711573\\
1595	-10.6569397654309\\
1596	-12.3133965410368\\
1597	-12.4926522319765\\
1598	-12.0600324025697\\
1599	-10.413450464504\\
1600	-10.7379753490497\\
1601	-9.86607781102566\\
1602	-14.0354286704248\\
1603	-5.71898022444634\\
1604	-6.45859358381676\\
1605	-7.62494976227572\\
1606	-12.4620152831747\\
1607	-1.40429614473533\\
1608	-3.14089006077393\\
1609	-6.81404042341251\\
1610	-7.77295925993258\\
1611	-8.32573692370602\\
1612	-10.387601076221\\
1613	-12.8802486690652\\
1614	-9.0140718825603\\
1615	-8.96408387090514\\
1617	-9.97615224730998\\
1618	-8.11998540932041\\
1619	-9.78425132067559\\
1620	-3.1815657339871\\
1621	-5.03601639594353\\
1622	-4.12711821578569\\
1623	-7.8089169628297\\
1624	-9.68371080255793\\
1625	-8.58733094504851\\
1626	-7.41587230921687\\
1627	-8.16982306854857\\
1628	-7.6968789110615\\
1629	-7.84913139982109\\
1630	-7.58142136033234\\
1631	-5.96010103558797\\
1632	-7.96462689558257\\
1633	-5.04243929666495\\
1634	-5.83686078820369\\
1635	-5.95365387437846\\
1636	-7.71822246052193\\
1637	-7.49904314880723\\
1638	-6.58430578125785\\
1639	-6.59119784507743\\
1640	-5.89084793606639\\
1641	-8.01144854702784\\
1642	-17.2039593609938\\
1643	-10.573416212334\\
1644	-9.57590246962991\\
1645	-10.8368360157222\\
1646	-10.2756113472185\\
1647	-12.2900167299433\\
1648	-9.21354801080838\\
1649	-8.70602404269289\\
1650	-9.15212533404952\\
1651	-9.2283208996746\\
1652	-7.04507454177065\\
1653	-7.86121310228464\\
1654	-8.33177716799355\\
1655	-8.94313819091462\\
1656	-10.0755150850657\\
1657	-8.59768066896254\\
1658	-8.94999939987815\\
1659	-10.0904898024205\\
1660	-11.0253671513651\\
1661	-12.4197094183542\\
1662	-20.9741041863274\\
1663	-15.3520679857709\\
1664	-10.4066847208085\\
1665	-10.856858902918\\
1666	-12.1584836498046\\
1667	-15.9970468181511\\
1668	-11.8223919016523\\
1669	-12.0241500792004\\
1670	-10.5911648532453\\
1671	-12.865835372799\\
1672	-8.35349618715441\\
1673	-7.3421706036338\\
1674	-11.8897803162683\\
1675	-9.91428843457948\\
1676	-11.9448693906204\\
1677	-13.039074930794\\
1678	-11.4368769825153\\
1679	-9.9748826510604\\
1680	-9.69710381112759\\
1681	-12.487881745569\\
1682	-11.2325878022996\\
1683	-10.2297079098403\\
1684	-10.9290884055949\\
1685	-9.95992929696763\\
1686	-8.61867064373564\\
1687	-8.98120759268227\\
1688	-7.65719917262345\\
1689	-10.7821927031416\\
1690	-13.2879434231415\\
1691	-14.0814257607481\\
1692	-13.0108372225327\\
1693	-11.6750664672945\\
1694	-10.4215829214065\\
1695	-8.77427843023884\\
1696	-9.66262125058597\\
1697	-10.8018967591909\\
1698	-14.2881614533258\\
1699	-14.964828116269\\
1700	-12.7950410693645\\
1701	-10.1894843911957\\
1702	-11.7213719359734\\
1703	-19.9426387973301\\
1704	-13.2377411890004\\
1705	-12.5252659696687\\
1706	-15.7812177868539\\
1707	-13.4755806803585\\
1708	-10.9948294606349\\
1709	-11.5468459896715\\
1710	-12.0110726161518\\
1711	-11.9000652281438\\
1712	-13.95265295124\\
1713	-12.2715944498093\\
1714	-12.0883647758578\\
1715	-12.2584952590062\\
1716	-12.3692444330959\\
1717	-11.00787906391\\
1718	-11.2868630828521\\
1719	-9.4147215485923\\
1720	-9.52625952828271\\
1721	-10.7953119197255\\
1722	-10.914854060559\\
1723	-8.98584099676759\\
1724	-11.8529871066112\\
1725	-10.4008347242584\\
1726	-3.56917117215426\\
1727	-9.58474822451876\\
1728	-17.6555606166032\\
1729	-12.9156160086343\\
1730	-10.2167010802812\\
1731	-10.5121379786501\\
1732	-11.478312493851\\
1733	-9.78422328877264\\
1734	-9.75699365363266\\
1735	-8.27788720256854\\
1736	-8.94591383777743\\
1737	-8.77102218441269\\
1739	-8.58849441183952\\
1741	-8.07288143171809\\
1742	-7.05900125091193\\
1743	-9.83023347272069\\
1744	-9.45554969218165\\
1745	-9.17816248384588\\
1746	-9.75106240207379\\
1747	-10.6622124665098\\
1748	-12.2844854518059\\
1749	-13.0380419050182\\
1750	-12.4950971505846\\
1751	-11.7549098039731\\
1752	-13.0875065363684\\
1753	-10.3369944972742\\
1754	-10.0736405981329\\
1755	-10.5725103430395\\
1756	-10.7498307994424\\
1757	-11.9155140739756\\
1758	-14.4725963726698\\
1759	-12.1416403421651\\
1760	-11.4689293083318\\
1761	-16.4129496417568\\
1762	-13.446300196631\\
1763	-12.3001698337748\\
1764	-11.191811198335\\
1765	-11.7362784796387\\
1766	-11.5847182302746\\
1767	-9.01995349249637\\
1768	-9.64123420179249\\
1769	-9.21546397637735\\
1770	-9.36659498760196\\
1771	-13.4893339663793\\
1772	-20.0199252698678\\
1773	-14.508754466804\\
1774	-15.2033276619277\\
1775	-14.8164749410371\\
1776	-11.877366799911\\
1777	-10.6431299509129\\
1778	-9.35682147946955\\
1779	-8.62858440142509\\
1780	-7.20123957612032\\
1781	-9.57352446071377\\
1782	-11.6917584013263\\
1783	-12.9708695128259\\
1784	-11.7505486811397\\
1785	-10.479264575167\\
1786	-12.9611069635696\\
1787	-17.4971316462847\\
1788	-18.1551512201004\\
1789	-13.3847648712792\\
1790	-22.0893319364932\\
1791	-18.3380437028902\\
1792	-10.5738552358675\\
1793	-11.3163592088983\\
1794	-9.65530186360206\\
1795	-10.8868794635484\\
1796	-15.1925713489195\\
1797	-18.3813377020722\\
1798	-13.961171084127\\
1799	-11.7432534750226\\
1800	-10.1145897670915\\
1801	-9.44727283285488\\
1802	-9.62204157640167\\
1803	-9.53895757051123\\
1804	-8.1186219422068\\
1805	-7.51177687301083\\
};
\addlegendentry{OSA predition}

\addplot [color=mycolor3, dotted, line width=2.0pt]
  table[row sep=crcr]{%
1006	-15.8689999999999\\
1007	-14.6479999999999\\
1008	-14.6479999999999\\
1009	-7.32400000000007\\
1010	-10.30741003042\\
1011	-11.3687055459736\\
1012	-12.5100303394918\\
1013	-11.2577265542154\\
1014	-11.1480847942689\\
1015	-15.5885485736151\\
1016	-14.077448355798\\
1017	-9.03250945447508\\
1018	-21.6958739624845\\
1019	-20.9798599539563\\
1020	-18.6428745810604\\
1021	-15.3568481526497\\
1022	-14.2764926125994\\
1023	-15.5243201075639\\
1024	-15.1414818114249\\
1025	-12.8113625580695\\
1026	-14.7915619135592\\
1027	-14.4770114118974\\
1028	-13.2438710916424\\
1029	-16.1408885209071\\
1030	-16.3246309543601\\
1031	-13.5226372316681\\
1032	-16.3402023760539\\
1033	-16.8487596529096\\
1034	-14.5795700797005\\
1035	-13.2531649238767\\
1036	-12.4976378213091\\
1037	-12.7859332926062\\
1038	-13.9590540655975\\
1039	-13.6004331277595\\
1040	-13.7772123888276\\
1041	-13.9940569323455\\
1042	-15.033756320468\\
1043	-18.6269458861132\\
1044	-17.5350437938844\\
1045	-13.31313200164\\
1046	-13.2611462955479\\
1047	-12.1061415556487\\
1048	-11.975595482643\\
1049	-11.7406042864632\\
1051	-11.231944327026\\
1052	-11.9543964202983\\
1053	-13.331226766237\\
1054	-12.8203678154944\\
1055	-16.7399887693823\\
1056	-14.7784889065595\\
1057	-11.9485627742024\\
1058	-12.3602125956472\\
1059	-11.6378003167447\\
1060	-12.3718029436466\\
1061	-11.1875028827019\\
1062	-11.2257882071553\\
1063	-10.4087622534751\\
1064	-10.4160245173659\\
1065	-10.3751696378818\\
1066	-9.93867751883477\\
1067	-10.3017900253506\\
1068	-12.5241645468043\\
1069	-13.7265564728418\\
1070	-14.3624442976377\\
1071	-14.4426940215074\\
1072	-15.2968748385097\\
1073	-13.7401656154852\\
1074	-13.3816059380613\\
1075	-12.960329940744\\
1076	-12.9111768871294\\
1077	-12.658047023825\\
1078	-16.3131430966762\\
1079	-27.0135506754914\\
1080	-26.7278450847589\\
1081	-23.9337969384367\\
1082	-16.5744674335383\\
1083	-22.4239324577272\\
1084	-29.1267752276465\\
1085	-27.8916909222446\\
1086	-21.1015983350851\\
1087	-26.6483807603274\\
1088	-38.4123338858692\\
1089	-27.1454002139949\\
1090	-20.4934662979852\\
1091	-16.30480233656\\
1092	-14.5144921425331\\
1093	-13.9296642051572\\
1094	-14.6646108667967\\
1095	-12.919552447888\\
1096	-11.3073098119964\\
1097	-11.4789242505863\\
1098	-11.2588659942664\\
1099	-13.4803383938704\\
1100	-13.6058514432732\\
1101	-13.2162263041621\\
1102	-14.7685677908339\\
1103	-16.1046890388839\\
1104	-20.3079509100151\\
1105	-18.0520822112776\\
1106	-18.7527635092672\\
1107	-16.5261399002582\\
1108	-14.6473159050106\\
1109	-16.0068338691804\\
1110	-13.8966511856017\\
1111	-13.0457676555131\\
1112	-13.885210270254\\
1113	-14.3533250879691\\
1114	-12.1567424479472\\
1115	-12.3304729497329\\
1116	-11.4661788312835\\
1117	-11.5461758022002\\
1118	-11.6046135956028\\
1119	-10.9422045010156\\
1120	-10.7795625864821\\
1121	-11.7819582158781\\
1122	-14.8312256408101\\
1123	-16.4069353472403\\
1124	-17.0166779623767\\
1125	-12.44794866296\\
1126	-13.9986158745908\\
1127	-21.6075139966181\\
1128	-17.2905377419352\\
1129	-17.9236358417988\\
1130	-18.6658473652953\\
1131	-13.2327321672894\\
1132	-12.8742474008302\\
1133	-13.2326137799239\\
1134	-14.545444300426\\
1135	-20.6095590734935\\
1136	-26.3184740834856\\
1137	-23.4746689416918\\
1138	-18.0155898532719\\
1139	-18.1899301251474\\
1140	-18.8157509100583\\
1141	-16.9878653936105\\
1142	-17.1096967880558\\
1143	-13.6502015391989\\
1144	-13.1981193300592\\
1145	-12.6490517011414\\
1146	-12.249892944222\\
1147	-12.4229010169313\\
1148	-15.211243296154\\
1149	-21.9776345785012\\
1150	-17.8469390358271\\
1151	-13.5234347736932\\
1152	-13.2086021859861\\
1153	-13.4214625899558\\
1154	-12.1290332439587\\
1155	-13.0001598154543\\
1156	-11.2728935222019\\
1157	-11.3459413858859\\
1158	-11.2533888258843\\
1159	-11.1914083094871\\
1160	-11.079860564932\\
1161	-10.8011336368179\\
1162	-10.6712229024154\\
1163	-11.3208801940998\\
1164	-11.5146159719798\\
1165	-8.9038411554227\\
1166	-12.2285827545022\\
1167	-20.8221754999727\\
1168	-22.1022185357945\\
1169	-14.7942720956203\\
1170	-13.8216250888288\\
1171	-13.6350791548025\\
1172	-14.4921807195224\\
1173	-11.303912009608\\
1174	-12.9986109219385\\
1175	-14.7458327920112\\
1176	-17.9785691836544\\
1177	-29.2371559196695\\
1178	-35.8629337467751\\
1179	-26.4645754428425\\
1180	-25.1192663253137\\
1181	-19.0104567223791\\
1183	-21.1614544708684\\
1184	-21.1923645468596\\
1185	-17.7433636166959\\
1186	-15.9606321117099\\
1187	-16.414497490571\\
1188	-16.2262437119398\\
1189	-16.7856232660804\\
1190	-13.4456579302309\\
1191	-17.9614078122881\\
1192	-24.3741587406644\\
1193	-16.7500982940742\\
1194	-13.0355509730398\\
1195	-13.2203233958712\\
1196	-19.4281328529758\\
1197	-24.9682778164226\\
1198	-28.8815005084866\\
1199	-28.4722832231796\\
1200	-27.1079908584632\\
1201	-21.9506977947049\\
1202	-19.7748523972596\\
1203	-23.150767852055\\
1204	-26.6273186428989\\
1205	-16.4690004327065\\
1206	-13.0198636786317\\
1207	-15.3600916837645\\
1208	-16.1978068352494\\
1209	-11.5326164166443\\
1210	-12.6173995767113\\
1211	-14.6829408560898\\
1212	-12.9670209307631\\
1213	-11.6762485599409\\
1214	-12.5597853096649\\
1215	-12.245460502972\\
1216	-13.5038672055234\\
1217	-17.2732622103238\\
1218	-16.2736924417804\\
1219	-13.1330173719641\\
1220	-13.241452773108\\
1221	-16.8492625589267\\
1222	-27.1690010201924\\
1223	-20.085128178712\\
1224	-13.828727445971\\
1225	-13.3346689691382\\
1226	-14.4832199302421\\
1227	-14.0301261121929\\
1228	-12.1331038881522\\
1229	-12.0554844959286\\
1230	-13.0237435222355\\
1231	-14.9936150113811\\
1232	-15.3544285349865\\
1233	-12.4944575967565\\
1234	-16.4945990526035\\
1235	-20.6749472682627\\
1236	-17.7114718368996\\
1237	-15.3482402249308\\
1238	-15.9778976728085\\
1239	-19.4576831402533\\
1240	-14.4128227503209\\
1241	-12.6202807439131\\
1242	-12.013281220291\\
1243	-12.5056324266202\\
1244	-14.327266382985\\
1245	-13.5583270822324\\
1246	-12.0919936101986\\
1247	-13.2827440833628\\
1248	-14.3153369406446\\
1249	-12.066886003214\\
1250	-12.4901670174531\\
1251	-12.3189983000386\\
1252	-11.9288381900333\\
1253	-12.1796433603445\\
1254	-11.0198275194509\\
1255	-10.9410712257356\\
1256	-10.6416229643455\\
1257	-10.439740140506\\
1258	-11.7291068965251\\
1259	-14.2260246308572\\
1260	-17.1529971150021\\
1261	-22.699284344566\\
1262	-15.6091346668825\\
1263	-13.1135360528551\\
1264	-13.1714056769792\\
1265	-12.7631414795817\\
1266	-12.7032488862512\\
1267	-11.4533340471276\\
1268	-11.3419122278979\\
1269	-11.3231835457491\\
1270	-15.6966561752949\\
1271	-16.4985596561257\\
1272	-18.506142426965\\
1273	-14.4808258377063\\
1274	-13.7414781470875\\
1275	-12.8548495226955\\
1276	-12.7941438915275\\
1277	-12.6354897804044\\
1278	-11.4788635846512\\
1279	-10.0234522496078\\
1280	-10.9383577911933\\
1281	-13.0382107897517\\
1282	-12.9621521719882\\
1283	-13.3441258832445\\
1284	-18.4793478338461\\
1285	-18.6483128295843\\
1286	-14.528415582123\\
1287	-15.2618491104568\\
1288	-16.7470773629998\\
1289	-19.4313520910739\\
1290	-18.3047549525993\\
1291	-12.9593378117281\\
1292	-13.4111187918445\\
1293	-14.9314215264553\\
1294	-13.2059338508557\\
1295	-11.0655414217383\\
1296	-10.846532246959\\
1297	-10.7572640236133\\
1298	-10.1791108782959\\
1299	-12.1052471747009\\
1300	-12.1897973193527\\
1301	-12.0302136804989\\
1302	-12.3666861447075\\
1303	-11.1086040002858\\
1304	-13.1388657365599\\
1305	-9.33892413288231\\
1306	-13.4288774996448\\
1307	-14.1396570759232\\
1308	-14.2258105011745\\
1309	-12.8472592612495\\
1310	-12.0705873561108\\
1311	-12.594185701988\\
1312	-16.8053813902518\\
1313	-16.7226538515779\\
1314	-13.4745319593362\\
1315	-17.6506385908704\\
1316	-16.6835807641942\\
1317	-14.7441306237856\\
1318	-16.4985787855003\\
1319	-16.2328662567309\\
1320	-14.4215643271716\\
1321	-13.2039686034734\\
1322	-18.8140802332189\\
1323	-27.3662416080874\\
1324	-19.8897642683964\\
1325	-13.5574244429663\\
1326	-13.8930732055551\\
1327	-14.9270994373098\\
1328	-16.3594039650927\\
1329	-17.4813839701526\\
1330	-14.6660536319789\\
1331	-14.4223072041891\\
1332	-17.8864287234057\\
1333	-18.042353363154\\
1334	-15.1819630213511\\
1335	-12.6292263536277\\
1336	-14.5215665075505\\
1337	-22.0331031014937\\
1338	-20.9609723037634\\
1339	-18.5934720669038\\
1340	-13.8778012169753\\
1341	-13.5857368850916\\
1342	-14.0162355738933\\
1343	-11.8812759558366\\
1344	-13.5286943667522\\
1345	-13.3678833396527\\
1346	-12.8674226426383\\
1347	-9.20885607479408\\
1348	-13.2871610705149\\
1349	-15.398605575803\\
1350	-13.1986838869291\\
1351	-12.4506230076972\\
1352	-13.2381546685147\\
1353	-11.6747881266076\\
1354	-11.4560979224211\\
1355	-10.9622439389345\\
1356	-10.8680934164713\\
1357	-10.2926239930916\\
1358	-12.2535109633625\\
1359	-15.8808819345645\\
1360	-11.9241166825941\\
1361	-11.7903270134759\\
1362	-11.3828146382489\\
1363	-11.0468472790478\\
1364	-10.742931108186\\
1365	-11.0916555486274\\
1366	-22.4998080039911\\
1367	-17.5775479382503\\
1368	-19.4493479312641\\
1369	-22.7314175827446\\
1370	-23.8676739205282\\
1371	-18.3716085724941\\
1372	-15.143100021204\\
1373	-15.0154719670529\\
1374	-15.5708602018469\\
1375	-16.2139733278245\\
1376	-17.681040158187\\
1377	-17.9170879430808\\
1378	-22.356705549332\\
1379	-25.7542558511004\\
1380	-30.1954714508577\\
1381	-20.9199890452785\\
1382	-20.9917436895346\\
1383	-24.7897734339183\\
1384	-20.4571156995262\\
1385	-21.5352104910537\\
1386	-20.4700243193845\\
1387	-13.1555693317223\\
1388	-12.4580533598264\\
1389	-12.4097651922293\\
1390	-14.1575199133654\\
1391	-12.9206241989873\\
1392	-11.6418966634869\\
1393	-11.8277357389516\\
1394	-11.3287149829573\\
1395	-11.3283148665519\\
1396	-10.4296712168482\\
1397	-10.6176448754425\\
1398	-10.4312986521636\\
1399	-10.8587084851556\\
1400	-14.200815463146\\
1401	-12.1363214236994\\
1402	-15.6125344305697\\
1403	-25.3289071985005\\
1404	-22.8441447188347\\
1405	-25.9300767436985\\
1406	-18.8625219836904\\
1407	-19.8294234833834\\
1408	-15.512473548968\\
1409	-13.3085247207819\\
1410	-13.4252295602932\\
1411	-12.7989849693224\\
1412	-11.960673189305\\
1413	-11.9169886898228\\
1414	-10.9664053549025\\
1415	-13.107528091379\\
1416	-10.6132561916377\\
1417	-10.6575116294198\\
1418	-15.3559520411859\\
1419	-17.3003833302046\\
1420	-16.9769856973646\\
1421	-15.7886408713618\\
1422	-16.4633568541262\\
1423	-15.6418036866066\\
1424	-13.6537974034532\\
1425	-13.6329438879716\\
1426	-12.9703464548784\\
1427	-14.4722464965357\\
1428	-12.6987512653957\\
1429	-11.9416456095435\\
1430	-12.8363807584574\\
1431	-16.6665609325771\\
1432	-13.4914461086696\\
1433	-12.4054347598517\\
1434	-12.0613377922239\\
1435	-12.3380275682146\\
1436	-11.4149140803574\\
1437	-11.257965603751\\
1438	-11.0298825831519\\
1439	-12.7877296552945\\
1440	-13.6313208665197\\
1441	-12.2303865268398\\
1442	-12.8863732255209\\
1443	-13.1206444941647\\
1444	-11.5829787552682\\
1445	-11.3940096342546\\
1446	-14.0038469792771\\
1447	-20.2936270948835\\
1448	-18.973423031076\\
1449	-15.9936597911808\\
1450	-15.4188054122249\\
1451	-14.1993635452836\\
1452	-13.7456854700361\\
1453	-12.6568417128397\\
1454	-13.7345342742478\\
1455	-13.1936065776983\\
1456	-11.7412227363784\\
1457	-12.0720977436172\\
1458	-14.20910455376\\
1459	-14.8136091164438\\
1460	-16.2311034790494\\
1461	-15.3150498238119\\
1462	-19.8361733308077\\
1463	-26.3785686277383\\
1464	-27.6034026051277\\
1465	-18.2587567402866\\
1466	-13.5030950016424\\
1467	-14.0575758117516\\
1468	-14.8779709221867\\
1469	-12.3794558957361\\
1470	-11.9713717583663\\
1471	-14.4264912430547\\
1472	-14.3242413312355\\
1473	-12.9122173012649\\
1474	-13.5786662350899\\
1475	-15.2073241306687\\
1476	-17.6956132473429\\
1477	-17.7183153638175\\
1478	-25.2568373411359\\
1479	-23.1078278561235\\
1480	-17.8171809797975\\
1481	-14.217644713465\\
1482	-14.908175844801\\
1483	-15.2476805183153\\
1484	-17.263821303231\\
1485	-13.6238439851863\\
1486	-13.4938344212712\\
1487	-13.7392908906074\\
1488	-11.7563532852871\\
1489	-12.2288560305217\\
1490	-17.4767741271398\\
1491	-13.7913439756576\\
1492	-13.1660602632473\\
1493	-20.8863238559607\\
1494	-18.3641278006012\\
1495	-15.6469831393235\\
1496	-20.2802836076924\\
1497	-20.5545232149602\\
1498	-14.2852084301737\\
1499	-12.6632437686424\\
1500	-12.7583429540744\\
1501	-15.9176197675481\\
1502	-26.2517034240066\\
1503	-25.1215473939758\\
1504	-23.6255154117573\\
1505	-19.6291942293881\\
1506	-15.0700917540403\\
1507	-14.5973269788917\\
1508	-13.3119402211469\\
1509	-12.8407312196068\\
1510	-12.08209347306\\
1511	-12.2982819806759\\
1512	-13.5279422094111\\
1513	-11.2801606055534\\
1514	-11.4292635634438\\
1515	-14.3779895093153\\
1516	-15.3548528396013\\
1518	-23.305365047541\\
1519	-28.6189844993964\\
1520	-18.7604104235706\\
1521	-17.5673200935087\\
1522	-18.5155414181593\\
1523	-17.2743566701358\\
1524	-13.7118384851356\\
1525	-14.2817203835039\\
1526	-21.6227796680744\\
1527	-25.3373445047109\\
1528	-14.677329345948\\
1529	-12.9357984125129\\
1530	-13.706036404272\\
1531	-17.3036902201052\\
1532	-14.7729919998869\\
1533	-10.4380791167891\\
1534	-11.8766087514477\\
1535	-13.2321655971164\\
1537	-11.4748062254496\\
1538	-11.0145621986283\\
1539	-10.6709331047809\\
1540	-10.3829071985936\\
1541	-10.0672246645322\\
1542	-10.9790147718961\\
1543	-15.4937908842244\\
1544	-17.1520939786221\\
1545	-15.7416704874975\\
1546	-16.9638552007218\\
1547	-21.6731375780842\\
1548	-22.1257083794758\\
1549	-25.485604349373\\
1550	-23.6239554097217\\
1551	-18.0827432892559\\
1552	-15.2221926159816\\
1553	-14.6358568614266\\
1554	-20.6786846570624\\
1555	-24.0157733811088\\
1556	-17.0016498191048\\
1557	-13.5724853382894\\
1558	-13.2809953323085\\
1559	-16.1772862171701\\
1560	-12.9338129154364\\
1561	-13.738193550299\\
1562	-9.56059965351665\\
1563	-11.4448728581899\\
1564	-11.8051371469869\\
1565	-11.3886939301051\\
1566	-11.2958531584998\\
1567	-12.0322484185524\\
1568	-9.3662866194029\\
1569	-9.32004274380211\\
1570	-9.17807560852566\\
1571	-9.47000038599754\\
1572	-9.77920793248836\\
1573	-7.79899846642388\\
1574	-14.7810293742295\\
1575	-22.3685684991656\\
1576	-21.9625447527549\\
1577	-16.3038298928218\\
1578	-19.7247112976975\\
1579	-30.6346156847248\\
1580	-28.7230111930226\\
1581	-25.3664325407321\\
1582	-20.2312116034968\\
1583	-21.2938864638863\\
1584	-21.3490414311555\\
1585	-15.7704998010076\\
1586	-15.0758367568055\\
1587	-12.4592807299648\\
1588	-12.6419681658679\\
1589	-15.4558721116946\\
1590	-12.9938818426522\\
1591	-13.1753162751036\\
1592	-14.1949492847821\\
1593	-13.7642255960347\\
1594	-13.3676679556838\\
1595	-13.4385635490687\\
1596	-14.5972584321305\\
1597	-15.5401768575766\\
1598	-14.5742004341591\\
1599	-12.7369786528507\\
1600	-13.4605772526804\\
1601	-11.3486492672866\\
1602	-16.5083044082608\\
1603	-11.986254579783\\
1604	-10.3281692459734\\
1605	-10.5796539353335\\
1606	-17.2610001052349\\
1607	-8.81892553161947\\
1608	-8.18884636601251\\
1609	-10.6950420101937\\
1610	-11.9181485805404\\
1611	-10.8762283401529\\
1612	-12.7742601761845\\
1613	-13.836247133029\\
1614	-11.4685991504887\\
1615	-11.8665070071279\\
1616	-10.6584608227813\\
1617	-12.0564749860591\\
1618	-11.7089777835329\\
1619	-13.3047274579897\\
1620	-8.67460760831159\\
1621	-9.15429796524904\\
1622	-8.23415598223119\\
1623	-10.7739619389429\\
1624	-11.3406022881181\\
1625	-10.4298861139328\\
1626	-10.056629539087\\
1627	-10.0851548082808\\
1628	-9.70085698492517\\
1629	-9.74429662691409\\
1630	-9.63132100942858\\
1631	-9.42886846162855\\
1632	-9.96897567161432\\
1633	-8.41363476262882\\
1634	-8.66675209715936\\
1635	-8.51251764454287\\
1636	-9.48964021887127\\
1637	-9.48129775139455\\
1638	-9.10449976346626\\
1639	-9.4095646527478\\
1640	-8.77882699289989\\
1641	-10.9977936484761\\
1642	-17.8635793569745\\
1643	-12.7802476883703\\
1644	-12.2875994677411\\
1645	-11.534062018301\\
1646	-12.7323319922978\\
1647	-13.5272470176142\\
1648	-11.4003590536022\\
1649	-11.2546843504856\\
1650	-10.9966905578112\\
1651	-11.7289691076419\\
1652	-10.5544372057432\\
1653	-10.7816668391408\\
1654	-10.0595134590487\\
1655	-11.1020074802016\\
1656	-10.4889882609016\\
1657	-10.3818562740867\\
1658	-10.3302928223209\\
1659	-12.1788923435461\\
1660	-12.099549353243\\
1661	-14.8226627588394\\
1662	-22.0198161462495\\
1663	-17.7219476148916\\
1664	-13.0738237475387\\
1665	-12.8785583194394\\
1666	-14.5320060985164\\
1667	-17.9912624399587\\
1668	-14.3891468971074\\
1669	-15.2598001234223\\
1670	-12.5190074073132\\
1671	-14.2334455941709\\
1672	-12.6063126588926\\
1673	-12.0834608226373\\
1674	-13.9342375104525\\
1675	-12.6896651659179\\
1676	-14.6696172790707\\
1677	-15.1677485533776\\
1678	-13.6196631794987\\
1679	-12.401237680245\\
1680	-12.6131454510178\\
1681	-14.8023231548193\\
1682	-13.3392199209225\\
1684	-12.7501333411467\\
1685	-11.8641716029904\\
1686	-11.7063253026179\\
1687	-11.2279773685591\\
1688	-11.0348695233017\\
1689	-13.3265353371412\\
1690	-15.4206525971358\\
1691	-15.5268449015923\\
1692	-14.3316626548865\\
1693	-12.5475293000429\\
1694	-12.1854025142982\\
1695	-11.769525321735\\
1696	-12.4399844829263\\
1697	-13.7282511828614\\
1698	-16.3909830309442\\
1699	-17.5511801323617\\
1700	-14.894032760049\\
1701	-12.3647126538622\\
1702	-14.2784573128056\\
1703	-21.9141860114362\\
1704	-15.5771073291444\\
1705	-14.7667514109971\\
1706	-16.7811318749939\\
1707	-14.8876637072165\\
1708	-13.4828200579536\\
1709	-13.741739753581\\
1710	-13.6407918197235\\
1711	-14.3352242950155\\
1712	-16.1784662368493\\
1713	-13.7241436432066\\
1714	-14.4051355702925\\
1715	-14.0155902302743\\
1716	-14.2424611165052\\
1717	-13.1359861915359\\
1718	-14.7683726157784\\
1719	-12.6698104375766\\
1720	-12.4997321546059\\
1721	-13.4025768208007\\
1722	-13.4150758668134\\
1723	-11.3105381683777\\
1724	-14.5588301607395\\
1725	-15.4445149309174\\
1726	-10.2332810226028\\
1727	-14.3367147989829\\
1728	-22.24201153938\\
1729	-18.7647244713448\\
1730	-14.9592883881912\\
1731	-14.9360714975946\\
1732	-14.8170734605969\\
1733	-13.6209113112436\\
1734	-12.6981963197884\\
1735	-11.8303104641341\\
1736	-11.8197781251249\\
1737	-11.6873075752749\\
1738	-11.7136016218844\\
1739	-11.3728630807691\\
1740	-11.0908544973709\\
1741	-11.0359511971515\\
1742	-10.1427383065034\\
1743	-13.2526566356205\\
1744	-11.8222301452774\\
1745	-12.0649113055847\\
1746	-11.8239993383581\\
1747	-13.2258713467429\\
1748	-13.8521590920311\\
1749	-15.3509678180724\\
1750	-13.7791757810919\\
1751	-14.2299965265986\\
1752	-14.4957632706055\\
1753	-11.6247137778096\\
1754	-12.5122491237148\\
1755	-11.9038138308822\\
1756	-12.2965779901381\\
1757	-13.2079899298828\\
1758	-15.3645898569507\\
1759	-13.2965562559621\\
1760	-14.1224325470732\\
1761	-17.7187769721538\\
1762	-15.4089453403544\\
1763	-13.5972878892376\\
1764	-12.9703487966922\\
1765	-13.75805965009\\
1766	-13.3771846076245\\
1767	-12.1829985626291\\
1768	-11.7707550982007\\
1769	-11.8353985752333\\
1770	-11.8836819958726\\
1771	-16.2933398442017\\
1772	-22.4836405321123\\
1773	-18.1324131731035\\
1774	-17.6018136202824\\
1775	-17.0462974883465\\
1776	-13.4669705115498\\
1777	-12.9663410373587\\
1778	-12.4095794392003\\
1779	-12.0050696819349\\
1780	-11.4352035794489\\
1781	-13.9963439016524\\
1782	-15.8453159497446\\
1783	-16.8370423428498\\
1784	-14.7329995803332\\
1785	-12.6719197646132\\
1786	-16.0482970855112\\
1787	-20.1875738359288\\
1788	-21.122040514766\\
1789	-16.7479348351562\\
1790	-24.9004069110415\\
1791	-21.2587039957639\\
1792	-13.3681566156745\\
1793	-13.2305007114514\\
1794	-13.1395538318741\\
1795	-14.5298592635963\\
1796	-18.2047863048538\\
1797	-22.0053564878131\\
1798	-17.2869470683825\\
1799	-14.9625182860179\\
1800	-12.8760481478525\\
1801	-13.1359713191466\\
1802	-12.4468343361032\\
1803	-12.7406826269319\\
1804	-11.5895261024052\\
1805	-11.3062682043735\\
};
\addlegendentry{MPO prediction}

\end{axis}

\begin{axis}[%
width=6.159cm,
height=1.831cm,
at={(8.104cm,5.085cm)},
scale only axis,
xmin=1000,
xmax=2000,
xlabel style={font=\color{white!15!black}},
xlabel={Sample index},
ymin=-404.053,
ymax=3.71940897576542,
ylabel style={font=\color{white!15!black}},
ylabel={$y(t)$},
axis background/.style={fill=white},
title style={font=\bfseries},
title={C6: RMSE(OSA) = 8.3465, RMSE(MPO) = 10.0061},
legend style={legend cell align=left, align=left, draw=white!15!black}
]
\addplot [color=mycolor1, line width=2.0pt]
  table[row sep=crcr]{%
1006	-185.547\\
1007	-219.727\\
1008	-169.678\\
1009	-86.6700000000001\\
1010	-107.422\\
1011	-102.539\\
1012	-126.953\\
1013	-103.76\\
1014	-41.5039999999999\\
1015	-31.7380000000001\\
1016	-28.076\\
1018	-157.471\\
1019	-167.236\\
1020	-174.561\\
1022	-74.463\\
1023	-158.691\\
1024	-111.084\\
1025	-83.008\\
1026	-142.822\\
1028	-103.76\\
1029	-173.34\\
1030	-164.795\\
1031	-124.512\\
1032	-180.664\\
1033	-179.443\\
1034	-142.822\\
1036	-89.1110000000001\\
1037	-108.643\\
1038	-137.939\\
1039	-130.615\\
1040	-136.719\\
1041	-140.381\\
1042	-151.367\\
1043	-213.623\\
1044	-191.65\\
1045	-126.953\\
1046	-115.967\\
1047	-64.6970000000001\\
1048	-69.5799999999999\\
1049	-96.4359999999999\\
1050	-74.463\\
1051	-78.125\\
1052	-111.084\\
1053	-128.174\\
1054	-119.629\\
1055	-190.43\\
1057	-81.787\\
1058	-63.4770000000001\\
1059	-85.4490000000001\\
1060	-101.318\\
1061	-51.27\\
1062	-58.5940000000001\\
1063	-81.787\\
1064	-65.9180000000001\\
1065	-56.152\\
1066	-74.463\\
1067	-86.6700000000001\\
1068	-139.16\\
1069	-130.615\\
1070	-163.574\\
1071	-146.484\\
1072	-170.898\\
1073	-139.16\\
1074	-133.057\\
1075	-115.967\\
1076	-111.084\\
1077	-111.084\\
1079	-280.762\\
1080	-291.748\\
1081	-280.762\\
1082	-175.781\\
1084	-303.955\\
1085	-317.383\\
1086	-239.258\\
1087	-313.721\\
1088	-404.053\\
1089	-289.307\\
1090	-238.037\\
1091	-158.691\\
1092	-122.07\\
1093	-102.539\\
1094	-129.395\\
1096	-61.0350000000001\\
1097	-59.8140000000001\\
1098	-75.684\\
1099	-118.408\\
1100	-107.422\\
1101	-111.084\\
1102	-146.484\\
1103	-152.588\\
1104	-207.52\\
1105	-167.236\\
1106	-208.74\\
1107	-148.926\\
1108	-130.615\\
1109	-151.367\\
1110	-108.643\\
1111	-98.877\\
1112	-122.07\\
1113	-123.291\\
1114	-85.4490000000001\\
1115	-95.2149999999999\\
1116	-67.1389999999999\\
1117	-76.904\\
1118	-81.787\\
1119	-57.373\\
1121	-93.9939999999999\\
1122	-151.367\\
1123	-152.588\\
1124	-170.898\\
1125	-97.6559999999999\\
1126	-140.381\\
1127	-211.182\\
1128	-161.133\\
1129	-186.768\\
1130	-191.65\\
1131	-112.305\\
1132	-75.684\\
1133	-107.422\\
1134	-123.291\\
1135	-205.078\\
1136	-252.686\\
1137	-252.686\\
1138	-184.326\\
1139	-189.209\\
1140	-167.236\\
1141	-163.574\\
1142	-162.354\\
1143	-108.643\\
1144	-96.4359999999999\\
1146	-78.125\\
1147	-87.8910000000001\\
1148	-133.057\\
1149	-203.857\\
1150	-161.133\\
1151	-109.863\\
1152	-92.7729999999999\\
1153	-89.1110000000001\\
1154	-52.49\\
1155	-41.5039999999999\\
1156	-47.607\\
1157	-90.3320000000001\\
1158	-67.1389999999999\\
1159	-78.125\\
1160	-85.4490000000001\\
1161	-80.566\\
1162	-54.932\\
1163	-37.8420000000001\\
1164	-30.518\\
1165	-54.932\\
1166	-119.629\\
1167	-172.119\\
1168	-194.092\\
1169	-136.719\\
1170	-91.5530000000001\\
1171	-64.6970000000001\\
1172	-43.9449999999999\\
1173	-98.877\\
1174	-91.5530000000001\\
1175	-137.939\\
1176	-168.457\\
1177	-275.879\\
1178	-316.162\\
1179	-260.01\\
1180	-249.023\\
1181	-159.912\\
1182	-185.547\\
1183	-195.313\\
1184	-212.402\\
1185	-158.691\\
1186	-144.043\\
1187	-142.822\\
1188	-129.395\\
1189	-156.25\\
1190	-109.863\\
1191	-195.313\\
1192	-234.375\\
1193	-175.781\\
1194	-104.98\\
1195	-111.084\\
1196	-192.871\\
1197	-234.375\\
1198	-289.307\\
1199	-303.955\\
1200	-302.734\\
1201	-228.271\\
1202	-205.078\\
1203	-235.596\\
1204	-275.879\\
1205	-170.898\\
1206	-104.98\\
1207	-152.588\\
1208	-122.07\\
1209	-79.346\\
1210	-109.863\\
1211	-122.07\\
1212	-95.2149999999999\\
1213	-75.684\\
1214	-101.318\\
1215	-80.566\\
1216	-115.967\\
1217	-162.354\\
1218	-147.705\\
1219	-100.098\\
1220	-101.318\\
1221	-157.471\\
1222	-261.23\\
1224	-118.408\\
1225	-85.4490000000001\\
1226	-101.318\\
1227	-90.3320000000001\\
1228	-67.1389999999999\\
1229	-76.904\\
1230	-92.7729999999999\\
1231	-120.85\\
1232	-134.277\\
1233	-89.1110000000001\\
1234	-156.25\\
1235	-179.443\\
1236	-170.898\\
1237	-140.381\\
1238	-150.146\\
1239	-202.637\\
1241	-53.711\\
1242	-78.125\\
1243	-85.4490000000001\\
1244	-119.629\\
1245	-101.318\\
1246	-74.463\\
1247	-118.408\\
1248	-112.305\\
1249	-79.346\\
1250	-102.539\\
1251	-90.3320000000001\\
1252	-80.566\\
1253	-95.2149999999999\\
1254	-54.932\\
1255	-68.3589999999999\\
1256	-47.607\\
1257	-51.27\\
1258	-106.201\\
1259	-122.07\\
1260	-167.236\\
1261	-219.727\\
1262	-142.822\\
1263	-83.008\\
1264	-63.4770000000001\\
1265	-62.2560000000001\\
1266	-91.5530000000001\\
1267	-54.932\\
1269	-85.4490000000001\\
1270	-150.146\\
1271	-133.057\\
1272	-190.43\\
1273	-124.512\\
1274	-117.188\\
1275	-56.152\\
1276	-62.2560000000001\\
1277	-34.1800000000001\\
1278	-46.3869999999999\\
1279	-51.27\\
1280	-95.2149999999999\\
1281	-100.098\\
1282	-117.188\\
1283	-123.291\\
1284	-197.754\\
1285	-170.898\\
1286	-128.174\\
1287	-153.809\\
1288	-163.574\\
1289	-213.623\\
1292	-53.711\\
1293	-40.2829999999999\\
1294	-41.5039999999999\\
1295	-52.49\\
1296	-54.932\\
1297	-51.27\\
1298	-70.8009999999999\\
1299	-108.643\\
1300	-98.877\\
1301	-103.76\\
1302	-118.408\\
1303	-69.5799999999999\\
1304	-36.6210000000001\\
1306	-125.732\\
1307	-125.732\\
1308	-142.822\\
1309	-118.408\\
1310	-87.8910000000001\\
1311	-123.291\\
1312	-172.119\\
1313	-172.119\\
1314	-120.85\\
1315	-207.52\\
1316	-155.029\\
1317	-151.367\\
1318	-173.34\\
1319	-172.119\\
1320	-130.615\\
1321	-112.305\\
1323	-266.113\\
1324	-202.637\\
1325	-124.512\\
1326	-118.408\\
1327	-115.967\\
1328	-150.146\\
1329	-173.34\\
1330	-125.732\\
1331	-134.277\\
1332	-177.002\\
1333	-192.871\\
1334	-120.85\\
1335	-97.6559999999999\\
1336	-130.615\\
1337	-218.506\\
1338	-195.313\\
1339	-203.857\\
1340	-115.967\\
1342	-101.318\\
1343	-63.4770000000001\\
1344	-41.5039999999999\\
1345	-34.1800000000001\\
1346	-29.297\\
1347	-78.125\\
1348	-108.643\\
1349	-131.836\\
1350	-93.9939999999999\\
1352	-119.629\\
1353	-72.021\\
1354	-79.346\\
1355	-73.242\\
1356	-54.932\\
1357	-75.684\\
1358	-120.85\\
1359	-156.25\\
1360	-95.2149999999999\\
1361	-73.242\\
1362	-58.5940000000001\\
1363	-74.463\\
1364	-45.1659999999999\\
1366	-195.313\\
1367	-163.574\\
1368	-216.064\\
1369	-241.699\\
1370	-249.023\\
1371	-181.885\\
1372	-131.836\\
1374	-129.395\\
1376	-184.326\\
1377	-181.885\\
1378	-247.803\\
1379	-270.996\\
1380	-328.369\\
1381	-217.285\\
1382	-238.037\\
1383	-264.893\\
1384	-205.078\\
1385	-239.258\\
1386	-200.195\\
1387	-113.525\\
1388	-74.463\\
1389	-79.346\\
1390	-117.188\\
1391	-97.6559999999999\\
1392	-64.6970000000001\\
1393	-90.3320000000001\\
1394	-63.4770000000001\\
1395	-48.828\\
1396	-64.6970000000001\\
1397	-72.021\\
1398	-48.828\\
1399	-100.098\\
1400	-135.498\\
1401	-97.6559999999999\\
1403	-241.699\\
1404	-220.947\\
1405	-279.541\\
1406	-187.988\\
1407	-206.299\\
1408	-134.277\\
1409	-85.4490000000001\\
1410	-108.643\\
1411	-87.8910000000001\\
1412	-63.4770000000001\\
1413	-92.7729999999999\\
1414	-50.049\\
1415	-32.9590000000001\\
1416	-43.9449999999999\\
1417	-98.877\\
1418	-125.732\\
1419	-157.471\\
1420	-152.588\\
1421	-146.484\\
1422	-168.457\\
1423	-135.498\\
1424	-108.643\\
1425	-112.305\\
1426	-89.1110000000001\\
1427	-146.484\\
1428	-96.4359999999999\\
1429	-80.566\\
1431	-164.795\\
1432	-123.291\\
1433	-93.9939999999999\\
1434	-80.566\\
1435	-91.5530000000001\\
1436	-61.0350000000001\\
1437	-61.0350000000001\\
1438	-87.8910000000001\\
1439	-109.863\\
1440	-124.512\\
1441	-96.4359999999999\\
1442	-128.174\\
1443	-114.746\\
1444	-61.0350000000001\\
1445	-69.5799999999999\\
1446	-135.498\\
1447	-190.43\\
1448	-186.768\\
1449	-157.471\\
1450	-152.588\\
1451	-114.746\\
1452	-109.863\\
1453	-87.8910000000001\\
1454	-125.732\\
1455	-117.188\\
1456	-70.8009999999999\\
1457	-107.422\\
1458	-126.953\\
1459	-150.146\\
1460	-164.795\\
1461	-163.574\\
1462	-222.168\\
1463	-267.334\\
1464	-302.734\\
1465	-190.43\\
1466	-106.201\\
1467	-68.3589999999999\\
1468	-45.1659999999999\\
1469	-70.8009999999999\\
1470	-64.6970000000001\\
1471	-119.629\\
1472	-102.539\\
1473	-93.9939999999999\\
1474	-124.512\\
1475	-145.264\\
1476	-172.119\\
1477	-183.105\\
1478	-258.789\\
1479	-231.934\\
1481	-120.85\\
1482	-120.85\\
1483	-123.291\\
1484	-157.471\\
1485	-108.643\\
1486	-113.525\\
1487	-107.422\\
1488	-58.5940000000001\\
1490	-152.588\\
1491	-107.422\\
1492	-115.967\\
1493	-216.064\\
1494	-159.912\\
1495	-156.25\\
1496	-219.727\\
1497	-206.299\\
1498	-129.395\\
1499	-85.4490000000001\\
1500	-85.4490000000001\\
1501	-146.484\\
1502	-236.816\\
1503	-246.582\\
1504	-251.465\\
1505	-192.871\\
1506	-122.07\\
1507	-103.76\\
1508	-72.021\\
1509	-69.5799999999999\\
1510	-58.5940000000001\\
1511	-89.1110000000001\\
1512	-102.539\\
1513	-58.5940000000001\\
1515	-126.953\\
1516	-145.264\\
1517	-185.547\\
1519	-281.982\\
1520	-197.754\\
1521	-172.119\\
1522	-178.223\\
1523	-156.25\\
1524	-103.76\\
1525	-128.174\\
1526	-214.844\\
1527	-246.582\\
1528	-141.602\\
1529	-79.346\\
1531	-28.076\\
1532	-35.4000000000001\\
1533	-64.6970000000001\\
1534	-74.463\\
1535	-102.539\\
1536	-81.787\\
1537	-64.6970000000001\\
1538	-62.2560000000001\\
1539	-61.0350000000001\\
1540	-54.932\\
1541	-65.9180000000001\\
1542	-102.539\\
1543	-148.926\\
1544	-157.471\\
1545	-155.029\\
1546	-183.105\\
1547	-225.83\\
1548	-225.83\\
1549	-270.996\\
1550	-247.803\\
1552	-125.732\\
1553	-122.07\\
1554	-205.078\\
1555	-236.816\\
1556	-166.016\\
1558	-53.711\\
1559	-40.2829999999999\\
1560	-42.7249999999999\\
1561	-25.635\\
1562	-61.0350000000001\\
1563	-84.229\\
1564	-89.1110000000001\\
1565	-78.125\\
1566	-47.607\\
1567	-35.4000000000001\\
1568	-52.49\\
1569	-56.152\\
1570	-62.2560000000001\\
1571	-47.607\\
1572	-37.8420000000001\\
1573	-69.5799999999999\\
1574	-164.795\\
1576	-217.285\\
1577	-147.705\\
1578	-208.74\\
1579	-291.748\\
1580	-261.23\\
1581	-273.438\\
1582	-200.195\\
1583	-202.637\\
1584	-212.402\\
1585	-139.16\\
1586	-133.057\\
1587	-80.566\\
1588	-75.684\\
1589	-135.498\\
1590	-93.9939999999999\\
1592	-115.967\\
1593	-111.084\\
1594	-113.525\\
1595	-120.85\\
1596	-136.719\\
1597	-147.705\\
1598	-125.732\\
1599	-92.7729999999999\\
1600	-120.85\\
1601	-57.373\\
1602	-26.855\\
1603	-46.3869999999999\\
1604	-73.242\\
1605	-37.8420000000001\\
1606	-17.0899999999999\\
1607	-39.0630000000001\\
1608	-70.8009999999999\\
1609	-84.229\\
1610	-104.98\\
1611	-87.8910000000001\\
1612	-142.822\\
1613	-123.291\\
1614	-83.008\\
1615	-125.732\\
1616	-70.8009999999999\\
1617	-56.152\\
1619	-24.414\\
1620	-48.828\\
1621	-36.6210000000001\\
1622	-65.9180000000001\\
1623	-111.084\\
1624	-106.201\\
1625	-76.904\\
1626	-86.6700000000001\\
1627	-64.6970000000001\\
1628	-92.7729999999999\\
1629	-67.1389999999999\\
1630	-63.4770000000001\\
1631	-58.5940000000001\\
1632	-43.9449999999999\\
1633	-57.373\\
1634	-51.27\\
1635	-90.3320000000001\\
1636	-83.008\\
1637	-67.1389999999999\\
1638	-64.6970000000001\\
1639	-51.27\\
1640	-61.0350000000001\\
1641	-135.498\\
1642	-179.443\\
1643	-123.291\\
1644	-114.746\\
1645	-80.566\\
1646	-136.719\\
1647	-125.732\\
1648	-80.566\\
1649	-102.539\\
1650	-76.904\\
1651	-108.643\\
1652	-64.6970000000001\\
1653	-64.6970000000001\\
1654	-80.566\\
1655	-104.98\\
1656	-74.463\\
1657	-81.787\\
1658	-85.4490000000001\\
1659	-135.498\\
1660	-115.967\\
1662	-231.934\\
1663	-183.105\\
1664	-117.188\\
1665	-86.6700000000001\\
1666	-135.498\\
1667	-173.34\\
1668	-133.057\\
1669	-163.574\\
1670	-98.877\\
1671	-50.049\\
1672	-52.49\\
1673	-119.629\\
1674	-107.422\\
1675	-100.098\\
1676	-156.25\\
1677	-146.484\\
1678	-133.057\\
1679	-90.3320000000001\\
1680	-114.746\\
1681	-148.926\\
1682	-120.85\\
1683	-125.732\\
1684	-112.305\\
1685	-80.566\\
1686	-97.6559999999999\\
1687	-70.8009999999999\\
1688	-79.346\\
1689	-135.498\\
1690	-156.25\\
1691	-164.795\\
1692	-146.484\\
1693	-104.98\\
1694	-72.021\\
1695	-85.4490000000001\\
1696	-101.318\\
1698	-159.912\\
1699	-190.43\\
1700	-146.484\\
1701	-84.229\\
1703	-222.168\\
1704	-155.029\\
1705	-155.029\\
1706	-181.885\\
1707	-137.939\\
1708	-113.525\\
1709	-129.395\\
1710	-115.967\\
1711	-131.836\\
1712	-167.236\\
1713	-125.732\\
1714	-146.484\\
1715	-137.939\\
1716	-141.602\\
1717	-109.863\\
1718	-144.043\\
1719	-102.539\\
1720	-106.201\\
1721	-119.629\\
1722	-115.967\\
1723	-58.5940000000001\\
1724	-35.4000000000001\\
1725	-28.076\\
1726	-47.607\\
1727	-122.07\\
1728	-161.133\\
1729	-161.133\\
1730	-101.318\\
1731	-119.629\\
1732	-104.98\\
1733	-92.7729999999999\\
1734	-58.5940000000001\\
1735	-81.787\\
1736	-81.787\\
1737	-80.566\\
1738	-95.2149999999999\\
1739	-79.346\\
1740	-75.684\\
1741	-50.049\\
1742	-72.021\\
1743	-139.16\\
1744	-96.4359999999999\\
1745	-119.629\\
1746	-107.422\\
1747	-145.264\\
1748	-135.498\\
1749	-186.768\\
1750	-137.939\\
1751	-151.367\\
1752	-155.029\\
1753	-72.021\\
1754	-131.836\\
1755	-95.2149999999999\\
1756	-113.525\\
1757	-128.174\\
1758	-166.016\\
1759	-123.291\\
1760	-158.691\\
1761	-201.416\\
1762	-164.795\\
1763	-142.822\\
1764	-113.525\\
1765	-136.719\\
1766	-113.525\\
1768	-81.787\\
1769	-96.4359999999999\\
1770	-83.008\\
1771	-172.119\\
1772	-225.83\\
1773	-191.65\\
1774	-187.988\\
1775	-181.885\\
1776	-101.318\\
1777	-89.1110000000001\\
1778	-72.021\\
1779	-62.2560000000001\\
1780	-68.3589999999999\\
1781	-115.967\\
1782	-146.484\\
1783	-167.236\\
1784	-141.602\\
1785	-93.9939999999999\\
1786	-169.678\\
1787	-201.416\\
1788	-224.609\\
1789	-178.223\\
1790	-288.086\\
1791	-209.961\\
1792	-117.188\\
1793	-84.229\\
1794	-69.5799999999999\\
1795	-125.732\\
1796	-162.354\\
1797	-218.506\\
1798	-166.016\\
1799	-126.953\\
1800	-69.5799999999999\\
1801	-72.021\\
1802	-78.125\\
1803	-87.8910000000001\\
1804	-56.152\\
1805	-70.8009999999999\\
};
\addlegendentry{True output}

\addplot [color=mycolor2, dashed, line width=2.0pt]
  table[row sep=crcr]{%
1006	-171.60756735679\\
1007	-207.372972410161\\
1008	-174.960700080454\\
1009	-86.1823724162059\\
1010	-110.995349893054\\
1011	-112.023314091797\\
1012	-130.286760047934\\
1013	-105.521457372687\\
1014	-31.3009211090216\\
1015	-14.2758473006229\\
1016	-17.9044403620314\\
1017	-102.248068986451\\
1018	-163.036122603209\\
1019	-161.191278483162\\
1020	-160.928268060267\\
1021	-131.985295260273\\
1022	-72.9806388441045\\
1023	-164.063749985325\\
1024	-114.542853108525\\
1025	-84.3001681052083\\
1026	-143.819111393111\\
1027	-119.111689982027\\
1028	-110.593550378328\\
1029	-162.632863195516\\
1030	-159.86140893243\\
1031	-123.280081345\\
1032	-173.069205017179\\
1033	-171.181174413626\\
1036	-96.1676893660672\\
1037	-107.842602842546\\
1038	-132.123227286532\\
1039	-125.008812171101\\
1040	-134.004555266186\\
1042	-148.069464793165\\
1043	-190.914077541743\\
1044	-184.20765536164\\
1045	-133.838571011551\\
1046	-122.769688606768\\
1047	-63.1680338191484\\
1048	-74.1423583827795\\
1049	-97.3808794557792\\
1050	-77.2013846939462\\
1051	-78.3115139394336\\
1052	-107.107139669424\\
1053	-127.475584347494\\
1054	-117.772307925938\\
1055	-173.023027508628\\
1056	-147.574105845098\\
1057	-86.3437201342206\\
1058	-66.3005347209328\\
1059	-89.0935902400213\\
1060	-102.862631529143\\
1061	-54.101214748802\\
1062	-52.0856294605719\\
1063	-89.0150319778511\\
1064	-66.7441862989297\\
1065	-56.9302768571335\\
1066	-79.4786112198153\\
1067	-86.6834765090168\\
1068	-128.573201333524\\
1069	-136.736464748564\\
1070	-147.208130303548\\
1071	-143.109015556528\\
1072	-166.171113989256\\
1073	-135.820405668215\\
1074	-141.248350723683\\
1075	-115.834481763208\\
1076	-115.288495826598\\
1077	-114.506334953746\\
1079	-256.647343694504\\
1080	-268.158224847341\\
1081	-262.256214482121\\
1082	-189.264294819265\\
1083	-233.524326057332\\
1084	-287.20326615306\\
1085	-290.159692962746\\
1086	-249.0635120674\\
1087	-281.692727180445\\
1088	-374.43012309209\\
1089	-319.707954776764\\
1090	-255.577067846775\\
1091	-156.741932613529\\
1092	-126.384580931613\\
1093	-110.302479075114\\
1094	-125.717550459032\\
1095	-102.323907884601\\
1096	-57.7341314747478\\
1097	-48.792317424078\\
1098	-80.4611846824305\\
1099	-117.587282370857\\
1100	-113.113300221172\\
1101	-113.364287678256\\
1102	-143.975618763065\\
1103	-149.359754213597\\
1104	-199.511201146069\\
1105	-177.494870303361\\
1106	-191.182351886397\\
1108	-134.600757275002\\
1109	-157.789366867725\\
1110	-114.683241206739\\
1111	-110.833571107961\\
1112	-120.246901337074\\
1113	-131.689759748769\\
1114	-90.6544538106116\\
1115	-102.303524389067\\
1116	-70.1999240658456\\
1117	-85.5980283767328\\
1118	-85.9671000733904\\
1119	-62.3082795925441\\
1120	-71.80369230372\\
1121	-97.0699024356288\\
1122	-147.506708994321\\
1123	-152.152079226968\\
1124	-163.681604523938\\
1125	-102.167338168934\\
1126	-136.641486064431\\
1127	-206.438065115638\\
1128	-165.169515409515\\
1129	-185.172468570194\\
1130	-183.18224555349\\
1131	-122.252563560329\\
1132	-83.1607033895111\\
1133	-107.643760490254\\
1134	-126.425747902242\\
1135	-191.776231936178\\
1136	-234.866982126499\\
1137	-238.707484955929\\
1138	-194.358952944519\\
1139	-188.06721222856\\
1140	-179.931143715754\\
1141	-173.725105141008\\
1142	-162.609828824815\\
1143	-113.668715307522\\
1144	-97.1568934606894\\
1145	-93.4406057260273\\
1146	-87.3739708476257\\
1147	-92.8675967833324\\
1148	-129.071628209816\\
1149	-196.662784146023\\
1150	-162.613478820537\\
1151	-115.456991879055\\
1152	-102.700696610724\\
1153	-94.3079414154411\\
1154	-52.6908917222938\\
1155	-34.3910191427019\\
1156	-44.0025802660605\\
1157	-96.1551967948546\\
1158	-73.983765498675\\
1159	-80.9045384756546\\
1160	-82.0479932759977\\
1161	-81.6501954425814\\
1162	-62.5939007144398\\
1163	-35.8717457526664\\
1164	-23.6236836028502\\
1165	-57.4937967023991\\
1166	-125.910471965547\\
1167	-172.639998011784\\
1168	-186.78510141708\\
1169	-145.574739591317\\
1170	-93.902502583454\\
1171	-66.9874576530995\\
1172	-35.3391658088319\\
1173	-102.77727009567\\
1174	-93.9035494247942\\
1175	-126.899955467471\\
1176	-172.554592118505\\
1177	-256.746498719573\\
1178	-318.617592517548\\
1179	-265.71707881409\\
1180	-252.83524829252\\
1181	-170.217673255881\\
1182	-176.427008910613\\
1183	-204.426900363193\\
1184	-199.913322723565\\
1185	-169.555426664912\\
1186	-146.462859294929\\
1187	-154.446183664432\\
1188	-134.504029793194\\
1189	-155.078054454768\\
1190	-120.668986436425\\
1191	-187.635786618039\\
1192	-215.762541435262\\
1193	-185.740691305797\\
1194	-109.7813352417\\
1195	-112.89133060463\\
1196	-192.244848294263\\
1197	-220.996325396744\\
1199	-286.610575916957\\
1200	-292.569005705594\\
1201	-238.801354632643\\
1202	-218.131503416166\\
1203	-229.959349272988\\
1204	-264.523977004674\\
1206	-103.84624290981\\
1207	-154.680799964568\\
1208	-128.065735161669\\
1209	-91.08777023467\\
1210	-104.817906257511\\
1211	-129.940234100648\\
1212	-102.933749851241\\
1213	-82.9835964507872\\
1214	-103.869962873596\\
1215	-91.3043894057328\\
1216	-117.649579755576\\
1217	-168.066805209796\\
1219	-114.326628027773\\
1220	-109.424411316551\\
1221	-156.013568514267\\
1222	-245.308731992177\\
1223	-195.305372266798\\
1224	-125.786278815001\\
1225	-84.9002379706615\\
1226	-106.404757355543\\
1227	-107.656759716062\\
1228	-71.308864455126\\
1229	-79.060473013431\\
1230	-101.859418183679\\
1232	-134.342440155234\\
1233	-92.9652612405623\\
1234	-157.836614296345\\
1235	-177.667338505431\\
1236	-162.393212003096\\
1237	-155.329390715426\\
1238	-144.834849341862\\
1239	-200.049554479291\\
1240	-130.702813420889\\
1241	-48.5188436646533\\
1242	-75.6237665504646\\
1243	-92.0762322369712\\
1244	-123.776764701007\\
1245	-107.435363853692\\
1246	-83.0435980332591\\
1247	-121.029323843135\\
1248	-116.668577215531\\
1249	-89.0592977961815\\
1250	-101.697968586044\\
1251	-94.6318533533026\\
1252	-91.4619067851654\\
1253	-99.7532086283559\\
1254	-55.976001769917\\
1255	-75.722676208844\\
1256	-48.2296209226556\\
1257	-58.8437225240191\\
1258	-113.204717424319\\
1259	-114.993874944\\
1260	-165.280913016054\\
1261	-205.962399976436\\
1262	-158.186773136123\\
1263	-89.9563547483629\\
1264	-54.8320047503692\\
1266	-95.3720010852107\\
1267	-56.0018107095029\\
1268	-68.5101991595425\\
1269	-89.5938637143925\\
1270	-149.319324622138\\
1271	-136.652236190324\\
1272	-179.965058482017\\
1273	-125.917125044516\\
1274	-126.078524265512\\
1275	-48.6912146082666\\
1276	-61.643669907045\\
1277	-31.2789154635775\\
1278	-45.0793940327628\\
1279	-50.1020521770292\\
1280	-99.1486735298968\\
1281	-101.507125901656\\
1283	-126.033914563285\\
1284	-186.354799632089\\
1285	-188.487568812567\\
1286	-133.138892793903\\
1287	-150.092378411877\\
1288	-161.419373043033\\
1289	-198.394779321282\\
1290	-165.331941256843\\
1291	-116.992365678863\\
1292	-40.8417857679776\\
1293	-28.2705411783743\\
1294	-38.3119466310359\\
1295	-54.4691402988522\\
1296	-58.7042309192384\\
1297	-50.1315909199043\\
1298	-71.860486956125\\
1299	-113.616150347283\\
1300	-104.078476997485\\
1301	-97.281962952987\\
1302	-118.891048604297\\
1304	-32.3368281983921\\
1305	-90.1339571281092\\
1306	-128.585212771559\\
1307	-115.39053016495\\
1308	-134.228843072427\\
1309	-126.339512762541\\
1310	-90.1546688026867\\
1311	-121.744618149766\\
1312	-167.262785248922\\
1313	-162.391433481384\\
1314	-128.718290149912\\
1315	-192.304106322933\\
1316	-161.513861114372\\
1317	-152.721053874013\\
1318	-171.069342924754\\
1319	-163.368848430852\\
1320	-139.059768528284\\
1321	-120.911491434619\\
1322	-175.585510626661\\
1323	-248.997591582228\\
1324	-206.287844067891\\
1325	-131.474717495517\\
1326	-120.869184689813\\
1327	-123.661834576401\\
1328	-148.94894537851\\
1329	-162.006710467626\\
1330	-136.884981417465\\
1331	-136.836503634947\\
1332	-170.641232004077\\
1333	-179.368997630635\\
1334	-132.071447282708\\
1335	-106.426457186634\\
1336	-135.910018615418\\
1337	-197.768071566046\\
1338	-196.361556133985\\
1339	-191.569249374759\\
1340	-122.855750364011\\
1341	-112.013394957334\\
1342	-116.97631527707\\
1343	-62.247558955695\\
1344	-32.3337304641125\\
1345	-23.4505327859472\\
1346	-23.2920969601932\\
1347	-82.4735962821476\\
1348	-111.550247289127\\
1349	-125.886267582702\\
1350	-102.819807762831\\
1351	-106.483284841543\\
1352	-117.387857258674\\
1353	-77.4834811705718\\
1354	-78.9396804119399\\
1355	-79.1744455326284\\
1356	-56.7381688655967\\
1357	-80.1115920251534\\
1359	-149.46214202716\\
1360	-98.0665562529359\\
1361	-82.838207661568\\
1362	-57.7932443171744\\
1363	-79.2883278913898\\
1364	-49.2543121577592\\
1366	-186.663653047807\\
1367	-159.724948172568\\
1369	-235.054265813664\\
1370	-238.182734694308\\
1371	-196.569353811115\\
1372	-141.448122863608\\
1373	-129.614985731915\\
1374	-139.200718721464\\
1375	-152.442314057459\\
1376	-170.007250263561\\
1377	-174.019087960666\\
1378	-224.786907203761\\
1379	-257.997251243094\\
1380	-304.158222948087\\
1381	-232.038820582253\\
1382	-238.883031330399\\
1383	-262.813860845776\\
1384	-209.608353073976\\
1385	-236.856163810697\\
1386	-198.304433960595\\
1388	-58.4849060422439\\
1389	-75.7572822223381\\
1390	-129.102314531604\\
1391	-103.06411412279\\
1392	-69.6945355384041\\
1393	-90.8020239017301\\
1395	-46.6558360982895\\
1396	-71.8447551334366\\
1397	-69.5978780839876\\
1398	-59.420560745905\\
1400	-140.645285687259\\
1401	-99.0844374856724\\
1402	-163.6313198654\\
1403	-241.882453896661\\
1404	-221.673669381069\\
1405	-266.843033154963\\
1406	-200.602236366809\\
1407	-203.468631831995\\
1408	-146.855894840801\\
1409	-78.5537638561395\\
1410	-111.752306681607\\
1411	-96.9702364069183\\
1412	-64.556885475738\\
1413	-96.3836220786325\\
1414	-47.9842150956117\\
1415	-28.4391877461519\\
1416	-37.0322449434502\\
1417	-108.851097548332\\
1418	-125.548155293032\\
1419	-153.110415964829\\
1420	-154.358434172048\\
1421	-148.468990037377\\
1422	-156.972280174365\\
1423	-143.930287990298\\
1424	-112.680191158512\\
1425	-115.644896330711\\
1426	-100.606777020107\\
1427	-140.99748620282\\
1428	-99.034121714059\\
1429	-85.06932983915\\
1430	-119.181061454035\\
1431	-161.855767233555\\
1432	-122.165161367068\\
1434	-86.5558365230038\\
1435	-96.4703879082231\\
1436	-72.7290912516237\\
1437	-56.5010909056664\\
1438	-91.3246635811288\\
1439	-107.92940953624\\
1440	-115.750679902097\\
1441	-110.471795378409\\
1442	-117.078707638738\\
1443	-122.110669773986\\
1444	-59.5027744735696\\
1445	-73.1989602405322\\
1446	-142.259510673237\\
1447	-179.442946704287\\
1449	-162.958399906291\\
1450	-153.2270198293\\
1451	-120.095354783296\\
1452	-116.967129377747\\
1453	-95.101612812877\\
1454	-124.966387869322\\
1455	-119.379974774395\\
1456	-74.2284222436642\\
1457	-117.050111034389\\
1458	-113.243520544356\\
1459	-147.710995883867\\
1460	-155.345165064858\\
1461	-157.973001530932\\
1464	-296.70056413222\\
1466	-114.134659773428\\
1467	-47.2861036667969\\
1468	-34.4196749340365\\
1469	-79.4150680912044\\
1470	-61.9958359766058\\
1471	-122.037145063034\\
1472	-105.913363505203\\
1473	-100.679503544843\\
1474	-124.400294166356\\
1475	-140.877707827376\\
1476	-166.79385884389\\
1477	-177.408980466373\\
1478	-246.743884228179\\
1479	-237.0663373856\\
1481	-127.780318158598\\
1482	-123.461437763226\\
1483	-130.242307107138\\
1484	-157.99938424945\\
1485	-117.895232777355\\
1486	-110.886345540873\\
1487	-116.51725239163\\
1488	-45.7254673566433\\
1489	-109.504605525919\\
1490	-148.969174304607\\
1491	-117.030861847922\\
1492	-114.978302933916\\
1493	-207.06981936384\\
1494	-164.999393756582\\
1495	-157.586031212647\\
1496	-204.180727663906\\
1497	-210.452729346202\\
1498	-134.990258356643\\
1499	-88.5365191610138\\
1500	-88.1767857855316\\
1502	-220.482783250664\\
1504	-239.782827260157\\
1505	-198.335591525767\\
1506	-124.848406638447\\
1507	-115.583419784407\\
1508	-73.5140752414509\\
1509	-76.8663577191646\\
1510	-59.6209269244798\\
1511	-96.7094511841408\\
1512	-102.312218166409\\
1513	-61.6821985961033\\
1515	-126.746732872514\\
1516	-140.553627514754\\
1518	-217.10027598038\\
1519	-277.204030707942\\
1520	-204.482317974963\\
1521	-180.964556070369\\
1522	-185.066763424233\\
1523	-169.389007480088\\
1524	-110.745430697401\\
1525	-126.495269707488\\
1526	-204.702792931656\\
1527	-223.307684298803\\
1529	-80.7103577503542\\
1531	-2.40589883018629\\
1533	-65.9119293064671\\
1534	-71.7793517267994\\
1535	-108.120600602799\\
1536	-88.1471441050785\\
1537	-62.5020671761065\\
1538	-70.4428646301644\\
1539	-59.4238852599178\\
1540	-66.994019530274\\
1541	-65.287121102202\\
1542	-99.7184409560666\\
1543	-146.88965482459\\
1544	-153.244045746714\\
1545	-156.201666674994\\
1546	-171.433961881505\\
1547	-219.373797403722\\
1548	-228.457834856338\\
1549	-254.909132953458\\
1550	-247.107678702122\\
1551	-200.83904856529\\
1552	-136.00815259104\\
1553	-124.348562317082\\
1554	-202.081610190082\\
1555	-218.481194677814\\
1557	-116.605564106341\\
1558	-31.6406653532852\\
1559	-22.9983349862637\\
1560	-44.9878636080805\\
1561	-12.0617102054132\\
1562	-66.7665726866996\\
1563	-88.0722447350079\\
1564	-86.9681454737499\\
1565	-82.0147012278851\\
1566	-42.816676554329\\
1567	-36.073174960798\\
1568	-51.9577986779186\\
1569	-59.4269547474169\\
1570	-62.9649598363521\\
1571	-46.1562996541559\\
1572	-39.2900091121312\\
1573	-75.9944203821726\\
1574	-172.592247320596\\
1575	-205.737637277609\\
1576	-194.683379007028\\
1577	-165.418644607196\\
1578	-202.196934061609\\
1579	-270.789435238416\\
1580	-266.467150328795\\
1581	-260.813900339197\\
1582	-209.592302773478\\
1583	-215.560907502821\\
1584	-201.548750063892\\
1585	-158.595473737212\\
1586	-131.448957388516\\
1587	-73.591818747487\\
1588	-69.8683472809146\\
1589	-145.956495260838\\
1590	-94.1743813119481\\
1592	-118.496518703989\\
1593	-116.443886875717\\
1594	-111.844416258739\\
1595	-125.879709178539\\
1596	-136.058206072074\\
1597	-142.885527017608\\
1598	-130.650881157905\\
1599	-101.98251284279\\
1600	-126.959857786229\\
1601	-53.0265709835598\\
1602	-9.71854915659742\\
1603	-35.0096628505928\\
1604	-81.9972696957257\\
1605	-28.855893253862\\
1606	-7.54144079534422\\
1607	-37.3575719529824\\
1608	-75.5331047793247\\
1609	-80.4097514028672\\
1610	-107.594884647822\\
1611	-89.9471840579497\\
1612	-146.595917779035\\
1613	-130.029317806485\\
1614	-81.3832835924816\\
1615	-130.457092417608\\
1616	-55.9182897589133\\
1617	-64.4146519679734\\
1618	-27.1221813402069\\
1619	-20.0187262680167\\
1620	-50.3980938034797\\
1621	-26.198683567151\\
1623	-112.12263345842\\
1624	-104.61707785814\\
1625	-80.8611997535318\\
1626	-85.0556351421978\\
1627	-65.9970697162203\\
1628	-95.1608630037258\\
1629	-59.473254731183\\
1630	-63.3917798887751\\
1631	-59.141587058112\\
1632	-48.172495722343\\
1633	-53.6299948007857\\
1634	-50.7100372205591\\
1635	-94.8179406758848\\
1636	-79.1159533408461\\
1638	-60.0683477990892\\
1639	-51.6971299986883\\
1640	-61.9493506406727\\
1641	-136.246714223892\\
1642	-178.807347260791\\
1643	-123.682462915286\\
1644	-117.284240166506\\
1645	-74.6650064539026\\
1646	-145.411674261479\\
1647	-122.11572408629\\
1648	-82.2320437457558\\
1649	-100.921330234516\\
1650	-76.0791875587634\\
1651	-108.705650508026\\
1652	-59.3707428134046\\
1653	-61.6287461493591\\
1654	-85.3791944574634\\
1655	-97.1440723636174\\
1656	-80.9412814100585\\
1657	-76.6204621781103\\
1658	-88.0196578256621\\
1659	-129.366864912269\\
1660	-115.09538509835\\
1662	-226.893288802622\\
1663	-190.489098677657\\
1664	-120.951669782991\\
1665	-80.0564716853785\\
1666	-140.147836321836\\
1667	-162.196722318017\\
1668	-131.879844906438\\
1669	-157.52538067543\\
1670	-96.1677325774642\\
1671	-45.2911475535002\\
1672	-47.9811742304373\\
1673	-127.682954728919\\
1674	-110.923767334309\\
1675	-97.4529933199908\\
1676	-146.738598924817\\
1677	-144.631754590723\\
1678	-127.682496784908\\
1679	-96.1839808414632\\
1680	-107.815571762234\\
1681	-148.078309025635\\
1682	-115.972902123401\\
1683	-132.263656483075\\
1685	-91.2365252371528\\
1686	-101.381553326329\\
1687	-68.6631287074047\\
1688	-83.7435655719964\\
1689	-125.202521237502\\
1690	-152.074232510418\\
1691	-154.986571104358\\
1692	-141.163223676032\\
1694	-75.9441748362624\\
1695	-86.4851914930623\\
1697	-121.01006733609\\
1699	-178.035473185592\\
1700	-149.749419490699\\
1701	-92.3557604215089\\
1703	-212.139686374269\\
1704	-156.613384425416\\
1705	-152.664724499956\\
1706	-173.454344821357\\
1707	-146.082955329448\\
1708	-114.287367914884\\
1709	-135.185095909484\\
1710	-116.466225219178\\
1711	-130.99752843648\\
1712	-158.315698995845\\
1713	-127.708480451293\\
1714	-144.443886581992\\
1715	-136.263352823333\\
1716	-142.110890192354\\
1717	-116.732111367337\\
1718	-140.974638462889\\
1719	-110.543953394173\\
1720	-107.495701378705\\
1721	-123.846008076907\\
1722	-113.446011120521\\
1723	-58.5836062234002\\
1724	-19.230722234113\\
1725	-12.9219271235443\\
1726	-50.5621965390997\\
1727	-133.818954295853\\
1728	-160.290350543724\\
1729	-154.53798305964\\
1730	-104.791365953509\\
1731	-119.793230248833\\
1732	-111.879120565535\\
1733	-99.9967368661271\\
1734	-55.9029213069637\\
1735	-87.4000933227785\\
1736	-80.4536971574892\\
1737	-81.8877150012215\\
1738	-95.3490023508054\\
1739	-81.0775451602162\\
1740	-79.2710457797473\\
1741	-55.178213245915\\
1742	-71.3860647022425\\
1743	-134.485372530812\\
1744	-97.8189711044706\\
1745	-118.286465854341\\
1746	-101.759226225685\\
1747	-146.73190557208\\
1748	-130.089731619173\\
1749	-172.809978241695\\
1750	-144.427470765847\\
1751	-144.895107538812\\
1752	-152.454166340822\\
1753	-74.2079621417176\\
1754	-139.629023880668\\
1755	-87.6569822885556\\
1756	-120.06022485249\\
1757	-118.80743993465\\
1758	-153.40502446019\\
1759	-126.768360353407\\
1761	-181.640893778087\\
1762	-170.437693853442\\
1763	-138.720023644465\\
1764	-124.216118756028\\
1765	-133.550691002028\\
1766	-114.899934723004\\
1767	-108.766442132003\\
1768	-81.2999498747131\\
1769	-105.893614301944\\
1770	-86.6484977223133\\
1771	-160.190555386243\\
1772	-202.671580102085\\
1774	-176.495180648755\\
1775	-186.435869190943\\
1776	-103.849234907339\\
1777	-97.9308867314876\\
1778	-69.9356187429767\\
1779	-65.1973371852691\\
1780	-78.2803059593102\\
1781	-118.621860340461\\
1782	-135.45062230059\\
1783	-157.196177473886\\
1784	-137.99680742193\\
1785	-103.274222541544\\
1786	-168.076924017269\\
1788	-210.955497032556\\
1789	-173.706804625807\\
1790	-272.394014733168\\
1791	-224.728629576137\\
1792	-125.763781132402\\
1793	-78.0547283396434\\
1794	-66.559288880758\\
1795	-133.356137610284\\
1796	-155.136801484478\\
1797	-203.425269410272\\
1798	-172.502424979007\\
1799	-131.992934089358\\
1800	-60.8873250655156\\
1801	-73.6290027426701\\
1803	-93.3329960627152\\
1804	-54.6116043238849\\
1805	-80.3090614263533\\
};
\addlegendentry{OSA predition}

\addplot [color=mycolor3, dotted, line width=2.0pt]
  table[row sep=crcr]{%
1006	-185.547\\
1007	-219.727\\
1008	-169.678\\
1009	-86.6700000000001\\
1010	-110.995349893054\\
1011	-113.48015936147\\
1012	-135.19327379848\\
1013	-110.360946472996\\
1014	-34.8172796545821\\
1015	-12.7660538094156\\
1016	-9.65849234555185\\
1017	-92.7438606403211\\
1018	-158.777941637327\\
1020	-155.55723983941\\
1021	-123.397786203265\\
1022	-70.5790363408887\\
1023	-160.232197873451\\
1024	-112.768851437712\\
1025	-84.9968505149641\\
1026	-144.543637685234\\
1027	-120.278561299612\\
1028	-109.823766628659\\
1029	-165.061806387555\\
1030	-157.200723171348\\
1031	-119.097894005191\\
1032	-169.969627169353\\
1033	-164.884288578847\\
1034	-136.93681106948\\
1036	-94.3296238556434\\
1037	-108.386459061253\\
1038	-132.061345973268\\
1039	-122.795101762155\\
1042	-144.987563571806\\
1043	-186.591636008599\\
1044	-171.552699409156\\
1045	-121.793716770385\\
1046	-117.382712461922\\
1047	-61.3617515859144\\
1048	-71.0846890973312\\
1049	-97.0969194975905\\
1050	-77.7171953450575\\
1051	-79.4321511093226\\
1052	-108.214999741604\\
1053	-126.768712571956\\
1054	-117.268519029459\\
1055	-171.732138212703\\
1056	-139.293646642306\\
1057	-83.9653427485123\\
1058	-67.5663048608785\\
1059	-89.1814629127123\\
1060	-104.983705149856\\
1061	-56.2687732641559\\
1062	-54.9440333895845\\
1063	-88.6123462529536\\
1064	-69.6021405326696\\
1065	-59.6268948915574\\
1066	-81.2118219435376\\
1067	-90.7005418767144\\
1068	-131.698036712644\\
1069	-135.132184658992\\
1070	-149.213618083935\\
1071	-137.752569292308\\
1072	-160.409434913501\\
1073	-130.397866925905\\
1074	-134.620869648923\\
1075	-114.336991028697\\
1076	-113.605579419946\\
1077	-114.503982773202\\
1079	-252.709425635659\\
1080	-256.100195234047\\
1081	-243.321067938582\\
1082	-167.846929090957\\
1084	-271.501809266627\\
1085	-269.835862016719\\
1086	-223.466043397849\\
1087	-264.697290837415\\
1088	-343.705053589925\\
1089	-282.446590473256\\
1090	-241.074931688262\\
1091	-154.624147103811\\
1092	-119.575679685124\\
1093	-107.907632054288\\
1094	-127.867241734112\\
1095	-101.992882400741\\
1096	-60.8914813638912\\
1097	-50.5074886380255\\
1098	-76.5055911478364\\
1099	-117.694450426636\\
1100	-112.227957160808\\
1101	-114.367986787761\\
1102	-145.934751460476\\
1103	-149.80255903684\\
1104	-199.13722509029\\
1105	-174.012522914705\\
1106	-192.615888642202\\
1107	-154.494309719521\\
1108	-133.985708880866\\
1109	-160.046217382638\\
1110	-117.086019165295\\
1111	-115.980182043542\\
1112	-128.865890340022\\
1113	-137.860394652278\\
1114	-98.5446884083733\\
1115	-111.319464692393\\
1116	-78.9686551400971\\
1117	-94.0740168090688\\
1118	-96.1378868847739\\
1119	-71.9370317764954\\
1120	-80.9870618245516\\
1121	-103.318988342816\\
1122	-154.858309128\\
1123	-156.671104906495\\
1124	-167.182376762871\\
1125	-102.125617214529\\
1126	-138.471185938093\\
1127	-206.298810579124\\
1128	-162.891798818806\\
1129	-185.567992510351\\
1130	-182.474501888118\\
1131	-117.86819812411\\
1132	-84.4504347069512\\
1133	-111.849825875277\\
1134	-128.637894166269\\
1135	-196.084166664454\\
1136	-233.3501233701\\
1137	-230.788200197408\\
1138	-183.004017563308\\
1139	-183.186662725417\\
1140	-174.957399238873\\
1141	-173.860156907509\\
1142	-167.490514202969\\
1143	-116.464793522313\\
1144	-101.608603627606\\
1146	-92.9794572410171\\
1147	-101.179648669014\\
1148	-137.546412571466\\
1149	-203.067339092969\\
1150	-165.497699265968\\
1151	-118.562528576525\\
1152	-107.27019552469\\
1153	-101.227855018227\\
1154	-59.6171916755654\\
1155	-39.0060038005088\\
1156	-44.9720947242135\\
1157	-96.3625826094931\\
1158	-76.7150090856637\\
1159	-85.0901700923714\\
1160	-85.9944545436288\\
1161	-83.6074286614437\\
1162	-64.9324175384934\\
1163	-40.9888270884699\\
1164	-26.0296488858671\\
1165	-56.4807375331598\\
1166	-127.38313011177\\
1167	-175.511058284223\\
1168	-188.839480223654\\
1170	-97.198219940946\\
1171	-70.8489428575774\\
1172	-37.5446485485013\\
1173	-101.889012706432\\
1174	-95.2353911865328\\
1175	-128.714348014696\\
1176	-168.942203915952\\
1177	-256.781930745541\\
1178	-309.495105810918\\
1179	-259.730767857054\\
1180	-250.585047172704\\
1181	-169.025458544714\\
1182	-179.696761887039\\
1183	-203.314095107743\\
1184	-203.033633709121\\
1185	-167.510594758444\\
1186	-148.297495921108\\
1187	-157.981224174138\\
1188	-140.454316933763\\
1189	-162.62488821973\\
1190	-125.356590046052\\
1191	-197.160870814102\\
1192	-220.227370495071\\
1193	-181.882173974361\\
1194	-111.925982697545\\
1195	-116.996894105499\\
1196	-194.545562133969\\
1198	-253.321180569697\\
1199	-271.334559888572\\
1200	-273.644236493625\\
1201	-220.166996686381\\
1202	-206.692707399118\\
1203	-225.314671915146\\
1204	-257.110743641904\\
1206	-104.947710910914\\
1207	-154.725134680761\\
1208	-127.0427598572\\
1209	-94.1274850203706\\
1210	-111.508201189256\\
1211	-132.908430402233\\
1212	-108.49968929726\\
1213	-91.239360632401\\
1214	-112.326805527275\\
1215	-99.0781977135923\\
1216	-128.760203161413\\
1217	-178.193282174354\\
1219	-119.300077490081\\
1220	-118.945633812569\\
1221	-167.343493199659\\
1222	-254.474370077357\\
1224	-131.659425078793\\
1225	-93.1696148678352\\
1226	-110.42980982086\\
1227	-113.722225889628\\
1228	-83.3205729684353\\
1229	-89.4424536616682\\
1231	-131.011329106772\\
1232	-143.256871195944\\
1233	-99.6005624580482\\
1234	-165.777795035362\\
1235	-184.646267007019\\
1236	-167.402895719238\\
1237	-156.069857497589\\
1238	-151.876944702146\\
1239	-203.038277205571\\
1241	-52.1924440096814\\
1242	-75.8601830102975\\
1243	-90.993161399675\\
1244	-126.183174633084\\
1245	-110.480168093163\\
1246	-87.4871303330906\\
1247	-128.35993394542\\
1248	-123.422188871953\\
1249	-96.0751800345415\\
1250	-111.509532228151\\
1251	-101.590589092212\\
1252	-98.7122095375021\\
1253	-110.541965195355\\
1254	-65.2684111602432\\
1255	-83.1871193586269\\
1256	-57.1831147003327\\
1257	-66.1445069589224\\
1258	-122.58737699298\\
1259	-125.726044277162\\
1261	-211.812944222738\\
1263	-95.1931191004796\\
1264	-62.576764050674\\
1265	-74.8030979880789\\
1266	-102.186494533485\\
1267	-63.0296079625696\\
1268	-73.2492107400956\\
1269	-93.2110425831274\\
1270	-154.707013424026\\
1271	-140.459150898716\\
1272	-184.688441558762\\
1273	-125.43666010532\\
1274	-126.375378113186\\
1275	-53.579784762991\\
1276	-61.1487999144256\\
1277	-30.1268752013273\\
1278	-44.2437051578838\\
1279	-48.4893969800523\\
1280	-97.3335315466434\\
1281	-101.464501573417\\
1282	-114.958317992413\\
1283	-125.17399899783\\
1284	-187.229905549205\\
1285	-184.322952402596\\
1286	-137.200516776599\\
1287	-155.236760757795\\
1288	-162.677416418589\\
1289	-200.319145122401\\
1291	-115.280857165873\\
1292	-45.8586680999897\\
1293	-25.1147950538464\\
1294	-30.5410735240039\\
1295	-49.2099551494898\\
1296	-54.8168101246517\\
1297	-47.8640255406297\\
1298	-69.20806570796\\
1299	-112.245021930481\\
1300	-105.067316242943\\
1301	-99.7753169209739\\
1302	-118.056414309156\\
1303	-72.4606265799728\\
1304	-34.2472086356108\\
1305	-89.1109647409958\\
1306	-131.502822059699\\
1307	-118.61707095479\\
1308	-132.404283204588\\
1309	-122.244420660296\\
1310	-90.9392370888143\\
1311	-122.339090629224\\
1312	-166.537575593008\\
1313	-160.656128516536\\
1314	-123.625749848343\\
1315	-191.440978782938\\
1316	-154.226044966168\\
1317	-149.384117599934\\
1318	-169.607836149202\\
1319	-159.96304017563\\
1320	-133.497399502222\\
1321	-120.29243097863\\
1323	-243.133837507078\\
1324	-196.088354911589\\
1325	-126.521710548268\\
1326	-119.786349637094\\
1327	-122.257484930297\\
1328	-151.132820604995\\
1329	-163.497082763436\\
1330	-133.391082491102\\
1331	-139.364191699968\\
1332	-173.219758827865\\
1333	-178.032546694327\\
1334	-126.424983213975\\
1335	-107.304464593157\\
1336	-139.838883766983\\
1337	-201.808979815136\\
1338	-192.20026790968\\
1339	-189.894708461745\\
1340	-117.743358822717\\
1341	-109.298496484061\\
1342	-117.20031591442\\
1343	-67.3690349092235\\
1344	-36.0501260139683\\
1345	-20.7561882498294\\
1346	-18.5461925176749\\
1347	-76.639312108591\\
1348	-107.794463573477\\
1349	-122.689864156151\\
1350	-97.8263570221779\\
1352	-117.100154062181\\
1353	-75.4918732557219\\
1354	-80.4229929948292\\
1355	-79.9441256701443\\
1356	-59.3510388548946\\
1357	-83.0817786348957\\
1359	-150.700368062312\\
1360	-96.3990002487928\\
1361	-83.3664358217347\\
1362	-62.005698703153\\
1363	-81.370001613459\\
1364	-52.5570062952297\\
1366	-192.711253547816\\
1367	-161.340511392172\\
1368	-199.852178679743\\
1369	-228.83315427262\\
1370	-230.401348888691\\
1372	-139.267168197548\\
1373	-132.181911620758\\
1374	-139.018966472279\\
1376	-173.022333593456\\
1377	-170.48365375595\\
1378	-219.866827781739\\
1379	-244.184359373487\\
1380	-287.016103134227\\
1381	-209.082299253106\\
1382	-225.969550548494\\
1383	-251.892978000681\\
1384	-198.349043458402\\
1385	-230.873572962284\\
1386	-192.490747649133\\
1387	-119.848529287689\\
1388	-59.9984546272819\\
1389	-70.3009346599645\\
1390	-122.95373670799\\
1391	-104.323289545051\\
1392	-71.5712629149168\\
1393	-93.6499323507533\\
1395	-51.4184740940452\\
1396	-74.7590992359292\\
1397	-74.5617817797852\\
1398	-62.6074501313544\\
1400	-145.393066584891\\
1401	-104.804388589965\\
1403	-243.068842561898\\
1404	-223.656116788186\\
1405	-268.331594598311\\
1406	-196.072605421755\\
1407	-205.50462007241\\
1409	-82.4678639190538\\
1410	-113.181126585173\\
1411	-98.2467441819022\\
1412	-70.7329483494877\\
1413	-100.45193899889\\
1414	-52.2729146722108\\
1415	-31.0294214052449\\
1416	-37.0271684570625\\
1417	-106.720836625809\\
1418	-127.963364287519\\
1419	-153.752802710967\\
1420	-153.190639289618\\
1421	-149.186029234218\\
1422	-158.157772486123\\
1423	-139.725265500493\\
1424	-113.192870610955\\
1425	-118.14573260246\\
1426	-102.575515635868\\
1427	-147.916281154035\\
1428	-101.877291768973\\
1429	-88.1727341091114\\
1431	-164.824407940039\\
1432	-123.584973765956\\
1433	-105.965991429126\\
1434	-91.8213629084973\\
1435	-102.564808761641\\
1436	-78.6866930158517\\
1437	-66.1895057842455\\
1438	-97.0578198446765\\
1439	-114.461819083953\\
1440	-120.842794572804\\
1441	-110.525707525639\\
1442	-123.525803303377\\
1443	-122.231167766686\\
1444	-61.3959435710251\\
1445	-75.6857725121488\\
1446	-144.858593154789\\
1447	-184.770305574069\\
1449	-156.975392916746\\
1450	-151.702260724142\\
1451	-119.036832890748\\
1452	-117.011361936425\\
1453	-98.5147775138603\\
1454	-130.28194147065\\
1455	-123.289933761513\\
1456	-78.0125072508592\\
1457	-121.983208361814\\
1458	-120.778015148778\\
1459	-148.141293643682\\
1460	-155.626237657072\\
1461	-155.017552860762\\
1463	-242.67312253545\\
1464	-281.722156351427\\
1466	-110.466376994711\\
1467	-48.7630729511172\\
1468	-24.8358989976675\\
1469	-68.6299530170179\\
1470	-58.9361258499107\\
1471	-115.657949113317\\
1472	-101.450848137918\\
1473	-99.2239707420308\\
1474	-125.586117412517\\
1475	-141.484537546206\\
1476	-165.758477572652\\
1477	-174.962464398205\\
1478	-242.253060491117\\
1479	-228.272521169602\\
1481	-127.823131264074\\
1482	-125.085295134772\\
1483	-132.546602252132\\
1484	-162.965315709038\\
1485	-122.121567667129\\
1486	-117.691277648927\\
1487	-120.969950603312\\
1488	-51.7620574045532\\
1489	-109.860318958032\\
1490	-152.495960898817\\
1491	-118.766434683824\\
1492	-119.285692196981\\
1493	-210.567487385024\\
1494	-163.974264878427\\
1495	-159.896151949835\\
1496	-206.123864145329\\
1497	-205.125750175049\\
1498	-133.221756388709\\
1499	-90.800038917562\\
1500	-89.5570608704368\\
1502	-224.827491711511\\
1503	-226.810745124581\\
1504	-231.363687068009\\
1505	-187.705259908497\\
1506	-119.419006319111\\
1507	-112.469817980158\\
1508	-74.9652102806276\\
1509	-78.7273941355136\\
1510	-63.3042979924915\\
1511	-100.726441610737\\
1512	-108.421135410228\\
1513	-66.4454792879048\\
1515	-133.267632321669\\
1516	-145.596265735836\\
1518	-217.051718188387\\
1519	-270.753471695916\\
1520	-197.572862514098\\
1521	-178.536759046202\\
1522	-185.84308626213\\
1523	-172.238002712188\\
1524	-118.186887057834\\
1525	-135.183165543245\\
1526	-211.498824371521\\
1527	-226.009745456101\\
1528	-143.914134300377\\
1529	-79.7869138513724\\
1530	-43.0018558619013\\
1531	3.71940897576542\\
1532	-18.5413309828018\\
1533	-55.1283246930745\\
1534	-63.1245539531751\\
1535	-97.540338408246\\
1536	-82.5247865977826\\
1537	-61.0171125219192\\
1538	-67.8351239059868\\
1539	-60.7731084434397\\
1540	-67.5070252852631\\
1541	-70.2047270301546\\
1542	-103.439650588156\\
1543	-149.18088618574\\
1544	-155.192689996941\\
1545	-155.812136156626\\
1546	-171.582642529035\\
1547	-214.570114494524\\
1548	-222.269388798872\\
1549	-250.824656986575\\
1550	-236.385188661907\\
1551	-192.272905425838\\
1552	-137.733964324411\\
1553	-128.562611692331\\
1554	-205.682509067561\\
1555	-221.494288218977\\
1557	-114.655020666315\\
1558	-35.8930041403153\\
1559	-15.3119067249452\\
1560	-32.0032519512749\\
1561	-6.39498688534104\\
1562	-55.0308733068603\\
1563	-80.1798859151797\\
1564	-81.8960152516702\\
1565	-76.3487549519971\\
1566	-40.6849196578755\\
1567	-32.8275584172268\\
1568	-49.3811669751501\\
1569	-57.4739282309115\\
1570	-62.3457365985785\\
1571	-46.0064003967768\\
1572	-38.3465473526126\\
1573	-76.1556743237422\\
1574	-175.082284707694\\
1575	-210.589243023739\\
1576	-205.179080153966\\
1577	-164.281934896043\\
1578	-209.821724891764\\
1579	-274.018452957888\\
1580	-260.004404854101\\
1581	-259.744880674325\\
1582	-203.495703925144\\
1583	-213.242918523729\\
1584	-205.817988235256\\
1585	-155.98776390428\\
1586	-137.683749832145\\
1587	-79.1263557576797\\
1588	-68.4652363658774\\
1589	-144.716781723043\\
1590	-97.4543263098133\\
1592	-120.154618897142\\
1593	-119.688705354891\\
1594	-116.371562079482\\
1596	-140.507846340443\\
1597	-146.431121275206\\
1598	-131.218935877415\\
1599	-104.877133618097\\
1600	-132.921272267032\\
1601	-59.1823259337509\\
1602	-11.4579369384091\\
1603	-29.5049005031606\\
1604	-74.517663475624\\
1605	-27.5143959310053\\
1606	-2.55687034031826\\
1607	-28.5629144324291\\
1608	-69.1463126213202\\
1609	-76.2058172010875\\
1610	-101.519957205873\\
1611	-86.493837701207\\
1612	-144.606136980466\\
1613	-129.526399899475\\
1614	-83.7558787152345\\
1615	-131.474386431832\\
1616	-58.3985121799001\\
1617	-60.6503088634167\\
1618	-27.2564007894964\\
1619	-16.931722427554\\
1620	-44.8162097528252\\
1621	-24.1722718827698\\
1623	-109.044254380261\\
1624	-101.918173251549\\
1625	-77.5815603564572\\
1626	-84.637358051812\\
1627	-65.0237286712311\\
1628	-94.6655030851996\\
1629	-60.3216457683109\\
1630	-60.632388144375\\
1631	-57.2378446856476\\
1632	-47.4540541903675\\
1633	-54.1723573850184\\
1634	-49.4925876876262\\
1635	-93.8747747226953\\
1637	-67.9496625083532\\
1638	-59.9899162606287\\
1639	-50.2235441634184\\
1640	-60.7550409197111\\
1641	-135.571838830798\\
1642	-178.175633792418\\
1643	-123.043186593823\\
1644	-117.007859182734\\
1645	-75.5896778606927\\
1646	-143.521983077267\\
1647	-124.376827802893\\
1648	-82.6189970550856\\
1649	-101.254903755818\\
1650	-76.486519420585\\
1651	-108.344822466927\\
1652	-59.4035513249733\\
1653	-59.3085861306079\\
1654	-82.4991548319117\\
1655	-96.9824039161017\\
1656	-77.142627805822\\
1657	-76.3057938776183\\
1658	-86.0331615387672\\
1659	-128.387523997971\\
1660	-112.153994136065\\
1662	-220.451635156299\\
1663	-183.224452892807\\
1664	-119.008616070849\\
1665	-80.0816990119029\\
1666	-136.562311677659\\
1667	-161.940694555677\\
1668	-127.169286101076\\
1669	-153.126010664913\\
1670	-91.3315991590937\\
1671	-40.1249808947307\\
1672	-42.9290139553971\\
1673	-121.108202016344\\
1675	-96.5834691386012\\
1676	-144.65584252095\\
1677	-139.722907522033\\
1678	-123.488216114443\\
1679	-91.0370679484604\\
1680	-105.657327037889\\
1681	-143.216857651435\\
1682	-111.62416338341\\
1683	-127.383042747583\\
1685	-89.288118716144\\
1686	-103.601072339297\\
1687	-72.3315067491058\\
1688	-84.8888866751879\\
1689	-128.910439175861\\
1690	-150.784042204629\\
1691	-152.66302885164\\
1692	-135.727905742337\\
1694	-73.5109955285561\\
1695	-85.6758468589439\\
1697	-122.13159169083\\
1699	-172.979922231163\\
1700	-141.140351942604\\
1701	-87.3666517251818\\
1703	-208.732010440957\\
1704	-150.36743158274\\
1705	-149.211882552193\\
1706	-169.531713326464\\
1708	-112.585183466665\\
1709	-134.05086095195\\
1710	-117.036334319076\\
1711	-132.063286344634\\
1712	-158.75997936465\\
1713	-124.783378763573\\
1714	-143.113550966566\\
1715	-134.509008522499\\
1716	-139.407117463593\\
1717	-115.240129235648\\
1718	-142.401720870981\\
1719	-110.17368637229\\
1720	-110.305721302346\\
1721	-126.953477329094\\
1722	-117.042068039509\\
1723	-60.6067745037994\\
1724	-20.0627075100888\\
1725	-7.55635433304792\\
1726	-40.4535422850649\\
1727	-126.937496461069\\
1728	-156.521931595342\\
1729	-150.126322015866\\
1730	-99.23367077085\\
1731	-117.946212869116\\
1732	-110.284586266651\\
1733	-100.813668678545\\
1734	-59.6494710783829\\
1735	-88.7282439410053\\
1736	-84.1157329066887\\
1737	-84.486930258231\\
1738	-97.6671352652747\\
1739	-83.3013058680724\\
1740	-81.5141647272219\\
1741	-58.4116972824017\\
1742	-75.8113601975449\\
1743	-137.944243989215\\
1744	-98.8731220891984\\
1745	-120.253750339119\\
1746	-102.645930038029\\
1747	-144.916412707557\\
1748	-129.646177916534\\
1749	-170.006444865264\\
1750	-136.379611109052\\
1752	-147.092836456599\\
1753	-68.1608024507268\\
1754	-136.515638926453\\
1755	-88.1276424519406\\
1756	-116.759068220153\\
1757	-119.081512831084\\
1758	-150.011873843234\\
1759	-118.579898893774\\
1760	-151.220654401967\\
1761	-175.867307535309\\
1762	-157.208000625357\\
1763	-131.712162475851\\
1764	-117.800798854048\\
1765	-131.157458406576\\
1766	-112.219115348509\\
1767	-106.523191406658\\
1768	-85.6704120344518\\
1769	-108.139062670644\\
1770	-91.9797998543909\\
1771	-166.651246994857\\
1772	-203.394140728446\\
1773	-181.24360298119\\
1774	-169.978717603356\\
1775	-176.229783739313\\
1776	-97.2228512823901\\
1777	-94.8981269742981\\
1778	-70.1795355230183\\
1779	-64.6940429242088\\
1780	-78.7205102063383\\
1781	-123.581900434756\\
1783	-156.696989914251\\
1784	-134.363399689335\\
1785	-99.7080368936579\\
1786	-168.154006305237\\
1788	-204.605159321083\\
1789	-164.143673953452\\
1790	-262.225643022105\\
1791	-209.58613547228\\
1792	-120.423919367915\\
1793	-79.4936038834514\\
1794	-62.5248366866076\\
1795	-129.948818351003\\
1796	-155.900843844142\\
1797	-199.884596345352\\
1799	-128.970455073577\\
1800	-61.8084940740246\\
1801	-68.9746064755602\\
1803	-94.3023901370616\\
1804	-56.8488968032398\\
1805	-81.1508170353807\\
};
\addlegendentry{MPO prediction}

\end{axis}

\begin{axis}[%
width=6.159cm,
height=1.831cm,
at={(0cm,2.542cm)},
scale only axis,
xmin=1000,
xmax=2000,
xlabel style={font=\color{white!15!black}},
xlabel={Sample index},
ymin=-400,
ymax=0,
ylabel style={font=\color{white!15!black}},
ylabel={$y(t)$},
axis background/.style={fill=white},
title style={font=\bfseries},
title={C7: RMSE(OSA) = 6.82, RMSE(MPO) = 9.0988},
legend style={legend cell align=left, align=left, draw=white!15!black}
]
\addplot [color=mycolor1, line width=2.0pt]
  table[row sep=crcr]{%
1006	-147.705\\
1007	-175.781\\
1008	-134.277\\
1009	-68.3589999999999\\
1010	-80.566\\
1011	-80.566\\
1012	-100.098\\
1013	-81.787\\
1014	-34.1800000000001\\
1015	-24.414\\
1016	-23.193\\
1018	-128.174\\
1020	-137.939\\
1021	-92.7729999999999\\
1022	-56.152\\
1023	-125.732\\
1024	-86.6700000000001\\
1025	-68.3589999999999\\
1026	-114.746\\
1028	-85.4490000000001\\
1029	-142.822\\
1030	-133.057\\
1031	-102.539\\
1032	-150.146\\
1033	-145.264\\
1034	-113.525\\
1036	-72.021\\
1037	-86.6700000000001\\
1038	-112.305\\
1039	-103.76\\
1040	-109.863\\
1041	-112.305\\
1042	-122.07\\
1043	-175.781\\
1044	-155.029\\
1045	-102.539\\
1046	-92.7729999999999\\
1047	-53.711\\
1048	-58.5940000000001\\
1049	-79.346\\
1050	-59.8140000000001\\
1051	-62.2560000000001\\
1052	-89.1110000000001\\
1053	-102.539\\
1054	-95.2149999999999\\
1055	-151.367\\
1057	-64.6970000000001\\
1058	-51.27\\
1059	-69.5799999999999\\
1060	-81.787\\
1061	-43.9449999999999\\
1062	-45.1659999999999\\
1063	-67.1389999999999\\
1064	-51.27\\
1065	-42.7249999999999\\
1066	-61.0350000000001\\
1067	-70.8009999999999\\
1068	-109.863\\
1069	-104.98\\
1070	-129.395\\
1071	-120.85\\
1072	-137.939\\
1073	-109.863\\
1074	-107.422\\
1075	-92.7729999999999\\
1076	-91.5530000000001\\
1077	-87.8910000000001\\
1079	-224.609\\
1080	-231.934\\
1081	-225.83\\
1082	-140.381\\
1083	-205.078\\
1084	-252.686\\
1085	-261.23\\
1086	-200.195\\
1088	-335.693\\
1089	-235.596\\
1090	-191.65\\
1091	-128.174\\
1092	-98.877\\
1093	-84.229\\
1094	-104.98\\
1095	-74.463\\
1096	-50.049\\
1097	-48.828\\
1098	-63.4770000000001\\
1099	-95.2149999999999\\
1100	-86.6700000000001\\
1101	-89.1110000000001\\
1102	-119.629\\
1103	-123.291\\
1104	-167.236\\
1105	-134.277\\
1106	-169.678\\
1107	-122.07\\
1108	-108.643\\
1109	-125.732\\
1110	-89.1110000000001\\
1111	-80.566\\
1112	-97.6559999999999\\
1113	-98.877\\
1114	-68.3589999999999\\
1115	-73.242\\
1116	-54.932\\
1117	-61.0350000000001\\
1118	-64.6970000000001\\
1119	-45.1659999999999\\
1121	-72.021\\
1122	-117.188\\
1123	-122.07\\
1124	-136.719\\
1125	-76.904\\
1126	-109.863\\
1127	-173.34\\
1128	-131.836\\
1129	-157.471\\
1130	-157.471\\
1131	-85.4490000000001\\
1132	-59.8140000000001\\
1133	-85.4490000000001\\
1134	-98.877\\
1135	-161.133\\
1136	-202.637\\
1137	-198.975\\
1138	-145.264\\
1139	-150.146\\
1140	-135.498\\
1141	-131.836\\
1142	-126.953\\
1143	-85.4490000000001\\
1145	-69.5799999999999\\
1146	-62.2560000000001\\
1147	-70.8009999999999\\
1148	-107.422\\
1149	-164.795\\
1151	-86.6700000000001\\
1152	-73.242\\
1153	-70.8009999999999\\
1154	-42.7249999999999\\
1155	-31.7380000000001\\
1156	-39.0630000000001\\
1157	-72.021\\
1158	-57.373\\
1159	-59.8140000000001\\
1160	-67.1389999999999\\
1161	-63.4770000000001\\
1162	-43.9449999999999\\
1163	-29.297\\
1164	-25.635\\
1165	-43.9449999999999\\
1166	-96.4359999999999\\
1167	-140.381\\
1168	-155.029\\
1169	-101.318\\
1170	-69.5799999999999\\
1171	-50.049\\
1172	-34.1800000000001\\
1173	-83.008\\
1174	-81.787\\
1175	-119.629\\
1176	-145.264\\
1177	-230.713\\
1178	-262.451\\
1179	-211.182\\
1180	-202.637\\
1181	-136.719\\
1182	-151.367\\
1183	-159.912\\
1184	-172.119\\
1185	-129.395\\
1186	-118.408\\
1187	-115.967\\
1188	-103.76\\
1189	-123.291\\
1190	-87.8910000000001\\
1191	-153.809\\
1192	-189.209\\
1194	-78.125\\
1195	-85.4490000000001\\
1196	-146.484\\
1197	-181.885\\
1198	-229.492\\
1199	-241.699\\
1200	-240.479\\
1201	-180.664\\
1202	-163.574\\
1203	-187.988\\
1204	-222.168\\
1205	-139.16\\
1206	-81.787\\
1207	-122.07\\
1208	-102.539\\
1209	-65.9180000000001\\
1210	-87.8910000000001\\
1211	-98.877\\
1212	-76.904\\
1213	-58.5940000000001\\
1214	-80.566\\
1215	-65.9180000000001\\
1216	-91.5530000000001\\
1217	-133.057\\
1218	-122.07\\
1219	-79.346\\
1220	-83.008\\
1221	-126.953\\
1222	-209.961\\
1224	-92.7729999999999\\
1225	-69.5799999999999\\
1226	-83.008\\
1227	-74.463\\
1228	-56.152\\
1229	-62.2560000000001\\
1230	-74.463\\
1231	-96.4359999999999\\
1232	-107.422\\
1233	-72.021\\
1234	-122.07\\
1235	-144.043\\
1236	-137.939\\
1237	-106.201\\
1238	-118.408\\
1239	-158.691\\
1241	-41.5039999999999\\
1242	-58.5940000000001\\
1243	-65.9180000000001\\
1244	-91.5530000000001\\
1245	-81.787\\
1246	-59.8140000000001\\
1247	-96.4359999999999\\
1248	-91.5530000000001\\
1249	-65.9180000000001\\
1250	-83.008\\
1251	-76.904\\
1252	-67.1389999999999\\
1253	-79.346\\
1254	-46.3869999999999\\
1255	-53.711\\
1256	-40.2829999999999\\
1257	-43.9449999999999\\
1258	-86.6700000000001\\
1259	-100.098\\
1261	-177.002\\
1262	-115.967\\
1263	-68.3589999999999\\
1264	-48.828\\
1265	-48.828\\
1266	-73.242\\
1267	-46.3869999999999\\
1268	-52.49\\
1269	-70.8009999999999\\
1270	-119.629\\
1271	-107.422\\
1272	-150.146\\
1273	-102.539\\
1274	-91.5530000000001\\
1275	-47.607\\
1276	-47.607\\
1277	-31.7380000000001\\
1278	-34.1800000000001\\
1279	-41.5039999999999\\
1280	-75.684\\
1281	-83.008\\
1282	-92.7729999999999\\
1283	-97.6559999999999\\
1284	-152.588\\
1285	-130.615\\
1286	-97.6559999999999\\
1287	-119.629\\
1288	-129.395\\
1289	-167.236\\
1290	-133.057\\
1291	-84.229\\
1292	-43.9449999999999\\
1293	-31.7380000000001\\
1294	-34.1800000000001\\
1295	-41.5039999999999\\
1296	-43.9449999999999\\
1297	-40.2829999999999\\
1298	-56.152\\
1299	-87.8910000000001\\
1300	-79.346\\
1301	-81.787\\
1302	-96.4359999999999\\
1303	-58.5940000000001\\
1304	-29.297\\
1306	-97.6559999999999\\
1307	-96.4359999999999\\
1308	-109.863\\
1309	-93.9939999999999\\
1310	-69.5799999999999\\
1311	-97.6559999999999\\
1312	-137.939\\
1313	-139.16\\
1314	-97.6559999999999\\
1315	-162.354\\
1316	-128.174\\
1317	-118.408\\
1318	-137.939\\
1319	-137.939\\
1320	-103.76\\
1321	-90.3320000000001\\
1323	-214.844\\
1324	-164.795\\
1325	-92.7729999999999\\
1326	-92.7729999999999\\
1327	-91.5530000000001\\
1329	-136.719\\
1330	-100.098\\
1331	-104.98\\
1332	-140.381\\
1333	-153.809\\
1334	-103.76\\
1335	-79.346\\
1336	-104.98\\
1337	-175.781\\
1338	-159.912\\
1339	-162.354\\
1340	-95.2149999999999\\
1341	-84.229\\
1342	-83.008\\
1343	-52.49\\
1344	-34.1800000000001\\
1345	-28.076\\
1346	-23.193\\
1347	-59.8140000000001\\
1348	-86.6700000000001\\
1349	-104.98\\
1350	-76.904\\
1352	-97.6559999999999\\
1353	-59.8140000000001\\
1354	-63.4770000000001\\
1355	-61.0350000000001\\
1356	-45.1659999999999\\
1357	-57.373\\
1358	-97.6559999999999\\
1359	-124.512\\
1360	-78.125\\
1361	-56.152\\
1362	-46.3869999999999\\
1363	-59.8140000000001\\
1364	-41.5039999999999\\
1365	-91.5530000000001\\
1366	-158.691\\
1367	-122.07\\
1368	-167.236\\
1369	-194.092\\
1370	-197.754\\
1371	-140.381\\
1372	-106.201\\
1373	-106.201\\
1374	-104.98\\
1375	-120.85\\
1376	-150.146\\
1377	-147.705\\
1378	-200.195\\
1379	-220.947\\
1380	-263.672\\
1381	-178.223\\
1382	-190.43\\
1383	-212.402\\
1384	-163.574\\
1385	-189.209\\
1386	-163.574\\
1387	-91.5530000000001\\
1388	-58.5940000000001\\
1389	-62.2560000000001\\
1390	-95.2149999999999\\
1391	-75.684\\
1392	-51.27\\
1393	-72.021\\
1394	-52.49\\
1395	-40.2829999999999\\
1397	-56.152\\
1398	-40.2829999999999\\
1400	-109.863\\
1401	-78.125\\
1403	-195.313\\
1404	-178.223\\
1405	-222.168\\
1406	-150.146\\
1407	-161.133\\
1409	-67.1389999999999\\
1410	-85.4490000000001\\
1412	-48.828\\
1413	-72.021\\
1414	-41.5039999999999\\
1415	-25.635\\
1416	-36.6210000000001\\
1417	-79.346\\
1419	-128.174\\
1420	-124.512\\
1421	-117.188\\
1422	-136.719\\
1423	-111.084\\
1424	-90.3320000000001\\
1425	-90.3320000000001\\
1426	-73.242\\
1427	-115.967\\
1428	-78.125\\
1429	-67.1389999999999\\
1431	-137.939\\
1432	-103.76\\
1433	-76.904\\
1434	-64.6970000000001\\
1435	-74.463\\
1436	-51.27\\
1437	-50.049\\
1438	-72.021\\
1439	-86.6700000000001\\
1440	-98.877\\
1441	-79.346\\
1442	-101.318\\
1443	-92.7729999999999\\
1444	-52.49\\
1445	-54.932\\
1446	-112.305\\
1447	-158.691\\
1448	-153.809\\
1449	-129.395\\
1450	-122.07\\
1451	-92.7729999999999\\
1452	-89.1110000000001\\
1453	-70.8009999999999\\
1454	-102.539\\
1455	-90.3320000000001\\
1456	-54.932\\
1457	-84.229\\
1459	-118.408\\
1460	-129.395\\
1461	-129.395\\
1463	-216.064\\
1464	-244.141\\
1465	-153.809\\
1466	-85.4490000000001\\
1467	-54.932\\
1468	-37.8420000000001\\
1469	-57.373\\
1470	-53.711\\
1471	-95.2149999999999\\
1473	-75.684\\
1474	-102.539\\
1475	-118.408\\
1476	-141.602\\
1477	-148.926\\
1478	-208.74\\
1479	-181.885\\
1481	-90.3320000000001\\
1483	-95.2149999999999\\
1484	-123.291\\
1485	-87.8910000000001\\
1486	-89.1110000000001\\
1487	-87.8910000000001\\
1488	-47.607\\
1489	-81.787\\
1490	-128.174\\
1491	-90.3320000000001\\
1492	-96.4359999999999\\
1493	-170.898\\
1494	-129.395\\
1495	-122.07\\
1496	-177.002\\
1497	-166.016\\
1498	-102.539\\
1499	-64.6970000000001\\
1500	-65.9180000000001\\
1501	-114.746\\
1502	-190.43\\
1503	-197.754\\
1504	-206.299\\
1506	-97.6559999999999\\
1507	-83.008\\
1508	-58.5940000000001\\
1509	-56.152\\
1510	-47.607\\
1511	-68.3589999999999\\
1512	-81.787\\
1513	-46.3869999999999\\
1514	-70.8009999999999\\
1515	-103.76\\
1516	-114.746\\
1517	-145.264\\
1519	-223.389\\
1520	-161.133\\
1521	-137.939\\
1522	-144.043\\
1523	-120.85\\
1524	-84.229\\
1525	-103.76\\
1526	-167.236\\
1527	-194.092\\
1528	-115.967\\
1529	-62.2560000000001\\
1531	-23.193\\
1532	-31.7380000000001\\
1533	-51.27\\
1534	-61.0350000000001\\
1535	-80.566\\
1536	-67.1389999999999\\
1537	-51.27\\
1539	-48.828\\
1540	-45.1659999999999\\
1541	-52.49\\
1542	-81.787\\
1543	-123.291\\
1544	-133.057\\
1545	-125.732\\
1547	-181.885\\
1548	-183.105\\
1549	-223.389\\
1550	-201.416\\
1551	-146.484\\
1552	-101.318\\
1553	-96.4359999999999\\
1554	-163.574\\
1555	-191.65\\
1556	-131.836\\
1558	-45.1659999999999\\
1559	-31.7380000000001\\
1560	-35.4000000000001\\
1561	-21.973\\
1563	-68.3589999999999\\
1564	-72.021\\
1565	-62.2560000000001\\
1566	-36.6210000000001\\
1567	-28.076\\
1568	-41.5039999999999\\
1569	-43.9449999999999\\
1570	-48.828\\
1571	-36.6210000000001\\
1572	-30.518\\
1573	-54.932\\
1574	-128.174\\
1575	-153.809\\
1576	-170.898\\
1577	-124.512\\
1578	-172.119\\
1579	-241.699\\
1580	-222.168\\
1581	-224.609\\
1582	-164.795\\
1583	-163.574\\
1584	-172.119\\
1585	-113.525\\
1586	-106.201\\
1587	-65.9180000000001\\
1588	-59.8140000000001\\
1589	-117.188\\
1590	-79.346\\
1591	-81.787\\
1592	-96.4359999999999\\
1593	-86.6700000000001\\
1594	-87.8910000000001\\
1595	-93.9939999999999\\
1597	-119.629\\
1598	-103.76\\
1599	-74.463\\
1600	-96.4359999999999\\
1601	-48.828\\
1602	-20.752\\
1603	-34.1800000000001\\
1604	-58.5940000000001\\
1605	-30.518\\
1606	-13.4280000000001\\
1607	-32.9590000000001\\
1608	-58.5940000000001\\
1609	-69.5799999999999\\
1610	-83.008\\
1611	-72.021\\
1612	-117.188\\
1613	-104.98\\
1614	-70.8009999999999\\
1615	-107.422\\
1616	-59.8140000000001\\
1618	-32.9590000000001\\
1619	-21.973\\
1620	-41.5039999999999\\
1621	-34.1800000000001\\
1622	-53.711\\
1623	-91.5530000000001\\
1624	-87.8910000000001\\
1625	-59.8140000000001\\
1626	-68.3589999999999\\
1627	-52.49\\
1628	-74.463\\
1629	-53.711\\
1631	-46.3869999999999\\
1632	-36.6210000000001\\
1633	-45.1659999999999\\
1634	-41.5039999999999\\
1635	-75.684\\
1636	-73.242\\
1637	-54.932\\
1638	-51.27\\
1639	-40.2829999999999\\
1640	-48.828\\
1641	-108.643\\
1642	-144.043\\
1643	-96.4359999999999\\
1644	-90.3320000000001\\
1645	-63.4770000000001\\
1646	-111.084\\
1647	-101.318\\
1648	-67.1389999999999\\
1649	-83.008\\
1650	-64.6970000000001\\
1651	-89.1110000000001\\
1652	-52.49\\
1653	-54.932\\
1654	-65.9180000000001\\
1655	-84.229\\
1656	-62.2560000000001\\
1657	-65.9180000000001\\
1658	-68.3589999999999\\
1659	-112.305\\
1660	-95.2149999999999\\
1662	-190.43\\
1663	-151.367\\
1664	-95.2149999999999\\
1665	-72.021\\
1666	-113.525\\
1667	-141.602\\
1668	-108.643\\
1669	-129.395\\
1670	-81.787\\
1671	-42.7249999999999\\
1672	-42.7249999999999\\
1673	-98.877\\
1675	-79.346\\
1676	-124.512\\
1677	-118.408\\
1678	-103.76\\
1679	-67.1389999999999\\
1680	-84.229\\
1681	-115.967\\
1682	-95.2149999999999\\
1683	-97.6559999999999\\
1684	-89.1110000000001\\
1685	-62.2560000000001\\
1686	-75.684\\
1687	-54.932\\
1688	-62.2560000000001\\
1689	-106.201\\
1690	-123.291\\
1691	-131.836\\
1692	-117.188\\
1693	-83.008\\
1694	-58.5940000000001\\
1695	-67.1389999999999\\
1696	-80.566\\
1699	-155.029\\
1701	-69.5799999999999\\
1703	-178.223\\
1704	-128.174\\
1705	-125.732\\
1706	-147.705\\
1707	-111.084\\
1708	-90.3320000000001\\
1709	-100.098\\
1710	-92.7729999999999\\
1711	-103.76\\
1712	-131.836\\
1713	-101.318\\
1714	-112.305\\
1715	-106.201\\
1716	-108.643\\
1717	-86.6700000000001\\
1718	-114.746\\
1719	-81.787\\
1720	-86.6700000000001\\
1721	-96.4359999999999\\
1722	-93.9939999999999\\
1723	-42.7249999999999\\
1724	-26.855\\
1725	-18.3109999999999\\
1726	-39.0630000000001\\
1727	-96.4359999999999\\
1728	-125.732\\
1729	-125.732\\
1730	-83.008\\
1731	-97.6559999999999\\
1732	-85.4490000000001\\
1733	-75.684\\
1734	-46.3869999999999\\
1735	-65.9180000000001\\
1736	-65.9180000000001\\
1737	-64.6970000000001\\
1738	-76.904\\
1739	-65.9180000000001\\
1740	-59.8140000000001\\
1741	-40.2829999999999\\
1742	-58.5940000000001\\
1743	-109.863\\
1744	-78.125\\
1745	-92.7729999999999\\
1746	-83.008\\
1747	-118.408\\
1748	-109.863\\
1749	-152.588\\
1750	-111.084\\
1751	-124.512\\
1752	-124.512\\
1753	-63.4770000000001\\
1754	-113.525\\
1755	-76.904\\
1756	-96.4359999999999\\
1757	-102.539\\
1758	-129.395\\
1759	-97.6559999999999\\
1760	-125.732\\
1761	-158.691\\
1762	-131.836\\
1763	-114.746\\
1764	-90.3320000000001\\
1765	-107.422\\
1766	-89.1110000000001\\
1767	-76.904\\
1768	-68.3589999999999\\
1769	-81.787\\
1770	-69.5799999999999\\
1771	-141.602\\
1772	-181.885\\
1773	-157.471\\
1774	-150.146\\
1775	-147.705\\
1776	-81.787\\
1777	-72.021\\
1778	-54.932\\
1779	-50.049\\
1780	-54.932\\
1781	-93.9939999999999\\
1782	-118.408\\
1783	-134.277\\
1784	-114.746\\
1785	-76.904\\
1786	-140.381\\
1787	-163.574\\
1788	-183.105\\
1789	-150.146\\
1790	-236.816\\
1791	-158.691\\
1792	-93.9939999999999\\
1793	-68.3589999999999\\
1794	-56.152\\
1795	-101.318\\
1796	-130.615\\
1797	-172.119\\
1798	-129.395\\
1799	-102.539\\
1800	-53.711\\
1801	-53.711\\
1802	-64.6970000000001\\
1803	-70.8009999999999\\
1804	-43.9449999999999\\
1805	-61.0350000000001\\
};
\addlegendentry{True output}

\addplot [color=mycolor2, dashed, line width=2.0pt]
  table[row sep=crcr]{%
1006	-151.11723943552\\
1007	-161.983221323869\\
1008	-137.995196727072\\
1009	-72.7626727761606\\
1010	-85.4686037505749\\
1011	-89.5987934740049\\
1012	-102.898894109405\\
1013	-87.9967473673973\\
1014	-23.8307185378378\\
1015	-9.67533439215072\\
1016	-16.1864695024394\\
1017	-87.0982779529395\\
1018	-135.643161541276\\
1019	-125.15707727804\\
1020	-143.418943774374\\
1021	-92.1394583394563\\
1022	-63.1600405179497\\
1023	-133.158208289984\\
1024	-87.1931484357422\\
1025	-70.5712429881601\\
1026	-112.767241521516\\
1027	-101.340474173738\\
1028	-93.3101722676577\\
1029	-131.519639493086\\
1030	-137.737798969717\\
1031	-100.07498836992\\
1032	-145.228362976789\\
1033	-143.076318707906\\
1034	-119.110267954399\\
1035	-100.161380317569\\
1036	-74.7501867136048\\
1037	-92.7513238581882\\
1038	-107.270623918935\\
1039	-99.1063468155119\\
1040	-109.047248655299\\
1041	-109.840347269106\\
1042	-119.960954714404\\
1043	-164.243041575465\\
1044	-152.251230514262\\
1045	-115.620452565116\\
1046	-97.9590431005943\\
1047	-49.7292327357416\\
1048	-61.4041342746966\\
1049	-80.4889935367025\\
1050	-67.3015006495593\\
1051	-61.3883820285098\\
1052	-92.5283029850461\\
1053	-100.102956007118\\
1054	-96.4499830268971\\
1055	-145.158656235709\\
1056	-114.922593665844\\
1057	-70.43319243822\\
1058	-54.276641957407\\
1059	-73.0552689081753\\
1060	-88.0302811128113\\
1061	-43.9069444456695\\
1062	-45.3546419180313\\
1063	-69.9641284111883\\
1064	-55.0530575697971\\
1065	-46.0432034720193\\
1066	-60.7880006448361\\
1067	-70.7567064927293\\
1068	-112.253803278054\\
1069	-102.853962816952\\
1070	-125.294858217704\\
1071	-118.904254989641\\
1072	-133.148050355062\\
1073	-116.548415000917\\
1074	-105.540527313905\\
1075	-104.193815539815\\
1076	-92.6366903790745\\
1077	-92.2519443095835\\
1079	-209.822644358566\\
1080	-217.849891269364\\
1081	-211.566701849692\\
1082	-153.391535734028\\
1084	-241.245223278482\\
1085	-242.317872234928\\
1086	-213.26322134249\\
1087	-245.282425215254\\
1088	-323.195508377903\\
1089	-256.727340541164\\
1090	-212.088830407031\\
1091	-129.527625974475\\
1092	-105.090779045958\\
1093	-87.1458515885047\\
1094	-112.984033129884\\
1096	-48.5444204321127\\
1097	-44.3726265114065\\
1098	-63.5094015675097\\
1099	-94.9543774647227\\
1100	-92.1382351888944\\
1101	-88.1705346127071\\
1102	-121.74009892591\\
1103	-117.096617264499\\
1104	-173.405606127332\\
1105	-134.193839582948\\
1106	-167.044629411646\\
1107	-130.604471349407\\
1108	-117.338125798235\\
1109	-127.977138920087\\
1110	-101.146071292675\\
1111	-87.8078518346974\\
1112	-100.931967316156\\
1113	-103.222179606532\\
1114	-73.6259018290248\\
1115	-78.9890154295074\\
1116	-59.5127843391342\\
1117	-65.4611995661785\\
1118	-70.2043279253023\\
1119	-52.3221081674385\\
1120	-58.2443421333717\\
1121	-78.9137784709569\\
1122	-119.01521223404\\
1123	-116.82738587693\\
1124	-131.366591762202\\
1125	-80.9890599924342\\
1126	-118.962816059881\\
1127	-169.136832094591\\
1128	-136.457950037825\\
1129	-157.684924880299\\
1130	-149.504678118703\\
1131	-98.0349507461724\\
1132	-66.5818280923593\\
1133	-83.4522543937835\\
1134	-106.324036373426\\
1135	-155.667978578261\\
1136	-191.236930968256\\
1137	-187.342310244252\\
1138	-154.893484232252\\
1140	-145.033737751978\\
1141	-134.99767580854\\
1142	-141.399175478924\\
1143	-89.2570147359261\\
1144	-79.8932073108936\\
1145	-76.1735163550127\\
1146	-66.8338939302112\\
1147	-78.8510560102859\\
1148	-104.741631385721\\
1149	-162.196778886423\\
1151	-96.4144920442061\\
1152	-78.2802583683822\\
1153	-77.0547592642918\\
1154	-44.76499460632\\
1155	-25.653485028593\\
1156	-39.2200475941047\\
1157	-81.6222019401309\\
1158	-53.7297318052154\\
1159	-64.7262535940488\\
1160	-71.5954220969495\\
1161	-64.7714626980026\\
1163	-32.4882764903286\\
1164	-21.5307700916583\\
1165	-45.5904029383985\\
1166	-107.919569851239\\
1167	-138.483991592694\\
1168	-154.77033156541\\
1170	-73.2789566326626\\
1172	-28.8280786008449\\
1173	-92.2599574627191\\
1174	-82.1105002322374\\
1175	-115.48557065217\\
1176	-141.755103221621\\
1177	-221.464965880117\\
1178	-263.255484002109\\
1179	-222.236870010612\\
1180	-203.084769959425\\
1181	-146.697399910002\\
1182	-159.736396149544\\
1183	-162.262546302067\\
1184	-172.431176309208\\
1185	-138.892616522109\\
1186	-127.442464622286\\
1187	-119.646433839042\\
1188	-116.630375599647\\
1189	-121.97112444744\\
1190	-98.261077858471\\
1192	-184.894844093207\\
1193	-132.560180864139\\
1194	-92.0177743861095\\
1195	-77.222059330353\\
1196	-162.289027132403\\
1197	-166.538326254517\\
1198	-214.851057141577\\
1199	-222.433778890919\\
1200	-236.13332586401\\
1201	-195.782659584486\\
1202	-170.285840544235\\
1203	-199.301715661468\\
1204	-202.218777858797\\
1206	-92.8752900843965\\
1207	-120.572509184559\\
1208	-119.465380217355\\
1209	-66.0422382547547\\
1210	-93.9450911278238\\
1211	-100.951038641667\\
1213	-67.4086351152494\\
1214	-84.1728457476922\\
1215	-73.6731645298755\\
1216	-96.3001011784906\\
1217	-132.165288799423\\
1218	-127.217962705065\\
1219	-80.9269882209019\\
1220	-95.8918524145158\\
1221	-121.399374780818\\
1222	-212.661022286512\\
1224	-103.890004625111\\
1225	-70.4268189014592\\
1226	-87.7851516439537\\
1227	-84.8798954441261\\
1228	-62.4558196206949\\
1229	-61.9719460283341\\
1230	-84.7148265578585\\
1231	-100.836710310483\\
1232	-101.490618817484\\
1233	-79.6952425633551\\
1234	-132.431977249603\\
1235	-140.441171507532\\
1236	-131.903351354189\\
1237	-116.353926510348\\
1238	-123.459659086731\\
1239	-154.829391130548\\
1240	-104.629057255707\\
1241	-34.6202571465055\\
1242	-61.8112373681786\\
1243	-75.1158990994556\\
1244	-98.4049015360501\\
1245	-84.8741080811419\\
1246	-66.2634691803948\\
1247	-102.043590633894\\
1248	-90.6170728131585\\
1249	-73.3895725644281\\
1250	-87.2376837351246\\
1251	-81.7069726665418\\
1252	-73.7371152605556\\
1253	-84.2012845186184\\
1254	-49.9317923560145\\
1255	-57.338237619352\\
1256	-47.1221352295984\\
1257	-48.242891227387\\
1258	-96.3956923644218\\
1259	-95.8802564526316\\
1260	-138.059195070432\\
1261	-171.820816416924\\
1262	-124.493267851625\\
1263	-66.1853992898623\\
1264	-54.3719588237427\\
1265	-55.8920041576744\\
1266	-81.779943693422\\
1267	-44.4221766330497\\
1269	-71.8288474244227\\
1270	-127.386058094289\\
1271	-107.122571136299\\
1272	-143.258454654687\\
1273	-109.028999235571\\
1274	-94.4553665322935\\
1275	-45.6410463519139\\
1276	-49.9814421916858\\
1277	-26.8309807321364\\
1278	-38.456440064982\\
1279	-46.659184913796\\
1280	-82.5308020270234\\
1281	-78.0596081319609\\
1282	-98.6438090099748\\
1283	-98.0723482562737\\
1284	-148.348362759229\\
1285	-139.535796951948\\
1286	-106.859081138705\\
1287	-120.719488012673\\
1288	-123.846305644005\\
1289	-158.170916151251\\
1290	-139.124098920907\\
1291	-94.9870525173558\\
1292	-33.3339910304649\\
1293	-28.0481767149745\\
1294	-34.3027666756859\\
1295	-42.8329299555867\\
1296	-45.8252960836501\\
1297	-43.3778531663429\\
1298	-58.0595616087503\\
1299	-93.3431708863088\\
1300	-81.3432279794224\\
1301	-82.7041166393224\\
1302	-99.8715902755916\\
1303	-56.7295568308555\\
1304	-27.2302248767378\\
1306	-102.032647990875\\
1307	-93.5842088629076\\
1308	-109.722117687422\\
1309	-96.1046440044372\\
1310	-72.248596280534\\
1311	-100.421622463213\\
1312	-134.933765101069\\
1313	-136.033960401415\\
1314	-100.153525055106\\
1315	-157.486541589797\\
1316	-132.969115197636\\
1317	-124.787547153505\\
1318	-135.028774900119\\
1319	-131.018379096665\\
1320	-116.012571572406\\
1321	-93.2350009496454\\
1322	-153.705022016753\\
1323	-198.019612662049\\
1324	-168.544161887842\\
1325	-101.779346783535\\
1326	-92.0631172109181\\
1327	-102.068387823065\\
1328	-118.665533949999\\
1329	-132.693300660786\\
1330	-103.849967304143\\
1331	-110.628173283864\\
1332	-135.741623131085\\
1333	-143.610799252852\\
1334	-109.755204889944\\
1335	-84.9757478332269\\
1336	-117.252249074205\\
1337	-167.313666590329\\
1338	-161.525247519645\\
1339	-153.350931084431\\
1340	-99.7265918495139\\
1341	-87.1179205535993\\
1342	-99.5317067763363\\
1343	-45.8081549759977\\
1345	-16.9786157926785\\
1346	-21.0031933360519\\
1347	-66.2543104455544\\
1348	-97.4445889135773\\
1349	-99.1578757891771\\
1350	-80.8082088100655\\
1352	-100.123350339352\\
1353	-64.5223129099386\\
1354	-63.6077562788169\\
1355	-62.1416347447639\\
1356	-49.9710324357388\\
1357	-62.6954300786826\\
1358	-96.6288740959994\\
1359	-121.920664839877\\
1360	-83.4796156561742\\
1361	-57.8611922520636\\
1362	-58.222299723451\\
1363	-58.9968691056617\\
1364	-42.4329382228152\\
1365	-97.2088141045133\\
1366	-161.592703461824\\
1367	-121.258210049495\\
1368	-159.978295152543\\
1369	-185.134229429822\\
1370	-196.800613902859\\
1371	-146.536868523144\\
1372	-117.22357437187\\
1373	-113.094373854753\\
1374	-115.234907782929\\
1375	-121.813108650407\\
1376	-137.348025340792\\
1377	-146.128170602716\\
1380	-248.936318538679\\
1381	-200.168546226462\\
1382	-186.425922440186\\
1383	-217.612372039376\\
1384	-167.165737055849\\
1385	-192.895083089467\\
1386	-165.794211046131\\
1387	-96.8744653360529\\
1388	-50.7891040320105\\
1389	-61.3843116528926\\
1390	-100.776738459086\\
1392	-54.7096231140463\\
1393	-71.9391057271407\\
1394	-63.2131875393347\\
1395	-34.6890319612\\
1396	-57.0605385377814\\
1397	-57.8248947627042\\
1398	-43.2169228382297\\
1399	-86.8821649890692\\
1400	-104.441900254321\\
1401	-84.6730877131831\\
1403	-191.901543836301\\
1404	-179.831343200195\\
1405	-203.663394806748\\
1406	-164.917630230929\\
1407	-167.273830004609\\
1408	-124.092002972129\\
1409	-65.0298143759528\\
1410	-92.7776053565963\\
1411	-69.4454754225476\\
1412	-58.0390595464789\\
1413	-72.1885548387768\\
1414	-35.3482304195129\\
1415	-23.6725652617351\\
1416	-32.2902446963924\\
1417	-95.7471969464934\\
1418	-108.796650855723\\
1419	-125.560574705019\\
1420	-124.294916600536\\
1421	-117.082394057727\\
1422	-136.547323265747\\
1423	-112.902827766254\\
1424	-98.7931114849382\\
1425	-100.550567420432\\
1426	-80.1611734283558\\
1427	-113.565109843263\\
1429	-65.0128936840165\\
1430	-105.511717971018\\
1431	-130.246207425763\\
1432	-111.638557949629\\
1433	-80.6754190338231\\
1434	-75.1785710369295\\
1435	-74.8616875400583\\
1436	-60.4135634480394\\
1437	-50.1567517750557\\
1438	-72.5739436568649\\
1439	-89.802244465395\\
1440	-97.8779226487584\\
1441	-81.8829444629087\\
1442	-100.707477117174\\
1443	-97.4703447047473\\
1444	-55.82823704357\\
1445	-62.7481509203903\\
1446	-115.617017454842\\
1447	-150.013915992626\\
1448	-147.317470901811\\
1449	-131.12562251367\\
1450	-129.260419576501\\
1451	-102.575354978645\\
1452	-92.9035264309298\\
1453	-80.0002832130986\\
1454	-97.9469108008375\\
1455	-98.1544350495985\\
1456	-58.1222082591387\\
1457	-86.9872914005464\\
1458	-102.937888455678\\
1459	-109.669265925073\\
1460	-131.404047898966\\
1461	-120.600441565501\\
1462	-171.244498509571\\
1464	-236.136449865088\\
1465	-176.49380142577\\
1466	-82.6024227464652\\
1468	-28.0274688320521\\
1469	-61.9997457388242\\
1470	-54.6930540368933\\
1471	-96.755385103055\\
1472	-95.4214519495711\\
1473	-79.4968299112211\\
1474	-104.570931770457\\
1475	-116.262949381394\\
1476	-133.656724873857\\
1477	-146.507961667717\\
1478	-202.798668956662\\
1479	-193.846775094304\\
1481	-97.1929842855075\\
1482	-97.3495744355509\\
1483	-101.761314489986\\
1484	-120.455792909437\\
1485	-91.7163389359903\\
1486	-96.3419722504266\\
1487	-89.3107242868273\\
1488	-52.9950906631816\\
1490	-121.860101847493\\
1491	-97.3441468903761\\
1492	-95.8069998500978\\
1493	-173.795916497962\\
1494	-134.434205977568\\
1495	-124.331930733344\\
1496	-166.098005816097\\
1497	-164.086481947059\\
1498	-118.49942936905\\
1499	-62.3492878506984\\
1500	-72.9440851274558\\
1502	-178.848343275708\\
1503	-192.609245446009\\
1504	-189.035749243761\\
1505	-157.926871531038\\
1506	-109.103842831946\\
1507	-84.5392253833274\\
1508	-66.6106205532938\\
1509	-57.453786608547\\
1510	-56.4503755286164\\
1511	-71.6649041778376\\
1512	-83.1895588459847\\
1513	-47.7217033450797\\
1514	-80.903100259183\\
1516	-114.991364205161\\
1517	-143.311098355966\\
1519	-219.196728411464\\
1520	-174.662765043856\\
1521	-146.822757418025\\
1522	-149.12266235735\\
1523	-132.356228741605\\
1524	-87.3712149599398\\
1525	-110.862558311093\\
1526	-153.523129445167\\
1527	-184.460995152376\\
1529	-63.5044182570023\\
1530	-36.3364427941826\\
1531	-2.4709507559171\\
1532	-30.3444455257056\\
1533	-51.8644559746708\\
1534	-62.2579671484486\\
1535	-85.4249558969293\\
1536	-71.982037461117\\
1537	-53.9580680467991\\
1538	-57.9242927438377\\
1539	-52.3099306979861\\
1540	-49.1804583341298\\
1541	-54.4324848222811\\
1543	-121.274764018879\\
1544	-133.211921882614\\
1545	-127.201540525137\\
1546	-148.286478209233\\
1547	-173.319266764375\\
1548	-182.906480942369\\
1549	-212.420282123998\\
1550	-215.408760011563\\
1551	-155.421445583992\\
1552	-108.193307024137\\
1553	-100.328734361177\\
1554	-168.350582526255\\
1555	-177.147142119666\\
1556	-142.238855883815\\
1557	-89.9598526813249\\
1558	-28.0724816974252\\
1559	-24.1783512337963\\
1560	-34.7242980305073\\
1561	-11.5925658406188\\
1562	-52.2785230748023\\
1563	-74.232792898106\\
1564	-67.2811975201025\\
1565	-67.5918449720746\\
1566	-34.4657685559746\\
1567	-32.4378736176793\\
1568	-42.5076216915809\\
1569	-48.8550394874776\\
1570	-48.1369600961134\\
1571	-37.5372219519488\\
1572	-31.1444730706464\\
1573	-61.5939458605276\\
1574	-140.480052591605\\
1575	-161.61028804959\\
1576	-166.794068647675\\
1577	-136.221299935249\\
1578	-164.47233241759\\
1579	-237.326612708478\\
1580	-220.861205011653\\
1581	-216.813759860506\\
1582	-181.67569778438\\
1583	-168.153751267672\\
1584	-173.624919799643\\
1585	-122.89188370037\\
1586	-114.243429256006\\
1587	-59.7279424942255\\
1588	-62.6810015828614\\
1589	-116.842675840091\\
1590	-80.4296152653992\\
1591	-85.6703869366227\\
1592	-101.151497123296\\
1593	-92.2716319082108\\
1594	-91.5628858119439\\
1595	-96.0633919756945\\
1596	-108.811335457291\\
1597	-117.58540704756\\
1598	-110.000311354392\\
1599	-85.0812984126765\\
1600	-96.7760157197927\\
1602	-6.02155906256576\\
1603	-31.518606802083\\
1604	-61.871205717993\\
1605	-23.563528605833\\
1606	-9.73925109996981\\
1607	-28.4240529511198\\
1608	-62.2997549901945\\
1609	-68.9776481983213\\
1610	-93.0333765621147\\
1611	-74.4631741009953\\
1612	-122.923333247456\\
1613	-108.049846201927\\
1614	-72.6310729346678\\
1615	-111.464629474083\\
1616	-53.361070181865\\
1617	-50.2490382169299\\
1618	-28.6887728716194\\
1619	-17.9085056232207\\
1620	-41.8599199522316\\
1621	-28.7342150052173\\
1623	-92.9873573769496\\
1624	-87.5460643949639\\
1625	-67.9305327216507\\
1626	-66.8642542219077\\
1627	-54.0660686397575\\
1628	-74.3495373456813\\
1629	-51.6494809063454\\
1630	-51.2332132508134\\
1631	-49.1311887886679\\
1632	-36.7558871562694\\
1633	-50.0814389582038\\
1634	-43.7292194885215\\
1635	-77.0477469105563\\
1636	-70.7766034148733\\
1637	-56.4624060656954\\
1638	-52.1946060054079\\
1639	-42.9748627975973\\
1640	-51.5817429749243\\
1641	-105.070920229715\\
1642	-145.629728149747\\
1643	-98.4054867313889\\
1644	-98.2609889602768\\
1645	-60.5755499246065\\
1646	-110.78473663893\\
1647	-106.768572266423\\
1648	-64.8576959571271\\
1649	-85.3244602741147\\
1650	-62.9944237381553\\
1651	-90.3114959845407\\
1652	-53.2557239325147\\
1653	-51.3512513702196\\
1654	-73.26360232876\\
1655	-80.8875837758164\\
1656	-63.4470677720956\\
1657	-64.4363163645892\\
1658	-75.6291258632093\\
1659	-106.832238108993\\
1660	-100.846056776882\\
1661	-136.625910457362\\
1662	-190.062948008012\\
1663	-151.842551011614\\
1664	-105.306804564792\\
1665	-71.5367718293021\\
1666	-114.569733509368\\
1667	-136.784975200353\\
1668	-112.586089394\\
1669	-127.169507971154\\
1670	-88.174019796433\\
1671	-33.9316647034241\\
1672	-44.4090340057683\\
1673	-102.353523716835\\
1674	-92.710146542315\\
1675	-77.4316551476263\\
1676	-124.882993876781\\
1677	-113.199300266322\\
1678	-109.100241363351\\
1679	-69.1367577852084\\
1680	-86.0814390951555\\
1681	-119.174440272398\\
1682	-94.7098787049442\\
1683	-98.7531073355046\\
1684	-91.9700298450571\\
1685	-67.9209628311391\\
1686	-78.1483052612373\\
1687	-56.7702454839825\\
1688	-69.30088425237\\
1689	-101.961940256909\\
1690	-118.568458569538\\
1691	-119.927216749009\\
1692	-120.993857367661\\
1693	-87.6786477696849\\
1694	-63.2837729758794\\
1695	-71.3112190177737\\
1696	-85.3099688281186\\
1697	-102.414268974657\\
1698	-124.503632040554\\
1699	-143.094019069712\\
1700	-117.931771084312\\
1701	-77.8801026338181\\
1702	-125.225985985148\\
1703	-181.078347171007\\
1704	-126.651572480675\\
1705	-122.735864115802\\
1706	-143.29386195501\\
1707	-123.397375497553\\
1708	-93.9737925849709\\
1709	-107.536711721088\\
1710	-93.3596668421208\\
1711	-105.365487698898\\
1712	-128.040892884316\\
1713	-101.698385888866\\
1714	-113.086231407315\\
1715	-112.716922540471\\
1716	-107.571518943528\\
1717	-96.8549890899096\\
1718	-112.266470917846\\
1719	-85.2195399616492\\
1720	-90.2426787534262\\
1721	-100.876197871885\\
1722	-93.0453878109076\\
1723	-41.0272909692217\\
1724	-22.1235855928576\\
1725	-7.30680888400434\\
1726	-40.5386970134409\\
1727	-103.942614576183\\
1728	-133.143137049338\\
1729	-122.655213569867\\
1730	-91.3978582373982\\
1731	-98.6256580154857\\
1733	-81.963681956955\\
1734	-49.5463105804067\\
1735	-65.2163668445676\\
1736	-66.5451920679757\\
1737	-65.2278431537297\\
1738	-77.4630191953013\\
1739	-69.5730459114914\\
1740	-65.6408604831383\\
1741	-39.8169301861794\\
1742	-61.0352795116221\\
1743	-112.788561280926\\
1744	-74.9472714709102\\
1745	-97.0061223871207\\
1746	-84.7475585950012\\
1747	-113.808663691861\\
1748	-114.87071992382\\
1749	-136.665633518564\\
1750	-119.968613052142\\
1751	-117.964376400702\\
1752	-126.199775677046\\
1753	-68.1155618270964\\
1754	-110.311978885582\\
1755	-85.5072562940486\\
1756	-94.9759897713882\\
1757	-98.3997243807416\\
1758	-121.535805148937\\
1759	-100.51454216192\\
1761	-146.390886331062\\
1762	-140.239189870437\\
1763	-111.812236682974\\
1764	-102.307928252149\\
1765	-108.497060061745\\
1766	-92.4226622083233\\
1768	-73.5635716490294\\
1769	-83.0833252101904\\
1770	-77.7418893308106\\
1772	-174.427882112191\\
1773	-149.385981672279\\
1774	-151.785358361117\\
1775	-147.236710654958\\
1776	-85.79702752652\\
1777	-76.6385707057163\\
1779	-54.9659580643836\\
1780	-59.0199209809796\\
1781	-93.8420194311184\\
1782	-113.097314423909\\
1783	-127.889493618698\\
1784	-117.378962348783\\
1785	-80.7481692860817\\
1786	-134.479411476903\\
1787	-164.43452953136\\
1788	-167.785502882258\\
1789	-154.59613790062\\
1790	-217.137248625196\\
1791	-176.118513462075\\
1792	-100.975173035974\\
1793	-62.8463207293983\\
1794	-61.2201849998698\\
1795	-103.86382075351\\
1797	-160.548009350228\\
1799	-108.212898376918\\
1800	-57.3989378924671\\
1801	-53.6328071054504\\
1802	-71.2891839837334\\
1803	-74.9243221500983\\
1804	-48.5423790903035\\
1805	-58.3919941835481\\
};
\addlegendentry{OSA predition}

\addplot [color=mycolor3, dotted, line width=2.0pt]
  table[row sep=crcr]{%
1006	-147.705\\
1007	-175.781\\
1008	-134.277\\
1009	-68.3589999999999\\
1010	-85.4686037505746\\
1011	-91.6047705115723\\
1012	-108.027257417993\\
1013	-92.8567005136913\\
1014	-29.4787978305205\\
1015	-9.65065853579017\\
1016	-9.65601966726103\\
1017	-80.5662900232217\\
1018	-134.218821102751\\
1019	-124.71863850667\\
1020	-140.311894734878\\
1021	-93.0218308250635\\
1022	-64.2554583511944\\
1023	-135.84533894183\\
1024	-92.5811203992041\\
1025	-74.4938168879769\\
1026	-117.457135870964\\
1027	-104.534738888153\\
1028	-96.3946435329667\\
1029	-137.389133869682\\
1030	-137.516094707514\\
1031	-102.217499616648\\
1032	-146.317336833964\\
1033	-141.236231088326\\
1034	-117.406666053946\\
1035	-101.104238140385\\
1036	-78.2894051922392\\
1038	-112.7924640166\\
1039	-101.65817957799\\
1040	-109.47198137959\\
1041	-110.376680299447\\
1042	-119.034252423239\\
1043	-162.511015239933\\
1044	-146.234453087022\\
1045	-110.031326719743\\
1046	-99.3858683527071\\
1047	-52.4080330590375\\
1048	-60.7548669179421\\
1049	-82.1602676344057\\
1050	-69.3158642104854\\
1051	-65.5669647077632\\
1052	-95.4270829150851\\
1053	-104.114798979683\\
1054	-99.0305732132258\\
1055	-147.940853728755\\
1056	-114.721012212928\\
1057	-71.7046193150491\\
1058	-58.0451645461296\\
1060	-92.4073927228053\\
1061	-49.8822151212419\\
1062	-49.7825236879337\\
1063	-73.6476539826897\\
1064	-59.5242388024083\\
1065	-50.9003867591412\\
1066	-65.8315263624886\\
1067	-74.7879199783238\\
1068	-116.170233857737\\
1069	-107.16490078462\\
1070	-127.818299000124\\
1071	-119.38229592331\\
1072	-133.148796875553\\
1073	-114.538894116459\\
1074	-106.639083753073\\
1075	-104.146467440436\\
1076	-96.8732509325853\\
1077	-96.2749666863904\\
1079	-211.938461741868\\
1080	-214.237235857703\\
1081	-203.290900441504\\
1082	-141.261486525997\\
1083	-191.585410843642\\
1084	-231.876372792619\\
1085	-229.889164533849\\
1086	-196.719761113574\\
1087	-236.407536070844\\
1088	-302.426703258745\\
1089	-234.957327219621\\
1090	-205.191544265953\\
1091	-132.395222574953\\
1092	-105.481879018029\\
1093	-89.9859676841279\\
1094	-117.672627907726\\
1096	-55.24204424441\\
1097	-48.2268436263316\\
1098	-65.3117973834549\\
1099	-97.0833585201487\\
1100	-93.5882999337771\\
1101	-91.2892047476346\\
1102	-123.82068762579\\
1103	-119.783953532074\\
1104	-173.443567970009\\
1105	-136.828108350636\\
1106	-168.954230401223\\
1107	-130.56002864417\\
1108	-121.41304345593\\
1109	-134.540328276548\\
1110	-106.548839030864\\
1111	-97.2131805145191\\
1112	-111.601992157591\\
1113	-112.795087460492\\
1114	-83.2369896427463\\
1115	-88.876017843183\\
1116	-69.2827446358872\\
1117	-75.0139503561354\\
1118	-79.7469512389862\\
1119	-62.0576409644846\\
1120	-68.8582971146327\\
1121	-87.3774749237416\\
1122	-129.868040952789\\
1123	-126.594977555435\\
1124	-137.316331399899\\
1125	-83.6032706056524\\
1126	-123.278323875847\\
1127	-175.989790010638\\
1128	-139.843355928618\\
1129	-163.25507138857\\
1130	-154.411934298005\\
1131	-97.8661574754112\\
1132	-71.9279295730171\\
1134	-110.301127843215\\
1135	-163.721408583301\\
1136	-195.667138762571\\
1137	-186.745496658829\\
1138	-150.271287897101\\
1139	-150.446281314915\\
1140	-144.720783916804\\
1141	-137.805126816111\\
1142	-145.5671605497\\
1143	-97.7904074744993\\
1144	-88.3046887579419\\
1145	-83.1045897545282\\
1146	-75.7979873220438\\
1147	-87.7444313694837\\
1148	-115.496949777659\\
1149	-170.738091671067\\
1150	-133.252844902431\\
1151	-102.23139106541\\
1152	-86.4342280036908\\
1153	-85.0978219326723\\
1154	-52.8100221311754\\
1155	-32.6730406391232\\
1156	-42.0225360517713\\
1157	-85.0313363397388\\
1158	-60.4428335061632\\
1159	-67.735980716109\\
1160	-76.173910697594\\
1161	-70.8266214193475\\
1163	-38.0120846841512\\
1164	-26.9817186790156\\
1165	-48.1205868509653\\
1166	-111.79130705908\\
1167	-146.315062956166\\
1168	-159.741717438628\\
1169	-116.954820848254\\
1170	-81.8228756020769\\
1171	-60.0092275304748\\
1172	-34.1909683358444\\
1173	-95.725306039822\\
1174	-89.2307199161271\\
1175	-120.962787486691\\
1176	-144.711405692196\\
1177	-223.979056843691\\
1178	-261.417677937934\\
1179	-221.019674047526\\
1180	-206.418783172139\\
1181	-148.940622977912\\
1182	-165.418010868716\\
1183	-170.831671539526\\
1184	-180.02379616507\\
1185	-145.417324955775\\
1186	-136.795453263469\\
1187	-130.620914258518\\
1188	-126.379986981066\\
1189	-135.68339552366\\
1190	-108.538094091439\\
1191	-156.395875767613\\
1192	-192.242109463641\\
1193	-136.699103701596\\
1194	-95.9637349460629\\
1195	-85.2193531152861\\
1196	-165.138721443571\\
1197	-176.523968764176\\
1198	-216.720351174206\\
1199	-218.771208915486\\
1200	-226.327983263188\\
1201	-185.645648668569\\
1202	-168.408141226146\\
1203	-198.893851152826\\
1204	-206.320224061865\\
1205	-142.566258359785\\
1206	-92.3070951871255\\
1207	-126.249236628219\\
1208	-121.120088962143\\
1209	-74.5592939087821\\
1210	-101.059265556851\\
1211	-108.660899028457\\
1213	-76.2473671720902\\
1214	-95.068579452553\\
1215	-83.3046605193463\\
1216	-107.961178133366\\
1217	-144.255680625751\\
1218	-136.899311866925\\
1219	-90.6764701422444\\
1220	-104.385125357734\\
1221	-133.610059654144\\
1222	-221.327619946146\\
1224	-114.418815171157\\
1225	-81.8709082612691\\
1226	-96.7950078597885\\
1227	-94.2354400565598\\
1228	-74.6504661856272\\
1229	-73.5134953796644\\
1231	-113.593184151233\\
1232	-113.620015214967\\
1233	-86.2596328992524\\
1234	-142.465752401784\\
1235	-152.860935122959\\
1236	-139.90394603898\\
1237	-120.701025455324\\
1238	-132.084856319116\\
1239	-163.33261848046\\
1240	-108.999879956698\\
1241	-39.124584653661\\
1242	-63.0863575498852\\
1243	-77.2260076726686\\
1244	-104.129184549593\\
1245	-91.455445382242\\
1246	-72.6010533929959\\
1247	-110.425501263298\\
1248	-99.7046348648748\\
1249	-79.8827095282488\\
1250	-95.8815118761283\\
1251	-90.4170540078476\\
1252	-82.1930001670444\\
1253	-94.0500872081595\\
1254	-59.2105550992487\\
1255	-66.0176840380182\\
1256	-55.3583382886147\\
1257	-57.636259994802\\
1258	-106.157433776608\\
1259	-108.14146262221\\
1261	-180.569570534273\\
1263	-72.9296273767034\\
1264	-58.7856693840797\\
1265	-60.6682042397006\\
1266	-89.4109010916434\\
1267	-53.128393012325\\
1268	-63.4770419941165\\
1269	-79.0905989821133\\
1270	-134.477622163534\\
1271	-115.967809704117\\
1272	-150.608328770823\\
1273	-111.915864812322\\
1274	-100.087857875543\\
1275	-51.1633682965007\\
1276	-52.3473484914064\\
1277	-29.8533853555905\\
1279	-48.9455376157009\\
1280	-86.4228888348368\\
1281	-83.5585836400367\\
1282	-101.324096453653\\
1283	-103.237708768236\\
1284	-152.778554292585\\
1285	-141.198508271015\\
1286	-112.387279895537\\
1287	-128.650062349794\\
1288	-130.004875840347\\
1289	-162.003671961967\\
1290	-139.157963176215\\
1291	-97.7234805362932\\
1292	-39.9848604460738\\
1293	-27.2499487027387\\
1294	-32.1041635958179\\
1295	-42.7342391127661\\
1296	-45.5200383520498\\
1297	-43.6298842968301\\
1298	-59.4733506380983\\
1299	-95.359854537229\\
1300	-85.2237345731473\\
1301	-86.5440759829919\\
1302	-103.392309945473\\
1303	-60.9964443218526\\
1304	-29.481652132361\\
1306	-104.067464661087\\
1307	-96.7697962604066\\
1308	-110.888756647637\\
1309	-97.2186585983559\\
1310	-74.3340263764494\\
1311	-102.848963162292\\
1312	-138.04617126224\\
1313	-137.517662857851\\
1314	-100.243708281599\\
1315	-158.862323462649\\
1316	-131.779654254144\\
1317	-125.82729671\\
1318	-138.569600634872\\
1319	-132.078513143968\\
1321	-97.5692237078729\\
1322	-157.596555850099\\
1323	-201.302535156389\\
1324	-165.283108660214\\
1325	-101.00974311065\\
1326	-96.0000049311811\\
1327	-103.151482382495\\
1328	-124.347731724666\\
1329	-139.669202871685\\
1330	-107.329262485936\\
1331	-115.636352475229\\
1332	-142.306215311941\\
1333	-146.627024998731\\
1334	-108.229819962204\\
1335	-86.946504094579\\
1336	-120.832326686338\\
1337	-174.543102493447\\
1338	-164.285035715383\\
1339	-157.157155758977\\
1340	-99.7298735506627\\
1341	-88.111775067561\\
1342	-102.068765178198\\
1343	-53.5090092991281\\
1345	-17.0062078762746\\
1346	-18.4732000542047\\
1347	-63.2171046798585\\
1348	-97.1136200538156\\
1349	-102.085390474884\\
1350	-80.6004546604984\\
1352	-103.15854540047\\
1353	-67.0839304308627\\
1354	-67.6997588594249\\
1355	-65.2678354741852\\
1356	-52.8486504038406\\
1357	-67.2263488374078\\
1358	-102.344592268619\\
1359	-126.314188477284\\
1360	-86.1950781446701\\
1361	-62.4150645459645\\
1362	-62.2499674225703\\
1363	-66.580583006443\\
1364	-47.9529451547312\\
1365	-102.647737187988\\
1366	-169.517842341099\\
1367	-128.424495489548\\
1368	-165.903019608553\\
1369	-187.634038106004\\
1370	-195.756965827951\\
1371	-145.442141486642\\
1372	-118.822085461991\\
1373	-118.274874161837\\
1374	-121.794829054061\\
1375	-131.484053397151\\
1376	-146.029743939117\\
1377	-148.318263839049\\
1378	-182.374423172499\\
1379	-205.15453260821\\
1380	-239.824375666554\\
1381	-187.046715289805\\
1382	-184.181530116077\\
1383	-213.106313519371\\
1384	-164.5686560897\\
1385	-193.841555962699\\
1386	-167.238541599867\\
1387	-99.0582388265825\\
1388	-54.1915717556835\\
1389	-60.9181671364529\\
1390	-100.296842857275\\
1391	-81.5514138066587\\
1392	-57.2659629590967\\
1393	-75.1395078623898\\
1394	-65.8648687635366\\
1395	-41.1267420573568\\
1396	-59.6880272580552\\
1397	-63.1515773807826\\
1398	-48.6876662908096\\
1399	-92.1035495985368\\
1400	-113.432188303965\\
1401	-89.2383758912795\\
1403	-199.046703013571\\
1404	-184.315985368434\\
1405	-208.893066641537\\
1406	-161.399990876133\\
1407	-170.80757636587\\
1408	-129.970051547803\\
1409	-72.3413347065205\\
1410	-98.5656374385098\\
1411	-76.4218233371055\\
1412	-65.3881249030426\\
1413	-81.0767622252704\\
1414	-42.32805735944\\
1415	-25.5856820581755\\
1416	-33.9053398195997\\
1417	-95.5986286811853\\
1418	-115.217061973091\\
1419	-131.158819339393\\
1420	-128.122740429105\\
1421	-121.225790263112\\
1422	-139.883390489676\\
1423	-115.169447441689\\
1424	-101.348684642178\\
1425	-106.085865452844\\
1426	-88.4022718245612\\
1427	-122.892334407663\\
1428	-94.4855048665036\\
1429	-74.4321386670329\\
1430	-112.43702859405\\
1431	-138.358348287812\\
1432	-115.495696145196\\
1433	-86.6344790979774\\
1434	-81.8755262342743\\
1435	-83.3866556443411\\
1436	-67.4980707769028\\
1437	-59.0642127018309\\
1438	-80.3142204202277\\
1439	-96.3115957337184\\
1440	-105.193358536218\\
1441	-86.9831352662293\\
1442	-106.042855325077\\
1443	-101.6386591744\\
1444	-60.7277589165312\\
1445	-68.0253391538045\\
1446	-123.165259157016\\
1447	-157.996573006865\\
1448	-150.807845597908\\
1449	-131.93951375\\
1450	-130.955843885811\\
1451	-106.406605455135\\
1452	-99.3778102982042\\
1453	-86.3165326586013\\
1454	-107.103987783564\\
1455	-103.847012598497\\
1456	-65.4517216441086\\
1457	-94.6968881546143\\
1458	-109.813873530908\\
1459	-116.626920392125\\
1460	-133.737880676677\\
1461	-123.793337902353\\
1462	-170.144962260344\\
1463	-202.266037188648\\
1464	-229.398610018034\\
1465	-167.713841203445\\
1466	-86.1664065736006\\
1468	-26.9398378377582\\
1469	-59.3412619413714\\
1470	-54.2995781345203\\
1471	-96.5603827524781\\
1472	-95.1362070385919\\
1473	-83.8520267284898\\
1474	-109.245606223134\\
1475	-120.898111016488\\
1476	-137.241870268561\\
1477	-146.475263383136\\
1478	-202.243103901877\\
1479	-190.639450624871\\
1481	-103.499300802691\\
1482	-103.817293603741\\
1483	-109.146664707821\\
1484	-129.690031659257\\
1485	-97.8304814412013\\
1486	-102.882349191968\\
1487	-97.7977023096157\\
1488	-59.3674577083148\\
1490	-130.933571209137\\
1491	-101.886523429186\\
1492	-102.823769985178\\
1493	-179.494115563346\\
1494	-139.983268382308\\
1495	-131.32517629978\\
1496	-172.604941933744\\
1497	-165.146115419852\\
1499	-69.9881923200955\\
1500	-76.8647498628577\\
1502	-188.744277685394\\
1503	-195.969170960677\\
1504	-190.963797710269\\
1505	-152.818985022547\\
1506	-107.969822484255\\
1508	-68.9585833011088\\
1509	-62.6517770658018\\
1510	-61.470056733667\\
1511	-79.1887009662082\\
1512	-90.7454900432735\\
1513	-53.6539331458014\\
1514	-86.9132642306531\\
1515	-107.613480129708\\
1516	-120.083528388404\\
1517	-148.565327082258\\
1519	-220.174077409775\\
1520	-173.974973723369\\
1521	-151.913835124236\\
1522	-156.326509305249\\
1523	-139.319014877946\\
1524	-97.8500863103925\\
1525	-120.619082033154\\
1526	-164.933294759423\\
1527	-189.136985430906\\
1528	-120.601803583157\\
1529	-66.4818394682304\\
1530	-38.7838218565917\\
1531	-0.955754744427395\\
1532	-21.1861002777919\\
1533	-45.2491360461327\\
1534	-56.7279563412376\\
1535	-79.6886111704237\\
1536	-69.6468651103696\\
1537	-54.1318084450913\\
1538	-58.9407352062935\\
1540	-53.5879183765826\\
1541	-59.3858326625318\\
1543	-128.138703003307\\
1544	-138.129091809071\\
1545	-131.580914640203\\
1547	-175.131557704816\\
1548	-181.273917319333\\
1549	-211.407454295146\\
1550	-209.603770544869\\
1551	-156.435058899319\\
1552	-113.263422996408\\
1553	-105.29758852229\\
1554	-175.131788375771\\
1555	-185.288233392823\\
1557	-94.5679993540266\\
1558	-34.9006260549947\\
1559	-19.6474438680461\\
1560	-29.1979732132384\\
1561	-8.91437417523116\\
1562	-44.7344999431946\\
1563	-70.3672211567618\\
1564	-66.1197421262702\\
1565	-63.9962216321414\\
1566	-34.3389755643859\\
1567	-31.9547009733039\\
1568	-43.2736027011124\\
1569	-50.0886306251216\\
1570	-50.8330714936499\\
1571	-39.4787986868823\\
1572	-32.8009981646212\\
1573	-63.5741954737789\\
1574	-144.929702254778\\
1575	-170.127253334415\\
1576	-177.071478963723\\
1577	-142.998454044305\\
1578	-175.838485790895\\
1579	-243.686489601513\\
1580	-225.044348443695\\
1581	-220.468012107115\\
1582	-180.829066273095\\
1583	-174.439995356246\\
1584	-180.367226487178\\
1585	-127.510007581545\\
1586	-122.818826704467\\
1587	-69.2190093802565\\
1588	-66.7278139314881\\
1589	-122.434198110027\\
1590	-85.2152286081607\\
1591	-89.3989949897568\\
1592	-105.970149727999\\
1593	-97.9261166897188\\
1594	-98.3278437209528\\
1595	-102.978595759398\\
1596	-115.460591300068\\
1597	-123.881771312617\\
1598	-114.323188265503\\
1599	-91.0622894525027\\
1600	-105.872755174616\\
1602	-10.3783945038799\\
1603	-29.5825298180521\\
1604	-59.9398979241043\\
1605	-24.2057431332853\\
1606	-6.3769721854178\\
1607	-24.3627627096801\\
1608	-57.7112400074902\\
1609	-66.3605103067343\\
1610	-89.8574392830667\\
1611	-75.9895034657661\\
1612	-124.862879258595\\
1613	-111.937953942146\\
1614	-77.2180263144674\\
1615	-115.836217763559\\
1616	-58.3552043611646\\
1617	-51.3262011702593\\
1618	-30.9164942739394\\
1619	-18.5440901299542\\
1620	-40.1718252957419\\
1621	-28.2585786149339\\
1622	-56.9123776450801\\
1623	-93.1607910436421\\
1624	-87.7743327046633\\
1625	-67.6781599716271\\
1626	-70.4466889713929\\
1627	-56.0074716514896\\
1628	-76.4172923354199\\
1629	-53.6918124176043\\
1630	-51.7487577278607\\
1631	-50.2157915082769\\
1632	-38.7099895933595\\
1633	-51.4296484150659\\
1634	-46.8301419795487\\
1635	-80.4794550372244\\
1636	-74.0181560176634\\
1637	-58.1974604734812\\
1638	-54.2750699249657\\
1639	-45.066384879213\\
1640	-54.0637931811759\\
1641	-108.40340856019\\
1642	-147.157042931782\\
1643	-100.647365244528\\
1644	-100.856880744175\\
1645	-65.373029904315\\
1646	-113.412882792665\\
1647	-109.074370638774\\
1648	-69.406646593827\\
1649	-87.474964900126\\
1650	-65.4194441098455\\
1651	-92.0381112749228\\
1652	-54.8392380323669\\
1653	-53.1100113273783\\
1654	-72.9862736865407\\
1655	-84.0077359712172\\
1656	-64.3099587857184\\
1657	-65.3437400351147\\
1658	-76.326474282811\\
1659	-110.191679220987\\
1660	-101.062575777933\\
1661	-139.636597614291\\
1662	-188.457510071935\\
1663	-150.569656762099\\
1664	-105.100805115948\\
1665	-74.8430935364615\\
1666	-116.790088541223\\
1667	-139.186079754978\\
1668	-113.171942645168\\
1669	-129.239534396363\\
1670	-88.9651271877433\\
1671	-36.4091562075118\\
1672	-43.1897263869587\\
1673	-102.169796446999\\
1674	-94.2865607965953\\
1675	-79.336197834828\\
1676	-125.856342945761\\
1677	-114.453456031805\\
1678	-108.155225874337\\
1679	-70.4560674107545\\
1680	-88.021329528659\\
1681	-121.057624302162\\
1682	-97.8889956654559\\
1683	-101.021898294514\\
1684	-94.2551889564552\\
1685	-70.9834174176092\\
1686	-82.722780273971\\
1687	-61.1125337101114\\
1688	-73.4985775209127\\
1689	-108.733924167913\\
1690	-122.262255495584\\
1691	-121.511424728258\\
1692	-117.936602853533\\
1693	-86.922124374727\\
1694	-64.9319993560662\\
1695	-73.3826375961889\\
1696	-88.8897091097379\\
1699	-143.755052066172\\
1700	-113.952298523697\\
1701	-76.3673658217022\\
1703	-180.751310471157\\
1704	-128.188785754113\\
1705	-123.732550439436\\
1706	-142.617389583831\\
1708	-97.4631162470621\\
1709	-111.398583521693\\
1710	-98.6964956072497\\
1711	-110.599755905017\\
1712	-133.163964744018\\
1713	-104.510671947577\\
1714	-115.532169994074\\
1715	-115.184635417487\\
1716	-111.945535862186\\
1717	-99.7773253298492\\
1718	-118.876265265792\\
1719	-89.5452830989407\\
1720	-94.7493195963868\\
1721	-106.657582494455\\
1722	-99.0162543629142\\
1723	-45.0108232649118\\
1724	-23.9662430902499\\
1725	-7.07073686337662\\
1726	-36.1001644068567\\
1727	-101.34761463222\\
1728	-132.609617902361\\
1729	-124.45527072749\\
1730	-91.6906702276322\\
1731	-102.787990688485\\
1732	-93.7503536058289\\
1733	-86.1553959550274\\
1734	-55.7074275876821\\
1735	-70.9927640925257\\
1736	-70.90376601875\\
1737	-69.305178806339\\
1738	-81.1539710300881\\
1739	-72.5875690413775\\
1740	-69.5535532179917\\
1741	-45.0716232566042\\
1742	-64.7797173423614\\
1743	-117.534679980679\\
1744	-80.106858696513\\
1745	-99.6911472399722\\
1746	-88.7855379595644\\
1747	-117.920285118162\\
1748	-116.096794555158\\
1749	-140.409688961801\\
1750	-116.294188919952\\
1751	-118.885425114622\\
1752	-124.506747168456\\
1753	-66.1767763303153\\
1754	-111.659046121415\\
1755	-84.5283806938646\\
1756	-98.0197610440816\\
1757	-100.152530189679\\
1758	-121.160581403804\\
1759	-97.6909507024161\\
1760	-123.833849991173\\
1761	-144.665747941135\\
1762	-133.453863661572\\
1763	-110.629335404296\\
1764	-100.316618888021\\
1765	-110.727818749449\\
1766	-95.0661136782132\\
1767	-85.9752298441947\\
1768	-79.0776409589241\\
1769	-89.1507013599494\\
1770	-83.0209344128953\\
1771	-137.622212883315\\
1772	-176.511321258626\\
1773	-148.888464028964\\
1774	-148.612885903819\\
1775	-144.815335063003\\
1776	-83.948486211068\\
1777	-76.3183411451769\\
1778	-67.3814005671504\\
1779	-60.313911954938\\
1780	-65.0559457900256\\
1781	-100.568285446563\\
1782	-119.268125075199\\
1783	-131.161274997786\\
1784	-117.713241331246\\
1785	-82.2407605129031\\
1786	-136.905733128809\\
1787	-163.676521271138\\
1788	-168.323253492348\\
1789	-148.812662227728\\
1790	-214.336305438218\\
1791	-165.773603700777\\
1792	-99.4689762458092\\
1793	-66.8926003782681\\
1794	-58.8423182143449\\
1795	-105.721494813303\\
1797	-162.744628126298\\
1798	-130.963192715724\\
1799	-108.221848137398\\
1800	-60.5446271369192\\
1801	-56.1950364779473\\
1802	-73.4000820953311\\
1803	-79.6513401100169\\
1804	-53.9385721640506\\
1805	-64.0595804565251\\
};
\addlegendentry{MPO prediction}

\end{axis}

\begin{axis}[%
width=6.159cm,
height=1.831cm,
at={(8.104cm,2.542cm)},
scale only axis,
xmin=1000,
xmax=2000,
xlabel style={font=\color{white!15!black}},
xlabel={Sample index},
ymin=-279.541,
ymax=0,
ylabel style={font=\color{white!15!black}},
ylabel={$y(t)$},
axis background/.style={fill=white},
title style={font=\bfseries},
title={C8: RMSE(OSA) = 6.0607, RMSE(MPO) = 7.4879},
legend style={legend cell align=left, align=left, draw=white!15!black}
]
\addplot [color=mycolor1, line width=2.0pt]
  table[row sep=crcr]{%
1006	-125.732\\
1007	-153.809\\
1008	-115.967\\
1009	-61.0350000000001\\
1010	-74.463\\
1011	-73.242\\
1012	-86.6700000000001\\
1013	-73.242\\
1014	-34.1800000000001\\
1015	-20.752\\
1016	-23.193\\
1017	-58.5940000000001\\
1018	-100.098\\
1019	-109.863\\
1020	-114.746\\
1022	-52.49\\
1023	-104.98\\
1025	-59.8140000000001\\
1026	-97.6559999999999\\
1027	-83.008\\
1028	-72.021\\
1029	-118.408\\
1030	-107.422\\
1031	-83.008\\
1032	-119.629\\
1033	-123.291\\
1034	-95.2149999999999\\
1035	-80.566\\
1036	-62.2560000000001\\
1037	-75.684\\
1038	-95.2149999999999\\
1039	-87.8910000000001\\
1040	-91.5530000000001\\
1041	-96.4359999999999\\
1042	-102.539\\
1043	-141.602\\
1044	-128.174\\
1045	-85.4490000000001\\
1046	-79.346\\
1047	-47.607\\
1048	-48.828\\
1049	-68.3589999999999\\
1050	-52.49\\
1051	-57.373\\
1052	-75.684\\
1053	-89.1110000000001\\
1054	-81.787\\
1055	-128.174\\
1056	-98.877\\
1057	-57.373\\
1058	-45.1659999999999\\
1059	-62.2560000000001\\
1060	-72.021\\
1061	-40.2829999999999\\
1062	-42.7249999999999\\
1063	-56.152\\
1064	-46.3869999999999\\
1065	-40.2829999999999\\
1066	-52.49\\
1067	-62.2560000000001\\
1068	-92.7729999999999\\
1069	-90.3320000000001\\
1070	-107.422\\
1071	-98.877\\
1072	-115.967\\
1073	-91.5530000000001\\
1074	-91.5530000000001\\
1075	-79.346\\
1076	-75.684\\
1077	-76.904\\
1079	-189.209\\
1080	-194.092\\
1081	-189.209\\
1082	-119.629\\
1083	-170.898\\
1084	-213.623\\
1085	-219.727\\
1086	-169.678\\
1087	-219.727\\
1088	-279.541\\
1089	-197.754\\
1090	-161.133\\
1091	-109.863\\
1092	-85.4490000000001\\
1093	-73.242\\
1094	-91.5530000000001\\
1095	-62.2560000000001\\
1096	-46.3869999999999\\
1097	-42.7249999999999\\
1098	-54.932\\
1099	-79.346\\
1100	-74.463\\
1101	-75.684\\
1102	-100.098\\
1103	-103.76\\
1104	-141.602\\
1105	-114.746\\
1106	-142.822\\
1107	-104.98\\
1108	-90.3320000000001\\
1109	-103.76\\
1110	-76.904\\
1111	-69.5799999999999\\
1112	-84.229\\
1113	-86.6700000000001\\
1114	-59.8140000000001\\
1115	-64.6970000000001\\
1116	-48.828\\
1117	-53.711\\
1118	-57.373\\
1119	-41.5039999999999\\
1120	-52.49\\
1121	-65.9180000000001\\
1122	-101.318\\
1123	-104.98\\
1124	-114.746\\
1125	-68.3589999999999\\
1126	-93.9939999999999\\
1127	-150.146\\
1128	-114.746\\
1129	-125.732\\
1130	-128.174\\
1131	-80.566\\
1132	-53.711\\
1133	-75.684\\
1134	-85.4490000000001\\
1135	-137.939\\
1136	-172.119\\
1137	-170.898\\
1138	-123.291\\
1139	-130.615\\
1140	-115.967\\
1142	-111.084\\
1143	-76.904\\
1144	-68.3589999999999\\
1145	-61.0350000000001\\
1146	-57.373\\
1147	-62.2560000000001\\
1148	-91.5530000000001\\
1149	-140.381\\
1151	-79.346\\
1152	-64.6970000000001\\
1153	-62.2560000000001\\
1154	-40.2829999999999\\
1155	-29.297\\
1156	-36.6210000000001\\
1157	-63.4770000000001\\
1158	-51.27\\
1159	-52.49\\
1160	-58.5940000000001\\
1161	-54.932\\
1162	-37.8420000000001\\
1163	-28.076\\
1164	-23.193\\
1165	-39.0630000000001\\
1166	-83.008\\
1167	-117.188\\
1168	-130.615\\
1169	-86.6700000000001\\
1170	-58.5940000000001\\
1172	-32.9590000000001\\
1173	-67.1389999999999\\
1174	-65.9180000000001\\
1176	-117.188\\
1177	-186.768\\
1178	-216.064\\
1179	-177.002\\
1180	-168.457\\
1181	-109.863\\
1182	-122.07\\
1183	-131.836\\
1184	-146.484\\
1185	-108.643\\
1186	-100.098\\
1187	-98.877\\
1188	-89.1110000000001\\
1189	-106.201\\
1190	-78.125\\
1191	-130.615\\
1192	-159.912\\
1194	-70.8009999999999\\
1195	-73.242\\
1196	-123.291\\
1197	-153.809\\
1198	-189.209\\
1199	-202.637\\
1200	-202.637\\
1201	-153.809\\
1202	-137.939\\
1203	-158.691\\
1204	-187.988\\
1205	-109.863\\
1206	-72.021\\
1207	-101.318\\
1208	-86.6700000000001\\
1209	-53.711\\
1210	-74.463\\
1211	-84.229\\
1213	-51.27\\
1214	-70.8009999999999\\
1215	-58.5940000000001\\
1216	-79.346\\
1217	-107.422\\
1218	-100.098\\
1219	-69.5799999999999\\
1220	-70.8009999999999\\
1221	-107.422\\
1222	-177.002\\
1224	-79.346\\
1225	-61.0350000000001\\
1226	-70.8009999999999\\
1227	-64.6970000000001\\
1228	-48.828\\
1229	-53.711\\
1230	-65.9180000000001\\
1231	-83.008\\
1232	-91.5530000000001\\
1233	-63.4770000000001\\
1234	-104.98\\
1235	-124.512\\
1236	-118.408\\
1237	-91.5530000000001\\
1238	-100.098\\
1239	-135.498\\
1241	-42.7249999999999\\
1242	-57.373\\
1243	-62.2560000000001\\
1244	-81.787\\
1245	-72.021\\
1246	-52.49\\
1247	-79.346\\
1248	-80.566\\
1249	-57.373\\
1250	-70.8009999999999\\
1252	-56.152\\
1253	-67.1389999999999\\
1254	-41.5039999999999\\
1255	-48.828\\
1256	-36.6210000000001\\
1257	-40.2829999999999\\
1258	-75.684\\
1259	-84.229\\
1260	-113.525\\
1261	-146.484\\
1262	-98.877\\
1263	-58.5940000000001\\
1264	-46.3869999999999\\
1265	-43.9449999999999\\
1266	-63.4770000000001\\
1267	-42.7249999999999\\
1268	-47.607\\
1269	-61.0350000000001\\
1270	-102.539\\
1271	-93.9939999999999\\
1272	-134.277\\
1273	-89.1110000000001\\
1274	-81.787\\
1275	-46.3869999999999\\
1276	-45.1659999999999\\
1277	-28.076\\
1278	-35.4000000000001\\
1279	-37.8420000000001\\
1280	-67.1389999999999\\
1281	-70.8009999999999\\
1282	-79.346\\
1283	-85.4490000000001\\
1284	-125.732\\
1285	-112.305\\
1286	-84.229\\
1287	-102.539\\
1288	-109.863\\
1289	-144.043\\
1290	-115.967\\
1291	-75.684\\
1292	-40.2829999999999\\
1293	-29.297\\
1294	-29.297\\
1295	-36.6210000000001\\
1296	-39.0630000000001\\
1297	-37.8420000000001\\
1298	-50.049\\
1299	-74.463\\
1300	-68.3589999999999\\
1301	-70.8009999999999\\
1302	-83.008\\
1303	-51.27\\
1304	-30.518\\
1305	-58.5940000000001\\
1306	-90.3320000000001\\
1307	-87.8910000000001\\
1308	-97.6559999999999\\
1309	-83.008\\
1310	-61.0350000000001\\
1311	-85.4490000000001\\
1312	-118.408\\
1313	-118.408\\
1314	-84.229\\
1315	-136.719\\
1316	-100.098\\
1317	-101.318\\
1318	-117.188\\
1319	-115.967\\
1320	-86.6700000000001\\
1321	-76.904\\
1323	-178.223\\
1324	-139.16\\
1325	-79.346\\
1326	-76.904\\
1327	-79.346\\
1328	-98.877\\
1329	-114.746\\
1330	-85.4490000000001\\
1331	-90.3320000000001\\
1332	-120.85\\
1333	-131.836\\
1334	-87.8910000000001\\
1335	-68.3589999999999\\
1336	-91.5530000000001\\
1337	-156.25\\
1338	-137.939\\
1339	-140.381\\
1340	-84.229\\
1341	-76.904\\
1342	-72.021\\
1343	-47.607\\
1344	-30.518\\
1346	-23.193\\
1347	-54.932\\
1349	-87.8910000000001\\
1350	-65.9180000000001\\
1351	-73.242\\
1352	-83.008\\
1353	-53.711\\
1354	-56.152\\
1355	-53.711\\
1356	-40.2829999999999\\
1357	-51.27\\
1358	-84.229\\
1359	-106.201\\
1360	-63.4770000000001\\
1361	-48.828\\
1362	-40.2829999999999\\
1363	-50.049\\
1364	-35.4000000000001\\
1365	-76.904\\
1366	-130.615\\
1367	-104.98\\
1368	-139.16\\
1369	-162.354\\
1370	-164.795\\
1371	-124.512\\
1372	-90.3320000000001\\
1373	-89.1110000000001\\
1374	-89.1110000000001\\
1375	-106.201\\
1376	-125.732\\
1377	-125.732\\
1378	-168.457\\
1379	-183.105\\
1380	-219.727\\
1381	-147.705\\
1383	-184.326\\
1384	-140.381\\
1385	-162.354\\
1386	-139.16\\
1387	-80.566\\
1388	-52.49\\
1389	-56.152\\
1390	-81.787\\
1391	-65.9180000000001\\
1392	-43.9449999999999\\
1393	-62.2560000000001\\
1394	-47.607\\
1395	-36.6210000000001\\
1396	-46.3869999999999\\
1397	-52.49\\
1398	-36.6210000000001\\
1399	-70.8009999999999\\
1400	-92.7729999999999\\
1401	-68.3589999999999\\
1403	-164.795\\
1404	-150.146\\
1405	-196.533\\
1406	-131.836\\
1407	-144.043\\
1408	-96.4359999999999\\
1409	-63.4770000000001\\
1410	-76.904\\
1411	-59.8140000000001\\
1412	-45.1659999999999\\
1413	-64.6970000000001\\
1414	-36.6210000000001\\
1415	-28.076\\
1416	-34.1800000000001\\
1417	-70.8009999999999\\
1418	-86.6700000000001\\
1419	-107.422\\
1421	-100.098\\
1422	-114.746\\
1423	-93.9939999999999\\
1424	-76.904\\
1425	-76.904\\
1426	-62.2560000000001\\
1427	-97.6559999999999\\
1428	-68.3589999999999\\
1429	-56.152\\
1430	-81.787\\
1431	-113.525\\
1432	-86.6700000000001\\
1433	-65.9180000000001\\
1434	-56.152\\
1435	-65.9180000000001\\
1436	-45.1659999999999\\
1437	-45.1659999999999\\
1439	-75.684\\
1440	-84.229\\
1441	-65.9180000000001\\
1442	-86.6700000000001\\
1443	-79.346\\
1444	-47.607\\
1445	-50.049\\
1446	-95.2149999999999\\
1447	-131.836\\
1448	-131.836\\
1449	-107.422\\
1450	-103.76\\
1451	-79.346\\
1452	-76.904\\
1453	-61.0350000000001\\
1454	-89.1110000000001\\
1455	-75.684\\
1456	-50.049\\
1457	-74.463\\
1458	-85.4490000000001\\
1459	-101.318\\
1460	-109.863\\
1461	-109.863\\
1462	-148.926\\
1463	-181.885\\
1464	-205.078\\
1465	-129.395\\
1466	-73.242\\
1467	-47.607\\
1468	-34.1800000000001\\
1469	-54.932\\
1470	-46.3869999999999\\
1471	-80.566\\
1472	-72.021\\
1473	-64.6970000000001\\
1474	-85.4490000000001\\
1475	-98.877\\
1476	-117.188\\
1477	-123.291\\
1478	-175.781\\
1479	-150.146\\
1480	-117.188\\
1481	-80.566\\
1482	-80.566\\
1483	-83.008\\
1484	-106.201\\
1485	-75.684\\
1486	-79.346\\
1487	-74.463\\
1488	-42.7249999999999\\
1490	-107.422\\
1491	-73.242\\
1492	-80.566\\
1493	-147.705\\
1494	-109.863\\
1495	-106.201\\
1496	-147.705\\
1497	-140.381\\
1498	-87.8910000000001\\
1499	-56.152\\
1500	-58.5940000000001\\
1501	-97.6559999999999\\
1502	-151.367\\
1503	-162.354\\
1504	-167.236\\
1505	-129.395\\
1506	-83.008\\
1507	-72.021\\
1508	-53.711\\
1509	-50.049\\
1510	-42.7249999999999\\
1511	-61.0350000000001\\
1512	-72.021\\
1513	-41.5039999999999\\
1515	-89.1110000000001\\
1516	-101.318\\
1517	-128.174\\
1519	-192.871\\
1520	-136.719\\
1521	-118.408\\
1522	-124.512\\
1523	-102.539\\
1524	-73.242\\
1525	-89.1110000000001\\
1526	-136.719\\
1527	-162.354\\
1528	-93.9939999999999\\
1529	-54.932\\
1530	-37.8420000000001\\
1531	-23.193\\
1532	-32.9590000000001\\
1533	-45.1659999999999\\
1534	-52.49\\
1535	-73.242\\
1537	-45.1659999999999\\
1538	-45.1659999999999\\
1539	-43.9449999999999\\
1540	-40.2829999999999\\
1541	-47.607\\
1542	-70.8009999999999\\
1543	-104.98\\
1544	-111.084\\
1545	-107.422\\
1546	-123.291\\
1547	-152.588\\
1548	-155.029\\
1549	-183.105\\
1550	-167.236\\
1552	-85.4490000000001\\
1553	-83.008\\
1554	-141.602\\
1555	-158.691\\
1557	-76.904\\
1558	-39.0630000000001\\
1559	-30.518\\
1560	-29.297\\
1561	-19.5309999999999\\
1562	-45.1659999999999\\
1563	-57.373\\
1564	-62.2560000000001\\
1565	-53.711\\
1566	-34.1800000000001\\
1567	-26.855\\
1568	-36.6210000000001\\
1569	-41.5039999999999\\
1570	-43.9449999999999\\
1571	-35.4000000000001\\
1572	-28.076\\
1573	-50.049\\
1574	-112.305\\
1576	-145.264\\
1577	-101.318\\
1578	-133.057\\
1579	-195.313\\
1580	-174.561\\
1581	-184.326\\
1582	-135.498\\
1583	-137.939\\
1584	-145.264\\
1585	-95.2149999999999\\
1586	-90.3320000000001\\
1587	-56.152\\
1588	-54.932\\
1589	-102.539\\
1590	-68.3589999999999\\
1591	-79.346\\
1592	-84.229\\
1593	-79.346\\
1594	-79.346\\
1595	-84.229\\
1596	-93.9939999999999\\
1597	-102.539\\
1598	-89.1110000000001\\
1599	-67.1389999999999\\
1600	-84.229\\
1601	-39.0630000000001\\
1602	-20.752\\
1604	-51.27\\
1605	-26.855\\
1606	-13.4280000000001\\
1608	-50.049\\
1609	-58.5940000000001\\
1610	-72.021\\
1611	-62.2560000000001\\
1612	-91.5530000000001\\
1613	-80.566\\
1614	-57.373\\
1615	-86.6700000000001\\
1616	-48.828\\
1617	-41.5039999999999\\
1618	-28.076\\
1619	-21.973\\
1620	-37.8420000000001\\
1621	-28.076\\
1622	-50.049\\
1623	-75.684\\
1624	-73.242\\
1625	-54.932\\
1626	-61.0350000000001\\
1627	-45.1659999999999\\
1628	-65.9180000000001\\
1629	-47.607\\
1630	-46.3869999999999\\
1631	-42.7249999999999\\
1632	-32.9590000000001\\
1633	-42.7249999999999\\
1634	-37.8420000000001\\
1635	-64.6970000000001\\
1636	-63.4770000000001\\
1637	-50.049\\
1638	-46.3869999999999\\
1639	-36.6210000000001\\
1640	-43.9449999999999\\
1641	-93.9939999999999\\
1642	-120.85\\
1643	-81.787\\
1644	-80.566\\
1645	-54.932\\
1646	-93.9939999999999\\
1647	-81.787\\
1648	-54.932\\
1649	-70.8009999999999\\
1650	-52.49\\
1651	-75.684\\
1652	-42.7249999999999\\
1654	-57.373\\
1655	-74.463\\
1656	-58.5940000000001\\
1657	-61.0350000000001\\
1658	-61.0350000000001\\
1659	-95.2149999999999\\
1660	-79.346\\
1661	-125.732\\
1662	-157.471\\
1663	-125.732\\
1664	-81.787\\
1665	-62.2560000000001\\
1666	-89.1110000000001\\
1667	-112.305\\
1668	-87.8910000000001\\
1669	-107.422\\
1670	-64.6970000000001\\
1671	-34.1800000000001\\
1672	-37.8420000000001\\
1673	-84.229\\
1674	-70.8009999999999\\
1675	-69.5799999999999\\
1676	-104.98\\
1677	-97.6559999999999\\
1678	-87.8910000000001\\
1679	-62.2560000000001\\
1680	-73.242\\
1681	-100.098\\
1682	-81.787\\
1683	-87.8910000000001\\
1684	-78.125\\
1685	-58.5940000000001\\
1686	-67.1389999999999\\
1687	-48.828\\
1688	-56.152\\
1689	-95.2149999999999\\
1690	-109.863\\
1691	-115.967\\
1692	-102.539\\
1693	-73.242\\
1694	-51.27\\
1695	-62.2560000000001\\
1696	-70.8009999999999\\
1699	-131.836\\
1700	-107.422\\
1701	-62.2560000000001\\
1702	-109.863\\
1703	-151.367\\
1704	-107.422\\
1705	-108.643\\
1706	-124.512\\
1707	-93.9939999999999\\
1708	-79.346\\
1709	-89.1110000000001\\
1710	-78.125\\
1711	-91.5530000000001\\
1712	-114.746\\
1713	-86.6700000000001\\
1714	-102.539\\
1715	-92.7729999999999\\
1716	-96.4359999999999\\
1717	-76.904\\
1718	-100.098\\
1719	-69.5799999999999\\
1721	-84.229\\
1722	-80.566\\
1723	-41.5039999999999\\
1724	-28.076\\
1725	-21.973\\
1726	-37.8420000000001\\
1727	-84.229\\
1728	-108.643\\
1729	-111.084\\
1730	-74.463\\
1731	-86.6700000000001\\
1732	-72.021\\
1733	-65.9180000000001\\
1734	-42.7249999999999\\
1735	-61.0350000000001\\
1736	-58.5940000000001\\
1737	-57.373\\
1738	-67.1389999999999\\
1739	-57.373\\
1740	-53.711\\
1741	-37.8420000000001\\
1742	-50.049\\
1743	-91.5530000000001\\
1744	-70.8009999999999\\
1745	-86.6700000000001\\
1746	-75.684\\
1747	-101.318\\
1748	-92.7729999999999\\
1749	-129.395\\
1750	-91.5530000000001\\
1751	-106.201\\
1752	-103.76\\
1753	-54.932\\
1754	-96.4359999999999\\
1755	-67.1389999999999\\
1756	-84.229\\
1757	-90.3320000000001\\
1758	-115.967\\
1759	-85.4490000000001\\
1761	-136.719\\
1762	-111.084\\
1763	-95.2149999999999\\
1764	-75.684\\
1765	-91.5530000000001\\
1767	-69.5799999999999\\
1768	-56.152\\
1769	-70.8009999999999\\
1770	-58.5940000000001\\
1771	-119.629\\
1772	-151.367\\
1773	-130.615\\
1774	-126.953\\
1775	-124.512\\
1776	-69.5799999999999\\
1777	-63.4770000000001\\
1778	-50.049\\
1779	-42.7249999999999\\
1780	-48.828\\
1781	-80.566\\
1782	-100.098\\
1783	-114.746\\
1784	-96.4359999999999\\
1785	-67.1389999999999\\
1786	-117.188\\
1787	-137.939\\
1788	-152.588\\
1789	-118.408\\
1790	-187.988\\
1792	-78.125\\
1793	-57.373\\
1794	-48.828\\
1795	-87.8910000000001\\
1796	-111.084\\
1797	-147.705\\
1798	-109.863\\
1799	-87.8910000000001\\
1800	-47.607\\
1801	-53.711\\
1802	-56.152\\
1803	-63.4770000000001\\
1804	-40.2829999999999\\
1805	-52.49\\
};
\addlegendentry{True output}

\addplot [color=mycolor2, dashed, line width=2.0pt]
  table[row sep=crcr]{%
1006	-115.168180771827\\
1007	-142.584446657916\\
1008	-120.611546293092\\
1009	-63.8143766301507\\
1010	-75.4486911407591\\
1011	-74.8694342537567\\
1012	-91.0410285618675\\
1013	-71.6653354872767\\
1014	-27.3605442009346\\
1015	-17.8569413719265\\
1016	-17.1070988521847\\
1017	-66.8991971135169\\
1018	-104.40611887042\\
1019	-104.409756520188\\
1020	-107.400278209422\\
1021	-87.5660173828328\\
1022	-52.2755822961888\\
1023	-108.485880208089\\
1024	-76.1431449397344\\
1025	-63.157024417186\\
1026	-98.3335530853026\\
1028	-72.7167854020056\\
1029	-110.073451784364\\
1030	-103.73720333007\\
1031	-85.0934784360618\\
1032	-110.989289201236\\
1033	-114.001952562031\\
1034	-100.418733536963\\
1035	-83.0917532790033\\
1036	-68.1883476732701\\
1038	-86.5671440278552\\
1039	-89.0672254000258\\
1040	-86.8316603626934\\
1041	-92.0750628801459\\
1042	-95.5928796979549\\
1043	-129.203836838746\\
1044	-125.150865194128\\
1045	-94.6177849862588\\
1046	-82.1592866092653\\
1047	-48.1299436869294\\
1048	-50.5261640306055\\
1049	-69.0627772505075\\
1050	-53.5673356128755\\
1051	-55.4179262183861\\
1052	-77.0685986794288\\
1053	-83.5799219217811\\
1054	-81.3805944546566\\
1055	-119.289990482762\\
1056	-96.3994009822891\\
1057	-63.479720992445\\
1058	-49.906805765366\\
1060	-74.4897826494969\\
1061	-37.9358544902911\\
1062	-43.2064596227087\\
1063	-59.4800657038281\\
1064	-47.5372743010951\\
1065	-41.0037077996858\\
1066	-55.0736982254191\\
1067	-56.9363705412823\\
1068	-96.2791651851098\\
1069	-82.2146517350966\\
1070	-104.220918621385\\
1071	-97.7338997926233\\
1072	-104.908440267729\\
1073	-96.594260020408\\
1075	-83.5717724308488\\
1076	-81.5316949242585\\
1077	-75.6141348649116\\
1078	-115.853444610257\\
1079	-168.802760649183\\
1080	-180.369948559761\\
1081	-178.473842822427\\
1082	-130.35574561343\\
1083	-163.75432122183\\
1084	-192.973082551627\\
1085	-203.62029222734\\
1086	-176.383892664682\\
1087	-198.546476899494\\
1088	-261.352492464363\\
1089	-214.640602717118\\
1090	-176.27390476006\\
1091	-113.034185743401\\
1092	-91.4503219264545\\
1093	-75.7661623027957\\
1094	-90.3092781126284\\
1096	-39.4157057542907\\
1097	-39.4287627731255\\
1098	-57.6883504845855\\
1099	-80.2673959709805\\
1100	-73.5719697376051\\
1101	-76.6767621087861\\
1102	-95.6005876903444\\
1103	-103.395097417128\\
1104	-133.249635434272\\
1105	-124.384984455313\\
1106	-129.315198416773\\
1108	-94.1230605798914\\
1109	-105.093767241302\\
1110	-83.0821472136611\\
1111	-73.4630943771194\\
1112	-86.9994800496088\\
1113	-85.736238370446\\
1114	-67.5731479291376\\
1115	-67.220429132559\\
1116	-50.2661527488474\\
1117	-53.4429790040692\\
1118	-61.6426905957069\\
1119	-45.2780394740532\\
1120	-53.4332988977076\\
1121	-66.8730043609121\\
1122	-100.608586254654\\
1123	-94.7999299917378\\
1124	-113.443207856619\\
1125	-69.360921864054\\
1126	-97.0741248009601\\
1127	-138.797500903872\\
1128	-118.350704469595\\
1129	-124.11268967917\\
1130	-121.96304809171\\
1132	-55.4190424355361\\
1133	-78.7146644532918\\
1134	-85.1236201774757\\
1135	-129.493832281641\\
1136	-157.173005565468\\
1137	-159.83083649986\\
1138	-126.97942676309\\
1139	-135.527740577161\\
1140	-119.419273972405\\
1141	-115.642413936071\\
1142	-114.234909288149\\
1143	-78.3292530105975\\
1144	-73.1798632328962\\
1145	-63.3918738903299\\
1146	-58.7309998739829\\
1147	-67.1605936132221\\
1148	-90.505529692608\\
1149	-125.788514361261\\
1150	-118.140927985982\\
1151	-78.9245327793315\\
1152	-72.4610305235285\\
1153	-66.9388599028275\\
1154	-38.777538349719\\
1155	-27.0906768189293\\
1156	-37.0445347366074\\
1157	-62.7891545739594\\
1158	-50.2355532089719\\
1159	-56.9943236923791\\
1160	-58.7747488175769\\
1161	-58.7848970624696\\
1162	-40.2959126275668\\
1164	-19.8413993781921\\
1165	-38.3795945524178\\
1166	-85.927834178136\\
1167	-116.715379930576\\
1168	-122.604165635155\\
1169	-94.7094902365725\\
1170	-63.328425772334\\
1171	-47.0031618959024\\
1172	-33.187105338284\\
1173	-68.4739901034297\\
1174	-64.1847099077547\\
1175	-86.4473103659732\\
1176	-111.517526889704\\
1177	-176.247077280754\\
1178	-208.103932826402\\
1179	-183.062621864288\\
1180	-166.002216583385\\
1181	-119.440809656983\\
1182	-124.547892841315\\
1183	-133.077579856399\\
1184	-137.386306143064\\
1185	-118.640320393961\\
1186	-97.2009151282195\\
1187	-106.893937250298\\
1188	-95.4224309188426\\
1189	-101.535230316042\\
1190	-83.0479996255142\\
1191	-121.477359750848\\
1192	-152.386985515019\\
1194	-79.8535209024144\\
1195	-72.9545634227768\\
1196	-118.674440326911\\
1197	-144.292546518471\\
1199	-185.721888490663\\
1200	-196.407268649021\\
1201	-159.291397657118\\
1202	-147.985315606196\\
1203	-154.807116668283\\
1204	-175.862753481701\\
1205	-117.80438533671\\
1206	-75.2154846858743\\
1207	-106.975275691696\\
1208	-90.3588428223461\\
1209	-56.7393430184025\\
1210	-74.5333425278504\\
1211	-85.0731531452604\\
1212	-70.0545881285052\\
1213	-60.8600832877371\\
1214	-70.1517101305672\\
1215	-63.3642009272182\\
1216	-73.7321839752076\\
1217	-111.138362379853\\
1218	-97.4396968430594\\
1219	-77.2651972388931\\
1220	-73.409265814857\\
1221	-104.09583284125\\
1222	-162.562549300479\\
1223	-138.512324273562\\
1224	-82.9704352912458\\
1225	-67.6005981857804\\
1226	-74.0745022478889\\
1227	-71.8366072098324\\
1228	-47.7015783438649\\
1229	-58.1548975735641\\
1230	-65.7585835853117\\
1231	-80.92989805246\\
1232	-93.0061950778681\\
1233	-66.8111146913106\\
1234	-100.590135723977\\
1235	-122.972869468598\\
1236	-112.098535359748\\
1237	-96.4590695565726\\
1238	-100.610653710991\\
1239	-126.246934569731\\
1240	-102.812445142566\\
1241	-36.0240435202702\\
1242	-54.5660524825596\\
1243	-68.0356948845931\\
1244	-84.6529326388227\\
1245	-74.8643913927763\\
1246	-59.2011595922127\\
1247	-83.1433299307644\\
1248	-75.7333966180456\\
1249	-59.8731663381459\\
1250	-76.1310182752172\\
1251	-65.0690070215903\\
1252	-59.2560837203976\\
1253	-71.1604174345905\\
1254	-42.5467495081382\\
1255	-51.9817112485114\\
1256	-36.87525816979\\
1257	-42.4579386451071\\
1258	-77.8485016077391\\
1259	-79.5195832338957\\
1260	-105.078019482067\\
1261	-140.992370761641\\
1263	-65.0944543121568\\
1264	-46.7350413171091\\
1265	-48.4702308983751\\
1266	-70.5096800611377\\
1267	-41.4895551433178\\
1268	-48.072495887774\\
1269	-61.5194596449026\\
1270	-100.26194206785\\
1271	-92.6147681415453\\
1272	-125.437573517272\\
1273	-96.4688132724548\\
1274	-80.0963235072129\\
1275	-44.7063052615956\\
1276	-47.2834986296182\\
1277	-28.1994501528955\\
1278	-34.8203414992283\\
1279	-40.297115710978\\
1280	-66.1279502840976\\
1281	-71.4220258056735\\
1282	-76.1059208252943\\
1283	-82.5361519355097\\
1284	-120.15353081871\\
1285	-121.533841819505\\
1286	-86.7628766181228\\
1287	-98.8415164837711\\
1288	-109.527601112322\\
1289	-130.578046977216\\
1290	-114.738712466346\\
1291	-83.1408287436263\\
1292	-34.9799198474172\\
1293	-27.8346795716943\\
1294	-32.0360329302455\\
1295	-38.1933286930528\\
1296	-39.0760513908315\\
1297	-38.1342281524471\\
1298	-47.0925498325739\\
1299	-79.5418725891309\\
1300	-70.4430049865291\\
1301	-67.5744117997931\\
1302	-80.7303622170944\\
1304	-30.9072089604695\\
1306	-88.4092284841663\\
1307	-83.0026014582636\\
1308	-96.2696143081296\\
1309	-82.0054183384411\\
1310	-65.5467564236876\\
1311	-82.4073156790369\\
1312	-109.40954907741\\
1313	-113.108130572197\\
1314	-88.0038562362329\\
1315	-126.903263179808\\
1316	-108.137631796599\\
1317	-95.9656504044056\\
1318	-117.269753559046\\
1319	-111.694559020459\\
1320	-94.1748546357073\\
1321	-82.6745350763902\\
1322	-117.546802971225\\
1323	-164.974524931576\\
1324	-136.213462254218\\
1325	-87.5565859362364\\
1326	-75.9529207516425\\
1327	-88.0008324270623\\
1328	-93.9069218331881\\
1329	-106.061641406436\\
1330	-89.1673301280978\\
1331	-94.4789960438752\\
1332	-113.238351881704\\
1333	-119.892005135046\\
1334	-93.2928242569485\\
1335	-79.7735583343706\\
1336	-91.5313925542953\\
1337	-147.026851868399\\
1338	-132.268381029231\\
1339	-130.540782171862\\
1340	-91.1637718854593\\
1341	-77.9942257437071\\
1342	-77.4103825207942\\
1343	-48.9949484249189\\
1344	-27.523977370837\\
1345	-22.255560594939\\
1346	-20.768748964249\\
1347	-53.4358187636692\\
1348	-75.1954584726388\\
1349	-86.172785961774\\
1350	-69.3071742022942\\
1351	-70.6834572186374\\
1352	-82.2449824825082\\
1353	-54.1735661679625\\
1354	-57.6756690815982\\
1355	-56.3522047755127\\
1356	-42.3511105795744\\
1357	-53.9726267118338\\
1358	-79.5566847546254\\
1359	-100.763109735315\\
1360	-65.1569783538823\\
1361	-52.5075875890927\\
1362	-46.4167774102627\\
1363	-55.160342208374\\
1364	-33.3824028274785\\
1366	-125.113510311896\\
1367	-105.053358945649\\
1368	-126.397861242185\\
1369	-151.812811972421\\
1370	-159.419568890393\\
1371	-125.911517648542\\
1372	-97.8974827847505\\
1373	-96.6315624064989\\
1374	-91.1656503481138\\
1376	-116.977468105433\\
1377	-117.026344843024\\
1378	-152.093851030648\\
1379	-173.637025559469\\
1380	-198.312403258321\\
1381	-160.431325923915\\
1382	-165.668042195733\\
1383	-171.833870897764\\
1384	-153.457809714303\\
1385	-155.487663674642\\
1386	-142.41641866984\\
1387	-88.1506398955416\\
1388	-48.5347529944588\\
1389	-55.9692862428992\\
1390	-81.5859671107046\\
1391	-67.2073076749364\\
1392	-47.6226625384952\\
1393	-66.1105008423692\\
1394	-48.5954440595017\\
1395	-38.6540212640466\\
1396	-48.3420820538342\\
1397	-49.4518658690756\\
1398	-42.9590644255386\\
1400	-92.4437509681047\\
1401	-70.9853781250526\\
1402	-113.247629043849\\
1403	-163.380071070643\\
1404	-149.526956213526\\
1405	-178.908645123228\\
1406	-139.421666776716\\
1407	-145.372412999727\\
1408	-101.409578007704\\
1409	-64.7796067260504\\
1410	-77.3344125874744\\
1411	-66.7236265210881\\
1412	-45.7876784695011\\
1413	-62.8854325290943\\
1414	-43.4916547168959\\
1415	-18.7000746897018\\
1416	-33.7345248583747\\
1417	-70.864288950901\\
1418	-90.3503000308335\\
1419	-100.98214552166\\
1420	-105.36592088072\\
1421	-96.6056364884673\\
1422	-109.515499311778\\
1423	-97.418029025539\\
1424	-82.2383721979854\\
1425	-81.5099372156849\\
1426	-67.5419206453337\\
1427	-95.2300394147042\\
1428	-66.9119791366518\\
1429	-59.6569893216631\\
1430	-84.4387262794546\\
1431	-105.155931896595\\
1433	-70.0689781436517\\
1434	-62.5906211275765\\
1435	-69.4777029013287\\
1436	-46.8799628101679\\
1437	-45.0544158681357\\
1438	-62.4060209371189\\
1439	-70.5766459866954\\
1440	-81.3102630577298\\
1441	-71.4217573960627\\
1442	-82.3686745718198\\
1443	-78.9055671236304\\
1444	-52.0617121223356\\
1445	-53.7668555209912\\
1447	-129.868925880792\\
1448	-120.655015226016\\
1449	-107.776019626\\
1450	-103.291107306524\\
1451	-89.4506561132162\\
1452	-78.0242439090605\\
1453	-68.8740925565576\\
1454	-81.2812570559997\\
1455	-82.5620210569584\\
1456	-50.2647511575949\\
1457	-76.1149221165149\\
1458	-80.1571129333513\\
1460	-107.600790277165\\
1461	-101.151170565749\\
1462	-144.426632483355\\
1463	-169.440398566456\\
1464	-190.458916356296\\
1465	-147.228872045074\\
1466	-76.7580773265142\\
1467	-43.6500943017543\\
1468	-35.087455156837\\
1469	-53.4502807807621\\
1470	-47.0587796917414\\
1471	-79.9649398338217\\
1472	-72.2535536757589\\
1473	-66.3670893041406\\
1474	-83.8023549000193\\
1475	-93.6714180307392\\
1476	-111.390146953928\\
1477	-118.344250595616\\
1478	-163.834963557768\\
1479	-158.346773301583\\
1480	-121.078160633321\\
1481	-91.3448899050929\\
1482	-81.2791758600204\\
1484	-100.204315658497\\
1485	-79.3424111577335\\
1486	-78.1113316384062\\
1487	-83.6700074699977\\
1488	-37.8524398034101\\
1489	-73.4875026296693\\
1490	-102.980623752006\\
1491	-77.5476493672109\\
1492	-75.5812601557648\\
1493	-146.067830115025\\
1494	-107.313862868227\\
1495	-107.804888944024\\
1496	-137.209012476735\\
1497	-136.442865972202\\
1499	-60.0484034785736\\
1500	-60.1754506764494\\
1502	-135.999443686608\\
1503	-150.986186215754\\
1504	-153.600815322603\\
1505	-140.563796470621\\
1506	-85.8323085014558\\
1507	-79.1885287346809\\
1508	-56.6929816913841\\
1509	-54.6500586518582\\
1510	-44.0198575844222\\
1511	-63.3366148499615\\
1512	-69.0876938208726\\
1513	-43.4910120129016\\
1514	-70.4358992916757\\
1515	-87.7014466137682\\
1516	-93.8177710263012\\
1517	-121.025419467664\\
1518	-155.247051298006\\
1519	-179.930719138045\\
1520	-145.029762038945\\
1521	-128.079839307811\\
1522	-126.075786740821\\
1523	-109.713846494595\\
1524	-75.3799964930397\\
1525	-85.2469735486536\\
1526	-132.678914240665\\
1527	-148.094576837354\\
1529	-55.4290662037822\\
1530	-35.4079889983946\\
1531	-11.9952827177806\\
1532	-24.8468061532201\\
1533	-47.5121622880704\\
1534	-49.4948719171023\\
1535	-79.3783002867146\\
1537	-47.175826092071\\
1538	-48.119672625133\\
1539	-45.8318331099811\\
1540	-42.8265813961175\\
1541	-49.1146301850977\\
1542	-70.2237278569703\\
1543	-103.704903487805\\
1544	-106.681494281585\\
1545	-105.629377461336\\
1546	-114.166680221678\\
1547	-151.428790758962\\
1548	-144.524812324485\\
1549	-171.279138163777\\
1550	-173.630064911743\\
1552	-93.279704275511\\
1553	-85.9378104311747\\
1554	-133.541424636663\\
1555	-150.355252370807\\
1556	-116.416782306548\\
1557	-87.0936487910567\\
1558	-20.6619707131931\\
1559	-27.867491027029\\
1560	-32.2960363461\\
1561	-11.5881447863699\\
1562	-44.2535087979854\\
1563	-61.683655887254\\
1564	-58.6958937321106\\
1565	-57.2794508953748\\
1566	-33.0123382329693\\
1567	-30.7456490421796\\
1568	-35.3637789740214\\
1569	-43.8598633272875\\
1570	-42.0173770834263\\
1572	-31.8987072672539\\
1573	-50.8665393948572\\
1574	-116.516801872041\\
1575	-135.067263980491\\
1576	-133.739228268035\\
1577	-110.728107467833\\
1578	-124.56357388411\\
1579	-193.82691155425\\
1580	-165.11193412139\\
1581	-179.423086496774\\
1582	-140.102068633263\\
1583	-142.152445433982\\
1584	-145.709669089539\\
1585	-103.946219752198\\
1586	-92.1702253584726\\
1587	-53.4498749225115\\
1588	-50.529509953242\\
1589	-105.348352246118\\
1590	-68.5316379689932\\
1591	-73.9727658342579\\
1592	-85.3694001804367\\
1593	-79.4541148906515\\
1594	-85.3034205162423\\
1595	-80.8191617765194\\
1596	-94.4459966507468\\
1597	-99.517913467243\\
1598	-89.9658646613213\\
1599	-73.2787941169122\\
1600	-88.6411795111192\\
1601	-32.3073756979584\\
1602	-15.9032262620581\\
1603	-28.1049511636536\\
1604	-55.9474362564115\\
1605	-23.3304784137995\\
1606	-12.2653633720261\\
1607	-27.3236547898844\\
1608	-49.2002866448938\\
1609	-59.9520980325881\\
1610	-73.1807279281907\\
1611	-63.5984358509052\\
1612	-89.8753876564806\\
1613	-88.0703389341004\\
1614	-58.5499906007869\\
1615	-85.0691175172735\\
1616	-44.9478926710881\\
1617	-41.4204010122189\\
1619	-16.509271622112\\
1620	-37.2152779494729\\
1621	-27.6327417104058\\
1622	-48.7344038573494\\
1623	-77.0073462719999\\
1624	-71.9176427168375\\
1625	-56.9008907674286\\
1626	-60.5016468029798\\
1627	-50.5631976287161\\
1628	-60.3999324158531\\
1629	-49.3615257018839\\
1630	-44.9866722901572\\
1631	-44.8126372334932\\
1632	-32.6598512601513\\
1633	-43.5191099900296\\
1634	-37.116992004321\\
1635	-63.8815321515174\\
1636	-58.9777889024767\\
1637	-49.2155331183108\\
1638	-50.807807437074\\
1639	-36.9288555910714\\
1640	-43.8034451309265\\
1641	-87.9358984709561\\
1642	-119.865292451861\\
1643	-83.1844778236607\\
1644	-79.4264802258633\\
1645	-55.4465317610279\\
1646	-90.0912240407431\\
1647	-85.1414253969083\\
1648	-54.6291066288352\\
1649	-74.1802067545318\\
1650	-53.6461562459299\\
1651	-71.013432977132\\
1652	-46.0271682316918\\
1653	-42.8852257207732\\
1654	-60.8657173508516\\
1655	-69.8805695122091\\
1656	-57.6326371770583\\
1657	-60.6847383263521\\
1658	-59.0441301803655\\
1659	-89.62230095271\\
1660	-83.0118736285801\\
1661	-110.990553902597\\
1662	-155.301735279085\\
1663	-128.118509954627\\
1664	-84.9322232275256\\
1665	-64.3032908553678\\
1666	-90.1832270801769\\
1667	-111.863974112723\\
1668	-85.4047995370354\\
1669	-101.117492611213\\
1671	-33.3431262109623\\
1672	-34.4997252989001\\
1673	-83.4226318903686\\
1675	-65.3867615395154\\
1676	-98.0032371275406\\
1677	-96.6616057590793\\
1678	-84.9792786161504\\
1679	-66.2444390951161\\
1680	-76.2921236623029\\
1681	-93.8472736804181\\
1682	-83.9999853957813\\
1683	-82.1039113122788\\
1684	-76.8492488082686\\
1685	-63.1231926231371\\
1686	-70.4381494173272\\
1687	-49.261903984657\\
1688	-60.0190668273628\\
1689	-91.3281285108062\\
1690	-97.8998837022491\\
1691	-107.234869981819\\
1692	-100.559378859788\\
1694	-57.2663461447494\\
1695	-62.1035978124128\\
1696	-70.8262095176722\\
1697	-85.9329789960098\\
1699	-122.460008977504\\
1700	-107.15432542564\\
1701	-66.5446755389369\\
1703	-144.944097035099\\
1704	-106.311713526629\\
1705	-108.916814936375\\
1706	-116.621211494112\\
1707	-101.100709130266\\
1708	-82.1791812691151\\
1709	-90.2196603928246\\
1710	-80.9246085814361\\
1711	-87.5245320855358\\
1712	-110.945601182235\\
1713	-83.8338615047494\\
1714	-98.6384687785987\\
1715	-93.7260893645932\\
1716	-95.9624023401007\\
1717	-80.1344704185378\\
1718	-96.007773184986\\
1719	-73.5831907258967\\
1720	-76.1802354037161\\
1721	-82.0819550058584\\
1722	-82.033594874129\\
1723	-42.8151809120116\\
1724	-25.0693926922486\\
1725	-16.1628343829559\\
1726	-33.9374666094545\\
1727	-93.4696951945273\\
1728	-109.31657527724\\
1729	-102.162386982884\\
1730	-80.1660901818132\\
1731	-84.8454857775212\\
1732	-78.595103813652\\
1733	-67.2942872264548\\
1734	-42.532895594182\\
1735	-58.2722797139247\\
1736	-60.2673001962771\\
1737	-59.9347965795905\\
1738	-64.4815907881343\\
1739	-58.2627812373303\\
1740	-56.0999659164822\\
1741	-38.9211031347165\\
1742	-50.6166325098623\\
1743	-92.3442352242855\\
1744	-68.3971082677849\\
1745	-82.7845340409913\\
1746	-73.5757134144624\\
1747	-97.4094304066291\\
1748	-90.4519357412996\\
1749	-116.372600241674\\
1750	-99.768535175928\\
1751	-97.7810557921946\\
1752	-105.131458741233\\
1753	-59.3566273441618\\
1754	-93.9049263334628\\
1755	-66.6538832658498\\
1756	-85.7311454137732\\
1757	-85.7113547369145\\
1758	-103.842938172871\\
1759	-88.9936584990271\\
1760	-102.886970950752\\
1761	-128.538529355068\\
1762	-111.471318729006\\
1763	-99.1480668805016\\
1764	-81.7003760459543\\
1765	-87.0899789090922\\
1766	-86.8417112086854\\
1767	-68.5565887633668\\
1768	-63.5128824422554\\
1769	-68.6053711737534\\
1770	-62.5018939167769\\
1771	-110.836083936304\\
1772	-138.506685955202\\
1773	-123.669228480981\\
1774	-119.675798261542\\
1775	-121.797713942757\\
1776	-77.1820614353717\\
1777	-65.9181989809426\\
1778	-52.2977612997447\\
1779	-48.7804688078513\\
1780	-50.8833826137877\\
1781	-79.8497798776327\\
1782	-90.3159384876581\\
1783	-104.801708229842\\
1784	-97.2930521527751\\
1785	-72.002807021312\\
1786	-114.641887831185\\
1787	-127.573872075798\\
1788	-137.826309691896\\
1789	-125.67663582191\\
1790	-163.59397734801\\
1791	-153.159947994354\\
1792	-82.7040424750103\\
1793	-55.1103510204905\\
1794	-55.1659738676033\\
1795	-84.319467598104\\
1796	-104.923274612831\\
1797	-130.600109419687\\
1798	-115.037779546829\\
1799	-91.630508581319\\
1800	-50.106269600471\\
1801	-53.6752737352504\\
1802	-59.9079950883588\\
1803	-61.7803952514175\\
1804	-45.9057838879417\\
1805	-53.2042553551562\\
};
\addlegendentry{OSA predition}

\addplot [color=mycolor3, dotted, line width=2.0pt]
  table[row sep=crcr]{%
1006	-125.732\\
1007	-153.809\\
1008	-115.967\\
1009	-61.0350000000001\\
1010	-75.4486911407587\\
1011	-75.273590954039\\
1012	-91.9974010179676\\
1013	-74.1635841232301\\
1014	-28.4763016825591\\
1015	-15.58547829847\\
1016	-14.7683590136419\\
1017	-62.4713099701394\\
1018	-103.667431219579\\
1019	-104.344399154594\\
1020	-105.224179373229\\
1021	-83.597439350508\\
1022	-51.6357112808241\\
1023	-106.881138583492\\
1024	-75.9129118022586\\
1025	-60.9912777972481\\
1026	-98.1603753765639\\
1028	-73.316897834477\\
1029	-111.085348151297\\
1030	-101.166651630305\\
1031	-81.9688028132518\\
1032	-109.276466465623\\
1033	-108.386676692763\\
1034	-92.3257764728487\\
1036	-65.9719623304993\\
1037	-77.5295980122701\\
1038	-87.4211853601189\\
1039	-86.2857129238785\\
1040	-85.7233434826896\\
1041	-89.0945839753892\\
1042	-90.9694301116033\\
1043	-122.738827699012\\
1044	-114.658020110243\\
1045	-85.2950322590611\\
1046	-78.5364698262665\\
1047	-46.0274065174576\\
1048	-48.4438364579642\\
1049	-68.6511964146134\\
1050	-53.4822584525032\\
1051	-55.6812158471344\\
1052	-76.4796328190937\\
1053	-83.8260558468794\\
1054	-79.2568491370664\\
1055	-117.543407495924\\
1056	-91.3727582765403\\
1057	-58.4971217448817\\
1058	-49.1359699652821\\
1059	-62.7105269184447\\
1060	-74.4979937413391\\
1061	-39.2411934668271\\
1062	-43.4778923861945\\
1063	-59.8370392283803\\
1064	-49.3250058141862\\
1065	-42.5634353333783\\
1066	-56.6473407821527\\
1067	-59.46948930385\\
1068	-96.1864308030404\\
1069	-83.9432298699728\\
1070	-102.091389469646\\
1071	-94.7838784783\\
1072	-102.352004364914\\
1073	-89.5214032301737\\
1074	-86.4523096098119\\
1075	-80.1339236723306\\
1076	-79.7657140608048\\
1077	-76.9018851093701\\
1078	-115.940048961137\\
1079	-162.515993446343\\
1080	-167.72377067455\\
1081	-162.000987193822\\
1082	-112.645570834156\\
1083	-152.454243556152\\
1084	-178.889649612391\\
1085	-183.345402653564\\
1086	-154.402392326466\\
1087	-181.955873265499\\
1088	-235.158961468324\\
1089	-186.046771155007\\
1090	-161.100646994791\\
1091	-107.428069228702\\
1092	-86.2052416088698\\
1093	-74.5832329412076\\
1094	-90.8619315207216\\
1096	-40.180702493479\\
1097	-37.2615920925862\\
1098	-54.6669263549356\\
1099	-79.1172037872041\\
1100	-72.3330959033433\\
1101	-75.2156760863722\\
1102	-95.0995242525805\\
1103	-101.059253038564\\
1104	-131.256162594272\\
1105	-119.238018431611\\
1106	-129.071746045583\\
1107	-105.80944924117\\
1108	-91.8358506459015\\
1109	-105.445845307298\\
1110	-82.6494137999211\\
1111	-76.1788422405316\\
1112	-90.6944580840193\\
1113	-89.7837232360055\\
1114	-70.5594469644982\\
1115	-72.9803145757519\\
1116	-55.7350174799467\\
1117	-57.9929006055579\\
1118	-65.7510532867184\\
1119	-50.3253356896964\\
1120	-58.9703978829784\\
1121	-71.6186280466759\\
1122	-105.619763242556\\
1123	-98.8417015717621\\
1124	-112.741293457566\\
1125	-68.5095784458738\\
1126	-96.8908335842948\\
1127	-139.169313577469\\
1128	-114.027362990482\\
1129	-122.758726990353\\
1130	-120.081241401326\\
1131	-82.3179613138382\\
1132	-55.0203484337183\\
1133	-78.916541263691\\
1134	-85.9294008674094\\
1135	-130.43743898075\\
1136	-154.790323173866\\
1137	-152.175731592669\\
1138	-116.533561136175\\
1139	-128.343335983389\\
1140	-114.88836759428\\
1141	-112.638680902423\\
1142	-112.913671509028\\
1143	-78.7066465573696\\
1144	-74.0286449861328\\
1145	-65.9045285605007\\
1146	-61.6846177987995\\
1147	-70.0173656193501\\
1148	-95.2319754543214\\
1149	-129.318433703848\\
1150	-115.393025294482\\
1151	-80.2718770976901\\
1152	-73.4779844077211\\
1153	-69.6710376769622\\
1154	-43.3574284859137\\
1155	-29.25526533375\\
1156	-38.3604000083767\\
1157	-64.451467200651\\
1158	-51.133939057775\\
1159	-57.177317848071\\
1160	-60.9160077815375\\
1161	-60.3749617359501\\
1162	-42.9561500358179\\
1163	-33.8011391288715\\
1164	-22.9931086488775\\
1165	-39.6881509281816\\
1166	-87.3887193835046\\
1167	-119.108179093813\\
1168	-123.963337618849\\
1169	-92.6738693283869\\
1170	-65.4338818353328\\
1172	-35.603688803923\\
1173	-71.2591919664935\\
1174	-67.2878591900389\\
1176	-111.146551489857\\
1177	-174.046074402157\\
1178	-201.654461122775\\
1179	-174.354909978007\\
1180	-161.215640138268\\
1181	-114.476839848732\\
1183	-133.738710569879\\
1184	-138.103169826281\\
1185	-116.037667731992\\
1186	-99.3836820471392\\
1187	-107.335286373486\\
1188	-98.3558836279842\\
1189	-107.103095572619\\
1190	-84.9619881124661\\
1191	-126.13466376395\\
1192	-152.46043472629\\
1193	-112.34747347989\\
1194	-78.9316696321052\\
1195	-75.5896556527073\\
1196	-119.88679175409\\
1197	-144.117037648972\\
1198	-160.73329400265\\
1199	-172.546292691823\\
1200	-178.747094212628\\
1201	-142.377924754599\\
1202	-135.507060037219\\
1203	-147.638368550844\\
1204	-167.511000195128\\
1205	-106.672709097517\\
1206	-70.9027643820662\\
1207	-104.229418329334\\
1208	-89.2527491667527\\
1209	-57.8308429468489\\
1210	-76.5758164265012\\
1211	-86.9202253046096\\
1212	-71.9279775984617\\
1213	-63.662582572714\\
1214	-76.1975743222642\\
1215	-67.6365683356973\\
1216	-79.4835199938166\\
1217	-114.144121028888\\
1218	-101.651966004836\\
1219	-79.632726507401\\
1220	-78.0356820223833\\
1221	-109.140288625375\\
1222	-165.89486624236\\
1223	-136.095499599988\\
1224	-85.6007546857622\\
1225	-71.3440116716381\\
1226	-78.1937756881969\\
1227	-77.2469025056573\\
1228	-54.6891231411521\\
1230	-71.9471857841493\\
1231	-86.3425795122469\\
1232	-96.6819754863122\\
1233	-70.6555245669697\\
1234	-105.075871045894\\
1235	-124.808976717152\\
1236	-113.488032889171\\
1237	-95.1928374922281\\
1238	-101.492208150445\\
1239	-126.86996145354\\
1240	-99.222643169865\\
1241	-39.9090029413026\\
1242	-55.0273357296605\\
1243	-66.2209239254103\\
1244	-86.8512001741403\\
1245	-76.933359611324\\
1246	-61.811227485289\\
1247	-88.3545927120006\\
1248	-81.4232869399591\\
1249	-62.2698988805782\\
1250	-79.7457713841613\\
1251	-70.2764305526603\\
1252	-63.4660162430566\\
1253	-76.0632894539458\\
1254	-48.1775018760654\\
1255	-56.7820980194467\\
1256	-41.8626901365044\\
1257	-46.6799800602562\\
1258	-82.519252014084\\
1259	-84.4102561348648\\
1260	-107.316619341977\\
1261	-140.106483381208\\
1263	-65.2334407750313\\
1264	-49.2651798901911\\
1265	-49.8755062352316\\
1266	-73.9693299790983\\
1267	-47.276477726326\\
1268	-51.7500723933806\\
1269	-65.0050564456756\\
1270	-103.968726528934\\
1271	-94.5909295513413\\
1272	-126.812729625668\\
1273	-93.9842537022585\\
1274	-81.3304757306298\\
1275	-45.2677154366706\\
1276	-46.0142661651741\\
1277	-28.8937380214786\\
1278	-35.1942727506359\\
1279	-40.2329378872159\\
1280	-67.2473240900754\\
1282	-76.755875716884\\
1283	-81.8357512403716\\
1284	-118.498172532877\\
1285	-117.91974830043\\
1286	-87.6615192806917\\
1287	-100.10771471481\\
1288	-108.673475935107\\
1289	-130.718261765623\\
1290	-109.253951960346\\
1291	-78.4047486999204\\
1292	-35.510526282663\\
1293	-25.2259745201429\\
1294	-29.0632708003575\\
1295	-37.7373423847459\\
1296	-38.8131168367113\\
1297	-37.7424683493339\\
1298	-47.0744796369663\\
1299	-78.3932918045814\\
1300	-71.685052695075\\
1301	-69.1151095962771\\
1302	-80.4613047956641\\
1303	-54.3713888167063\\
1304	-32.2029187385458\\
1306	-89.377016120942\\
1307	-83.3035638790993\\
1308	-94.6382351560774\\
1309	-80.3146739121664\\
1310	-63.7969128361253\\
1311	-82.5675249900742\\
1312	-107.999764517376\\
1313	-108.538834274783\\
1314	-82.7573609027124\\
1315	-123.714031097531\\
1316	-100.8776302186\\
1317	-93.5896091811264\\
1318	-113.005117184177\\
1319	-107.770827430575\\
1320	-89.8871579661945\\
1321	-81.996519262229\\
1322	-118.906759155454\\
1323	-161.476153645646\\
1324	-129.092742426408\\
1325	-81.4168003979487\\
1326	-74.3948556014743\\
1328	-95.1727284329834\\
1329	-105.253252780483\\
1330	-84.9627924522506\\
1331	-93.4054988288178\\
1332	-113.239611335371\\
1333	-116.178908286981\\
1334	-86.045770354966\\
1335	-77.0230736932895\\
1336	-93.2314621773965\\
1337	-147.038934157991\\
1338	-129.567348459838\\
1339	-126.976534242021\\
1340	-84.4230176177423\\
1341	-75.2623529449513\\
1342	-75.6098478989945\\
1343	-48.9363792459599\\
1344	-28.5646662478066\\
1345	-21.372927084135\\
1346	-18.6197849782691\\
1347	-50.9247045261761\\
1348	-72.2131625234035\\
1349	-85.1377400801271\\
1350	-67.5240981955537\\
1351	-70.7166646027247\\
1352	-81.347658604057\\
1353	-52.9758829502493\\
1354	-57.22889453777\\
1355	-56.3899608964375\\
1356	-43.3915925076856\\
1357	-55.5482776581682\\
1358	-82.0375355793162\\
1359	-101.019540416127\\
1360	-63.3965300339478\\
1361	-52.2073564565885\\
1362	-47.5105626946167\\
1363	-58.0187808111111\\
1364	-37.5849429265556\\
1366	-129.107118025643\\
1367	-106.017496604623\\
1369	-147.385827121934\\
1370	-151.768251996196\\
1371	-117.866787620378\\
1372	-91.8422060757648\\
1373	-94.4702047622518\\
1374	-91.987830382523\\
1376	-117.463597115828\\
1377	-114.263658178791\\
1378	-146.457383363334\\
1379	-161.947168879911\\
1380	-184.312100493266\\
1381	-140.099861261112\\
1382	-153.831818458573\\
1383	-160.821115756034\\
1384	-138.147245074551\\
1385	-149.618132558744\\
1386	-134.172205116665\\
1387	-82.0297406725135\\
1388	-48.5987703486126\\
1389	-53.378464464904\\
1390	-79.5242439369322\\
1391	-65.8461264555665\\
1392	-46.8333175214204\\
1393	-66.8903093372724\\
1394	-50.6272438595756\\
1395	-40.522440917432\\
1396	-50.7832848720807\\
1397	-52.3305537897422\\
1398	-43.8884319314807\\
1399	-70.8938508308413\\
1400	-93.7504901507455\\
1401	-72.0281662522907\\
1402	-115.595846965114\\
1403	-164.345931314174\\
1404	-150.08262837334\\
1405	-179.315691319192\\
1406	-132.404322405801\\
1407	-143.249957897843\\
1408	-100.697510243435\\
1409	-64.7192065165568\\
1410	-78.4557539810821\\
1411	-67.5448474382647\\
1412	-49.5973732822317\\
1413	-65.8640729223737\\
1415	-22.9569459369252\\
1416	-33.0818121733678\\
1417	-70.6128236434015\\
1418	-90.2823813240364\\
1419	-101.778382129934\\
1420	-103.347614814603\\
1421	-96.0302701105038\\
1422	-107.671285328245\\
1423	-93.4209662031087\\
1424	-80.8710296074039\\
1425	-82.3155666385508\\
1426	-69.5433209031096\\
1427	-99.1717218886338\\
1428	-69.1094098667891\\
1429	-60.976338946748\\
1430	-87.3336527750528\\
1431	-108.252004506255\\
1432	-86.2578171357852\\
1433	-70.0021745352644\\
1434	-64.6048278287799\\
1435	-72.9175585475673\\
1436	-50.8683491108116\\
1437	-48.8966490103326\\
1438	-65.8659718285853\\
1440	-82.5257162330286\\
1441	-71.4124811350764\\
1442	-84.9291264780154\\
1443	-78.6492234031762\\
1444	-51.5584222750942\\
1445	-55.7892963918714\\
1447	-131.109108552859\\
1448	-121.561100990684\\
1449	-104.047890875796\\
1450	-100.552222631746\\
1451	-87.2051596387448\\
1452	-79.6897138850679\\
1453	-70.5746736032775\\
1454	-85.7197906801166\\
1455	-83.3232549059774\\
1456	-53.5652741318431\\
1457	-79.2318693699622\\
1458	-82.7481030637553\\
1460	-105.400330251025\\
1461	-98.7368036995465\\
1462	-138.223247406281\\
1463	-162.364254462904\\
1464	-178.961634765957\\
1465	-131.961163314807\\
1466	-73.2829367165316\\
1467	-43.3226683631644\\
1468	-30.6387588537207\\
1469	-51.703994221295\\
1470	-45.1472662418689\\
1471	-78.2952848477883\\
1472	-70.5415866746398\\
1473	-65.083534391187\\
1474	-83.4727659267571\\
1475	-92.5516542003904\\
1476	-108.455254775803\\
1477	-113.810077536755\\
1478	-157.747450547442\\
1479	-148.050268645153\\
1480	-116.308228055216\\
1481	-89.2906373136041\\
1482	-82.9258153593573\\
1483	-92.8838187703545\\
1484	-105.156628185123\\
1485	-81.1598326629887\\
1486	-80.9906904070385\\
1487	-85.960219258418\\
1488	-42.7014355373979\\
1489	-75.5460252367591\\
1490	-104.684502022924\\
1491	-77.6857282944272\\
1492	-77.1432804723702\\
1493	-144.974396518632\\
1494	-105.918362658012\\
1495	-105.97497122001\\
1496	-135.999576409288\\
1497	-130.975117335179\\
1498	-93.9601367696316\\
1499	-61.2566231412966\\
1500	-61.8892441007386\\
1502	-138.886310010987\\
1503	-147.228745665937\\
1504	-146.52185465099\\
1505	-129.328250233598\\
1506	-81.5498843655064\\
1507	-76.9020379966939\\
1508	-56.4598309819407\\
1509	-56.3751808268273\\
1510	-46.9063052200661\\
1511	-66.4713245583346\\
1512	-72.5824615958409\\
1513	-45.167996168148\\
1514	-72.809156396135\\
1515	-91.5618643214109\\
1516	-96.0990723720613\\
1517	-120.421884769502\\
1518	-152.453466141458\\
1519	-174.824007814952\\
1520	-135.232888270401\\
1521	-123.761807401656\\
1522	-126.219901099541\\
1523	-109.285550350098\\
1524	-78.3427112263776\\
1525	-88.8238119346772\\
1526	-134.07610641539\\
1527	-148.316255206038\\
1528	-95.8125311677352\\
1529	-54.3151861544218\\
1530	-35.6899117986291\\
1531	-9.52466468958482\\
1532	-19.171744912677\\
1533	-40.0820477622922\\
1534	-44.598744697465\\
1535	-72.5210862406875\\
1537	-46.6482759644584\\
1538	-48.0598252943962\\
1539	-47.1662100688454\\
1540	-44.6612838550448\\
1541	-51.5963981355374\\
1542	-72.9457420714102\\
1543	-105.961023880811\\
1544	-108.300961511743\\
1545	-105.236102790534\\
1546	-113.231181665007\\
1547	-146.785208362149\\
1548	-140.351754424976\\
1549	-163.003268568451\\
1550	-161.827714283396\\
1552	-91.1598703800526\\
1553	-86.3694023206717\\
1554	-135.240749143761\\
1555	-148.986366983082\\
1556	-112.493834201813\\
1557	-84.5054305255712\\
1558	-23.2483432252595\\
1559	-21.6361075361372\\
1560	-26.113209618497\\
1561	-10.3141190321587\\
1562	-38.065421089686\\
1563	-56.5651337185623\\
1564	-56.3155381803001\\
1565	-53.2664929870464\\
1566	-31.4256825999264\\
1567	-29.4320209729265\\
1568	-35.5029589009227\\
1569	-43.4893218827453\\
1570	-42.6081140264648\\
1571	-36.8086969157619\\
1572	-32.2855554042142\\
1573	-52.8523732747917\\
1574	-118.235355221572\\
1575	-138.757297937004\\
1576	-139.648634425264\\
1577	-110.537976924458\\
1578	-129.097867300357\\
1579	-193.631377583735\\
1580	-164.526099393517\\
1581	-175.495856207471\\
1582	-134.628295356833\\
1584	-144.754817412768\\
1585	-103.088096304437\\
1586	-95.3073355475512\\
1587	-56.7405328710424\\
1588	-51.3296575301238\\
1589	-105.134799532819\\
1590	-69.673983828849\\
1591	-74.618707531333\\
1592	-83.4118615404259\\
1593	-78.8676553001681\\
1594	-84.7823685983815\\
1595	-82.46811940152\\
1596	-94.280592087071\\
1597	-99.8424226440375\\
1598	-89.3034594249766\\
1599	-72.8577098254507\\
1600	-90.8665516544756\\
1601	-35.4846452075476\\
1602	-14.9974938419907\\
1603	-25.7300022047373\\
1604	-51.2592045865026\\
1605	-21.55386208706\\
1606	-9.59210165915488\\
1607	-24.0734116702395\\
1608	-45.2464376033536\\
1609	-56.2256735114529\\
1610	-70.3617673511908\\
1611	-61.4727494635658\\
1612	-88.6329601216105\\
1613	-86.3900143812762\\
1614	-60.3903528979679\\
1615	-86.7017967807201\\
1616	-45.1493155251251\\
1617	-40.5274620757914\\
1619	-16.7627519349617\\
1620	-34.761532696924\\
1621	-25.7216536867988\\
1622	-47.0618117364443\\
1623	-74.6374349614962\\
1624	-70.5913406997802\\
1625	-55.2441552852026\\
1626	-59.9682284970261\\
1627	-49.9610465426326\\
1628	-62.0010301417428\\
1629	-48.2930606770838\\
1630	-44.91123447021\\
1631	-44.6053261071688\\
1632	-32.8916670899537\\
1633	-43.7008526437012\\
1634	-37.4446377435145\\
1635	-64.0054481415832\\
1636	-58.7259304790616\\
1637	-47.2623139452462\\
1638	-49.0469598971129\\
1639	-37.5045224670398\\
1640	-43.8800038686909\\
1641	-88.0436841732492\\
1642	-117.821635030521\\
1643	-81.3575105647678\\
1644	-78.4733305105178\\
1645	-53.9065733700522\\
1646	-89.0849223529283\\
1647	-82.8029510753245\\
1648	-54.2136271118134\\
1649	-73.6128064745456\\
1650	-54.2608827618278\\
1651	-72.1224067776259\\
1652	-44.7967294428424\\
1653	-43.6397294375145\\
1654	-58.4704499043166\\
1655	-69.3985330970897\\
1656	-55.4456739954551\\
1657	-58.2403843202578\\
1658	-57.3698448654727\\
1659	-86.9839329300851\\
1660	-78.645928024974\\
1661	-109.008353107459\\
1662	-146.765519322945\\
1663	-120.802125727686\\
1664	-80.7218121005908\\
1665	-61.5698292190777\\
1666	-88.4951532134348\\
1667	-110.938757347275\\
1668	-84.6099842383728\\
1669	-99.482693459074\\
1671	-32.401187346522\\
1672	-33.6106903005887\\
1673	-80.7095924848838\\
1675	-65.1397865293688\\
1676	-95.6143228809892\\
1677	-92.0907095004895\\
1678	-81.3971143535089\\
1679	-62.1129320244049\\
1680	-74.2310738064605\\
1681	-93.156385150214\\
1682	-80.6227753140554\\
1683	-80.8427116089183\\
1684	-73.5433148032712\\
1685	-59.6497102879305\\
1686	-69.8284928516564\\
1687	-49.6672694183537\\
1688	-60.2891861434393\\
1689	-93.4953531853128\\
1691	-102.743802492358\\
1692	-93.9519205359268\\
1693	-72.59512035227\\
1694	-54.5816702614779\\
1695	-61.8348555734794\\
1696	-70.1801444403216\\
1699	-117.637861219742\\
1700	-99.5574462076327\\
1701	-60.739195389936\\
1703	-138.556159428799\\
1704	-98.9849852078235\\
1705	-102.94428422854\\
1706	-111.357850243762\\
1707	-93.2394369260662\\
1708	-79.1161965985257\\
1709	-88.7769474361844\\
1710	-79.4253809195864\\
1711	-87.9256152252115\\
1712	-109.566891529764\\
1713	-81.3087235758715\\
1714	-95.6429232235919\\
1715	-89.5405808141682\\
1716	-92.8797541885185\\
1717	-77.3837788975193\\
1718	-94.8425394091414\\
1719	-71.009714896127\\
1720	-75.7864265945198\\
1721	-81.5318977426296\\
1722	-80.3663312730037\\
1723	-42.5235903987657\\
1724	-25.4128988972498\\
1725	-14.9395379790967\\
1726	-30.7302994763663\\
1727	-89.5048208173589\\
1728	-109.227703359876\\
1729	-101.201238507129\\
1730	-75.8586643710387\\
1731	-84.736841183565\\
1732	-77.5659946987207\\
1733	-68.421429371105\\
1734	-44.3528383022858\\
1735	-59.2656052742877\\
1736	-60.4132366787892\\
1737	-60.8271223514589\\
1738	-66.1442318532227\\
1739	-58.1921336249186\\
1740	-56.6775250015417\\
1741	-40.5508026130979\\
1742	-52.0168156549971\\
1743	-93.8797299681216\\
1744	-70.1161545039536\\
1745	-83.2342074207404\\
1746	-72.4207811082179\\
1747	-95.8231378499079\\
1748	-87.4603253293694\\
1749	-112.831327098046\\
1750	-91.5804253448787\\
1751	-94.8050569949041\\
1752	-98.7700790984736\\
1753	-54.2448783146642\\
1754	-92.2253621912619\\
1755	-63.5395615404916\\
1756	-83.2004627750866\\
1757	-84.4653789155584\\
1758	-100.547666553981\\
1759	-81.5230789870477\\
1760	-98.7724522223875\\
1761	-120.825788948112\\
1762	-101.567185162082\\
1763	-92.0529682147919\\
1764	-77.3935230487568\\
1765	-85.3466491886395\\
1766	-83.3946567412625\\
1767	-68.709452991556\\
1768	-63.4363489783104\\
1769	-71.0162384614316\\
1770	-63.7203088733008\\
1771	-113.754195149718\\
1772	-137.470386728632\\
1773	-117.907119431351\\
1774	-112.705994343065\\
1775	-112.789315497032\\
1776	-68.9381403266925\\
1777	-62.7321783252512\\
1778	-50.4847316648672\\
1779	-47.7012817181551\\
1780	-52.8840027910294\\
1781	-82.0755805096444\\
1783	-102.643816753876\\
1784	-91.8788750206902\\
1785	-68.4361924603375\\
1786	-112.704537163105\\
1787	-124.214619485128\\
1788	-131.363457200892\\
1789	-114.961374980003\\
1790	-157.363481743212\\
1791	-136.850465622811\\
1792	-78.0879286563609\\
1793	-55.7762819859674\\
1794	-51.4339606349486\\
1795	-85.5552043382572\\
1796	-104.109321394116\\
1797	-127.672848355037\\
1798	-105.983300566087\\
1799	-86.8744072696584\\
1800	-48.6452170590969\\
1801	-51.9971251600396\\
1802	-58.9259452037725\\
1803	-62.6680587330841\\
1804	-45.9363010158418\\
1805	-55.3976765030961\\
};
\addlegendentry{MPO prediction}

\end{axis}

\begin{axis}[%
width=6.159cm,
height=1.831cm,
at={(0cm,0cm)},
scale only axis,
xmin=1000,
xmax=2000,
xlabel style={font=\color{white!15!black}},
xlabel={Sample index},
ymin=-200,
ymax=0,
ylabel style={font=\color{white!15!black}},
ylabel={$y(t)$},
axis background/.style={fill=white},
title style={font=\bfseries},
title={C9: RMSE(OSA) = 3.9697, RMSE(MPO) = 6.995},
legend style={legend cell align=left, align=left, draw=white!15!black}
]
\addplot [color=mycolor1, line width=2.0pt]
  table[row sep=crcr]{%
1006	-86.6700000000001\\
1007	-103.76\\
1008	-79.346\\
1009	-45.1659999999999\\
1010	-57.373\\
1011	-51.27\\
1012	-63.4770000000001\\
1013	-48.828\\
1014	-24.414\\
1015	-17.0899999999999\\
1016	-17.0899999999999\\
1017	-43.9449999999999\\
1018	-73.242\\
1019	-76.904\\
1020	-79.346\\
1022	-36.6210000000001\\
1023	-76.904\\
1024	-53.711\\
1025	-43.9449999999999\\
1026	-69.5799999999999\\
1027	-61.0350000000001\\
1028	-48.828\\
1029	-84.229\\
1030	-79.346\\
1031	-59.8140000000001\\
1032	-87.8910000000001\\
1033	-85.4490000000001\\
1034	-67.1389999999999\\
1035	-54.932\\
1036	-46.3869999999999\\
1037	-56.152\\
1038	-64.6970000000001\\
1039	-64.6970000000001\\
1040	-67.1389999999999\\
1041	-68.3589999999999\\
1042	-73.242\\
1043	-100.098\\
1044	-87.8910000000001\\
1045	-61.0350000000001\\
1046	-56.152\\
1047	-34.1800000000001\\
1048	-34.1800000000001\\
1049	-48.828\\
1050	-37.8420000000001\\
1051	-39.0630000000001\\
1052	-56.152\\
1053	-62.2560000000001\\
1054	-58.5940000000001\\
1055	-90.3320000000001\\
1056	-64.6970000000001\\
1057	-41.5039999999999\\
1058	-34.1800000000001\\
1059	-45.1659999999999\\
1060	-51.27\\
1061	-28.076\\
1062	-25.635\\
1063	-41.5039999999999\\
1064	-34.1800000000001\\
1065	-29.297\\
1066	-40.2829999999999\\
1067	-43.9449999999999\\
1068	-68.3589999999999\\
1069	-63.4770000000001\\
1070	-79.346\\
1071	-69.5799999999999\\
1072	-79.346\\
1073	-63.4770000000001\\
1074	-62.2560000000001\\
1075	-54.932\\
1076	-53.711\\
1077	-53.711\\
1079	-126.953\\
1080	-131.836\\
1081	-129.395\\
1082	-83.008\\
1083	-114.746\\
1084	-141.602\\
1085	-147.705\\
1086	-111.084\\
1087	-146.484\\
1088	-185.547\\
1089	-131.836\\
1090	-112.305\\
1091	-75.684\\
1092	-61.0350000000001\\
1093	-52.49\\
1094	-64.6970000000001\\
1095	-46.3869999999999\\
1096	-31.7380000000001\\
1097	-31.7380000000001\\
1098	-40.2829999999999\\
1099	-57.373\\
1100	-52.49\\
1101	-53.711\\
1102	-72.021\\
1103	-72.021\\
1104	-97.6559999999999\\
1105	-79.346\\
1106	-101.318\\
1107	-72.021\\
1108	-64.6970000000001\\
1109	-73.242\\
1110	-54.932\\
1111	-51.27\\
1112	-59.8140000000001\\
1113	-62.2560000000001\\
1114	-43.9449999999999\\
1115	-47.607\\
1116	-34.1800000000001\\
1117	-40.2829999999999\\
1118	-42.7249999999999\\
1119	-30.518\\
1121	-47.607\\
1122	-72.021\\
1123	-74.463\\
1124	-81.787\\
1125	-48.828\\
1126	-70.8009999999999\\
1127	-101.318\\
1128	-84.229\\
1129	-93.9939999999999\\
1130	-91.5530000000001\\
1131	-54.932\\
1132	-40.2829999999999\\
1133	-56.152\\
1134	-61.0350000000001\\
1135	-97.6559999999999\\
1136	-119.629\\
1137	-117.188\\
1138	-89.1110000000001\\
1139	-92.7729999999999\\
1140	-81.787\\
1142	-79.346\\
1143	-54.932\\
1144	-48.828\\
1145	-45.1659999999999\\
1146	-40.2829999999999\\
1147	-45.1659999999999\\
1148	-64.6970000000001\\
1149	-96.4359999999999\\
1151	-52.49\\
1152	-45.1659999999999\\
1153	-46.3869999999999\\
1154	-29.297\\
1155	-23.193\\
1156	-26.855\\
1157	-46.3869999999999\\
1158	-35.4000000000001\\
1159	-41.5039999999999\\
1161	-39.0630000000001\\
1162	-28.076\\
1163	-21.973\\
1164	-17.0899999999999\\
1165	-29.297\\
1166	-58.5940000000001\\
1167	-80.566\\
1168	-91.5530000000001\\
1169	-59.8140000000001\\
1170	-43.9449999999999\\
1171	-32.9590000000001\\
1172	-23.193\\
1173	-47.607\\
1174	-46.3869999999999\\
1175	-67.1389999999999\\
1176	-81.787\\
1177	-131.836\\
1178	-147.705\\
1179	-120.85\\
1180	-117.188\\
1181	-78.125\\
1182	-81.787\\
1183	-91.5530000000001\\
1184	-98.877\\
1185	-73.242\\
1186	-68.3589999999999\\
1187	-69.5799999999999\\
1188	-62.2560000000001\\
1189	-75.684\\
1190	-53.711\\
1191	-93.9939999999999\\
1192	-108.643\\
1194	-51.27\\
1195	-54.932\\
1196	-92.7729999999999\\
1197	-109.863\\
1198	-134.277\\
1199	-140.381\\
1200	-140.381\\
1201	-106.201\\
1202	-96.4359999999999\\
1203	-111.084\\
1204	-129.395\\
1205	-78.125\\
1206	-50.049\\
1207	-73.242\\
1208	-57.373\\
1209	-39.0630000000001\\
1210	-56.152\\
1211	-61.0350000000001\\
1212	-46.3869999999999\\
1213	-37.8420000000001\\
1214	-50.049\\
1215	-39.0630000000001\\
1216	-53.711\\
1217	-76.904\\
1218	-72.021\\
1219	-51.27\\
1220	-51.27\\
1221	-78.125\\
1222	-123.291\\
1223	-86.6700000000001\\
1224	-56.152\\
1225	-43.9449999999999\\
1226	-48.828\\
1227	-43.9449999999999\\
1228	-34.1800000000001\\
1229	-37.8420000000001\\
1230	-47.607\\
1231	-59.8140000000001\\
1232	-64.6970000000001\\
1233	-46.3869999999999\\
1234	-74.463\\
1235	-85.4490000000001\\
1236	-83.008\\
1237	-65.9180000000001\\
1238	-73.242\\
1239	-96.4359999999999\\
1240	-61.0350000000001\\
1241	-29.297\\
1242	-42.7249999999999\\
1243	-43.9449999999999\\
1244	-58.5940000000001\\
1245	-50.049\\
1246	-40.2829999999999\\
1247	-58.5940000000001\\
1248	-54.932\\
1249	-39.0630000000001\\
1250	-48.828\\
1252	-39.0630000000001\\
1253	-50.049\\
1254	-29.297\\
1255	-36.6210000000001\\
1256	-26.855\\
1257	-31.7380000000001\\
1258	-53.711\\
1259	-61.0350000000001\\
1260	-76.904\\
1261	-101.318\\
1262	-67.1389999999999\\
1263	-42.7249999999999\\
1264	-32.9590000000001\\
1265	-31.7380000000001\\
1266	-47.607\\
1267	-30.518\\
1268	-36.6210000000001\\
1269	-43.9449999999999\\
1270	-65.9180000000001\\
1271	-62.2560000000001\\
1272	-86.6700000000001\\
1273	-59.8140000000001\\
1274	-56.152\\
1275	-31.7380000000001\\
1276	-32.9590000000001\\
1277	-20.752\\
1278	-24.414\\
1279	-26.855\\
1280	-47.607\\
1281	-47.607\\
1282	-56.152\\
1283	-59.8140000000001\\
1284	-93.9939999999999\\
1285	-79.346\\
1286	-62.2560000000001\\
1287	-73.242\\
1288	-79.346\\
1289	-101.318\\
1290	-80.566\\
1291	-52.49\\
1292	-30.518\\
1293	-21.973\\
1294	-23.193\\
1295	-29.297\\
1296	-29.297\\
1297	-26.855\\
1298	-36.6210000000001\\
1299	-53.711\\
1300	-47.607\\
1301	-51.27\\
1302	-57.373\\
1303	-37.8420000000001\\
1304	-20.752\\
1306	-63.4770000000001\\
1307	-61.0350000000001\\
1308	-69.5799999999999\\
1309	-58.5940000000001\\
1310	-45.1659999999999\\
1311	-59.8140000000001\\
1312	-84.229\\
1313	-85.4490000000001\\
1314	-59.8140000000001\\
1315	-102.539\\
1316	-76.904\\
1317	-74.463\\
1318	-84.229\\
1319	-83.008\\
1320	-63.4770000000001\\
1321	-54.932\\
1323	-124.512\\
1324	-95.2149999999999\\
1325	-57.373\\
1326	-53.711\\
1327	-54.932\\
1328	-72.021\\
1329	-83.008\\
1330	-59.8140000000001\\
1331	-64.6970000000001\\
1332	-83.008\\
1333	-90.3320000000001\\
1334	-61.0350000000001\\
1335	-50.049\\
1336	-63.4770000000001\\
1337	-104.98\\
1338	-92.7729999999999\\
1339	-95.2149999999999\\
1340	-58.5940000000001\\
1341	-53.711\\
1342	-51.27\\
1343	-34.1800000000001\\
1344	-23.193\\
1345	-20.752\\
1346	-17.0899999999999\\
1347	-40.2829999999999\\
1348	-54.932\\
1349	-62.2560000000001\\
1350	-47.607\\
1352	-57.373\\
1353	-37.8420000000001\\
1354	-39.0630000000001\\
1355	-39.0630000000001\\
1356	-28.076\\
1357	-39.0630000000001\\
1358	-57.373\\
1359	-73.242\\
1360	-48.828\\
1361	-35.4000000000001\\
1362	-31.7380000000001\\
1363	-36.6210000000001\\
1364	-26.855\\
1365	-54.932\\
1366	-93.9939999999999\\
1367	-72.021\\
1368	-97.6559999999999\\
1369	-113.525\\
1370	-114.746\\
1371	-85.4490000000001\\
1372	-63.4770000000001\\
1374	-63.4770000000001\\
1375	-75.684\\
1376	-89.1110000000001\\
1377	-86.6700000000001\\
1378	-115.967\\
1379	-125.732\\
1380	-150.146\\
1381	-96.4359999999999\\
1382	-109.863\\
1383	-122.07\\
1384	-92.7729999999999\\
1385	-107.422\\
1386	-92.7729999999999\\
1387	-54.932\\
1388	-39.0630000000001\\
1389	-40.2829999999999\\
1390	-57.373\\
1391	-42.7249999999999\\
1392	-32.9590000000001\\
1393	-45.1659999999999\\
1394	-34.1800000000001\\
1395	-26.855\\
1396	-34.1800000000001\\
1397	-37.8420000000001\\
1398	-25.635\\
1399	-48.828\\
1400	-64.6970000000001\\
1401	-46.3869999999999\\
1403	-115.967\\
1404	-106.201\\
1405	-133.057\\
1406	-92.7729999999999\\
1407	-97.6559999999999\\
1408	-69.5799999999999\\
1409	-43.9449999999999\\
1410	-53.711\\
1411	-42.7249999999999\\
1412	-34.1800000000001\\
1413	-45.1659999999999\\
1414	-26.855\\
1415	-20.752\\
1416	-25.635\\
1417	-50.049\\
1418	-61.0350000000001\\
1419	-78.125\\
1420	-74.463\\
1421	-69.5799999999999\\
1422	-80.566\\
1423	-64.6970000000001\\
1424	-53.711\\
1425	-54.932\\
1426	-43.9449999999999\\
1427	-69.5799999999999\\
1428	-50.049\\
1429	-40.2829999999999\\
1430	-58.5940000000001\\
1431	-80.566\\
1432	-59.8140000000001\\
1433	-46.3869999999999\\
1434	-41.5039999999999\\
1435	-47.607\\
1436	-32.9590000000001\\
1437	-31.7380000000001\\
1438	-43.9449999999999\\
1439	-53.711\\
1440	-59.8140000000001\\
1441	-48.828\\
1442	-61.0350000000001\\
1443	-57.373\\
1444	-34.1800000000001\\
1445	-37.8420000000001\\
1446	-65.9180000000001\\
1447	-91.5530000000001\\
1448	-90.3320000000001\\
1449	-75.684\\
1450	-73.242\\
1451	-57.373\\
1452	-53.711\\
1453	-43.9449999999999\\
1454	-61.0350000000001\\
1455	-54.932\\
1456	-36.6210000000001\\
1457	-52.49\\
1458	-61.0350000000001\\
1459	-70.8009999999999\\
1460	-76.904\\
1461	-76.904\\
1462	-102.539\\
1463	-124.512\\
1464	-140.381\\
1465	-91.5530000000001\\
1466	-52.49\\
1467	-34.1800000000001\\
1468	-25.635\\
1469	-36.6210000000001\\
1470	-32.9590000000001\\
1471	-58.5940000000001\\
1472	-53.711\\
1473	-46.3869999999999\\
1474	-62.2560000000001\\
1476	-81.787\\
1477	-86.6700000000001\\
1478	-122.07\\
1481	-58.5940000000001\\
1482	-58.5940000000001\\
1483	-59.8140000000001\\
1484	-75.684\\
1485	-54.932\\
1486	-56.152\\
1487	-53.711\\
1488	-30.518\\
1489	-54.932\\
1490	-70.8009999999999\\
1491	-50.049\\
1492	-53.711\\
1493	-100.098\\
1494	-75.684\\
1495	-72.021\\
1496	-102.539\\
1497	-98.877\\
1498	-62.2560000000001\\
1499	-40.2829999999999\\
1500	-42.7249999999999\\
1501	-70.8009999999999\\
1502	-107.422\\
1503	-112.305\\
1504	-115.967\\
1505	-90.3320000000001\\
1506	-59.8140000000001\\
1507	-52.49\\
1508	-39.0630000000001\\
1510	-31.7380000000001\\
1511	-45.1659999999999\\
1512	-50.049\\
1513	-30.518\\
1514	-42.7249999999999\\
1515	-61.0350000000001\\
1516	-68.3589999999999\\
1517	-84.229\\
1518	-107.422\\
1519	-128.174\\
1520	-95.2149999999999\\
1521	-81.787\\
1522	-85.4490000000001\\
1523	-72.021\\
1524	-52.49\\
1525	-63.4770000000001\\
1526	-102.539\\
1527	-115.967\\
1528	-69.5799999999999\\
1529	-40.2829999999999\\
1530	-29.297\\
1531	-17.0899999999999\\
1532	-21.973\\
1533	-32.9590000000001\\
1534	-36.6210000000001\\
1535	-52.49\\
1536	-43.9449999999999\\
1537	-31.7380000000001\\
1538	-34.1800000000001\\
1539	-31.7380000000001\\
1540	-30.518\\
1541	-34.1800000000001\\
1542	-51.27\\
1543	-75.684\\
1544	-76.904\\
1545	-75.684\\
1546	-86.6700000000001\\
1547	-107.422\\
1548	-107.422\\
1549	-128.174\\
1550	-115.967\\
1551	-86.6700000000001\\
1552	-61.0350000000001\\
1553	-58.5940000000001\\
1554	-98.877\\
1555	-111.084\\
1556	-79.346\\
1558	-29.297\\
1559	-20.752\\
1560	-24.414\\
1561	-18.3109999999999\\
1562	-29.297\\
1563	-43.9449999999999\\
1564	-42.7249999999999\\
1565	-39.0630000000001\\
1566	-26.855\\
1567	-20.752\\
1568	-29.297\\
1569	-29.297\\
1570	-31.7380000000001\\
1571	-24.414\\
1572	-21.973\\
1573	-34.1800000000001\\
1574	-78.125\\
1575	-91.5530000000001\\
1576	-100.098\\
1577	-70.8009999999999\\
1578	-97.6559999999999\\
1579	-140.381\\
1580	-128.174\\
1581	-128.174\\
1582	-97.6559999999999\\
1583	-95.2149999999999\\
1584	-102.539\\
1585	-68.3589999999999\\
1586	-64.6970000000001\\
1587	-42.7249999999999\\
1588	-41.5039999999999\\
1589	-69.5799999999999\\
1590	-53.711\\
1591	-51.27\\
1592	-57.373\\
1593	-54.932\\
1594	-54.932\\
1595	-59.8140000000001\\
1597	-72.021\\
1598	-61.0350000000001\\
1599	-45.1659999999999\\
1600	-59.8140000000001\\
1601	-26.855\\
1602	-18.3109999999999\\
1603	-23.193\\
1604	-37.8420000000001\\
1605	-23.193\\
1606	-10.9860000000001\\
1607	-21.973\\
1608	-36.6210000000001\\
1609	-42.7249999999999\\
1610	-51.27\\
1611	-43.9449999999999\\
1612	-68.3589999999999\\
1613	-57.373\\
1614	-40.2829999999999\\
1615	-58.5940000000001\\
1616	-41.5039999999999\\
1617	-29.297\\
1618	-23.193\\
1619	-15.8689999999999\\
1620	-25.635\\
1621	-19.5309999999999\\
1622	-35.4000000000001\\
1623	-54.932\\
1624	-53.711\\
1625	-36.6210000000001\\
1626	-43.9449999999999\\
1627	-34.1800000000001\\
1628	-43.9449999999999\\
1629	-36.6210000000001\\
1630	-31.7380000000001\\
1631	-31.7380000000001\\
1632	-25.635\\
1633	-31.7380000000001\\
1634	-28.076\\
1635	-43.9449999999999\\
1636	-43.9449999999999\\
1637	-34.1800000000001\\
1638	-32.9590000000001\\
1639	-26.855\\
1640	-30.518\\
1641	-64.6970000000001\\
1642	-84.229\\
1643	-56.152\\
1644	-56.152\\
1645	-37.8420000000001\\
1646	-65.9180000000001\\
1647	-62.2560000000001\\
1648	-39.0630000000001\\
1649	-48.828\\
1650	-41.5039999999999\\
1651	-51.27\\
1652	-36.6210000000001\\
1653	-32.9590000000001\\
1654	-43.9449999999999\\
1655	-52.49\\
1656	-37.8420000000001\\
1657	-42.7249999999999\\
1658	-45.1659999999999\\
1659	-64.6970000000001\\
1660	-58.5940000000001\\
1661	-81.787\\
1662	-111.084\\
1663	-86.6700000000001\\
1664	-57.373\\
1665	-43.9449999999999\\
1666	-67.1389999999999\\
1667	-80.566\\
1668	-63.4770000000001\\
1669	-75.684\\
1671	-25.635\\
1672	-26.855\\
1673	-57.373\\
1674	-53.711\\
1675	-48.828\\
1676	-74.463\\
1677	-68.3589999999999\\
1678	-61.0350000000001\\
1679	-45.1659999999999\\
1680	-54.932\\
1681	-73.242\\
1682	-59.8140000000001\\
1683	-59.8140000000001\\
1684	-56.152\\
1685	-42.7249999999999\\
1686	-46.3869999999999\\
1687	-36.6210000000001\\
1688	-40.2829999999999\\
1689	-67.1389999999999\\
1690	-78.125\\
1691	-80.566\\
1692	-69.5799999999999\\
1693	-52.49\\
1694	-37.8420000000001\\
1695	-42.7249999999999\\
1696	-51.27\\
1697	-63.4770000000001\\
1699	-90.3320000000001\\
1700	-72.021\\
1701	-42.7249999999999\\
1702	-73.242\\
1703	-106.201\\
1704	-75.684\\
1705	-72.021\\
1706	-86.6700000000001\\
1707	-65.9180000000001\\
1708	-54.932\\
1709	-63.4770000000001\\
1710	-57.373\\
1711	-64.6970000000001\\
1712	-79.346\\
1713	-62.2560000000001\\
1714	-70.8009999999999\\
1715	-67.1389999999999\\
1716	-67.1389999999999\\
1717	-54.932\\
1718	-69.5799999999999\\
1719	-52.49\\
1720	-52.49\\
1721	-59.8140000000001\\
1722	-57.373\\
1723	-31.7380000000001\\
1724	-20.752\\
1725	-15.8689999999999\\
1726	-25.635\\
1727	-59.8140000000001\\
1728	-78.125\\
1729	-75.684\\
1730	-52.49\\
1731	-62.2560000000001\\
1732	-57.373\\
1733	-47.607\\
1734	-34.1800000000001\\
1735	-42.7249999999999\\
1736	-43.9449999999999\\
1737	-41.5039999999999\\
1738	-46.3869999999999\\
1739	-40.2829999999999\\
1740	-37.8420000000001\\
1741	-29.297\\
1742	-39.0630000000001\\
1743	-63.4770000000001\\
1744	-47.607\\
1745	-54.932\\
1746	-54.932\\
1747	-67.1389999999999\\
1748	-64.6970000000001\\
1749	-85.4490000000001\\
1750	-67.1389999999999\\
1751	-72.021\\
1752	-73.242\\
1753	-41.5039999999999\\
1754	-64.6970000000001\\
1755	-42.7249999999999\\
1756	-58.5940000000001\\
1757	-63.4770000000001\\
1758	-76.904\\
1759	-59.8140000000001\\
1760	-75.684\\
1761	-93.9939999999999\\
1763	-64.6970000000001\\
1764	-56.152\\
1765	-63.4770000000001\\
1766	-56.152\\
1767	-47.607\\
1768	-41.5039999999999\\
1769	-48.828\\
1770	-43.9449999999999\\
1771	-79.346\\
1772	-107.422\\
1773	-91.5530000000001\\
1774	-87.8910000000001\\
1775	-86.6700000000001\\
1776	-50.049\\
1777	-46.3869999999999\\
1778	-37.8420000000001\\
1779	-32.9590000000001\\
1780	-35.4000000000001\\
1781	-57.373\\
1782	-73.242\\
1783	-80.566\\
1784	-68.3589999999999\\
1785	-47.607\\
1786	-84.229\\
1788	-103.76\\
1789	-84.229\\
1790	-134.277\\
1791	-101.318\\
1792	-56.152\\
1793	-42.7249999999999\\
1794	-36.6210000000001\\
1795	-61.0350000000001\\
1796	-79.346\\
1797	-100.098\\
1798	-79.346\\
1799	-63.4770000000001\\
1800	-35.4000000000001\\
1801	-36.6210000000001\\
1802	-39.0630000000001\\
1803	-43.9449999999999\\
1804	-30.518\\
1805	-36.6210000000001\\
};
\addlegendentry{True output}

\addplot [color=mycolor2, dashed, line width=2.0pt]
  table[row sep=crcr]{%
1006	-88.0285879115352\\
1007	-102.215711035965\\
1008	-81.6831289873805\\
1009	-44.4920181216498\\
1010	-54.985306145418\\
1011	-56.4484542579223\\
1012	-64.5118884980077\\
1013	-57.9839263317706\\
1014	-15.3954170179704\\
1015	-14.134123304558\\
1016	-15.0184584540866\\
1017	-52.9679495002224\\
1018	-79.9739725675174\\
1019	-80.7112588203793\\
1020	-78.9805772924817\\
1021	-62.5924597068915\\
1022	-37.2633624061154\\
1023	-80.3173608188365\\
1024	-55.2767804427144\\
1025	-43.8943133300031\\
1026	-70.9058074400448\\
1027	-62.1622231295964\\
1028	-54.8703675468307\\
1029	-84.6958816751428\\
1030	-79.3475972429915\\
1031	-60.5499688864111\\
1032	-83.543437228959\\
1033	-88.0062185179556\\
1034	-66.8904322958708\\
1035	-63.3578313779424\\
1036	-45.2458956911898\\
1037	-57.9837803079092\\
1038	-64.3173400806147\\
1039	-64.1325785430283\\
1040	-64.4937360726706\\
1041	-68.6029765182579\\
1042	-75.3440551683345\\
1043	-94.7097413133313\\
1044	-92.822450051226\\
1045	-60.1677385859389\\
1046	-64.8565941483398\\
1047	-30.7670965701113\\
1048	-37.9171489639798\\
1049	-48.3410616706255\\
1050	-38.7814124809793\\
1051	-44.6216654885179\\
1052	-52.5688349231084\\
1053	-65.674709617547\\
1054	-58.4728392821621\\
1055	-87.0264710373924\\
1056	-72.0287160996199\\
1057	-44.2040641636693\\
1058	-36.9531152536574\\
1059	-45.2589305291085\\
1060	-52.9261926880365\\
1061	-28.9114240055046\\
1062	-33.4936966288146\\
1063	-41.099549468745\\
1064	-35.3171522943915\\
1065	-30.4828537260851\\
1066	-41.2019679074913\\
1067	-45.7849303524986\\
1068	-68.6534598731182\\
1069	-67.0778155985163\\
1070	-74.4460082106423\\
1071	-73.8007122818619\\
1072	-76.7434704679131\\
1073	-66.145620320927\\
1074	-64.4597041673103\\
1075	-58.5765627136689\\
1076	-57.8885953262118\\
1077	-56.8517050632317\\
1078	-88.5924541764355\\
1079	-126.841186343893\\
1080	-126.793506303385\\
1081	-125.786298417855\\
1082	-88.0450261745218\\
1084	-141.231457589399\\
1085	-137.971730374414\\
1086	-113.23217427389\\
1087	-143.799593485876\\
1088	-179.112564427322\\
1089	-148.840168720271\\
1091	-80.7548784944224\\
1092	-61.8164131242606\\
1093	-56.4945495446345\\
1094	-65.764968386057\\
1095	-45.758830168103\\
1096	-28.6855709551546\\
1097	-30.7994283379639\\
1098	-39.1720566762408\\
1099	-63.1364450832671\\
1100	-53.7189856288587\\
1101	-59.4439681575623\\
1102	-68.890908266754\\
1103	-74.1716657860939\\
1104	-97.4251887370201\\
1105	-91.481510677623\\
1106	-93.2564174116239\\
1107	-83.3150065701034\\
1108	-63.359709850246\\
1109	-80.4355065266595\\
1110	-55.9135815171687\\
1111	-53.5366342533653\\
1112	-64.4399950566783\\
1113	-62.7596422382162\\
1114	-44.0568106735677\\
1115	-52.2108899649932\\
1116	-38.2361727628274\\
1118	-44.9250858249509\\
1119	-32.6137614950287\\
1120	-39.879192232376\\
1121	-50.1758846971734\\
1122	-75.847469257825\\
1123	-75.7258123532113\\
1124	-78.3152065913346\\
1125	-55.4764170742947\\
1126	-68.5760370759867\\
1127	-109.816351102771\\
1128	-80.3816442708276\\
1129	-95.2964227082816\\
1130	-91.2658572348821\\
1131	-59.5543480046686\\
1132	-41.5834270197906\\
1133	-59.1339002953912\\
1134	-64.5568894975361\\
1135	-95.2532168648877\\
1136	-117.810036776633\\
1137	-111.732428271931\\
1138	-98.3383492533032\\
1139	-92.1238143515361\\
1140	-91.0498051866114\\
1141	-82.3628874194176\\
1142	-80.7765547877084\\
1143	-55.610329919004\\
1144	-55.0191894253658\\
1145	-46.4118724301356\\
1146	-43.8627966506917\\
1147	-49.9326264675267\\
1148	-63.0601857906647\\
1149	-96.9460034446645\\
1151	-56.8473502213412\\
1152	-49.3404150668575\\
1153	-49.9879118633232\\
1154	-27.6532674561656\\
1155	-20.678471833975\\
1156	-27.0550891996945\\
1157	-49.5567753830769\\
1158	-35.149882108471\\
1159	-42.6713636673044\\
1160	-42.6007404153261\\
1161	-42.9084052116521\\
1162	-34.3305470947375\\
1163	-21.7491725852469\\
1164	-17.5065602343291\\
1165	-30.4442818999403\\
1166	-64.9166702150192\\
1167	-89.899943137314\\
1168	-90.4022695370861\\
1169	-64.7763726968433\\
1170	-46.9334668186007\\
1171	-34.5851308531676\\
1172	-25.4573126078676\\
1173	-51.02920958753\\
1174	-47.5806744448539\\
1175	-64.5485712800919\\
1176	-84.1940594167752\\
1177	-133.107958664201\\
1178	-155.39452382523\\
1179	-125.654575621333\\
1180	-123.971144783733\\
1181	-78.7963637543378\\
1182	-90.4078300892377\\
1183	-90.6949578396279\\
1184	-97.5486261762956\\
1185	-81.5933547477209\\
1186	-69.4250052956027\\
1187	-72.6772338530095\\
1188	-67.8764416168758\\
1189	-75.0924001703027\\
1190	-59.0154345776018\\
1191	-89.7911939816038\\
1192	-109.86653171068\\
1193	-82.6712415270667\\
1194	-52.1711464557804\\
1195	-59.3189878067574\\
1196	-92.5100524989336\\
1197	-108.739546852764\\
1198	-131.066166677327\\
1199	-134.847747373575\\
1200	-140.188607915595\\
1201	-115.294815259751\\
1202	-97.2877002867399\\
1203	-115.47128843651\\
1204	-122.621256166135\\
1205	-80.2909481368183\\
1206	-54.2443287114208\\
1207	-72.8507757858545\\
1208	-68.0519799851349\\
1209	-39.2360468545833\\
1210	-55.7164346579821\\
1211	-61.8468283118912\\
1212	-53.6946443043319\\
1213	-38.8834860484969\\
1214	-55.3856895127451\\
1215	-45.139420136928\\
1216	-53.7265308404665\\
1217	-78.898897781602\\
1218	-70.5106769417475\\
1219	-54.7636333166286\\
1220	-56.5159461889755\\
1221	-78.1436885154371\\
1222	-128.016659727485\\
1223	-96.6034725974375\\
1224	-56.5213816694479\\
1225	-43.1629452476939\\
1226	-56.0663444984154\\
1227	-47.399024797953\\
1228	-37.8047914856286\\
1229	-41.7740470348238\\
1230	-48.9306879813503\\
1231	-60.0538703584934\\
1232	-66.5958095274698\\
1233	-47.895236598283\\
1234	-80.1480360713015\\
1235	-88.9064046577673\\
1236	-81.293995974057\\
1237	-71.0615143121836\\
1238	-71.0098650907833\\
1239	-97.4795751136326\\
1240	-67.5274957428715\\
1241	-25.3165361553615\\
1242	-41.2723621761431\\
1243	-48.3522711316996\\
1244	-65.236262383452\\
1245	-52.1125110142571\\
1246	-40.9436562820583\\
1247	-61.7086651090569\\
1248	-57.9341428752709\\
1249	-42.2139942769188\\
1250	-55.9318936910422\\
1251	-46.5027930207264\\
1252	-46.2539916279727\\
1253	-47.4973510063176\\
1254	-32.3039150626544\\
1255	-37.4783547078125\\
1256	-27.8596384759633\\
1257	-33.7631383274745\\
1258	-57.8842108752945\\
1259	-62.3822225896406\\
1260	-79.2078156574555\\
1261	-105.011506316376\\
1262	-72.9542368116613\\
1263	-44.5970740181237\\
1264	-34.321971850401\\
1265	-35.445592693015\\
1266	-50.0429496709532\\
1267	-31.0769067153899\\
1268	-37.4187282235105\\
1269	-47.1829018856631\\
1270	-71.5323790399918\\
1271	-65.1194828703537\\
1272	-86.3962086467134\\
1273	-61.3097648281357\\
1274	-57.9584099058316\\
1275	-29.8272806371017\\
1276	-32.1182189821159\\
1277	-22.4797206551364\\
1278	-24.0619315359629\\
1279	-31.0671600066203\\
1280	-48.5665930938112\\
1281	-55.2377570101964\\
1282	-54.8991441051862\\
1283	-63.0980632206824\\
1284	-95.1665198547043\\
1285	-89.1885337175017\\
1286	-63.6563499748186\\
1287	-76.6645632945458\\
1288	-76.1972745847061\\
1289	-97.7248127245248\\
1290	-84.8490253285302\\
1292	-25.4487171090907\\
1293	-23.1289064266309\\
1294	-24.5230790222022\\
1295	-28.3729104782369\\
1296	-31.1634903364111\\
1297	-29.4953474474469\\
1298	-37.8092428682621\\
1299	-59.1172281011748\\
1300	-50.9490471940624\\
1301	-51.3004837071874\\
1302	-59.8528371647274\\
1303	-39.5264881236017\\
1304	-23.3482678043422\\
1306	-66.6596044013495\\
1307	-61.1549359294845\\
1308	-68.2646794989512\\
1309	-61.9549976007377\\
1310	-46.6050107654498\\
1311	-64.3721926849737\\
1312	-84.6534732331484\\
1313	-81.5835323657298\\
1314	-62.1406993076673\\
1315	-103.011274976048\\
1316	-80.8364824991265\\
1317	-77.5278342311076\\
1318	-84.5607591126545\\
1319	-84.2644071305572\\
1320	-65.4515693310286\\
1321	-61.4945052789865\\
1322	-88.0646991415645\\
1323	-124.124411939208\\
1324	-94.2128698976351\\
1325	-59.7251382585757\\
1326	-57.3455323298817\\
1327	-58.8836257583087\\
1328	-73.3414946660159\\
1329	-81.1800386383736\\
1330	-60.4222486457616\\
1332	-83.6210890337081\\
1333	-84.4673434528195\\
1334	-64.7670628694514\\
1335	-51.1215718336587\\
1336	-69.4125817493939\\
1337	-104.471022337675\\
1338	-96.4368016505243\\
1339	-91.022615944521\\
1340	-62.0418612150374\\
1341	-54.2282726110691\\
1342	-60.3308395067681\\
1343	-31.4906598855071\\
1344	-21.3381964963864\\
1345	-17.3631066851053\\
1346	-16.7842640925958\\
1347	-42.4334188615953\\
1348	-58.3647469563321\\
1349	-65.4505703715822\\
1350	-51.4676621563892\\
1351	-52.4494996251208\\
1352	-61.5069965888295\\
1353	-38.1891023000392\\
1354	-42.4019879188766\\
1355	-41.1598698760481\\
1356	-32.8767990451302\\
1357	-36.4734110065585\\
1358	-62.8281244024843\\
1359	-73.9772370645753\\
1360	-48.1272194543358\\
1361	-40.0054357503734\\
1362	-33.6054983803444\\
1363	-42.2154412837037\\
1364	-25.218828631791\\
1366	-100.237801985772\\
1367	-72.9216561215148\\
1368	-96.4440432029032\\
1369	-111.199224087713\\
1370	-113.012705007757\\
1371	-91.6760225608218\\
1372	-67.8507827878352\\
1373	-67.6076709944373\\
1374	-68.5811136665691\\
1375	-75.5899135883706\\
1376	-84.1080658379797\\
1377	-88.2887133073361\\
1378	-111.62395664271\\
1379	-123.360514678596\\
1380	-143.068182171462\\
1381	-106.049741304573\\
1382	-107.688158921769\\
1383	-121.529446164638\\
1384	-101.798126459501\\
1385	-110.587346242849\\
1386	-93.1703824717913\\
1387	-57.0050647485918\\
1388	-31.9285330457294\\
1389	-42.7314806554293\\
1390	-59.4147952325002\\
1391	-50.8099093592496\\
1392	-34.4891186248351\\
1393	-45.636130637007\\
1394	-38.0634035426253\\
1395	-25.3659259689832\\
1396	-36.9525252359974\\
1397	-38.5309418663958\\
1398	-31.8415397434806\\
1400	-68.9307862475118\\
1401	-53.1742790919288\\
1402	-84.5589504039622\\
1403	-123.948066779581\\
1404	-105.500606464663\\
1405	-124.069393242502\\
1406	-98.9575513192815\\
1407	-98.975282522186\\
1408	-73.5811804994423\\
1409	-43.6885741227534\\
1410	-61.8306912370817\\
1411	-42.158828474179\\
1412	-33.8735605622999\\
1413	-50.6573806709373\\
1414	-26.8089135375558\\
1415	-18.3891582880356\\
1416	-25.405904586177\\
1417	-52.9539212183709\\
1418	-71.2151304019255\\
1419	-75.6403610205693\\
1420	-76.5781355356371\\
1421	-73.5883471793484\\
1422	-80.0123102865875\\
1423	-65.485208862533\\
1424	-60.8945265739267\\
1425	-60.3567584058123\\
1426	-46.8807185199373\\
1427	-70.8011754882029\\
1428	-47.4239629706838\\
1429	-45.9615167897955\\
1430	-58.7397984652378\\
1431	-81.3428114339397\\
1432	-62.2014507650715\\
1433	-50.3671309824983\\
1434	-45.5057130606895\\
1435	-51.5736342149055\\
1436	-34.0409260673839\\
1437	-33.6634006217707\\
1438	-46.6088991410381\\
1439	-55.3597820580587\\
1440	-60.9295922043277\\
1441	-50.4010221604919\\
1442	-61.9735368300276\\
1443	-59.7164443280819\\
1444	-33.402233050409\\
1445	-42.7150765644537\\
1447	-98.1498875603704\\
1448	-84.9150411494263\\
1449	-78.9398873722043\\
1450	-73.5874117369169\\
1451	-61.858754835094\\
1452	-61.3571420522958\\
1453	-46.8642046970783\\
1454	-63.434070006381\\
1455	-54.2596954139626\\
1456	-36.5959110288923\\
1457	-58.1150552948659\\
1458	-61.5317596950358\\
1459	-70.4238331831648\\
1461	-78.3947331502843\\
1462	-100.029508193404\\
1463	-127.70776765095\\
1464	-136.56423643696\\
1466	-55.1907478645328\\
1467	-35.5919547077767\\
1468	-22.2279734470428\\
1469	-42.5285250085599\\
1470	-36.8062545071093\\
1471	-61.4172156966388\\
1472	-55.1673511372387\\
1473	-47.6141700350827\\
1474	-63.3166427764384\\
1475	-70.3189160192105\\
1476	-84.0059042674379\\
1477	-87.5296841410425\\
1478	-119.151823331266\\
1479	-117.778779906445\\
1480	-83.0474718524902\\
1481	-59.0910821723571\\
1482	-64.0996664374472\\
1483	-63.1040289332\\
1484	-75.102950201652\\
1485	-56.7598010923621\\
1486	-57.5100179303613\\
1487	-58.919491247543\\
1488	-26.6751705959978\\
1489	-51.9251143902495\\
1490	-79.5676572951459\\
1491	-47.7574417978612\\
1492	-63.6666792771875\\
1493	-101.708622010086\\
1494	-75.4774835751391\\
1495	-74.9158206884269\\
1496	-101.860854022534\\
1497	-97.0246255762329\\
1498	-66.9114417930389\\
1499	-42.5256464806944\\
1500	-48.0614731122134\\
1501	-69.2361408398674\\
1502	-116.356294181352\\
1503	-109.396485155378\\
1504	-110.928727364589\\
1505	-92.69931154712\\
1506	-64.6070611596722\\
1508	-42.2369650566293\\
1509	-39.6467056390604\\
1510	-32.2506146825872\\
1511	-45.5956093052057\\
1512	-51.8332616438024\\
1513	-28.511151635161\\
1514	-50.2120524930474\\
1515	-62.0496172884129\\
1516	-68.8224316938138\\
1517	-89.8682721426919\\
1518	-104.750992114102\\
1519	-130.759602047643\\
1520	-99.6901483056911\\
1521	-86.0910402209402\\
1522	-92.5599818332144\\
1523	-75.3083806955567\\
1524	-60.7667177673957\\
1525	-59.6906756303742\\
1526	-104.031847943533\\
1527	-112.682046114503\\
1528	-66.5828343140677\\
1529	-43.1374648464466\\
1531	-9.58708478978474\\
1532	-21.3328446963367\\
1533	-36.6564687341045\\
1534	-40.2439542936395\\
1535	-57.1204408437175\\
1536	-44.8610618114699\\
1537	-37.0486656925623\\
1538	-33.841701521882\\
1539	-37.1135699549727\\
1540	-30.3841790661161\\
1541	-38.0339067327495\\
1542	-52.9064307523026\\
1543	-79.1382825950795\\
1544	-82.8882194239609\\
1545	-74.4709088040506\\
1546	-87.1567041521428\\
1547	-108.511183658024\\
1548	-108.794375681202\\
1549	-124.696165733878\\
1550	-126.402872483139\\
1551	-85.6019348567443\\
1552	-67.9434942033217\\
1553	-60.8607679212066\\
1554	-105.737917449366\\
1555	-104.296600299321\\
1556	-83.0532247052233\\
1557	-51.2398046268374\\
1558	-25.524621290145\\
1559	-18.2294265207856\\
1560	-24.2183129897035\\
1561	-11.4724226325598\\
1562	-35.4928358670595\\
1563	-44.7630803587679\\
1564	-45.793143201437\\
1565	-42.1332252298055\\
1566	-24.4611577920625\\
1567	-23.6524713660606\\
1568	-30.2798473108946\\
1569	-30.8266318521069\\
1570	-34.4607543934335\\
1572	-23.093835053721\\
1573	-37.6593357796503\\
1574	-90.2934729058493\\
1575	-101.850624642375\\
1576	-99.7553547803727\\
1577	-78.2210655550919\\
1578	-93.4979572506418\\
1579	-147.242547185352\\
1580	-128.40329012497\\
1581	-127.186502706852\\
1582	-105.222365257344\\
1583	-99.4533939374999\\
1584	-103.020585507872\\
1585	-71.7963053541914\\
1586	-68.877889828103\\
1587	-37.7081528881047\\
1588	-40.3324045581014\\
1589	-73.3808908535577\\
1590	-48.5423831106993\\
1591	-57.1772622940609\\
1592	-64.2948511704103\\
1593	-56.6008896721667\\
1594	-60.0879879964193\\
1595	-57.6858878132145\\
1596	-68.4845487451471\\
1597	-72.4546789101646\\
1598	-61.9849488993848\\
1599	-54.105305270537\\
1600	-60.7297258389322\\
1601	-22.3490584253875\\
1602	-12.4386294311721\\
1603	-23.9567741836095\\
1604	-40.6215985542258\\
1605	-15.9527626718209\\
1606	-13.8818814903516\\
1607	-23.3666457762595\\
1608	-37.551447092035\\
1609	-45.2224399434608\\
1610	-55.8651434580454\\
1611	-46.8464134417104\\
1612	-69.7077819791439\\
1613	-64.2518578779295\\
1614	-41.2454118827786\\
1615	-62.528984161213\\
1616	-32.4359655689332\\
1617	-32.9933571791012\\
1618	-21.9773228688728\\
1619	-15.5717976134099\\
1620	-29.9554486492048\\
1621	-17.8188279501092\\
1622	-33.7823522659678\\
1623	-59.8847584128387\\
1624	-54.7496525026838\\
1625	-43.160730744921\\
1626	-42.7300145282356\\
1627	-37.2832019891423\\
1628	-47.0016816069356\\
1629	-33.1364370891033\\
1630	-37.9234598639741\\
1631	-31.2472607724631\\
1632	-26.8586317893532\\
1633	-32.1955433774604\\
1634	-28.8558167678193\\
1635	-48.6649153374967\\
1636	-41.3767716789453\\
1637	-35.7517605471874\\
1638	-36.7033162502025\\
1639	-28.441220422656\\
1640	-32.8062347708435\\
1641	-66.4179886200611\\
1642	-90.0023760638455\\
1643	-61.5681890655706\\
1644	-57.3077992583837\\
1645	-39.8818040780432\\
1646	-64.0420348190062\\
1647	-64.2960267205901\\
1648	-39.9231296534674\\
1649	-51.9045678160926\\
1650	-40.5302712243601\\
1651	-55.8535966440477\\
1652	-29.1018800936406\\
1653	-38.0250752885574\\
1654	-42.7483176984549\\
1655	-53.9568704192905\\
1656	-42.4260546104936\\
1657	-41.3367984263994\\
1658	-46.9765438910003\\
1659	-67.2669258672584\\
1660	-58.1725773504322\\
1661	-83.5979491295595\\
1662	-112.708471210604\\
1664	-61.6271100892764\\
1665	-42.5269470583005\\
1666	-68.7913303849964\\
1667	-83.6763536539074\\
1668	-62.4726800988155\\
1669	-75.2771294031147\\
1670	-49.3472206133954\\
1671	-26.1310987246625\\
1672	-29.3185964401696\\
1673	-63.543890255595\\
1675	-47.2524839773207\\
1676	-73.0757638678933\\
1677	-70.3148518030746\\
1678	-64.1642492749118\\
1679	-44.9144082314276\\
1680	-59.2202991022843\\
1681	-70.5124897545984\\
1682	-60.3855592914324\\
1683	-63.8914590870299\\
1684	-55.4934121957031\\
1685	-44.201500952156\\
1686	-52.5990398762549\\
1687	-34.6023563138656\\
1688	-45.121940622015\\
1689	-68.8724064980845\\
1690	-72.4443949406648\\
1691	-79.1618592222881\\
1692	-69.7855836427411\\
1693	-57.2144855719637\\
1694	-40.7433922681062\\
1695	-46.5484038622385\\
1696	-53.6693133304379\\
1697	-59.4847327909474\\
1698	-79.020253663301\\
1699	-88.7876144698082\\
1700	-70.7680492715056\\
1701	-45.1586027729165\\
1702	-78.7813186416674\\
1703	-105.487163022612\\
1704	-71.9523175769361\\
1705	-77.9904029598338\\
1706	-81.8352719891973\\
1707	-73.62601458872\\
1708	-55.550822924409\\
1709	-65.2313518107435\\
1710	-60.0637437033674\\
1712	-74.3868794313166\\
1713	-67.5762224133571\\
1714	-68.6679539767345\\
1715	-68.8999878510742\\
1716	-72.4038742196092\\
1717	-54.9163619722542\\
1718	-71.0075269917666\\
1719	-51.8250869092114\\
1720	-54.2375966475236\\
1721	-64.550014103721\\
1722	-56.800550401756\\
1723	-31.5732746259366\\
1724	-18.1089066926327\\
1725	-14.1926963502558\\
1726	-26.3642264095076\\
1727	-68.6816716031037\\
1728	-83.1854190170325\\
1729	-77.7874989480142\\
1730	-56.3809351717218\\
1731	-62.312076520247\\
1732	-60.0407545525677\\
1733	-51.0624309319178\\
1734	-32.4462338857784\\
1735	-45.5624938184073\\
1736	-47.6483624487132\\
1737	-41.5536169521722\\
1738	-49.2793205174135\\
1739	-41.540227450874\\
1740	-43.3119335624949\\
1741	-26.534126241402\\
1742	-38.9695825723734\\
1743	-66.0266476867446\\
1744	-47.0371741585238\\
1745	-58.940927384017\\
1746	-54.5013275048711\\
1747	-68.4041350476489\\
1748	-66.1128433755739\\
1749	-83.6528580241261\\
1750	-66.6713195092495\\
1751	-72.8085691969902\\
1752	-74.4644742761436\\
1753	-40.7929620547363\\
1754	-75.208954460868\\
1755	-38.8490426754015\\
1756	-61.1158143252235\\
1757	-57.3093841148602\\
1758	-79.3738089839997\\
1759	-60.49482992346\\
1760	-75.013148215148\\
1761	-92.3373494764771\\
1762	-77.8859332746788\\
1763	-70.77119993256\\
1764	-60.0913209900184\\
1765	-66.6826790555181\\
1766	-58.2789814033515\\
1767	-51.4972271978042\\
1768	-47.0011589867402\\
1769	-46.5789958408018\\
1770	-47.6686757410823\\
1771	-80.6539042112538\\
1772	-101.400106164962\\
1773	-88.2005150744524\\
1774	-91.8746032787092\\
1775	-84.1806340991229\\
1776	-59.4898112879116\\
1777	-47.4062247507859\\
1778	-44.1245608102111\\
1779	-32.6714661266394\\
1780	-40.5715751181187\\
1782	-70.6168668944827\\
1783	-78.6028152263254\\
1784	-71.3943813619389\\
1785	-49.4703668405057\\
1786	-82.4100257781897\\
1787	-100.14382449023\\
1788	-96.2018538807063\\
1789	-86.5980134209233\\
1790	-129.391606303591\\
1791	-106.884201275107\\
1792	-61.0165024250202\\
1793	-43.4518724433453\\
1794	-37.1644814614701\\
1795	-67.2374748146965\\
1796	-77.6022842927121\\
1797	-98.9554803530682\\
1798	-78.48670952467\\
1799	-65.6652213965067\\
1800	-38.1812331318513\\
1801	-37.779983090883\\
1802	-46.5213961698248\\
1803	-42.5221749576385\\
1804	-33.4584395914771\\
1805	-37.5433385370898\\
};
\addlegendentry{OSA predition}

\addplot [color=mycolor3, dotted, line width=2.0pt]
  table[row sep=crcr]{%
1006	-86.6700000000001\\
1007	-103.76\\
1008	-79.346\\
1009	-45.1659999999999\\
1010	-54.985306145418\\
1011	-55.4723037098422\\
1012	-65.9310471815597\\
1013	-59.3187067273664\\
1014	-19.7738936413107\\
1015	-13.8877055822079\\
1016	-13.1087466889533\\
1017	-51.5662156054195\\
1018	-81.6744259845425\\
1019	-83.9720117462523\\
1020	-83.4706478623773\\
1021	-66.5339544065932\\
1022	-42.0103416319337\\
1023	-84.5950035389658\\
1024	-60.0473926037491\\
1025	-48.5097871076734\\
1026	-74.769172394585\\
1027	-66.1127204620491\\
1028	-58.5496360324389\\
1029	-90.3343503207998\\
1030	-84.0900200048302\\
1031	-64.5002863120983\\
1032	-87.5956773260118\\
1033	-89.5381263265219\\
1034	-69.3507643315261\\
1035	-65.2252723651466\\
1036	-49.8464071027258\\
1037	-61.236677837717\\
1038	-67.8776848261393\\
1039	-67.377165825668\\
1040	-66.7593325070368\\
1041	-69.5462635629835\\
1042	-76.3404780665473\\
1043	-96.3424663503281\\
1044	-91.770255507389\\
1045	-61.5882149064362\\
1046	-65.6864851657565\\
1047	-34.3884232570856\\
1048	-39.5999797699262\\
1049	-51.180545490115\\
1050	-41.34285884271\\
1051	-46.7495970736563\\
1053	-67.4725081611748\\
1054	-61.7582449910374\\
1055	-89.7924437617896\\
1056	-72.7552381826167\\
1057	-48.0977806358167\\
1058	-40.9687624970613\\
1059	-49.1465394308495\\
1060	-56.6612523221925\\
1061	-32.530655877332\\
1062	-36.7963258914949\\
1063	-46.9262894870365\\
1064	-39.6454104015229\\
1065	-34.385889876736\\
1066	-45.3916511089792\\
1067	-49.4370469972262\\
1068	-72.6499522774479\\
1069	-70.5557267964639\\
1070	-78.9899701728189\\
1071	-75.466765516227\\
1072	-80.1990110713855\\
1073	-67.8979131570218\\
1074	-66.7407105619452\\
1075	-61.5961107806356\\
1076	-61.5447659042454\\
1077	-61.6301229961382\\
1078	-94.043087163873\\
1079	-130.80636604649\\
1080	-130.764437774528\\
1081	-126.961217008852\\
1082	-87.3957500289514\\
1083	-115.840064908424\\
1084	-141.875757441045\\
1085	-138.581759562654\\
1086	-110.074167394769\\
1088	-175.948257884421\\
1089	-143.320387469622\\
1090	-116.7923881165\\
1091	-83.5557815674599\\
1092	-64.8326372029721\\
1093	-60.3605010243391\\
1094	-70.3744223486322\\
1095	-50.0526651112996\\
1096	-31.426012392262\\
1097	-32.0106096393488\\
1098	-39.9587535867402\\
1099	-63.3652548251655\\
1100	-56.1264291305865\\
1101	-61.638743039385\\
1102	-73.143585677486\\
1103	-76.5641069888304\\
1104	-100.816869974129\\
1105	-94.2509457300728\\
1106	-100.378598183261\\
1108	-69.9041733151878\\
1109	-85.7405516282358\\
1110	-62.3112049310462\\
1111	-59.7677811451049\\
1112	-70.2590991829913\\
1113	-69.9210202134825\\
1114	-49.6874037944922\\
1115	-56.9855766121643\\
1116	-44.1333013486833\\
1117	-48.0128805228405\\
1118	-50.4744830077439\\
1119	-38.1181543571267\\
1120	-45.37129971385\\
1121	-55.042120594477\\
1122	-81.3590085983312\\
1123	-81.9718540953622\\
1124	-84.0045121523926\\
1125	-58.6267235730718\\
1126	-74.3422014252969\\
1127	-113.535135390311\\
1128	-87.0970879454328\\
1129	-99.3719372888572\\
1130	-95.2862652601273\\
1131	-63.1132814285843\\
1132	-45.7820630384904\\
1133	-63.1610100770602\\
1134	-69.1158075563396\\
1135	-100.981822161234\\
1136	-121.823412678296\\
1137	-114.872659564717\\
1138	-98.7962739674401\\
1140	-93.8308617074235\\
1141	-88.2359615631694\\
1142	-86.7250156417385\\
1143	-60.5044326163782\\
1144	-59.7817465430428\\
1145	-52.657185913675\\
1146	-49.3000859294127\\
1147	-55.8218940477066\\
1148	-70.3135785815175\\
1149	-102.572928567921\\
1151	-62.4285042867614\\
1152	-55.2663648508951\\
1153	-56.3655097402559\\
1154	-34.0196730743482\\
1155	-24.8717169608205\\
1156	-29.5299517667986\\
1157	-52.2208847681216\\
1158	-38.4334850754951\\
1159	-44.9621123215077\\
1160	-45.0367362064642\\
1161	-46.0437457132775\\
1162	-38.2321930546923\\
1163	-27.1949555259848\\
1164	-21.3816881530242\\
1165	-33.992591238637\\
1166	-68.9850611459115\\
1167	-95.9339527279108\\
1168	-99.0616755503197\\
1169	-71.1324213373734\\
1170	-54.1213879372251\\
1171	-41.7828153035891\\
1172	-31.0935095037134\\
1173	-57.1698083024146\\
1174	-54.1656684250054\\
1175	-70.643477510285\\
1176	-88.5853873771234\\
1177	-138.698846209218\\
1178	-160.488635490564\\
1179	-133.096066769908\\
1180	-132.109904260244\\
1181	-87.7893875505024\\
1182	-98.3691776731296\\
1183	-101.070830728057\\
1184	-106.039147700348\\
1185	-88.0503745206363\\
1186	-78.6558955468795\\
1187	-80.2586362752575\\
1188	-75.2708172297037\\
1189	-84.1402934669029\\
1190	-65.8539988132736\\
1191	-98.3131766912086\\
1192	-115.575725632599\\
1193	-88.1110774378592\\
1194	-57.5346137323133\\
1195	-63.5752676997824\\
1196	-98.2814042952155\\
1197	-113.694766228207\\
1198	-135.374094324226\\
1199	-137.451182614884\\
1200	-140.177774911753\\
1201	-115.281234828344\\
1202	-100.89440719741\\
1203	-118.257723099809\\
1204	-127.101323651908\\
1205	-81.4910947915564\\
1206	-55.876248134814\\
1207	-76.2966226463811\\
1208	-70.0514504915018\\
1209	-45.3797812517676\\
1210	-60.6975790583938\\
1211	-65.7406461328424\\
1212	-57.9038943776638\\
1213	-45.0313702270259\\
1214	-60.5532670097436\\
1215	-51.5068133503523\\
1216	-61.8412602555757\\
1217	-85.6862631669321\\
1218	-77.4574713343768\\
1219	-59.8895033080298\\
1220	-62.0572778304336\\
1221	-85.0262930307451\\
1222	-134.156349623851\\
1223	-104.202364178877\\
1224	-66.5684275737831\\
1225	-50.5825632345438\\
1226	-61.8387302591666\\
1227	-55.8178277303155\\
1228	-45.6289743675338\\
1229	-49.314824950279\\
1230	-57.1790948817659\\
1231	-67.6409668367048\\
1232	-73.2988007982708\\
1233	-54.1720045384443\\
1234	-86.1832443604396\\
1235	-96.4576230296382\\
1236	-88.9771433841231\\
1237	-76.6247386688999\\
1238	-78.0944092431635\\
1239	-102.562913232764\\
1240	-72.0294098973573\\
1241	-31.6624455195104\\
1242	-44.2527159365388\\
1243	-50.3534872155715\\
1244	-69.3645005778681\\
1246	-45.8912592037882\\
1247	-66.4582676993166\\
1248	-63.5170208974412\\
1249	-47.7718998291562\\
1250	-61.6969541264573\\
1251	-54.161425660446\\
1252	-53.436951813937\\
1253	-56.4003083799696\\
1254	-38.3380583647534\\
1255	-43.6767122726512\\
1256	-33.4701048408112\\
1257	-38.3933338589893\\
1258	-62.9851929051849\\
1259	-68.4085355911336\\
1260	-84.8642498175234\\
1261	-111.227336132376\\
1262	-79.6310337457774\\
1263	-51.9830211762132\\
1264	-40.5604406354082\\
1265	-40.9126361393032\\
1266	-56.5892676700537\\
1267	-37.1486766411235\\
1268	-42.4972225687657\\
1269	-51.9420191539818\\
1270	-77.1783660964584\\
1271	-71.9939447663487\\
1272	-93.3706770989234\\
1273	-66.943525846893\\
1274	-63.4372680949734\\
1275	-34.7328394243243\\
1276	-34.9936372545319\\
1277	-24.4112092077876\\
1278	-26.5867304760839\\
1279	-32.78150477482\\
1280	-51.8033665858763\\
1281	-58.2235623813674\\
1282	-60.5874300074245\\
1283	-67.1659086716154\\
1284	-100.608409532237\\
1285	-94.3889385319301\\
1286	-71.6837305524666\\
1287	-83.6948096800268\\
1288	-83.5261969441603\\
1289	-103.329504719683\\
1290	-88.1372463898022\\
1291	-60.2978196609763\\
1292	-30.0125371371\\
1293	-23.9409691940034\\
1294	-26.0172663665267\\
1295	-30.5073992753998\\
1296	-32.0522885170083\\
1297	-31.129193346844\\
1298	-40.2380567350287\\
1299	-61.5111098338548\\
1300	-55.2615649072081\\
1301	-56.0810117434942\\
1302	-63.7833691100525\\
1303	-43.8957984013146\\
1304	-27.4065630114058\\
1306	-71.1007359727237\\
1307	-66.3193294250318\\
1308	-72.6268504290242\\
1309	-65.0373134302608\\
1310	-50.6344545913696\\
1311	-68.1580957413605\\
1312	-89.7717373592241\\
1313	-86.1401210754216\\
1314	-64.2881608211962\\
1315	-106.407622851297\\
1316	-83.5936907680468\\
1317	-81.2192891395496\\
1318	-88.963340596227\\
1319	-88.0343782629429\\
1320	-69.217301618032\\
1321	-65.4174132502703\\
1322	-94.0494199100574\\
1323	-128.333605357335\\
1324	-98.068554819577\\
1325	-62.6045427873971\\
1326	-60.3372052119958\\
1327	-62.7785396632762\\
1328	-78.1263045455735\\
1329	-85.8910373068447\\
1330	-63.6180525850468\\
1331	-75.4487661975891\\
1332	-89.3840697433197\\
1333	-89.1168330046203\\
1334	-66.2114568686786\\
1335	-54.3942533446941\\
1336	-72.4070646367832\\
1337	-109.226157500349\\
1338	-100.362630743748\\
1339	-96.1120210794372\\
1340	-64.4922060657711\\
1341	-57.4074482378701\\
1342	-63.4080927417215\\
1343	-37.1414365372866\\
1344	-24.6552830147446\\
1345	-18.5800672780417\\
1346	-17.2668771910292\\
1347	-42.6618153690556\\
1348	-59.2344816060477\\
1349	-67.2806704969435\\
1350	-54.2592652484464\\
1351	-56.3260130556371\\
1352	-64.7132801560344\\
1353	-42.4689661908569\\
1354	-46.0300939106401\\
1355	-45.3219822302419\\
1356	-37.1944953222596\\
1357	-41.8004531256911\\
1358	-66.3511778645263\\
1359	-79.7724692167892\\
1360	-52.9695114565639\\
1361	-43.1528634705078\\
1362	-38.3307894088127\\
1363	-46.7081535545603\\
1364	-30.8187455787399\\
1365	-66.6558730572131\\
1366	-107.803486429204\\
1367	-81.4466391315134\\
1368	-103.981995306269\\
1369	-117.873859702508\\
1370	-118.138200372721\\
1372	-73.0713541315299\\
1373	-73.4005699778968\\
1374	-74.808818288577\\
1375	-83.2833341592404\\
1376	-90.6267120035841\\
1377	-92.0388911303769\\
1378	-116.108880881911\\
1379	-125.063262608116\\
1380	-143.762375642831\\
1381	-103.890565680041\\
1382	-109.826337311036\\
1383	-121.897156229133\\
1384	-101.692187352297\\
1385	-114.938563485686\\
1386	-97.4453883868875\\
1387	-60.1842590787562\\
1388	-35.504147132365\\
1389	-42.7641955009581\\
1390	-60.7612710500318\\
1391	-52.7450436596084\\
1392	-38.7828311479423\\
1393	-49.7109269424016\\
1394	-41.4739594156722\\
1395	-30.0814400777692\\
1396	-39.9619039771012\\
1397	-42.1829132350456\\
1398	-35.1958749861692\\
1399	-56.2105749979953\\
1400	-74.159161218987\\
1401	-59.2871969210025\\
1402	-92.8851541101687\\
1403	-132.370337223955\\
1404	-116.418659475398\\
1405	-133.215430845209\\
1406	-102.854701929626\\
1407	-105.557664071072\\
1408	-79.1917962875114\\
1409	-48.7736565236876\\
1410	-66.419940553061\\
1411	-49.0891119795729\\
1412	-39.1707608246436\\
1413	-54.7758199819746\\
1414	-32.8220904405898\\
1415	-22.6829687754584\\
1416	-27.6384995031078\\
1417	-55.4874825450438\\
1418	-74.519140112249\\
1419	-82.2533473926021\\
1420	-80.7131490005265\\
1421	-78.471851830277\\
1422	-86.1490987058967\\
1423	-69.7366877751017\\
1424	-64.8223966081075\\
1425	-66.7692725922582\\
1426	-53.9233065337264\\
1427	-77.8923579596221\\
1428	-53.888921296247\\
1429	-50.2717845915713\\
1430	-64.8981829094412\\
1431	-86.4020372449011\\
1432	-66.6727507336402\\
1433	-55.2691020731688\\
1434	-50.8741856276238\\
1435	-57.496552755875\\
1436	-40.284454234132\\
1437	-39.1252018601274\\
1438	-52.1105178740722\\
1439	-61.2330111180199\\
1440	-66.4759432444305\\
1441	-55.4058630486686\\
1442	-66.9302030441313\\
1443	-64.2013216649061\\
1444	-37.8070716717384\\
1445	-45.9465895301032\\
1447	-103.914598076152\\
1448	-92.4978736138648\\
1449	-83.0314166481749\\
1450	-78.6210187021779\\
1451	-66.3353032475977\\
1452	-66.3536449258174\\
1453	-54.144740279942\\
1454	-70.4281012244148\\
1455	-61.1595663649441\\
1456	-41.9799607635712\\
1457	-62.7666293185409\\
1458	-67.8525176288906\\
1459	-75.646259717566\\
1460	-78.8491737279012\\
1461	-81.4845535963891\\
1462	-103.41048644405\\
1463	-129.410586941684\\
1464	-139.585530180318\\
1465	-95.7848017473557\\
1466	-56.9752475428936\\
1467	-38.5039881018586\\
1468	-24.3870518219119\\
1469	-43.1366201360968\\
1470	-39.9963367280955\\
1471	-65.3365067082725\\
1472	-59.2889564524246\\
1473	-51.8972239712525\\
1474	-67.5453627799302\\
1475	-74.3632997984782\\
1476	-86.8118242415453\\
1477	-91.032675881693\\
1478	-122.42709160428\\
1479	-119.381038475062\\
1480	-91.3607904009773\\
1481	-65.8522143395287\\
1482	-69.0711977917472\\
1483	-70.6510390054398\\
1484	-82.4732362404638\\
1485	-62.3441783350138\\
1486	-63.1459017331035\\
1487	-64.2303089021364\\
1488	-32.6644082678747\\
1489	-55.2649315149768\\
1490	-81.7669850592347\\
1491	-53.4925413167316\\
1492	-66.642224743159\\
1493	-108.916300709542\\
1494	-82.1268548573696\\
1495	-80.2301638274407\\
1496	-108.297272312422\\
1497	-101.988526375593\\
1498	-70.1008103670156\\
1499	-47.0427525156122\\
1500	-52.5072768218622\\
1501	-75.0314401131695\\
1502	-121.152928608619\\
1503	-117.76611508984\\
1504	-116.438105197263\\
1505	-95.1720540569452\\
1506	-68.1923043096501\\
1507	-58.1097049695197\\
1508	-45.766962825704\\
1509	-43.8279169651503\\
1510	-37.544601377024\\
1511	-49.9967305625116\\
1512	-55.8779968972867\\
1513	-32.5752066974326\\
1514	-52.6839940558152\\
1515	-67.4044289315864\\
1516	-73.4137242489905\\
1517	-94.2732820974879\\
1518	-111.374414776181\\
1519	-135.072431140591\\
1520	-104.624080540579\\
1521	-92.13099233252\\
1522	-98.9513820229263\\
1524	-68.6767421727498\\
1525	-69.5643526668468\\
1526	-111.310289509171\\
1527	-120.456501580856\\
1528	-71.6104170200899\\
1529	-45.2748554626644\\
1530	-29.8474532127998\\
1531	-10.8027178403852\\
1532	-18.9807634122878\\
1533	-35.2029980963546\\
1534	-40.3012973872274\\
1535	-57.876466567848\\
1536	-47.3923691491671\\
1537	-39.5260793117127\\
1538	-38.0409962836375\\
1539	-40.3236642768111\\
1540	-35.0775025838529\\
1541	-41.9120208802665\\
1542	-57.8093867790183\\
1543	-84.181748252812\\
1544	-88.6709269130804\\
1545	-81.8079061377962\\
1546	-92.6768371976823\\
1547	-114.055224971812\\
1548	-114.195667962896\\
1549	-129.741510555362\\
1550	-129.338164529677\\
1551	-92.4121492406857\\
1552	-72.8184964711436\\
1553	-66.9979093374659\\
1554	-112.668314485467\\
1555	-113.080477601831\\
1556	-87.622378945324\\
1557	-56.1785589349904\\
1558	-28.4536248446766\\
1559	-17.9230182829692\\
1560	-23.7628742660231\\
1561	-10.9940595499966\\
1562	-32.1641364458549\\
1563	-44.7710937962022\\
1564	-45.5843785709801\\
1565	-42.9255791557462\\
1566	-26.68087295333\\
1567	-24.1647183830194\\
1568	-32.1015270140006\\
1569	-32.7258628832139\\
1570	-36.4146976754264\\
1572	-27.0395055346187\\
1573	-41.2341807579196\\
1574	-95.4189217817723\\
1575	-111.387619837437\\
1576	-111.554138934663\\
1577	-87.7624703280524\\
1578	-105.551397559072\\
1579	-156.389187273586\\
1580	-139.551786482746\\
1581	-136.626707214005\\
1582	-112.26038913929\\
1583	-108.948233743207\\
1584	-112.430381662526\\
1585	-79.2492537826117\\
1586	-76.7909607171259\\
1587	-45.5680742194259\\
1588	-44.2398588319691\\
1589	-76.9426687972882\\
1590	-53.2764078839232\\
1591	-58.3978152732261\\
1592	-68.0261476785183\\
1593	-62.5318339155244\\
1594	-65.0609548504378\\
1595	-64.2874651399579\\
1596	-73.3048633003666\\
1597	-77.7566237377466\\
1598	-66.6005274301399\\
1599	-57.9821158249711\\
1600	-67.8080021058254\\
1601	-27.8548784596046\\
1602	-13.9679155099241\\
1603	-23.5560616021421\\
1604	-41.1037945852154\\
1605	-17.2276952482553\\
1606	-11.3767300935301\\
1607	-22.9840091819156\\
1608	-37.8348022894991\\
1609	-45.2836997913666\\
1610	-57.2062166710912\\
1611	-49.7698041421672\\
1612	-73.1867326223828\\
1613	-67.7559216757293\\
1614	-46.9967037396386\\
1615	-67.4967564528602\\
1616	-37.7167168007768\\
1617	-33.699874725155\\
1618	-24.0312786311852\\
1619	-17.3212063008559\\
1620	-30.4859207431371\\
1621	-20.4285945755098\\
1622	-34.8656622105495\\
1623	-60.5128164166833\\
1624	-57.5605268509451\\
1625	-45.3870703161315\\
1626	-47.0627604891108\\
1627	-40.3168864369072\\
1628	-50.8956632955833\\
1629	-37.7171600968475\\
1630	-39.8300296043881\\
1631	-35.6188493033596\\
1632	-30.0971005567244\\
1633	-35.0073522071114\\
1634	-31.6934910310435\\
1635	-51.3218311486296\\
1636	-45.5201104284697\\
1637	-37.78095311177\\
1638	-39.1267141461017\\
1639	-32.1669242826549\\
1640	-36.0305584097855\\
1641	-70.377957950421\\
1642	-94.4052827528922\\
1643	-67.5526668905313\\
1644	-64.1776763879466\\
1645	-45.4482325434026\\
1646	-70.018366468415\\
1647	-68.7190442453334\\
1648	-44.266012582646\\
1649	-55.8276663548331\\
1650	-44.7245192750941\\
1651	-59.0625270406063\\
1652	-33.4607587950102\\
1653	-38.4865348084872\\
1654	-45.4109756472178\\
1655	-55.6798886631252\\
1656	-44.0062457872259\\
1657	-44.8683260634541\\
1658	-48.942813527924\\
1659	-70.0341423383964\\
1660	-61.6100948676717\\
1661	-86.1517047978932\\
1662	-116.042848935147\\
1663	-91.4990432533609\\
1664	-64.7859494595457\\
1665	-46.6805313730426\\
1666	-71.6465260579523\\
1667	-87.1037871310982\\
1668	-66.6698642522836\\
1669	-78.089127711585\\
1670	-51.4707520950237\\
1671	-27.1481842137225\\
1672	-30.2852886015123\\
1673	-65.3862077813703\\
1674	-59.1602744254837\\
1675	-50.8606127828932\\
1676	-75.7539525372974\\
1677	-72.4038226483954\\
1678	-66.6926135178562\\
1679	-47.9882932750729\\
1681	-74.4461583142124\\
1682	-62.4633619275796\\
1683	-65.9530889295513\\
1684	-59.0527623655662\\
1685	-46.3842381797187\\
1686	-55.0879574299429\\
1687	-39.3274276461214\\
1688	-47.784463571875\\
1689	-73.5459639523267\\
1691	-80.6888445918378\\
1692	-71.0112066684801\\
1693	-58.3873440648804\\
1694	-43.2407054747023\\
1695	-49.5694273208458\\
1696	-57.6851432404101\\
1697	-64.0111502053389\\
1698	-81.356472876347\\
1699	-92.1645312957733\\
1700	-72.7419118447058\\
1701	-45.8890094920735\\
1702	-80.7518827157817\\
1703	-109.230375178908\\
1704	-74.4992872032108\\
1705	-78.8582998563297\\
1706	-85.4528262877513\\
1707	-74.2168745731806\\
1708	-59.0363825476009\\
1709	-68.5691508460689\\
1710	-63.0787785522041\\
1711	-71.4106439696848\\
1712	-78.7396984983875\\
1713	-69.1406909855007\\
1714	-72.5250055980098\\
1715	-71.0925255511424\\
1716	-74.7558051596227\\
1717	-59.3124622103664\\
1718	-74.2569038113099\\
1719	-55.107043597704\\
1720	-56.9092108880382\\
1721	-67.4783382536857\\
1722	-61.1378316373043\\
1723	-34.4526230575277\\
1724	-20.0358572721543\\
1725	-14.898932381974\\
1726	-26.324637628755\\
1727	-69.0255257965375\\
1728	-86.7882182544108\\
1729	-82.3504315072792\\
1730	-60.9642769082348\\
1731	-68.1046857927108\\
1732	-64.7694687824821\\
1733	-55.9663279879626\\
1734	-37.7737936017622\\
1735	-49.0021276276175\\
1736	-51.9287103513761\\
1737	-46.6270297768974\\
1738	-53.1643513480876\\
1739	-46.0418831160716\\
1740	-47.5909150531779\\
1741	-31.9451423746389\\
1742	-42.1876459236621\\
1743	-69.2287110070324\\
1744	-50.9432112395793\\
1745	-61.5380925818513\\
1746	-58.3505219716276\\
1747	-71.4831693337237\\
1748	-69.2454636970954\\
1749	-87.083037558961\\
1750	-68.5974874155736\\
1751	-74.4214550246825\\
1752	-76.208884415767\\
1753	-42.5132859045818\\
1754	-76.3417379279233\\
1755	-44.093885343894\\
1756	-63.4588139517371\\
1757	-60.4264475095256\\
1758	-80.0104034685917\\
1759	-61.9287199867113\\
1760	-76.4437454778458\\
1761	-92.9582771853072\\
1762	-78.061405101261\\
1764	-62.2694435212425\\
1765	-69.72134748336\\
1766	-61.8488000913669\\
1767	-55.530571655634\\
1768	-51.8884703964666\\
1769	-52.7546697199205\\
1770	-51.6080585359257\\
1771	-86.293816559564\\
1772	-106.775004479619\\
1773	-90.131216957164\\
1774	-92.6910811200519\\
1775	-86.5887003141252\\
1776	-59.9582839372908\\
1777	-51.4440861594674\\
1778	-47.946533098778\\
1779	-37.6960355974174\\
1780	-44.9861506495411\\
1781	-62.0559884141985\\
1782	-75.2232115251009\\
1783	-81.6372860210072\\
1784	-73.478100324133\\
1785	-52.2584163048421\\
1786	-85.2550381127135\\
1787	-101.872775170537\\
1788	-100.663149569231\\
1789	-86.8837106128519\\
1790	-131.236588585345\\
1791	-106.443279830367\\
1792	-62.5295332264452\\
1793	-46.9982025215008\\
1794	-39.4562009121196\\
1795	-69.9195556505513\\
1796	-82.6995456407751\\
1797	-102.242130392465\\
1798	-80.9989376870562\\
1799	-67.7096600581906\\
1800	-40.4824442246349\\
1801	-40.6402936565776\\
1802	-49.1661651686659\\
1803	-47.8835422670102\\
1804	-36.986808076362\\
1805	-41.6190395187846\\
};
\addlegendentry{MPO prediction}

\end{axis}

\begin{axis}[%
width=6.159cm,
height=1.831cm,
at={(8.104cm,0cm)},
scale only axis,
xmin=1000,
xmax=2000,
xlabel style={font=\color{white!15!black}},
xlabel={Sample index},
ymin=-158.691,
ymax=-9.766,
ylabel style={font=\color{white!15!black}},
ylabel={$y(t)$},
axis background/.style={fill=white},
title style={font=\bfseries},
title={C10: RMSE(OSA) = 3.4069, RMSE(MPO) = 5.9801},
legend style={legend cell align=left, align=left, draw=white!15!black}
]
\addplot [color=mycolor1, line width=2.0pt]
  table[row sep=crcr]{%
1006	-76.904\\
1007	-90.3320000000001\\
1008	-68.3589999999999\\
1009	-40.2829999999999\\
1010	-48.828\\
1011	-46.3869999999999\\
1012	-54.932\\
1013	-45.1659999999999\\
1014	-21.973\\
1015	-17.0899999999999\\
1016	-18.3109999999999\\
1018	-62.2560000000001\\
1019	-67.1389999999999\\
1020	-69.5799999999999\\
1021	-52.49\\
1022	-34.1800000000001\\
1023	-64.6970000000001\\
1024	-52.49\\
1025	-39.0630000000001\\
1026	-61.0350000000001\\
1027	-53.711\\
1028	-45.1659999999999\\
1029	-73.242\\
1030	-67.1389999999999\\
1031	-52.49\\
1032	-75.684\\
1033	-74.463\\
1034	-58.5940000000001\\
1036	-41.5039999999999\\
1037	-47.607\\
1038	-58.5940000000001\\
1039	-54.932\\
1040	-59.8140000000001\\
1041	-59.8140000000001\\
1042	-65.9180000000001\\
1043	-85.4490000000001\\
1044	-76.904\\
1045	-54.932\\
1046	-50.049\\
1047	-35.4000000000001\\
1048	-34.1800000000001\\
1049	-45.1659999999999\\
1050	-34.1800000000001\\
1051	-36.6210000000001\\
1052	-51.27\\
1053	-54.932\\
1054	-51.27\\
1055	-78.125\\
1056	-62.2560000000001\\
1057	-36.6210000000001\\
1058	-30.518\\
1059	-40.2829999999999\\
1060	-46.3869999999999\\
1061	-29.297\\
1062	-29.297\\
1063	-36.6210000000001\\
1065	-26.855\\
1066	-35.4000000000001\\
1067	-40.2829999999999\\
1068	-58.5940000000001\\
1069	-56.152\\
1070	-65.9180000000001\\
1071	-61.0350000000001\\
1072	-70.8009999999999\\
1073	-57.373\\
1074	-58.5940000000001\\
1075	-51.27\\
1077	-48.828\\
1079	-111.084\\
1080	-114.746\\
1081	-112.305\\
1082	-74.463\\
1084	-125.732\\
1085	-128.174\\
1086	-96.4359999999999\\
1087	-124.512\\
1088	-158.691\\
1089	-115.967\\
1090	-97.6559999999999\\
1091	-68.3589999999999\\
1092	-53.711\\
1093	-47.607\\
1094	-56.152\\
1095	-45.1659999999999\\
1096	-30.518\\
1097	-30.518\\
1098	-36.6210000000001\\
1099	-50.049\\
1101	-47.607\\
1102	-62.2560000000001\\
1103	-64.6970000000001\\
1104	-85.4490000000001\\
1105	-72.021\\
1106	-85.4490000000001\\
1107	-63.4770000000001\\
1108	-54.932\\
1109	-64.6970000000001\\
1110	-48.828\\
1111	-43.9449999999999\\
1112	-53.711\\
1113	-54.932\\
1114	-40.2829999999999\\
1115	-43.9449999999999\\
1116	-32.9590000000001\\
1118	-40.2829999999999\\
1119	-30.518\\
1120	-34.1800000000001\\
1121	-43.9449999999999\\
1122	-63.4770000000001\\
1123	-65.9180000000001\\
1124	-72.021\\
1125	-45.1659999999999\\
1126	-56.152\\
1127	-89.1110000000001\\
1128	-70.8009999999999\\
1129	-81.787\\
1130	-80.566\\
1131	-50.049\\
1132	-36.6210000000001\\
1133	-47.607\\
1134	-52.49\\
1135	-81.787\\
1136	-103.76\\
1137	-102.539\\
1138	-76.904\\
1139	-80.566\\
1140	-72.021\\
1141	-69.5799999999999\\
1142	-68.3589999999999\\
1143	-51.27\\
1145	-39.0630000000001\\
1146	-37.8420000000001\\
1147	-41.5039999999999\\
1148	-56.152\\
1149	-84.229\\
1150	-70.8009999999999\\
1151	-50.049\\
1152	-42.7249999999999\\
1153	-40.2829999999999\\
1154	-28.076\\
1155	-23.193\\
1156	-24.414\\
1157	-41.5039999999999\\
1158	-32.9590000000001\\
1159	-34.1800000000001\\
1160	-37.8420000000001\\
1161	-35.4000000000001\\
1162	-26.855\\
1163	-21.973\\
1164	-18.3109999999999\\
1165	-25.635\\
1166	-51.27\\
1167	-73.242\\
1168	-78.125\\
1169	-54.932\\
1170	-39.0630000000001\\
1172	-24.414\\
1173	-41.5039999999999\\
1174	-42.7249999999999\\
1175	-54.932\\
1176	-73.242\\
1177	-108.643\\
1178	-125.732\\
1179	-103.76\\
1180	-101.318\\
1181	-67.1389999999999\\
1182	-75.684\\
1183	-79.346\\
1184	-86.6700000000001\\
1185	-69.5799999999999\\
1186	-62.2560000000001\\
1187	-61.0350000000001\\
1188	-56.152\\
1189	-67.1389999999999\\
1190	-52.49\\
1191	-76.904\\
1192	-97.6559999999999\\
1194	-45.1659999999999\\
1195	-47.607\\
1196	-78.125\\
1197	-93.9939999999999\\
1198	-112.305\\
1199	-119.629\\
1200	-119.629\\
1201	-91.5530000000001\\
1202	-84.229\\
1203	-96.4359999999999\\
1204	-111.084\\
1205	-74.463\\
1206	-46.3869999999999\\
1207	-62.2560000000001\\
1208	-54.932\\
1209	-34.1800000000001\\
1210	-47.607\\
1211	-53.711\\
1212	-42.7249999999999\\
1213	-34.1800000000001\\
1214	-43.9449999999999\\
1215	-40.2829999999999\\
1216	-52.49\\
1217	-65.9180000000001\\
1218	-62.2560000000001\\
1219	-45.1659999999999\\
1220	-45.1659999999999\\
1221	-67.1389999999999\\
1222	-104.98\\
1223	-79.346\\
1224	-51.27\\
1225	-40.2829999999999\\
1226	-47.607\\
1227	-40.2829999999999\\
1228	-31.7380000000001\\
1229	-35.4000000000001\\
1230	-42.7249999999999\\
1231	-51.27\\
1232	-57.373\\
1233	-42.7249999999999\\
1234	-63.4770000000001\\
1235	-75.684\\
1236	-70.8009999999999\\
1237	-57.373\\
1238	-62.2560000000001\\
1239	-81.787\\
1240	-52.49\\
1241	-28.076\\
1242	-36.6210000000001\\
1243	-42.7249999999999\\
1244	-51.27\\
1245	-46.3869999999999\\
1246	-36.6210000000001\\
1247	-52.49\\
1248	-52.49\\
1249	-37.8420000000001\\
1250	-47.607\\
1251	-41.5039999999999\\
1252	-37.8420000000001\\
1253	-45.1659999999999\\
1254	-29.297\\
1255	-31.7380000000001\\
1257	-26.855\\
1258	-47.607\\
1259	-53.711\\
1260	-69.5799999999999\\
1261	-89.1110000000001\\
1263	-36.6210000000001\\
1264	-31.7380000000001\\
1265	-29.297\\
1266	-40.2829999999999\\
1267	-31.7380000000001\\
1268	-30.518\\
1269	-39.0630000000001\\
1270	-61.0350000000001\\
1271	-53.711\\
1272	-79.346\\
1273	-54.932\\
1274	-48.828\\
1275	-32.9590000000001\\
1276	-29.297\\
1277	-23.193\\
1278	-23.193\\
1279	-25.635\\
1280	-40.2829999999999\\
1281	-46.3869999999999\\
1283	-51.27\\
1284	-79.346\\
1285	-73.242\\
1286	-53.711\\
1287	-63.4770000000001\\
1288	-69.5799999999999\\
1289	-85.4490000000001\\
1290	-72.021\\
1291	-47.607\\
1292	-28.076\\
1293	-21.973\\
1294	-21.973\\
1295	-26.855\\
1296	-26.855\\
1297	-25.635\\
1298	-32.9590000000001\\
1299	-47.607\\
1300	-43.9449999999999\\
1301	-45.1659999999999\\
1302	-52.49\\
1303	-35.4000000000001\\
1304	-20.752\\
1305	-35.4000000000001\\
1306	-56.152\\
1307	-54.932\\
1308	-59.8140000000001\\
1309	-53.711\\
1310	-40.2829999999999\\
1311	-51.27\\
1312	-72.021\\
1313	-70.8009999999999\\
1314	-51.27\\
1315	-81.787\\
1316	-62.2560000000001\\
1317	-63.4770000000001\\
1318	-73.242\\
1319	-70.8009999999999\\
1320	-54.932\\
1321	-47.607\\
1323	-108.643\\
1324	-85.4490000000001\\
1325	-50.049\\
1326	-51.27\\
1327	-51.27\\
1328	-61.0350000000001\\
1329	-72.021\\
1330	-54.932\\
1331	-56.152\\
1332	-73.242\\
1333	-79.346\\
1334	-56.152\\
1335	-45.1659999999999\\
1336	-57.373\\
1337	-87.8910000000001\\
1338	-78.125\\
1339	-80.566\\
1340	-54.932\\
1341	-45.1659999999999\\
1342	-47.607\\
1343	-31.7380000000001\\
1344	-20.752\\
1345	-20.752\\
1346	-17.0899999999999\\
1347	-31.7380000000001\\
1348	-48.828\\
1349	-54.932\\
1350	-40.2829999999999\\
1351	-45.1659999999999\\
1352	-52.49\\
1353	-35.4000000000001\\
1355	-35.4000000000001\\
1356	-29.297\\
1357	-31.7380000000001\\
1358	-51.27\\
1359	-64.6970000000001\\
1360	-41.5039999999999\\
1361	-32.9590000000001\\
1362	-28.076\\
1363	-32.9590000000001\\
1364	-26.855\\
1365	-45.1659999999999\\
1366	-80.566\\
1367	-63.4770000000001\\
1368	-81.787\\
1369	-98.877\\
1370	-97.6559999999999\\
1371	-76.904\\
1372	-58.5940000000001\\
1373	-54.932\\
1374	-57.373\\
1375	-65.9180000000001\\
1376	-76.904\\
1377	-76.904\\
1378	-100.098\\
1379	-108.643\\
1380	-130.615\\
1381	-91.5530000000001\\
1382	-98.877\\
1383	-108.643\\
1384	-85.4490000000001\\
1385	-96.4359999999999\\
1386	-85.4490000000001\\
1387	-51.27\\
1388	-34.1800000000001\\
1389	-36.6210000000001\\
1390	-51.27\\
1391	-43.9449999999999\\
1392	-31.7380000000001\\
1393	-41.5039999999999\\
1394	-32.9590000000001\\
1395	-23.193\\
1396	-31.7380000000001\\
1397	-34.1800000000001\\
1398	-25.635\\
1399	-42.7249999999999\\
1400	-58.5940000000001\\
1401	-45.1659999999999\\
1402	-68.3589999999999\\
1403	-100.098\\
1404	-87.8910000000001\\
1405	-109.863\\
1406	-81.787\\
1407	-81.787\\
1408	-62.2560000000001\\
1409	-37.8420000000001\\
1410	-46.3869999999999\\
1411	-42.7249999999999\\
1412	-29.297\\
1413	-40.2829999999999\\
1414	-21.973\\
1415	-19.5309999999999\\
1416	-24.414\\
1417	-45.1659999999999\\
1418	-57.373\\
1419	-65.9180000000001\\
1420	-65.9180000000001\\
1421	-62.2560000000001\\
1422	-69.5799999999999\\
1423	-57.373\\
1424	-48.828\\
1425	-50.049\\
1426	-43.9449999999999\\
1427	-59.8140000000001\\
1428	-46.3869999999999\\
1429	-36.6210000000001\\
1430	-51.27\\
1431	-69.5799999999999\\
1432	-54.932\\
1433	-41.5039999999999\\
1434	-37.8420000000001\\
1435	-41.5039999999999\\
1436	-32.9590000000001\\
1437	-30.518\\
1438	-41.5039999999999\\
1439	-47.607\\
1440	-52.49\\
1441	-42.7249999999999\\
1442	-53.711\\
1443	-51.27\\
1444	-32.9590000000001\\
1445	-32.9590000000001\\
1446	-58.5940000000001\\
1447	-80.566\\
1448	-78.125\\
1449	-67.1389999999999\\
1450	-62.2560000000001\\
1451	-50.049\\
1452	-48.828\\
1453	-41.5039999999999\\
1454	-54.932\\
1455	-50.049\\
1456	-32.9590000000001\\
1457	-45.1659999999999\\
1458	-52.49\\
1459	-61.0350000000001\\
1460	-68.3589999999999\\
1461	-68.3589999999999\\
1463	-107.422\\
1464	-119.629\\
1465	-80.566\\
1466	-46.3869999999999\\
1467	-32.9590000000001\\
1468	-24.414\\
1469	-34.1800000000001\\
1470	-29.297\\
1471	-51.27\\
1472	-48.828\\
1473	-40.2829999999999\\
1474	-52.49\\
1475	-63.4770000000001\\
1476	-70.8009999999999\\
1477	-75.684\\
1478	-103.76\\
1479	-92.7729999999999\\
1481	-51.27\\
1482	-52.49\\
1483	-54.932\\
1484	-65.9180000000001\\
1485	-51.27\\
1487	-48.828\\
1488	-30.518\\
1489	-45.1659999999999\\
1490	-68.3589999999999\\
1491	-48.828\\
1492	-52.49\\
1493	-86.6700000000001\\
1494	-63.4770000000001\\
1495	-63.4770000000001\\
1496	-85.4490000000001\\
1497	-81.787\\
1498	-53.711\\
1499	-35.4000000000001\\
1500	-36.6210000000001\\
1501	-61.0350000000001\\
1502	-90.3320000000001\\
1503	-97.6559999999999\\
1504	-100.098\\
1505	-79.346\\
1506	-53.711\\
1507	-46.3869999999999\\
1508	-37.8420000000001\\
1509	-34.1800000000001\\
1510	-29.297\\
1511	-40.2829999999999\\
1512	-42.7249999999999\\
1513	-30.518\\
1514	-42.7249999999999\\
1515	-56.152\\
1516	-62.2560000000001\\
1517	-78.125\\
1519	-114.746\\
1520	-84.229\\
1521	-73.242\\
1522	-76.904\\
1523	-63.4770000000001\\
1524	-46.3869999999999\\
1525	-57.373\\
1526	-90.3320000000001\\
1527	-101.318\\
1528	-61.0350000000001\\
1529	-36.6210000000001\\
1530	-26.855\\
1531	-18.3109999999999\\
1532	-21.973\\
1533	-30.518\\
1534	-31.7380000000001\\
1535	-42.7249999999999\\
1536	-37.8420000000001\\
1537	-30.518\\
1538	-29.297\\
1539	-29.297\\
1540	-26.855\\
1541	-31.7380000000001\\
1542	-43.9449999999999\\
1543	-63.4770000000001\\
1544	-68.3589999999999\\
1545	-65.9180000000001\\
1546	-74.463\\
1547	-91.5530000000001\\
1548	-92.7729999999999\\
1549	-106.201\\
1550	-100.098\\
1551	-74.463\\
1552	-54.932\\
1553	-51.27\\
1554	-84.229\\
1555	-97.6559999999999\\
1556	-69.5799999999999\\
1557	-46.3869999999999\\
1558	-26.855\\
1559	-21.973\\
1560	-20.752\\
1561	-14.6479999999999\\
1562	-25.635\\
1563	-40.2829999999999\\
1564	-40.2829999999999\\
1565	-35.4000000000001\\
1566	-24.414\\
1567	-19.5309999999999\\
1568	-28.076\\
1569	-28.076\\
1570	-30.518\\
1571	-24.414\\
1572	-21.973\\
1573	-32.9590000000001\\
1574	-69.5799999999999\\
1575	-79.346\\
1576	-86.6700000000001\\
1577	-63.4770000000001\\
1578	-80.566\\
1579	-114.746\\
1580	-103.76\\
1581	-106.201\\
1582	-81.787\\
1583	-81.787\\
1584	-86.6700000000001\\
1585	-59.8140000000001\\
1586	-56.152\\
1587	-40.2829999999999\\
1588	-35.4000000000001\\
1589	-62.2560000000001\\
1590	-47.607\\
1591	-48.828\\
1592	-53.711\\
1593	-50.049\\
1594	-50.049\\
1595	-52.49\\
1596	-57.373\\
1597	-64.6970000000001\\
1598	-56.152\\
1599	-41.5039999999999\\
1600	-52.49\\
1601	-31.7380000000001\\
1602	-18.3109999999999\\
1603	-25.635\\
1604	-34.1800000000001\\
1606	-9.76600000000008\\
1607	-21.973\\
1608	-32.9590000000001\\
1609	-37.8420000000001\\
1610	-43.9449999999999\\
1611	-39.0630000000001\\
1612	-58.5940000000001\\
1613	-48.828\\
1614	-36.6210000000001\\
1615	-54.932\\
1616	-39.0630000000001\\
1617	-26.855\\
1618	-23.193\\
1619	-17.0899999999999\\
1620	-25.635\\
1621	-19.5309999999999\\
1622	-29.297\\
1623	-43.9449999999999\\
1624	-42.7249999999999\\
1625	-32.9590000000001\\
1626	-40.2829999999999\\
1627	-29.297\\
1628	-43.9449999999999\\
1629	-31.7380000000001\\
1630	-31.7380000000001\\
1631	-30.518\\
1632	-24.414\\
1633	-26.855\\
1634	-26.855\\
1635	-41.5039999999999\\
1636	-36.6210000000001\\
1637	-30.518\\
1638	-32.9590000000001\\
1639	-26.855\\
1640	-31.7380000000001\\
1641	-58.5940000000001\\
1642	-75.684\\
1643	-51.27\\
1644	-48.828\\
1645	-39.0630000000001\\
1646	-58.5940000000001\\
1647	-58.5940000000001\\
1648	-37.8420000000001\\
1649	-46.3869999999999\\
1650	-39.0630000000001\\
1651	-51.27\\
1652	-32.9590000000001\\
1654	-37.8420000000001\\
1655	-47.607\\
1656	-36.6210000000001\\
1657	-41.5039999999999\\
1658	-41.5039999999999\\
1659	-57.373\\
1660	-51.27\\
1661	-72.021\\
1662	-93.9939999999999\\
1663	-78.125\\
1664	-50.049\\
1665	-40.2829999999999\\
1667	-69.5799999999999\\
1668	-56.152\\
1669	-63.4770000000001\\
1670	-43.9449999999999\\
1671	-23.193\\
1672	-24.414\\
1673	-52.49\\
1674	-45.1659999999999\\
1675	-42.7249999999999\\
1676	-64.6970000000001\\
1677	-61.0350000000001\\
1678	-56.152\\
1679	-39.0630000000001\\
1680	-53.711\\
1681	-65.9180000000001\\
1682	-52.49\\
1683	-53.711\\
1684	-48.828\\
1685	-37.8420000000001\\
1686	-42.7249999999999\\
1687	-35.4000000000001\\
1688	-36.6210000000001\\
1689	-57.373\\
1690	-67.1389999999999\\
1691	-70.8009999999999\\
1692	-62.2560000000001\\
1693	-45.1659999999999\\
1694	-35.4000000000001\\
1695	-39.0630000000001\\
1696	-46.3869999999999\\
1699	-79.346\\
1700	-65.9180000000001\\
1701	-40.2829999999999\\
1702	-68.3589999999999\\
1703	-91.5530000000001\\
1704	-65.9180000000001\\
1705	-64.6970000000001\\
1706	-75.684\\
1707	-59.8140000000001\\
1708	-50.049\\
1709	-56.152\\
1710	-51.27\\
1711	-57.373\\
1712	-70.8009999999999\\
1713	-56.152\\
1714	-63.4770000000001\\
1715	-58.5940000000001\\
1716	-59.8140000000001\\
1717	-48.828\\
1718	-63.4770000000001\\
1719	-45.1659999999999\\
1721	-52.49\\
1722	-50.049\\
1723	-30.518\\
1724	-19.5309999999999\\
1725	-15.8689999999999\\
1726	-28.076\\
1727	-53.711\\
1728	-67.1389999999999\\
1729	-65.9180000000001\\
1730	-46.3869999999999\\
1731	-54.932\\
1732	-50.049\\
1733	-43.9449999999999\\
1734	-30.518\\
1735	-42.7249999999999\\
1736	-37.8420000000001\\
1737	-36.6210000000001\\
1738	-41.5039999999999\\
1740	-34.1800000000001\\
1741	-25.635\\
1742	-35.4000000000001\\
1743	-58.5940000000001\\
1744	-42.7249999999999\\
1745	-52.49\\
1746	-45.1659999999999\\
1747	-59.8140000000001\\
1748	-57.373\\
1749	-76.904\\
1750	-58.5940000000001\\
1751	-64.6970000000001\\
1752	-65.9180000000001\\
1753	-37.8420000000001\\
1754	-61.0350000000001\\
1755	-46.3869999999999\\
1756	-51.27\\
1757	-57.373\\
1758	-69.5799999999999\\
1759	-53.711\\
1760	-70.8009999999999\\
1761	-81.787\\
1762	-69.5799999999999\\
1763	-58.5940000000001\\
1764	-50.049\\
1765	-58.5940000000001\\
1766	-52.49\\
1767	-43.9449999999999\\
1768	-37.8420000000001\\
1769	-45.1659999999999\\
1770	-39.0630000000001\\
1771	-72.021\\
1772	-90.3320000000001\\
1773	-79.346\\
1774	-75.684\\
1775	-74.463\\
1776	-46.3869999999999\\
1777	-42.7249999999999\\
1778	-35.4000000000001\\
1779	-30.518\\
1780	-32.9590000000001\\
1781	-52.49\\
1782	-59.8140000000001\\
1783	-70.8009999999999\\
1784	-59.8140000000001\\
1785	-42.7249999999999\\
1786	-72.021\\
1787	-84.229\\
1788	-90.3320000000001\\
1789	-73.242\\
1790	-115.967\\
1791	-85.4490000000001\\
1792	-51.27\\
1793	-39.0630000000001\\
1794	-34.1800000000001\\
1795	-56.152\\
1796	-69.5799999999999\\
1797	-89.1110000000001\\
1798	-68.3589999999999\\
1799	-53.711\\
1800	-31.7380000000001\\
1801	-36.6210000000001\\
1802	-34.1800000000001\\
1803	-39.0630000000001\\
1804	-25.635\\
1805	-35.4000000000001\\
};
\addlegendentry{True output}

\addplot [color=mycolor2, dashed, line width=2.0pt]
  table[row sep=crcr]{%
1006	-75.4433374680916\\
1007	-84.0131295580413\\
1008	-69.0521531457325\\
1009	-41.527862705022\\
1010	-50.5808092701227\\
1011	-53.3625838437169\\
1012	-52.5017454682952\\
1013	-51.5106805168393\\
1014	-18.3152294117463\\
1015	-18.9672076785489\\
1016	-17.730321148327\\
1017	-43.2030631259656\\
1018	-72.9292681188454\\
1019	-66.6886816977985\\
1020	-71.3581898214229\\
1021	-55.3233104702945\\
1022	-36.514111246973\\
1023	-71.427209012443\\
1024	-44.7514131601715\\
1025	-42.7899249928632\\
1026	-62.6255684833764\\
1027	-50.858585042017\\
1028	-53.4864617855922\\
1029	-67.9489849394838\\
1030	-67.1505177859704\\
1031	-54.161303006955\\
1032	-70.9053420612499\\
1033	-73.0246900685511\\
1034	-62.9101834577809\\
1035	-53.4351092488355\\
1036	-46.9141874363636\\
1037	-49.1769087784583\\
1038	-58.1147918722631\\
1039	-54.4745437030183\\
1040	-56.1134228944545\\
1041	-59.6427521359253\\
1042	-61.105974492328\\
1043	-85.4202746155054\\
1044	-76.3024919667121\\
1045	-57.1519754043952\\
1046	-55.6472744623397\\
1047	-33.372551878673\\
1048	-38.2535725437513\\
1049	-43.6092154873616\\
1050	-37.4448121423302\\
1051	-38.9574595795589\\
1052	-52.3246912950588\\
1053	-54.2996922510822\\
1054	-51.0278827598008\\
1055	-77.7176092383484\\
1056	-60.5711513896563\\
1057	-42.3306640600572\\
1058	-35.4781300453008\\
1059	-41.7247465378443\\
1060	-48.739186031517\\
1061	-28.5324165041038\\
1062	-30.7095459668142\\
1063	-40.6355143111768\\
1064	-32.6668799334741\\
1065	-31.3268566441484\\
1066	-36.0907810430976\\
1067	-39.1544663946975\\
1068	-62.3441326385387\\
1069	-55.7754810683552\\
1070	-65.0399772947783\\
1071	-64.7824089778883\\
1072	-64.2177056768451\\
1073	-61.5697431649955\\
1074	-57.0616948640775\\
1075	-54.6913619264408\\
1076	-51.4659168054789\\
1077	-54.0646638591479\\
1078	-77.2066854772547\\
1079	-106.56861212406\\
1080	-112.131431872272\\
1081	-104.859099469491\\
1082	-78.9165675167469\\
1083	-104.033774013299\\
1084	-119.268857150666\\
1085	-120.579689027033\\
1086	-98.5849720037509\\
1087	-123.563350501141\\
1088	-152.310641361103\\
1089	-123.394540827739\\
1090	-102.288223801237\\
1091	-72.9029665109572\\
1092	-55.506571677382\\
1093	-51.7417372410766\\
1094	-57.6000576631634\\
1095	-43.6042877455441\\
1096	-27.8562168576732\\
1097	-29.7102015534165\\
1098	-37.4197331277919\\
1099	-54.8524532996428\\
1100	-47.8278864837109\\
1101	-51.1172833279811\\
1102	-61.6155266685366\\
1103	-63.7502114220056\\
1104	-84.0688846564087\\
1105	-76.679519764763\\
1106	-80.9482121987025\\
1107	-66.7399763773915\\
1108	-60.4725378884127\\
1109	-65.5915331640574\\
1110	-52.670580533699\\
1111	-49.2952238617108\\
1112	-53.7676530433412\\
1113	-55.3581246539075\\
1114	-43.5585599995411\\
1115	-44.9141813770348\\
1116	-38.8383469899659\\
1117	-38.4682653346154\\
1118	-39.5955104422471\\
1119	-33.4419012643518\\
1120	-37.2609400240256\\
1121	-44.8544876382387\\
1122	-64.1183234911039\\
1123	-65.8666129119429\\
1124	-70.1062712424305\\
1125	-46.5877845426476\\
1126	-62.6801705044209\\
1127	-90.369721866329\\
1128	-70.220397182929\\
1129	-77.6611052879584\\
1130	-80.5197054217313\\
1131	-51.5926342304451\\
1132	-42.1694961479768\\
1133	-50.291593642318\\
1134	-55.6892851288001\\
1135	-84.5365030589865\\
1136	-95.9887991010849\\
1137	-98.6075858448826\\
1138	-79.4051204197388\\
1139	-82.4833122716714\\
1140	-73.2095802697593\\
1141	-74.2388413111735\\
1142	-71.4308853217333\\
1143	-50.0408752041317\\
1144	-46.777940616963\\
1145	-44.8429540267291\\
1146	-38.7152080832257\\
1147	-45.0437149546137\\
1148	-59.1706650520287\\
1149	-82.8915748964932\\
1150	-66.1056946502115\\
1151	-55.5468229287017\\
1152	-46.7112788272855\\
1153	-45.474535386187\\
1154	-28.0507902845648\\
1155	-22.2821012118\\
1156	-27.9874739970203\\
1157	-43.2103685238005\\
1158	-34.8633901665269\\
1159	-37.4562252494923\\
1160	-39.4920946566008\\
1161	-37.4922794809286\\
1163	-22.6954088928064\\
1164	-20.1728081544522\\
1165	-28.5457691796644\\
1166	-55.2962754802988\\
1167	-76.8765897917249\\
1168	-81.1023719764321\\
1169	-56.9779638526877\\
1170	-43.0632710509749\\
1172	-25.0640713236705\\
1173	-46.7526681671582\\
1174	-42.023413761824\\
1175	-57.0263010058052\\
1176	-70.7278001637728\\
1177	-112.693417144016\\
1178	-128.95280059711\\
1179	-105.846089465673\\
1180	-105.242093089805\\
1181	-69.9529425756227\\
1182	-80.5422226926403\\
1183	-77.9265933183672\\
1184	-89.9680907044724\\
1185	-66.2236255644011\\
1186	-67.1308387170059\\
1187	-64.8088849275589\\
1188	-58.5102076490546\\
1189	-66.1845325184574\\
1190	-52.7729125338203\\
1191	-83.7071120678834\\
1192	-87.7235641926513\\
1193	-71.5533582305859\\
1194	-48.511178250589\\
1195	-50.2939227371173\\
1196	-80.2826984743067\\
1197	-90.4798196839301\\
1198	-106.772558294258\\
1199	-114.446020118896\\
1200	-120.361457367064\\
1201	-92.6474100563123\\
1202	-92.776469121218\\
1203	-95.7084974572481\\
1204	-106.034162526593\\
1205	-75.278458764403\\
1206	-47.9081468508389\\
1207	-66.3152907402318\\
1208	-58.8771975574816\\
1209	-36.7479142865216\\
1210	-48.745510600213\\
1211	-52.6710028904188\\
1212	-47.2697790914604\\
1213	-37.9404638290007\\
1214	-46.5520094672161\\
1215	-42.584213374404\\
1216	-48.7674813641129\\
1217	-72.7688831171527\\
1218	-61.9532442922321\\
1219	-49.834658171309\\
1220	-50.485292144466\\
1221	-64.8340333662743\\
1222	-107.098292514133\\
1223	-78.8327282246012\\
1224	-54.6803526388403\\
1225	-41.5451921031965\\
1226	-50.9965180491247\\
1227	-46.8862384082456\\
1228	-34.1394649020847\\
1229	-37.159091696667\\
1230	-47.2652903147091\\
1231	-50.5169644050009\\
1232	-58.0373801626952\\
1233	-41.2350802908984\\
1234	-68.1685523267129\\
1235	-77.4952832484755\\
1236	-70.6153829844602\\
1237	-62.1721153652154\\
1238	-63.8926471243701\\
1239	-83.0520927517932\\
1240	-56.4857371652895\\
1241	-27.6725931577487\\
1242	-37.9121784728841\\
1243	-41.2069875273401\\
1244	-55.4261673039641\\
1245	-47.9185382005317\\
1246	-39.6190913742719\\
1247	-54.8598031452389\\
1248	-52.230626854873\\
1249	-42.3284313257464\\
1250	-51.2369287858558\\
1251	-44.8753139677715\\
1252	-42.0526300962222\\
1253	-44.5272226855668\\
1254	-31.5586318396417\\
1255	-37.337407073363\\
1256	-25.5531583397119\\
1257	-32.2201060140876\\
1258	-52.7458773402384\\
1259	-52.7608451281264\\
1260	-68.9930736648041\\
1261	-90.1050990198337\\
1262	-64.9377518366648\\
1263	-41.6237508763625\\
1264	-33.8018326567924\\
1265	-33.7574521616452\\
1266	-46.1313219616529\\
1267	-26.9377810872775\\
1268	-36.0229905528652\\
1269	-40.600870718913\\
1270	-61.0883452453581\\
1271	-60.2202248172907\\
1272	-76.3337975292591\\
1273	-56.7256838468779\\
1274	-53.9501315578782\\
1275	-29.9120069027033\\
1276	-32.9957024379662\\
1277	-23.701600340358\\
1278	-24.0991903174286\\
1279	-29.7463098970695\\
1280	-44.6522340795534\\
1281	-44.8952568540574\\
1282	-51.9829284463244\\
1283	-52.7182915717938\\
1284	-83.0740800344188\\
1285	-73.5594115095219\\
1286	-57.9927902159368\\
1287	-65.8968543774117\\
1288	-68.5551444430639\\
1289	-84.0863625888855\\
1290	-73.3458901080749\\
1291	-48.8244466757108\\
1292	-29.8525782006859\\
1293	-24.268192056014\\
1294	-25.3325082467297\\
1296	-29.0279409301929\\
1297	-26.7498145731722\\
1298	-34.3322345614349\\
1299	-50.6625497148075\\
1300	-47.1192778473569\\
1301	-44.6942254945532\\
1302	-52.7706227893652\\
1303	-35.8078712297126\\
1304	-26.8167426754635\\
1305	-39.3003444161343\\
1306	-58.8406934644488\\
1307	-51.8753586421512\\
1308	-61.8734581817921\\
1309	-52.040238603747\\
1310	-43.984331962773\\
1311	-54.0697870945853\\
1312	-71.4433621607272\\
1313	-71.5734468714074\\
1314	-55.4789138089257\\
1315	-79.471233811546\\
1316	-67.1455022189496\\
1317	-60.3465892276574\\
1318	-72.3324672718452\\
1319	-70.606696197613\\
1320	-59.08015417518\\
1321	-53.7129613757077\\
1322	-76.9480684363068\\
1323	-101.727517224951\\
1324	-86.9506924393061\\
1325	-52.3316494261662\\
1326	-53.2036640556962\\
1327	-55.9092701514141\\
1328	-63.2044886234801\\
1329	-67.536980911455\\
1330	-57.7286496232095\\
1331	-57.0098862836039\\
1332	-74.1109219395598\\
1333	-76.7427991541294\\
1334	-58.3135647149752\\
1335	-48.3096693361215\\
1336	-59.4088889576269\\
1337	-86.7275782727663\\
1338	-78.7139482754651\\
1339	-78.5872212004638\\
1340	-51.889006176046\\
1341	-54.0844531981822\\
1342	-50.0627547220936\\
1343	-32.0316680058199\\
1344	-23.2266930441876\\
1345	-18.697025098036\\
1346	-18.6714527429021\\
1347	-36.4549417478131\\
1348	-48.3291700717834\\
1349	-58.2671796111822\\
1350	-45.5885567141124\\
1351	-47.48249194291\\
1352	-53.3742432697011\\
1353	-35.0684614492422\\
1354	-39.173094624244\\
1355	-37.3783361861576\\
1356	-30.0021744948954\\
1357	-37.545797453263\\
1358	-53.2124704895143\\
1359	-62.9496496805491\\
1360	-40.6449726718226\\
1361	-37.6486518777569\\
1362	-32.8812125464869\\
1363	-37.9131682149443\\
1364	-24.9507257550554\\
1365	-52.0829712428417\\
1366	-82.7799525282769\\
1367	-62.398571132838\\
1368	-83.7595986601439\\
1369	-93.4437223789062\\
1370	-96.8618560148659\\
1372	-61.4407232532321\\
1373	-63.7157421605323\\
1374	-56.1622868308191\\
1375	-66.4362687295834\\
1376	-73.8674099811467\\
1377	-73.3177546034269\\
1378	-98.0590246589486\\
1379	-107.897758069125\\
1380	-119.04403245842\\
1381	-96.2761740859569\\
1382	-101.725417911519\\
1383	-109.414614083687\\
1384	-86.4544543415866\\
1385	-98.5515281254218\\
1386	-85.8924331895073\\
1387	-52.2858500853665\\
1388	-36.7809563478111\\
1389	-36.2836305740748\\
1390	-56.6036258654876\\
1391	-41.2275696484867\\
1392	-33.9513888458598\\
1393	-41.2660453139963\\
1394	-35.6147957025023\\
1395	-27.8634929228003\\
1396	-31.7589428333104\\
1397	-36.8927560783006\\
1398	-28.2075369093268\\
1399	-48.1290397949351\\
1400	-57.4006496881764\\
1401	-45.0559265457659\\
1402	-72.0244562187343\\
1403	-102.108261815335\\
1404	-87.0226531089279\\
1405	-110.762503578218\\
1406	-79.7878735336442\\
1407	-88.6705891556262\\
1408	-61.0124733474283\\
1409	-41.858297315146\\
1410	-53.2395384920501\\
1411	-37.9665055039329\\
1412	-35.1566642840482\\
1413	-42.3444644561544\\
1415	-17.2365168038818\\
1416	-23.5102519216084\\
1417	-46.1234836216042\\
1418	-58.7600807407507\\
1419	-66.7848798283778\\
1420	-67.4545970209751\\
1421	-62.3931805965644\\
1422	-72.265345271966\\
1423	-57.2691347864127\\
1425	-52.039832277934\\
1426	-43.3469195035029\\
1427	-64.3746750312084\\
1428	-45.1633795743944\\
1429	-39.7965765048355\\
1430	-53.8958621891163\\
1431	-68.9823591153852\\
1432	-55.0148231296046\\
1433	-48.0740999281302\\
1434	-40.1716020877107\\
1435	-46.5230762288388\\
1436	-33.2779512585421\\
1437	-31.198919737571\\
1438	-40.7257849067503\\
1439	-50.9422046763307\\
1440	-53.4434566769935\\
1441	-45.7160504256774\\
1442	-53.9360972242339\\
1443	-50.5899316200441\\
1444	-35.6781854944902\\
1445	-37.7359529595049\\
1446	-62.0412288139185\\
1447	-79.2034421969447\\
1448	-78.6717153813579\\
1449	-65.5852610907043\\
1450	-68.2375137602339\\
1451	-52.2819866723269\\
1452	-52.3653066256818\\
1453	-42.6147382547624\\
1454	-57.2871116877841\\
1455	-48.892278081368\\
1456	-37.3009651076509\\
1457	-49.4999944745805\\
1458	-51.6150658874153\\
1459	-61.8717974366511\\
1460	-65.4693564068812\\
1461	-66.3552102963147\\
1462	-88.5264615389249\\
1463	-105.400964925194\\
1464	-119.663426547141\\
1465	-81.5704585938049\\
1466	-50.0252500045078\\
1467	-34.6248038275805\\
1468	-27.4741983331639\\
1469	-34.0492654185834\\
1470	-34.5628274242313\\
1471	-52.9892194273025\\
1472	-46.1514965164954\\
1473	-45.1544115059814\\
1474	-55.7881537889748\\
1475	-58.1582931493497\\
1476	-72.0365882666083\\
1477	-73.830428635549\\
1478	-103.196549908081\\
1479	-94.176771445231\\
1480	-79.4778726887321\\
1481	-58.1429129448572\\
1482	-53.994411944655\\
1483	-57.232825896227\\
1484	-67.2936745854267\\
1485	-50.8500682383112\\
1486	-52.5564627090228\\
1487	-50.9581545529536\\
1488	-29.2982716295023\\
1490	-68.5560875731173\\
1491	-48.1054111623635\\
1492	-56.5810311689117\\
1493	-82.6530484998164\\
1494	-69.5517172340349\\
1495	-63.6310150398758\\
1496	-87.612742605327\\
1497	-76.5744400186775\\
1498	-59.3984297311847\\
1499	-40.2580288644035\\
1500	-39.7827280819038\\
1501	-62.0268854208352\\
1502	-88.6847155227222\\
1503	-96.688086924979\\
1504	-92.3038769742705\\
1505	-81.0817341402858\\
1506	-58.2494280859912\\
1507	-51.7018744187926\\
1508	-37.7103104294961\\
1509	-38.2142092725746\\
1510	-34.2747809802224\\
1511	-42.7982507078434\\
1512	-45.3110933705243\\
1513	-30.733690255802\\
1514	-47.0267358712044\\
1515	-55.9221223566703\\
1516	-62.6149707252371\\
1517	-74.8051799319387\\
1518	-98.166715312474\\
1519	-111.310765420224\\
1520	-87.4201682532209\\
1521	-79.4038791701573\\
1522	-76.9486435361032\\
1523	-70.9077734904672\\
1524	-52.0125743561211\\
1525	-55.7017171048994\\
1526	-89.2205537065438\\
1527	-94.595042777195\\
1528	-64.7008526769007\\
1529	-38.6683969432027\\
1530	-26.5907081674288\\
1531	-16.1993991247903\\
1532	-21.7869382814947\\
1533	-33.2962683813057\\
1534	-36.4673553358739\\
1535	-45.752775675023\\
1536	-38.8076934562689\\
1537	-32.6387394398703\\
1538	-32.8435004488158\\
1539	-32.7869015926099\\
1540	-31.3328167697659\\
1541	-32.7079034927622\\
1542	-44.9630797612858\\
1543	-65.9680160225305\\
1544	-67.0207880529556\\
1545	-66.4125518012213\\
1546	-76.1795696957902\\
1547	-89.6602208842983\\
1548	-95.303843612568\\
1549	-105.131057474371\\
1550	-103.171637045094\\
1551	-79.6153673146005\\
1552	-57.7361988253363\\
1553	-57.7893451699763\\
1554	-81.8549238861679\\
1555	-93.5013329909982\\
1556	-68.5950935935971\\
1557	-54.630739768236\\
1558	-19.2995777209346\\
1559	-24.3054576535885\\
1560	-23.6220686472268\\
1561	-14.6508147936677\\
1562	-30.0244444883956\\
1563	-37.0199058042322\\
1564	-40.3928832327197\\
1565	-38.2443120213227\\
1566	-26.2626063229047\\
1567	-24.8655067222626\\
1568	-26.2710903240438\\
1570	-32.5575025077683\\
1571	-27.0634544772386\\
1572	-24.3961456356078\\
1573	-35.1274956764564\\
1574	-77.3948270052865\\
1575	-88.6251512621211\\
1576	-85.7588985135546\\
1577	-68.6731016106703\\
1578	-80.9514511522409\\
1579	-116.31860736415\\
1580	-105.843329622528\\
1581	-105.539180586678\\
1582	-84.5858084662912\\
1583	-86.0956169918711\\
1584	-89.1541143849392\\
1585	-62.4084769181623\\
1586	-58.2401851696659\\
1587	-36.0553613304062\\
1588	-36.0372684156607\\
1589	-67.4775926293689\\
1590	-43.4726544020334\\
1591	-49.8291216018588\\
1592	-55.1123403023439\\
1593	-49.465384301496\\
1594	-55.0933282412793\\
1595	-53.5636725625161\\
1596	-60.3056998385682\\
1597	-60.9710046961113\\
1598	-59.4617738201505\\
1599	-45.2764177871768\\
1600	-56.6356916376683\\
1601	-27.7305434860809\\
1602	-16.9414998775981\\
1603	-24.9235052251965\\
1604	-35.6763709459492\\
1605	-21.094530173275\\
1606	-16.5563172420229\\
1607	-20.7585941601683\\
1608	-34.2722348243649\\
1609	-39.4017360091295\\
1610	-48.8687382596506\\
1611	-43.1757835034621\\
1612	-56.2901527781012\\
1613	-57.6404056320075\\
1614	-38.0333022360765\\
1615	-53.0535023487366\\
1616	-33.1098850541734\\
1617	-32.5934348130991\\
1618	-23.2927718418216\\
1619	-19.7056282473031\\
1620	-26.4427055955173\\
1621	-20.9302745398893\\
1622	-29.583728374324\\
1623	-50.3081716562522\\
1624	-43.9936212724315\\
1625	-38.2896075922863\\
1626	-38.4103716437639\\
1627	-34.6452773773374\\
1628	-39.4408272694327\\
1629	-34.4645353866911\\
1630	-33.2065539206503\\
1631	-30.7779782019572\\
1632	-27.3089026022399\\
1633	-31.1751503448279\\
1634	-26.8621407577903\\
1635	-38.6698272806993\\
1636	-40.8043376082915\\
1637	-31.2810987834293\\
1638	-33.9795718402399\\
1639	-27.4956396398434\\
1640	-32.1818638227435\\
1641	-59.6075791974565\\
1642	-79.1140767876755\\
1643	-54.6123177364982\\
1644	-52.3230012125516\\
1645	-39.1932798701334\\
1646	-63.3290109535453\\
1647	-56.3678099830001\\
1648	-40.3415823855075\\
1649	-48.3503068094074\\
1650	-36.2516995628873\\
1651	-50.6637021590172\\
1652	-33.856420996513\\
1653	-33.5453750125969\\
1654	-43.3087835056619\\
1655	-45.7842639226217\\
1656	-41.500987694677\\
1657	-38.8362104950049\\
1658	-42.5946978411737\\
1659	-59.074107098181\\
1660	-51.9824087299805\\
1661	-71.787706049123\\
1662	-98.807713093277\\
1663	-75.8373008086833\\
1664	-56.58953363399\\
1665	-40.8597144754278\\
1666	-58.3674994164719\\
1667	-70.4909706676137\\
1668	-51.9594166374716\\
1669	-66.6542001515136\\
1670	-43.3301446859707\\
1671	-26.7533611572742\\
1672	-27.0988375353138\\
1673	-50.9961291160196\\
1674	-52.3272423663602\\
1675	-40.966426866016\\
1676	-65.1264976136365\\
1677	-59.0321494241318\\
1678	-54.8907508813213\\
1679	-47.5779408720825\\
1680	-48.2734011139808\\
1681	-64.3798678411033\\
1682	-56.2386358514686\\
1684	-52.94686718905\\
1685	-40.2433284030128\\
1686	-46.9714761914011\\
1687	-33.3267985707134\\
1688	-40.0821088384293\\
1689	-59.4214420854889\\
1690	-62.1826815490356\\
1691	-65.7930753070375\\
1692	-65.244975417389\\
1693	-47.6520349881814\\
1694	-38.9073140337516\\
1695	-41.655612160336\\
1696	-47.4794739762035\\
1697	-56.0551225035665\\
1699	-77.7009418672294\\
1700	-64.7007104925899\\
1701	-45.3215080746052\\
1702	-68.1351729152427\\
1703	-87.7959919664445\\
1704	-68.2250708545912\\
1705	-65.4298230651714\\
1706	-75.5139341517608\\
1707	-62.3040322951342\\
1708	-53.384684248781\\
1709	-58.0324712385379\\
1710	-52.2144470791648\\
1711	-59.1121983237408\\
1712	-69.092751755712\\
1713	-55.9128832540334\\
1714	-63.7836371332273\\
1715	-60.8058809921927\\
1716	-59.4605174304265\\
1717	-51.9079377212834\\
1718	-61.0688771311102\\
1719	-50.1867662590089\\
1720	-48.1646870023765\\
1721	-55.9481272834405\\
1722	-50.6356346322127\\
1723	-30.1815811002925\\
1724	-20.499557064031\\
1725	-18.3844415032343\\
1726	-23.422680331415\\
1727	-59.1868124746154\\
1728	-73.842690167962\\
1729	-69.8719098000106\\
1730	-49.2841635809682\\
1731	-55.2393854220813\\
1732	-52.8321919345249\\
1733	-43.964086232028\\
1734	-32.2034821632658\\
1735	-40.1666529921235\\
1736	-44.0185061280424\\
1737	-37.673857469327\\
1738	-44.9936655549659\\
1739	-38.0825338776078\\
1740	-38.3180228999774\\
1741	-26.6255391141351\\
1742	-35.6678496459026\\
1743	-60.5147559233244\\
1744	-42.1331121209678\\
1745	-51.0619909662896\\
1746	-48.2447860562313\\
1747	-59.9155166366779\\
1748	-57.8672386988094\\
1749	-72.386242786022\\
1750	-61.8524216341616\\
1751	-63.9640423701424\\
1752	-64.4357160883176\\
1753	-41.7265309434777\\
1754	-59.2174319105993\\
1755	-46.6668044160776\\
1756	-54.732824495512\\
1757	-55.6411728732669\\
1758	-68.339513624667\\
1759	-56.1817221253984\\
1760	-65.6482924074146\\
1761	-79.6130341195428\\
1763	-62.7567935136785\\
1764	-53.4565455872234\\
1765	-58.9275099998083\\
1766	-52.6982031835414\\
1767	-47.8636519609381\\
1768	-41.8432559617358\\
1769	-46.2585613546651\\
1770	-43.4530187971636\\
1771	-70.0412890990553\\
1772	-91.152484824875\\
1773	-75.7784915661978\\
1774	-72.655837737748\\
1775	-75.3350188341381\\
1776	-51.8107710607646\\
1777	-44.6641122245837\\
1778	-38.0077668408367\\
1779	-33.3359012864082\\
1780	-36.3702572073646\\
1781	-51.1787613300612\\
1782	-61.9092656132407\\
1783	-66.6326691007969\\
1784	-60.1594774827895\\
1785	-47.035030379179\\
1786	-72.2206113769296\\
1787	-80.6752071394151\\
1788	-88.623103169381\\
1789	-77.3501924969605\\
1790	-106.018312202308\\
1791	-96.001684628581\\
1792	-51.2032971023432\\
1793	-42.6712569243562\\
1794	-37.5564062399801\\
1795	-56.6523852999096\\
1796	-68.3077985077348\\
1797	-84.4637529127199\\
1798	-72.1836104484087\\
1799	-58.2979986279331\\
1800	-31.4178737500022\\
1802	-39.7093289763129\\
1803	-42.5550671851713\\
1804	-29.072201530407\\
1805	-35.8005668366116\\
};
\addlegendentry{OSA predition}

\addplot [color=mycolor3, dotted, line width=2.0pt]
  table[row sep=crcr]{%
1006	-76.904\\
1007	-90.3320000000001\\
1008	-68.3589999999999\\
1009	-40.2829999999999\\
1010	-50.5808092701232\\
1011	-54.0821540872721\\
1012	-55.8788220715292\\
1013	-53.0134977745779\\
1014	-22.0048170934431\\
1016	-19.3264634857696\\
1017	-44.8141520938855\\
1018	-75.5139019738849\\
1019	-73.1467221127898\\
1020	-76.1058887789843\\
1021	-60.3861605584868\\
1022	-42.1375237198029\\
1023	-76.9856703017342\\
1024	-52.0585945756527\\
1025	-45.4474025069674\\
1026	-67.2162926458113\\
1027	-55.3505763999976\\
1028	-55.4766429741451\\
1029	-73.7294177081371\\
1030	-69.4140573039758\\
1031	-56.2491664374727\\
1032	-73.7980163944617\\
1033	-73.0174547392314\\
1034	-62.6668008474146\\
1035	-55.1492872998947\\
1036	-49.2570514183938\\
1037	-53.1679691272427\\
1038	-62.2703641701976\\
1039	-57.9364215835981\\
1040	-59.1100171113599\\
1041	-60.6246805813653\\
1042	-62.004069941788\\
1043	-84.1158778643769\\
1044	-75.2931172944709\\
1045	-56.0750941966319\\
1046	-55.4342047195066\\
1047	-35.5507624168474\\
1048	-38.889920220844\\
1049	-46.0620847030975\\
1050	-38.9171769112229\\
1051	-41.4054760467657\\
1052	-55.4107805408546\\
1053	-57.1806128763651\\
1054	-53.39789511706\\
1055	-79.8930055352464\\
1056	-62.1993322226112\\
1057	-42.8796288066774\\
1058	-38.3162614305943\\
1059	-45.8862864042255\\
1060	-52.5566290294607\\
1061	-32.8303648359056\\
1062	-33.9625002955954\\
1063	-43.9941220743576\\
1064	-37.1069909346354\\
1065	-35.0614078126564\\
1066	-41.131315072188\\
1067	-43.6564616631917\\
1068	-66.0071767967224\\
1069	-60.7110263267105\\
1070	-68.7699006877463\\
1071	-67.6515786309899\\
1072	-68.4925903980732\\
1073	-62.1540420595861\\
1074	-59.5692310521606\\
1075	-56.1607389651349\\
1076	-53.7399944106367\\
1077	-56.7410955776261\\
1078	-81.6473250073013\\
1079	-109.097088804536\\
1080	-113.195882036124\\
1081	-104.900814603931\\
1082	-75.6111601999962\\
1083	-103.130429082776\\
1084	-119.515162949437\\
1085	-117.875241672881\\
1086	-93.9687343336859\\
1088	-148.127425898107\\
1089	-117.184161111939\\
1090	-100.673512739027\\
1091	-73.3664949939794\\
1092	-56.9973506750234\\
1093	-54.0372866025716\\
1094	-61.3459198223561\\
1095	-47.2906906977698\\
1096	-29.9590631007725\\
1097	-30.5603052543077\\
1098	-38.0024661784116\\
1099	-55.5708142378287\\
1100	-50.2080226121416\\
1101	-52.4913962675755\\
1102	-64.5455232810907\\
1103	-65.9884516671586\\
1104	-85.7055279152639\\
1105	-77.7051540538648\\
1106	-83.6980211514608\\
1107	-66.9099219769228\\
1108	-62.1102388360193\\
1109	-69.3916465906409\\
1110	-55.6104602116238\\
1111	-53.5562580548108\\
1112	-59.6883716763462\\
1113	-60.1206852206133\\
1114	-47.8380177038798\\
1115	-50.0627052743007\\
1116	-43.216243190793\\
1117	-44.4839475014792\\
1118	-45.296105911362\\
1119	-37.7302265731842\\
1120	-42.4286172420091\\
1121	-50.3467859598411\\
1122	-69.1618032892716\\
1123	-70.7676633847\\
1124	-74.4206786349309\\
1125	-49.2205643569762\\
1126	-65.6472838068876\\
1127	-95.5881660528537\\
1128	-74.7861232030159\\
1129	-81.6799097191172\\
1130	-82.6174707712373\\
1131	-53.3048410586125\\
1132	-44.1597469407161\\
1133	-53.930700284302\\
1134	-59.7359122306184\\
1135	-89.7461625028832\\
1136	-101.932386450318\\
1137	-100.597675450981\\
1138	-79.9229711159983\\
1139	-84.0674671857996\\
1140	-74.7177770621197\\
1141	-75.8977760868672\\
1142	-75.0098108770258\\
1143	-54.1232634856824\\
1144	-49.4461254038649\\
1145	-47.9756873493641\\
1146	-43.7469606320906\\
1147	-49.2072118722842\\
1148	-64.4836448964613\\
1149	-89.0322587992493\\
1150	-70.6450718786477\\
1151	-57.5685411364057\\
1152	-50.8743787638286\\
1153	-50.2566590837903\\
1154	-33.4547563577844\\
1155	-26.6647009211572\\
1156	-31.301529272461\\
1157	-47.9404566600426\\
1158	-39.1448290321205\\
1159	-41.7077942721787\\
1160	-44.5384873097485\\
1161	-42.2583516892919\\
1162	-34.6951198364\\
1163	-27.7320252802253\\
1164	-24.3180744400979\\
1165	-32.8488507368206\\
1166	-60.4692788400428\\
1167	-83.0678292476271\\
1168	-88.016366181982\\
1169	-63.9237771090088\\
1170	-49.4440490630407\\
1171	-40.9162629158434\\
1172	-31.2652236594993\\
1173	-52.4462626696552\\
1174	-49.2159917048355\\
1176	-76.8824784134201\\
1177	-117.414489011073\\
1178	-135.19589286819\\
1179	-112.171514749521\\
1180	-111.41385205079\\
1181	-76.725213301544\\
1182	-87.4762519772016\\
1183	-85.7987042012578\\
1184	-96.3150584861512\\
1185	-73.0609514159198\\
1186	-71.5441247649885\\
1187	-70.5072773096865\\
1188	-64.8528799920325\\
1189	-72.2563599147845\\
1190	-57.6706378301642\\
1191	-88.5409117573113\\
1192	-94.7238017468915\\
1193	-72.9138222241761\\
1194	-49.6608110338689\\
1195	-53.2450713266189\\
1196	-83.0028658711612\\
1197	-93.9416956187151\\
1198	-108.758362526627\\
1199	-114.302144001072\\
1200	-118.170989773957\\
1201	-90.9572477202869\\
1202	-91.4883102696003\\
1203	-97.9369373872903\\
1204	-107.428204571842\\
1205	-74.5607783683301\\
1206	-48.2190479301819\\
1207	-66.9836667236859\\
1208	-60.7140063752211\\
1209	-39.8431221208973\\
1210	-52.3553192773747\\
1211	-56.3123462917738\\
1212	-49.9983374898993\\
1213	-42.1422955763587\\
1214	-51.4161437819341\\
1215	-47.4845360967686\\
1216	-54.1276636351561\\
1217	-76.0866857516287\\
1218	-67.9235623066618\\
1219	-54.271508765215\\
1220	-55.890825211347\\
1221	-71.9868117502306\\
1222	-112.539461911205\\
1223	-84.9219158866222\\
1224	-59.3833618054912\\
1225	-46.3675449730727\\
1226	-55.630272909973\\
1227	-52.0901956379073\\
1228	-41.1742737290795\\
1229	-43.6480473802553\\
1230	-53.6701739086373\\
1231	-58.1772309624616\\
1232	-64.0914784297472\\
1233	-46.5278649055572\\
1234	-72.4383441938014\\
1235	-83.1999357734594\\
1236	-75.9432572286257\\
1237	-66.4897874174665\\
1238	-69.8641344602217\\
1239	-88.7138450151849\\
1241	-33.4170928277376\\
1242	-42.3062367942584\\
1243	-45.4434190660929\\
1244	-58.7030877225213\\
1245	-52.4037170534145\\
1246	-43.7573132592165\\
1247	-59.5293059048042\\
1248	-57.3100829970617\\
1249	-46.3313842720902\\
1250	-56.6633802035342\\
1251	-50.7265888576694\\
1252	-48.1013940869457\\
1253	-51.4702092184395\\
1254	-36.8189765603543\\
1255	-42.8113239813131\\
1256	-32.3347303843468\\
1257	-35.8735747328287\\
1258	-58.8085172969288\\
1259	-59.8655572069565\\
1260	-74.4285207748005\\
1261	-95.492178266021\\
1262	-69.9467356773944\\
1263	-46.45444982852\\
1264	-39.4817195571741\\
1265	-39.0939450309725\\
1266	-52.6996963859835\\
1267	-34.6748247111325\\
1268	-40.1807655054174\\
1269	-46.9296924637674\\
1270	-67.0464331641977\\
1271	-65.1362973971611\\
1272	-83.8472203345054\\
1273	-61.3459794058672\\
1274	-58.6768042273759\\
1275	-36.039528024296\\
1276	-36.1498806121217\\
1277	-27.7721121395859\\
1278	-27.8798495885626\\
1279	-32.9540705316192\\
1280	-49.3677062365477\\
1281	-50.4446042969648\\
1282	-56.0177070921345\\
1283	-57.8525491928081\\
1284	-88.1319596161445\\
1285	-79.3207646669307\\
1286	-62.9197004323521\\
1287	-71.9302963947223\\
1288	-74.6664884413442\\
1289	-89.0285608045947\\
1290	-77.3124497696226\\
1291	-52.6171860138172\\
1292	-33.1515572338108\\
1293	-27.3533705537175\\
1294	-28.8131234854782\\
1295	-31.343976424926\\
1296	-32.5484244534828\\
1297	-30.6120615743691\\
1298	-38.0757430029962\\
1299	-54.4437916411439\\
1300	-51.6357081233864\\
1301	-49.644329774351\\
1302	-56.7106808644357\\
1303	-39.2369784286993\\
1304	-29.8016388427163\\
1305	-44.2124615461519\\
1306	-64.4021349518227\\
1307	-57.7690943636489\\
1308	-65.9931376189818\\
1310	-46.903252936914\\
1311	-57.9481306658449\\
1312	-75.9480380070888\\
1313	-75.1284987349813\\
1314	-59.031658362856\\
1315	-84.3345933224039\\
1316	-69.9921262170096\\
1317	-65.0014217413107\\
1318	-74.9722531710288\\
1319	-72.5406631209141\\
1320	-60.9042573890113\\
1321	-56.6661111393219\\
1322	-81.8047621618284\\
1323	-105.486786368137\\
1324	-87.9566497039184\\
1325	-54.0951443423919\\
1326	-55.4010653801415\\
1327	-57.991440895124\\
1328	-67.1354740031775\\
1329	-71.7597768551477\\
1330	-59.4332374130206\\
1331	-60.0203486766752\\
1332	-76.8463397628027\\
1333	-79.2312597069649\\
1334	-59.5190170167643\\
1335	-50.2840001336951\\
1336	-62.2931667094238\\
1337	-89.8393998873792\\
1338	-81.1410809904319\\
1339	-81.1913150322457\\
1340	-53.1492033117831\\
1341	-53.80047374828\\
1342	-53.7078238558486\\
1343	-35.6203226742809\\
1344	-25.5979354368746\\
1345	-22.0901873833454\\
1346	-20.4789622040587\\
1347	-38.8104562354761\\
1348	-52.2547127741934\\
1349	-61.0721471989041\\
1350	-49.5917223439417\\
1351	-53.0439019129601\\
1352	-58.6616164989637\\
1353	-39.7408845099503\\
1354	-43.1308997031458\\
1355	-42.2531675307534\\
1356	-34.5870466803724\\
1357	-41.541151488339\\
1358	-59.3976913491424\\
1359	-68.8121125689549\\
1360	-44.6835314895827\\
1361	-40.8881497376615\\
1362	-37.5100104433345\\
1363	-43.3660680112414\\
1364	-31.0784828109072\\
1365	-56.8658756810191\\
1366	-90.7191511853239\\
1367	-69.5249058701843\\
1368	-89.6220254269565\\
1369	-99.9715342453096\\
1370	-100.135198858202\\
1371	-81.5033740697611\\
1372	-64.4969161986012\\
1373	-67.0303083673061\\
1374	-62.4807463132315\\
1375	-71.1226818094019\\
1376	-78.6389445352888\\
1377	-76.4642579151143\\
1378	-99.3433377136139\\
1379	-108.297585513428\\
1380	-118.90414061125\\
1381	-91.2951509321429\\
1383	-108.417697111482\\
1384	-85.3314954329637\\
1385	-98.5642231919385\\
1386	-86.8297929083842\\
1387	-53.1128655065297\\
1388	-37.7554857546897\\
1389	-38.2451337715929\\
1390	-58.0260467260928\\
1391	-44.6843846844808\\
1392	-35.5356607328122\\
1393	-43.5599554685389\\
1394	-37.6374039265752\\
1395	-30.3108716318156\\
1396	-35.7499623886722\\
1397	-39.9596295990468\\
1398	-31.9986781498449\\
1399	-52.5500112035686\\
1400	-63.3008647982031\\
1401	-49.3591209323179\\
1402	-76.253889880667\\
1403	-107.610368674624\\
1404	-92.1635985580622\\
1405	-115.159370340434\\
1406	-84.0574060388021\\
1407	-91.489441336897\\
1408	-66.0614442254425\\
1409	-45.3080085364481\\
1410	-57.6241589456986\\
1411	-44.7058495362123\\
1412	-38.1567442233413\\
1413	-47.8192596645945\\
1414	-35.0005792366344\\
1415	-23.9740421765803\\
1416	-28.0737173141736\\
1417	-50.2003598338276\\
1418	-63.1374064510885\\
1419	-70.7091757413052\\
1420	-71.1203777872224\\
1421	-66.2179212315521\\
1422	-75.6386672754522\\
1423	-61.1742913900284\\
1424	-57.8728807286259\\
1425	-57.0867263978867\\
1426	-48.219254035442\\
1427	-68.348428976851\\
1428	-50.6653304503695\\
1429	-43.5772055346138\\
1431	-74.1530626428744\\
1432	-58.9314121183522\\
1433	-51.5519428418561\\
1434	-45.7784890211196\\
1435	-51.7887565859401\\
1436	-39.4851514815064\\
1437	-36.5627152251448\\
1439	-55.0219947090964\\
1440	-58.3090708056982\\
1441	-49.9630093026328\\
1442	-58.7492236922499\\
1443	-54.7925013344448\\
1444	-38.7368811042891\\
1445	-41.5492348524663\\
1446	-67.1639837110333\\
1447	-84.9674596773805\\
1448	-83.3668894022801\\
1449	-70.0726306643598\\
1450	-71.3648782966347\\
1451	-57.1865621489992\\
1452	-57.1880078648733\\
1453	-47.7977079890773\\
1454	-62.486459631062\\
1455	-54.2105189842073\\
1456	-41.1662041512066\\
1457	-54.6469907972978\\
1458	-57.6585622621194\\
1459	-66.526398053144\\
1460	-70.3035010767774\\
1461	-69.342284208062\\
1462	-90.4293021928834\\
1463	-107.405727445411\\
1464	-120.336402098544\\
1465	-82.2265117201543\\
1466	-51.050191803778\\
1467	-36.7679435826615\\
1468	-29.7322396933703\\
1469	-37.0832063947817\\
1470	-37.1159298681241\\
1471	-57.551510935217\\
1472	-50.481071941045\\
1473	-47.5992935936986\\
1474	-60.4133261707989\\
1475	-63.0981050989631\\
1476	-73.9387961770356\\
1477	-76.7777276802174\\
1478	-104.753360503624\\
1479	-95.1186148191566\\
1480	-80.9955245423096\\
1481	-62.311295995891\\
1482	-59.9633277191863\\
1483	-62.6501952330693\\
1484	-73.4642280098915\\
1485	-56.659585919356\\
1486	-57.105833734867\\
1487	-55.8903903564703\\
1488	-34.0541758642639\\
1489	-52.0208804329848\\
1490	-73.3522736212208\\
1491	-52.0332275503797\\
1492	-59.5621655499865\\
1493	-87.2892385309797\\
1494	-71.5486360293494\\
1495	-68.08800553177\\
1496	-91.3223684320722\\
1497	-80.5450005428552\\
1498	-60.816458946492\\
1499	-43.7012494546018\\
1500	-44.6490218822696\\
1501	-66.9607516235067\\
1502	-94.0934476713396\\
1503	-101.137525734276\\
1504	-95.8628410897334\\
1505	-80.7668721906709\\
1506	-58.830927496871\\
1507	-54.1326536118579\\
1508	-41.2050549786436\\
1509	-40.9767930291246\\
1510	-38.4207048746503\\
1511	-48.4254528451877\\
1512	-50.7500385383448\\
1513	-36.3137427164302\\
1514	-52.031645643819\\
1515	-62.1295692062538\\
1516	-67.7019620841297\\
1517	-79.6180852798946\\
1518	-101.303324399302\\
1519	-114.991375353477\\
1520	-88.8636173617963\\
1521	-81.9427982437985\\
1522	-81.6493579192688\\
1523	-74.4304914041543\\
1524	-58.1640003062682\\
1525	-63.1959550238696\\
1526	-95.0417225767471\\
1527	-100.10082171543\\
1528	-66.550671335453\\
1529	-41.5089107672634\\
1531	-17.8019813927256\\
1532	-22.5341789836177\\
1533	-34.1503124789147\\
1534	-38.2587691282333\\
1535	-48.8470643289093\\
1537	-36.0623936110305\\
1538	-36.6894191769438\\
1539	-37.3844129561146\\
1540	-36.3592409005912\\
1541	-38.6666405453941\\
1542	-50.4660392715223\\
1543	-71.5833225248302\\
1544	-73.0194608278116\\
1545	-70.7842193355416\\
1546	-80.3802993643219\\
1547	-94.1225569831176\\
1548	-98.242171373332\\
1549	-109.053907270136\\
1550	-105.980110595631\\
1551	-83.1818224738415\\
1552	-62.8059143392838\\
1553	-62.8155037966685\\
1554	-89.1780357412526\\
1555	-98.9873725800476\\
1556	-71.8025935503154\\
1557	-57.1814038829496\\
1558	-24.5058652578803\\
1559	-24.8688801771484\\
1560	-24.8745316247077\\
1561	-17.7044099236625\\
1562	-31.7379961974896\\
1563	-40.5746720993172\\
1564	-41.9221823472205\\
1565	-39.8268861916308\\
1566	-28.8819472230118\\
1567	-27.3079182876252\\
1568	-30.3788790837039\\
1569	-31.9482631878911\\
1570	-35.5080307540964\\
1571	-30.465025734117\\
1572	-27.8814369101983\\
1573	-39.0344843937892\\
1574	-82.048823478598\\
1575	-96.1378534975977\\
1576	-95.7194686246307\\
1577	-76.4443001794771\\
1578	-90.5307875052426\\
1579	-125.136465414746\\
1580	-114.22538340402\\
1581	-113.8817865212\\
1582	-91.1123364643865\\
1583	-92.9502815605383\\
1584	-96.8134936059851\\
1585	-69.5139463107623\\
1586	-65.262093537347\\
1587	-42.5225142440172\\
1588	-39.4707131248556\\
1589	-71.344610307825\\
1590	-48.7982464256054\\
1591	-52.0282712586129\\
1592	-57.7980405519045\\
1593	-52.5338589920239\\
1594	-57.041936685215\\
1595	-57.4395807111271\\
1596	-63.867255259938\\
1597	-65.2430864743678\\
1598	-61.6693032433786\\
1599	-48.5557398809572\\
1600	-60.914196942349\\
1601	-32.3594751326013\\
1602	-18.7246221075097\\
1603	-26.0076657873844\\
1604	-36.7467558079031\\
1605	-22.2758850743851\\
1606	-17.0117379541023\\
1607	-23.9117278814751\\
1608	-36.1484659801265\\
1609	-41.8117501671488\\
1610	-51.7999942936374\\
1611	-47.3673173606126\\
1612	-61.3828236802253\\
1613	-61.0218874341933\\
1614	-44.7793042667938\\
1615	-58.9448777074185\\
1616	-36.6482811894625\\
1617	-33.6222963208968\\
1618	-26.6784002758482\\
1619	-22.2797276531205\\
1620	-29.1126979280889\\
1621	-23.7844876626596\\
1622	-32.4863517300141\\
1623	-53.030537016657\\
1624	-48.9258880220164\\
1625	-42.5525156855974\\
1626	-44.0947760956749\\
1627	-38.6181799923613\\
1628	-45.111877361433\\
1629	-37.224062911029\\
1630	-36.5963326806384\\
1631	-34.3909172206556\\
1632	-29.856371715309\\
1633	-34.7218355177843\\
1634	-31.4361400229504\\
1635	-42.3238939517316\\
1636	-42.949761209866\\
1637	-35.0949183756043\\
1638	-37.0850117539519\\
1639	-30.1222680247165\\
1640	-34.921274211147\\
1641	-62.2668322975601\\
1642	-81.9229559661196\\
1643	-58.3050478664227\\
1644	-56.5983476224824\\
1645	-43.933836266777\\
1646	-67.6094949275789\\
1647	-62.1522652317387\\
1648	-43.9979214638377\\
1649	-52.4963407663606\\
1650	-40.6013681263601\\
1651	-52.9322109920797\\
1652	-35.6921921622611\\
1653	-35.5501783528725\\
1654	-43.9757936136186\\
1655	-48.7441914838766\\
1656	-42.9533781585012\\
1657	-42.0713368295731\\
1658	-44.2691136739847\\
1659	-61.1083728879253\\
1660	-54.5201812205273\\
1661	-73.9894841425278\\
1662	-100.884808804407\\
1664	-58.573893142772\\
1665	-45.0043120331952\\
1666	-62.204645588728\\
1667	-75.1916174068547\\
1668	-56.4493475898032\\
1669	-68.7252116801897\\
1670	-46.5883330180636\\
1671	-29.0512877235731\\
1672	-29.9782931806392\\
1673	-54.759629654475\\
1674	-54.8453707708184\\
1675	-46.3519129014937\\
1676	-68.5947596036451\\
1677	-62.3278315905504\\
1678	-57.1973821268161\\
1679	-48.8012253601948\\
1680	-52.8815299513151\\
1681	-65.4791527567081\\
1682	-57.0167879257215\\
1683	-57.2947548645222\\
1684	-54.8631403760571\\
1685	-43.4058779727545\\
1686	-50.7236968606865\\
1687	-37.993480705996\\
1688	-43.1069838387102\\
1689	-63.852568358373\\
1690	-66.7232075454408\\
1691	-67.4619375590571\\
1692	-65.1423642064228\\
1693	-49.0251115605438\\
1694	-40.6585943608209\\
1695	-44.1372427465935\\
1696	-50.8118781268608\\
1697	-59.4383652873123\\
1699	-79.5138866003153\\
1700	-65.5622613147143\\
1701	-45.4646440788536\\
1702	-70.3420116302527\\
1703	-89.2384607649508\\
1704	-68.0851766374474\\
1705	-66.7326460458894\\
1706	-76.5946105756625\\
1707	-62.8975834816124\\
1708	-55.1308202057548\\
1709	-60.7896728539931\\
1710	-55.0972251506423\\
1711	-62.061553261208\\
1712	-72.5271848837399\\
1713	-58.0211624391245\\
1714	-65.6153866036752\\
1715	-62.5376802282683\\
1716	-61.6917645546514\\
1717	-53.5293051189212\\
1718	-63.8213722628584\\
1719	-51.4546568228654\\
1720	-51.3465092565909\\
1721	-58.2652815706999\\
1722	-53.906470765311\\
1723	-33.2164972838984\\
1724	-22.3806065391898\\
1725	-20.4557313088173\\
1726	-26.1290605253841\\
1727	-59.554809416559\\
1728	-77.1082963686595\\
1729	-74.7307830776813\\
1730	-54.4483954582493\\
1731	-61.0480653286954\\
1732	-58.0174529405504\\
1733	-49.428858176535\\
1734	-36.5015921406323\\
1735	-44.4007159241385\\
1736	-46.6435734684703\\
1737	-42.5121204681925\\
1738	-49.189070642911\\
1739	-42.7902891301944\\
1740	-42.572448736724\\
1741	-31.6189657279824\\
1742	-40.2166530594425\\
1743	-64.7177243976816\\
1744	-46.6350497914818\\
1745	-54.4754502472481\\
1746	-50.5143858217291\\
1747	-63.3715634224304\\
1748	-60.5634917654875\\
1749	-74.9968737514\\
1750	-62.3433085577165\\
1751	-65.9367605146485\\
1752	-65.5520475563096\\
1753	-41.7442163833234\\
1754	-61.2502212036652\\
1755	-47.3426580651537\\
1756	-55.4836545054147\\
1757	-57.8915963058239\\
1758	-69.2000416522731\\
1759	-56.5893363423711\\
1760	-67.251121093394\\
1761	-78.5237992711625\\
1762	-69.7200551598155\\
1763	-62.3591736549677\\
1764	-54.4460086606832\\
1765	-60.9072772918748\\
1766	-54.5177989632746\\
1768	-45.0387967007935\\
1769	-50.3139072383572\\
1770	-47.1061697990542\\
1771	-75.4833174528385\\
1772	-95.0797764252136\\
1773	-79.7665060197271\\
1774	-74.5924668652547\\
1775	-75.7236061539927\\
1776	-52.6750373824916\\
1777	-47.3949805189234\\
1778	-40.6870580133486\\
1779	-36.5040751958218\\
1780	-40.3999831423928\\
1781	-55.9663367494279\\
1782	-65.5023513810208\\
1783	-70.9697069310739\\
1784	-62.0222730332941\\
1785	-48.7016802085566\\
1786	-75.5295608576939\\
1787	-83.1645427002306\\
1788	-89.7042307787092\\
1789	-77.9471850817752\\
1790	-108.059419121774\\
1791	-93.1953882875075\\
1792	-53.5692493368699\\
1793	-44.8948744070772\\
1794	-39.6675979081945\\
1795	-60.6113118556041\\
1796	-71.7663926579653\\
1797	-87.3200038958507\\
1798	-72.8003981998243\\
1799	-60.5161603759761\\
1800	-34.960507543472\\
1801	-37.8469353566363\\
1802	-41.4270907798268\\
1803	-46.6206580141218\\
1804	-33.4328285039894\\
1805	-40.571416270813\\
};
\addlegendentry{MPO prediction}

\end{axis}
\end{tikzpicture}%
	\caption{Validation results for the model estimated using LASSO. Compression direction d3.}\label{fig:spgl_vs}
\end{figure}
\end{document}